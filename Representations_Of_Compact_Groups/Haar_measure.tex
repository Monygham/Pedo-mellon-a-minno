\input ../pree.tex

\begin{document}

\title{Haar Measure}
\date{}
\maketitle

\section{Introduction}
\noindent
In this notes we introduce Haar measure, which is a fundamental technical tool in representation theory of locally compact topological groups. We send the interested reader to \cite{diestel2014joys} for excellent exposition of aspects and applications of this notion beyond our rudimentary presentation.

\section{Existence of Haar measure}

\begin{definition}
Let $G$ be a topological group and let $\mu$ be a Borel measure. Then $\mu$ is \textit{left-invariant} if $\mu(xA) = \mu(A)$ for every $A$ in $\cB(G)$. Similarly $\mu$ is right-invariant if $\mu(Ax) = \mu(A)$ for every $A$ in $\cB(G)$.
\end{definition}

\begin{definition}
Let $G$ be a locally compact group and $\mu$ be a Borel measure. If $\mu$ is a nonzero, left-invariant, regular Borel measure on $G$, then we say that $\mu$ is \textit{a left Haar measure on $G$}. Similarly if $\mu$ is a nonzero, right-invariant, regular Borel measure on $G$, then we say that $\mu$ is \textit{a right Haar measure on $G$} 
\end{definition}

\begin{theorem}
Let $G$ be a locally compact topological group. Then there exists a left (right) Haar measure $\mu$ on $G$. If in addition $G$ is $\sigma$-compact, then $\mu$ is inner regular.
\end{theorem}
\noindent
We denote by $\cK$ the set of all compact subsets of $G$ and by $\cU$ the set of all open neighborhoods of identity in $G$. Let $U$ be an open nonempty subset of $G$ and $K$ be a compact subset of $G$. We define
$$(K:U) = \inf \big\{n\in \NN\,\big|\mbox{ there exist }x_1,...,x_n\in G\mbox{ such that }K\subseteq \bigcup_{i=1}^nx_iU\big\}$$
Throughout the proof we fix a compact subset $Q$ of $G$ such that $\bd{int}(Q) \neq \emptyset$.

\begin{lemma}\label{lemma:approximationoncompactsets}
Fix $U\in \cU$. There exists a real valued function $h_U$ on $\cK$ such that the following assertions hold.
\begin{enumerate}[label=\emph{\textbf{(\arabic*)}}, leftmargin=1.5em]
\item For every compact subset $K$ in $\cK$ we have $h_U(K)\geq 0$, $h_U(\emptyset) = 0$ and $h_U(Q)=1$.
\item For every compact subset $K$ in $\cK$ and for every element $x$ in $G$ we have $h_U(xK) = h_U(K)$.
\item If $K\subseteq L$ are compact subsets in $\cK$, then $h_U(K)\subseteq h_U(L)$.
\item For every compact subset $K$ in $\cK$ we have $h_U(K) \leq \left(K:\bd{int}(Q)\right)$.
\item If $K, L$ are compact subsets in $\cK$, then
$$h_U(K\cup L) \leq h_U(K) + h_U(L)$$
and if $K\cdot U^{-1}\cap L\cdot U^{-1} = \emptyset$, then the equality holds.
\end{enumerate}
\end{lemma}
\begin{proof}[Proof of the lemma]
For every compact subset $K$ of $G$ we define
$$h_U(K) = \frac{(K:U)}{(Q:U)}$$
Now we check that $h_U$ admits the properties above. Properties \textbf{(1)}, \textbf{(2)} and \textbf{(3)} are clear. For \textbf{(4)} note that
$$(K:U) \leq (Q:U)\cdot\left(K:\bd{int}(Q)\right)$$
Indeed, if $K\subseteq \bigcup_{i=1}^ny_i\cdot \bd{int}(Q)$ and $Q\subseteq \bigcup_{j=1}^mz_jU$, then $K\subseteq \bigcup_{i=1}^n\bigcup_{j=1}^my_iz_jU$ and this implies the inequality above. Observe that $xU \cap K \neq \emptyset$ implies that $x\in K\cdot U^{-1}$ and similarly $xU\cap L\neq \emptyset$ implies that $x\in L\cdot U^{-1}$. Assuming that for compact subsets $K, L$ in $G$ we have $ K\cdot U^{-1}\cap L\cdot U^{-1} = \emptyset$ we derive from this that for every $x\in G$ we have $xU\cap \left(K\cap L\right) = \emptyset$. Thus if $K\cdot U^{-1}\cap L\cdot U^{-1} = \emptyset$, then we have $(K\cup L:U) = (K:U)+(K:L)$ and hence $h_U(K\cup L) = h_U(K)+h_U(L)$. Note that in general case we have $(K\cup L:U) \leq (K:U)+(K:L)$ and hence also \textbf{(5)} holds for $h_U$.
\end{proof}

\begin{lemma}\label{lemma:compactsetsareseparablebyopens}
Let $K, L$ in $\cK$ and suppose that $K\cap L =\emptyset$. Then there exists $U\in \cU$ such that
$$K\cdot U^{-1}\cap L\cdot U^{-1} = \emptyset$$
\end{lemma}
\begin{proof}[Proof of the lemma]
Left as an exercise.
\end{proof}

\begin{lemma}\label{lemma:definitiononcompactsets}
There exists a real valued function $h$ on $\cK$ such that the following assertions hold.
\begin{enumerate}[label=\emph{\textbf{(\arabic*)}}, leftmargin=1.5em]
\item For every compact subset $K$ in $\cK$ we have $h(K)\geq 0$, $h(\emptyset) = 0$ and $h(Q)=1$.
\item For every compact subset $K$ in $\cK$ and for every element $x$ in $G$ we have $h(xK) = h(K)$.
\item If $K\subseteq L$ are compact subsets in $\cK$, then $h(K)\subseteq h(L)$.
\item For every compact subset $K$ in $\cK$ we have $h(K) \leq \left(K:\bd{int}(Q)\right)$.
\item If $K, L$ are compact subsets in $\cK$, then
$$h(K\cup L) \leq h(K) + h(L)$$
and if $K \cap L = \emptyset$, then the equality holds.
\end{enumerate}
\end{lemma}
\begin{proof}[Proof of the lemma]
Consider a topological space
$$X = \prod_{K\in \cK}\big[0,(K:\bd{int}(Q)\big]$$
By Tichonoff's theorem $X$ is compact. For every $U\in \cU$ we define a subset $F_U\subseteq X$ that consists of tuples $\{a_K\}_{K\in \cK}$ such that $a_{\emptyset} = 0$, $a_{Q}=1$, $a_{xK}= a_K$ for $x\in G$ and $K$ in $\cK$, $a_K\leq a_L$ for $K\subseteq L$ in $\cK$, $a_{K\cup L}\leq a_K + a_L$ for $K,L$ in $\cK$ and the equality holds if $K\cdot U^{-1} \cap L\cdot U^{-1} = \emptyset$. Conditions imposed on tuples in $F_U$ imply that $F_U$ is a closed subset. Note that $\{h_U(K)\}_{K\in \cK}\in F_U$ for every $U\in \cU$. Moreover, we have
$$F_{U_1\cap U_2\cap ...\cap U_n} \subseteq F_{U_1}\cap F_{U_2}\cap ...\cap F_{U_n}$$
for $U_1, U_2,...,U_n\in \cU$. This implies that $\big\{F_U\big\}_{U\in \cU}$ is a centered family of nonempty closed subsets of a compact space $X$. Thus
$$\bigcap_{U\in \cU}F_U \neq \emptyset$$
by compactness of $X$. Hence there exists $\{c_K\}_{K\in \cK}$ in the intersection. We define a real function $h$ on $\cK$ by $h(K) = c_K$ for $K$ in $\cK$. The fact that properties \textbf{(1)}, \textbf{(2)}, \textbf{(3)} and \textbf{(4)} hold for $h$ follows by definition of $F_U$ for $U\in \cU$. Since $\{c_K\}_{K\in \cK}$ is an element in $F_U$ for every $U\in \cU$ we derive that
$$c_{K\cup L} \leq c_K+c_L$$
for $K, L$ in $\cK$. This implies $h(K\cup L) \leq h(K) + h(L)$ for $K, L\in \cK$. Moreover, $c_{K\cup L} = c_K + c_L$ if $K\cdot U^{-1}\cap L\cdot U^{-1} = \emptyset$ for some $U\in \cU$. This implies that $c_{K\cup L} = c_K + c_L$ if $K \cap L = \emptyset$ by Lemma \ref{lemma:compactsetsareseparablebyopens}. Thus $h$ admits \textbf{(4)}.
\end{proof}

\begin{proof}[Proof of the theorem]
We fix $h$ as in Lemma \ref{lemma:definitiononcompactsets} and we define $\mu^*:\cP(G)\ra [0,+\infty]$. First if $U$ is an open subset of $G$, then we define
$$\mu^*(U) = \sup_{K\in \cK,\,K\subseteq U}h(K)$$
Note that if $U, V$ are open subsets of $G$ and $U\subseteq V$, then $\mu^*(U)\leq \mu^*(V)$. Thus it makes sense to define
$$\mu^*(A) = \inf \big\{\mu^*(U)\,\big|\,U\mbox{ is an open subset of }G\mbox{ containing }A\big\}$$
for arbitrary subset $A\subseteq G$. Note that $\mu^*(xA) = \mu^*(A)$ by definition of $\mu^*$ and the corresponding property of $h$. By {\cite[Theorem 1.3]{Measuresonlocallycompactspaces}} we have that Borel sets $\cB(G)$ are $\mu^*$-measurable, $\mu^*_{\mid \cB(G)} = \mu$ is a regular Borel measure on $G$. According to this result if $G$ is $\sigma$-compact, then $\mu$ is inner regular. Clearly $\mu$ is left-invariant and since
$$1 = h(Q)\leq \mu(Q)$$
we derive that it is nonzero measure.
\end{proof}

\section{Uniqueness of Haar measure}

\begin{theorem}\label{theorem:}
Let $G$ be a locally compact group. If $\mu_1$ and $\mu_2$ are left (right) Haar measures on $G$, then there exists positive constant $a\in \RR$ such that
$$\mu_1 = a\cdot \mu_2$$
\end{theorem}
\noindent
For the proof we need the following result.

\begin{lemma}\label{lemma:sigmacompactopensubgroup}
Let $G$ be a locally compact group. Then there exists a $\sigma$-compact, open subgroup $H$ of $G$.
\end{lemma}
\begin{proof}[Proof of the lemma]
Let $U$ be an open neighborhood of identity in $G$ such that $\bd{cl}(U)$ is compact. Consider $V = U\cap U^{-1}$. Then $V$ is open neighborhood of identity in $G$ such that $V = V^{-1}$ and $\bd{cl}(V)$ is compact. We define $H = \bigcup_{n\in \NN}V^n$. Then $H$ is an open subgroup of $G$. We have
$$H= G\setminus \left(\bigcup_{g \in G\setminus H}gH\right)$$
and hence $H$ is also a closed subgroup of $G$. Moreover, for every $n\in \NN$ set $\bd{cl}\left(V^n\right)$ is compact in $G$. Since
$$H = \bigcup_{n\in \NN}\left(H\cap \bd{cl}\left(V^n\right)\right)$$
we derive that $H$ is $\sigma$-compact.
\end{proof}

\begin{proof}[Proof of the theorem]
By Lemma \ref{lemma:sigmacompactopensubgroup} there exists an open subgroup $H$ of $G$ that is $\sigma$-compact. We prove now that there exists $a\in \RR$ such that
$${\mu_1}_{\mid \cB(H)} = a\cdot {\mu_2}_{\mid \cB(H)}$$
For this consider $\mu = {\mu_1}_{\mid \cB(H)} + {\mu_2}_{\mid \cB(H)}$ and denote ${\mu_2}_{\mid \cB(H)}$ by $\nu$. Measures $\mu, \nu$ are $\sigma$-finite as they are finite on compact subsets of $H$ and $H$ is $\sigma$-compact space. Moreover, $\nu \ll \mu$ and hence by {\cite[Theorem 5.3]{RadonNikodymHahnJordanLebesguedecomposition}} there exists a Borel function $f:H\ra \CC$ such that
$$\nu(A) = \int_Afd\mu$$
for every Borel subset $A$ in $H$. Since $\mu$ and $\nu$ are nonnegative measures, we derive that $f$ is real and nonnegative $\mu$-almost everywhere. Hence we may assume that $f$ takes only nonnegative real values. We define
$$E = \big\{(x,y)\in H\times H\,\big|\,f(xy) - f(y) \neq 0\big\}$$
Next as $\nu, \mu$ are left-invariant, we deduce that 
$$0 = \nu\big(l_x(A)\big) - \nu(A) = \int_{l_x(A)}fd\mu -\int_{A}fd\mu = \int_{A}\left(f\cdot l_x - f\right)d\mu$$
for every $x\in H$ and $A\in \cB(H)$, where $l_x:H\ra H$ is a continuous map given by left multiplication by $x$. This implies that for all $x\in H$ we have
$$\mu(E_x) = 0$$
By {\cite[Theorem 7.5]{Integration}} applied to measure $\mu\otimes \mu$ on $H\times H$, we deduce that there exists $y\in H$ such that the set
$$E_y = \big\{x\in H\,\big|\,f(xy) -f(y) \neq 0\big\}$$
has measure $\mu$ zero. This implies that $f$ is constant almost everywhere with respect to $\mu$ and thus there exists nonzero $b\in \RR$ such that $\nu = b\cdot \mu$. Hence we have
$${\mu_1}_{\mid \cB(H)} = a\cdot {\mu_2}_{\mid \cB(H)}$$
for $a = (1-b)b^{-1}$. Let $K$ be a compact subset of $G$. Since $H$ is an open subgroup of $G$, there exists $x_1,...,x_n\in G$ such that
$$K\subseteq x_1H\cup ...\cup x_nH$$
and the sum is disjoint. Therefore, we have
$$\mu_1(K) = \sum_{i=1}^n\mu_1\left(K\cap x_iH\right) = \sum_{i=1}^n\mu_1\left(x_i^{-1}K\cap H\right) =a\cdot \sum_{i=1}^n\mu_2\left(x_i^{-1}K\cap H\right) = a\cdot \sum_{i=1}^n\mu_2\left(K\cap x_iH\right) = a\cdot \mu_2(K)$$
This implies that $\mu_1 = a\cdot \mu_2$ because $\mu_1,\mu_2$ are regular Borel measures.
\end{proof}

\section{Modular function and invariance of Haar measure on compact groups}






\small
\bibliographystyle{alpha}
\bibliography{../zzz}

\end{document}