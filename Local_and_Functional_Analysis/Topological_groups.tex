% arara: indent: {overwrite: yes, silent: yes}
\documentclass[10pt]{amsart}
\input ../pree.tex

\begin{document}
\title{Topological groups}
\date{}
\maketitle

\section{Introduction}
\noindent
In these notes we study topological groups which are very important objects in analysis. We heavily use \cite{Topological_Spaces} and \cite{Uniform_Spaces}.

In the first section we introduce topological groups and study their basic properties. Next section studies left and right uniform structures on topological groups and their relation to topological notions. The last section is devoted to abelian topological groups and their completions.

\section{Topological groups}

\begin{definition}
	A group together with a topology such that group operations are continuous is a \textit{topological group}.
\end{definition}

\begin{definition}
	Let $G,H$ be a topological groups. A map $f:G\ra H$ which is continuous homomorphism is a \textit{morphism} of topological groups.
\end{definition}

\begin{definition}
	Let $i:G\hookrightarrow H$ be a morphism of topological groups and a topological embedding. Then $i$ is an \textit{embedding} of topological groups.
\end{definition}
\noindent
Let $G$ be a topological group. For a subset $S$ of $G$ we define
$$S^{-1} = \big\{x^{-1}\,\big|\,x\in S\,\big\}$$

\begin{definition}
	A subset $S$ of a topological group $G$ is \textit{symmetric} if $S$ and $S^{-1}$ coincide.
\end{definition}

\begin{fact}\label{fact:symmetric_neighborhoods_of_unit_generate_base}
	Let $G$ be a topological group and let $O$ be an open neighborhood of identity in $G$. Then there exists an open and symmetric neighborhood $Q$ of identity in $G$ such that $Q\subseteq O$.
\end{fact}
\begin{proof}
	Since $(-)^{-1}:G\ra G$ is homeomorphism, $O \cap O^{-1}$ is symmetric and open neighborhood of identity contained in $O$.
\end{proof}

\begin{fact}\label{fact:closure_of_identity}
	Let $G$ be a topological group and let $1$ be its identity element. Then the following assertions hold.
	\begin{enumerate}[label=\emph{\textbf{(\arabic*)}}, leftmargin=*]
		\item The intersection of all open neighborhoods of identity in $G$ is $\bd{cl}\left(\{1\}\right)$.
		\item $\bd{cl}\left(\{1\}\right)$ is the smallest closed subgroup of $G$ and it is normal.
	\end{enumerate}
\end{fact}
\begin{proof}
	We use Fact \ref{fact:symmetric_neighborhoods_of_unit_generate_base}. We have $x \in \bd{cl}(\{1\})$ if and only if for every open and symmetric neighborhood $O$ of $1$ in $G$ we have $1 \in Ox$. This is equivalent with $x \in O$ for every open and symmetric neighborhood $O$ of $1$ in $G$. Hence \textbf{(1)} holds.

	The assertion \textbf{(2)} is left for the reader.
\end{proof}

\begin{theorem}\label{theorem:quotients_of_topological_groups}
	Let $G$ be a topological group and let $N$ be its normal subgroup. Consider the quotient map $q:G\twoheadrightarrow G/N$ in the category of groups and equip $G/N$ with quotient topology. Then the following assertions holds.
	\begin{enumerate}[label=\emph{\textbf{(\arabic*)}}, leftmargin=*]
		\item $q$ is an open map.
		\item $G/N$ is a topological group and $q$ is a homomorphism of groups.
		\item Let $f:G\ra H$ be a continuous homomorphism of topological groups and suppose that $N \subseteq \Ker(f)$. Then there exists a unique continuous homomorphism $g:G/N\ra H$ such that $g\cdot q = f$.
		\item $N$ is a closed in $G$ if and ony if $G/N$ is Hausdorff.
	\end{enumerate}
\end{theorem}
\noindent
For the proof we need the following result.

\begin{lemma}\label{lemma:Hausdorff_topological_groups}
	Let $G$ be a topological group. Then $G$ is Hausdorff if and only if identity subgroup of $G$ is closed.
\end{lemma}
\begin{proof}[Proof of the lemma]
	If $G$ is Hausdorff, then each singleton subset of $G$ is closed. Hence identity subgroup of $G$ is closed.

	Conversely, assume that identity subgroup in $G$ is closed. Pick two distinct elements $g_1,g_2 \in G$. Since $G$ is a topological group, the map
	\begin{center}
		\begin{tikzpicture}
			[description/.style={fill=white,inner sep=2pt}]
			\matrix (m) [matrix of math nodes, row sep=4em, column sep=4em,text height=1.5ex, text depth=0.25ex]
			{ G\times G     & G\times G & G \\} ;
			\path[->,line width=0.8pt,font=\scriptsize]
			(m-1-1) edge node[above] {$ 1_G\times (-)^{-1} $} (m-1-2)
			(m-1-2) edge node[above] {$ \cdot_G $} (m-1-3);
		\end{tikzpicture}
	\end{center}
	is continuous. Hence there exists an open neighborhood $O$ of identity in $G$ such that $g_1g_2^{-1}\not \in OO^{-1}$. Then
	$$Og_1\cap Og_2 = \emptyset$$
	Thus $G$ is a Hausdorff topological space.
\end{proof}

\begin{proof}[Proof of the theorem]
	Fix an open subset $Q$ of $G$, then the set
	$$q^{-1}\left(q\left(Q\right)\right) = QN$$
	is open. According to the fact that $q:G \twoheadrightarrow G/N$ is a quotient topological map, we infer that $q(Q)$ is open in $G/N$. Hence $q$ is an open map and the proof of \textbf{(1)} is completed.

	Since $q$ is open, we derive that $q\times q$ is open. Since squares
	\begin{center}
		\begin{tikzpicture}
			[description/.style={fill=white,inner sep=2pt}]
			\matrix (m) [matrix of math nodes, row sep=4em, column sep=5em,text height=1.5ex, text depth=0.25ex]
			{ G \times G     & G   & G   & G                            \\
			  G/N \times G/N & G/N & G/N & G/N \\} ;
			\path[->,line width=0.8pt,font=\scriptsize]
			(m-1-1) edge node[above] {$ \cdot_{G} $} (m-1-2)
			(m-1-1) edge node[left] {$ q\times q $} (m-2-1)
			(m-1-2) edge node[right] {$ q $} (m-2-2)
			(m-2-1) edge node[below] {$ \cdot_{G/N} $} (m-2-2)
			(m-1-3) edge node[above] {$ (-)^{-1}_{G} $} (m-1-4)
			(m-1-3) edge node[left] {$ q $} (m-2-3)
			(m-1-4) edge node[right] {$ q $} (m-2-4)
			(m-2-3) edge node[below] {$ (-)^{-1}_{G/N} $} (m-2-4);
		\end{tikzpicture}
	\end{center}
	are commutative, we deduce that the addition $\cdot_{G/N}:G/N \times G/N \ra G/N$ and the inverse map $(-)^{-1}_{G/N}:G/N\ra G/N$ are continuous. Therefore, $G/N$ is a topological group. It follows that $q$ is a morphism of topological groups and hence \textbf{(2)} holds.

	The assertion \textbf{(3)} describes the universal property which follows easily from \textbf{(2)} and the fact that $q$ is a topological quotient.

	For \textbf{(4)} observe that
	$$N\mbox{ is closed subgroup of }G\,\Leftrightarrow\,\mbox{identity subgroup of }G/N\mbox{ is closed }$$
	Thus it suffices to prove that
	$$\mbox{ identity subgroup of }G/N\mbox{ is closed }\,\Leftrightarrow\,G/N\mbox{ is a Hausdorff topological space}$$
	but this is a consequence of Lemma \ref{lemma:Hausdorff_topological_groups}.
\end{proof}

\section{Uniform structures on topological groups}
\noindent
In this section we introduce uniform structures on topological groups and study their properties.

\begin{fact}\label{fact:left_uniform_structure_on_topological_group}
	Let $G$ be a topological group. For every open and symmetric neighborhood $O$ of identity in $G$ we define
	$$L_O = \big\{(g_1,g_2)\in G\times G\,\big|\,g_1^{-1}g_2 \in O\big\}$$
	The collection
	$$\big\{U \in \fD_G\,\big|\,L_O\subseteq U\mbox{ for some open and symmetric neighborhood }O\mbox{ of identity in }G\big\}$$
	is uniform structure on $G$.
\end{fact}
\begin{proof}
	Note that $L_O$ is reflexive and symmetric relation on $G$ for every open and symmetric neighborhood $O$ of identity in $G$. Next it is clear that the collection in the statement is closed under finite intersections and is upward closed in $\fD_G$. Pick $U$ in the collection. Then there exists open and symmetric neighborhood $O$ of identity in $G$ such that $L_O\subseteq U$. Since the multiplication $G\times G\ra G$ is continuous, there exists an open neighborhood $Q$ of unit in $G$ such that $QQ \subseteq O$. By Fact \ref{fact:symmetric_neighborhoods_of_unit_generate_base} we may assume that $Q$ is symmetric. Hence $L_Q\cdot L_Q \subseteq L_O\subseteq U$.
\end{proof}

\begin{definition}
	Let $G$ be a topological group. Then the uniformity introduced above is \textit{left uniform structure} on $G$.
\end{definition}

\begin{proposition}\label{proposition:left_uniform_structure_on_topological_group}
	Let $G$ be a topological group. Then the following assertions hold.
	\begin{enumerate}[label=\emph{\textbf{(\arabic*)}}, leftmargin=3.0em]
		\item Left uniform structure induces the topology on $G$.
		\item For every $g \in G$ maps
		      $$G\ni x \mapsto gx \in G,\,G\ni x \mapsto xg \in G$$
		      are uniform with respect to left uniform structure on $G$.
	\end{enumerate}
\end{proposition}
\begin{proof}
	Suppose that $O$ is an open and symmetric neighborhood of identity in $G$. Then
	$$L_O(x) = xO^{-1} = xO$$
	for every $x \in G$. Hence $Q \subseteq G$ is open set in topology induced by left uniform structure if and only if
	$$Q = \bigcup_{x\in Q}xO_x$$
	where $O_x$ is an open and symmetric neighborhood of the identity in the original topology of $G$ for each $x \in Q$. Fact \ref{fact:symmetric_neighborhoods_of_unit_generate_base} implies that left uniform structure on induces the original topology on $G$. This completes the proof of \textbf{(1)}.

	For $g \in G$ we denote by $l_g$ and $r_g$ maps $G \ni x \mapsto gx \in G$ and $G \ni x \mapsto xg \in G$, respectively. Then
	$$\left(l_g\times l_g\right)^{-1}(L_O) = L_O,\,\left(r_g \times r_g\right)^{-1}(L_O) = L_{gOg^{-1}}$$
	for every open and symmetric neighborhood $O$ of identity in $G$. Thus $l_g$ and $r_g$ are uniform with respect to left uniform structure on $G$. Hence \textbf{(2)} holds.
\end{proof}

\begin{fact}\label{fact:each_continuous_homomorphism_is_uniform_with_respect_to_left_uniformity}
	Let $f:G\ra H$ be a continuous homomorphism of topological groups and let $O$ be an open and symmetric neighborhood of identity in $G$. Then
	$$\left(f\times f\right)^{-1}(L_O) = L_{f^{-1}(O)}$$
	In particular, $f$ is uniform with respect to left uniform structures on $G$ and $H$.
\end{fact}
\begin{proof}
	We have
	$$\left(f\times f\right)^{-1}(L_O) = \big\{(g_1, g_2) \in G\times G\,\big|\,f(g_1)^{-1}f(g_2) \in O\big\} = L_{f^{-1}(O)}$$
	Thus $f$ is a uniform map.
\end{proof}

\begin{corollary}\label{corollary:topological_groups_embeddings_is_uniform_embedding}
	Let $i:G\hookrightarrow H$ be an embedding of topological groups. Then $i$ is a uniform embedding with respect to left uniformities on $G$ and $H$.
\end{corollary}
\begin{proof}
	Since $i$ is a topological embedding, the map $O\mapsto f^{-1}(O)$ defined which takes open and symmetric neighborhoods of $H$ to open and syummetric neighborhoods of $G$ is surjective. By Fact \ref{fact:each_continuous_homomorphism_is_uniform_with_respect_to_left_uniformity} we derive
	$$L_{i^{-1}(O)} = \left(i\times i\right)^{-1}(L_O)$$
	for every open and symmetric neighborhood of $H$. Hence $i$ induces left uniform structure on $G$. Thus $i$ is an embedding of $G$ into $H$ with respect to left uniform structure.
\end{proof}

\begin{corollary}\label{corollary:uniform_Kolmogorov_quotients_of_topological_groups}
	Let $G$ be a topological group and let $q:G\twoheadrightarrow \tilde{G}$ be the quotient of $G$ with respect to the closure of identity in $G$. If $\tilde{G}$ and $G$ are considered with their left uniform structures, then $q$ is a uniform Kolmogorov quotient.
\end{corollary}
\begin{proof}
	We denote by $1$ the identity of $G$. It follows from Fact \ref{fact:closure_of_identity} that intersection of $L_O$ for all open and symmetric neighborhoods $O$ of identity in $G$ consists of $(g_1,g_2) \in G\times G$ such that $g_1g_2^{-1} \in \bd{cl}\left(\{1\}\right)$. Pick an open and symmetric neighborhood $O$ of identity in $G$. Then $q(O)$ is an open and symmetric neighborhood of identity in $\tilde{G}$ and such that $q^{-1}(q(O)) = O$. Thus
	$$\left(q \times q\right)^{-1}(L_{q(O)}) = L_O$$
	Hence we derive that $q$ is a uniform Kolmogorov quotient of $G$.
\end{proof}

\begin{remark}\label{remark:right_uniform_structure}
	Let $G$ be a topological group. For every open and symmetric neighborhood $O$ of identity in $G$ we define
	$$R_O = \big\{(g_1,g_2)\in G\times G\,\big|\,g_1g_2^{-1} \in O\big\}$$
	The collection
	$$\big\{U \in \fD_G\,\big|\,\mbox{there exists open and symmetric neighborhood }O\mbox{ of identity in }G\mbox{ such that }R_O\subseteq U\big\}$$
	is a uniform structure on $G$. It is called \textit{right uniform structure} on $G$. Analogical version of results in this section hold for right uniform structures. We left details for the reader.
\end{remark}

\begin{fact}\label{fact:inverse_is_an_isomorphism_of_left_and_right_uniform_structures}
	Let $G$ be a topological group. Then $(-)^{-1}:G\ra G$ is a uniform map between left and right uniformities on $G$.
\end{fact}

\begin{definition}
	Let $G$ be a topological group and let $\cF$ be a filter on $G$. If $\cF$ is Cauchy with respect to left (right) uniform structure on $G$, then $\cF$ is \textit{left} (\textit{right}) Cauchy.
\end{definition}

\begin{definition}
	Let $G$ be a topological group which is complete with respect to its left (right) uniform structure. Then $G$ is \textit{left} (\textit{right}) complete.
\end{definition}

\section{Abelian topological groups}
\noindent
As usual for abelian groups we use additive notation.

\begin{proposition}\label{proposition:abelian_topological_groups_are_uniform_groups}
	Let $A$ be an abelian topological group. Then the following assertions hold.
	\begin{enumerate}[label=\emph{\textbf{(\arabic*)}}, leftmargin=3.0em]
		\item Left and right uniform structure on $A$ coincide.
		\item Group operations on $A$ are uniform maps.
	\end{enumerate}
\end{proposition}
\begin{proof}
	Left for the reader.
\end{proof}
\noindent
From now when discussing uniform structures and uniform maps in the context of abelian topological groups we consider them with left uniform structures (which coincide with right uniform structures according to Proposition \ref{proposition:abelian_topological_groups_are_uniform_groups}).

\begin{definition}
	Let $A$ be an abelian topological group. Then $A$ is \textit{complete} if it is complete as the uniform space.
\end{definition}

\begin{theorem}\label{theorem:completion_of_topological_group}
	Let $A$ be a Hausdorff abelian topological group. There exists a complete Hausdorff abelian topological group $\overline{A}$ and a continuous embedding of topological groups $A \hookrightarrow \overline{A}$ with dense image.
\end{theorem}
\begin{proof}
	General results on uniform spaces show that there exists an embedding of uniform spaces $i:A\hookrightarrow \overline{A}$ with dense image such that $\overline{A}$ is a Hausdorff complete uniform space. Let $+_A:A\times A\ra A$ be the addition on $A$ and let $(-)^{-1}_A:A\ra A$ be the map taking opposite element. Both these maps are uniform by Proposition \ref{proposition:abelian_topological_groups_are_uniform_groups}. By results on extension of uniform maps there exist unique uniform maps $+_{\overline{A}}:\overline{A}\times \overline{A}\ra \overline{A}$ and $(-)^{-1}_{\overline{A}}:\overline{A}\ra \overline{A}$ such that the diagrams
	\begin{center}
		\begin{tikzpicture}
			[description/.style={fill=white,inner sep=2pt}]
			\matrix (m) [matrix of math nodes, row sep=4em, column sep=4em,text height=1.5ex, text depth=0.25ex]
			{ A\times A                        & A            &  & A            & A                                                          \\
			  \overline{A} \times \overline{A} & \overline{A} &  & \overline{A} & \overline{A}                      \\} ;
			\path[->,line width=0.8pt,font=\scriptsize]
			(m-1-1) edge node[above] {$ +_A $} (m-1-2)
			(m-2-1) edge node[below] {$ +_{\overline{A}} $} (m-2-2)
			(m-1-4) edge node[above] {$ (-)^{-1}_A $} (m-1-5)
			(m-2-4) edge node[below] {$ (-)^{-1}_{\overline{A}} $} (m-2-5);
			\path[right hook->,line width=0.8pt,font=\scriptsize]
			(m-1-1) edge node[left] {$ i\times i $} (m-2-1)
			(m-1-2) edge node[right] {$ i $} (m-2-2)
			(m-1-4) edge node[left] {$ i $} (m-2-4)
			(m-1-5) edge node[right] {$ i $} (m-2-5);
		\end{tikzpicture}
	\end{center}
	are commutative. Next $i$ has dense image, all maps considered above are continuous and $\overline{A}$ is Hausdorff. Thus $+_{\overline{A}}$ and $-_{\overline{A}}$ make $\overline{A}$ into an abelian topological group and $i$ into a homomorphism of groups.

	Next let $\hat{A}$ be $\overline{A}$ with left unifrom structure. Then $\hat{A}$ and $\overline{A}$ are isomorphic topological groups. Since $i$ is a continuous embedding, we may consider it as a continuous embedding $A\hookrightarrow \hat{A}$. According to Corollary \ref{corollary:topological_groups_embeddings_is_uniform_embedding} we may consider $i$ as an embedding $j:A\hookrightarrow \hat{A}$ of uniform spaces. By general result on extension of uniform maps there exists a unique uniform map $h:\hat{A}\ra \overline{A}$ such that the triangle
	\begin{center}
		\begin{tikzpicture}
			[description/.style={fill=white,inner sep=2pt}]
			\matrix (m) [matrix of math nodes, row sep=3em, column sep=2em,text height=1.5ex, text depth=0.25ex]
			{ \hat{A} &   & \overline{A}           \\
			          & A & \\} ;
			\path[->,line width=0.8pt,font=\scriptsize]
			(m-1-1) edge node[above] {$ h $} (m-1-3);
			\path[right hook->,line width=0.8pt,font=\scriptsize]
			(m-2-2) edge node[left = 3pt, below = 1pt] {$ j $} (m-1-1)
			(m-2-2) edge node[right = 3pt, below = 1pt] {$ i $} (m-1-3);
		\end{tikzpicture}
	\end{center}
	is commutative. It follows that $h$ is the identity set theoretically. It follows that if $\cF$ is a Cauchy filter on $\hat{A}$, then it is also Cauchy on $\overline{A}$. Thus it is convergent in $\overline{A}$ and hence also in $\hat{A}$. This shows that $\hat{A}$ is complete.
\end{proof}


\section{Completions of topological groups}
\noindent
First we need to prove the following result.

\begin{proposition}\label{proposition:multiplication_preserves_one_sided_Cauchy_filters}
	Let $G$ be a topological group. Then the multiplication $G\times G\ra G$ sends left (right) Cauchy filters on $G\times G$ to left (right) Cauchy filters.
\end{proposition}
\begin{proof}
	We prove proposition for left uniform structures.

	Let $\cF$ be a left Cauchy filter in $G\times G$ and let $O$ be an open and symmetric neighborhood of $G$. Let $\cF_l$ and $\cF_r$ be images of $\cF$ under left and right projections $G\times G\ra G$. Then $\cF_l$ and $\cF_r$ are left Cauchy filters on $G$. Fix open and symmetric neighborhood $Q$ of $G$ such that $QQQ \subseteq O$. There exists $F_r \in \cF_r$ such that
	$$F_r \times F_r \subseteq L_Q$$
	Next fix $x \in F_r$. Next there exists $F_l \in \cF_l$ such that
	$$F_l\times F_l \subseteq L_{xQx^{-1}}$$
	Define $F = F_l\times F_r$ and $F \in \cF$. Pick $(l_1,r_1),(l_2,r_2) \in F$. Then
	$$(l_1r_1)^{-1}(l_2r_2) = r_1^{-1}(l_1^{-1}l_2)r_2 = (r_1^{-1}x)x^{-1}(l_1^{-1}l_2)x(x^{-1}r_2) \subseteq Qx^{-1}\left(xQx^{-1}\right)xQ = QQQ \subseteq O$$
	Hence the preimage of $L_O$ under the multiplication of $G$ contains $F \times F$. It follows that the image of $\cF$ under the multiplication of $G$ is a Cauchy filter on $G$.
\end{proof}

\begin{definition}
	Let $G$ be a topological group. A left (right) complete group $\overline{G}$ and an embedding $G \hookrightarrow \overline{G}$ of topological groups with dense image is a \textit{left} (\textit{right}) group completion of $G$.
\end{definition}

\begin{theorem}\label{theorem:criterion_for_existence_of_one_sided_completions}
	Let $G$ be a topological group. Then the following are equivalent.
	\begin{enumerate}[label=\emph{\textbf{(\roman*)}}, leftmargin=*]
		\item $G$ admits left (right) completion.
		\item Left Cauchy filters and right Cauchy filters on $G$ coincide.
	\end{enumerate}
\end{theorem}
\begin{proof}
	Suppose that $G$ admits left completion $i:G \hookrightarrow \overline{G}$. Let $\cF$ be a left Cauchy filter on $G$. We define
	$$\cF^{-1} = \big\{F^{-1}\,\big|\,F\in \cF\big\}$$
	Then there exists a continuous map $\overline{G}\ra \overline{G}$ such that the diagram
	\begin{center}
		\begin{tikzpicture}
			[description/.style={fill=white,inner sep=2pt}]
			\matrix (m) [matrix of math nodes, row sep=3em, column sep=3em,text height=1.5ex, text depth=0.25ex]
			{  G            & G                                                          \\
			   \overline{G} & \overline{G}                      \\} ;
			\path[->,line width=0.8pt,font=\scriptsize]
			(m-1-1) edge node[above] {$ (-)^{-1} $} (m-1-2)
			(m-2-1) edge node[below] {$  $} (m-2-2);
			\path[right hook->,line width=0.8pt,font=\scriptsize]
			(m-1-1) edge node[left] {$ i $} (m-2-1)
			(m-1-2) edge node[right] {$ i $} (m-2-2);
		\end{tikzpicture}
	\end{center}
	is commutative. From the commutativity it follows that $i(\cF^{-1})$ is convergent in $\overline{G}$. Since $\overline{G}$ is left complete and $i$ is an embedding of uniform spaces (Corollary \ref{corollary:topological_groups_embeddings_is_uniform_embedding}), we derive that $\cF^{-1}$ is a left Cauchy filter on $G$. According to Fact \ref{fact:inverse_is_an_isomorphism_of_left_and_right_uniform_structures} the map $(-)^{-1}$ induces bijection between left and right Cauchy filters on $G$. Thus $\cF$ is a right Cauchy filter on $G$. This completes the proof of $\textbf{(i)}\Rightarrow \textbf{(ii)}$.


\end{proof}



























\small
\bibliographystyle{apalike}
\bibliography{../zzz}

\end{document}

