\input ../pree.tex

\begin{document}

\title{Conditional Expectations}
\date{}
\maketitle
\section{Introduction}
These notes introduce notion of conditional expectation of a random variable and discuss its properties. Aside basic measure-theoretic and probabilistic tools we use here Radon-Nikodym theorem {\cite[Theorem 5.1]{Radon_Nikodym_Hahn_Jordan_Lebesgue_decomposition}}.

\section{Existence of conditional expectations}
\noindent
Fix a probability space $(\Omega, \cF, P)$.

\begin{theorem}\label{theorem:existenceofconditionalexpectationforintegrable}
Let $X:\Omega \ra \CC$ be an integrable random variable and $\cG$ be a $\sigma$-subalgebra of $\cF$. Then there exists $\cG$-measurable and integrable function $f:\Omega \ra \CC$ such that
$$\int_GXdP = \int_G f dP$$
for every $G$ in $\cG$. Moreover, the set of all $\cG$-measurable functions having the property described by the system of equations above is
$$\big\{g:\Omega \ra \CC\,\big|\,g\mbox{ is }\cG\mbox{-measurable and }f(\omega)=g(\omega)\mbox{ almost surely}\big\}$$
\end{theorem}
\begin{proof}
We define a complex measure $\nu:\cG\ra \CC$ by formula
$$\nu(G) = \int_G X\,dP$$
for $G\in \cG$. Since $\nu \ll P_{\mid \cG}$ and by Radon-Nikodym theorem, we derive that there exists a $\cG$-measurable function $f:\Omega \ra \CC$ such that
$$\nu(G) = \int_G f\,dP$$
The last statement is clear and is left for the reader as an exercise.
\end{proof}

\begin{definition}
Let $X:\Omega \ra \CC$ be an integrable random variable and $\cG$ be a $\sigma$-subalgebra of $\cF$. Suppose that $f:\Omega \ra \CC$ is a $\cG$-measurable and integrable function $f:\Omega \ra \CC$ such that 
$$\int_GX\,dP = \int_G f\, dP$$
for every $G$ in $\cG$. Then $f$ is called \textit{a version of the conditional expectation of $X$ with respect to $\cG$}.
\end{definition}
\noindent
No we define important special case.

\begin{definition}
Let $\cG$ be a $\sigma$-subalgebra of $\cF$. Let $f:\Omega \ra \CC$ be a $\cG$-measurable, integrable function such that
$$P(A\cap G) = \int_G f\, dP$$
for every $G\in \cG$. Then $f$ is called \textit{a version of conditional probability of $A$ with respect to $\cG$}.
\end{definition}
\noindent
Now that we discuss basic existence and uniqueness results concerning conditional expectation let us introduce some notation. Let $(\Omega, \cF, P)$ be a probability space, $X:\Omega \ra \CC$ be an integrable random variable and $\cG$ be a $\sigma$-subalgebra of $\cF$. We denote any version of the conditional expectation of $X$ with respect to $\cG$ by a symbol
$$\EE[X\,|\,\cG]$$ 
and for every set $A\in \cF$ we denote by
$$P[A\,|\,\cG]$$
any version of conditional probability of $A$ with respect to $\cG$. We also often omit the word version and speak about conditional expectation and conditional probabilities. Nevertheless one should always keep in mind that these are $\cG$-measurable and integrable functions defined up to sets in $\cG$ of probability zero.

\section{Properties of conditional expectation}
\noindent
Let $(\Omega, \cF, P)$ be a probability space and $\cG$ be a $\sigma$-sublagebra of $\cF$.

\begin{theorem}\label{theorem:mainpropertiesofconditionalexpectation}
Let $Y$, $X$, $\{X_n\}_{n\in \NN}$ be integrable random variables $\Omega \ra \CC$. Then the following results hold.
\begin{enumerate}[label=\emph{\textbf{(\arabic*)}}, leftmargin=*]
\item If $X,Y$ have real values and $X \leq Y$ almost surely, then $\EE[X\,|\,\cG] \leq \EE[Y\,|\,\cG]$ almost surely.
\item $\EE[a\cdot X+b\cdot Y\,|\,\cG] = a\cdot \EE[X\,|\,\cG]+b\cdot \EE[Y\,|\,\cG]$ almost surely for $a$, $b\in \CC$.
\item $\big|\EE[X\,|\,\cG]\big|\leq \EE[|X|\,|\,\cG]$ almost surely.
\item If $\{X_n\}_{n\in \NN}$ converges almost surely to $X$ and 
$$|X_n|\leq Y,\,|X|\leq Y$$
almost surely for every $n\in \NN$, then $\big\{\EE[X_n\,|\,\cG]\big\}_{n\in \NN}$ converges almost surely to $\EE[X\,|\,\cG]$.
\end{enumerate}
\end{theorem}
\begin{proof}
For the proof of \textbf{(1)}. We have 
$$\int_G\EE[X\,|\,\cG]\,dP = \int_GX\,dP \leq \int_GY\,dP = \int_G\EE[Y\,|\,\cG]\,dP$$
Since conditional expectations with respect to $\cG$ are $\cG$-measurable, we deduce that $\EE[X\,|\,\cG] \leq \EE[Y\,|\,\cG]$.\\
Next we prove \textbf{(2)}. Pick $a, b\in \CC$. We have
$$\int_G\EE[a\cdot X+b\cdot Y\,|\,\cG]\,dP = \int_G(a\cdot X+b\cdot Y)\,dP = a\cdot \int_GX\,dP+b\cdot \int_GY\,dP =$$
$$= a\cdot \int_G\EE[X\,|\,\cG]\,dP+b\cdot \int_G\EE[Y\,|\,\cG]\,dP =\int_G\big(a\cdot \EE[X\,|\,\cG]+b\cdot \EE[Y\,|\,\cG]\big)\,dP$$
for every $G\in \cG$. Since conditional expectations with respect to $\cG$ is $\cG$-measurable, we derive that $\EE[a\cdot X+b\cdot Y\,|\,\cG] = a\cdot \EE[X\,|\,\cG]+b\cdot \EE[Y\,|\,\cG]$.\\
For \textbf{(3)} assume pick $\alpha \in \CC$ such that $|\alpha| = 1$ and
$$\alpha \cdot \EE[X\,|\,\cG] = \big|\EE[X\,|\,\cG]\big|$$
Then
$$\big|\EE[X\,|\,\cG]\big| = \alpha \cdot \EE[X\,|\,\cG] = \EE[\alpha \cdot X\,|\,\cG] =  \EE[\mathrm{re}\left(\alpha \cdot X\right)\,|\,\cG] \leq \EE[|\alpha \cdot X|\,|\,\cG] = \EE[|X|\,|\,\cG]$$
almost surely. Thus \textbf{(3)} holds.\\
Finally we prove that \textbf{(4)}. Set $Z_n = \sup_{k\leq n}|X_k-X|$. Then $Z_n$ is nonnegative measurable function and $\lim_{n\ra +\infty}Z_n = 0$. Moreover, $\big\{Z_n\big\}_{n\in \NN}$ is pointwise decreasing and dominated by $2\cdot |Y|$. Thus by dominated convergence theorem
$$\lim_{n\ra +\infty}\int_{\Omega} Z_n \,dP= 0$$
Next $\big\{\EE[Z_n\,|\,\cG]\big\}_{n\in \NN}$ are measurable, almost surely pointwise decreasing and nonnegative functions. Moreover, we derive that
$$\lim_{n\ra +\infty}\int_{\Omega} \EE[Z_n\,|\,\cG]\,dP =\lim_{n\ra +\infty}\int_{\Omega} Z_n\,dP = 0$$
and hence 
$$\int_{\Omega} \left(\lim_{n\ra +\infty}\EE[Z_n\,|\,\cG]\right)\,dP = 0$$
This implies that $\lim_{n\ra +\infty}\EE[Z_n\,|\,\cG]=0$ almost surely. By \textbf{(1)} and \textbf{(3)} we have
$$\sup_{k\geq n}\big|\EE[X_k\,|\,\cG]-\EE[X\,|\,\cG]\big|= \sup_{k\geq n}\EE[|X_k-X|\,\big|\,\cG] \leq \EE[Z_n\,|\,\cG] $$
Therefore
$$\lim_{n\ra +\infty}\sup_{k\geq n}\big|\EE[X_k\,|\,\cG]-\EE[X\,|\,\cG]\big| = 0$$
and hence $\lim_{n\ra +\infty}\EE[X_n\,|\,\cG] = \EE[X\,|\,\cG]$.
\end{proof}

\begin{theorem}\label{theorem:multiplicationofconditionalexpectation}
Let $X$, $Y:\Omega \ra \CC$ be random variables such that $X$, $Y\cdot X$ are integrable and $Y$ is $\cG$-measurable. Then
$$\EE[Y\cdot X\,|\,\cG] = Y\cdot \EE[X\,|\,\cG]$$
\end{theorem}
\begin{proof}
First note that the result is clear for $Y = \mathbb{1}_G$ where $G\in \cG$ and also for $\RR_{>0}$-linear combination of such functions. Next suppose that $Y:\Omega \ra \CC$ is integrable $\cG$-measurable function with nonnegative real values. Then there exists an nondecreasing sequence $\{Y_n\}_{n\in \NN}$ of positive combinations of indicator functions of sets in $\cG$ that converges to $Y$. Note that $|Y_n\cdot X|\leq |Y_n|\cdot |X|$ and $|Y_n\cdot \EE[X\,|\,\cG]|\leq Y\cdot |\EE[X\,|\,\cG]|$ for $n\in \NN$. Then by dominated convergence theorem
$$\int_G \EE[Y\cdot X\,|\,\cG]\,dP = \int_G Y\cdot X\,dP = \lim_{n\ra +\infty}\int_G Y_n\cdot X\,dP = \lim_{n\ra +\infty} \int_G Y_n\cdot \EE[X\,|\,\cG]\,dP = \int_G Y\cdot \EE[X\,|\,\cG]\,dP$$
for every $G\in \cG$. This implies that $\EE[Y\cdot X\,|\,\cG] = Y\cdot \EE[X\,|\,\cG]$. Suppose now that $Y:\Omega \ra \CC$ is a $\cG$-measurable and integrable random variable taking real values. We write $Y_+ = \max\{0,Y\}$ and $Y_- = \min\{0,Y\}$. Then
$$\EE[Y\cdot X\,|\,\cG] = \EE[Y_+\cdot X\,|\,\cG]+\EE[Y_-\cdot X\,|\,\cG]= Y_+\cdot\EE[ X\,|\,\cG] + Y_-\cdot\EE[ X\,|\,\cG] = Y\cdot \EE[X\,|\,\cG]$$
This proves the assertion for every real-valued, integrable and $\cG$-measurable random variable $Y$. Finally suppose that $Y$ is complex valued, $\cG$-measurable and integrable. Write $Y = Y_r + i\cdot Y_i$ for real valued $Y_r, Y_i$ random variables. Then $Y_r, Y_i$ are $\cG$-measurable and integrable. Hence
$$\EE[Y\cdot X\,|\,\cG] = \EE[Y_r\cdot X\,|\,\cG]+i\cdot \EE[Y_i\cdot X\,|\,\cG]= Y_r\cdot\EE[ X\,|\,\cG] +i\cdot Y_i\cdot\EE[ X\,|\,\cG] = Y\cdot \EE[X\,|\,\cG]$$
Thus assertion holds for any $\cG$-measurable, integrable random variable $Y:\Omega \ra \CC$. Suppose now that $Y$ is $\cG$-measurable and $Y\cdot X$, $X$ are integrable. Define $W_n=\{\omega \in \Omega\,|\,|Y(\omega)|\leq n\}$ and $Y_n = \mathbb{1}_{W_n}\cdot Y$. Then $\{Y_n\}_{n\in \NN}$ is a sequence of integrable $\cG$-measurable random variables convergent to $Y$ and $|Y_n\cdot X|\leq |Y\cdot X|$ for every $n\in \NN$. Hence
$$Y\cdot \EE[X\,|\,\cG] = \lim_{n\ra +\infty}Y_n\cdot \EE[X\,|\,\cG] = \lim_{n\ra +\infty}\EE[Y_n\cdot X\,|\,\cG] = \EE[Y\cdot X\,|\,\cG]$$
and the last equality follow from \textbf{(4)} of Theorem \ref{theorem:mainpropertiesofconditionalexpectation}
\end{proof}

\begin{theorem}[Tower Property]\label{theorem:towerproperty}
Let $\cG_2\subseteq \cG_1\subseteq \cF$ be $\sigma$-algebras and $X:\Omega \ra \CC$ be an integrable random variable. Then
$$\EE[\EE[X\,|\,\cG_1]\,|\,\cG_2] = \EE[X\,|\,\cG_2]$$
\end{theorem}
\begin{proof}
Fix $G\in \cG_2$. Then also $G\in \cG_1$ and
$$\int_G\EE[\EE[X\,|\,\cG_1]\,|\,\cG_2]\,dP = \int_G \EE[X\,|\,\cG_1]\,dP = \int_G X\,dP = \int_G \EE[X\,|\,\cG_2]\,dP$$ 
Therefore, we derive that $\EE[\EE[X\,|\,\cG_1]\,|\,\cG_2] = \EE[X\,|\,\cG_2]$.
\end{proof}

\begin{theorem}\label{theorem:Jenseninequality}
Let $\cG$ be a $\sigma$-subalgebra of $\cF$, $X:\Omega \ra \RR$ be an integrable random variable and $\phi:\RR\ra \RR$ be a convex function. Suppose that $\phi(X)$ is integrable. Then
$$\phi\left(\EE[X\,|\,\cG]\right)\leq \EE[\phi(X)\,|\,\cG]$$
\end{theorem}
\begin{proof}
Let $L_{\phi}$ be a set of functions $\RR\ni x\mapsto a\cdot x+b\in \RR$ for $a$, $b\in \RR$ such that $a\cdot x+b\leq \phi(x)$ for every $x\in \RR$. Since $\phi$ is convex, we derive that for every $x\in \RR$ we have $\phi(x) = \sup_{l\in L_{\phi}}l(x)$. Hence
$$\phi\left(\EE[X\,|\,\cG]\right)=\sup_{l\in L_{\phi}} l\left(\EE[X\,|\,\cG]\right) = \sup_{l\in L_{\phi}}\EE[l(X)\,|\,\cG]\leq \EE[\phi(X)\,|\,\cG]$$
\end{proof}




































































\small
\bibliographystyle{apalike}
\bibliography{../zzz}

\end{document}