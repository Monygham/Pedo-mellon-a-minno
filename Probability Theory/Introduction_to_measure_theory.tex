\input pree.tex

\begin{document}

\title{Introduction to measure theory}
\date{}
\maketitle

\section{Families of sets}
\noindent
In this section we study various families of sets that are important in the development of measure theory.

\begin{definition}
Let $X$ be a set and $\cF\subseteq \cP(X)$ be a family of subsets of $X$. We define the following types of families.
\begin{enumerate}[label=\textbf{(\arabic*)}, leftmargin=*]
\item $\cF$ is \textit{an algebra} if it contains $X$ and is closed under finite unions, intersections and completions. 
\item $\cF$ is \textit{a $\sigma$-algebra} if it is an algebra and is closed under countable unions.
\item $\cF$ is \textit{a monotone family} if it is closed under unions of countable non-decreasing sequences and under intersections of countable non-increasing sequences.
\item $\cF$ is \textit{a $\pi$-system} if it is closed under finite intersections.
\item $\cF$ is \textit{a $\lambda$-system} if it contains $X$ and is closed under complements and countable disjoint unions. 
\end{enumerate}
\end{definition}

\begin{fact}\label{intersections}
Let $X$ be a set and $\{\cF_i\}_{i\in I}$ be a class of families subsets of $X$. Suppose that $\cF_i$ is an algebra ($\sigma$-algebra, monotone family, $\pi$-system, $\lambda$-system) for every $i\in I$. Then the intersection $\bigcap_{i\in I}\cF_i$
is an algebra ($\sigma$-algebra, monotone family, $\pi$-system, $\lambda$-system).
\end{fact}
\begin{proof}
Left as an exercise.
\end{proof}

\begin{definition}
Let $\cF$ be a family of subsets of $X$. We denote by $\sigma(\cF)$, $\lambda(\cF)$ and $\cM(\cF)$ intersections of all $\sigma$-algebras, $\lambda$-systems and monotone families containing $\cF$, respectively. We call them \textit{$\sigma$-algebra, $\lambda$-system and monotone family generated by $\cF$}, respectively. 
\end{definition}

\begin{theorem}[Dynkin's $\pi$-$\lambda$ lemma]\label{theorem:dynkinlemma}
Let $X$ be a set and $\cP$ be a $\pi$-system of its subsets. Then $\lambda(\cP)=\sigma(\cP)$.
\end{theorem}
\noindent
For the proof we need the following result.

\begin{lemma}\label{theorem:dynkinlemmal}
Let $\cL$ be a $\lambda$-system. Then for every $A\in \cL$ family
$$\cL_A=\big\{B\in \cL\,\big|\,A\cap B\in \cL \big\}$$
is a $\lambda$-system.
\end{lemma}
\begin{proof}[Proof of the lemma]
Since $A\in \cL$, we have $X\in \cL_A$. Suppose now that $B\in \cL_A$. Then $A\cap B\in \cL$. Since $X\setminus A\in \cL$, we derive that also $(A\cap B)\cup (X\setminus A)\in \cL$ and hence
$$A\cap (X\setminus B)=X\setminus \left(\left(A\cap B\right)\cup (X\setminus A)\right)\in \cL$$
Thus $X\setminus B\in \cL_A$. Finally note that $\cL_A$ is closed under countable disjoint unions. 
\end{proof}

\begin{proof}[Proof of the theorem]
Fix $A\in \cP$. Define $\cL_A$ as in Lemma \ref{theorem:dynkinlemmal} with $\cL = \lambda(\cP)$. Then $\cL_A$ is a $\lambda$-system. Moreover, $\cL_A$ contains $\cP$. Hence $\cL_A=\lambda(\cP)$. This shows that $\lambda(\cP)$ is closed under intersections with members of $\cP$. Now fix $A\in \lambda(\cP)$ and define $\cL_A$ as in Lemma \ref{theorem:dynkinlemmal} with $\cL = \lambda(\cP)$. Then $\cP\subseteq \cL_A$ and $\cL_A$ is a $\lambda$-system. Thus $\cL_A=\lambda(\cP)$. This proves that $\lambda(\cP)$ is a $\pi$-system. A $\pi$-system that is simultaneously a $\lambda$-system is a $\sigma$-algebra. Thus $\sigma(\cP)\subseteq \lambda(\cP)$. Since it is clear that $\lambda(\cP)\subseteq \sigma(\cP)$, we derive that $\lambda(\cP)=\sigma(\cP)$.
\end{proof}

\begin{theorem}[Halmos's lemma on monotone classes]\label{theorem:monotoneclasses}
Let $X$ be a set and $\cA$ be an algebra of its subsets. Then $\cM(\cA)=\sigma(\cA)$.
\end{theorem}
\noindent
For the proof we need the following easy results. Their proofs are left to the reader.

\begin{lemma}\label{lemma:firstmonotoneclasses}
Let $\cM$ be a monotone family. Then for every $A\in \cM$ family
$$\cM_A=\big\{B\in \cM\,\big|\,A\cap B\in \cM\big\}$$
is monotone.
\end{lemma}

\begin{lemma}\label{lemma:secondmonotoneclasses}
Let $\cM$ be a monotone family. Then a family
$$\cM^c=\big\{A\in \cM\,\big|\,X\setminus A\in \cM\big\}$$
is monotone.
\end{lemma}

\begin{proof}[Proof of the theorem]
Fix $A\in \cA$. Define $\cM_A$ as in Lemma \ref{lemma:firstmonotoneclasses} with $\cM = \cM(\cA)$. Then $\cM_A$ is a monotone family. Moreover, $\cM_A$ contains $\cA$. Hence $\cM_A=\cM(\cA)$. This shows that $\cM(\cA)$ is closed under intersections with members of $\cA$. Now fix $A\in \cM(\cA)$ and define $\cM_A$ as in Lemma \ref{lemma:firstmonotoneclasses} with $\cM = \cM(\cA)$. Then $\cA\subseteq \cM_A$ and $\cM_A$ is a monotone family. Thus $\cM_A=\cM(\cA)$. This proves that $\cM(\cA)$ is closed under finite intersections. According to Lemma \ref{lemma:secondmonotoneclasses} we derive that $\cM(\cA)^c$ is a monotone family and contains $\cA$. Hence $\cM(\cA)^c=\cM(\cA)$ and $\cM(\cA)$ is closed under complements. Therefore, $\cM(\cA)$ is a $\sigma$-algebra. Thus $\sigma(\cA)\subseteq \cM(\cA)$. Since it is clear that $\cM(\cA)\subseteq \sigma(\cA)$, we derive that $\cM(\cA)=\sigma(\cA)$.
\end{proof}

\section{Measurable spaces and measures}

\begin{definition}
A pair $(X,\Sigma)$ consisting of a set $X$ together with a $\sigma$-algebra $\Sigma$ of its subsets is called \textit{a measurable space}.
\end{definition}

\begin{definition}
Let $(X_1,\Sigma_1)$ and $(X_2,\Sigma_2)$ be measurable spaces. A function $f:X_1\ra X_2$ is called \textit{a measurable map} if $f^{-1}(A)\in \Sigma_1$ for every $A\in \Sigma_1$.
\end{definition}
\noindent
Measurable spaces and their morphisms form a category. 

\begin{definition}
Let $X$ be a set and $\Sigma$ be an algebra of its subsets. A function $\mu:\Sigma \ra [0,+\infty]$ such that $\mu(\emptyset)=0$ and 
$$\mu\left(\bigcup_{n=0}^mA_n\right)=\sum_{n\in \NN}\mu(A_n)$$
for every family  $\{A_n\}_{0\leq n\leq m}$ of pairwise disjoint subsets in $\Sigma$ is called \textit{an additive function}. If in addition $\mu$ satisfies
$$\mu\left(\bigcup_{n\in \NN}A_n\right)=\sum_{n\in \NN}\mu(A_n)$$
for every family  $\{A_n\}_{n\in \NN}$ of pairwise disjoint subsets in $\Sigma$ such that $\bigcup_{n\in \NN}A_n\in \Sigma$, then $\mu$ is called \textit{a $\sigma$-additive function}. Moreover, if $\mu:\Sigma \ra [0,+\infty]$ is a $\sigma$-additive function and $\Sigma$ is a $\sigma$-algebra, then $\mu$ is called \textit{a measure}. 
\end{definition}

\begin{definition}
A tuple $(X,\Sigma,\mu)$ consisting of a measurable space $\left(X,\Sigma\right)$ and a measure $\mu:\Sigma \ra [0,+\infty]$ is called \textit{a space with measure}.
\end{definition}

\begin{definition}
Let $(X,\Sigma,\mu)$ be a space with measure. We say that it is \textit{finite} if $\mu(X)$ is finite. We say that it is \textit{$\sigma$-finite} if there exists a sequence $\{X_n\}_{n\in \NN}$ of subsets of $\Sigma$ such that $\mu(X_n)$ is finite for every $n\in \NN$ and $X=\bigcup_{n\in \NN}X_n$.
\end{definition}

\begin{theorem}\label{theorem:uniquenessonpisystem}
Let $(X,\Sigma)$ be a measurable space and $\mu_1$, $\mu_2:\Sigma\ra [0,+\infty]$ be measures such that $\mu_1(X)=\mu_2(X)$ is finite. Suppose that $\cP$ is a $\pi$-system of subsets of $X$ such that $\Sigma=\sigma(\cP)$ and $\mu_1(A)=\mu_2(A)$ for every $A\in \cP$. Then $\mu_1=\mu_2$.
\end{theorem}
\begin{proof}
Define $\cF=\big\{A\in \Sigma\,\big|\,\mu_1(A)=\mu_2(A)\big\}$. Straightforward verification shows that $\cF$ is a $\lambda$-system. By assumption $\cP\subseteq \cF$. Therefore, $\lambda(\cP)\subseteq \cF$. By Theorem \ref{theorem:monotoneclasses} we deduce that $\Sigma=\sigma(\cP)=\lambda(\cP)\subseteq \cF\subseteq \Sigma$. Hence $\cF=\Sigma$.
\end{proof}

\begin{definition}
Let $(X_1,\Sigma_1,\mu_1)$ and $(X_2,\Sigma_2,\mu_2)$ be spaces with measures. A function $f:X_1\ra X_2$ is called \textit{a morphism of spaces with measures} if $f$ is a morphism of measurable spaces and for every $A\in \Sigma_2$ we have equality $\mu_2(A)=\mu_1(f^{-1}(A))$.
\end{definition}
\noindent
Spaces with measures and their morphisms form a category.

\section{Outer measures and Carath{\'e}odory's construction}

\begin{definition}
Let $X$ be a set and $\mu^*:\cP(X)\ra [0,+\infty]$ be a function. Suppose that $\mu^*(\emptyset)=0$, $\mu^*(A)\leq \mu^*(B)$ for every subset $A$ of a set $B$ contained in $X$ and 
$$\mu^*\left(\bigcup_{n\in \NN}A_n\right)\leq \sum_{n\in \NN}\mu^*(A_n)$$ 
for every family  $\{A_n\}_{n\in \NN}$ of subsets of $X$. Then we say that $\mu^*$ is \textit{an outer measure on $X$}.
\end{definition}

\begin{theorem}[Carath{\'e}odory's construction]\label{theorem:carath{\'e}odoryconstruction}
Let $X$ be a set and $\mu^*:\cP(X)\ra [0,+\infty]$ be an outer measure on $X$. We define a family of sets $\Sigma_{\mu^*}\subseteq \cP(X)$ by condition
$$A\in \Sigma_{\mu^*}\,\Leftrightarrow\,\forall_{E\subseteq X}\,\mu^*(E)=\mu^*(E\cap A)+\mu^*(E\setminus A)$$
Then the following assertions hold.
\begin{enumerate}[label=\emph{\textbf{(\arabic*)}}, leftmargin=*]
\item $\Sigma_{\mu^*}$ is an $\sigma$-algebra of subsets of $X$.
\item For every family $\{A_n\}_{n\in \NN}$ of pairwise disjoint subsets of $\Sigma_{\mu^*}$ and every subset $E$ of $X$ we have
$$\mu^*\left(E\cap \bigcup_{n\in \NN}A_n\right)=\sum_{n\in \NN}\mu^*(E\cap A_n)$$
In particular, $\mu^*_{\mid \Sigma_{\mu^*}}$ is a measure.
\item Every subset $A$ of $X$ such that $\mu^*(A)=0$ is contained in $\Sigma_{\mu^*}$. In particular, $\mu^*_{\mid \Sigma_{\mu^*}}$ is complete.
\end{enumerate}
\end{theorem}
\noindent
The proof is encapsulated in two lemmas.

\begin{lemma}\label{lemma:outeralgebra}
$\Sigma_{\mu^*}$ is an algebra of sets.
\end{lemma}
\begin{proof}[Proof of the lemma]
Clearly $\emptyset \in \Sigma_{\mu^*}$ and $A\in \Sigma_{\mu^*}\,\Leftrightarrow\,X\setminus A\in \Sigma_{\mu^*}$. It suffices to prove that $\Sigma_{\mu^*}$ is closed under unions. For a subset $B$ of $X$ we denote $X\setminus B$ by $B^c$. Now assume that $A_1$, $A_2\in \Sigma_{\mu^*}$ and pick a subset $E$ of $X$. Then
$$\mu^*(E)=\mu^*(E\cap A_1)+\mu^*(E\cap A_1^c)=\mu^*(E\cap A_1)+\mu^*(E\cap A_1^c\cap A_2)+\mu^*(E\cap A_1^c\cap A_2^c)$$
Since we have equalities
$$E\cap A_1=\left(E\cap (A_1\cup A_2)\right)\cap A_1,\,E\cap A_1^c\cap A_2=\left(E\cap (A_1\cup A_2)\right)\cap A_1^c$$
we derive that $\mu^*(E\cap (A_1\cup A_2))=\mu^*(E\cap A_1)+\mu^*(E\cap A_1^c\cap A_2)$. Similarly we have equality 
$$E\cap A_1^c\cap A_2^c=E\cap (A_1\cup A_2)^c$$
and hence $\mu^*(E\cap A_1^c\cap A_2^c)=\mu^*(E\cap (A_1\cup A_2)^c)$. Therefore, we have 
$$\mu^*(E)=\mu^*(E\cap (A_1\cup A_2))+\mu^*(E\cap (A_1\cup A_2)^c)$$
Thus we proved that $A_1\cup A_2\in \Sigma$. Therefore, $\Sigma_{\mu^*}$ is a family of subsets of $X$ closed under finite unions, complements and containing $\emptyset$. Thus $\Sigma_{\mu^*}$ is an algebra of sets.
\end{proof}

\begin{lemma}\label{lemma:outersigma}
Let $\{A_n\}_{n\in \NN}$ be a family of pairwise disjoint subsets of $\Sigma_{\mu^*}$. Then $\bigcup_{n\in \NN}A_n\in \Sigma_{\mu^*}$ and for every subset $E$ of $X$ there is an equality
$$\mu^*\left(E\cap \bigcup_{n\in \NN}A_n\right)=\sum_{n\in \NN}\mu^*(E\cap A_n)$$
\end{lemma}
\begin{proof}[Proof of the lemma]
We prove that $\bigcup_{n\in \NN}A_n\in \Sigma_{\mu^*}$. For this observe that we have 
$$\mu^*(E) \leq \mu^*\left(E\cap \bigcup_{n\in \NN}A_n\right)+\mu^*\left(E\setminus \bigcup_{n\in \NN}A_n\right) \leq  \sum_{n\in \NN}\mu^*(E\cap A_n)+\mu^*\left(E\setminus \bigcup_{n\in \NN}A_n\right) =$$
$$= \lim_{m\ra +\infty}\left(\mu^*\left(E\cap \bigcup_{n=0}^mA_n\right)+\mu^*\left(E\setminus \bigcup_{n\in \NN}A_n\right)\right)\leq \lim_{m\ra +\infty}\left(\mu^*\left(E\cap \bigcup_{n=0}^mA_n\right)+\mu^*\left(E\setminus \bigcup_{n=0}^{m}A_n\right)\right) = \mu^*(E)$$
and the last equality holds, since $\bigcup_{n=0}^mA_n\in \Sigma_{\mu^*}$ by Lemma \ref{lemma:outeralgebra}. This implies that we have equalities everywhere above. Hence
$$\mu^*\left(E\cap \bigcup_{n\in \NN}A_n\right)= \sum_{n\in \NN}\mu^*(E\cap A_n)$$
and $\bigcup_{n\in \NN}A_n\in \Sigma_{\mu^*}$.
\end{proof}

\begin{proof}[Proof of the theorem]
Lemma \ref{lemma:outeralgebra} and Lemma \ref{lemma:outersigma} imply that $\Sigma_{\mu^*}$ is a $\sigma$-algebra and statement \textbf{(2)} holds. It suffices to verify that statement \textbf{(3)} holds. For this pick a subset $A$ of $X$ such that $\mu^*(A)=0$. Then for every subset $E$ of $X$ we have
$$\mu^*(E)\leq \mu^*(E\cap A)+\mu^*(E\setminus A)=\mu^*(E\setminus A)\leq \mu^*(E)$$
Hence $\mu^*(E)=\mu^*(E\cap A)+\mu^*(E\setminus A)$ and thus $A\in \Sigma_{\mu^*}$.
\end{proof}
\noindent
Next result is a general tool of constructing measures.

\begin{theorem}[Carath{\'e}odory extension]\label{theorem:caratheodoryextensionresult}
Let $X$ be a set and $\Sigma$ be some algebra of its subsets. Suppose that $\mu:\Sigma\ra [0,+\infty]$ is a $\sigma$-additive function. Now for every subset $A$ in $X$ we define
$$\mu^*(A) = \inf \bigg\{\sum_{n\in \NN}\mu(A_n)\,\big|\,A_n\in \Sigma\mbox{ for every }n\in \NN\mbox{ and }A\subseteq \bigcup_{n\in \NN}A_n\bigg\}$$
Then $\mu^*:\cP(X)\ra [0,+\infty]$ is an outer measure, $\Sigma \subseteq \Sigma_{\mu^*}$ and $\mu^*_{\mid \Sigma} = \mu$. Moreover, if $\mu(X)$ is finite, then $\mu^*_{\mid \sigma(\Sigma)}$ is a unique extension of $\Sigma$ to a measure on $\sigma(\Sigma)$.
\end{theorem}
\begin{proof}
Standard verification shows that $\mu^*$ is an outer measure. Note that 
$$\mu^*(A) = \inf \bigg\{\sum_{n\in \NN}\mu(A_n)\,\big|\,\{A_n\}_{n\in \NN}\mbox{ is a family of pairwise disjoint subsets of }\Sigma \mbox{ and }A\subseteq \bigcup_{n\in \NN}A_n\bigg\}$$
for every subset $A$ of $X$. Let $A$ be element of $A$ and let $E$ be an arbitrary subset of $X$. Fix $\epsilon > 0$. By the remark above there exists a family $\{A_n\}_{n\in \NN}$ of pairwise disjoint elements of $\Sigma$ such that
$$E\subseteq \bigcup_{n\in \NN}A_n,\,\sum_{n\in \NN}\mu(A_n)\leq \mu^*(E)+\epsilon$$
By definition of $\mu^*$ we have $\mu^*(E\cap A)\leq \sum_{n\in \NN}\mu(A_n\cap A)$, $\mu^*(E\setminus A) \leq \sum_{n\in \NN}\mu(A_n\setminus A)$ and hence
$$\mu^*(E)\leq \mu^*(E\cap A)+\mu^*(E\setminus A) \leq \sum_{n\in \NN}\mu(A_n\cap A)+\sum_{n\in \NN}\mu(A_n\setminus A)=$$
$$=\sum_{n\in \NN}\left(\mu(A_n\cap A)+\mu(A_n\setminus A)\right)=\sum_{n\in \NN}\mu(A_n)\leq \mu^*(E)+\epsilon$$
Since $\epsilon > 0$ is arbitrary, we derive that $\mu^*(E) = \mu^*(E\cap A)+\mu^*(E\setminus A)$ and hence $A\in \Sigma_{\mu^*}$. Thus $\Sigma \subseteq \Sigma_{\mu^*}$. Once again fix $A\in \Sigma$. Then for every family $\{A_n\}_{n\in \NN}$ of pairwise disjoint elements of $\Sigma$ such that $A\subseteq \bigcup_{n\in \NN}A_n$ we have $\mu(A) = \sum_{n\in \NN}\mu(A_n\cap A)\leq \sum_{n\in \NN}\mu(A_n)$ and thus $\mu(A)\leq \mu^*(A)$. Obviously $\mu^*(A)\leq \mu(A)$. Therefore, for every $A\in \Sigma$ we have $\mu(A) = \mu^*(A)$. Together with $\Sigma\subseteq \Sigma_{\mu^*}$ this implies that $\mu^*_{\mid \sigma(\Sigma)}$ is a measure that extends $\mu$. Now we prove the uniqueness of extension under the assumption that $\mu(X)$ is finite. This follows immediately from Theorem \ref{theorem:uniquenessonpisystem}.
\end{proof}

\section{Outer metric measures}

\begin{definition}
Let $X$ be a topological space. The $\sigma$-algebra $\cB_X$ generated by all open sets of $X$ is called \textit{the $\sigma$-algebra of Borel subsets of $X$}.
\end{definition}

\begin{definition}
Let $(X,d)$ be a metric space and $\mu^*:\cP(X)\ra [0,+\infty]$ be an outer measure. We say that $\mu^*$ is \textit{a metric outer measure} if  
$$\mu^*(E_1\cup E_2)=\mu^*(E_1)+\mu^*(E_2)$$
for any two subsets $E_1$, $E_2$ of $X$ with $\mathrm{dist}(E_1,E_2)=\inf_{x_1\in E_1,\,x_2\in E_2}d(x_1,x_2)>0$.
\end{definition}

\begin{theorem}[Carath{\'e}odory]
Let $(X,d)$ be a metric space and $\mu^*:\cP(X)\ra [0,+\infty]$ be an outer metric measure on $X$. Then the $\sigma$-algebra $\cB_X$ of Borel subsets of $X$ is contained in $\Sigma_{\mu^*}$.
\end{theorem}
\begin{proof}
Let $U$ be an open subset of $X$. Define $F=X\setminus U$ and $U_n=\{x\in X\mid \mathrm{dist}(x,F)>\frac{1}{2^n}\}$ for $n\in \NN$. Then $\{U_n\}_{n\in \NN}$ form an ascending family of open sets and $U=\bigcup_{n\in \NN}U_n$. Fix now a subset $E$ of $X$ such that $\mu^*(E)\in \RR$. We define $E_n=E\cap U_n$ for every $n\in \NN$. Since $\mu^*$ is an outer metric measure, we derive that
$$\mu^*\left(\bigcup_{n=0}^mE_{2n+1}\setminus E_{2n}\right)=\sum_{n=0}^m\mu^*(E_{2n+1}\setminus E_{2n}),\,\mu^*\left(\bigcup_{n=1}^mE_{2n}\setminus E_{2n-1}\right)=\sum_{n=1}^m\mu^*(E_{2n}\setminus E_{2n-1})$$
for every positive integer $m$. Thus we derive 
$$\sum_{n\in \NN}\mu^*(E_{2n+1}\setminus E_{2n})\leq \mu^*(E)\in \RR,\,\sum_{n\in \NN}\mu^*(E_{2n}\setminus E_{2n-1})\leq \mu^*(E)\in \RR$$
Hence we have $\sum_{n\in \NN}\mu^*(E_{n+1}\setminus E_{n})\leq 2\cdot \mu^*(E)\in \RR$. Using the fact that $\mu^*$ is an outer measure, we derive that
$$\mu(E_m)\leq \mu^*(E\cap U)\leq \mu^*(E_m)+\sum_{n\geq m}\mu^*(E_{n+1}\setminus E_n)$$
for every $m\in \NN$. Hence these inequalities yield $\lim_{m\ra +\infty}\mu^*(E_m)=\mu^*(E\cap U)$. Now we have $\mu^*(E_m)+\mu^*(E\setminus U)\leq \mu^*(E)\leq \mu^*(E\cap U)+\mu^*(E\setminus U)$ for every $m\in \NN$. The first inequality holds due to the fact that $\mu^*$ is an outer metric measure. We derive that $\mu^*(E)=\mu^*(E\cap U)+\mu^*(E\setminus U)$. Note that if $\mu^*(E)=+\infty$, then inequality $\mu^*(E)\leq \mu^*(E\cap U)+\mu^*(E\setminus U)$ must be equality. Hence for every subset $E$ of $X$ we have $\mu^*(E)=\mu^*(E\cap U)+\mu^*(E\setminus U)$. This implies that $U\in \Sigma_{\mu^*}$. Since $U$ is an arbitrary open subset of $X$, we deduce that $\cB_X\subseteq \Sigma_{\mu^*}$.
\end{proof}































































\end{document}