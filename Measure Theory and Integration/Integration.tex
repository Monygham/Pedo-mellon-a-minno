\input ../pree.tex

\begin{document}

\title{Integration}
\date{}
\maketitle

\section{Introduction}
In these notes we develop theory of Bochner-Lebesgue integral. Our exhibition is to some extent different from the standard one. The first step is typical - we start with integration of nonnegative functions and we prove monotone convergence theorem. Then we immediately introduce Lebesgue's spaces and prove their completeness. Lebesgue's dominated convergence theorem is presented as a result about convergence in Lebesgue's spaces. After this we introduce integral as a linear operator on Lebesgue's spaces. The last part of the notes is devoted to product measures and theorems on iterated integration (due to Tonelli and Fubini). Prerequisites consists of material contained in first notes on measure theory \cite{Introduction_to_measure_theory}. Most of the theory of Lebesgue's spaces (this does not embrace Bochner's integral itself due to obvious reasons) works for Banach spaces defined over fields with complete absolute values. The reader may always assume for hers convenience that the field is either $\CC$ or $\RR$.

\begin{definition}
Let $(X,\Sigma)$ be a measurable space and let $Y$ be a topological space. A map $f:X\ra Y$ is \textit{measurable} if $f$ is a measurable map $(X,\Sigma)\ra (Y,\cB(Y))$, where $\cB(Y)$ is the $\sigma$-algebra of Borel sets on $Y$. 
\end{definition}

\begin{definition}
Let $X$ be a set and let $Y$ be a topological space. Consider a sequence $\{f_n:X\ra Y\}_{n\in \NN}$ and a map $f:X\ra Y$. If 
$$f(x) = \lim_{n\ra +\infty}f_n(x)$$
then $\{f_n\}_{n\in \NN}$ is \textit{pointwise convergent to $f$}. In this case we write
$$f = \lim_{n\ra +\infty}f_n$$
\end{definition}

\section{Properties of measurable real functions}
\noindent
Let $\ol{\RR}=\{-\infty\} \cup \RR \cup \{+\infty\}$ be completion of $\RR$ to linearly ordered set with the smallest and the greatest elements. Clearly $\ol{\RR}$ is complete linear order. Addition is partially defined operation on $\ol{\RR}$ given by the following rules
$$(+\infty)+r=+\infty=r+(+\infty),\,(-\infty)+r=-\infty=r+(-\infty)$$
for every $r\in \RR$. Moreover, $\ol{\RR}$ with order topology is the two point compactification of $\RR$.

Let $\{f_n:X\ra \ol{\RR}\}_{n\in \NN}$ be a sequence of functions on a set $X$. We define functions
$$\sup_{n\in \NN}f_n,\inf_{n\in \NN}f_n:X\ra \ol{\RR}$$
by formulas
$$\left(\sup_{n\in \NN}f_n\right)(x) = \sup_{n\in \NN}f_n(x),\,\left(\inf_{n\in \NN}f_n\right)(x) = \inf_{n\in \NN}f_n(x)$$
for every $x\in X$. We define functions
$$\limsup_{n\ra +\infty}f_n = \inf_{m\in \NN}\sup_{n\geq m}f_n,\,\liminf_{n\ra +\infty}f_n = \sup_{m\in \NN}\inf_{n\geq m}f_n$$ 
If 
$$\liminf_{n\ra +\infty}f_n = \limsup_{n\ra +\infty}f_n$$
then $\{f_n\}_{n\in \NN}$ is pointwise convergent.

Let $X$ be a set and let $f,g:X\ra \ol{\RR}$ be functions. We write $f \leq g$ if $f(x) \leq g(x)$ for all $x\in X$.

\begin{definition}
Let $X$ be a set and let $f:X\ra \ol{\RR}$ be a function. We say that $f$ is \textit{nonnegative} if $f \geq 0$.
\end{definition}

\begin{definition}
Let $X$ be a set. A sequence $\{f_n:X\ra \ol{\RR}\}_{n\in \NN}$ is \textit{nondecreasing} if $f_n \leq f_m$ for all pairs $n, m\in \NN$ such that $n\leq m$.
\end{definition}

\begin{proposition}\label{proposition:measurable_functions_closed_under_supremum}
Let $\{f_n:X\ra \ol{\RR}\}_{n\in \NN}$ be a sequence of measurable functions on a measurable space $(X,\Sigma)$. Then functions
$$\sup_{n\in \NN}f_n,\,\inf_{n\in \NN}f_n$$
are measurable.
\end{proposition}
\begin{proof}
Note that $\inf_{n\in \NN}f_n = -\sup_{n\in \NN}\left(-f_n\right)$. Thus it suffices to prove the proposition for $\sup_{n\in \NN}f_n$. Fix $r \in \RR$ and note that
$$\big\{x\in X\,\big|\,\sup_{n\in \NN}f_n(x) > r\big\} = \bigcup_{q\in \QQ,\,q > r}\bigcup_{n\in \NN}\big\{x\in X\,\big|\,f_n(x) \geq q\big\}$$
Therefore, we derive that $f = \sup_{n\in \NN}f_n$ satisfies $f^{-1}\left((r,+
\infty]\right)\in \Sigma$ for every $r \in \RR$. Family of all left-open intervals in $\ol{\RR}$ generate $\cB(\ol{\RR})$. Hence $f$ is measurable.
\end{proof}

\begin{corollary}\label{corollary:measurable_closed_under_limes_inferior}
Let $\{f_n:X\ra \ol{\RR}\}_{n\in \NN}$ be a sequence of measurable functions on a measurable space $(X,\Sigma)$. Then functions
$$\liminf_{n \ra +\infty}f_n,\,\limsup_{n\ra +\infty}f_n$$
are measurable. In particular, if $\{f_n(x)\}_{n\in \NN}$ is convergent for every $x\in X$, then also $$\lim_{n\ra +\infty}f_n$$
is measurable.
\end{corollary}
\begin{proof}
Follows directly from Proposition \ref{proposition:measurable_functions_closed_under_supremum} and definitions.
\end{proof}

\begin{proposition}\label{proposition:simple_approximation_for_nonnegative}
Let $f:X\ra \ol{\RR}$ be a nonnegative, measurable function on a measurable space $(X,\Sigma)$. Then there exists a nondecreasing sequence $\{s_n:X\ra \ol{\RR}\}_{n\in \NN}$ of nonnegative, measurable functions such that $s_n(X)$ is a finite subset of $\RR$ for every $n\in \NN$ and $\{s_n\}_{n\in \NN}$ is pointwise convergent to $f$. Moreover, $s_n\leq f$ for every $n\in \NN$.
\end{proposition}
\begin{proof}
For every $n\in \NN$ and integer $0 \leq k < n\cdot 2^n$ we define 
$$A_{n,k}=f^{-1}\left(\left[\frac{k}{2^n},\frac{k+1}{2^n}\right]\right)$$
Then $A_{n,k}$ is a measurable set. We define
$$s_n(x)=\begin{cases}\frac{k}{2^n}& \mbox{ if } x\in A_{n,k}\\
0& \mbox{ if } x\in X\setminus \bigcup_{k=0}^{n\cdot 2^n-1}A_{n,k}
\end{cases}$$
Then each $s_n:X\ra \ol{\RR}$ is a nonnegative, measurable function such that $s_n(X)$ is a finite subset of $\RR$. Moreover, we have
$$|s_n(x) - f(x)| \leq \frac{1}{2^n}$$
for every $x \in X$ such that $f(x) \leq n$. It follows that
$$f = \lim_{n\ra +\infty}s_n$$
By definition of $s_n$ we have $s_n \leq f$ for each $n\in \NN$. This completes the proof.
\end{proof}

\section{Lebesgue's integral of nonnegative functions}\label{section:lebesgues_integration}

\begin{definition}
Let $(X,\Sigma,\mu)$ be a space with measure. A measurable function $s:X\ra \ol{\RR}$ such that $s(X)$ is a finite subset of $\RR$ and 
$$\mu\left(\big\{x\in X\,\big|\,s(x)\neq 0\big\}\right) \in \RR$$
is \textit{a $\mu$-simple function}.
\end{definition}

\begin{definition}
Let $(X,\Sigma,\mu)$ be a space with measure and $s:X\ra \ol{\RR}$ be a nonnegative, $\mu$-simple function. Then
$$\int_Xs\,d\mu = \sum_{y\in \ol{\RR}}y\cdot \mu\left(s^{-1}(y)\right)$$
is \textit{the integral of $s$ with respect to $\mu$}.
\end{definition}

\begin{fact}\label{fact:basics_for_simple_functions}
Let $(X,\Sigma,\mu)$ be a space with measure and $s_1, s_2:X\ra \ol{\RR}$ be nonnegative, $\mu$-simple functions. Then the following assertions hold.
\begin{enumerate}[label=\emph{\textbf{(\arabic*)}}, leftmargin=*]
\item If $a, b\in \RR$ and $a, b\geq 0$, then $a s_1 + b s_2$ is a nonnegative, $\mu$-simple function and 
$$\int_X\left(a s_1 + b s_2\right)\,d\mu = a \int_Xs_1\,d\mu + b \int_Xs_2\,d\mu$$
\item If $s_1\leq s_2$, then
$$\int_Xs_1\,d\mu \leq \int_Xs_2\,d\mu$$
\end{enumerate}
\end{fact}
\begin{proof}
Left for the reader as an exercise.
\end{proof}

\begin{definition}
Let $f:X\ra \ol{\RR}$ be a nonnegative, measurable function on a space $(X,\Sigma,\mu)$ with measure. Then we define 
$$\int_X f d\mu = \sup\bigg\{\int_X s\, d\mu\,\bigg|\,s\mbox{ is a nonnegative, }\mu\mbox{-simple function and }s\leq f\bigg\}$$
We call it \textit{the integral of $f$ with respect to $\mu$}.
\end{definition}

\begin{fact}\label{fact:integral_is_monotone}
Let $f,g:X\ra \ol{\RR}$ be a nonnegative, measurable functions on a space $(X,\Sigma,\mu)$ with measure. If $f\leq g$, then
$$\int_Xf\,d\mu \leq \int_Xg\,d\mu$$
\end{fact}
\begin{proof}
Left for the reader as an exercise.
\end{proof}

\begin{theorem}[Monotone Convergence Theorem]\label{theorem:monotone_convergence}
Let $\{f_n:X\ra \ol{\RR}\}_{n\in \NN}$ be a sequence of nonnegative, measurable functions on a space $(X,\Sigma,\mu)$ with measure. Assume that $\{f_n\}_{n\in \NN}$ is nondecreasing and let $f$ be a nonnegative function which is a limit of $\{f_n\}_{n\in \NN}$. Then $f:X\ra \ol{\RR}$ is a nonnegative, measurable function and
$$\lim_{n\ra +\infty}\int_X f_n\, d\mu = \int_X f\, d\mu$$
\end{theorem}
\begin{proof}
By Corollary \ref{corollary:measurable_closed_under_limes_inferior} function $f$ is measurable. It is also nonnegative. By Fact \ref{fact:integral_is_monotone} we deduce that
$$\int_X f_n\,d\mu \leq \int_X f_{n+1}\,d\mu \leq \int_X f\, d\mu$$
for every $n\in \NN$ and hence 
$$\lim_{n\ra +\infty}\int f_n d\mu \leq \int f d\mu$$
Fix a number $\alpha \in (0,1)$. Pick a $\mu$-simple, nonnegative function $s:X\ra \ol{\RR}$ such that $s\leq f$. Consider the set
$$A_n = \big\{x\in X\,\big|\, f_n(x)< \alpha s(x)\big\}$$
Then $A_n\in \Sigma$ for every $n\in \NN$. Since $\{f_n\}_{n\in \NN}$ is nondecreasing sequence, we derive that $\{A_n\}_{n\in \NN}$ is nonincreasing sequence of sets. Since $s(X)$ is a finite subset of $\RR$ and
$$s(x)\leq f(x) = \lim_{n\ra +\infty}f_n(x)$$
we derive that
$$\bigcap_{n\in \NN}A_n = \emptyset,\,A_1\subseteq \big\{x\in X\,\big|\,s(x)\neq 0\big\}$$
In particular, $\mu(A_1) \in \RR$ and 
$$\lim_{n\ra +\infty}\mu(A_n)=0$$
We have inequality
$$\alpha  \int_Xs\,d\mu = \int_X\alpha s\,d\mu = \int_X  \mathbb{1}_{X\setminus A_n}\cdot \left(\alpha \cdot s\right)\,d\mu +  \int_X \mathbb{1}_{A_n}\cdot \left(\alpha \cdot s\right)\,d\mu \leq $$
$$\leq \int_X f_n\,d\mu + \mu(A_n)\cdot \sup_{x\in X}\left(\alpha s(x)\right) = \int_X f_n\,d\mu + \mu(A_n)\cdot \sup_{x\in X}\left(\alpha s(x)\right)$$
By virtue of
$$\lim_{n\ra +\infty} \mu(A_n) = 0$$
we have
$$\alpha  \int_Xs\,d\mu \leq \lim_{n\ra +\infty}\int_Xf_n\,d\mu$$
Since $s$ is an arbitrary nonnegative and $\mu$-simple function such that $s\leq f$, we deduce that
$$\alpha  \int_Xf\,d\mu \leq \lim_{n\ra +\infty}\int_Xf_n\,d\mu$$
Finally for $\alpha \ra 1$ we obtain
$$\int_Xf\,d\mu \leq  \lim_{n\ra +\infty}\int_Xf_n\,d\mu$$
and this completes the proof.
\end{proof}
\noindent
The theorem above is a reason why Lebesgue's integration theory is such a powerful tool.

\begin{theorem}[Fatou's lemma]\label{theorem:fatous_lemma}
Let $\{f_n:X\ra \ol{\RR}\}_{n\in \NN}$ be a sequence of nonnegative, measurable functions on a space $(X,\Sigma,\mu)$ with measure. Then
$$\int_X\liminf_{n\ra +\infty}f_n\,d\mu \leq \liminf_{n\ra +\infty}\int_Xf_n\,d\mu$$
\end{theorem}
\begin{proof}
For every $m\in \NN$ denote $\inf_{n\geq m}f_n$ by $g_m$. Corollary \ref{corollary:measurable_closed_under_limes_inferior} implies that $\{g_m:X\ra \ol{\RR}\}_{m\in \NN}$ is a nondecreasing sequence of nonnegative, measurable functions on $(X,\Sigma)$. By Theorem \ref{theorem:monotone_convergence} we have
$$\lim_{m\ra +\infty}\int_X\inf_{n\geq m}f_n = \lim_{m\ra +\infty}\int_Xg_m\,d\mu = \int_X\lim_{m\ra +\infty}g_m\,d\mu = \int_X\liminf_{n\ra +\infty}f_n\,d\mu$$
Hence
$$\int_X\liminf_{n\ra +\infty}f_n\,d\mu = \lim_{m\ra +\infty}\int_X\inf_{n\geq m}f_n \leq \liminf_{n\ra +\infty}\int_Xf_n\,d\mu$$
\end{proof}

\begin{proposition}\label{proposition:integral_is_linear}
Let $f,g:X\ra \ol{\RR}$ be a nonnegative, measurable functions on a space $(X,\Sigma,\mu)$ with measure. Fix numbers $a, b\in \{0\}\cup \RR_+$. Then the function $a f+b  g$ is measurable and 
$$\int_X\left(a f+b  g\right)\,d\mu = a \int_Xf\,d\mu + b \int_Xg\,d\mu$$
\end{proposition}
\begin{proof}
By Proposition \ref{proposition:simple_approximation_for_nonnegative} there exist nondecreasing sequences $\{s_n\}_{n\in \NN}$ and $\{t_n\}_{n\in \NN}$ of nonnegative, measurable functions such that
\begin{enumerate}[label=\textbf{(\arabic*)}, leftmargin=*]
\item $s_n(X),t_n(X)$ are finite subsets of $\RR$ for each $n\in \NN$. 
\item $$f(x) = \lim_{n\ra +\infty}s_n(x),\,g(x) = \lim_{n\ra +\infty}t_n(x)$$
\item $s_n \leq f,\,t_n\leq g$ for all $n \in \NN$.
\end{enumerate} 
It follows that
$$\lim_{n \ra +\infty}\left(as_n + bt_n\right) = af + bg$$
Thus $af+bg$ is measurable by Corollary \ref{corollary:measurable_closed_under_limes_inferior}.
By definition
$$a \int_Xf\,d\mu + b \int_Xg\,d\mu \leq \int_X\left(a f+b  g\right)\,d\mu $$
Hence if one of the integrals
$$\int_Xf\,d\mu,\,\int_Xg\,d\mu$$
is infinite, then the assertion holds. Suppose that both integrals are finite. Then $\{s_n\}_{n\in \NN}$ and $\{t_n\}_{n\in \NN}$ consist of nonnegative, $\mu$-simple functions. By Theorem \ref{theorem:monotone_convergence} and Fact \ref{fact:basics_for_simple_functions} we have
$$\int_X\left(a f + b g\right)\,d\mu = \lim_{n\ra +\infty}\int_X\left(a s_n + b t_n\right)\,d\mu = \lim_{n\ra +\infty}\left(a \int_Xs_n\,d\mu + b \int_Xt_n\,d\mu\right) =$$
$$=a  \left(\lim_{n\ra +\infty}\int_Xs_n\,d\mu\right) + b \left(\lim_{n\ra +\infty}\int_Xt_n\,d\mu\right) = a \int_Xf\,d\mu + b \int_Xg\,d\mu$$
\end{proof}

\section{Strongly measurable functions}
\noindent
In this section we introduce a class of measurable functions which form the basis of integration in Banach spaces.

\begin{proposition}\label{proposition:measurable_functions_closed_under_pointwise_limits}
Let $(Y,d)$ be a metric space and let $(X,\Sigma)$ be a measurable space. Suppose that a sequence $\{f_n:X\ra Y\}_{n\in \NN}$ of measurable functions is pointwise convergent to some function $f:X\ra Y$. Then $f$ is measurable.
\end{proposition}
\begin{proof}
Let $U$ be an open subset of $Y$. We define
$$U_k = \big\{y \in Y\,\big|\,\mathrm{dist}\left(y,X\setminus U\right) > 2^{-k}\big\}$$
for every $k\in \NN$. Then $\{U_k\}_{k\in \NN}$ are open subsets of $Y$. We have
$$f^{-1}(U) =\bigcup_{k\in \NN}\bigcup_{m\in \NN}\bigcap_{n\geq m}f_n^{-1}(U_k)$$
and the left hand side is clearly an element of $\Sigma$. Hence preimages of open subsets of $Y$ under $f$ are in $\Sigma$. Since $\sigma$-algebra $\cB(Y)$ is generated by open sets, we derive the assertion.
\end{proof}
\noindent
We fix a field $\mathbb{K}$ together with an absolute value $|-|$. Suppose that $Y$ is a normed vector space over $\mathbb{K}$ and suppose that $\lVert-\rVert$ is its norm. Let $X$ be a set and let $f:X\ra Y$ be a function. We define a nonnegative function $\lVert f \rVert:X\ra \ol{\RR}$ by formula 
$$\lVert f\rVert (x) = \lVert f(x) \rVert$$
for every $x\in X$.

\begin{definition}
Let $Y$ be a normed vector space over $\mathbb{K}$ and let $(X,\Sigma)$ be a measurable space. A function $f:X\ra Y$ is \textit{strongly measurable} if it is measurable and $f(X)$ is a separable subspace of $Y$.
\end{definition}

\begin{proposition}\label{proposition:strongly_measurable_functions_closed_under_pointwise_limits}
Let $Y$ be a normed vector space over $\mathbb{K}$ and let $(X,\Sigma)$ be a measurable space. Suppose that a sequence $\{f_n:X\ra Y\}_{n\in \NN}$ of strongly measurable functions is pointwise convergent to some function $f:X\ra Y$. Then $f$ is strongly measurable.
\end{proposition}
\begin{proof}
According to Proposition \ref{proposition:measurable_functions_closed_under_pointwise_limits} function $f$ is measurable. Moreover, we have
$$f(X) \subseteq \bd{cl}\bigg(\bigcup_{n\in \NN}f_n(X)\bigg)$$
and hence $f(X)$ is a separable subspace of $Y$. Thus $f$ is strongly measurable.
\end{proof}

\begin{proposition}\label{proposition:strongly_measurable_closed_under_diagonal}
Let $n\in \NN$ and let $Y_0, ...,Y_n$ be normed vector spaces over $\mathbb{K}$. Suppose that $(X,\Sigma)$ is a measurable space and $f_i:X\ra Y_i$ for $0 \leq i \leq n$ are strongly measurable functions. Then the function 
$$X\ni x \mapsto \bigg(f_0(x),...,f_n(x)\bigg) \in \prod_{i=0}^nY_i$$
is strongly measurable.
\end{proposition}
\begin{proof}
Note that the family of open subsets of 
$$\prod_{i=0}^nf_i(X)$$
is contained in $\sigma$-algebra generated by sets 
$$\prod_{i=0}^n\left(U_i\cap f_i(X)\right)$$
where $U_i$ is an open subset of $Y_i$ for $0\leq i \leq n$. Indeed, this is a consequence of the fact that $f_i(X)$ are separable for $0\leq i\leq n$. It follows that the function in question is measurable. Since finite product of separable metric spaces is separable, we derive that its image is separable. Hence the function in the statement is strongly measurable.
\end{proof}

\begin{corollary}\label{corollary:strongly_measurable_form_vector_space}
Let $Y$ be a normed space over $\mathbb{K}$ and let $(X,\Sigma)$ be a measurable space. Let $f, g:X\ra Y$ be strongly measurable functions. Then 
$$\alpha f + \beta g$$
is strongly measurable for all $\alpha,\beta \in \mathbb{K}$.
\end{corollary}
\begin{proof}
This is a consequence of Proposition \ref{proposition:strongly_measurable_closed_under_diagonal} and the fact that $Y$ is topological vector space over $\mathbb{K}$. Details are left for the reader.
\end{proof}

\begin{theorem}\label{theorem:simple_approximation_strongly_measurable}
Let $Y$ be a normed space over $\mathbb{K}$ and let $(X,\Sigma)$ be a measurable space. Let $f:X\ra Y$ be a function. Then the following assertions are equivalent.
\begin{enumerate}[label=\emph{\textbf{(\roman*)}}, leftmargin=*]
\item $f$ is strongly measurable.
\item There exists a sequence $\{s_n:X\ra Y\}_{n\in \NN}$ of measurable functions pointwise convergent to $f$ such that $s_n(X)\subseteq Y$ is finite and the inequality
$$||f - s_n||\leq ||f||$$
holds for every $n\in \NN$.
\end{enumerate}
\end{theorem}
\noindent
For the proof we need the following lemma.

\begin{lemma}\label{lemma:argmin_sets_lemma}
Let $n,k\in \NN$ satisfy $k \leq n$. Then
$$\big\{(r_0,...,r_n)\in \RR^{n+1}\,\big|\,\min_{0\leq i\leq n} r_i < r_j\mbox{ for }j < k\mbox{ and }r_k = \min_{0\leq i\leq n} r_i\big\} \subseteq \RR^{n+1}$$
is a Borel subset.
\end{lemma}
\begin{proof}[Proof of the lemma]
Left for the reader.
\end{proof}

\begin{proof}[Proof of the theorem]
Suppose that $f$ is strongly measurable. Consider a countable subset $\{y_k\}_{k\in \NN}$ of $Y$ which closure contains $f(X)$ and assume that $y_0$ is zero in $Y$. From Proposition \ref{proposition:strongly_measurable_closed_under_diagonal} we deduce that the function
$$X\ni x\mapsto \bigg(\lVert y_0 - f(x)\rVert,...,\lVert y_n - f(x)\rVert \bigg)\in \RR^{n+1}$$
is measurable for each $n \in \NN$. Thus by Lemma \ref{lemma:argmin_sets_lemma} the set
$$A_{n,k} = \big\{x \in X\,\big|\,\min_{0\leq i\leq n}\lVert y_i - f(x)\rVert < \lVert y_j - f(x)\rVert \mbox{ for }j < k\mbox{ and }\lVert y_k - f(x)\rVert = \min_{0\leq i\leq n}\lVert y_i - f(x)\rVert\big\}$$
is in $\Sigma$ for all $k,n\in \NN$ such that $k \leq n$. For $n\in \NN$ we define a function $s_n:X\ra Y$ by formula
$$s_n(x) = \sum_{k=0}^ny_k\cdot \mathbb{1}_{A_{n,k}}$$
Note that $s_n$ is measurable, $s_n(X)$ is finite and
$$\lVert s_n(x) - f(x)\rVert = \min_{0\leq i\leq n}\lVert y_i - f(x)\rVert$$
for every $x \in X$. Thus
$$\lim_{n\ra +\infty}s_n = f$$
and $\lVert s_n - f\rVert \leq  \lVert f\rVert$. This completes the proof of $\textbf{(i)}\Rightarrow \textbf{(ii)}$.\\
Suppose now that there exists a sequence $\{s_n:X\ra Y\}_{n\in \NN}$ of measurable functions pointwise convergent to $f$ such that $s_n(X)\subseteq Y$ is finite. Then Proposition \ref{proposition:strongly_measurable_functions_closed_under_pointwise_limits} asserts that $f$ is strongly measurable. This proves that $\textbf{(ii)}\Rightarrow \textbf{(i)}$.
\end{proof}

\section{Lebesgue's spaces}\label{section:lebesgue_spaces}
\noindent
In this section we fix a Banach space $Y$ with norm $\lVert-\rVert$ over a field $\mathbb{K}$ with absolute value $|-|$.

\begin{definition}
Let $f:X\ra Y$ be a strongly measurable function on a space $(X,\Sigma,\mu)$ with measure. If
$$\lVert f\rVert_1 = \int_X\lVert f\rVert\,d\mu \in \RR$$
then $f$ is \textit{integrable with respect to $\mu$} or shortly \textit{$\mu$-integrable}.
\end{definition}

\begin{definition}
Let $(X,\Sigma,\mu)$ be a space with measure. Then the set of all $Y$-valued, $\mu$-integrable functions is denoted by $L^1(\mu,Y)$ and is called \textit{the Lebesgue's space for $Y$ with respect to $\mu$}.
\end{definition}
\noindent
By Corollary \ref{corollary:strongly_measurable_form_vector_space} the set of all strongly measurable, $Y$-valued functions on a space $(X,\Sigma)$ is a $\mathbb{K}$-vector space with respect to the usual operations.

\begin{theorem}[Riesz]\label{theorem:riesz_theorem}
Let $(X,\Sigma,\mu)$ be a space with measure. Then the following assertions hold.
\begin{enumerate}[label=\emph{\textbf{(\arabic*)}}, leftmargin=*]
\item $L^1(\mu,Y)$ is a $\mathbb{K}$-linear subspace of the space of all $Y$-valued, strongly measurable functions.
\item The map
$$L^1(\mu,Y)\ni f \mapsto \lVert f \rVert_1\in [0,+\infty)$$
is a seminorm.
\item If $\{f_n:X\ra Y\}_{n\in \NN}$ is a sequence of elements of $L^1(\mu,Y)$ which is Cauchy with respect to $||-||_1$, then there exist an increasing sequence $\{n_k\}_{k\in \NN}$ of natural numbers and a $\mu$-integrable function $f:X\ra Y$ such that
$$\lim_{k\ra +\infty}f_{n_k}(x) = f(x)$$
for all $x$ outside some set in $\Sigma$ of measure $\mu$ equal to zero. Moreover, $\{f_n\}_{n\in \NN}$ converges to $f$ with respect to $\lVert-\rVert_1$.
\end{enumerate}
\end{theorem}
\begin{proof}
Suppose that $f, g:X\ra Y$ are integrable and $\alpha, \beta\in \mathbb{K}$. Then by Fact \ref{fact:integral_is_monotone}, Proposition \ref{proposition:integral_is_linear} we have
$$0 \leq \lVert \alpha f+ \beta g\rVert_1 = \int_X\lVert \alpha f + \beta g\rVert\,d\mu \leq \int_X\left(|\alpha|\cdot \lVert f\rVert+ |\beta|\cdot \lVert g \rVert\right)\,d\mu= $$
$$= |\alpha| \int_X\lVert f\rVert \,d\mu + |\beta| \int_X\lVert g \rVert\,d\mu = |\alpha| \cdot \lVert f \rVert_1+ |\beta| \cdot \lVert g \rVert_1 $$
Hence $\alpha f+ \beta g$ is $\mu$-integrable and $\lVert -\rVert_1$ is a seminorm on $L^1(\mu,Y)$. This proves both \textbf{(1)} and \textbf{(2)}.\\
Now we focus on \textbf{(3)}. We consider an increasing sequence $\{n_k\}_{k\in \NN}$ of natural numbers such that
$$\lVert f_{n_{k+1}} - f_{n_k}\rVert_1\leq 4^{-k}$$
for every $k\in \NN$. For $k\in \NN$ consider a set
$$A_k = \big\{x \in X\,\big|\,\lVert f_{n_{k+1}}(x) - f_{n_k}(x)\rVert \geq 2^{-k}\big\}$$
in $\Sigma$. Then
$$2^{-k} \cdot \mu(A_k)\leq \int_X\lVert f_{n_{k+1}}-f_{n_k}\rVert \,d\mu = \lVert f_{n_{k+1}} - f_{n_k}\rVert_1 \leq 4^{-k}$$
Hence $\mu(A_k)\leq 2^{-k}$ for each $k\in \NN$ and we have
$$\mu\left(\bigcap_{m\in \NN}\bigcup_{k\geq m}A_k\right) = \lim_{m\ra +\infty} \mu\left(\bigcup_{k\geq m}A_k\right)\leq $$
$$\leq \limsup_{m\ra +\infty}\left(\sum_{k\geq m}\mu(A_k)\right) \leq \limsup_{m\ra +\infty}\left(\sum_{k\geq m}2^{-k}\right) = 0$$
For each $m\in \NN$ let
$$B_m = \bigcup_{k\geq m}A_k$$
Then $\{B_m\}_{m\in \NN}$ is a nonincreasing sequence of subsets of $\Sigma$ and we proved that set
$$B = \bigcap_{m\in \NN}B_m$$
in $\Sigma$ satisfy $\mu(B) = 0$. For $x\not \in B_m$ we have
$$\sum_{k\geq m}\lVert f_{n_{k+1}}(x) - f_{n_k}(x)\rVert \leq \sum_{k=m}^{+\infty}2^{-k} = 2^{1-m}$$
Since $Y$ is a Banach space, we deduce that for $x\not \in B_m$ series
$$f_{n_0}(x) + \sum_{k\in \NN}\left(f_{n_{k+1}}(x) - f_{n_k}(x)\right)$$
is convergent. Therefore, it is also convergent for $x\not \in B$. We define $f:X\ra Y$ as a sum
$$f_{n_0}(x) + \sum_{k\in \NN}\left(f_{n_{k+1}}(x) - f_{n_k}(x)\right)$$
for $x\not \in B$ and $f(x) = 0$ for $x\in B$. Then
$$\lim_{k\ra +\infty}f_{n_k}(x) = f(x)$$
for $x \not \in B$. Hence
$$\lim_{k\ra +\infty}\mathbb{1}_{X\setminus B}\cdot f_{n_k} = f$$
and Proposition \ref{proposition:strongly_measurable_functions_closed_under_pointwise_limits} asserts that $f$ is a strongly measurable function. Moreover, by Theorem \ref{theorem:monotone_convergence} we have
$$\lVert f \rVert_1 = \int_X\lVert f\rVert \,d\mu = \int_X\mathbb{1}_{X\setminus B}\cdot \big \lVert f_{n_0} + \sum_{k\in \NN}(f_{n_{k+1}} - f_{n_k})\big \rVert \,d\mu \leq \int_X\left(\lVert f_{n_0}\rVert  + \sum_{k\in \NN}\lVert f_{n_{k+1}}-f_{n_k}\rVert \right)\,d\mu = $$
$$= \int_X\lVert f_{n_0}\rVert \,d\mu + \sum_{k\in \NN}\int_X \lVert f_{n_{k+1}}-f_{n_k}\rVert \,d\mu \leq \lVert f_{n_0}\rVert_1 + \sum_{k\in \NN}4^{-k} $$
Thus $f\in L^1(\mu,Y)$ and by Theorem \ref{theorem:monotone_convergence} we have
$$\lVert f - f_{n_m}\rVert_1 = \int_X\lVert f - f_{n_m}\rVert \,d\mu = \int_X\mathbb{1}_{X\setminus B}\cdot \big\lVert \sum_{k\geq m}\left(f_{n_{k+1}} - f_{n_k}\right)\big\rVert \,d\mu \leq  \int_X \big\lVert \sum_{k\geq m}\left(f_{n_{k+1}} - f_{n_k}\right)\big\rVert \,d\mu \leq $$
$$\leq \int_X\sum_{k\geq m}\lVert f_{n_{k+1}} - f_{n_k}\rVert \,d\mu =\sum_{k\geq m}\int_X\lVert f_{n_{k+1}} - f_{n_k}\rVert \,d\mu = \sum_{k\geq m}4^{-k} = 4^{-m}\cdot \frac{4}{3} $$
Therefore, $\{f_{n_k}\}_{k\in \NN}$ converges to $f$ with respect to $\lVert -\rVert_1$. Since $\{f_n\}_{n\in \NN}$ is a Cauchy sequence with respect to $\lVert-\rVert_1$ with a subsequence convergent to $f$, we derive that $\{f_n\}_{n\in \NN}$ converges to $f$ with respect to $\lVert -\rVert_1$.
\end{proof}
\noindent
Next theorem is a criterion connecting pointwise convergence and convergence with respect to $\lVert -\rVert_1$.

\begin{theorem}[Lebesgue's dominated convergence theorem]\label{theorem:dominated_convergence}
Let $(X,\Sigma,\mu)$ be a space with measure and let $\{f_n:X\ra Y\}_{n\in \NN}$ be a sequence of $\mu$-integrable functions. Suppose that $f:X\ra Y$ is a pointwise limit of $\{f_n\}_{n\in \NN}$ and assume that there exists nonnegative, measurable function $g:X\ra \ol{\RR}$ such that $\lVert f_n\rVert \leq g$ holds for every $n\in \NN$ and
$$\int_X g\,d\mu\in \RR$$
Then $f\in L^1(\mu,Y)$ and $\{f_n\}_{n\in \NN}$ converges to $f$ with respect to $\lVert -\rVert_1$.
\end{theorem}
For the proof we need the following result.

\begin{lemma}\label{lemma:difference_of_integrals}
Let $f, g:X\ra \ol{\RR}$ be a nonnegative, measurable functions on a space $(X,\Sigma,\mu)$ with measure. Suppose that $f \leq g$ and
$$\int_Xf\,d\mu, \int_Xg\,d\mu\in \RR$$
Then
$$\int_X(g-f)\,d\mu = \int_Xg\,d\mu - \int_Xf\,d\mu$$
\end{lemma}
\begin{proof}[Proof of the lemma]
According to Proposition \ref{proposition:integral_is_linear} we obtain that
$$\int_Xg\,d\mu = \int_X\big((g-f) + f\big)\,d\mu = \int_X(g-f)\,d\mu + \int_Xf\,d\mu$$
Since integrals above are finite, we have
$$\int_X(g-f)\,d\mu = \int_Xg\,d\mu - \int_Xf\,d\mu$$
\end{proof}

\begin{proof}[Proof of the theorem]
Since $\{f_n\}_{n\in \NN}$ converges pointwise to $f$, we deduce that $f$ is strongly measurable. Moreover, a sequence $\big\{\lVert f\rVert_n:X\ra \ol{\RR}\big\}_{n\in \NN}$ converges pointwise to $\lVert f\rVert$. Since $\lVert f_n\rVert\leq g$ holds for every $n\in \NN$, we deduce that $\lVert f\rVert\leq g$. Thus
$$\lVert f\rVert_1 = \int_X\lVert f\rVert\,d\mu \leq \int_Xg\,d\mu \in \RR$$
Hence $f\in L^1(\mu, Y)$. Note that $\lVert f - f_n\rVert \leq 2g$ holds for every $n\in \NN$. Thus by Theorem \ref{theorem:fatous_lemma} and Lemma \ref{lemma:difference_of_integrals} we have
$$\int_X2g\,d\mu - \int_X\limsup_{n\ra +\infty}\lVert f-f_n\rVert\,d\mu  = \int_X \big(2g - \limsup_{n\ra +\infty}\lVert f-f_n\rVert\big)\,d\mu = $$
$$= \int_X\liminf_{n\ra +\infty}\left(2g - \lVert f-f_n||\right)\,d\mu \leq \liminf_{n\ra +\infty}\int_X\left(2g - \lVert f- f_n\rVert\right)\,d\mu =$$
$$= \int_X 2g\,d\mu - \limsup_{n\ra +\infty}\int_X \lVert f-f_n\rVert\,d\mu $$
Hence
$$\limsup_{n\ra +\infty}\int_X\lVert f - f_n\rVert\,d\mu \leq \int_X\limsup_{n\ra +\infty}\lVert f - f_n||\,d\mu = 0$$
Thus we deduce that
$$\lim_{n\ra +\infty}\lVert f - f_n\rVert_1 = \lim_{n\ra +\infty}\int_X\lVert f - f_n\rVert\,d\mu = 0$$
\end{proof}
\noindent
It turns out that Lebesgue's space $L^1(\mu, Y)$ contains certain dense subspace which can be easily described. We shall define this space and then prove that in fact it is dense.

\begin{definition}
Let $(X,\Sigma, \mu)$ be a space with measure. A measurable function $s:X\ra Y$ such that $s(X)\subseteq Y$ is finite and
$$\mu\left(\{x\in X\,\big|\,s(x)\neq 0\big\}\right) \in \RR$$
is \textit{$\mu$-simple}. The set of all $\mu$-simple, $Y$-valued functions defined on $(X,\Sigma, \mu)$ is denoted by $S(\mu, Y)$.
\end{definition}

\begin{theorem}\label{theorem:simple_are_dense}
Let $(X,\Sigma, \mu)$ be a space with measure. For each $f \in L^1(\mu,Y)$ there exists a sequence $\{s_n:X\ra Y\}_{n\in \NN}$ of $\mu$-simple functions and a nonnegative, measurable function $g:X\ra \ol{\RR}$ such that the following assertions hold.
\begin{enumerate}[label=\emph{\textbf{(\arabic*)}}, leftmargin=*]
\item
$$\int_Xg\,d\mu \in \RR$$
\item $\lVert s_n \rVert \leq g$ for every $n\in \NN$
\item $\{s_n\}_{n\in \NN}$ converges pointwise to $f$.
\end{enumerate}
In particular, the space $S(\mu, Y)$ is a dense $\mathbb{K}$-linear subspace of $L^1(\mu, Y)$.
\end{theorem}
\begin{proof}
Clearly every $\mu$-simple function is strongly measurable and $\mu$-integrable. Moreover, $\mu$-simple functions are closed under $\mathbb{K}$-vector space operations defined on the space of strongly measurable functions. Hence $S(\mu, Y) \subseteq L^1(\mu, Y)$ is a $\mathbb{K}$-linear subspace.\\
Suppose now that $f:X\ra Y$ is a $\mu$-integrable function. By Theorem \ref{theorem:simple_approximation_strongly_measurable} there exists a sequence $\{s_n:X\ra Y\}_{n\in \NN}$ of measurable functions pointwise convergent to $f$ such that $s_n(X)$ is finite and the inequality
$$\lVert f - s_n\rVert\leq \lVert f\rVert$$
holds for every $n\in \NN$. Let $g = 2\cdot \lVert f\rVert$. Then 
$$\int_Xg\,d\mu \in \RR$$
Moreover, for every $n\in \NN$ we have $\lVert s_n \rVert\leq g$. Hence $s_n$ is $\mu$-simple for every $n\in \NN$. This completes the proof of the first part of the  theorem.\\
Now by virtue of Theorem \ref{theorem:dominated_convergence} we deduce that $S(\mu, Y)\subseteq L^1(\mu, Y)$ is dense.
\end{proof}

\section{Bochner's integral}\label{section:bochner_integration}
\noindent
In this section $\mathbb{K}$ is either field $\RR$ or $\CC$ with their usual absolute values.

\begin{definition}
Let $Y$ be a Banach space over $\mathbb{K}$ and let $(X,\Sigma,\mu)$ be a space with measure. For every $s \in S(\mu, Y)$ we define
$$\int_X s\,d\mu = \sum_{y\in Y}y\cdot \mu\left(s^{-1}(y)\right)$$
and we call it \textit{the integral of $s$ with respect to $\mu$}.
\end{definition}

\begin{fact}\label{fact:integral_of_simple_properties}
Let $Y$ be a Banach space over $\mathbb{K}$ and let $(X,\Sigma,\mu)$ be a space with measure. Then
$$S(\mu, Y)\ni s \mapsto \int_Xs\,d\mu \in Y$$
is a $\mathbb{K}$-linear operator such that
$$\bigg\lVert\int_Xs\,d\mu\bigg\rVert \leq \lVert s \rVert_1$$
\end{fact}
\begin{proof}
We left the proof (direct calculation) for the reader as an exercise.
\end{proof}
\noindent
Let $Y$ be a Banach space over $\mathbb{K}$ and let $(X,\Sigma,\mu)$ be a space with measure. By Theorem \ref{theorem:simple_are_dense} space $S(\mu, Y)$ is a dense $\mathbb{K}$-linear subspace of $L^1(\mu, Y)$. By Theorem \ref{theorem:riesz_theorem} space $L^1(\mu, Y)$ is complete and by Fact \ref{fact:integral_of_simple_properties} operator
$$S(\mu, Y)\ni s \mapsto \int_Xs\,d\mu \in Y$$
is a $\mathbb{K}$-linear operator with norm equal to one. These imply that there exists a unique $\mathbb{K}$-linear operator
$$L^1(\mu, Y)\ni f\mapsto \int_Xf\,d\mu\in Y$$
with norm equal to one extending the integral on $S(\mu, Y)$.

\begin{definition}
Let $Y$ be a Banach space over $\mathbb{K}$ and let $(X,\Sigma,\mu)$ be a space with measure. The operator
$$L^1(\mu, Y)\ni f\mapsto \int_Xf\,d\mu\in Y$$
is called \textit{the Bochner's integral with respect to $\mu$}. For every $f\in L^1(\mu, Y)$ element
$$\int_Xf\,d\mu\in Y$$
is called \textit{the integral of $f$ with respect to $\mu$}.
\end{definition}

\begin{corollary}\label{corollary:convergence_of_integral}
Let $Y$ be a Banach space over $\mathbb{K}$ and let $(X,\Sigma,\mu)$ be a space with measure. Suppose that $\{f_n:X\ra Y\}_{n\in \NN}$ is a sequence of $\mu$-integrable functions convergent in $L^1(\mu, Y)$ to some $\mu$-integrable function $f:X\ra Y$. Then
$$\lim_{n\ra +\infty}\int_Xf_n\,d\mu = \int_Xf\,d\mu$$
in $Y$.
\end{corollary}
\begin{proof}
By definition Bochner's integral is continuous with respect to $\lVert-\rVert_1$.
\end{proof}

\section{Lebesgue integral of scalar functions and induction}
\noindent
First we compare Bochner's integration with Lebesgue's integration of nonnegative functions. We introduce precise terminology.

\begin{definition}
Let $X$ be a set and let $f:X\ra \CC$ be a function. If $f(x)\in \RR$ for every $x\in X$, then we say that $f$ is \textit{real valued}. If in addition $f(x)\geq 0$ for every $x\in X$, then $f$ is \textit{nonnegative}.
\end{definition}
\noindent
As careful reader may notice there is certain ambiguity in theory developed so far. Indeed, if $(X,\Sigma,\mu)$ is a space with measure and $f:X\ra \CC$ is a $\mu$-integrable, nonnegative function, then we have a twofold interpretation of
$$\int_Xf\,d\mu$$
Firstly, if we consider $f$ as a nonnegative, $\mu$-measurable function with values in $\ol{\RR}$, then we may consider integral of this nonnegative function described as in Section \ref{section:lebesgues_integration}. On the other hand it may be considered as the Bochner integral of $f$ with respect to $\mu$ as defined in Section \ref{section:bochner_integration}. We explain now why these two numbers are equal. For this note that there is no ambiguity in definitions of simple functions and their integrals between Section \ref{section:lebesgues_integration} on the one hand and Sections \ref{section:lebesgue_spaces}, \ref{section:bochner_integration} on the other. By Proposition \ref{proposition:simple_approximation_for_nonnegative} there exists a nondecreasing sequence of nonnegative, $\mu$-simple functions $\{s_n:X\ra \CC\}_{n\in \NN}$ which is pointwise convergent to $f$. By Theorem \ref{theorem:monotone_convergence} we have
$$\int_Xf\,d\mu = \lim_{n\ra +\infty}\int_Xs_n\,d\mu$$
where we understand the left hand side as the integral in the sense of Section \ref{section:lebesgues_integration}. On the other hand by Theorem \ref{theorem:dominated_convergence} the sequence $\{s_n\}_{n\in \NN}$ converges to $f$ also in $L^1(\mu,\CC)$. Hence by Corollary \ref{corollary:convergence_of_integral} we deduce that
$$\int_Xf\,d\mu = \lim_{n\ra +\infty}\int_Xs_n\,d\mu$$
where we understand the left hand side as the Bochner integral of $f$ with respect to $\mu$. Thus the two numbers are equal.\\
Let $(X,\Sigma, \mu)$ be a space with measure. In case of $\CC$ or $\RR$ valued $\mu$-integrable function $f$ on $X$ its Bochner integral
$$\int_X f\,d\mu$$
is also called Lebesgue integral.\\
The following sequence of results is an useful tool for studying classes of functions in integration theory.

\begin{corollary}\label{corollary:measurable_induction_for_nonnegative}
Let $(X,\Sigma)$ be a measurable space and let $\cF$ be a family of functions defined on $X$ and with values in $\ol{\RR}$. Suppose that the following assertions hold.
\begin{enumerate}[label=\emph{\textbf{(\arabic*)}}, leftmargin=*]
\item $\mathbb{1}_A\in \cF$ for every $A\in \Sigma$.
\item $\cF$ is closed under $\RR$-linear combinations of nonnegative functions with nonnegative coefficients.
\item $\cF$ is closed under pointwise limits of nondecreasing sequences of nonnegative functions.
\end{enumerate}
Then $\cF$ contains all nonnegative, measurable functions on $X$ with values in $\ol{\RR}$.
\end{corollary}
\begin{proof}
This follows from Proposition \ref{proposition:simple_approximation_for_nonnegative}.
\end{proof}

\begin{corollary}\label{corollary:measurable_induction_for_complex}
Let $(X,\Sigma,\mu)$ be a space with measure and let $\cF$ be a family of complex valued, $\mu$-integrable functions defined on $X$. Suppose that the following assertions hold.
\begin{enumerate}[label=\emph{\textbf{(\arabic*)}}, leftmargin=*]
\item $\mathbb{1}_A\in \cF$ for every $A\in \Sigma$ with $\mu(A)\in \RR$.
\item If $f, g\in \cF$ and $\alpha, \beta\in \CC$, then
$$\alpha f + \beta g\in \cF$$
\item If $\{f_n:X\ra \CC\}_{n\in \NN}$ is a nondecreasing sequence of nonnegative functions in $\cF$ which converges to $\mu$-integrable function $f$, then $f\in \cF$.
\end{enumerate}
Then $\cF$ is $L^1(\mu, \CC)$.
\end{corollary}
\begin{proof}
By \textbf{(1)} and \textbf{(2)} family $\cF$ contains all $\mu$-simple functions. In particular, it contains all nonnegative, $\mu$-simple functions. According to \textbf{(3)} and Proposition \ref{proposition:simple_approximation_for_nonnegative} this implies that $\cF$ contains all nonnegative, $\mu$-integrable functions. Suppose now that $f:X\ra \CC$ is real valued and $\mu$-integrable. Then $f_+ = \sup \{f, 0\}$ and $f_- = \sup\{-f, 0\}$ are $\mu$-integrable and nonnegative. Hence they are elements of $\cF$. By \textbf{(2)} we deduce that $f = f_+ - f_-$ is in $\cF$. Finally, if $f:X\ra \CC$ is an arbitrary function in $L^1(\mu, \CC)$, then we write $f = f_r + i\cdot f_i$, where $f_r, f_i$ are real valued and $\mu$-integrable. Then by previous considerations $f_r,f_i\in \cF$ and hence $f\in \cF$ as their $\CC$-linear combination.
\end{proof}

\begin{corollary}\label{corollary:measurable_induction_for_banach_valued}
Let $(X,\Sigma,\mu)$ be a space with measure and let $Y$ be a Banach space over a field $\mathbb{K}$ with absolute value. Suppose that $\cF$ is a family of $\mu$-integrable, $Y$-valued functions defined on $X$. Suppose that the following assertions hold.
\begin{enumerate}[label=\emph{\textbf{(\arabic*)}}, leftmargin=*]
\item $y\cdot \mathbb{1}_A\in \cF$ for every $y\in Y$ and $A\in \Sigma$ with $\mu(A)\in \RR$.
\item If $f, g\in \cF$ and $\alpha, \beta\in \mathbb{K}$, then
$$\alpha f + \beta g\in \cF$$
\item Suppose that $\{f_n:X\ra Y\}_{n\in \NN}$ is a sequence of functions in $\cF$ and $g:X\ra \ol{\RR}$ is a nonnegative, $\mu$-integrable function such that $\lVert f_n\rVert \leq g$ for every $n\in \NN$. Let $f$ be a pointwise limit of $\{f_n\}_{n\in \NN}$. Then $f\in \cF$.
\end{enumerate}
Then $\cF$ is $L^1(\mu,Y)$.
\end{corollary}
\begin{proof}
By \textbf{(1)} and \textbf{(2)} family $\cF$ contains all $\mu$-simple functions. According to Theorem \ref{theorem:simple_are_dense} and \textbf{(3)} we derive that $\cF$ contains every element of $L^1(\mu,Y)$.
\end{proof}

\section{Product measures}
\noindent
In this section we discuss integration on the product of spaces with measures.

\begin{fact}\label{fact:productalgebra}
Let $(X_1,\Sigma_1)$ and $(X_2,\Sigma_2)$ be measurable spaces. Then a family consisting of disjoint sums of sets of the form $A_1\times A_2$ where $A_1\in \Sigma_1, A_2\in \Sigma_2$ is an algebra of subsets of $X_1\times X_2$.
\end{fact}
\begin{proof}
Left to the reader as an exercise.
\end{proof}

\begin{definition}
Let $(X_1,\Sigma_1)$ and $(X_2,\Sigma_2)$ be measurable spaces. Let $\Sigma_1\times \Sigma_2$ be the algebra of subsets of $X_1\times X_2$ consisting of disjoint subsets of the form $A_1\times A_2$ where $A_1 \in \Sigma_1,A_2\in \Sigma_2$. Then $\Sigma_1\times \Sigma_2$ is \textit{the product algebra of $\Sigma_1$ and $\Sigma_2$}. A $\sigma$-algebra $\Sigma_1\otimes \Sigma_2$ generated by $\Sigma_1\times \Sigma_2$ is \textit{the product $\sigma$-algebra of $\Sigma_1$ and $\Sigma_2$}.
\end{definition}
\noindent
Suppose that $Y$ is a set and $f:X_1\times X_2\ra Y$ is a function. For every $x_1\in X_1$ we define a function $f_{x_1}:X_2\ra Y$ by formula 
$$f_{x_1}(x) = f(x_1,x)$$
for every $x$ in $X_2$. Similarly for every $x_2\in X_2$ we define a function $f_{x_2}:X_1\ra Y$ by formula
$$f_{x_2}(x) = f(x,x_2)$$
for every $x$ in $X_1$. There is also a version of this notation for sets. Let $E\subseteq X_1\times X_2$ be a subset. Then we define
$$E_{x_1} = \{x\in X_2\,|\,(x_1,x)\in E\},\,E_{x_2} = \{x\in X_1\,|\,(x,x_2)\in E\}$$
for each $x_1\in X_1$ and $x_2\in X_2$. Note that
$$\mathbb{1}_{E_{x_1}} = \left(\mathbb{1}_E\right)_{x_1},\,\mathbb{1}_{E_{x_2}} = \left(\mathbb{1}_E\right)_{x_2}$$

\begin{proposition}\label{proposition:measurable_functions_have_measurable_sections}
Let $(X_1,\Sigma_1), (X_2,\Sigma_2)$ be measurable spaces and $Y$ be a Banach space over a field $\mathbb{K}$ with absolute value. Then the following assertions hold.
\begin{enumerate}[label=\emph{\textbf{(\arabic*)}}, leftmargin=*]
\item For every function $f:X_1\times X_2\ra \ol{\RR}$ measurable with respect to $\Sigma_1\otimes \Sigma_2$ and any $x_1\in X_1,x_2\in X_2$ function $f_{x_1}$ is measurable with respect to $\Sigma_2$ and function $f_{x_2}$ is measurable with respect to $\Sigma_1$.
\item For every function $f:X_1\times X_2\ra Y$ strongly measurable with respect to $\Sigma_1\otimes \Sigma_2$ and any $x_1\in X_1,x_2\in X_2$ function $f_{x_1}$ is strongly measurable with respect to $\Sigma_2$ and $f_{x_2}$ is strongly measurable with respect to $\Sigma_1$.
\end{enumerate}
\end{proposition}
\begin{proof}
First let $\cS$ be a family of all subsets $E$ in $\Sigma_1\otimes \Sigma_2$ such that $E_{x_1}\in \Sigma_2$ and $E_{x_2}\in \Sigma_1$ for every $x_1\in X_1$ and $x_2\in X_2$. Then $\Sigma_1\times \Sigma_2\subseteq \cS$ and $\cS$ is a monotone family. Thus by Sierpiński's theorem on monotone classes we have $\Sigma_1\otimes \Sigma_2\subseteq \cS$.\\
Now we prove the first assertion. Let $\cF$ be a family of all functions $f:X_1\times X_2\ra \ol{\RR}$ such that $f_{x_1}$ is measurable with respect to $\Sigma_2$ and $f_{x_2}$ is measurable with respect to $\Sigma_1$ for every $x_1\in X_1,x_2\in X_2$. Since $\Sigma_1\otimes \Sigma_2\subseteq \cS$, we derive that $\cF$ contains $\mathbb{1}_E$ for every $E\in \Sigma_1\otimes \Sigma_2$. Thus the intersection of $\cF$ with nonnegative, $\ol{\RR}$-valued functions on $X_1\times X_2$ satisfy all conditions of Corollary \ref{corollary:measurable_induction_for_nonnegative} and hence $\cF$ contains all nonnegative, $\Sigma_1\times \Sigma_2$-measurable functions with values in $\ol{\RR}$. Now suppose that $f:X_1\times X_2\ra \ol{\RR}$ is $\Sigma_1\otimes \Sigma_2$-measurable. Write $f_+ = \sup\{f, 0\}$ and $f_- = \sup\{-f, 0\}$. Then $f = f_+ - f_-$ and both functions $f_+,f_-:X_1\times X_2\ra \ol{\RR}$ are measurable with respect to $\Sigma_1\otimes \Sigma_2$ and nonnegative. Thus $f_+,f_-\in \cF$. Hence also $f\in \cF$. This proves \textbf{(1)}.\\
Now we prove \textbf{(2)}. Let $\cF$ be a family of all functions $f:X_1\times X_2\ra Y$ such that $f_{x_1}$ is measurable with respect to $\Sigma_2$ and $f_{x_2}$ is measurable with respect to $\Sigma_1$ for every $x_1\in X_1,x_2\in X_2$. As above we can derive that for every $y\in Y$ and for every $E\in \Sigma_1\otimes \Sigma_2$ we have $y\cdot \mathbb{1}_E\in \cF$. Moreover, $\cF$ is a $\mathbb{K}$-vector space with respect to pointwise operations. Hence $\cF$ contains every $\Sigma_1\times \Sigma_2$-measurable function $s:X_1\times X_2\ra Y$ such that $s(X_1\times X_2)$ is finite. Next by Theorem \ref{theorem:simple_approximation_strongly_measurable} for every strongly $\Sigma_1\times \Sigma_2$-measurable function $f:X_1\times X_2\ra Y$ there exists a sequence $\{s_n:X_1\times X_2\ra Y\}_{n\in \NN}$ of strongly $\Sigma_1\times \Sigma_2$-measurable functions such that $s_n(X_1\times X_2)$ is finite for every $n\in \NN$ and
$$f = \lim_{n\ra +\infty}s_n$$
Since $\cF$ is closed under pointwise limits, we derive that $f$ is in $\cF$.
\end{proof}

\begin{theorem}\label{theorem:fubinis_theorem_basic}
Let $(X,\Sigma_1,\mu_1)$ and $(X_2,\Sigma_2,\mu_2)$ be spaces with $\sigma$-finite measures. Then the following assertions hold.
\begin{enumerate}[label=\emph{\textbf{(\arabic*)}}, leftmargin=*]
\item For every $E\in \Sigma_1\otimes \Sigma_2$ function
$$X_1\ni x_1\mapsto \mu_2(E_{x_1})\in \ol{\RR}$$
is measurable with respect to $\Sigma_1$.
\item For every $E\in \Sigma_1\otimes \Sigma_2$ function
$$X_2\ni x_2\mapsto \mu_1(E_{x_2})\in \ol{\RR}$$
is measurable with respect to $\Sigma_2$.
\item There exists a unique measure $\mu_1\otimes \mu_2$ defined on $\Sigma_1\otimes \Sigma_2$ such that
$$\left(\mu_1\otimes \mu_2\right)\left(A_1\times A_2\right) = \mu_1(A_1)\mu_2(A_2)$$
for $A_1\in \Sigma_1, A_2\in \Sigma_2$.
\item Measure $\mu_1\otimes \mu_2$ is $\sigma$-finite.
\item For every $E\in \Sigma_1\otimes \Sigma_2$ we have
$$\int_{X_1}\mu_2(E_{x_1})\,d\mu_1 = \left(\mu_1\otimes \mu_2\right)(E) = \int_{X_2}\mu_1(E_{x_2})\,d\mu_2$$
\end{enumerate}
\end{theorem}
\begin{proof}
We prove \textbf{(1)}. For every $E$ in $\Sigma_1\otimes \Sigma_2$ we denote by $f_E$ the function
$$X_1\ni x_1 \mapsto \mu_2\left(E_{x_1}\right)\in \ol{\RR}$$
This function is well defined according to Proposition \ref{proposition:measurable_functions_have_measurable_sections}. Let $\cF$ be a family of all subsets $E\in \Sigma_1\otimes \Sigma_2$ such that $f_E$ is measurable with respect to $\Sigma_1$. First note that if $E=A_1 \times A_2$ for $A_1\in \Sigma_1$ and $A_2\in \Sigma_2$, then  $f_E = \mu_2(A_2)\cdot \mathbb{1}_{A_1}$. Now suppose that
$$E= \bigcup_{n=1}^mA_{1,n}\times A_{2,n}$$
where $A_{1,n}\in \Sigma_1,A_{2,n}\in \Sigma_2$ for every $1\leq n\leq m$. Then
$$f_E = \sum_{n=1}^m\mu_2(A_{2,n})\mathbb{1}_{A_{1,n}}$$
and hence $\Sigma_1\times \Sigma_2\subseteq \cF$. Moreover, $\cF$ is a monotone family of sets. Sierpiński's theorem on monotone classes shows that $\Sigma_1\otimes \Sigma_2\subseteq \cF$. This proves \textbf{(1)} and by symmetry also \textbf{(2)}.\\ Now by \textbf{(1)} it makes sense to define
$$(\mu_1\otimes \mu_2)(E) = \int_{X_1}\mu_2(E_{x_1})\,d\mu_1$$
for every $E\in \Sigma_1\otimes \Sigma_2$. Clearly $(\mu_1\otimes \mu_2)(\emptyset) = 0$ and if $\{E_n\}_{n\in \NN}$ is a family of disjoint subsets in $\Sigma_1\otimes \Sigma_2$, then by Theorem \ref{theorem:monotone_convergence} we have
$$(\mu_1\otimes \mu_2)\left(\bigcup_{n\in \NN}E_n\right) = \sum_{n\in \NN}(\mu_1\otimes \mu_2)(E_n)$$
Hence $\mu_1\otimes \mu_2$ is a measure on $\Sigma_1\otimes \Sigma_2$. We also have
$$(\mu_1\otimes \mu_2)\left(A_1\times A_2\right) = \int_{X_1}\mu_2(A_2)\mathbb{1}_{A_1}\,d\mu_1 = \mu_1(A_1)\mu_2(A_2)$$
for every $A_1\in \Sigma_1, A_2\in \Sigma_2$. This gives the first part of \textbf{(3)}. Suppose now that $\mu,\nu$ are measures on $\Sigma_1\otimes \Sigma_2$ such that
$$\mu(A_1\times A_2) = \mu_1(A_1)\mu_2(A_2) = \nu(A_1\times A_2)$$
for every $A_1\in \Sigma_1,A_2\in \Sigma_2$. Let 
$$X_1 = \bigcup_{n\in \NN}X_{1,n},\,X_2 = \bigcup_{n\in \NN}X_{2,n}$$
be partitions such that $X_{1,n}\in \Sigma_1,X_{2,n}\in \Sigma_2$ and $\mu_1(X_{1,n})\in \RR,\mu_2(X_{2,n})\in \RR$ for every $n\in \NN$. Fix now $n,m\in \NN$ and for every $E\in \Sigma_1\otimes \Sigma_2$ define
$$\mu_{n,m}(E) = \mu\left(E\cap (X_{1,n}\times  X_{2,m})\right),\,\nu_{n,m}(E) = \nu_{n,m}\left(E\cap (X_{1,n}\times X_{2,m})\right)$$
Note that $\mu_{n,m},\,\nu_{n,m}$ are finite measures on $\Sigma_1\otimes \Sigma_2$. Family $\big\{A_1\times A_2\big\}_{A_1\in \Sigma_1,A_2\in \Sigma_2}$ is a $\pi$-system that generates $\sigma$-algebra $\Sigma_1\otimes \Sigma_2$. Dynkin's theorem on $\pi-\lambda$-systems shows that $\mu_{n,m} = \nu_{n,m}$. This implies that
$$\mu(E) = \sum_{n\in \NN}\sum_{m\in \NN}\mu_{n,m}(E) = \sum_{n\in \NN}\sum_{m\in \NN}\nu_{n,m}(E) = \nu(E)$$
Thus $\mu_1\otimes \mu_2$ is unique and \textbf{(3)} is proved. Moreover, it is easy to observe that \textbf{(4)} holds i.e. measure $\mu_1\otimes \mu_2$ is $\sigma$-finite. Indeed, we have
$$X_1\times X_2 = \bigcup_{n\in \NN}\bigcup_{m\in \NN}X_{1,n}\times X_{2,m}$$
and
$$\left(\mu_1\otimes \mu_2\right)\left(X_{1,n}\times X_{2,m}\right) = \mu_1(X_{1,n})\mu_2(X_{2,m}) \in \RR$$
Finally by symmetry we derive that
$$\Sigma_1\otimes \Sigma_2\ni E \mapsto \int_{X_2}\mu_1(E_{x_2})\,d\mu_2\in [0,+\infty]$$
is a measure on $\Sigma_1\otimes \Sigma_2$ which takes exactly the same values on sets $\big\{A_1\times A_2\big\}_{A_1\in \Sigma_1,A_2\in \Sigma_2}$ as $\mu_1\otimes \mu_2$. By uniqueness of $\mu_1\otimes \mu_2$ we have
$$(\mu_1\otimes \mu_2)(E) = \int_{X_2}\mu_1(E_{x_2})\,d\mu_2$$
This finishes the proof of \textbf{(5)}.
\end{proof}

\begin{definition}
Let $(X,\Sigma_1,\mu_1)$ and $(X_2,\Sigma_2,\mu_2)$ be spaces with $\sigma$-finite measures. The unique measure $\mu_1\otimes \mu_2$ on $\Sigma_1\otimes \Sigma_2$ such that
$$\left(\mu_1\otimes \mu_2\right)\left(A_1\times A_2\right) = \mu_1(A_1)\mu_2(A_2)$$
for every $A_1\in \Sigma_1, A_2\in \Sigma_2$ is \textit{the product measure of $\mu_1$ and $\mu_2$}.
\end{definition}
\noindent
Next results relate integration with respect to $\mu_1\otimes \mu_2$ to iterated integration with respect to $\mu_1$ and $\mu_2$.

\begin{theorem}[Tonelli]\label{theorem:Tonellis_theorem_for_measurable_functions}
Let $(X,\Sigma_1,\mu_1)$ and $(X_2,\Sigma_2,\mu_2)$ be spaces with $\sigma$-finite measures. Suppose that $f:X_1\times X_2\ra \ol{\RR}$ is a nonnegative function measurable with respect to $\Sigma_1\otimes \Sigma_2$. Then functions
$$X_1\ni x_1 \mapsto \int_{X_2}f_{x_1}\,d\mu_2\in \ol{\RR}$$
and 
$$X_2\ni x_2 \mapsto \int_{X_1}f_{x_2}\,d\mu_1\in \ol{\RR}$$
are measurable with respect to $\Sigma_1$ and $\Sigma_2$, respectively. Moreover, we have equality
$$\int_{X_1}\int_{X_2}f_{x_1}\,d\mu_2d\mu_1 = \int_{X_1\times X_2}f\,d(\mu_1\otimes \mu_2) = \int_{X_2}\int_{X_1}f_{x_2}\,d\mu_1d\mu_2$$
\end{theorem}
\begin{proof}
Let $\cF$ be a family of all nonnegative functions $f:X_1\times X_2\ra \ol{\RR}$ that are measurable with respect to $\Sigma_1\otimes \Sigma_2$ such that functions
$$X_1\ni x_1\mapsto \int_{X_2}f_{x_1}\,d\mu_2\in \ol{\RR},\,X_2\ni x_2\mapsto \int_{X_1}f_{x_2}\,d\mu_1\in \ol{\RR}$$
are measurable with respect to $\Sigma_1, \Sigma_2$, respectively, and the formula
$$\int_{X_1}\int_{X_2}f_{x_1}\,d\mu_2d\mu_1 = \int_{X_1\times X_2}f\,d(\mu_1\otimes \mu_2) = \int_{X_2}\int_{X_1}f_{x_2}\,d\mu_1d\mu_2$$
holds. Then $\cF$ is closed under linear combinations of its elements with nonnegative coefficients. Next if $\{f_n:X_1\times X_2\ra \ol{\RR}\}_{n\in \NN}$ is a nondecreasing sequence of elements of $\cF$, then
$$\lim_{n\ra +\infty}f_n\in \cF$$
by Theorem \ref{theorem:monotone_convergence}. Finally $\mathbb{1}_E\in \cF$ for every $E\in \Sigma_1\otimes \Sigma_2$ by Theorem \ref{theorem:fubinis_theorem_basic}. According to Corollary \ref{corollary:measurable_induction_for_nonnegative} we derive that $\cF$ consists of all nonnegative functions measurable with respect to $\Sigma_1\otimes \Sigma_2$.
\end{proof}

\begin{theorem}[Fubini's theorem for integrable functions]\label{corollary:fubinis_for_integrable}
Let $(X,\Sigma_1,\mu_1)$ and $(X_2,\Sigma_2,\mu_2)$ be spaces with $\sigma$-finite measures and let $Y$ be a Banach space over $\RR$ or $\CC$. Suppose that $f:X_1\times X_2\ra Y$ is a function integrable with respect to $\mu_1\otimes \mu_2$. Then there are sets $N_i$ in $\Sigma_i$ for $i=1,2$ such that
$$\mu_1(N_1) = \mu_2(N_2) = 0$$ 
and functions
$$X_1\setminus N_1\ni x_1\mapsto \int_{X_2}f_{x_1}\,d\mu_2\in Y,\,X_2\setminus N_2\ni x_2\mapsto \int_{X_1}f_{x_2}\,d\mu_1\in Y$$
are well defined and integrable with respect to $\mu_1, \mu_2$, respectively. Moreover, we have equality
$$\int_{X_1}\int_{X_2} \left(\mathbb{1}_{\left(X_1\setminus N_1\right)\times X_2}\right)_{x_1}\cdot f_{x_1}\,d\mu_2d\mu_1 = \int_{X_1\times X_2}f\,d(\mu_1\otimes \mu_2) = \int_{X_2}\int_{X_1} \left(\mathbb{1}_{X_1\times \left(X_2\setminus N_2\right)}\right)_{x_2}\cdot f_{x_2}\,d\mu_1d\mu_2$$
\end{theorem}
\begin{proof}
Let $\cF$ be a family of all $\left(\mu_1\otimes \mu_2\right)$-integrable functions $f:X_1\times X_2\ra Y$ such that the statement holds for $\cF$. Then according to Theorem \ref{theorem:Tonellis_theorem_for_mu_measurable_functions} for every $y\in Y$ and $E\in \Sigma_1\otimes \Sigma_2$ such that $\left(\mu_1\otimes \mu_2\right)(E) \in \RR$ we have $y\cdot \mathbb{1}_E \in \cF$. Moreover, if $f, g\in \cF$, then for scalars $\alpha,\beta$ we have $\alpha f + \beta g\in \cF$. Suppose that $\{f_n:X_1\times X_2\ra Y\}_{n\in \NN}$ is a sequence of functions in $\cF$ which is $\mu$-almost everywhere pointwise convergent and $g:X_1\times X_2\ra \ol{\RR}$ is a nonnegative, $\mu$-measurable function such that
$$\int_{X_1\times X_2}g\,d\mu \in \RR$$
and $||f_n||\leq g$ holds $\mu$-almost everywhere for $n\in \NN$. Let $f$ be a $\mu$-almost everywhere pointwise limit of $\{f_n\}_{n\in \NN}$. Then by Theorems \ref{theorem:dominated_convergence} and \ref{theorem:Tonellis_theorem_for_mu_measurable_functions} we have $f\in \cF$. From Corollary \ref{corollary:measurable_induction_for_banach_valued} we derive that $\cF$ contains all $\left(\mu_1\otimes \mu_2\right)$-integrable functions.
\end{proof}










\small
\bibliographystyle{apalike}
\bibliography{../zzz}

\end{document}