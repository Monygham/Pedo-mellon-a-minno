\input ../pree.tex

\begin{document}

\title{Introduction to measure theory}
\date{}
\maketitle

\section{Introduction}
\noindent
These notes discuss basic notions related to measure theory. We introduce important classes of sets, introduce measurable spaces and maps and discuss rudimentary notions related to measures. In the last two sections we develop outer measure constructions due to Carath{\'e}odory.

\section{Families of sets}
\noindent
In these section we study families of sets closed under certain set-theoretic operations.

\begin{definition}\label{definition:families_of_sets}
Let $X$ be a set and let $\cF$ be a family of subsets of $X$. 
\begin{enumerate}[label=\textbf{(\arabic*)}, leftmargin=*]
\item $\cF$ is \textit{an algebra} if it contains $X$ and is closed under finite unions, intersections and complements. 
\item $\cF$ is \textit{a $\sigma$-algebra} if it is an algebra and is closed under countable unions.
\item $\cF$ is \textit{an $\omega$-monotone family} if it is closed under unions of countable non-decreasing sequences of subsets and under intersections of countable non-increasing sequences of subsets.
\item $\cF$ is \textit{a $\pi$-system} if it is closed under finite intersections.
\item $\cF$ is \textit{a $\lambda$-system} if it contains $X$ and is closed under complements and countable disjoint unions. 
\end{enumerate}
\end{definition}

\begin{remark}\label{remark:intersections}
Let $X$ be a set and let $\{\cF_i\}_{i\in I}$ be a class of families of subsets of $X$. Suppose that $\cF_i$ is an algebra for each $i \in I$, then $\bigcap_{i\in I}\cF_i$ is an algebra. Similarly for other types of families described in Definition \ref{definition:families_of_sets}.
\end{remark}

\begin{definition}
Let $X$ be a set and let $\cF$ be a family of subsets of $X$. 
\begin{enumerate}[label=\textbf{(\arabic*)}, leftmargin=*]
\item Then \textit{the $\sigma$-algebra generated by $\cF$ in $X$} is the intersection of all $\sigma$-algebras of subsets of $X$ which contain $\cF$. It is also denoted by $\sigma(\cF)$.
\item Then \textit{the $\lambda$-system generated by $\cF$ in $X$} is the intersection of all $\lambda$-systems of subsets of $X$ which contain $\cF$. It is also denoted by $\lambda(\cF)$.
\item Then \textit{the $\omega$-monotone family generated by $\cF$ in $X$} is the intersection of all $\omega$-monotone families of subsets of $X$ which contain $\cF$. It is also denoted by $\cM(\cF)$.
\end{enumerate}
\end{definition}

\begin{theorem}[Dynkin's $\pi$-$\lambda$ lemma]\label{theorem:dynkins_lemma}
Let $X$ be a set and let $\cP$ be a $\pi$-system of its subsets. Then $\lambda(\cP)=\sigma(\cP)$.
\end{theorem}
\noindent
For the proof we need the following result.

\begin{lemma}\label{lemma:dynkins_lemmal}
Let $\cL$ be a $\lambda$-system and let $A \in \cL$. Then 
$$\cL_A=\big\{B\subseteq X\,\big|\,A\cap B\in \cL \big\}$$
is a $\lambda$-system.
\end{lemma}
\begin{proof}[Proof of the lemma]
We verify that $\cL_A$ is closed under complements. Fix $B \in \cL_A$. We have $X\setminus A \in \cL$ and $A\cap B\in \cL$. Since $\cL$ is $\lambda$-system, we obtain
$$\left(X\setminus A\right)\cup \left(A\cap B\right)\in \cL$$
and thus 
$$A\setminus B = X\setminus \bigg(\left(X\setminus A\right)\cup \left(A\cap B\right)\bigg) \in \cL$$
This proves that $X\setminus B \in \cL_A$ and this proves that $\cL_A$ is closed under complements. Since $A\in cL$, we derive that $X\in \cL_A$ and the fact that $\cL_A$ is closed under disjoint countable unions is straightforward.
\end{proof}

\begin{proof}[Proof of the theorem]
Fix $A\in \cP$. Define $\cL_A$ as in Lemma \ref{lemma:dynkins_lemmal} with $\cL = \lambda(\cP)$. Then $\cL_A$ is a $\lambda$-system. Moreover, $\cL_A$ contains $\cP$. Hence $\lambda(\cP)\subseteq \cL_A$. This shows that $\lambda(\cP)$ is closed under intersections with members of $\cP$. Now fix $A\in \lambda(\cP)$ and define $\cL_A$ as in Lemma \ref{lemma:dynkins_lemmal} with $\cL = \lambda(\cP)$. Then $\cP \subseteq \cL_A$ and $\cL_A$ is a $\lambda$-system. Thus $\lambda(\cP)\subseteq \cL_A$. This proves that $\lambda(\cP)$ is a $\pi$-system. A $\pi$-system that is simultaneously a $\lambda$-system is a $\sigma$-algebra. Thus $\sigma(\cP)\subseteq \lambda(\cP)$. Since it is clear that $\lambda(\cP) \subseteq \sigma(\cP)$, we derive that $\lambda(\cP)=\sigma(\cP)$.
\end{proof}

\begin{theorem}[Sierpi{\'n}ski's lemma on $\omega$-monotone classes]\label{theorem:monotone_classes}
Let $X$ be a set and let $\cA$ be an algebra of its subsets. Then $\cM(\cA)=\sigma(\cA)$.
\end{theorem}
\noindent
For the proof we need the following easy results. Their proofs are left to the reader.

\begin{lemma}\label{lemma:first_monotone_classes}
Let $\cM$ be an $\omega$-monotone family. Then for every $A \in \cM$ family
$$\cM_A = \big\{B\subseteq X\,\big|\,A\cap B\in \cM\big\}$$
is $\omega$-monotone.
\end{lemma}

\begin{lemma}\label{lemma:second_monotone_classes}
Let $\cM$ be an $\omega$-monotone family. Then a family
$$\cM^c= \big\{A \subseteq X \,\big|\,A\in \cM\mbox{ and }X\setminus A\in \cM\big\}$$
is $\omega$-monotone.
\end{lemma}

\begin{proof}[Proof of the theorem]
Fix $A\in \cA$. Define $\cM_A$ as in Lemma \ref{lemma:first_monotone_classes} with $\cM = \cM(\cA)$. Then $\cM_A$ is an $\omega$-monotone family. Moreover, $\cM_A$ contains $\cA$. Hence $\cM(\cA)\subseteq \cM_A$. This shows that $\cM(\cA)$ is closed under intersections with members of $\cA$. Now fix $A\in \cM(\cA)$ and define $\cM_A$ as in Lemma \ref{lemma:first_monotone_classes} with $\cM = \cM(\cA)$. Then $\cA \subseteq \cM_A$ and $\cM_A$ is a monotone family. Thus $\cM(\cA)\subseteq \cM_A$. This proves that $\cM(\cA)$ is closed under finite intersections. According to Lemma \ref{lemma:second_monotone_classes} we derive that $\cM(\cA)^c$ is an $\omega$-monotone family and contains $\cA$. Hence $\cM(\cA)\subseteq \cM(\cA)^c$ and thus $\cM(\cA)$ is closed under complements. Therefore, $\cM(\cA)$ is a $\sigma$-algebra. Thus $\sigma(\cA)\subseteq \cM(\cA)$. Since it is clear that $\cM(\cA) \subseteq \sigma(\cA)$, we derive that $\cM(\cA) = \sigma(\cA)$.
\end{proof}

\section{Measures}

\begin{definition}
A pair $(X,\Sigma)$ consisting of a set $X$ together with a $\sigma$-algebra $\Sigma$ of its subsets is \textit{a measurable space}.
\end{definition}

\begin{definition}
Let $(X,\Sigma)$ and $(Y,\Delta)$ be measurable spaces. A map $f:X\ra Y$ is \textit{measurable} if $f^{-1}(B)\in \Sigma$ for every $B\in \Delta$.
\end{definition}

\begin{definition}
Let $X$ be a set and let $\Sigma$ be an algebra of its subsets. Consider a function $\mu:\Sigma \ra [0,+\infty]$.
\begin{enumerate}[label=\textbf{(\arabic*)}, leftmargin=*]
\item $\mu$ is \textit{a finitely additive function on $\Sigma$} if $\mu(\emptyset) = 0$ and 
$$\mu\left(A\cup B\right) = \mu(A) + \mu(B)$$
for each pair of disjoint sets $A,B\in \Sigma$.
\item $\mu$ is \textit{a $\sigma$-additive function on $\Sigma$} if $\mu(\emptyset) = 0$ and
$$\mu\left(\bigcup_{n\in \NN}A_n\right)=\sum_{n\in \NN}\mu(A_n)$$
for every family  $\{A_n\}_{n\in \NN}$ of pairwise disjoint subsets in $\Sigma$ such that $\bigcup_{n\in \NN}A_n\in \Sigma$.
\item $\mu$ is \textit{a measure on $\Sigma$} if $\Sigma$ is a $\sigma$-algebra and $\mu$ is $\sigma$-additive function.
\end{enumerate}
\end{definition}

\begin{definition}
A tuple $(X,\Sigma,\mu)$ consisting of a measurable space $\left(X,\Sigma\right)$ and a measure $\mu$ on $\Sigma$ is called \textit{a space with measure}.
\end{definition}

\begin{definition}
Let $(X,\Sigma,\mu)$ be a space with measure, let $(Y,\Delta)$ be a measurable space and let $f:X\ra Y$ be a measurable map. Then a measure on $\Delta$ given by formula
$$\Delta \ni B \mapsto \mu\left(f^{-1}(B)\right) \in [0,+\infty]$$
is \textit{a pushforward of $\mu$ along $f$} and is usually denoted by $f_*\mu$.
\end{definition}

\begin{definition}
Let $(X,\Sigma,\mu)$ and $(Y,\Delta,\nu)$ be spaces with measures. A map $f:X\ra Y$ is \textit{a morphism of spaces with measures} if $f$ is a morphism of measurable spaces and $\nu = f_*\mu$.
\end{definition}

\begin{definition}
Let $(X,\Sigma,\mu)$ be a space with measure.
\begin{enumerate}[label=\textbf{(\arabic*)}, leftmargin=*]
\item $\mu$ is \textit{finite} if $\mu(X) \in \RR$.
\item $\mu$ is \textit{$\sigma$-finite} if there exists a sequence $\{X_n\}_{n\in \NN}$ in $\Sigma$ such that 
$$X = \bigcup_{n\in \NN}X_n$$
and $\mu(X_n)\in \RR$ for each $n\in \NN$.
\end{enumerate}
\end{definition}

\begin{theorem}\label{theorem:uniqueness_on_pi_system}
Let $(X,\Sigma)$ be a measurable space and let $\mu_1,\mu_2$ be finite measures on $\Sigma$ such that $\mu_1(X)=\mu_2(X)$. Suppose that $\cP$ is a $\pi$-system of subsets of $X$ such that $\mu_1(A)=\mu_2(A)$ for every $A\in \cP$. Then $\mu_1(A) = \mu_2(A)$ for every $A\in \sigma(\cP)$.
\end{theorem}
\begin{proof}
Define $\cF = \big\{A\in \Sigma\,\big|\,\mu_1(A)=\mu_2(A) \big\}$. Straightforward verification shows that $\cF$ is a $\lambda$-system. By assumption $\cP \subseteq \cF$. Therefore, $\lambda(\cP) \subseteq \cF$. By Theorem \ref{theorem:dynkins_lemma} we deduce that $\sigma(\cP)=\lambda(\cP)$. Hence $\mu_1$ and $\mu_2$ are equal on sets in $\sigma(\cP)$.
\end{proof}



\section{Outer measures and Carath{\'e}odory's construction}

\begin{definition}
Let $X$ be a set and let $\mu^*:\cP(X)\ra [0,+\infty]$ be a function. Suppose that $\mu^*$ satisfies the following conditions.
\begin{enumerate}[label=\textbf{(\arabic*)}, leftmargin=*]
\item $\mu^*(\emptyset) = 0$
\item $\mu^*(A) \leq \mu^*(B)$ for every pair of subsets $A,B\subseteq X$ such that $A\subseteq B$.
\item $$\mu^*\left( \bigcup_{n\in \NN}A_n \right)\leq \sum_{n\in \NN}\mu^*(A_n)$$ 
for every family  $\{A_n\}_{n\in \NN}$ of subsets of $X$.
\end{enumerate}
Then $\mu^*$ is \textit{an outer measure on $X$}.
\end{definition}

\begin{definition}
Let $X$ be a set and let $\mu^*$ be an outer measure on $X$. A subset $A$ of $X$ satisfies \textit{Carath{\'e}dory's condition with respect to $\mu^*$} if 
$$\mu^*(E)=\mu^*(E\cap A)+\mu^*(E\setminus A)$$
for each $E\subseteq X$.
\end{definition}

\begin{theorem}[Carath{\'e}odory's construction]\label{theorem:caratheodory_construction}
Let $X$ be a set and let $\mu^*$ be an outer measure on $X$. Let $\Sigma_{\mu^*}$ be the family of all subsets of $X$ which satisfy Carath{\'e}odory's condition for $\mu^*$. Then the following assertions hold.
\begin{enumerate}[label=\emph{\textbf{(\arabic*)}}, leftmargin=*]
\item $\Sigma_{\mu^*}$ is a $\sigma$-algebra of subsets of $X$.
\item For every family $\{A_n\}_{n\in \NN}$ of pairwise disjoint subsets of $\Sigma_{\mu^*}$ and every subset $E$ of $X$ we have
$$\mu^*\left(E\cap \bigcup_{n\in \NN}A_n\right)=\sum_{n\in \NN}\mu^*(E\cap A_n)$$
In particular, $\mu^*_{\mid \Sigma_{\mu^*}}$ is a measure.
\item Every subset $A$ of $X$ such that $\mu^*(A)=0$ is contained in $\Sigma_{\mu^*}$.
\end{enumerate}
\end{theorem}

\begin{proof}
We first claim that $\Sigma_{\mu^*}$ is an algebra of subsets of $X$. Clearly $X \in \Sigma_{\mu^*}$ and $A \in \Sigma_{\mu^*}$ if and only if $X\setminus A \in \Sigma_{\mu^*}$. Thus in order to prove the claim it suffices to show that $\Sigma_{\mu^*}$ is closed under intersections. Pick $A,B\in \Sigma_{\mu^*}$ and let $E \subseteq X$. Then
$$\mu^*(E) = \mu^*(E\cap A) + \mu^*(E\setminus A) = \mu^*\big(E\cap (A\cap B)\big) + \mu^*\big((E\cap A) \setminus B\big) + \mu^*(E\setminus A)$$
where we first apply Carth{\'e}odory's condition to $A \in \Sigma_{\mu^*}$ with $E\subseteq X$ and then we apply Carth{\'e}odory's condition to $B\in \Sigma_{\mu^*}$ with $E\cap A\subseteq X$. Note that
$$\big(E\setminus (A \cap B)\big)\cap A = (E\cap A) \setminus B$$
and
$$\big(E\setminus (A\cap B)\big)\setminus A = E\setminus A$$
Thus applying Carath{\'e}odory's condition to $A\in \Sigma_{\mu^*}$ with $E\setminus (A\cap B)\subseteq X$ yields
$$\mu^*\big(E\setminus (A\cap B)\big) = \mu^*\big((E\cap A) \setminus B\big) + \mu^*(E\setminus A)$$
Therefore, we have
$$\mu^*(E) = \mu^*\big(E\cap (A\cap B)\big) + \underbrace{\mu^*\big((E\cap A) \setminus B\big) + \mu^*(E\setminus A)}_{=\mu^*\big(E\setminus (A\cap B)\big)} = $$
$$=\mu^*\big(E\cap (A\cap B)\big) + \mu^*\big(E\setminus (A\cap B)\big)$$
Since $E\subseteq X$ is arbitrary, we derive that $A\cap B$ satisfies Carath{\'e}odory condition. Hence $A\cap B \in \Sigma_{\mu^*}$ and $\Sigma_{\mu^*}$ is an algebra of subsets of $X$.\\
Next we claim that 
$$\mu^*\bigg(E\cap \bigcup_{n=0}^mA_n\bigg) = \sum_{n=0}^m\mu^*(E\cap A_n)$$
for every family $A_0,...,A_m$ of pairwise disjoint elements of $\Sigma_{\mu^*}$. Indeed, this follows by induction on $m \in \NN$. For $m = 0$ it is clear. Suppose that the equality holds for some $m$. We have
$$\mu^*\bigg(E\cap \bigcup_{n=0}^{m+1}A_n\bigg) = \mu^*(E\cap A_{m+1}) + \mu^*\bigg(E\cap \bigcup_{n=0}^mA_n\bigg)
$$
by Carath{\'e}odory's condition for $A_{m+1}$ and the fact that $A_{m+1}$ is disjoint from $A_0,...,A_m$. Now by induction assumption
$$\mu^*\bigg(E\cap \bigcup_{n=0}^mA_n\bigg) = \sum_{n=0}^m\mu^*(E\cap A_n)$$
and hence
$$\mu^*\bigg(E\cap \bigcup_{n=0}^{m+1}A_n\bigg) = \sum_{n=0}^{m+1}\mu^*(E\cap A_n)$$
which proves the claim.\\
Now we complete the proof of the theorem. Fix a family $\{A_n\}_{n\in \NN}$ of pairwise disjoint elements of $\Sigma_{\mu^*}$. Let $E$ be a subset of $X$. From the first fact proved above we have
$$\bigcup_{n=0}^mA_n \in \Sigma_{\mu^*}$$
for every $m\in \NN$. Next from the second fact we have
$$\sum_{n=0}^m\mu^*(E\cap A_n) + \mu^*\bigg(E\setminus \bigcup_{n=0}^mA_n\bigg) = \mu^*\bigg(E\cap \bigcup_{n=0}^mA_n\bigg) + \mu^*\bigg(E\setminus \bigcup_{n\in \NN}A_n\bigg)$$
for every $m\in \NN$. Hence combining them we get
$$\mu^*(E) = \sum_{n=0}^m\mu^*(E\cap A_n) + \mu^*\bigg(E\setminus \bigcup_{n=0}^mA_n\bigg)$$
for every $m\in \NN$. Now from the fact that $\mu^*$ is an outer measure on $X$ we have
$$\mu^*(E) \leq \mu^*\bigg(E\cap \bigcup_{n\in \NN}A_n\bigg) + \mu^*\bigg(E\setminus \bigcup_{n\in \NN}A_n\bigg) \leq \sum_{n\in \NN}\mu^*(E\cap A_n) + \mu^*\bigg(E\setminus \bigcup_{n\in \NN}A_n\bigg) = $$
$$= \lim_{m\ra +\infty}\sum_{n=0}^m\mu^*(E\cap A_n) + \mu^*\bigg(E\setminus \bigcup_{n\in \NN}A_n\bigg) \leq \lim_{m\ra +\infty}\bigg(\sum_{n=0}^m\mu^*(E\cap A_n) + \mu^*\bigg(E\setminus \bigcup_{n=0}^mA_n\bigg)\bigg)$$
but the right hand side is equal to $\mu^*(E)$. Thus every inequality in the formula above must be equality. Therefore, we have
$$\mu^*(E) = \mu^*\bigg(E\cap \bigcup_{n\in \NN}A_n\bigg) + \mu^*\bigg(E\setminus \bigcup_{n\in \NN}A_n\bigg)$$
and
$$\mu^*\bigg(E\cap \bigcup_{n\in \NN}A_n\bigg) = \sum_{n\in \NN}\mu^*(E\cap A_n)$$
Since $E\subseteq X$ is arbitrary, we deduce that the union of $\{A_n\}_{n\in \NN}$ is an element of $\Sigma_{\mu^*}$. This proves that the algebra $\Sigma_{\mu^*}$ of subsets of $X$ is a $\sigma$-algebra. The second assertion in the statement of the theorem is also proved according to the fact that $E\subseteq X$ is arbitrary. It suffice to prove that if $\mu^*(A) = 0$ for some $A\subseteq X$, then $A$ satisfies Carath{\'e}odory condition for $\mu^*$. This is left for the reader as an exercise.
\end{proof}
\noindent
Next result is a general tool for constructing measures.

\begin{theorem}[Carath{\'e}odory extension theorem]\label{theorem:caratheodory_extension_result}
Let $X$ be a set and let $\Sigma$ be an algebra of its subsets. Suppose that $\mu$ is a $\sigma$-additive function on $\Sigma$. For every subset $A$ in $X$ we define
$$\mu^*(A) = \inf \bigg\{\sum_{n\in \NN}\mu(A_n)\,\big|\,A_n\in \Sigma\mbox{ for every }n\in \NN\mbox{ and }A\subseteq \bigcup_{n\in \NN}A_n\bigg\}$$
Then the following assertions hold.
\begin{enumerate}[label=\emph{\textbf{(\arabic*)}}, leftmargin=*]
\item $\mu^*$ is an outer measure on $X$.
\item $\Sigma \subseteq \Sigma_{\mu^*}$ and $\mu^*_{\mid \Sigma} = \mu$.
\item If $\mu(X)$ is finite, then $\mu^*_{\mid \sigma(\Sigma)}$ is a unique extension of $\Sigma$ to a measure on $\sigma(\Sigma)$.
\end{enumerate}
\end{theorem}
\begin{proof}
Standard verification shows that $\mu^*$ is an outer measure on $X$. Note that 
$$\mu^*(A) = \inf \bigg\{\sum_{n\in \NN}\mu(A_n)\,\big|\,\{A_n\}_{n\in \NN}\mbox{ is a family of pairwise disjoint subsets of }\Sigma \mbox{ and }A\subseteq \bigcup_{n\in \NN}A_n\bigg\}$$
for every subset $A$ of $X$. Let $A$ be element of $\Sigma$ and let $E$ be an arbitrary subset of $X$. Fix $\epsilon > 0$. By the remark above there exists a family $\{A_n\}_{n\in \NN}$ of pairwise disjoint elements of $\Sigma$ such that
$$E\subseteq \bigcup_{n\in \NN}A_n,\,\sum_{n\in \NN}\mu(A_n)\leq \mu^*(E)+\epsilon$$
By definition of $\mu^*$ we have $\mu^*(E\cap A)\leq \sum_{n\in \NN}\mu(A_n\cap A)$, $\mu^*(E\setminus A) \leq \sum_{n\in \NN}\mu(A_n\setminus A)$ and hence
$$\mu^*(E)\leq \mu^*(E\cap A)+\mu^*(E\setminus A) \leq \sum_{n\in \NN}\mu(A_n\cap A)+\sum_{n\in \NN}\mu(A_n\setminus A)=$$
$$=\sum_{n\in \NN}\big(\mu(A_n\cap A)+\mu(A_n\setminus A)\big)=\sum_{n\in \NN}\mu(A_n)\leq \mu^*(E)+\epsilon$$
Since $\epsilon > 0$ is arbitrary, we derive that $\mu^*(E) = \mu^*(E\cap A)+\mu^*(E\setminus A)$ and hence $A\in \Sigma_{\mu^*}$. Thus $\Sigma \subseteq \Sigma_{\mu^*}$. Once again fix $A\in \Sigma$. Then for every family $\{A_n\}_{n\in \NN}$ of pairwise disjoint elements of $\Sigma$ such that $A\subseteq \bigcup_{n\in \NN}A_n$ we have $\mu(A) = \sum_{n\in \NN}\mu(A_n\cap A)\leq \sum_{n\in \NN}\mu(A_n)$ and thus $\mu(A)\leq \mu^*(A)$. Obviously $\mu^*(A)\leq \mu(A)$. Therefore, for every $A\in \Sigma$ we have $\mu(A) = \mu^*(A)$. Together with $\Sigma\subseteq \Sigma_{\mu^*}$ this implies that $\mu^*_{\mid \sigma(\Sigma)}$ is a measure that extends $\mu$. Now Theorem \ref{theorem:uniqueness_on_pi_system} implies the uniqueness of extension under the assumption that $\mu(X)$ is finite.
\end{proof}

\section{Outer metric measures}

\begin{definition}
Let $X$ be a topological space. Then $\sigma$-algebra $\cB(X)$ generated by all open sets of $X$ is called \textit{the $\sigma$-algebra of Borel subsets of $X$}.
\end{definition}

\begin{definition}
Let $(X,d)$ be a metric space and $\mu^*$ be an outer measure on $X$. We say that $\mu^*$ is \textit{a metric outer measure on $(X,d)$} if  
$$\mu^*(E_1\cup E_2)=\mu^*(E_1)+\mu^*(E_2)$$
for any two subsets $E_1$, $E_2$ of $X$ with $\mathrm{dist}(E_1,E_2) > 0$.
\end{definition}

\begin{theorem}[Carath{\'e}odory]
Let $(X,d)$ be a metric space and let $\mu^*$ be an outer metric measure on $(X,d)$. Then the $\sigma$-algebra $\cB(X)$ of Borel subsets of $X$ is contained in $\Sigma_{\mu^*}$.
\end{theorem}
\begin{proof}
Let $U$ be an open subset of $X$. Define $U_n=\big\{x\in X\mid \mathrm{dist}\left(x,X\setminus U\right)>\frac{1}{2^n}\big\}$ for $n\in \NN$. Then $\{U_n\}_{n\in \NN}$ form an ascending family of open sets and $U=\bigcup_{n\in \NN}U_n$. Fix now a subset $E$ of $X$ such that $\mu^*(E)\in \RR$. We define $E_n=E\cap U_n$ for every $n\in \NN$. Since $\mu^*$ is an outer metric measure, we derive that
$$\mu^*\left(\bigcup_{n=0}^mE_{2n+1}\setminus E_{2n}\right)=\sum_{n=0}^m\mu^*(E_{2n+1}\setminus E_{2n}),\,\mu^*\left(\bigcup_{n=1}^mE_{2n}\setminus E_{2n-1}\right)=\sum_{n=1}^m\mu^*(E_{2n}\setminus E_{2n-1})$$
for every positive integer $m$. Thus we derive 
$$\sum_{n\in \NN}\mu^*(E_{2n+1}\setminus E_{2n})\leq \mu^*(E)\in \RR,\,\sum_{n\in \NN}\mu^*(E_{2n}\setminus E_{2n-1})\leq \mu^*(E)\in \RR$$
Hence we have
$$\sum_{n\in \NN}\mu^*(E_{n+1}\setminus E_{n})\leq 2\cdot \mu^*(E)\in \RR$$
Using the fact that $\mu^*$ is an outer measure, we derive that
$$\mu^*(E_m)\leq \mu^*(E\cap U)\leq \mu^*(E_m)+\sum_{n=m}^{+\infty}\mu^*(E_{n+1}\setminus E_n)$$
for every $m\in \NN$. Hence these inequalities yield
$$\lim_{m\ra +\infty}\mu^*(E_m)=\mu^*(E\cap U)$$
Now we have 
$$\mu^*(E)\leq \mu^*(E\cap U)+\mu^*(E\setminus U) \leq \lim_{m\ra +\infty}\mu^*(E_m) + + \mu^*(E\setminus U) = $$
$$= \lim_{m\ra +\infty}\bigg(\mu^*(E_m) + + \mu^*(E\setminus U)\bigg) = \lim_{m\ra +\infty}\mu^*\bigg(E_m\cup \big(E\setminus U\big)\bigg) \leq \mu^*(E)$$ 
for every $m\in \NN$. Note that if $\mu^*(E)=+\infty$, then inequality $\mu^*(E)\leq \mu^*(E\cap U)+\mu^*(E\setminus U)$ must be equality. Hence $U$ satisfies Carath{\'e}odory condition with respect to $\mu^*$. This implies that $U \in \Sigma_{\mu^*}$. Since $U$ is an arbitrary open subset of $X$, we deduce that $\cB(X)\subseteq \Sigma_{\mu^*}$.
\end{proof}































































\end{document}