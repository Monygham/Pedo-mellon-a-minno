\input ../pree.tex

\begin{document}

\title{Radon-Nikodym theorem, Hahn-Jordan decomposition and Lebesgue decomposition}
\date{}
\maketitle

\section{Introduction}
\noindent
This notes are devoted to some more advanced notions in measure theory. Tools presented here are indispensable in probability theory and statistics. We refer to \cite{Introductiontomeasuretheory} for extensive introduction to measure theory. 

\section{Signed and Complex Measures}
\noindent
In this section we define extension of the usual notion of measure. Denote by $\ol{\RR}=\RR\cup \{-\infty,+\infty\}$ the topological space obtained as a two-point compactification of the line $\RR$. Addition is partially defined operation on $\ol{\RR}$ given by the following rules
$$(+\infty)+r=+\infty=r+(+\infty),\,(-\infty)+r=-\infty=r+(-\infty)$$
for every $r\in \RR$

\begin{definition}
Let $\left(X,\Sigma\right)$ be a measurable space. \textit{A signed measure on $(X,\Sigma)$} is a function $\nu:\Sigma\ra \ol{\RR}$ such that $\nu(\emptyset)=0$ and 
$$\nu\left(\bigcup_{n\in \NN}A_n\right)=\sum_{n\in \NN}\nu(A_n)$$
for every family $\{A_n\}_{n\in \NN}$ of pairwise disjoint subsets of $\Sigma$. We also say that $\nu$ is \textit{a real measure on $(X,\Sigma)$} if it is signed measure and takes values in $\RR$.
\end{definition}

\begin{definition}
Let $\left(X,\Sigma\right)$ be a measurable space. \textit{A complex measure} is a function $\nu:\Sigma\ra \CC$ such that $\nu(\emptyset)=0$ and 
$$\nu\left(\bigcup_{n\in \NN}A_n\right)=\sum_{n\in \NN}\nu(A_n)$$
for every family $\{A_n\}_{n\in \NN}$ of pairwise disjoint subsets of $\Sigma$. 
\end{definition}

\begin{definition}
Let $(X,\Sigma)$ be a measurable space and let $\mu,\nu$ be two measures (either complex or signed) on $(X,\Sigma)$. Suppose that for every set $A$ in $\Sigma$ we have
$$\mu(A) = 0\,\Rightarrow \nu(A)=0$$
Then we write $\nu \ll \mu$ and say that $\nu$ is \textit{absolutely continuous with respect to $\mu$}.
\end{definition}

\begin{definition}
Let $(X,\Sigma)$ be a measurable space and let $\mu$, $\nu$ be two measures (either complex or signed) on $(X,\Sigma)$. Suppose that there exists a set $S\in \Sigma$ such that
$$\mu(A\cap S) = 0,\,\nu(A\setminus S) = 0$$
for every $A \in \Sigma$. Then we write $\nu \perp \mu$ and say that $\nu$ is \textit{singular with respect to $\mu$}.
\end{definition}

\section{Hahn-Jordan Decomposition}

\begin{theorem}[Hahn-Jordan decomposition]\label{theorem:jordansdecomposition}
Let $\left(X,\Sigma\right)$ be a measurable space and $\nu:\Sigma\ra \ol{\RR}$ be a signed measure. Then there exists the unique pair of measures $\nu_+,\nu_-:\Sigma \ra [0,+\infty]$ such that  
$$\nu = \nu_+ - \nu_-$$
and $\nu_+ \perp \nu_-$.
\end{theorem}
\noindent
For the proof we need the following notion.

\begin{definition}
Let $(X,\Sigma,\nu)$ be a space with signed measure. A set $A\in \Sigma$ is \textit{positive} if for every subset $B$ of $A$ such that $B\in \Sigma$ we have inequality $\nu(B)\geq 0$. 
\end{definition}

\begin{lemma}\label{lemma:positiveexist}
Let $B\in \Sigma$ be a set such that $\nu(B)\in \RR$ and $\nu(B)>0$. Then there exists a positive set $C\subseteq B$ such that $\nu(B)\leq \nu(C)$.
\end{lemma}
\begin{proof}[Proof of the lemma]
First note that all sets $A\in \Sigma$ contained in $B$ have finite measure (we left the proof as an exercise for the reader). For every subset $A\in \Sigma$ contained in $B$ we define
$$\delta(A) = \max\bigg\{\frac{1}{2}\inf\big\{\nu(D)\,\big|\,D\mbox{ is a subset of }A\mbox{ in }\Sigma\big\},-1\bigg\}$$
Note that $\delta(A)\leq 0$. Now we define a sequence $\{D_n\}_{n\in \NN}$ of disjoint subsets of $B$ and members of $\Sigma$. This is done recursively as follows. If $D_0,...,D_n$ are defined, then we pick $D_{n+1}$ as a subset of $B\setminus \left(D_0\cup...\cup D_n\right)$ in $\Sigma$ such that
$$\nu(D_{n+1})\leq \delta\big(B\setminus \left(D_0\cup...\cup D_n\right)\big)$$
Let
$$C = B\setminus \bigcup_{n\in \NN}D_n$$
be a subset of $B$. Clearly $C\in \Sigma$ and for every $n\in \NN$ we have
$$\delta(C) \geq \delta\big(B\setminus \left(D_0\cup...\cup D_n\right)\big)$$
Thus
$$\nu(C) = \nu(B)-\sum_{n\in \NN}\nu(D_n) \geq \nu(B) -\sum_{n\in \NN}\delta\big(B\setminus \left(D_0\cup...\cup D_n\right)\big) = \nu(B) -\sum_{n\in \NN}\delta(C)$$
Since $\nu(C)\in \RR$, we derive that $\delta(C)=0$. This implies that $C$ is a positive set and $\nu(C)\geq \nu(B)$.
\end{proof}

\begin{proof}[Proof of the theorem]
Assume that for every $A\in \Sigma$ we have $\nu(A) \in \RR\cup \{-\infty\}$. Now consider 
$$\alpha=\sup \big\{\nu(C)\,\big|\,C\mbox{ is positive}\big\}$$
We can find a nondecreasing sequence $\{\alpha_n\}_{n\in \NN}$ of nonnegative real numbers that converges to $\alpha$ and such that for every $n\in \NN$ there exists a positive set $C_n$ with $\nu(C_n)=\alpha_n$. Now pick $P=\bigcup_{n\in \NN}C_n$. Obviously $P$ is positive and $\nu(P)=\alpha$. In particular, $\alpha \in \RR$. Assume that there exists $B\in \Sigma$ such that $B\subseteq X\setminus P$ and $\nu(B)>0$. According to Lemma \ref{lemma:positiveexist} we deduce that there exists a positive set $C$ inside $B$ such that $\nu(B)\leq \nu(C)$. Then we get
$$\alpha=\nu(P)<\nu(P)+\nu(C)=\nu(P\cup C)$$
and $P\cup C$ is positive. This contradicts the definition of $\alpha$. Hence for every $B\subseteq X\setminus P$ such that $B\in \Sigma$ we have $\nu(B)\leq 0$. Thus measures
$$\nu_+(A) = \nu(A\cap P),\,\nu_-(A) = -\nu(A\setminus P)$$
defined for $A\in \Sigma$ satisfy the assertion of the theorem. This finishes the proof of the Hahn-Jordan decomposition under the assumption that $\nu(A)\in \RR\cup \{-\infty\}$ for all $A\in \Sigma$.\\
If $\nu(A)\in \RR\cup \{+\infty\}$ for every $A\in \Sigma$, then we apply the result above for $-\nu$. Finally the case $\nu(A_1)=-\infty$ and $\nu(A_2)=+\infty$ for some $A_1$, $A_2\in \Sigma$ yields to the contradiction. Hence Hahn-Jordan decomposition is proved.
\end{proof}

\begin{corollary}\label{corollary:oneisfinite}
Let $\left(X,\Sigma\right)$ be a measurable space and $\nu:\Sigma\ra \ol{\RR}$ be a signed measure. Then either $\nu_+$ or $\nu_-$ is finite.
\end{corollary}
\begin{proof}
According to Theorem \ref{theorem:jordansdecomposition} we have $\nu = \nu_+-\nu_-$ and both $\nu_+$, $\nu_-$ are nonnegative measures such that $\nu_+\perp \nu_-$. We cannot have $\nu_+(X)=\nu_-(X)=+\infty$, because then $\nu(X)$ would be undefined in $\ol{\RR}$. This implies that either $\nu_+(X)\in \RR$ or $\nu_-(X)\in \RR$.
\end{proof}

\section{Lebesgue decomposition}

\begin{definition}
Let $(X,\Sigma)$ be a measurable space and $\mu:\Sigma \ra \ol{\RR}$ be a signed measure. We say that $\mu$ is $\sigma$-finite if there exists a decomposition
$$X = \bigcup_{n\in \NN}X_n$$
onto pairwise disjoint elements of $\Sigma$ such that $\mu(X_n)\in \RR$ for every $n\in \NN$.
\end{definition}

\begin{theorem}[Lebesgue decomposition]\label{theorem:lebesguedecomposition}
Let $(X,\Sigma)$ be a measurable space and let $\mu$ be a $\sigma$-finite, measure on $(X,\Sigma)$. Suppose that $\nu$ is either a signed and $\sigma$-finite measure or a complex measure on $(X,\Sigma)$. Then there exists a unique decomposition 
$$\nu = \nu_s + \nu_a$$
of measure $\nu$ such that $\nu_s \perp \mu$ and $\nu_a \ll \mu$.
\end{theorem}
\begin{proof}
Suppose first that $\nu$ is finite measure. Consider
$$\alpha = \sup_{A\in \Sigma,\,\mu(A)=0}\nu(A)$$
Since $\nu$ is finite, we derive that $\alpha \in \RR$. Consider a sequence $\{A_n\}_{n\in \NN}$ such that $A_n\in \Sigma$, $\mu(A_n)=0$ for every $n\in \NN$ and $\lim_{n\ra +\infty}\nu(A_n)=\alpha$. Define $S = \bigcup_{n\in \NN}A_n$. Then $\mu(S) = 0$ and $\nu(S) = \alpha$. Moreover, if $A\in \Sigma$ and $A\cap S = \emptyset$, then $\mu(A) = 0$ implies $\nu(A) = 0$. Now we define $\nu_s(A) = \nu(A\cap S)$ and $\nu_a(A) = \nu(A\setminus S)$ for every $A\in \Sigma$. Then $\nu = \nu_s+\nu_a$ and $\nu_s \perp \mu$, $\nu_a \ll \mu$.\\
Now assume that $\nu$ is $\sigma$-finite measure. There exists a decomposition
$$X = \bigcup_{n\in \NN}X_n$$
onto pairwise disjoint elements of $\Sigma$ such that $\mu(X_n)\in \RR$ for every $n\in \NN$. We define $\nu_n(A) = \nu(A\cap X_n)$ for each $n\in \NN$ and $A\in \Sigma$. Then $\nu_n$ is a finite measure. By the case above we find $\nu_n = \nu_{ns}+ \nu_{na}$ and $\nu_{ns}\perp \mu$, $\nu_{na} \ll \mu$ for some measures on $\Sigma$. Now we define
$$\nu_s = \sum_{n\in \NN}\nu_{ns},\,\nu_a=\sum_{n\in \NN}\nu_{an}$$
Then $\nu = \nu_s+\nu_a$ and $\nu_s \perp \mu$, $\nu_a \ll \mu$.\\
Now consider the case when $\nu$ is $\sigma$-finite and signed measure. According to Theorem \ref{theorem:jordansdecomposition} we write $\nu = \nu_+-\nu_-$ for measures $\nu_+,\nu_-$ such that $\nu_+\perp \nu_-$. Then $\nu_+,\nu_-$ are $\sigma$-finite measures. According to previous case we can write $\nu_+ = \nu_{+s}+\nu_{+a}$, $\nu_- = \nu_{-s} + \nu_{-a}$ for measures such that $\nu_{+s} \perp \mu,\nu_{-s} \perp \mu,\nu_{+a}\ll \mu,\nu_{-a}\ll\mu$. Then $\nu_s = \nu_{+s} - \nu_{-s}, \nu_a=\nu_{+a}-\nu_{-a}$ are signed measures and $\nu_s \perp \mu, \nu_a\ll\mu$.\\
Finally assume that $\nu$ is complex. Then $\nu = \nu^r + i\cdot \nu^i$, where $\nu^r$ and $\nu^i$ are finite, signed measures. Form the case above we have decompositions
$$\nu^r = \nu^r_s+\nu^r_a,\,\nu^i=\nu^i_s+\nu^i_s$$
and $\nu^r_s \perp \mu$, $\nu^i_s \perp \mu$, $\nu^r_a \ll\mu$, $\nu^i_a \ll \mu$. Then complex measures 
$$\nu_s = \nu^r_s+i\cdot \nu^i_s,\,\nu_a = \nu^r_a+i\cdot \nu^i_a$$
satisfy $\nu_s \perp \mu$, $\nu_a \ll \mu$.
\end{proof}

\section{Radon-Nikodym theorem}
\noindent
In this section we prove famous result of Radon and Nikodym.

\begin{theorem}[Radon-Nikodym]\label{theorem:radonnikodymmain}
Let $(X,\Sigma)$ be a measurable space and let $\mu$ be a $\sigma$-finite measure on $(X,\Sigma)$. Suppose that $\nu \ll \mu$ for $\nu$ that is either complex measure or $\sigma$-finite, signed measure. Then there exists a measurable function $f:X\ra \CC$ such that
$$\nu(A) = \int_A f d\mu$$
for every $A\in \Sigma$.
\end{theorem}
\begin{proof}[Proof for finite measures $\mu,\nu$]
Fix $n\in \NN$ and $k\in \NN$. According to Theorem \ref{theorem:jordansdecomposition} there exists a set $P_{n,k}\in \Sigma$ such that
$$\left(\nu - \frac{k}{2^n}\cdot \mu\right)\left(A\cap P_{n,k}\right)\geq 0,\,\left(\nu - \frac{k}{2^n}\cdot \mu\right)\left(A\setminus P_{n,k}\right)\leq 0$$
for every $A\in \Sigma$. We may assume that $P_{n,0}=X$, $P_{n,k+1}\subseteq P_{n,k}$ and $P_{n,k} =P_{n+1,2k}$ for every $n, k\in \NN$. Since $\nu \ll \mu$ and $\nu$ is finite, we derive that
$$\mu\left(\bigcap_{k\in \NN}P_{n,k}\right) = \nu\left(\bigcap_{k\in \NN}P_{n,k}\right)=0$$
and may assume that this set is empty for every $n\in \NN$. Pick $n\in \NN$. We define a function $f_n:X\ra \CC$ by formula
$$f_n(x) =\sum_{k\in \NN} \frac{k}{2^n}\cdot \chi_{P_{n,k}\setminus P_{n,k+1}}(x)$$
Clearly $f_n$ is a measurable function with real nonnegative values. If $m,n\in \NN$ and $n\leq m$, then we have
$$f_n(x)\leq f_m(x)\leq f_n(x)+\frac{1}{2^n}$$
Thus $\{f_n\}_{n\in \NN}$ is a nondecreasing sequence of measurable functions convergent uniformly to a measurable function $f:X\ra \CC$. Moreover, for every $A\in \Sigma$ and $n\in \NN$ we have
$$\nu(A)-\frac{1}{2^n}\mu(A) = \sum_{k\in \NN}\nu\left(A\cap \left(P_{n,k}\setminus P_{n,k+1}\right)\right)-\frac{1}{2^n}\mu(A)  \leq$$
$$\leq \sum_{k\in \NN}\frac{k+1}{2^n}\mu\left(A\cap \left(P_{n,k}\setminus P_{n,k+1}\right)\right) - \frac{1}{2^n}\sum_{k\in \NN}\mu\left(A\cap \left(P_{n,k}\setminus P_{n,k+1}\right)\right) \leq  $$
$$\leq  \sum_{k\in \NN}\frac{k}{2^n}\mu\left(A\cap \left(P_{n,k}\setminus P_{n,k+1}\right)\right)\leq \sum_{k\in \NN}\nu\left(A\cap \left(P_{n,k}\setminus P_{n,k+1}\right)\right) = \nu(A)$$
and since
$$\int_Af_n\,d\mu =  \sum_{k\in \NN}\frac{k}{2^n}\mu\left(A\cap \left(P_{n,k}\setminus P_{n,k+1}\right)\right)$$
we derive that
$$\nu(A)-\frac{1}{2^n}\mu(A) \leq \int_Af_n\,d\mu \leq \nu(A)$$
This inequality together with monotone convergence theorem imply that
$$\nu(A) =  \lim_{n\ra +\infty}\int_Af_n\,d\mu = \int_Af\,d\mu$$
This finishes the proof for finite measures $\nu,\mu$.
\end{proof}

\begin{proof}[Reduction to finite case]
Assume now that $\nu$ and $\mu$ are $\sigma$-finite measures on $(X,\Sigma)$. Then there exists a decomposition
$$X= \bigcup_{n\in \NN}X_n$$
onto disjoint subsets in $\Sigma$ such that $\nu(X_n)\in \RR$ and $\mu(X_n)\in \RR$ for every $n\in \NN$. For every $n\in \NN$ we define $\nu_n(A) = \nu(A\cap X_n)$ and $\mu_n(A) = \mu(A\cap X_n)$ for $A\in \Sigma$. Since $\nu \ll \mu$, we derive that $\nu_n\ll \mu_n$ for every $n\in \NN$. Measures $\{\nu_n\}_{n\in \NN}$ and $\{\mu_n\}_{n\in \NN}$ are finite. By finite case of the theorem we deduce that for each $n\in \NN$ there exists a measurable function $f_n:X\ra \CC$ such that
$$\nu_n(A) = \int_A f_n\,d\mu_n$$
for every $A\in \Sigma$. Note that $f_n$ is real nonnegative and can be set equal to zero outside $X_n$. Thus
$$\nu_n(A) = \int_A f_n\,d\mu_n = \int_A f_n\,d\mu$$
for every $A\in \Sigma$. Therefore, we deduce that
$$\nu(A) = \sum_{n\in \NN}\nu(A\cap X_n) = \sum_{n\in \NN}\nu_n(A) =\sum_{n\in \NN}\int_Af_n\,d\mu = \int_A \left(\sum_{n\in \NN}f_n\right)\,d\mu$$
by monotone convergence theorem.\\
Assume now that both $ \nu$ is $\sigma$-finite, signed measure. In this situation we may write $\nu = \nu_+-\nu_-$ for measures $\nu_+, \nu_-$ such that $\nu_+\perp \nu_-$. There exists a set $P\in \Sigma$ such that $\nu_-(P)=\nu_+(X\setminus P)=0$. Then $\nu_+ \ll \mu$ and $\nu_-\ll \mu$. By the case considered previously there exist measurable functions $f_+:X\ra \CC$, $f_-:X\ra \CC$ such that
$$\nu_+(A) = \int_Af_+\,d\mu,\,\nu_-(A) = \int_Af_-\,d\mu$$
for every $A\in \Sigma$. Moreover, we may assume that $f_+$ is equal to zero outside $P$ and $f_-$ is equal to zero outside $X\setminus P$. From this we have
$$\nu(A) = \nu_+(A) + \nu_-(A) = \int_Af_+\,d\mu + \int_Af_-\,d\mu = \int_A\left(f_+-f_-\right)\,d\mu $$
for every $A\in \Sigma$.\\
Suppose that $\nu$ is complex measure. Write $\nu = \nu_r - i\cdot \nu_i$. Then both $\nu_r, \nu_-$ are finite, signed measures. Moreover, we have $\nu_r\ll\mu,\nu_i\ll \mu$. There exist measurable functions $f_r:X\ra \CC$ and $f_i:X\ra \CC$ that are real valued and satisfy
$$\nu_r(A) = \int_Af_r\,d\mu,\,\nu_i(A) = \int_Af_i\,d\mu$$
for every $A\in \Sigma$. Thus
$$\nu(A) = \nu_r(A) + i\cdot \nu_i(A) = \int_Af_r\,d\mu + i\cdot \int_Af_i\,d\mu = \int_A\left(f_r+i\cdot f_i\right)\,d\mu$$
for every $A\in \Sigma$.
\end{proof}

\section{Applications of Radon-Nikodyn theorem}

\begin{proposition}\label{proposition:derivativeandintegration}
Let $\mu$ be a measure on $(X,\Sigma)$ and $f:X\ra \CC$ be a measurable function taking nonnegative values. We define
$$\nu(A) = \int_Ag\,d\mu$$
for every $A\in \Sigma$. Then $\nu$ is a measure on $(X,\Sigma)$ and the equality
$$\int_X g\,d\nu =\int_Xg\cdot f\,d\mu$$
holds for every measurable function $g:X\ra \CC$ that is either $\nu$-integrable or takes nonnegative values.
\end{proposition}
\begin{proof}
Suppose that $A = \bigcup_{n\in \NN}A_n$ for $A\in \Sigma$ and $A_n\in \Sigma$ for every $n\in \NN$. Assume also that $\{A_n\}_{n\in \NN}$ are pairwise disjoint. Then by monotone convergence theorem
$$\nu(A) = \int_A f\,d\mu = \int_X\chi_A\cdot f\,d\mu = \int_X\left(\sum_{n\in \NN}\chi_{A_n}\cdot f\right)\,d\mu = \sum_{n\in \NN}\int_X\chi_{A_n}\cdot f\,d\mu = \sum_{n\in \NN}\int_{A_n}f\,d\mu = \sum_{n\in \NN}\nu(A_n)$$
Moreover, we have $\nu(\emptyset) = 0$. Thus $\nu$ is a measure on $(X,\Sigma)$.\\
For the second part of the statement note that the family of measurable functions $g:X\ra \CC$ satisfying equality
$$\int_Xg\,d\nu = \int_Xg\cdot f\,d\mu$$
contains $\{\chi_A\}_{A\in \Sigma}$, is closed under $\RR_{\geq 0}$-linear combinations of measurable functions taking nonnegative values, if it contains nondecreasing sequence $\{g_n:X\ra \CC\}_{n\in \NN}$ of measurable functions taking only nonnegative values, then it also contains its limit. Thus this family contains all measurable functions $g:X\ra \CC$ taking nonnegative values. Since every real valued, $\nu$-integrable function $g:X\ra \CC$ is a difference of a two $\nu$-integrable functions taking nonnegative values, we deduce that this family contains all real, $\nu$-integrable functions. Finally, if it contains two $\nu$-integrable, real valued functions, then it contains its $\CC$-linear combination. Thus it contains all $\nu$-integrable functions.
\end{proof}

\begin{theorem}\label{theorem:structureofcomplexmeasures}
Let $\mu$ be a complex measure on $(X,\Sigma)$. There exists a measurable function $f:X\ra \CC$ such that
$$\mu(A) = \int_Af\,d|\mu|$$
for every $A\in \Sigma$ and $|f(x)|=1$ for every $x$ in $X$.
\end{theorem}
\noindent
For the proof we need the following result.
\begin{lemma}\label{lemma:convexvaluesofintegral}
Let $\mu$ be a measure on $(X,\Sigma)$. Suppose that $f:X\ra \CC$ is a measurable function and $F$ is a closed subset of $\CC$. Assume that for every $A\in \Sigma$ such that $\mu(A)>0$, we have
$$\frac{1}{\mu(A)}\int_A f\,d\mu \in F$$
Then $\mu\left(X\setminus f^{-1}(F)\right)=0$.
\end{lemma}
\begin{proof}[Proof of the lemma]
Let $D$ be a closed disc in $\CC$ such that $D\cap F = \emptyset$. If $\mu\left(f^{-1}(D)\right) > 0$, then
$$\frac{1}{\mu\left(f^{-1}(D)\right)}\int_{f^{-1}(D)}f\,d\mu \in D$$
by convexity of $D$. This implies that for every closed disc in $\CC$ disjoint from $F$ we have $\mu\left(f^{-1}(D)\right) = 0$. Since $\CC\setminus F$ can be covered by such discs, we derive that $\mu\left(X\setminus f^{-1}(F)\right)=0$.
\end{proof}

\begin{proof}[Proof of the theorem]
Consider Radon-Nikodym derivative $f:X\ra \CC$ of $\mu$ with respect to $|\mu|$. It exists according to Theorem \ref{theorem:radonnikodymmain}. Since
$$\frac{1}{\mu(A)}\bigg|\int_Af\,d|\mu|\bigg| \leq \frac{1}{\mu(A)}\int_A|f|\,d|\mu| = \frac{|\mu|(A)}{\mu(A)}\leq 1$$
for every $A\in \Sigma$ such that $A\in \Sigma$, we derive by Lemma \ref{lemma:convexvaluesofintegral} that $f(x)\in D$ almost everywhere with respect to measure $|\mu|$, where $D$ is a closed unit disc in $\CC$. Changing values of $f$ on a set of measure $|\mu|$ equal to zero, we may assume that $f(x)\in D$ for every $x$ in $X$.\\
Suppose next that  $0<\alpha < 1$ and denote $A_{\alpha} = f^{-1}\left(\{z\in \CC\mid|f(z)|\leq \alpha \}\right)$. Let
$$A_{\alpha} = \bigcup_{n\in \NN}A_n$$
be a decomposition on disjoint subsets in $\Sigma$. Then
$$\sum_{n\in \NN}|\mu(A_n)| = \sum_{n\in \NN}\bigg|\int_{A_n}f\,d|\mu|\bigg|\leq  \sum_{n\in \NN}\int_{A_n}|f|\,d|\mu| \leq \alpha \cdot \sum_{n\in \NN} |\mu|(A_n) = \alpha\cdot |\mu|(A_{\alpha})$$
Hence
$$|\mu|(A_{\alpha}) \leq \alpha\cdot |\mu|(A_{\alpha})$$
Therefore, $|\mu|(A_{\alpha})=0$. Since $\alpha$ is arbitrary number in $(0,1)$, we deduce that
$$|\mu|\bigg(\big\{z\in \CC\,\big|\,|f(z)| < 1\big\}\bigg) = 0$$
Thus changing values of $f$ on a set of measure $|\mu|$ equal to zero, we may assume that $|f(x)|=1$ for every $x$ in $X$.
\end{proof}

\begin{corollary}\label{corollary:totalvariationofcontinuousmeasure}
Let $\mu$ be a measure on $(X,\Sigma)$ and $f:X\ra \CC$ be a $\mu$-integrable function. Define
$$\nu(A) = \int_Af\,d\mu$$
for every $A\in \CC$. Then $\nu$ is a complex measure on $(X,\Sigma)$ and
$$|\nu|(A) = \int_A|f|\,d\mu$$
for every $A\in \Sigma$.
\end{corollary}
\begin{proof}
Clearly $\nu(A)\in \CC$ for every $A\in \Sigma$. Suppose that $A = \bigcup_{n\in \NN}A_n$ for $A\in \Sigma$ and $A_n\in \Sigma$ for every $n\in \NN$. Assume also that $\{A_n\}_{n\in \NN}$ are pairwise disjoint. Then by dominated convergence theorem
$$\nu(A) = \int_A f\,d\mu = \int_X\chi_A\cdot f\,d\mu = \int_X\left(\sum_{n\in \NN}\chi_{A_n}\cdot f\right)\,d\mu = \sum_{n\in \NN}\int_X\chi_{A_n}\cdot f\,d\mu = \sum_{n\in \NN}\int_{A_n}f\,d\mu = \sum_{n\in \NN}\nu(A_n)$$
Moreover, we have $\nu(\emptyset) = 0$. Thus $\nu$ is a complex measure on $(X,\Sigma)$. Since $f$ is $\mu$-integrable, there exists a $\sigma$-finite subset $\Omega\in \Sigma$ such that $|f(x)|= 0$ for $x\not \in \Omega$. We define $\tilde{\mu}(A) = \mu(A\cap \Omega)$ for every $A\in \Sigma$. Clearly
$$\nu(A) = \int_Af\,d\mu = \int_Af\,d\tilde{\mu}$$
for every $A\in \Sigma$. Hence we have $|\nu|\ll\tilde{\mu}$ by definition of $\nu$ and $|\nu|$. Note that $\tilde{\mu}$ is a $\sigma$-finite measure. By Theorem \ref{theorem:radonnikodymmain} there exists a measurable function $g:X\ra \CC$ equal to zero outside $\Omega$ such that
$$|\nu|(A) = \int_Ag\,d\tilde{\mu} = \int_Ag\,d\mu$$
for every $A\in \Sigma$. We may assume that $g$ takes only nonnegative values. By Theorem \ref{theorem:structureofcomplexmeasures} there exists a function $h:X\ra \CC$ such that
$$\nu(A) = \int_Ah\,d|\nu|$$
for every $A\in \Sigma$ and $|h(x)|=1$ for all $x$ in $X$. By Proposition \ref{proposition:derivativeandintegration} we deduce that
$$\int_Af\,d\mu = \nu(A) = \int_Ah\,d|\nu| = \int_Ah\cdot g\,d\mu$$
for every $A\in \Sigma$. Therefore, $f = h\cdot g$ almost everywhere with respect to $\mu$. Thus
$$g(x) = |h(x)|\cdot g(x) = |f(x)|$$
almost everywhere with respect to $\mu$.
\end{proof}

\section{Banach spaces of complex and real measures}

\begin{proposition}\label{proposition:variationismeasure}
Let $\mu$ be a complex measure on $(X,\Sigma)$. For every $A\in \Sigma$ we define
$$|\mu|(A) = \sup \bigg\{\sum_{n\in \NN}|\mu(A_n)|\,\bigg|\,A = \bigcup_{n\in \NN}A_n\mbox{ is a partition of }A\mbox{ onto subsets in }\Sigma\bigg\}$$
Then $|\mu|$ is a finite measure on $(X,\Sigma)$.
\end{proposition}
\begin{proof}
Let $\mu = \mu^r + i\cdot \mu^i$ be decomposition onto real and imaginary part. Then $\mu^r,\mu^i$ are finite, signed measures. By Theorem \ref{theorem:jordansdecomposition} we derive that there exist decompositions $\mu^r = \mu^r_+ - \mu^r_-$, $\mu^i = \mu^i_+ - \mu^i_-$ such that $\mu^r_+,\mu^r_-,\mu^i_+,\mu^i_-$ are finite measures and $\mu^r_+\perp \mu^r_-,\mu^i_+\perp \mu^i_-$. Then for every partition
$$A = \bigcup_{n\in \NN}A_n$$
of $A \in \Sigma$ onto sets in $\Sigma$ we have
$$\sum_{n\in \NN}|\mu(A_n)| = \sum_{n\in \NN}\sqrt{\left(\mu^r(A_n)\right)^2 + \left(\mu^i(A_n)\right)^2} \leq \sum_{n\in \NN}\left(|\mu^r(A_n)| + |\mu^i(A_n)|\right) \leq$$
$$\leq \sum_{n\in \NN}\left(\mu^r_+(A_n)+\mu^r_-(A_n) + \mu^i_+(A_n) + \mu^i_-(A_n)\right) =\mu^r_+(A)+\mu^r_-(A) + \mu^i_+(A) + \mu^i_-(A)$$
Left hand side of the inequality does not depend on the partition and hence
$$|\mu|(A) \leq \mu^r_+(A)+\mu^r_-(A) + \mu^i_+(A) + \mu^i_-(A)$$
This implies that $|\mu|(A)\in \RR$ for every $A\in \Sigma$. Note also that $|\mu|(\emptyset)=0$. Suppose now that $A\in \Sigma$ and we have partitions
$$A = \bigcup_{n\in \NN}A_{n} = \bigcup_{n\in \NN}C_n,\,A_n = \bigcup_{m\in \NN}A_{n,m}\mbox{ for every }n\in \NN$$
onto subsets in $\Sigma$. Then
$$\sum_{n\in \NN}|\mu(C_n)| \leq \sum_{n\in \NN}\sum_{m\in \NN}|\mu(A_n\cap C_m)|\leq \sum_{n\in \NN}|\mu|(A_n)$$
and
$$\sum_{n\in \NN}\left(\sum_{m\in \NN}|\mu(A_{n,m})|\right)\leq |\mu|(A)$$
These inequalities imply that
$$|\mu|(A) \leq \sum_{n\in \NN}|\mu|(A_n)\leq |\mu|(A)$$
Therefore, $|\mu|$ is a finite measure.
\end{proof}

\begin{definition}
Let $\mu$ be a complex measure on $(X,\Sigma)$. Then we define
$$||\mu|| = |\mu|(X)$$
and call it \textit{the total variation of $\mu$}.
\end{definition}

\begin{theorem}\label{theorem:Banachspaceofmeasures}
Let $(X,\Sigma)$ be a measurable space and $\cM(X,\Sigma)$ be a space of all complex measures on $(X,\Sigma)$. Then the following assertions hold.
\begin{enumerate}[label=\emph{\textbf{(\arabic*)}}, leftmargin=1.5em]
\item $\cM(X,\Sigma)$ is a $\CC$-linear space.
\item The mapping
$$\cM(X,\Sigma)\ni \mu \mapsto ||\mu||\in [0,+\infty)$$
is a norm on that space.
\item Suppose that $\{\mu_n\}_{n\in \NN}$ is a sequence of complex measures on $(X,\Sigma)$ that is a Cauchy sequence with respect to total variation. Then there exists a complex measure $\mu$ such that
$$\lim_{n\ra +\infty}\mu_n = \mu$$
Moreover, for every $A\in \Sigma$ we have
$$\lim_{n\ra +\infty}\mu_n(A) = \mu(A)$$
\end{enumerate}
\end{theorem}
\begin{proof}
We left \textbf{(1)} and \textbf{(2)} for the reader as an exercise. Fix $A\in \Sigma$. Then
$$|\mu_n(A) - \mu_m(A)| \leq |\mu_n - \mu_m|(A) \leq ||\mu_n - \mu_m||$$
for every $n,m\in \NN$. Since $\{\mu_n\}_{n\in \NN}$ is a Cauchy sequence with respect to total variation, we derive that there exists the limit of $\{\mu_n(A)\}_{n\in \NN}$. We denote
$$\mu(A) = \lim_{n\ra +\infty}\mu_n(A)$$
Suppose that
$$A = \bigcup_{k\in \NN}A_k$$
for $A\in \Sigma$ and $A_k\in \Sigma$ for $k\in \NN$. Assume that sets $\{A_k\}_{k \in \NN}$ are disjoint. Pick $N\in \NN$. Then
$$\sum_{k=0}^N|\mu_n(A_k) - \mu(A_k)| = \lim_{m\ra +\infty}\sum_{k=0}^N|\mu_n(A_k) - \mu_m(A_k)| \leq$$
$$\leq \limsup_{m\ra +\infty}\sum_{k\in \NN}|\mu_n(A_k) - \mu_m(A_k)|\leq \limsup_{m\ra +\infty}|\mu_n - \mu_m|(A) = \limsup_{m\ra +\infty}||\mu_n - \mu_m||$$
This implies that
$$\sum_{k\in \NN}|\mu_n(A_k) - \mu(A_k)| \leq  \limsup_{m\ra +\infty}||\mu_n - \mu_m||$$
regardless of set $A$ and partition $\{A_k\}_{k\in \NN}$. Thus we deduce that there exists a sequence $\{a_n\}_{n\in \NN}$ of real numbers, convergent to zero such that
$$\sum_{k\in \NN}|\mu_n(A_k) - \mu(A_k)| \leq a_n$$
for every $n\in \NN$, $A\in \Sigma$ and partition $\{A_k\}_{k\in \NN}$ as above. Therefore, for fixed $N\in \NN$ we have
$$\big|\mu(A) - \sum_{k=0}^N\mu(A_k)\big| \leq |\mu(A) - \mu_n(A)| + \big|\mu_n(A) - \sum_{k=0}^N\mu_n(A_k)\big| + \sum_{k=0}^N |\mu_n(A_k) - \mu(A_k)\big| \leq $$
$$\leq |\mu(A) - \mu_n(A)| + \big|\mu_n(A) - \sum_{k=0}^N\mu_n(A_k)\big| + \sum_{k\in \NN} |\mu_n(A_k) - \mu(A_k)\big| \leq  2a_n +  \big|\mu_n(A) - \sum_{k=0}^N\mu_n(A_k)\big|$$
Hence we derive that 
$$\mu(A) = \sum_{k\in \NN}\mu(A_k)$$
thus $\mu$ is a complex measure and according to
$$\sum_{k\in \NN}|\mu_n(A_k) - \mu(A_k)| \leq a_n$$
for every $n\in \NN$ we deduce that
$$\lim_{n\ra +\infty}|\mu_n - \mu|(A) = 0$$
for every $A\in \Sigma$. Hence also $\lim_{n\ra +\infty}||\mu_n-\mu|| = 0$. This finishes the proof of \textbf{(3)}.
\end{proof}

\begin{corollary}
Let $(X,\Sigma)$ be a measurable space and $\mu$ be a measure on $\Sigma$. Then there exists an isometrical embedding
$$L^1(X,\mu)\ni f\mapsto \left(\Sigma \ni A \mapsto \int_Af\,d\mu\in \CC\right)\in \cM(X,\Sigma)$$
of Banach spaces. If in addition $\mu$ is $\sigma$-finite, then it is surjective map onto the subspace of $\cM(X,\Sigma)$ consisting of complex measures which are absolutely continous with respect to $\mu$.
\end{corollary}
\begin{proof}
The first assertion follows from Corollary \ref{corollary:totalvariationofcontinuousmeasure} and Theorem \ref{theorem:Banachspaceofmeasures}. The second is a recapitulation of Theorem \ref{theorem:radonnikodymmain}.
\end{proof}


\small
\bibliographystyle{apalike}
\bibliography{../zzz}

\end{document}