\input ../pree

\begin{document}

\title{Projective bundles and projective schemes}
\date{}
\maketitle

\section{Actions of multiplicative group on relatively affine schemes}
\noindent
We start by introducing basic object of our study and its relative version.

\begin{example}[Multiplicative group]\label{example:multiplication_group}
We denote by $\mathbb{G}_{m}$ a presheaf on $\Sch$ defined by
$$\mathbb{G}_{m}\big(X\big) = \Gamma(X,\cO_X)^*$$
if $f:X_1\ra X_2$ is a morphism of schemes, then 
$$\mathbb{G}_{m}(f) = \Gamma(X_2,\cO_{X_2})^*\ni r \mapsto f^{\#}(r)\in \Gamma(X_1,\cO_{X_1})^*$$
Clearly $\mathbb{G}_{m}$ is a commutative group presheaf with multiplication determined by the usual multiplication of invertible regular functions. Note that for every scheme $X$ we have an identification
$$\mathbb{G}_{m}\big(X\big) = \Gamma(X,\cO_X)^* = \Mor\big(X, \Spec \ZZ[t,t^{-1}]\big)$$
Hence $\mathbb{G}_{m}$ is a group scheme over $\ZZ$. We call it \textit{the multiplicative group of $\ZZ$}. Note that the multiplication $\mu:\mathbb{G}_{m}\times_{\Spec \ZZ}\mathbb{G}_{m}\ra \mathbb{G}_{m}$ and the unit $e:\Spec \ZZ\ra \mathbb{G}_{m}$ induce bialgebra structure on $\ZZ$-algebra $\ZZ[t,t^{-1}]$. Hence there exist comultiplication and counit 
$$\Delta:\ZZ[t,t^{-1}]\ra \ZZ[t,t^{-1}]\otimes_{\ZZ}\ZZ[t,t^{-1}],\,\xi:\ZZ[t,t^{-1}]\ra \ZZ$$
which are morphisms of rings. Concretely 
$$\Delta(f(t)) = f(t)\otimes f(t),\,\xi(f(t)) = f(1)$$
where $f(t)\in \ZZ[t,t^{-1}]$ is Laurent polynomial.
\end{example}

\begin{example}[Relative multiplicative group]
Let $S$ be a scheme. We denote group $S$-scheme $\mathbb{G}_{m}\times_{\Spec \ZZ}S$ by $\mathbb{G}_{m,S}$ and call it \textit{the multiplicative group of $S$}. Note that $\mathbb{G}_{m,S}$ is an affine group scheme over $S$. Its quasi-coherent algebra $\cO_S[t,t^{-1}]$ on $S$ is a quasi-coherent bialgebra on $S$ with the comultiplication $\Delta_S$ and counit $\xi_S$ given by formulas
$$\Delta_S(f(t)) = f(t)\otimes f(t),\,\xi_S(f(t)) = f(1)$$
where $f(t) \in \cO_S(V)[t,t^{-1}]$ is a Laurent polynomial and $V$ is an arbitrary open affine subset of $S$.
\end{example}

\begin{definition}
Let $X$ be an $S$-scheme with the structural morphism $p:X\ra S$. Suppose that there is an action $a:\mathbb{G}_{m}\times_{\Spec \ZZ}X\ra X$ of $\mathbb{G}_{m}$ on $X$ such that the triangle
\begin{center}
\begin{tikzpicture}
[description/.style={fill=white,inner sep=2pt}]
\matrix (m) [matrix of math nodes, row sep=3em, column sep=3em,text height=1.5ex, text depth=0.25ex] 
{ \mathbb{G}_{m}\times_{\Spec \ZZ}X &     &  X \\
                                    &  S  &     \\} ;
\path[->,line width=1.0pt,font=\scriptsize]
(m-1-1) edge node[above] {$ a $} (m-1-3)
(m-1-1) edge node[left = 7pt, below=0pt] {$ p\cdot \mathrm{pr}_X $} (m-2-2)
(m-1-3) edge node[right = 5pt, below = 0pt] {$ p $} (m-2-2) ;
\end{tikzpicture}
\end{center}
is commutative. Then \textit{$\mathbb{G}_{m}$ acts on $X$ over $S$}.
\end{definition}
\noindent
Let $X$ be an $S$-scheme with affine structural morphism $p:X\ra S$. Suppose that 
$$a:\mathbb{G}_{m}\times_{\Spec \ZZ}X\ra X$$
is an action of $\mathbb{G}_m$ over $S$. Since $\mathbb{G}_{m,S} = \mathbb{G}_m\times_{\Spec \ZZ}S$ we can view $a$ as an action of $\mathbb{G}_{m,S}$ on $X$ over $S$. Denote $p_*\cO_X$ by $\cA$. This is a quasi-coherent algebra on $S$. Note that $a$ corresponds to a unique morphism 
$$c_a:\cA \ra \cO_S[t,t^{-1}]\otimes_{\cO_S}\cA$$
of quasi-coherent algebras on $S$. The fact that $a$ is an action is equivalent with the fact that $c_a$ is a coaction of the sheaf of bialgebras $\cO_S[t,t^{-1}]$ on $\cA$. Fix $n\in \ZZ$. The map
$$V\mapsto \big\{f\in \cA(V)\,\big|\,c_a(f) = t^n\otimes f\big\}$$
defined for all open affine subsets of $S$ gives rise to a subsheaf of $\cA$. We denote this subsheaf by $\cA_n$. Observe that $\cA_n$ is a quasi-coherent subsheaf of $\cA$. We have $\cA_n\cap \cA_m = \{0\}$ for $n\neq m$ and $\cA_n\cdot \cA_m \subseteq \cA_{n+m}$ for $n,m\in \ZZ$. Moreover, the image of the structural morphism $\cO_S\ra \cA$ is contained in $\cA_0$. Thus 
$$\bigoplus_{n\in \NN}\cA_n \subseteq \cA$$
is a $\ZZ$-graded quasi-coherent subalgebra. Fix an open affine subset $V$ of $S$ and an element $f\in \cA(V)$. Write
$$c_a(f) = \sum_{n\in \ZZ}t^n\otimes f_n$$
where almost all $f_n$ are zero for $n\in \NN$. We claim that $f_n$ is a section of $\cA_n$ over $V$. For this observe that
$$\sum_{n\in \ZZ}t^n\otimes c_a(f_n) = \Gamma\left(V,1_{\cO_S[t,t^{-1}]}\otimes_{\cO_S}c_a\right)(c_a(f)) = \Gamma\left(V,\Delta_S\otimes_{\cO_S}1_{\cA}\right)(c_a(f)) = \sum_{n\in \ZZ}t^n\otimes t^n\otimes f_n$$
Hence $c_a(f_n) = t^n\otimes f_n$ and this proves the claim. Moreover, note that
$$1\otimes f = \Gamma\left(V,\xi_S\otimes_{\cO_S}1_{\cA}\right)(c_a(f)) = \sum_{n\in \ZZ}1\otimes f_n = 1\otimes \left(\sum_{n\in \ZZ}f_n\right)$$
Hence $f$ is a sum of $f_n$ for $n\in \ZZ$. This implies that
$$\cA = \bigoplus_{n\in \ZZ}\cA_n$$
Thus $c_a$ makes $\cA$ into a $\ZZ$-graded quasi-coherent algebra on $S$. On the other hand it is a routine verification to check that $\ZZ$-grading of this sort induces a coaction of $\cO_S[t,t^{-1}]$ on $\cA$. Thus we proved the following result.

\begin{theorem}\label{theorem:description_of_multiplicative_group_actions}
Let $p:X\ra S$ be an affine $S$-scheme. Then the following sets can be identified.
\begin{enumerate}[label=\emph{\textbf{(\arabic*)}}, leftmargin=3.0em]
\item The set of actions of $\mathbb{G}_m$ on $X$ over $S$.
\item The set of actions of $\mathbb{G}_{m,S}$ on $X$ over $S$.
\item The set of coactions 
$$c:p_*\cO_X\ra \cO_{S}[t,t^{-1}]\otimes_{\cO_S}p_*\cO_X$$
which are morphisms of quasi-coherent algebras on $S$.
\item The set of $\ZZ$-gradations on $p_*\cO_X$ making it into a quasi-coherent $\ZZ$-graded algebra on $S$ such that the image of the structural morphism $\cO_S\ra p_*\cO_X$ lies in zeroth graded component.
\end{enumerate}
\end{theorem}
\noindent
Identifications described above are functorial as it is indicated in the theorem below.

\begin{theorem}\label{theorem:multiplicative_group_equivariant_morphisms}
Let $p_1:X_1\ra S$ and $p_2:X_2\ra S$ be affine $S$-schemes equipped with actions of $\mathbb{G}_m$ over $S$. Then the following sets can be identified.
\begin{enumerate}[label=\emph{\textbf{(\arabic*)}}, leftmargin=3.0em]
\item The set of $\mathbb{G}_m$-equivariant morphism $f:X_1\ra X_2$ over $S$.
\item The set of $\mathbb{G}_{m,S}$-equivariant morphism $f:X_1\ra X_2$ over $S$.
\item Morphism ${p_1}_*\cO_{X_1}\ra {p_2}_*\cO_{X_2}$ of quasi-coherent algebras on $S$ which preserve coactions determined by actions of $\mathbb{G}_m$.
\item Morphism ${p_1}_*\cO_{X_1}\ra {p_2}_*\cO_{X_2}$ of quasi-coherent algebras on $S$ which preserve $\ZZ$-gradations determined by actions of $\mathbb{G}_m$.
\end{enumerate}
\end{theorem}
\begin{proof}
This follows from similar considerations as these preceding Theorem \ref{theorem:description_of_multiplicative_group_actions}.
\end{proof}

\section{Categorical quotients of multiplicative group}
\noindent
Using results of the previous section we can describe certain quotients of $\mathbb{G}_m$-actions. First we need to introduce appropriate notion of the quotient.

\begin{definition}
Let $X$ be a scheme equipped with an action of $a:\mathbb{G}_m\times_{\Spec \ZZ}X\ra X$. Suppose that $q:X\ra Y$ is a morphism of schemes such that 
\begin{center}
\begin{tikzpicture}
[description/.style={fill=white,inner sep=2pt}]
\matrix (m) [matrix of math nodes, row sep=3em, column sep=3em,text height=1.5ex, text depth=0.25ex] 
{\mathbb{G}_m\times_{\Spec \ZZ}X &  X & Y\\} ;
\path[->,line width=1.0pt,font=\scriptsize]
(m-1-1) edge[transform canvas={yshift=0.5ex}] node[above] {$ a  $} (m-1-2)
(m-1-1) edge[transform canvas={yshift=-0.5ex}] node[below] {$ \mathrm{pr}_X $} (m-1-2)
(m-1-2) edge node[above] {$ q $} (m-1-3);
\end{tikzpicture}
\end{center}
is a cokernel in the category of schemes. Then $q$ is \textit{a categorical quotient of $X$ by $\mathbb{G}_m$}.
\end{definition}
\noindent
Let $X$ be a scheme equipped with an action of $a:\mathbb{G}_m\times_{\Spec \ZZ}X\ra X$ and let $q:X\ra Y$ be a morphism of schemes such that $q\cdot \mathrm{pr}_X = q\cdot a$. For a morphism $g:Y'\ra Y$ of schemes consider the cartesian square
\begin{center}
\begin{tikzpicture}
[description/.style={fill=white,inner sep=2pt}]
\matrix (m) [matrix of math nodes, row sep=2em, column sep=2em,text height=1.5ex, text depth=0.25ex] 
{ X' &    X                           \\
    Y' &   Y                 \\} ;
\path[->,line width=1.0pt,font=\scriptsize]  
(m-1-1) edge node[auto] {$ g'$} (m-1-2)
(m-2-1) edge node[below] {$ g$} (m-2-2)
(m-1-1) edge node[left] {$q' $} (m-2-1)
(m-1-2) edge node[auto] {$ q$} (m-2-2);
\end{tikzpicture}
\end{center} 
Then there exists a unique action $a':\mathbb{G}_m \times_{\Spec \ZZ}X' \ra X'$ of $\mathbb{G}_m$ on $X'$ such that the square above consists of $\mathbb{G}_m$-equivariant morphism (we consider $Y,Y'$ as $\mathbb{G}_m$-schemes equipped with trivial $\mathbb{G}_m$-actions). Keeping this in mind we have the following.

\begin{definition}
Let $X$ be a scheme equipped with an action of $a:\mathbb{G}_m\times_{\Spec \ZZ}X\ra X$. Then a morphism $q:X\ra Y$ of schemes is \textit{a universal categorical quotient of $X$ by $\mathbb{G}_m$} if for every morphism $g:Y'\ra Y$ of schemes a base change $q':X'\ra Y'$ of $q$ along $g$ is a categorical quotient of $X'$ by $\mathbb{G}_m$. 
\end{definition}

\begin{theorem}\label{theorem:universal_categorical_quotients_for_multiplicative_group_exists_for_relatively_affine_schemes}
Let $p:X\ra S$ be an affine $S$-scheme equipped with an action of $\mathbb{G}_m$ on $X$ over $S$ and let 
$$p_*\cO_X = \bigoplus_{n\in \ZZ}\left(p_*\cO_X\right)_n$$
be $\ZZ$-gradation determined by the action. Suppose that $Y = \Spec_S\,\left(p_*\cO_X\right)_0$ and consider the canonical morphism $q:X \ra Y$ given by the inclusion $\left(p_*\cO_X\right)_0 \hookrightarrow p_*\cO_X$ of quasi-coherent algebras on $S$. Then $q$ is a universal categorical quotient of $X$ by $\mathbb{G}_m$.
\end{theorem}
\noindent
We first prove the result in affine case.

\begin{lemma}
Let $A$ be a $k$-algebra and suppose that $\mathbb{G}_m$ acts on $\Spec A$. Let 
$$A = \bigoplus_{n\in \ZZ}A_n$$
be the $\ZZ$-gradation determined by the action of $\mathbb{G}_m$. Then for every $A_0$-algebra $B$ we have a kernel diagram
\begin{center}
\begin{tikzpicture}
[description/.style={fill=white,inner sep=2pt}]
\matrix (m) [matrix of math nodes, row sep=3em, column sep=3.3em,text height=1.5ex, text depth=0.25ex] 
{B & A\otimes_{A_0}B  & & \ZZ[t,t^{-1}]\otimes_{\ZZ}A\otimes_{A_0}B\\} ;
\path[->,line width=1.0pt,font=\scriptsize]
(m-1-2) edge[transform canvas={yshift=0.5ex}] node[above] {$ a_n \otimes b\mapsto t^n\otimes a_n \otimes b  $} (m-1-4)
(m-1-2) edge[transform canvas={yshift=-0.5ex}] node[below] {$ a_n \otimes b\mapsto 1\otimes a_n \otimes b $} (m-1-4)
(m-1-1) edge node[above] {$ b\mapsto 1\otimes b $} (m-1-2);
\end{tikzpicture}
\end{center}
where $a_n$ is an element of $A_n$ for $n\in \ZZ$.
\end{lemma}
\begin{proof}[Proof of the lemma]
\end{proof}































































\small
\bibliographystyle{apalike}
\bibliography{../zzz}

\end{document}