\input ../pree.tex

\begin{document}

\title{Geometry of $k$-functors}
\date{}
\maketitle

\section{Introduction}
\noindent
These notes are devoted to study $k$-functors and presheaves on $\Sch_k$. We introduce elements of the functorial language that will be used significantly in our other notes on algebraic groups. Mostly we follow first part of \cite{demazure1970groupes}, but in order to make our presentation self-contained we introduce all notions and add some results from other sources. The reader may find this  tedious and formal. Since it is somewhat obvious that the category of $k$-functors i.e. the category of copresheaves on $\Alg_k$ and the category of presheaves on $\Sch_k$ have equivalent subcategories of Zariski sheaves, she may even find a bit irritating the fact that we express each notion in these two linguistically different, but geometrically equivalent, settings. Nevertheless this was the only route that we had found to present this material in a clear and complete way. On the other hand the fact that subcategories of sheaves with respect to Zariski topology in these two categories are equivalent is not entirely obvious result in the theory of sheaves.\\
Throughout these notes $k$ is a fixed commutative ring and $\Alg_k$ denote the category of commutative $k$-algebras. If $A, B$ are $k$-algebras, then we denote by $\Mor_k(A, B)$ the set of all morphisms $A\ra B$ of $k$-algebras. Similarly if $X, Y$ are $k$-schemes (i.e. schemes together with morphism to $\Spec k$), then we denote by $\Mor_k(X,Y)$ the set of all morphisms $X\ra Y$ of $k$-schemes (morphisms of schemes that preserve structure morphisms to $\Spec k$).

\section{\textit{k}-functors}

\begin{definition}
The category $\Fun(\Alg_k,\Set)$ of copresheaves on $\Alg_k$ is called \textit{the category of $k$-functors}.
\end{definition}
\noindent
Since $\Spec:\Alg_k^{\mathrm{op}} \ra \Aff_k$ is an equivalence of categories, the category of $k$-functors is equivalent with the category of presheaves $\widehat{\Aff_k}$.\\
If $\fX$ and $\fY$ are $k$-functors, then we denote by $\Mor_k(\fX,\fY)$ the class of morphisms $\fX\ra \fY$ of $k$-functors. If $\sigma:\fX\ra \fY$ is a morphism of $k$-functors, then for every $k$-algebra $A$ we denote by $\sigma^A:\fX(A)\ra \fY(A)$ the corresponding component of $\sigma$.\\
Let $\fX$ and $\fY$ be $A$-functors for some $k$-algebra $A$. Then we denote by $\Mor_A\left(\fX,\fY\right)$ the class of morphisms of $A$-functors $\fX\ra \fY$. For every $A$-algebra $B$ and a morphism $\sigma:\fX\ra \fY$ of $A$-functors we denote by $\fX_{B}$, $\fY_{B}$, $\sigma_{B}$ the restrictions $\fX_{\mid \Alg_B}$, $\fY_{\mid \Alg_B}$, $\sigma_{\mid \Alg_B}$ of these entities to the category of $B$-algebras. We note the following result.

\begin{fact}\label{fact:restriction_works_as_expected}
Let $\fX$ and $\fY$ be $k$-functors. Assume that $A$ is a $k$-algebra, $B$ is an $A$-algebra, $C$ is an $B$-algebra. Then the composition of maps of classes
\begin{center}
\begin{tikzpicture}
[description/.style={fill=white,inner sep=2pt}]
\matrix (m) [matrix of math nodes, row sep=3em, column sep=3em,text height=1.5ex, text depth=0.25ex] 
{ \Mor_A\left(\fX_A,\fY_A\right) &  \Mor_B\left(\fX_B,\fY_B\right) & \Mor_C\left(\fX_C,\fY_C\right)\\} ;
\path[->,line width=1.0pt,font=\scriptsize]  
(m-1-1) edge node[above] {$\sigma\mapsto \sigma_{B} $} (m-1-2)
(m-1-2) edge node[above] {$\sigma\mapsto \sigma_{C} $} (m-1-3);
\end{tikzpicture}
\end{center}
equals
\begin{center}
\begin{tikzpicture}
v[description/.style={fill=white,inner sep=2pt}]
\matrix (m) [matrix of math nodes, row sep=3em, column sep=3em,text height=1.5ex, text depth=0.25ex] 
{ \Mor_A\left(\fX_A,\fY_A\right) &  \Mor_C\left(\fX_C,\fY_C\right)\\} ;
\path[->,line width=1.0pt,font=\scriptsize]  
(m-1-1) edge node[above] {$\sigma\mapsto \sigma_C $} (m-1-2);
\end{tikzpicture}
\end{center}
\end{fact}

\begin{definition}
Let $\fX$ and $\fY$ be $k$-functors and suppose that for every $k$-algebra $A$ the class $\Mor_A\left(\fX_A,\fY_A\right)$ is a set. We define
$$\iMor_k(\fX,\fY)(A)=\Mor_A\left(\fX_A,\fY_A\right)$$
for every $k$-algebra $A$. This is a $k$-functor. Indeed, for every $k$-algebra $A$ and $A$-algebra $B$ we can compose a morphism $\sigma:\fX_A\ra \fY_A$ of $k$-functors with the forgetful functor $\Alg_B \ra \Alg_A$. This induces a map 
$$\iMor_{k}(\fX,\fY)(A)\ni \sigma \mapsto \sigma_{B}\in \iMor_{k}(\fX,\fY)(B)$$
and according to Fact \ref{fact:restriction_works_as_expected} these maps make $\iMor_{k}(\fX,\fY)$ a $k$-functor. The $k$-functor $\iMor_{\cC}(\fX,\fY)$ is called \textit{a hom $k$-functor of $\fX$ and $\fY$}.
\end{definition}

\begin{definition}
Let $\fX$ be a $k$-functor and let $A$ be a $k$-algebra. Then elements of $\fX(A)$ are called \textit{$A$-points of $\fX$}.
\end{definition}
\noindent
We denote by $\bd{1}$ a $k$-functor that assigns to every $k$-algebra a set with one element. Then for every $k$-algebra $A$ the restriction $\bd{1}_A$ is a terminal object in the category of $A$-functors.\\
Let $\fX$ be a $k$-functor. Suppose that $A$ is a $k$-algebra and $x\in \fX(A)$. Then $x$ determines a morphism $\bd{1}_{A}\ra \fX_A$ that for every $A$-algebra $B$ with structural morphism $f:A\ra B$ sends the unique element of $\bd{1}_{A}(B)$ to $\fX(f)(x)\in \fX_A(B)$. This gives rise to a bijection
$$\fX(A)\simeq \Mor_{A}\left(\bd{1}_{A},\fX_A\right)$$
natural in $k$-algebra $A$.

\begin{definition}
Let $\fZ,\fX,\fY$ be $k$-functors and let $\sigma:\fZ\times \fX\ra \fY$ be a morphism of $k$-functors. Fix $z\in \fZ(A)$ for some $k$-algebra $A$. We denote by $i_z:\bd{1}_A\ra \fZ_A$ the morphism of $A$-functors corresponding to $z$. Since $\bd{1}_A$ is a terminal $A$-functor, $\sigma_A\cdot \left(i_z\times 1_{\fX_A}\right)$ is isomorphic to a morphism $\sigma_z:\fX_A\ra \fY_A$ of $A$-functors. We call $\sigma_z$ \textit{the slice of $\sigma$ along $z$}.
\end{definition}
\noindent
Consider now $k$-functors $\fX,\fY,\fZ$ and assume that the internal hom $\iMor_k\left(\fX,\fY\right)$ exists. Let $\sigma:\fZ\times \fX\ra \fY$ be a morphism. Then the family of maps
$$\fZ(A)\ni z\mapsto \sigma_z\in \iMor_k\left(\fX,\fY\right)(A)$$
give rise to a morphism $\tau:\fZ\ra \iMor_k\left(\fX,\fY\right)$ of $k$-functors. Indeed, for a morphism $f:A\ra B$ of $k$-algebras and $z\in \fZ(A)$ we have
$$\sigma_B\cdot \left(i_{\fZ(f)(z)}\times 1_{\fX_B}\right) = \left(\sigma_A\cdot \left(i_{z}\times 1_{\fX_A}\right)\right)_B$$
and hence $\sigma_{\fZ(f)(z)} = \left(\sigma_z\right)_B$. This gives rise to a map $\Phi$ of classes
$$\Mor_k\left(\fZ\times \fX,\fY\right)\ni \sigma \mapsto \tau \in \Mor_k\big(\fZ,\iMor_k(\fX,\fY)\big)$$
Consider next a morphism $\tau:\fZ\ra \iMor_k(\fX,\fY)$ of $k$-functors and define $\sigma:\fZ\times \fX\ra \fY$ by formula $\sigma^A(z,x) = \left(\tau^A(z)\right)^A(x)$ for every $k$-algebra $A$ and points $z\in \fZ(A),\,x\in \fX(A)$. Let $f:A\ra B$ be a morphism of $k$-algebras. Then
$$\sigma^B\left(\fZ(f)(z),\fX(f)(x)\right) = \left(\tau^B\left(\fZ(f)(z)\right)\right)^B\left(\fX(f)(x)\right) = \left(\left(\tau^A(z)\right)_B\right)^B\left(\fX(f)(x)\right) =$$
$$ = \left(\tau^A(z)\right)^B\left(\fX(f)(x)\right) = \fY(f)\left(\left(\tau^A(z)\right)^A(x)\right) = \fY(f)\left(\sigma^A(z,x)\right)$$
Thus $\sigma:\fZ\times \fX\ra \fY$ is a well defined morphism of $k$-functors. This gives rise to a map $\Psi$ of classes
$$\Mor_k\big(\fZ,\iMor_k(\fX,\fY)\big) \ni \tau \mapsto \sigma \in \Mor_k(\fZ\times \fX,\fY)$$

\begin{theorem}\label{theorem:category_of_k_functors_is_cartesian_closed}
Let $\fZ,\fX,\fY$ be $k$-functors and assume that the $k$-functor $\iMor_k(\fX,\fY)$ exists. Then maps $\Phi$ and $\Psi$ are mutually inverse bijections and hence they induce a bijection
$$\Mor_k\left(\fZ\times \fX,\fY\right) \simeq  \Mor_k\big(\fZ,\iMor_k(\fX,\fY)\big)$$
\end{theorem}
\begin{proof}
Pick a morphism $\tau:\fZ\ra \iMor_k(\fX,\fY)$ of $k$-functors. Let $A$ be a $k$-algebra and $z\in \fZ(A)$. Let us first prove that $\Psi(\tau)_z = \tau^A(z)$. Indeed, let $f:A\ra B$ be a morphism of $k$-algebras and $x$ be an element in $\fX(B)$. Then we have
$$\left(\Psi(\tau)_z\right)^B(x) = \Psi(\tau)^B\left(\fZ(f)(z),x\right) = \left(\tau^B\left(\fZ(f)(z)\right)\right)^B(x) = \left(\left(\tau^A(z)\right)_B\right)^B(x) = \left(\tau^A(z)\right)^B(x)$$
Hence $\Psi(\tau)_z = \tau^A(z)$ because $B$ and a $B$-point $x$ are arbitrary. Now we use this fact and obtain
$$\left(\Phi(\Psi(\tau))\right)^A(z) = \Psi(\tau)_z = \tau^A(z)$$
and hence $\Phi\cdot \Psi$ is the identity. On the other hand fix a morphism $\sigma:\fZ\times \fX \ra \fY$. Let $A$ be a $k$-algebra and let $z\in \fZ(A)$, $x\in \fX(A)$ be points. Then
$$\left(\Psi\left(\Phi(\sigma)\right)\right)^A(z,x) = \left(\Phi(\sigma)^A(z)\right)^A(x) = \sigma_z^A(x) = \sigma^A(z,x)$$
Thus $\Psi\cdot \Phi$ is the identity map. Therefore, $\Phi$ and $\Psi$ are mutually inverse bijections.
\end{proof}

\section{Zariski local $k$-functors and Zariski sheaves}
\noindent
In this part we use the notion of a Grothendieck topology on a category. For this notion we refer the reader to \cite{Sheaves}.

\begin{definition}
Let $\big\{f_i:X_i\ra X\big\}_{i\in I}$ be a family of morphisms of $k$-schemes. We say that $\{f_i\}_{i\in I}$ is \textit{a Zariski covering of $X$} if the following conditions are satisfied.
\begin{enumerate}[label=\textbf{(\arabic*)}, leftmargin=3.0em]
\item For every $i\in I$ morphism $f_i$ is an open immersion of schemes.
\item Morphism $\coprod_{i\in I}X_i\ra X$ induced by $\big\{f_i\big\}_{i\in I}$ is surjective.
\end{enumerate}
\end{definition}
\noindent
The collection of all Zariski coverings on $\Sch_k$ is a Grothendieck pretopology.

\begin{definition}
We call the Grothendieck topology generated by the pretopology consisting of Zariski coverings on $\Sch_k$ \textit{the Zariski topology on $\Sch_k$}. A presheaf on $\Sch_k$ that is a sheaf with respect to Zariski topology on $\Sch_k$ is called \textit{a Zariski sheaf}.
\end{definition}
\noindent
Let $\fX$ be a presheaf on the category of $k$-schemes. Recall that by {\cite[Theorem 3.5]{Sheaves}} $\fX$ is a Zariski sheaf if and only if for every $k$-scheme $X$ and for every Zariski covering $\big\{f_i:X_i\ra X\big\}$ of $X$ the diagram
\begin{center}
\begin{tikzpicture}
[description/.style={fill=white,inner sep=2pt}]
\matrix (m) [matrix of math nodes, row sep=3em, column sep=6em,text height=1.5ex, text depth=0.25ex] 
{\fX(X) &   \prod_{i\in I}\fX(X_i)&  \prod_{(i,j)\in I\times I} \fX(X_i\times_XX_j)  \\} ;
\path[->,line width=0.8pt,font=\scriptsize]
(m-1-1) edge node[above] {$ \langle \fX(f_i) \rangle_{i\in I} $} (m-1-2)
(m-1-2) edge[transform canvas={yshift=0.5ex}] node[above] {$ \langle \fX(f'_{ij}) \cdot pr_i\rangle_{(i,j)}$} (m-1-3)
(m-1-2) edge[transform canvas={yshift=-0.5ex}] node[below] {$ \langle \fX(f''_{ij}) \cdot pr_j\rangle_{(i,j)}$} (m-1-3);
\end{tikzpicture}
\end{center}
is a kernel of a pair of arrows, where for every $(i,j)\in I\times I$ morphisms $f'_{ij}$ and $f''_{ij}$ form a cartesian square
\begin{center}
\begin{tikzpicture}
[description/.style={fill=white,inner sep=2pt}]
\matrix (m) [matrix of math nodes, row sep=3em, column sep=3em,text height=1.5ex, text depth=0.25ex] 
{X_i\times_XX_j   &   X_j   \\
 X_i  & X   \\} ;
\path[->,line width=0.8pt,font=\scriptsize]
(m-1-1) edge node[above] {$ f''_{ji}$} (m-1-2)
(m-2-1) edge node[below] {$ f_i $} (m-2-2)
(m-1-1) edge node[left] {$ f'_{ij} $} (m-2-1)
(m-1-2) edge node[right] {$ f_j  $} (m-2-2);
\end{tikzpicture}
\end{center}
\noindent
Now we repeat this definitions for $k$-algebras and $k$-functors.

\begin{definition}
Let $\big\{f_i:A\ra A_i\big\}_{i\in I}$ be a family of morphisms of $k$-algebras. We say that $\{f_i\}_{i\in I}$ is \textit{a Zariski covering of $A$} if the following conditions are satisfied.
\begin{enumerate}[label=\textbf{(\arabic*)}, leftmargin=3.0em]
\item For every $i\in I$ morphism $\Spec f_i$ is an open immersion of schemes.
\item Morphism $\coprod_{i\in I}\Spec A_i\ra \Spec A$ induced by $\big\{\Spec f_i\big\}_{i\in I}$ is surjective.
\end{enumerate}
\end{definition}
\noindent
The collection of all Zariski coverings on $\Alg_k$ induces on its opposite category $\Aff_k$ of affine $k$-schemes a Grothendieck pretopology.

\begin{definition}
We call the Grothendieck topology generated by the pretopology consisting of Zariski coverings on $\Aff_k$ \textit{the Zariski topology on $\Aff_k$}. A $k$-functor that is a sheaf with respect to Zariski topology on $\Aff_k$ is called \textit{a Zariski local $k$-functor}.
\end{definition}
\noindent
Let $\fX$ be a $k$-functor. Again by {\cite[Theorem 3.5]{Sheaves}} $\fX$ is a Zariski local $k$-functor if and only if for every $k$-algebra $A$ and for every Zariski covering $\big\{f_i:A\ra A_i\big\}$ of $A$ the diagram
\begin{center}
\begin{tikzpicture}
[description/.style={fill=white,inner sep=2pt}]
\matrix (m) [matrix of math nodes, row sep=3em, column sep=6em,text height=1.5ex, text depth=0.25ex] 
{\fX(A) &   \prod_{i\in I}\fX(A_i)&  \prod_{(i,j)\in I\times I} \fX(A_i\otimes_AA_j)  \\} ;
\path[->,line width=0.8pt,font=\scriptsize]
(m-1-1) edge node[above] {$ \langle \fX(f_i) \rangle_{i\in I} $} (m-1-2)
(m-1-2) edge[transform canvas={yshift=0.5ex}] node[above] {$ \langle \fX(f'_{ij}) \cdot pr_i\rangle_{(i,j)}$} (m-1-3)
(m-1-2) edge[transform canvas={yshift=-0.5ex}] node[below] {$ \langle \fX(f''_{ij}) \cdot pr_j\rangle_{(i,j)}$} (m-1-3);
\end{tikzpicture}
\end{center}
is a kernel of a pair of arrows, where for every $(i,j)\in I\times I$ morphisms $f'_{ij}$ and $f''_{ij}$ form a cocartesian square
\begin{center}
\begin{tikzpicture}
[description/.style={fill=white,inner sep=2pt}]
\matrix (m) [matrix of math nodes, row sep=3em, column sep=3em,text height=1.5ex, text depth=0.25ex] 
{A &  A_j   \\
 A_i&  A_i\otimes_AA_j   \\} ;
\path[->,line width=0.8pt,font=\scriptsize]
(m-1-1) edge node[above] {$ f_j $} (m-1-2)
(m-2-1) edge node[below] {$ f'_{ij} $} (m-2-2)
(m-1-1) edge node[left] {$ f_i $} (m-2-1)
(m-1-2) edge node[right] {$ f'_{ji}  $} (m-2-2);
\end{tikzpicture}
\end{center}
\noindent
Now we state the main result of this section.

\begin{theorem}\label{theorem:sheaves_on_schemes_are_local_kfunctors}
Let
\begin{center}
\begin{tikzpicture}
[description/.style={fill=white,inner sep=2pt}]
\matrix (m) [matrix of math nodes, row sep=3em, column sep=3em,text height=1.5ex, text depth=0.25ex] 
{ \widehat{\Sch_k}  & \mbox{\emph{the category of $k$-functors}} \\};
\path[->,line width=1.0pt,font=\scriptsize]  
(m-1-1) edge node[auto] {$ $} (m-1-2);
\end{tikzpicture}
\end{center}
be the restriction of presheaves on $\Sch_k$ to copresheaves on $\Alg_k$ ($k$-functors) induced by the contravariant functor $\Spec:\Alg_k\ra \Sch_k$. Then it induces an equivalence of categories between Zariski sheaves on $\Sch_k$ and Zariski local $k$-functors.
\end{theorem}
\begin{proof}
Note that $\Aff_k$ with Zariski topology is a dense subsite ({\cite[definition 4.4]{Sheaves}}) of $\Sch_k$ with Zariski topology. Hence the result is a special case of a more general theorem {\cite[Theorem 4.6]{Sheaves}}. 
\end{proof}
The notion of creation of limits and colimits ({\cite[Definition on page 112]{Maclane}}) is essential to our discussion below. Recall that Yoneda embedding
\begin{center}
\begin{tikzpicture}
[description/.style={fill=white,inner sep=2pt}]
\matrix (m) [matrix of math nodes, row sep=3em, column sep=3em,text height=1.5ex, text depth=0.25ex] 
{ \Sch_k  &  \widehat{\Sch_k} \\};
\path[right hook->,line width=1.0pt,font=\scriptsize]  
(m-1-1) edge node[auto] {$ $} (m-1-2);
\end{tikzpicture}
\end{center}
is full and faithful. Moreover, it creates limits. {\cite[Proposition 8.8]{gortz2010algebraic}} states that every representable functor $\widehat{\Sch_k}$ is a Zariski sheaf. Let
\begin{center}
\begin{tikzpicture}
[description/.style={fill=white,inner sep=2pt}]
\matrix (m) [matrix of math nodes, row sep=3em, column sep=3em,text height=1.5ex, text depth=0.25ex] 
{ \Sch_k  &  \mbox{$k$-functors} \\};
\path[right hook->,line width=1.0pt,font=\scriptsize]  
(m-1-1) edge node[auto] {$\fP $} (m-1-2);
\end{tikzpicture}
\end{center}
be the functor defined by the composition of the Yoneda embedding and the restriction $\widehat{\Sch_k}\ra \widehat{\Aff_k}$. This functor is full, faithful and creates limits and its image consists of Zariski local $k$-functors. Thus Theorem \ref{theorem:sheaves_on_schemes_are_local_kfunctors} and the discussion above imply that we have the following result.

\begin{corollary}\label{corollary:functors_of_points_zariski_local_k_functors_zariski_sheaves}
There exists the commutative triangle of functors and categories
\begin{center}
\begin{tikzpicture}
[description/.style={fill=white,inner sep=2pt}]
\matrix (m) [matrix of math nodes, row sep=2em, column sep=0.5em,text height=1.5ex, text depth=0.25ex] 
{ \mbox{\emph{Zariski sheaves on $\Sch_k$}}  &        & \mbox{\emph{Zariski local $k$-functors}} \\
                                               & \Sch_k &                                          \\};
\path[->,line width=1.0pt,font=\scriptsize]  
(m-1-1) edge node[auto] {$ \simeq $} (m-1-3);
\path[right hook->,line width=1.0pt,font=\scriptsize]
(m-2-2) edge node[auto] {$ $} (m-1-1)
(m-2-2) edge node[right = 1pt, below = 2pt] {$  $} (m-1-3);
\end{tikzpicture}
\end{center}
where the horizontal functor is an equivalence, the left hand side functor is the Yoneda embedding and the right hand side functor is the restriction of $\fP$ to the category of Zariski local $k$-functors, which contains its essential image. In particular, both nonhorizontal functors in the diagram are full, faithful and create limits.
\end{corollary}

\begin{definition}
Let $X$ be a $k$-scheme. Then the image of $X$ under $\fP$ is a $k$-functor given by formula
$$\Alg_k\ni A \mapsto \Mor_k\left(\Spec A,X\right) \in \Set$$
We call this $k$-functor \textit{the functor of points of $X$}.
\end{definition}

\begin{remark}\label{remark:functors_of_points}
By means of identifications in Corollary \ref{corollary:functors_of_points_zariski_local_k_functors_zariski_sheaves} we do not make any formal and notational distinction between $k$-scheme $X$ and its functor of points. In particular, we denote by $X$ the functor of points of a $k$-scheme $X$. According to the same result we also do not distinguish between functor of points as a Zariski local $k$-functor and as a Zariski sheaf on $\Sch_k$.
\end{remark}

\begin{definition}
Let $\fX$ be a $k$-functor (or a presheaf on $\Sch_k$). We say that $\fX$ is \textit{representable} or is \textit{a scheme} if it is a functor of points of some $k$-scheme.
\end{definition}
\noindent
Let us observe that:

\begin{fact}\label{fact:internal_hom_exists_for_schemes}
Let $X,Y$ be $k$-schemes. Then $\iMor_k(X,Y)$ exists. 
\end{fact}
\begin{proof}
Fix a $k$-algebra $A$ and observe that the class $\Mor_A\left(X_A,Y_A\right)$ of natural transformations (morphisms of $A$-functors) is in bijective correspondence (via Yoneda lemma) with the set of morphisms $\Mor_A\left(\Spec A\times_kX,\Spec A\times_kY\right)$ of $A$-schemes.
\end{proof}
\noindent
Finally we note here the following result that we need for the later use.

\begin{proposition}\label{proposition:representable_monomorphisms_are_sheaves}
Let $\sigma:\fX \ra \fY$ be a monomorphism of $k$-functors and $\fY$ be a Zariski local $k$-functor. Assume that for every $k$-algebra $A$ and every morphism $\tau:\Spec A \ra \fY$ of $k$-functors there exist a Zariski local $k$-functor $\fZ$ that fits into a cartesian square
\begin{center}
\begin{tikzpicture}
[description/.style={fill=white,inner sep=2pt}]
\matrix (m) [matrix of math nodes, row sep=3em, column sep=3em,text height=1.5ex, text depth=0.25ex] 
{  \fZ       & \fX           \\
   \Spec A   & \fY           \\};
\path[->,line width=1.0pt,font=\scriptsize]
(m-1-1) edge node[above] {$  $} (m-1-2)
(m-2-1) edge node[below] {$ \tau $} (m-2-2)
(m-1-2) edge node[right] {$ \sigma $} (m-2-2)
(m-1-1) edge node[left] {$   $} (m-2-1);
\end{tikzpicture}
\end{center}
Then $\fX$ is a Zariski local $k$-functor.
\end{proposition}
\begin{proof}
Suppose that $A$ is a $k$-algebra and $S$ is a covering sieve on $A$ with respect to Zariski topology. Recall that by {\cite[page 2]{Sheaves}} we may consider $S$ as a subcopresheaf of $\Spec A$. Suppose that $\tau:\Spec A \ra \fY$ and $m:S\ra \fX$ are morphisms of $k$-functors such that $\sigma \cdot m$ is equal to the composition of $S\hookrightarrow \Spec A$ with $\tau$. Next there exists a Zariski local $k$-functor $\fZ$ that fits into a cartesian square
\begin{center}
\begin{tikzpicture}
[description/.style={fill=white,inner sep=2pt}]
\matrix (m) [matrix of math nodes, row sep=3em, column sep=3em,text height=1.5ex, text depth=0.25ex] 
{  \fZ       & \fX           \\
   \Spec A   & \fY           \\} ;
\path[->,line width=1.0pt,font=\scriptsize]
(m-1-1) edge node[above] {$\tau'  $} (m-1-2)
(m-2-1) edge node[below] {$\tau  $} (m-2-2)
(m-1-2) edge node[right] {$ \sigma $} (m-2-2)
(m-1-1) edge node[left] {$ \sigma'  $} (m-2-1);
\end{tikzpicture}
\end{center}
of $k$-functors. By universal property of cartesian squares there exists a unique morphism $n:S\ra \fZ$ of $k$-functors such that the diagram
\begin{center}
\begin{tikzpicture}
[description/.style={fill=white,inner sep=2pt}]
\matrix (m) [matrix of math nodes, row sep=3em, column sep=3em,text height=1.5ex, text depth=0.25ex] 
{ S & {}            &  {}               \\
   {} &  \fZ        & \fX     \\
   {} & \Spec A &  \fY       \\} ;
\path[->,line width=1.0pt,font=\scriptsize]
(m-1-1) edge node [right= 2pt, below= 2pt] {$ n $} (m-2-2)
(m-2-2) edge node [above = 2pt] {$\tau' $} (m-2-3)
(m-2-2) edge node [left= 2pt] {$\sigma' $} (m-3-2)
(m-2-3) edge node [right] {$\sigma $} (m-3-3)
(m-3-2) edge node [below] {$\tau $} (m-3-3);
\path[->, bend left, line width=1.0pt, font=\scriptsize]
(m-1-1) edge node [left= 2pt, above= 2pt] {$m $} (m-2-3);
\path[right hook->, bend right, line width=1.0pt, font=\scriptsize]
(m-1-1) edge node [left= 2pt, below= 2pt] {$ $} (m-3-2);
\end{tikzpicture} 
\end{center}
is commutative. Since $\fZ$ is Zariski local, there exists a morphism $\rho:\Spec A \ra \fZ$ such that $\rho_{\mid S} = n$. Then $\left(\tau'\cdot \rho\right)_{\mid S} = \tau'\cdot n = m$ and hence matching family $m$ admits an amalgamation. Since $\sigma$ is a monomorphism, this suffices to prove that $\fX$ is a Zariski local $k$-functor.
\end{proof}

\section{Closed, open $k$-subfunctors and criterion for representability}
\noindent
Suppose now that $A$ is a $k$-algebra and $\ideal{a}\subseteq A$ is an ideal. Then we define $V(\ideal{a}) = \Spec A/\ideal{a}$ as a closed subscheme $\Spec A$ induced by the quotient morphism $A\ra A/\ideal{a}$. We define an open subscheme $D(\ideal{a}) = \Spec A\setminus V(\ideal{a})$ of $\Spec A$.

\begin{definition}
Let $\sigma:\fX\ra \fY$ be a morphism of $k$-functors. Assume that for every $k$-algebra $A$ and every morphism $\tau:\Spec A \ra \fY$ of $k$-functors there exists an ideal $\ideal{a}$ in $A$ and a morphism $\tau':D(\ideal{a})\ra \fX$ of $k$-functors such that the square
\begin{center}
\begin{tikzpicture}
[description/.style={fill=white,inner sep=2pt}]
\matrix (m) [matrix of math nodes, row sep=3em, column sep=3em,text height=1.5ex, text depth=0.25ex] 
{  D(\ideal{a})        & \fX           \\
\Spec A             & \fY           \\} ;
\path[->,line width=1.0pt,font=\scriptsize]
(m-1-1) edge node[above] {$ \tau' $} (m-1-2)
(m-2-1) edge node[below] {$ \tau $} (m-2-2)
(m-1-2) edge node[right] {$ \sigma $} (m-2-2);
\path[right hook->,line width=1.0pt,font=\scriptsize]
(m-1-1) edge node[left] {$   $} (m-2-1);
\end{tikzpicture}
\end{center}
is cartesian. Then $\sigma$ is \textit{an open immersion of $k$-functors}.
\end{definition}
\noindent
We make an easy observation.

\begin{fact}\label{fact:open_immersions_are_closed_under_base_change_and_composition}
The class of open immersions of $k$-functors is closed under base change and composition.
\end{fact}
\noindent
Now we define open covers.

\begin{definition}
Let $\fX$ be a $k$-functor and $\big\{\sigma_i:\fX_i\ra \fX\big\}_{i\in I}$ be a family of open immersions. Then for every $k$-algebra $A$ and $x\in \fX(A)$ we have a family of ideals $\{\ideal{a}_i\}_{i\in I}$ defined by cartesian squares
\begin{center}
\begin{tikzpicture}
[description/.style={fill=white,inner sep=2pt}]
\matrix (m) [matrix of math nodes, row sep=3em, column sep=3em,text height=1.5ex, text depth=0.25ex] 
{ D(\ideal{a}_i)   &    \fX_i   \\
\Spec A          &    \fX  \\} ;
\path[->,line width=1.0pt,font=\scriptsize]  
(m-1-1) edge node[above] {$ \tau'  $} (m-1-2)
(m-2-1) edge node[below] {$ \tau $} (m-2-2)
(m-1-2) edge node[right] {$\sigma_i$} (m-2-2);
\path[right hook->,line width=1.0pt,font=\scriptsize]
(m-1-1) edge node[left] {$ $} (m-2-1);
\end{tikzpicture}
\end{center}
in which bottom vertical morphism $\tau:\Spec A \ra \fX$ corresponds to $x$. We say that $\{\sigma_i\}_{i\in I}$ is \textit{an open cover of $\fX$} if for every $k$-algebra $A$ and $x\in \fX(A)$ we have
$$\Spec A = \bigcup_{i\in I}D(\ideal{a}_i)$$
or in other words $A = \sum_{i\in I}\ideal{a}_i$.
\end{definition}
\noindent
These notions are intertwined in the following elementary yet beautiful result.

\begin{theorem}\label{theorem:representability_basic_result}
Let $\fX$ be a $k$-functor. Then the following are equivalent.
\begin{enumerate}[label=\emph{\textbf{(\roman*)}}, leftmargin=3.0em]
\item $\fX$ is isomorphic with functor of points of some $k$-scheme.
\item $\fX$ is a Zariski local $k$-functor and there exists an open cover $\big\{\sigma_i:X_i \ra \fX\big\}_{i\in I}$ of $k$-functors for some family $\{X_i\}_{i\in I}$ of $k$-schemes.
\item $\fX$ is a Zariski local $k$-functor and there exists an open cover $\big\{\sigma_i:\Spec A_i \ra \fX\big\}_{i\in I}$ of $k$-functors for some family $\{A_i\}_{i\in I}$ of $k$-algebras.
\end{enumerate}
\end{theorem}
\noindent
The proof depends on two lemmas. Check {\cite[Definition 7.1]{Sheaves}} for the notion of a locally surjective morphism.

\begin{lemma}\label{lemma:open_immersions_locally_on_domain_are_zariski_local_k_functors}
Let $f:X\ra Y$ be a morphism of $k$-schemes. Suppose that $f$ is surjective morphism and an open immersion locally on $X$. Then $f$ is a locally surjective morphism of Zariski local $k$-functors. 
\end{lemma}
\begin{proof}[Proof of the lemma]
Let $A$ be a $k$-algebra and $g:\Spec A\ra Y$ be a morphism of $k$-schemes. Since $f$ is surjective and an open immersion locally on $X$, there exist a Zariski cover $\big\{f_i:A\ra A_i\big\}_{i\in I}$ and a family $\big\{g_i:\Spec A_i\ra X\big\}_{i\in I}$ of morphisms of $k$-schemes such that $f\cdot g_i = g\cdot \Spec f_i$ for every $i\in I$. This by definition implies that $f$ is a locally surjective morphism of Zariski local $k$-functors.
\end{proof}

\begin{lemma}\label{lemma:recollement}
Let $X = \coprod_{i\in I}X_i, R = \coprod_{i,j\in I}R_{ij}$ be disjoint sums of $k$-schemes and let $p,q:R\ra X$ be morphisms of $k$-schemes such that the following conditions are satisfied.
\begin{enumerate}[label=\emph{\textbf{(\arabic*)}}, leftmargin=3.0em]
\item For any $i,j\in I$ morphism $p_{\mid R_{ij}}$ induces an open immersion $R_{ij}\hookrightarrow X_i$ and morphism $q_{\mid R_{ij}}$ induces an open immersion $R_{ij}\hookrightarrow X_j$.
\item For every $i\in I$ morphisms $p_{\mid R_{ii}}$ and $q_{\mid R_{ii}}$ are equal and induce an isomorphisms $R_{ii}\ra X_i$.  
\item Triple $\left(R,p,q\right)$ is an equivalence relation on $X$ in the category of $k$-schemes.
\end{enumerate}
Then there exist a $k$-scheme $Y$ and a morphism $f:X\ra Y$ of $k$-schemes such that
\begin{center}
\begin{tikzpicture}
[description/.style={fill=white,inner sep=2pt}]
\matrix (m) [matrix of math nodes, row sep=3em, column sep=3em,text height=1.5ex, text depth=0.25ex] 
{ R &  X & Y  \\} ;
\path[->,line width=0.8pt,font=\scriptsize]
(m-1-1) edge[transform canvas={yshift=0.5ex}] node[above] {$ p  $} (m-1-2)
(m-1-1) edge[transform canvas={yshift=-0.5ex}] node[below] {$ q $} (m-1-2)
(m-1-2) edge node[above] {$ f  $} (m-1-3);
\end{tikzpicture}
\end{center}
is a cokernel of a pair $(p,q)$ in the category of Zariski local $k$-functors.
\end{lemma}
\begin{proof}[Proof of the lemma]
Let
\begin{center}
\begin{tikzpicture}
[description/.style={fill=white,inner sep=2pt}]
\matrix (m) [matrix of math nodes, row sep=3em, column sep=3em,text height=1.5ex, text depth=0.25ex] 
{ R &  X & Y  \\} ;
\path[->,line width=0.8pt,font=\scriptsize]
(m-1-1) edge[transform canvas={yshift=0.5ex}] node[above] {$ p  $} (m-1-2)
(m-1-1) edge[transform canvas={yshift=-0.5ex}] node[below] {$ q $} (m-1-2)
(m-1-2) edge node[above] {$ f  $} (m-1-3);
\end{tikzpicture}
\end{center}
be a cokernel in the category of ringed spaces. It exists according to {\cite[Remark 2.3]{LocallyRingedSpaces}}. Moreover, {\cite[Theorem 3.2]{LocallyRingedSpaces}} states that for every $i\in I$ subset $f(X_i)$ is open in $Y$ and we have an isomorphism of ringed spaces $X_i\cong f(X_i)$ induced by $f$. Therefore, $Y$ is a $k$-scheme and $f:X\ra Y$ is a morphism of $k$-schemes.\\
Now we verify that $f$ is the quotient in the category of Zariski local $k$-functors. For this note that we proved above that $f$ is open immersion of $k$-schemes locally on $X$ and it is surjective. Thus by Lemma \ref{lemma:open_immersions_locally_on_domain_are_zariski_local_k_functors} we derive that $f$ is a locally surjective morphism of Zariski local $k$-functors. Therefore ({\cite[Theorem 7.3]{Sheaves}}), it suffices to show that the square
\begin{center}
\begin{tikzpicture}
[description/.style={fill=white,inner sep=2pt}]
\matrix (m) [matrix of math nodes, row sep=3em, column sep=3em,text height=1.5ex, text depth=0.25ex] 
{ R  &   X   \\
  X  &   Y   \\};
\path[->,line width=1.0pt,font=\scriptsize]  
(m-1-1) edge node[above] {$ q $} (m-1-2)
(m-2-1) edge node[below] {$ f $} (m-2-2)
(m-1-1) edge node[left] {$ p $} (m-2-1)
(m-1-2) edge node[right] {$ f $} (m-2-2);
\end{tikzpicture}
\end{center}
is cartesian. Since the functor
\begin{center}
\begin{tikzpicture}
[description/.style={fill=white,inner sep=2pt}]
\matrix (m) [matrix of math nodes, row sep=3em, column sep=3em,text height=1.5ex, text depth=0.25ex] 
{ \Sch_k  &  \mbox{Zariski local $k$-functors} \\};
\path[right hook->,line width=1.0pt,font=\scriptsize]  
(m-1-1) edge node[auto] {$ $} (m-1-2);
\end{tikzpicture}
\end{center}
described in Corollary \ref{corollary:functors_of_points_zariski_local_k_functors_zariski_sheaves} preserves limits, we derive that it suffices to check that
\begin{center}
\begin{tikzpicture}
[description/.style={fill=white,inner sep=2pt}]
\matrix (m) [matrix of math nodes, row sep=3em, column sep=3em,text height=1.5ex, text depth=0.25ex] 
{ R  &   X   \\
  X  &   Y   \\};
\path[->,line width=1.0pt,font=\scriptsize]  
(m-1-1) edge node[above] {$ q $} (m-1-2)
(m-2-1) edge node[below] {$ f $} (m-2-2)
(m-1-1) edge node[left] {$ p $} (m-2-1)
(m-1-2) edge node[right] {$ f $} (m-2-2);
\end{tikzpicture}
\end{center}
is cartesian square of $k$-schemes. By {\cite[Remark 2.3]{LocallyRingedSpaces}} we have $R_{ij} = X_i\times_YX_j$ for every $i, j\in I$ and hence
$$X\times_YX = \left(\coprod_{i\in I}X_i\right)\times_Y \left(\coprod_{i\in I}X_i\right) = \coprod_{i,j\in I}\left(X_i\times_YX_j\right) = \coprod_{i,j\in I}R_{ij} = R$$
Thus the result follows.
\end{proof}

\begin{proof}[Proof of the theorem]
Every $k$-scheme $Y$ is a Zariski local $k$-functor and clearly $1_{Y}:Y \ra Y$ is an open cover. Thus $\textbf{(i)}\Rightarrow \textbf{(ii)}$.\\
Every functor of points of a $k$-scheme admits open cover by functors of points of affine $k$-schemes. Indeed, it suffices to take open affine subschemes that cover given $k$-scheme. This implies that every open cover of a $k$-functor $\fX$ by functors of points of $k$-schemes admits refinement by open cover of functors of points of affine $k$-schemes. Therefore, implication $\textbf{(ii)}\Rightarrow \textbf{(iii)}$ holds.\\
Suppose that $\fX$ is a Zariski local $k$-functor and $\big\{\sigma_i:\Spec A_i\ra \fX\big\}_{i\in I}$ is an open cover of $\fX$. Note that for every $i,j\in I$ there exist a $k$-scheme $R_{ij}$ and open immersions $p_{ij}:R_{ij}\hookrightarrow \Spec A_i$, $q_{ij}:R_{ij}\hookrightarrow \Spec A_j$ such that the square
\begin{center}
\begin{tikzpicture}
[description/.style={fill=white,inner sep=2pt}]
\matrix (m) [matrix of math nodes, row sep=3em, column sep=3em,text height=1.5ex, text depth=0.25ex] 
{  R_{ij}                &   \Spec A_j   \\
  \Spec A_i  &   \fX   \\};
\path[->,line width=1.0pt,font=\scriptsize]  
(m-1-1) edge node[above] {$ q_{ij} $} (m-1-2)
(m-2-1) edge node[below] {$ \sigma_i $} (m-2-2)
(m-1-1) edge node[left] {$ p_{ij} $} (m-2-1)
(m-1-2) edge node[right] {$ \sigma_j $} (m-2-2);
\end{tikzpicture}
\end{center}
is cartesian. Consider a $k$-scheme $X = \coprod_{i\in I}\Spec A_i$ and a morphism $\sigma:X \ra \fX$ induced by $\{\sigma_i\}_{i\in I}$. Moreover, consider a $k$-scheme $R = \coprod_{i,j\in I}R_{ij}$ and morphisms $p,q:R\ra X$ induced by $\{p_{ij}\}_{i,j\in I}$ and $\{q_{ij}\}_{i,j\in I}$, respectively. Note that the square
\begin{center}
\begin{tikzpicture}
[description/.style={fill=white,inner sep=2pt}]
\matrix (m) [matrix of math nodes, row sep=3em, column sep=3em,text height=1.5ex, text depth=0.25ex] 
{ R  &   X   \\
  X  &   \fX   \\};
\path[->,line width=1.0pt,font=\scriptsize]  
(m-1-1) edge node[above] {$ q $} (m-1-2)
(m-2-1) edge node[below] {$ \sigma $} (m-2-2)
(m-1-1) edge node[left] {$ p $} (m-2-1)
(m-1-2) edge node[right] {$ \sigma $} (m-2-2);
\end{tikzpicture}
\end{center}
is cartesian and hence $\left(R,p,q\right)$ is an equivalence relation in the category of Zariski local $k$-functors. By Lemma \ref{lemma:recollement} there exist a $k$-scheme $Y$ and a morphism $f:X\ra Y$ such that
\begin{center}
\begin{tikzpicture}
[description/.style={fill=white,inner sep=2pt}]
\matrix (m) [matrix of math nodes, row sep=3em, column sep=3em,text height=1.5ex, text depth=0.25ex] 
{R &  X & Y  \\} ;
\path[->,line width=0.8pt,font=\scriptsize]
(m-1-1) edge[transform canvas={yshift=0.5ex}] node[above] {$ p  $} (m-1-2)
(m-1-1) edge[transform canvas={yshift=-0.5ex}] node[below] {$ q $} (m-1-2)
(m-1-2) edge node[above] {$ f  $} (m-1-3);
\end{tikzpicture}
\end{center}
is a cokernel of $\left(p,q\right)$ in the category of Zariski local $k$-functors. Moreover, $\sigma$ is locally surjective morphism of Zariski local $k$-functors and hence also
\begin{center}
\begin{tikzpicture}
[description/.style={fill=white,inner sep=2pt}]
\matrix (m) [matrix of math nodes, row sep=3em, column sep=3em,text height=1.5ex, text depth=0.25ex] 
{ R &  X & \fX  \\} ;
\path[->,line width=0.8pt,font=\scriptsize]
(m-1-1) edge[transform canvas={yshift=0.5ex}] node[above] {$ p $} (m-1-2)
(m-1-1) edge[transform canvas={yshift=-0.5ex}] node[below] {$ q $} (m-1-2)
(m-1-2) edge node[above] {$ \sigma  $} (m-1-3);
\end{tikzpicture}
\end{center}
is a cokernel of $\left(p,q\right)$. Thus $Y$ is isomorphic with $\fX$. This proves $\textbf{(iii)}\Rightarrow \textbf{(i)}$.
\end{proof}

\section{Representable morphisms of $k$-functors}

\begin{definition}
Let $\sigma:\fX\ra \fY$ be a morphism of Zariski local $k$-functors. Assume that for every $k$-algebra $A$ and every morphism $\tau:\Spec A\ra \fY$ of $k$-functors there exist a $k$-scheme $X$, a morphism $f:X\ra \Spec A$ of $k$-schemes and a morphism $\tau':X\ra \fX$ of $k$-functors such that the square
\begin{center}
\begin{tikzpicture}
[description/.style={fill=white,inner sep=2pt}]
\matrix (m) [matrix of math nodes, row sep=3em, column sep=3em,text height=1.5ex, text depth=0.25ex] 
{  X        & \fX           \\
   \Spec A             & \fY           \\} ;
\path[->,line width=1.0pt,font=\scriptsize]
(m-1-1) edge node[above] {$ \tau' $} (m-1-2)
(m-2-1) edge node[below] {$ \tau $} (m-2-2)
(m-1-2) edge node[right] {$ \sigma $} (m-2-2)
(m-1-1) edge node[left] {$ f  $} (m-2-1);
\end{tikzpicture}
\end{center}
is cartesian. Then $\sigma$ is \textit{a representable morphism of $k$-functors}.
\end{definition}
\noindent
It is easy to proof the following. 

\begin{fact}\label{fact:representable_morphisms_under_base_change_and_composition}
The class of representable morphisms of $k$-functors is closed under base change and composition.
\end{fact}

\begin{proposition}\label{proposition:representable_are_representable_after_arbitrary_base_change}
Let $\sigma:\fX\ra \fY$ be a representable morphism of Zariski local $k$-functors. Fix a $k$-scheme $Y$ and a morphism $\tau:Y\ra \fY$. Then there exist a $k$-scheme $X$, a morphism $f:X\ra Y$ of $k$-schemes and a morphism $\tau':X\ra \fX$ such that the square
\begin{center}
\begin{tikzpicture}
[description/.style={fill=white,inner sep=2pt}]
\matrix (m) [matrix of math nodes, row sep=3em, column sep=3em,text height=1.5ex, text depth=0.25ex] 
{  X    & \fX           \\
   Y    & \fY           \\} ;
\path[->,line width=1.0pt,font=\scriptsize]
(m-1-1) edge node[above] {$ \tau' $} (m-1-2)
(m-2-1) edge node[below] {$ \tau $} (m-2-2)
(m-1-2) edge node[right] {$ \sigma $} (m-2-2)
(m-1-1) edge node[left] {$ f  $} (m-2-1);
\end{tikzpicture}
\end{center}
is cartesian.
\end{proposition}
\begin{proof}
Let
\begin{center}
\begin{tikzpicture}
[description/.style={fill=white,inner sep=2pt}]
\matrix (m) [matrix of math nodes, row sep=3em, column sep=3em,text height=1.5ex, text depth=0.25ex] 
{  \fZ        & \fX           \\
   Y    & \fY           \\} ;
\path[->,line width=1.0pt,font=\scriptsize]
(m-1-1) edge node[above] {$ \tau' $} (m-1-2)
(m-2-1) edge node[below] {$ \tau $} (m-2-2)
(m-1-2) edge node[right] {$ \sigma $} (m-2-2)
(m-1-1) edge node[left] {$ \sigma'  $} (m-2-1);
\end{tikzpicture}
\end{center}
be a cartesian square. According to {\cite[Theorem 2.12]{Sheaves}} $k$-functor $\fZ$ is Zariski local. Suppose that $\big\{f_i:\Spec A_i\ra Y\big\}_{i\in I}$ is an open cover of $Y$ as a $k$-scheme. Then it is also an open cover of $Y$ as a $k$-functor and hence its base change $\big\{\tau_i:\fZ_i\ra \fZ\big\}_{i\in I}$ is an open cover of $\fZ$. Since $\sigma$ is representable, we deduce that $\fZ_i$ is a functor of points of some $k$-scheme for $i\in I$. Now by Theorem \ref{theorem:representability_basic_result} we derive that there exists a $k$-scheme $X$ such that $\fZ$ is isomorphic with $X$. This proves the result.
\end{proof}

\begin{definition}
Suppose that $\cC$ is any class of morphism of $k$-schemes, which is local on the target. Let $\sigma:\fX\ra \fY$ be a morphism of Zariski local $k$-functors. Assume that for every $k$-algebra $A$ and every morphism $\tau:\Spec A\ra \fY$ of $k$-functors there exist a $k$-scheme $X$, a morphism $f:X\ra \Spec A$ of $k$-schemes and a morphism $\tau':X\ra \fX$ of $k$-functors such that the square
\begin{center}
\begin{tikzpicture}
[description/.style={fill=white,inner sep=2pt}]
\matrix (m) [matrix of math nodes, row sep=3em, column sep=3em,text height=1.5ex, text depth=0.25ex] 
{  X        & \fX           \\
\Spec A             & \fY           \\} ;
\path[->,line width=1.0pt,font=\scriptsize]
(m-1-1) edge node[above] {$ \tau' $} (m-1-2)
(m-2-1) edge node[below] {$ \tau $} (m-2-2)
(m-1-2) edge node[right] {$ \sigma $} (m-2-2)
(m-1-1) edge node[left] {$ f  $} (m-2-1);
\end{tikzpicture}
\end{center}
is cartesian and $f$ is in $\cC$. Then we say that is \textit{$\sigma$ is in $\cC$}.
\end{definition}
\noindent
We prove the result which allows us to translate Grothendieck's dictionary describing classes of morphisms of $k$-schemes to the language of Zariski local $k$-functors.

\begin{proposition}\label{proposition:classes_of_morphisms}
Let $\sigma:\fX\ra \fY$ be a representable morphism of Zariski local $k$-functors and let $\cC$ be a class of morphism of $k$-schemes that is local on the target. Then the following assertions are equivalent.
\begin{enumerate}[label=\emph{\textbf{(\roman*)}}, leftmargin=3.0em]
\item $\sigma$ is in $\cC$.
\item For every $k$-scheme $Y$ and every morphism $\tau:Y\ra \fY$ of $k$-functors there exist a $k$-scheme $X$, a morphism $f:X\ra Y$ of $k$-schemes and a morphism $\tau':X\ra \fX$ of $k$-functors such that the square
\begin{center}
\begin{tikzpicture}
[description/.style={fill=white,inner sep=2pt}]
\matrix (m) [matrix of math nodes, row sep=3em, column sep=3em,text height=1.5ex, text depth=0.25ex] 
{  X        & \fX           \\
   Y             & \fY           \\} ;
\path[->,line width=1.0pt,font=\scriptsize]
(m-1-1) edge node[above] {$ \tau' $} (m-1-2)
(m-2-1) edge node[below] {$ \tau $} (m-2-2)
(m-1-2) edge node[right] {$ \sigma $} (m-2-2)
(m-1-1) edge node[left] {$ f  $} (m-2-1);
\end{tikzpicture}
\end{center}
is cartesian and $f$ as a morphism of $k$-schemes is in $\cC$.
\end{enumerate} 
\end{proposition}
\begin{proof}
It suffices to prove that $\textbf{(i)}\Rightarrow \textbf{(ii)}$. Suppose that \textbf{(i)} holds. Pick a $k$-scheme $Y$ and consider a morphism $\tau:Y\ra \fY$ of $k$-functors. Then by Proposition \ref{proposition:representable_are_representable_after_arbitrary_base_change} there exist a $k$-scheme $X$, a morphism $f:X\ra Y$ of $k$-schemes and a morphism $\tau':X\ra \fX$ of $k$-functors such that the square
\begin{center}
\begin{tikzpicture}
[description/.style={fill=white,inner sep=2pt}]
\matrix (m) [matrix of math nodes, row sep=3em, column sep=3em,text height=1.5ex, text depth=0.25ex] 
{  X        & \fX           \\
   Y             & \fY           \\} ;
\path[->,line width=1.0pt,font=\scriptsize]
(m-1-1) edge node[above] {$ \tau' $} (m-1-2)
(m-2-1) edge node[below] {$ \tau $} (m-2-2)
(m-1-2) edge node[right] {$ \sigma $} (m-2-2)
(m-1-1) edge node[left] {$ f  $} (m-2-1);
\end{tikzpicture}
\end{center}
is cartesian. Now pick an open affine subset $V$ of $Y$. Then the square 
\begin{center}
\begin{tikzpicture}
[description/.style={fill=white,inner sep=2pt}]
\matrix (m) [matrix of math nodes, row sep=3em, column sep=3em,text height=1.5ex, text depth=0.25ex] 
{  f^{-1}(V)        & X           \\
   V             & Y           \\} ;
\path[->,line width=1.0pt,font=\scriptsize]
(m-1-1) edge node[left] {$   $} (m-2-1)
(m-1-2) edge node[right] {$ f $} (m-2-2);
\path[right hook->,line width=1.0pt,font=\scriptsize]
(m-2-1) edge node[below] {$  $} (m-2-2)
(m-1-1) edge node[above] {$  $} (m-1-2);
\end{tikzpicture}
\end{center}
is cartesian and the left vertical arrow is in $\cC$ by \textbf{(i)}. Since $V$ is arbitrary and $\cC$ is local on the target, this implies that $f\in \cC$. Hence $\sigma$ admits \textbf{(ii)}.
\end{proof}


\noindent
Now we can apply the result above to Zariski sheaves on $\Sch_k$.

\begin{corollary}\label{corollary:representable_morphisms_of_Zariski_sheaves}
Let $\sigma:\fX\ra \fY$ be a representable morphism of Zariski sheaves on $\Sch_k$. Then the following assertions are equivalent.
\begin{enumerate}[label=\emph{\textbf{(\roman*)}}, leftmargin=3.0em]
\item For every $k$-algebra $A$ and every morphism $\tau:\Spec A\ra \fY$ of $k$-presheaves there exist a $k$-scheme $X$, a morphism $f:X\ra \Spec A$ of $k$-schemes and a morphism $\tau':X\ra \fX$ of $k$-functors such that the square
\begin{center}
\begin{tikzpicture}
[description/.style={fill=white,inner sep=2pt}]
\matrix (m) [matrix of math nodes, row sep=3em, column sep=3em,text height=1.5ex, text depth=0.25ex] 
{  X        & \fX           \\
   \Spec A             & \fY           \\} ;
\path[->,line width=1.0pt,font=\scriptsize]
(m-1-1) edge node[above] {$ \tau' $} (m-1-2)
(m-2-1) edge node[below] {$ \tau $} (m-2-2)
(m-1-2) edge node[right] {$ \sigma $} (m-2-2)
(m-1-1) edge node[left] {$ f  $} (m-2-1);
\end{tikzpicture}
\end{center}
is cartesian.
\item For every $k$-scheme $Y$ and every morphism $\tau:Y\ra \fY$ of $k$-functors there exist a $k$-scheme $X$, a morphism $f:X\ra Y$ of $k$-schemes and a morphism $\tau':X\ra \fX$ of $k$-functors such that the square
\begin{center}
\begin{tikzpicture}
[description/.style={fill=white,inner sep=2pt}]
\matrix (m) [matrix of math nodes, row sep=3em, column sep=3em,text height=1.5ex, text depth=0.25ex] 
{  X        & \fX           \\
   Y             & \fY           \\} ;
\path[->,line width=1.0pt,font=\scriptsize]
(m-1-1) edge node[above] {$ \tau' $} (m-1-2)
(m-2-1) edge node[below] {$ \tau $} (m-2-2)
(m-1-2) edge node[right] {$ \sigma $} (m-2-2)
(m-1-1) edge node[left] {$ f  $} (m-2-1);
\end{tikzpicture}
\end{center}
is cartesian.
\end{enumerate} 
\end{corollary}
\begin{proof}
It suffices to prove that $\textbf{(i)}\Rightarrow \textbf{(ii)}$. Suppose that \textbf{(i)} holds. This implies according to Proposition \ref{proposition:representable_are_representable_after_arbitrary_base_change} that $\sigma_{\mid \Alg_k}$ admits analogical property to \textbf{(ii)} but for $k$-functors. By Corollary \ref{corollary:functors_of_points_zariski_local_k_functors_zariski_sheaves} it follows that $\sigma$ itself admits \textbf{(ii)}.
\end{proof}




\noindent
Now for completeness we state the analogical definitions for presheaves on $\Sch_k$.

\begin{definition}
Let $\sigma:\fX\ra \fY$ be a morphism of presheaves on $\Sch_k$. Assume that for every $k$-scheme $Y$ and every morphism $\tau:Y \ra \fY$ of presheaves on $\Sch_k$ there exist an open subscheme $X\hookrightarrow Y$ and a morphism $\tau':X \ra \fX$ of presheaves such that the square
\begin{center}
\begin{tikzpicture}
[description/.style={fill=white,inner sep=2pt}]
v\matrix (m) [matrix of math nodes, row sep=3em, column sep=3em,text height=1.5ex, text depth=0.25ex] 
{  X        & \fX           \\
Y             & \fY           \\} ;
\path[->,line width=1.0pt,font=\scriptsize]
(m-1-1) edge node[above] {$ \tau' $} (m-1-2)
(m-2-1) edge node[below] {$ \tau $} (m-2-2)
(m-1-2) edge node[right] {$ \sigma $} (m-2-2);
\path[right hook->,line width=1.0pt,font=\scriptsize]
(m-1-1) edge node[left] {$   $} (m-2-1);
\end{tikzpicture}
\end{center}
is cartesian. Then $\sigma$ is \textit{an open immersion of presheaves on $\Sch_k$}.
\end{definition}

\begin{definition}
Let $\sigma:\fX\ra \fY$ be a morphism of presheaves on $\Sch_k$. Assume that for every $k$-scheme $Y$ and every morphism $\tau:Y \ra \fY$ of presheaves on $\Sch_k$ there exist a closed subscheme $X\hookrightarrow Y$ and a morphism $\tau':X \ra \fX$ of presheaves such that the square
\begin{center}
\begin{tikzpicture}
[description/.style={fill=white,inner sep=2pt}]
\matrix (m) [matrix of math nodes, row sep=3em, column sep=3em,text height=1.5ex, text depth=0.25ex] 
{  X        & \fX           \\
Y             & \fY           \\} ;
\path[->,line width=1.0pt,font=\scriptsize]
(m-1-1) edge node[above] {$ \tau' $} (m-1-2)
(m-2-1) edge node[below] {$ \tau $} (m-2-2)
(m-1-2) edge node[right] {$ \sigma $} (m-2-2);
\path[right hook->,line width=1.0pt,font=\scriptsize]
(m-1-1) edge node[left] {$   $} (m-2-1);
\end{tikzpicture}
\end{center}
is cartesian. Then $\sigma$ is \textit{a closed immersion of presheaves on $\Sch_k$}.
\end{definition}

\begin{remark}\label{remark:open_and_closed_immersions_for_Zariski_local_and_Zariski_sheaves_are_the_same}
Equivalence described in Theorem \ref{theorem:sheaves_on_schemes_are_local_kfunctors} identifies the class of open (closed) immersions of Zariski local $k$-functors on the one hand and the class of open (closed) immersions of Zariski sheaves on $\Sch_k$ on the other. Moreover, the equivalence preserves and reflects open covers.
\end{remark}


\begin{definition}
Let $\fX$ be a presheaf on $\Sch_k$ and $\big\{\sigma_i:\fX_i\ra \fX\big\}_{i\in I}$ be a family of open immersions of presheaves. Then for every $k$-scheme $X$ and $x\in \fX(X)$ we have a family of open subschemes $\{X_i\}_{i\in I}$ of $X$ defined by cartesian squares
\begin{center}
\begin{tikzpicture}
[description/.style={fill=white,inner sep=2pt}]
\matrix (m) [matrix of math nodes, row sep=3em, column sep=3em,text height=1.5ex, text depth=0.25ex] 
{ X_i   &    \fX_i   \\
X          &    \fX  \\} ;
\path[->,line width=1.0pt,font=\scriptsize]  
(m-1-1) edge node[above] {$ \tau'  $} (m-1-2)
(m-2-1) edge node[below] {$ \tau $} (m-2-2)
(m-1-2) edge node[right] {$\sigma_i$} (m-2-2);
\path[right hook->,line width=1.0pt,font=\scriptsize]
(m-1-1) edge node[left] {$ $} (m-2-1);
\end{tikzpicture}
\end{center}
in which bottom vertical morphism $\tau:X \ra \fX$ corresponds to $x$. We say that $\{\sigma_i\}_{i\in I}$ is \textit{an open cover of $\fX$} if for every $k$-scheme $X$ and $x\in \fX(X)$ we have $X = \bigcup_{i\in I}X_i$.
\end{definition}




\begin{definition}
Let $\sigma:\fX\ra \fY$ be a morphism of $k$-functors. Assume that for every $k$-algebra $A$ and every morphism $\tau:\fP_{\Spec_A}\ra \fY$ of $k$-functors there exist an ideal $\ideal{a}$ in $A$ and morphism $\tau':\fP_{V(\ideal{a})}\ra \fX$ such that the square
\begin{center}
\begin{tikzpicture}
[description/.style={fill=white,inner sep=2pt}]
\matrix (m) [matrix of math nodes, row sep=3em, column sep=3em,text height=1.5ex, text depth=0.25ex]
{  \fP_{V(\ideal{a})} = \fP_{\Spec A/\ideal{a}}  & \fX      \\
   \fP_{\Spec A}  & \fY           \\};
\path[->,line width=1.0pt,font=\scriptsize]
(m-1-1) edge node[above] {$\tau' $} (m-1-2)
(m-2-1) edge node[below] {$ \tau $} (m-2-2)
(m-1-1) edge node[left] {$ \fP_{\Spec q} $} (m-2-1)
(m-1-2) edge node[right] {$ \sigma $} (m-2-2);
\end{tikzpicture}
\end{center}
is cartesian, where $q:A\ra A/\ideal{a}$ is the quotient map. Then $\sigma$ is \textit{a closed immersion of $k$-functors}.
\end{definition}

\begin{fact}\label{fact:closed_immersions_closed_under_base_change_and_composition}
The class of closed immersions of $k$-functors is closed under base change and composition.
\end{fact}
\begin{proof}
Left to the reader.
\end{proof}

\begin{proposition}\label{proposition:open_closed_immersions}
Let $\sigma:\fX\ra \fY$ be a closed (open) immersion of $k$-functors. Fix a $k$-scheme $Y$ and a morphism $\tau:\fP_Y\ra \fY$. Then there exist a $k$-scheme $X$, a closed (open) immersion $f:X\ra Y$ of schemes and a morphism $\tau':\fP_X\ra \fX$ of $k$-functors such that the square
\begin{center}
\begin{tikzpicture}
[description/.style={fill=white,inner sep=2pt}]
\matrix (m) [matrix of math nodes, row sep=3em, column sep=3em,text height=1.5ex, text depth=0.25ex] 
{  \fP_{X}        & \fX           \\
   \fP_{Y}             & \fY           \\} ;
\path[->,line width=1.0pt,font=\scriptsize]
(m-1-1) edge node[above] {$ \tau' $} (m-1-2)
(m-2-1) edge node[below] {$ \tau $} (m-2-2)
(m-1-2) edge node[right] {$ \sigma $} (m-2-2)
(m-1-1) edge node[left] {$ \fP_f  $} (m-2-1);
\end{tikzpicture}
\end{center}
is cartesian.
\end{proposition}
\begin{proof}
According to Fact \ref{fact:closed_immersions_closed_under_base_change_and_composition} (Fact \ref{fact:open_immersions_closed_under_base_change_and_composition}) pullback $\fX\times_{\fY}\fP_Y\ra \fP_Y$ of $\sigma$ along $\tau$ is a closed (open) immersion of $k$-functors. Since $\fP_Y$ is a Zariski local $k$-functor by Fact \ref{fact:functors_of_points} and closed (open) immersions are monomorphisms, we derive by Proposition \ref{proposition:representable_monomorphisms_are_sheaves} that a fiber-product $\fX\times_{\fY}\fP_Y$ of $\sigma$ and $\tau$ is a Zariski local $k$-functor. Since closed (open) immersions of $k$-functors are representable, we deduce by Proposition \ref{proposition:representable_are_representable_after_arbitrary_base_change} that there exists a $k$-scheme $X$, a morphism $f:X\ra Y$ of $k$-schemes and a morphism $\tau':\fP_X\ra \fX$ of $k$-functors such that the square
\begin{center}
\begin{tikzpicture}
[description/.style={fill=white,inner sep=2pt}]
\matrix (m) [matrix of math nodes, row sep=3em, column sep=3em,text height=1.5ex, text depth=0.25ex] 
{  \fX\times_{\fY}\fP_Y\cong \fP_{X}        & \fX           \\
   \fP_{Y}        & \fY           \\} ;
\path[->,line width=1.0pt,font=\scriptsize]
(m-1-1) edge node[above] {$ \tau' $} (m-1-2)
(m-2-1) edge node[below] {$ \tau $} (m-2-2)
(m-1-2) edge node[right] {$ \sigma $} (m-2-2)
(m-1-1) edge node[left] {$ \fP_f  $} (m-2-1);
\end{tikzpicture}
\end{center}
is cartesian and $\fP_f$ is a closed (open) immersion of $k$-functors. Since the functor
\begin{center}
\begin{tikzpicture}
[description/.style={fill=white,inner sep=2pt}]
\matrix (m) [matrix of math nodes, row sep=3em, column sep=3em,text height=1.5ex, text depth=0.25ex] 
{ \widehat{\Sch_k}  & \mbox{the category of $k$-functors} \\};
\path[->,line width=1.0pt,font=\scriptsize]  
(m-1-1) edge node[auto] {$\fP $} (m-1-2);
\end{tikzpicture}
\end{center}
preserves finite limits, it follows that for every open affine subset $V$ of $Y$ we have a cartesian square
\begin{center}
\begin{tikzpicture}
[description/.style={fill=white,inner sep=2pt}]
\matrix (m) [matrix of math nodes, row sep=3em, column sep=3em,text height=1.5ex, text depth=0.25ex] 
{  \fP_{f^{-1}(V)}        & \fP_X           \\
   \fP_{V}                & \fP_Y           \\} ;
\path[right hook->,line width=1.0pt,font=\scriptsize]
(m-1-1) edge node[above] {$  $} (m-1-2)
(m-2-1) edge node[below] {$  $} (m-2-2);
\path[->,line width=1.0pt,font=\scriptsize]
(m-1-2) edge node[right] {$ \fP_f $} (m-2-2)
(m-1-1) edge node[left] {$ \fP_{f_V}  $} (m-2-1);
\end{tikzpicture}
\end{center}
where $f_V:f^{-1}(V)\ra V$ is the restriction of $f$. Next as $\fP_f$ is a closed (open) immersion and $V$ is affine, we derive that $f_V$ is a closed (open) immersion of schemes. Since this holds for every affine open subset $V$ of $Y$, we deduce that $f$ is a closed (open) immersion.
\end{proof}
\noindent
The next result is frequently used in the theory of \textit{algebraic spaces}.

\begin{proposition}\label{proposition:diagonal_representability_of_morphisms}
Let $\fY$ be a $k$-functor such that the diagonal $\fY\ra \fY\times \fY$ is representable. Then every morphism $\sigma:\fX\ra \fY$ of $k$-functors is representable.
\end{proposition}
\begin{proof}
Fix a morphism of $k$-functors $\sigma:\fX\ra \fY$. Let $Y$ be a $k$-scheme and let $\tau:\fP_Y\ra \fY$ be a morphism of $k$-functors. Consider the cartesian square
\begin{center}
\begin{tikzpicture}
[description/.style={fill=white,inner sep=2pt}]
\matrix (m) [matrix of math nodes, row sep=3em, column sep=3em,text height=1.5ex, text depth=0.25ex] 
{  \fZ        & \fX           \\
   \fP_{Y}    & \fY           \\} ;
\path[->,line width=1.0pt,font=\scriptsize]
(m-1-1) edge node[above] {$ \tau' $} (m-1-2)
(m-2-1) edge node[below] {$ \tau $} (m-2-2)
(m-1-2) edge node[right] {$ \sigma $} (m-2-2)
(m-1-1) edge node[left] {$ \sigma'  $} (m-2-1);
\end{tikzpicture}
\end{center}
Then there exists a cartesian square
\begin{center}
\begin{tikzpicture}
[description/.style={fill=white,inner sep=2pt}]
\matrix (m) [matrix of math nodes, row sep=3em, column sep=3em,text height=1.5ex, text depth=0.25ex] 
{  \fZ                  & \fY           \\
   \fP_{Y}\times \fY    & \fY\times \fY           \\} ;
\path[->,line width=1.0pt,font=\scriptsize]
(m-1-1) edge node[above] {$  $} (m-1-2)
(m-2-1) edge node[below] {$ \tau\times \sigma $} (m-2-2)
(m-1-2) edge node[right] {$ \textbf{diagonal} $} (m-2-2)
(m-1-1) edge node[left] {$   $} (m-2-1);
\end{tikzpicture}
\end{center}
Since the diagonal of $\fY$ is representable, we derive that $\fZ$ is isomorphic with functor of points of some $k$-scheme. This finishes the proof.
\end{proof}

\section{Example: Grassmannians}
\noindent
In this section we use abstract results from previous sections to prove the existence of $k$-scheme representing grassmanian $k$-functor (to be defined below). We start by recalling the notion of quotient.

\begin{definition}
Let $\cC$ be a category and let $X$ be an object of $\cC$. Suppose that $f_1:X\twoheadrightarrow X_1$ and $f_2:X\twoheadrightarrow X_2$ are epimorphisms in $\cC$. We say that $f_1$ and $f_2$ are \textit{equivalent} if there exists a commutative triangle 
\begin{center}
\begin{tikzpicture}
[description/.style={fill=white,inner sep=2pt}]
\matrix (m) [matrix of math nodes, row sep=2em, column sep=1em,text height=1.5ex, text depth=0.25ex] 
{     X_1     &   &  X_2        \\
              & X &  \\} ;
\path[->,line width=1.0pt,font=\scriptsize] 
(m-1-1) edge node[above] {$\cong $} (m-1-3);
\path[->>,line width=1.0pt,font=\scriptsize] 
(m-2-2) edge node[auto] {$f_1 $} (m-1-1)
(m-2-2) edge node[below = 7pt, right = 1pt] {$f_2 $} (m-1-3);
\end{tikzpicture}
\end{center}
in $\cC$ in which horizontal arrow is an isomorphism. Class of epimorphisms with domain in $X$ which are equivalent with respect to the relation above is called \textit{a quotient of $X$}. 
\end{definition}

\begin{definition}
Let $V$ be a $k$-module and let $n$ be a positive integer. For $k$-algebra $A$ we define
$$\mathrm{Grass}_{V,n}(A) =\bigg\{    \begin{subarray}{c}
\mbox{Quotients of $A\otimes_kV$ represented by epimorphisms }\\
\mbox{with codomains that are projective $A$-modules of rank $n$}
\end{subarray}\bigg\}$$
 Note that if $f:A\ra B$ is a morphism of $k$-algebras (making $B$ into an $A$-algebra), then the functor $B\otimes_A(-)$ induces the canonical map
$$\mathrm{Grass}_{V,n}(f):\mathrm{Grass}_{V,n}(A)\ra \mathrm{Grass}_{V,n}(B)$$
This makes $\mathrm{Grass}_{V,n}$ into a $k$-functor. We call it \textit{the grassmannian $k$-functor of quotients of rank $n$ of $V$}.
\end{definition}

\begin{theorem}\label{theorem:representability_of_grassmannian}
Let $V$ be a $k$-module and let $n$ be a positive integer. Then the $k$-functor $\mathrm{Grass}_{V,n}$ is representable and if $V$ is finitely generated, then it is represented by a scheme locally of finite type over $k$.
\end{theorem}
\noindent
We start with the following general result.

\begin{lemma}\label{lemma:isomorphism_locus_of_module_morphism}
Let $X$ be a locally ringed space and $\phi:\cP \ra Q$ be a morphism of $\cO_X$-modules such that $\cP$ is of finite type and $Q$ is locally free of finite rank. Then for every point $x$ in $X$ the following assertions are equivalent.
\begin{enumerate}[label=\emph{\textbf{(\roman*)}}, leftmargin=3.0em]
\item $1_{k(x)}\otimes_{\cO_{X,x}}\phi_{x}$ is an isomorphism of vector spaces over $k(x)$.
\item $\phi_{x}$ is an isomorphism of $\cO_{X,x}$-modules.
\end{enumerate}
Moreover, the subset
$$\big\{x\in X\,\big|\,\phi_{x}\mbox{ is an isomorphism of }\cO_{X,x}\mbox{-modules}\big\}$$
of $X$ is open.
\end{lemma}
\begin{proof}[Proof of the lemma]
Suppose that $\cK = \Ker\left(\phi\right),\,\cL = \Coker\left(\phi\right)$. Note first that $\cL$ is $\cO_X$-module of finite type as the homomorphic image of $Q$. Fix a point $x$ in $X$ such that $1_{k(x)}\otimes_{\cO_{X,x}}\phi_{x}$ is an isomorphisms of $k(x)$ vector spaces. This implies that $k(x)\otimes_{\cO_{X,x}}\cL_x = 0$ and hence by Nakayama lemma we derive that $\cL_{x} = 0$. Thus we have a short exact sequence
\begin{center}
\begin{tikzpicture}
[description/.style={fill=white,inner sep=2pt}]
\matrix (m) [matrix of math nodes, row sep=3em, column sep=3em,text height=1.5ex, text depth=0.25ex] 
{  0 & \cK_{x} & \cP_{x} & Q_{x} & 0      \\} ;
\path[->,line width=1.0pt,font=\scriptsize]
(m-1-1) edge node[above] {$  $} (m-1-2)
(m-1-2) edge node[above] {$  $} (m-1-3)
(m-1-3) edge node[above] {$ \phi_{x} $} (m-1-4)
(m-1-4) edge node[above] {$  $} (m-1-5);
\end{tikzpicture}
\end{center}
Facts that $Q_{x}$ is finitely presented and $\cP_{x}$ is finitely generated over $\cO_{X,x}$ imply that $\cK_{x}$ is finitely generated over $\cO_{X,x}$. Since $Q_{x}$ is free, we derive that the sequence above is split exact. Therefore, also the sequence
\begin{center}
\begin{tikzpicture}
[description/.style={fill=white,inner sep=2pt}]
\matrix (m) [matrix of math nodes, row sep=3em, column sep=3.5em,text height=1.5ex, text depth=0.25ex] 
{  0 & k(x)\otimes_{\cO_{X,x}}\cK_{x} & k(x)\otimes_{\cO_{X,x}}\cP_{x} & k(x)\otimes_{\cO_{X,x}}Q_{x} & 0      \\} ;
\path[->,line width=1.0pt,font=\scriptsize]
(m-1-1) edge node[above] {$  $} (m-1-2)
(m-1-2) edge node[above] {$  $} (m-1-3)
(m-1-3) edge node[above] {$ 1_{k(x)}\otimes_{\cO_{X,x}}\phi_{x} $} (m-1-4)
(m-1-4) edge node[above] {$  $} (m-1-5);
\end{tikzpicture}
\end{center}
is exact and hence $k(x)\otimes_{\cO_{X,x}}\cK_{x} = 0$. Nakayama lemma implies that $\cK_{x} = 0$. Thus we derive that $1_{k(x)}\otimes_{\cO_{X,x}}\phi_{x}$ is an isomorphisms of $k(x)$ vector spaces if and only if $\phi_{x}$ is an isomorphisms of $\cO_{X,x}$-modules. In other words
$$\big\{x\in X\,\big|\,1_{k(x)}\otimes_{\cO_{X,x}}\phi_{x}\mbox{ is an isomorphism of vector spaces over }k(x)\big\} = $$
$$= \big\{x\in X\,\big|\,\phi_{x}\mbox{ is an isomorphism of }\cO_{X,x}\mbox{-modules}\big\}$$
Note that
$$\big\{x\in X\,\big|\,\phi_{x}\mbox{ is an isomorphism of }\cO_{X,x}\mbox{-modules}\big\}\subseteq \big\{x\in X\,\big|\,\phi_{x}\mbox{ is an epimorphism of }\cO_{X,x}\mbox{-modules}\big\}$$
and
$$\big\{x\in X\,\big|\,\phi_{x}\mbox{ is an epimorphism of }\cO_{X,x}\mbox{-modules}\big\} = X\setminus \mathrm{supp}(\cL)$$
Since $\cL$ is finitely generated, we derive that $\mathrm{supp}(\cL)$ is closed and $X\setminus \mathrm{supp}(\cL)$ is open. Now there is a short exact sequence
\begin{center}
\begin{tikzpicture}
[description/.style={fill=white,inner sep=2pt}]
\matrix (m) [matrix of math nodes, row sep=3em, column sep=3em,text height=1.5ex, text depth=0.25ex] 
{  0 & \cK_{\mid X\setminus \mathrm{supp}(\cL)} & \cP_{\mid X\setminus \mathrm{supp}(\cL)} & Q_{\mid X\setminus \mathrm{supp}(\cL)} & 0      \\} ;
\path[->,line width=1.0pt,font=\scriptsize]
(m-1-1) edge node[above] {$  $} (m-1-2)
(m-1-2) edge node[above] {$  $} (m-1-3)
(m-1-3) edge node[above] {$ \phi_{\mid X\setminus \mathrm{supp}(\cL)} $} (m-1-4)
(m-1-4) edge node[above] {$  $} (m-1-5);
\end{tikzpicture}
\end{center}
It follows that $\cK_{\mid X\setminus \mathrm{supp}(\cL)}$ is $\cO_X$-module of finite type. Thus
$$\big\{x\in X\,\big|\,\phi_{x}\mbox{ is an isomorphism of }\cO_{X,x}\mbox{-modules}\big\} = \left(X\setminus \mathrm{supp}(\cL)\right)\setminus \mathrm{supp}(\cK_{\mid X\setminus \mathrm{supp}(\cL)})$$
is an open subset of $X$.
\end{proof}
\noindent
Let $V$ be a free $k$-module and let $n$ be a positive integer. Consider a morphism $u:k^{\oplus n} \ra V$ of $k$-modules. Now we define a $k$-subfunctor $\mathrm{Grass}_{V}^u$ of $\mathrm{Grass}_{V,n}$ by formula
$$\mathrm{Grass}_{V}^u(A) =\bigg\{\begin{subarray}{c}
\mbox{Elements of $\mathrm{Grass}_{V,n}(A)$ which are represented by epimorphisms $\phi:A\otimes_kV\ra U$}\\
\mbox{such that the composition $\phi\cdot \left(1_A\otimes_k u\right)$ is an isomorphism}
\end{subarray}\bigg\}$$
for every $k$-algebra. Next we proceed in steps.

\begin{lemma}\label{lemma:standard_open_cover_of_grassmannian}
Let $V$ be a free $k$-module and let $n$ be a positive integer. Then
$$\big\{\mathrm{Grass}_{V}^u\hookrightarrow \mathrm{Grass}_{V,n} \big\}_{u\in \Hom_k\left(k^{\oplus n},V\right)}$$
is an open cover of $\mathrm{Grass}_{V,n}$.
\end{lemma}
\begin{proof}[Proof of the lemma]
Let $A$ be a $k$-algebra. Consider a morphism $\tau:\fP_{\Spec A}\ra \mathrm{Grass}_{V,n}$ that corresponds to some quotient of $A\otimes_kV$ that is represented by an epimorphism $\phi:A\otimes_kV\ra U$ of $A$-modules with projective $A$-module $U$ of rank $n$. Let $u:k^{\oplus n}\ra V$ be a morphism of $k$-modules. Consider a cartesian square
\begin{center}
\begin{tikzpicture}
[description/.style={fill=white,inner sep=2pt}]
\matrix (m) [matrix of math nodes, row sep=3em, column sep=3em,text height=1.5ex, text depth=0.25ex] 
{  \fX        & \mathrm{Grass}^u_V           \\
   \fP_{\Spec A}             & \mathrm{Grass}_{V,n}           \\} ;
\path[->,line width=1.0pt,font=\scriptsize]
(m-1-1) edge node[above] {$  $} (m-1-2)
(m-2-1) edge node[below] {$ \tau $} (m-2-2);
\path[right hook->,line width=1.0pt,font=\scriptsize]
(m-1-1) edge node[left]  {$  $} (m-2-1)
(m-1-2) edge node[right] {$  $} (m-2-2);
\end{tikzpicture}
\end{center}
Pick a $k$-algebra $B$ and a morphism $f:A\ra B$ of $k$-algebras. Note that $f$ makes $B$ into an $A$-algebra. Then $f\in \fX(B)$ if and only if $\left(1_B\otimes_A\phi\right)\cdot \left(1_B\otimes_ku\right)$ is an isomorphism of $B$-modules. Thus by Lemma \ref{lemma:isomorphism_locus_of_module_morphism} we deduce that $f\in \fX(B)$ if and only if $\Spec f:\Spec B\ra \Spec A$ factors through an open subscheme
$$W_u = \bigg\{\ideal{q}\in \Spec A\,\bigg|\,\big(\phi\cdot \left(1_A\otimes_ku\right)\big)_{\ideal{q}}\mbox{ is an isomorphism of }A_{\ideal{q}}\mbox{-modules}\bigg\} =$$
$$= \bigg\{\ideal{q}\in \Spec A\,\bigg|\,k(\ideal{q})\otimes_{A_{\ideal{q}}}\big(\phi\cdot \left(1_A\otimes_ku\right)\big)_{\ideal{q}}\mbox{ is an isomorphism of }k(\ideal{q})\mbox{-vector spaces}\bigg\}$$
This implies that $\fX \hookrightarrow \fP_{\Spec A}$ is isomorphic to an open immersion $\fP_{W_u}\hookrightarrow \fP_{\Spec A}$.\\
Pick now $\ideal{q}\in \Spec A$ and an epimorphism $\theta:k^{\oplus I}\twoheadrightarrow V$ for some set $I$. Then there exist $J\subseteq I$ with $\bd{card}(J) = n$ such that the restriction to $k(\ideal{q})^{\oplus J}$ of the morphism
$$1_{k(\ideal{q})}\otimes_{A_{\ideal{q}}}\big(\phi\cdot \left(1_A\otimes \theta\right)\big)_{\ideal{q}}:k(\ideal{q})^{\oplus I}\ra k(\ideal{q})\otimes_{A_{\ideal{q}}}U_{\ideal{q}}$$
is an isomorphism of $k(\ideal{q})$-vector spaces. Let $u:k^{\oplus n}\ra V$ be a morphism given as the composition of the canonical monomorphism $k^{\oplus n} = k^{\oplus J}\hookrightarrow k^{\oplus I}$ with $\theta$. Then
$$\bigg(1_{k(\ideal{q})}\otimes_{A_{\ideal{q}}}\big(\phi\cdot \left(1_A\otimes u\right)\big)_{\ideal{q}}\bigg)$$
is an isomorphism of $k(\ideal{q})$-vector spaces. Note that module $U_{\ideal{q}}$ is a free $A_{\ideal{q}}$-module of rank $n$. Hence by Lemma \ref{lemma:isomorphism_locus_of_module_morphism} we derive that
$$\big(\phi\cdot \left(1_A\otimes u\right)\big)_{\ideal{q}}$$
is an isomorphism of $A_{\ideal{q}}$-modules. Thus $\ideal{q} \in W_u$. Since $\ideal{q}$ is arbitrary, we deduce that
$$\Spec A = \bigcup_{u\in \Hom_{k}(k^{\oplus n},V)}W_u$$
This finishes the proof.
\end{proof}

\begin{lemma}\label{lemma:representability_of_open_chart}
Let $V$ be a $k$-module and let $n$ be a positive integer. Suppose that $u:k^{\oplus n}\ra V$ is a morphism of $k$-modules. Then $\mathrm{Grass}_{V}^u$ is representable by a $k$-scheme. Moreover, if $V$ is finitely generated over $k$, then it is represented by a $k$-scheme of finite type.
\end{lemma}
\begin{proof}[Proof of the lemma]
Pick $k$-algebra $A$ and let $\phi:A\otimes_kV\twoheadrightarrow U$ be a morphism of $A$-modules that represents an element of $\mathrm{Grass}_{V}^u(A)$. Let $v$ be an inverse of $A$-module isomorphism $\phi\cdot \left(1_A\otimes_ku\right)$. Then $\theta = v\cdot \phi:A\otimes_kV\twoheadrightarrow A^{\oplus n}$ represents the same element of $\mathrm{Grass}_{V}^u(A)$ as $\phi$. Moreover, it is a unique representative of this element having the property that $\theta \cdot \left(1_A\otimes_k u\right) = 1_{A^{\oplus n}}$. Thus we derive that
$$\mathrm{Grass}_{V}^u(A) = \big\{\theta:A\otimes_kV\twoheadrightarrow A^{\oplus n}\,\big|\,\theta\cdot \left(1_A\otimes_ku\right) = 1_{A^{\oplus n}}\big\}$$
is natural identification. Note that for every $k$-algebra we have a cartesian square of sets
\begin{center}
\begin{tikzpicture}
[description/.style={fill=white,inner sep=2pt}]
\matrix (m) [matrix of math nodes, row sep=3em, column sep=5.5em,text height=1.5ex, text depth=0.25ex] 
{ \mathrm{Grass}_{V}^u(A)              & \bd{1}\\
  \Hom_A\left(A\otimes_k V,A^{\oplus n}\right)    & \Hom_A\left(A^{\oplus n},A^{\oplus n}\right) \\} ;
\path[->,line width=1.0pt,font=\scriptsize]
(m-1-1) edge node[above] {$  $} (m-1-2)
(m-2-1) edge node[below] {$ \Hom_A\left(u,1_{A^{\oplus n}}\right) $} (m-2-2);
\path[right hook->,line width=1.0pt,font=\scriptsize]
(m-1-1) edge node[left]  {$  $} (m-2-1)
(m-1-2) edge node[right] {$1_{A^{\oplus n}} $} (m-2-2);
\end{tikzpicture}
\end{center}
These cartesian squares induce cartesian square of $k$-functors. Thus it suffices to prove that $k$-functors
$$\Alg_k\ni A\mapsto \Hom_A\left(A\otimes_kV,A^{\oplus n}\right)\in \Set,\,\Alg_k\ni A\mapsto \Hom_A\left(A^{\oplus n},A^{\oplus n}\right)\in \Set$$
are representable. For this it suffices to prove that for every $k$-module $W$ a $k$-functor
$$\Alg_k\ni A\mapsto \Hom_A\left(A\otimes_kW,A^{\oplus n}\right)\in \Set$$
Indeed, we have chain of bijections
$$\fP_{\underbrace{\Spec \Sym(W)\times_k ... \times_k \Spec \Sym(W)}_{n\,\mathrm{times}}}\left(A\right) \cong \Mor_k\left(\Spec A, \Spec \Sym(W)\right)^n \cong$$
$$\cong \Hom_k\left(W,A\right)^n \cong \Hom_k\left(W, A^{\oplus n}\right) \cong \Hom_A\left(A\otimes_k W,A^{\oplus n}\right)$$
natural in $k$-algebra $A$. Note that if $W$ is finitely generated, then
$$\underbrace{\Spec \Sym(W)\times_k ... \times_k \Spec \Sym(W)}_{n\,\mathrm{times}}$$
is of finite type over $k$.
\end{proof}

\begin{lemma}\label{lemma:sheaf_property_for_grassmannian}
Let $X$ be a locally ringed space, let $\cF$ be a sheaf of $\cO_X$-modules and let $\{U_i\}_{i \in I}$ be an open cover of $X$. Fix a positive integer $n$. Suppose that for each $i\in I$ there is an epimorphism $\phi_i:\cF_{\mid U_i}\twoheadrightarrow \cO_{U_i}^{\oplus n}$ such that for any  $i,j\in I$ there exists a commutative triangle
\begin{center}
\begin{tikzpicture}
[description/.style={fill=white,inner sep=2pt}]
\matrix (m) [matrix of math nodes, row sep=3em, column sep=1em,text height=1.5ex, text depth=0.25ex] 
{     \cO_{U_i\cap U_j}^{\oplus n}     &   & \cO_{U_i\cap U_j}^{\oplus n}         \\
                        & \cF_{\mid U_i\cap U_j} &  \\} ;
\path[densely dotted,->,line width=1.0pt,font=\scriptsize] 
(m-1-1) edge node[above] {$\cong $} (m-1-3);
\path[->>,line width=1.0pt,font=\scriptsize] 
(m-2-2) edge node[auto] {$ {\phi_j}_{\mid U_i\cap U_j} $} (m-1-1)
(m-2-2) edge node[below = 7pt, right = 1pt] {$ {\phi_i}_{\mid U_i\cap U_j} $} (m-1-3);
\end{tikzpicture}
\end{center}
with dotted arrow that is an isomorphism of $\cO_{U_i\cap U_j}$-modules. Then there exists a locally free sheaf $\cU$ of rank $n$ and an epimorphism $\phi:\cF \twoheadrightarrow \cU$ such that there are commutative triangles
\begin{center}
\begin{tikzpicture}
[description/.style={fill=white,inner sep=2pt}]
\matrix (m) [matrix of math nodes, row sep=3em, column sep=1em,text height=1.5ex, text depth=0.25ex] 
{         \cU_{\mid U_i} &   & \cO_{U_i}^{\oplus n}         \\
                        & \cF_{\mid U_i} &  \\} ;
\path[densely dotted,->,line width=1.0pt,font=\scriptsize] 
(m-1-1) edge node[above] {$ \psi_i $} (m-1-3);
\path[->>,line width=1.0pt,font=\scriptsize] 
(m-2-2) edge node[auto] {$ {\phi}_{\mid U_i} $} (m-1-1)
(m-2-2) edge node[below = 7pt, right = 1pt] {$ \phi_i $} (m-1-3);
\end{tikzpicture}
\end{center}
with isomorphisms $\psi_i$ for every $i\in I$. Moreover, $\phi$ with these properties determine a unique quotient of $\cF$.
\end{lemma}
\begin{proof}[Proof of the lemma]
By assumption for every pair $i,j\in I$ there exists an isomorphism $\theta_{ij}$ such that the triangle
\begin{center}
\begin{tikzpicture}
[description/.style={fill=white,inner sep=2pt}]
\matrix (m) [matrix of math nodes, row sep=3em, column sep=1em,text height=1.5ex, text depth=0.25ex] 
{     \cO_{U_i\cap U_j}^{\oplus n}     &   & \cO_{U_i\cap U_j}^{\oplus n}         \\
                        & \cF_{\mid U_i\cap U_j} &  \\} ;
\path[densely dotted,->,line width=1.0pt,font=\scriptsize] 
(m-1-1) edge node[above] {$\theta_{ij} $} (m-1-3);
\path[->>,line width=1.0pt,font=\scriptsize] 
(m-2-2) edge node[auto] {$ {\phi_j}_{\mid U_i\cap U_j} $} (m-1-1)
(m-2-2) edge node[below = 7pt, right = 1pt] {$ {\phi_i}_{\mid U_i\cap U_j} $} (m-1-3);
\end{tikzpicture}
\end{center}
is commutative. Note that $\theta_{ij}$ is a unique isomorphism that makes the triangle commutative. Thus $\{\theta_{ij}\}_{i,j\in I}$ satisfy cocycle condition. Hence there exists a unique locally free sheaf $\cU$ of rank $n$ with $\{\theta_{ij}\}_{i,j\in I}$ as the family of transition isomorphisms. Moreover, $\{\phi_i\}_{i\in I}$ induce an epimorphism $\phi:\cF\twoheadrightarrow \cU$. This constructs $\phi$ with properties as in the statement.
\end{proof}

\begin{proof}[Proof of the theorem]
By Lemma \ref{lemma:sheaf_property_for_grassmannian} we derive that $\mathrm{Grass}_{V.n}$ is a Zariski local $k$-functor. Now the theorem follows from Theorem \ref{theorem:representability_basic_result} in the light of Lemma \ref{lemma:standard_open_cover_of_grassmannian} and Lemma \ref{lemma:representability_of_open_chart}.
\end{proof}

\begin{definition}
Let $V$ be a $k$-module and let $n$ be a positive integer. Then a $k$-scheme $\bd{Gr}(V,n)$ that represents the $k$-functor $\mathrm{Grass}_{V,n}$ is called \textit{the grassmannian scheme of rank $n$ quotients of $V$}.
\end{definition}

\begin{theorem}\label{theorem:separatedness_of_grassmannian}
Let $V$ be a $k$-module and let $n$ be a positive integer. Then the grassmannian $k$-scheme $\bd{Gr}(V,n)$ is separated over $k$.
\end{theorem}
\noindent
For the proof we need one easy result.

\begin{lemma}\label{lemma:representability_of_vanishing_locus}
Let $A$ be a commutative ring and let $\psi:W\ra U$ be a morphism of $A$-modules. Suppose that $U$ is projective and finitely generated. Let $\fX$ be an $k$-subfunctor of $\fP_{\Spec A}$ such that
$$\fX(B) = \big\{f:A\ra B\,\big|\,1_B\otimes_A\psi = 0\big\}$$
Then $\fX$ is represented by a closed subscheme of $\Spec A$.
\end{lemma}
\begin{proof}[Proof of the lemma]
Since every projective and finitely generated $A$-module is a direct summand of a finitely generated and free $A$-module, we may assume that $U = A^{\oplus n}$ for some positive integer $n$. Let $p_i:A^{\oplus n}\ra A$ be the projection on $i$-th factor for $1\leq i\leq n$. Then $\fX$ is represented by the vanishing locus of
$$\ideal{a} = \sum_{i=1}^n p_i\left(\psi\left(W\right)\right) \subseteq \Spec A$$
\end{proof}

\begin{proof}[Proof of the theorem]
For the proof it suffices to show that the diagonal
$$\mathrm{Grass}_{V,n}\hookrightarrow \mathrm{Grass}_{V,n}\times \mathrm{Grass}_{V,n}$$
is a closed immersion of $k$-functors. Consider a $k$-algebra $A$ and let morphism $\tau:\fP_{\Spec A}\ra \mathrm{Grass}_{V,n} \times \mathrm{Grass}_{V,n}$ of $k$-functors be determined by a pair of quotients of $A\otimes_kV$ given by
$$\big(\phi_1:A\otimes_kV_1\twoheadrightarrow U_1,\phi_2:A\otimes_kV\twoheadrightarrow U_2\big)$$
Consider a cartesian square
\begin{center}
\begin{tikzpicture}
[description/.style={fill=white,inner sep=2pt}]
\matrix (m) [matrix of math nodes, row sep=3em, column sep=3em,text height=1.5ex, text depth=0.25ex] 
{  \fX                       & \mathrm{Grass}_{V,n}           \\
   \fP_{\Spec A}             & \mathrm{Grass}_{V,n}\times \mathrm{Grass}_{V,n}           \\} ;
\path[->,line width=1.0pt,font=\scriptsize]
(m-1-1) edge node[above] {$  $} (m-1-2)
(m-2-1) edge node[below] {$ \tau $} (m-2-2);
\path[right hook->,line width=1.0pt,font=\scriptsize]
(m-1-1) edge node[left]  {$  $} (m-2-1)
(m-1-2) edge node[right] {$\bd{diagonal}  $} (m-2-2);
\end{tikzpicture}
\end{center}
Out goal is to show that the left hand side vertical arrow is a closed immersion. Suppose that $K_i = \Ker(\phi_i)$ and $v_i:K_i\hookrightarrow A\otimes_kV$ is the canonical inclusion for $i=1,2$. Note that the short exact sequence
\begin{center}
\begin{tikzpicture}
[description/.style={fill=white,inner sep=2pt}]
\matrix (m) [matrix of math nodes, row sep=3em, column sep=3em,text height=1.5ex, text depth=0.25ex] 
{  0 & K_i & A\otimes_kV & U_i & 0      \\} ;
\path[->,line width=1.0pt,font=\scriptsize]
(m-1-1) edge node[above] {$  $} (m-1-2)
(m-1-2) edge node[above] {$ v_i $} (m-1-3)
(m-1-3) edge node[above] {$ \phi_i  $} (m-1-4)
(m-1-4) edge node[above] {$  $} (m-1-5);
\end{tikzpicture}
\end{center}
is split exact for $i=1,2$. Indeed, this follows from the fact that $U_i$ is projective for $i=1,2$. Thus for every $A$-algebra $B$ given by a morphism of $k$-algebras $f:A\ra B$ we have $f\in \fX(B)$ if and only if $B\otimes_AK_1$ and $B\otimes_AK_2$ are isomorphic as a subobjects of $B\otimes_kV$. Note that this holds precisely if and only if $1_B\otimes_A\left(\phi_1\cdot v_2\right) = 0$ and $1_B\otimes_A\left(\phi_2\cdot v_1\right)$ because these equalities are equivalent with
$$B\otimes_AK_1 \subseteq B\otimes_AK_2,\,B\otimes_AK_2\subseteq B\otimes_AK_1$$
Lemma \ref{lemma:representability_of_vanishing_locus} implies that $k$-functors $\fX_1,\fX_2$ given by
$$\fX_1(B) = \big\{f:A\ra B\,\big|\,1_B\otimes_A\left(\phi_1\cdot v_2\right) = 0\big\},\,\fX_2(B) =  \big\{f:A\ra B\,\big|\,1_B\otimes_A\left(\phi_2\cdot v_1\right) = 0\big\}$$
are represented by closed subschemes of $\Spec A$. Clearly $\fX$ is an intersection of $\fX_1$ and $\fX_2$ inside $\fP_{\Spec A}$ and hence this $k$-functor is represented by a closed subscheme of $\Spec A$.
\end{proof}

\section{Closed immersions and hom $k$-functors}

\begin{definition}
Let $X$ be a $k$-scheme. Suppose that there exists an open affine cover $X = \bigcup_{i\in I}X_i$ such that $k$-algebra $\Gamma(X_i,\cO_{X_i})$ is free as a $k$-module. Then we say that $X$ is \textit{a locally free $k$-scheme}.
\end{definition}
\noindent
Next theorem is the main result of this section.

\begin{theorem}\label{theorem:closed_immersions_and_internal_hom}
Let $j:\fY'\ra \fY$ be a closed immersion of $k$-functors and $X$ be a locally free $k$-scheme. Suppose that classes $\Mor_A\left((\fP_X)_A,\fY_A\right)$ are sets for every $k$-algebra $A$. Then classes $\Mor_A\left((\fP_X)_A,\fY'_A\right)$ are sets for every $k$-algebra $A$ and the morphism
$$\iMor_k\left(1_{\fP_X},j\right):\iMor_k\left(\fP_X,\fY'\right)\ra \iMor_k\left(\fP_X,\fY\right)$$
is a closed immersion of $k$-functors.
\end{theorem}
\noindent
It is useful to isolate crucial steps in the argument. For this we proceed by proving some lemmas.

\begin{lemma}\label{lemma:for_affine_local_factorization}
Suppose that $A$ is a commutative ring. Let $j:\fY'\ra \fY$ be a closed immersion of $A$-functors and $X$ be an affine $A$-scheme such that $\Gamma(X,\cO_X)$ is a free $A$-module. Assume that $\tau:\fP_X\ra \fY$ is a morphism of $A$-functors. Then there exists an ideal $\ideal{a}\subseteq A$ such that for every $A$-algebra $B$ the restriction $\tau_B$ factors through $j_B$ if and only if the structure morphism $f:A\ra B$ of $B$ satisfies $\ideal{a}\subseteq \ker(f)$.
\end{lemma}
\begin{proof}[Proof of the lemma]
Since $j$ is a closed immersion of $A$-functors and $X$ is affine $k$-scheme there exists an affine $A$-scheme $X'$, a closed immersion $j':X'\ra X$ of schemes and a cartesian square
\begin{center}
\begin{tikzpicture}
[description/.style={fill=white,inner sep=2pt}]
\matrix (m) [matrix of math nodes, row sep=3em, column sep=3em,text height=1.5ex, text depth=0.25ex] 
{  \fP_{X'} & \fY' \\
   \fP_X  & \fY           \\} ;
\path[->,line width=1.0pt,font=\scriptsize]  
(m-1-1) edge node[above] {$ $} (m-1-2)
(m-2-1) edge node[below] {$\tau  $} (m-2-2)
(m-1-1) edge node[left] {$ \fP_{j'} $} (m-2-1)
(m-1-2) edge node[right] {$j $} (m-2-2);
\end{tikzpicture}
\end{center}
of $A$-functors. Next let $B$ be an $A$-algebra with the structure morphism $f:A\ra B$. Then $\tau_B$ factors through $j_B$ if and only if the projection $\Spec B\times_{\Spec A}X\ra X$ induced by $f$ factors through $X'$. Let $A[X]$ be the $A$-algebra of global regular functions on $X$ and let $\ideal{J}$ be an ideal in $A[X]$ such that $A[X]/\ideal{J} = A[X']$ is the $A$-algebra of global regular functions of $X'$. With this notation we derive that the projection $\Spec B\times_{\Spec A}X\ra X$ induced by $f$ factors through $X'$ if and only if the morphism $A[X]\ra B\otimes_AA[X]$ induced by $f$ sends every element of $\ideal{J}$ to zero. Since $A[X]$ is a free $A$-module, we write $A[X] = A^{\oplus I}$ for some index set $I$. Then the morphism $A[X]\ra B\otimes_AA[X]$ induced by $f$ is just $f^{\oplus I}:A^{\oplus I}\ra B^{\oplus I}$. We have $f^{\oplus I}\left(\ideal{J}\right)=0$ if and only if $\left(pr^B_i\cdot f^{\oplus I}\right)\left(\ideal{J}\right)=$ for every $i\in I$, where $pr^B_i:B^{\oplus I}\ra B$ is the projection on $i$-th component. Pick $i\in I$ and consider the commutative diagram
\begin{center}
\begin{tikzpicture}
[description/.style={fill=white,inner sep=2pt}]
\matrix (m) [matrix of math nodes, row sep=3em, column sep=3em,text height=1.5ex, text depth=0.25ex] 
{  A^{\oplus I} & B^{\oplus I}  \\
   A  & B           \\} ;
\path[->,line width=1.0pt,font=\scriptsize]  
(m-1-1) edge node[above] {$ f^{\oplus I} $} (m-1-2)
(m-2-1) edge node[below] {$ f  $} (m-2-2)
(m-1-1) edge node[left] {$ pr^A_i $} (m-2-1)
(m-1-2) edge node[right] {$ pr^B_i $} (m-2-2);
\end{tikzpicture}
\end{center}
In the diagram $pr^A_i$ is the projection on $i$-th component. Diagram implies that $\left(pr^B_i\cdot f^{\oplus I}\right)\left(\ideal{J}\right)=$ for every $i\in I$ if and only if $\left(f\cdot pr_i^A\right)(\ideal{J}) = 0$ for every $i\in I$. This is equivalent with the condition that $f(\ideal{a})=0$ for ideal $\ideal{a}$ in $A$ generated by $\sum_{i\in I}pr_i^A(\ideal{J})$. Thus the lemma is proved.
\end{proof}

\begin{lemma}\label{lemma:covers_and_factorizations}
Suppose that $A$ is a commutative ring. Let $j:\fY'\ra \fY$ be a closed immersion of $A$-functors and $X$ be an $A$-scheme with open cover
$$X=\bigcup_{i\in I}X_i$$
Assume that $\tau:\fP_X\ra \fY$ is a morphism of $A$-functors. Fix an $A$-algebra $B$. Then $\tau_B$ factors through $j_B$ if and only if $\left(\tau_{\mid \fP_{X_i}}\right)_B$ factors through $j_B$ for every $i\in I$.
\end{lemma}
\begin{proof}[Proof of the lemma]
If $\tau_B$ factors through $j_B$, then also $\left(\tau_{\mid \fP_{X_i}}\right)_B$ factors through $j_B$ for every $i\in I$. It suffices to prove the converse. So suppose that $\left(\tau_{\mid \fP_{X_i}}\right)_B$ factors through $j_B$ for every $i\in I$. Since $j$ is a closed immersion of $A$-functors and $X$ is an $A$-scheme, Proposition \ref{proposition:open_closed_immersions} implies that there exists a cartesian square
\begin{center}
\begin{tikzpicture}
[description/.style={fill=white,inner sep=2pt}]
\matrix (m) [matrix of math nodes, row sep=3em, column sep=3em,text height=1.5ex, text depth=0.25ex] 
{  \fP_{X'} & \fY' \\
   \fP_{X}  & \fY           \\} ;
\path[->,line width=1.0pt,font=\scriptsize]  
(m-1-1) edge node[above] {$ $} (m-1-2)
(m-2-1) edge node[below] {$\tau  $} (m-2-2)
(m-1-1) edge node[left] {$ \fP_{j'} $} (m-2-1)
(m-1-2) edge node[right] {$ j $} (m-2-2);
\end{tikzpicture}
\end{center}
where $j':X'\ra X$ is a closed immersion of $A$-schemes. For each $i\in I$ let $j'_i:j'^{-1}(X_i)\ra X_i$ be the restriction of $j'$. We have the induced cartesian square
\begin{center}
\begin{tikzpicture}
[description/.style={fill=white,inner sep=2pt}]
\matrix (m) [matrix of math nodes, row sep=3em, column sep=3em,text height=1.5ex, text depth=0.25ex] 
{  \fP_{j'^{-1}(X_i)} & \fY' \\
   \fP_{X_i}  & \fY           \\} ;
\path[->,line width=1.0pt,font=\scriptsize]  
(m-1-1) edge node[above] {$ $} (m-1-2)
(m-2-1) edge node[below] {$\tau_{\mid \fP_{X_i}}  $} (m-2-2)
(m-1-1) edge node[left] {$ \fP_{j'_i}  $} (m-2-1)
(m-1-2) edge node[right] {$ j $} (m-2-2);
\end{tikzpicture}
\end{center}
Now $\left(\tau_{\mid \fP_{X_i}}\right)_B$ factors through $j_B$. This implies that $(\fP_{j'_i})_B$ admits a section for every $i\in I$. Then $(\fP_{j'_i})_B$ is an isomorphism for every $i\in I$. Thus $j'_i\times_{\Spec A}1_{\Spec B}$ is an isomorphism for every $i\in I$ and hence $j'\times_{\Spec A}1_{\Spec B}$ is an isomorphism of $B$-schemes. This means that $\tau_B$ factors through $j_B$.
\end{proof}

\begin{proof}[Proof of the theorem]
Let $A$ be a $k$-algebra. The restriction functor $(-)_{\mid \Alg_A} = (-)_A$ preserves all closed immersions. Thus $j_A$ is a closed immersion of $A$-functors and hence we derive that $j_A:\fY'_A\ra \fY_A$ is a monomorphism of $A$-functors. Thus we have an injective  map of classes
$$\Mor_A\left(1_{(\fP_X)_A},j_A\right):\Mor_A\left((\fP_X)_A,\fY'_A\right)\hookrightarrow \Mor_A\left((\fP_X)_A,\fY_A\right)$$
Hence if $\Mor_A\left((\fP_X)_A,\fY_A\right)$ is a set, then $\Mor_A\left((\fP_X)_A,\fY'_A\right)$ is a set. All these facts imply that both internal homs
$$\iMor_k\left(\fP_X,\fY'\right),\,\iMor_k\left(\fP_X,\fY\right)$$
exist and morphism $\iMor_k(1_{\fP_X},j)$ of $k$-functors is a monomorphism. Our task is to prove that it is a closed immersion. For this consider a $k$-algebra $A$ and a morphism $\sigma:\fP_{\Spec A}\ra \iMor_k\left(\fP_X,\fY\right)$ of $k$-functors that sends $1_A$ to some morphism $\tau:(\fP_X)_A\ra \fY_A$ of $A$-functors. Consider a cartesian square
\begin{center}
\begin{tikzpicture}
[description/.style={fill=white,inner sep=2pt}]
\matrix (m) [matrix of math nodes, row sep=3em, column sep=3em,text height=1.5ex, text depth=0.25ex] 
{  \fU  & \iMor_k\left(\fP_X,\fY'\right) \\
   \fP_{\Spec A}  & \iMor_k\left(\fP_X,\fY\right)           \\} ;
\path[->,line width=1.0pt,font=\scriptsize]  
(m-1-1) edge node[above] {$ $} (m-1-2)
(m-2-1) edge node[below] {$ \sigma $} (m-2-2)
(m-1-1) edge node[left] {$  $} (m-2-1)
(m-1-2) edge node[right] {$ \iMor_k\left(1_{\fP_X},j\right) $} (m-2-2);
\end{tikzpicture}
\end{center}
Since $\iMor_k\left(1_{\fP_X},j\right)$ is a monomorphism, we may consider $\fU$ as a $k$-subfunctor of $\fP_{\Spec A}$. For every $k$-algebra $B$ subset $\fU(B)\subseteq \Mor_k(A,B)= \Mor_k\left(\Spec B,\Spec A\right)$ consists of $A$-algebras $B$ with structure morphisms $f:A\ra B$ such that $\tau_B$ factors through $j_B:\fY'_B\ra \fY_B$. Since $X$ is a locally free $k$-scheme, we deduce that $(\fP_{X})_A$ is a functor of points of a locally free $A$-scheme
$$\Spec A\times_{\Spec k}X$$
Pick an open affine cover $\bigcup_{i\in I}X_i$ of this $A$-scheme such that $\Gamma(X_i,\cO_X)$ is a free $A$-module. Now Lemma \ref{lemma:covers_and_factorizations} implies that $\tau_B$ factors through $j_B$ if and only if $\left(\tau_{\mid X_i}\right)_B$ factors through $j_B$ for every $i\in I$. Next by Lemma \ref{lemma:for_affine_local_factorization} we deduce that $\left(\tau_{\mid X_i}\right)_B$ factors through $j_B$ for given $i\in I$ if and only if $f(\ideal{a}_i)=0$ for some ideal $\ideal{a}_i\subseteq A$ independent of $f$. Thus $\fU$ consists of all morphisms $f:A\ra B$ of $k$-algebras such that $f(\ideal{a})=0$ where $\ideal{a} = \sum_{i\in I}\ideal{a}_i$. Therefore, $\fU\hookrightarrow \fP_{\Spec A}$ is isomorphic with $\fP_{V(\ideal{a})} = \fP_{\Spec A/\ideal{a}}\hookrightarrow \fP_{\Spec A}$ induced by the quotient map $A\ra A/\ideal{a}$ and hence $\iMor_k(1_{\fP_X},j)$ is a closed immersion of $k$-functors.
\end{proof}







































\small
\bibliographystyle{alpha}
\bibliography{../zzz}


\end{document}