\input ../pree.tex

\begin{document}

\title{Completeness of first order logic}
\date{}
\maketitle

\section{Introduction}
\noindent
This notes are devoted to study classical first order logic and its completeness and were influenced by presentations in \cite{enderton2001mathematical} and \cite{lyndon1966notes}. We start by defining first order languages, their terms and formulas and proving unique readability theorems for these parts of syntax. Next we introduce classical semantics for first order languages and a deductive system for classical first order logic. Our deductive system is of Hilbert’s type. There are other deductive systems for classical predicate logic. They are all equivalent, but have different properties from the point of view of automatic theorem proving. There is a whole branch of logic devoted to study of such systems called proof theory. After introduction of semantics and deductive system we are able to state and prove soundness theorem. Next we investigate some fundamental metaproperties of the deductive system and introduce the notion of consistent sets of formulas. Then we prove Henkin’s theorem on existence of models for consistent sets of formulas. The proof is taken from the Lyndon’s book \cite{lyndon1966notes}, but more details are provided. Next we prove completeness theorem due to Gödel. As a corollary we deduce standard compactness theorem.

\section{First order languages and their syntax}

\begin{definition}
\textit{A signature} consists of the following data. A set $R$ of \textit{relation symbols} equipped with a decomposition
$$R = \coprod_{n\in \NN,n\geq 1}R_n$$
where elements of $R_n$ are called \textit{$n$-ary relation symbols} and of a set $F$ of function symbols equipped with a decomposition
$$F = \coprod_{n\in \NN}F_n$$
where elements of $F_n$ are called \textit{$n$-ary function symbols}. Elements of $F_0$ are called \textit{constants}.
\end{definition}

\begin{definition}
\textit{A first order language $\cL$} consists of the following data. A signature called \textit{the signature of $\cL$} and an infinite set $\bd{V}$ called \textit{the set of variables of $\cL$}.
\end{definition}
\noindent
In what follows we fix a first order language $\cL$ with signature $(R,F)$ and set of variables $\bd{V}$.

\begin{definition}
The set of all words over the alphabet consisting of
$$\{\neg, \ra, \forall, (, )\} \cup R \cup F \cup \bd{V}$$
is called \textit{the set of expressions of $\cL$}. The set $\{\neg, \ra, \forall, (, )\}$ is called \textit{the set of logical constants}.
\end{definition}
\noindent
For each expression $e$ of $\cL$ denote by $|e|$ the length of $e$. Hence $|e|$ is the total number of occurrences of symbols $\{\neg, \ra, \forall, (, )\} \cup R \cup F \cup \bd{V}$.

\begin{definition}
We define the set of terms of $\cL$ by recursion.
\begin{enumerate}[label=\textbf{(\arabic*)}, leftmargin=3.0em]
\item Every variable of $\cL$ is a term.
\item If $t_1,...,t_n$ are terms of $\cL$ and $f$ is $n$-ary function symbol, then $ft_1...t_n$ is a term of $\cL$
\end{enumerate}
\end{definition}
\noindent
The following result studies the syntax of terms of $\cL$.

\begin{theorem}[Unique readability of terms]\label{theorem:unique_readebility_of_terms}
Let $t$ be a term of $\cL$. Then either $t$ is a variable or there exists a unique $n$-ary function symbol $f$ and unique terms $t_1,...,t_n$ such that
$$t = ft_1...t_n$$
for some $n \in \NN$.
\end{theorem}
\noindent
For the proof we need some additional results.

\begin{lemma}\label{lemma:term_presentation}
Let $t$ be a term of $\cL$. Then either $t$ is a variable or there exists an $n$-ary function symbol
and terms $t_1,...,t_n$ such that
$$t = ft_1...t_n$$
for some $n\in\NN$.
\end{lemma}
\begin{proof}[Proof of the lemma]
This follows from the definition of the set of terms of L. Details are left for the
reader.
\end{proof}

\begin{lemma}\label{lemma:terminal_segment_of_a_term_is_a_concatenation_of_terms}
Let $t$ be a term and let $s$ be its terminal segment. Then $s$ is a concatenation of terms.
\end{lemma}
\begin{proof}[Proof of the lemma]
The proof goes on induction on $|t|$. If $|t| = 1$, then the assertion holds because $t$ is a variable. Next suppose that $|t| \geq 1$ and $s$ is a nonempty terminal segment of $t$. By Lemma \ref{lemma:term_presentation} we can write $t = ft_1...t_m$, where $m$ is a positive integer, $f \in F_m$ and $t_1,...,t_m$ are terms. Then
either $s = t$ or
$$s = s_it_{i+1}...t_m$$
where $s_i$ is a terminal segment of a term $t_i$ for some $1 \leq i \leq m$. Since $|t_i|<|t|$, we derive by induction that $s_i$ is a concatenation of terms. Thus $s$ is a concatenation of terms.
\end{proof}

\begin{lemma}
Let $t,s$ be terms. If $s$ is a initial term of $t$, then $t$ is equal to $s$.
\end{lemma}
\begin{proof}[Proof of the lemma]
We define $\ZZ$-valued function $K$ on a set of words over the alphabet consisting of $F \cup \bd{V}$. We set
$$K(s) =\begin{cases}1 & \mbox{ if }s\mbox{ is a variable}\\
1-n & \mbox{ if }s\mbox{ is a function symbol}
\end{cases}$$
Next we extend $K$ to more complicated words over $F \cup \bd{V}$ by
$$K(s_1 ...s_m) = K(s_1) + ... + K(s_m)$$
We first check that if $t$ is a term, then $K(t) = 1$. Note that $K(t) = 1$ if $t$ is a variable. Moreover, if $K(t_1) = ... = K(t_n) = 1$ for some terms of $\cL$ and $f$ is $n$-ary function symbol, then also $K( ft_1...t_n) = 1$.
By recursion we deduce that $K(t) = 1$ for every term $t$. Now we prove the lemma. If $t, s$ are terms and $t$ is equal to $sw$ for some expression $w$, then by Lemma \ref{lemma:terminal_segment_of_a_term_is_a_concatenation_of_terms} expression $w$ is a concatenation of terms. This shows that if $w$ is nonempty, then $K(w) \geq 1$ and hence
$$1 = K(t) = K(s) + K(w) \geq 2$$
This contradiction shows that $w$ must be empty. Hence $t$ is equal to $s$.
\end{proof}

\begin{proof}[Proof of the theorem]
Lemma \ref{lemma:term_presentation} is the existence part of the theorem. Suppose that
$$ft_1 ...t_n = gs_1 ...s_m$$
where $f \in F_n$ , $g \in F_m$ and $t_1, ..., t_n , s_1 , ..., s_m$ are terms. Then $f = g$ and thus $n = m$. We have
$$t_1 ...t_n = s_1 ...s_n$$
Without loss of generality we may assume that $|t_1| \leq |s_1|$. By Lemma \ref{lemma:terminal_segment_of_a_term_is_a_concatenation_of_terms} we deduce that $s_1 = t_1$.
Thus
$$t_2 ...t_n = s_2 ...s_n$$
and by repeating this argument we obtain $t_2 = s_2 ,...,t_n = s_n$.
\end{proof}
 
\begin{definition}
\textit{An atomic formula of $\cL$} is an expression of $\cL$ of the form
$$rt_1...t_n$$
where $r$ is $n$-ary relation symbol and $t_1, ...,t_n$ are terms of $\cL$.
\end{definition}

\begin{definition}
We define the set of formulas of $\cL$ by recursion.
\begin{enumerate}[label=\textbf{(\arabic*)}, leftmargin=3.0em]
\item Every atomic formula is a formula of $\cL$.
\item If $\phi$ is a formula of $\cL$, then $(\neg \phi)$ is a formula of $\cL$.
\item If $\phi$ and $\psi$ are formulas of $\cL$, then $(\phi\ra \psi)$ is a formula of $\cL$.
\item If $x$ is a variable and $\phi$ is a formula of $\cL$, then $\forall x\phi$ is a formula of $\cL$.
\end{enumerate}
\end{definition}
\noindent
Now we state the main property of the syntax of formulas of $\cL$.

\begin{theorem}[Unique readability of formulas]\label{theorem:unique_readability_of_formulas}
Let $\gamma$ be a formula of $\cL$. Then precisely one of the following assertions holds.
\begin{enumerate}[label=\textbf{\emph{(\arabic*)}}, leftmargin=3.0em]
\item $\gamma$ is of the form $rt_1...t_n$ for a unique $n$-ary relation symbol $r$ and unique terms $t_1 , ..., t_n$.
\item $\gamma$ is of the form $(\neg \phi)$ for a unique formula $\phi$ of $\cL$.
\item $\gamma$ is of the form $(\phi \ra \psi)$ for unique formulas $\phi, \psi$ of $\cL$.
\item $\gamma$ is of the form $\forall x \phi$ for a unique variable $x$ and a unique formula $\phi$ of $\cL$.
\end{enumerate}
\end{theorem}
\noindent
For the proof we need the following results.

\begin{lemma}\label{lemma:presentation_of_formulas}
Let $\gamma$ be a formula of $\cL$. Then precisely one of the following assertions holds.
\begin{enumerate}[label=\textbf{\emph{(\arabic*)}}, leftmargin=3.0em]
\item $\gamma$ is an atomic formula.
\item $\gamma$ is of the form $(\neg \phi)$ for some formula $\phi$ of $\cL$.
\item $\gamma$ is of the form $(\phi \ra \psi)$ for some formulas $\phi, \psi$ of $\cL$.
\item $\gamma$ is of the form $\forall x \phi$ for some variable $x$ and some formula $\phi$ of $\cL$.
\end{enumerate}
\end{lemma}
\begin{proof}[Proof of the lemma]
The fact that each formula of $\cL$ is of one of the types enumerated above follows from the recursive definition of the set of formulas of $\cL$. It remains to prove that each formula is precisely of one of the forms \textbf{(1)}, \textbf{(2)}, \textbf{(3)}, \textbf{(4)}. By looking at first symbols we see that the only possibility that must be excluded is $(\neg \phi) = (\psi \ra \theta)$ for some formulas $\phi, \psi, \theta$ of $\cL$, but note that if this is the case, then $\psi$ starts with $\neg$. This is impossible.
\end{proof}

\begin{lemma}\label{lemma:no_formula_is_a_proper_initial_segment_of_any_other_formula}
Let $\phi$ be a formula of $\cL$. Then $\phi$ has no proper initial segments that are formulas.
\end{lemma}
\begin{proof}[Proof of the lemma]
We define a $\ZZ$-valued function $K$ on the set of words over the alphabet consisting of union of the set $\{\forall, \neg, \ra, (,)\}$ and the set of terms of $\cL$. We set
$$K(s) = \begin{cases}
1 & \mbox{ if }s\mbox{ is a term}\\
1-n & \mbox{ if }s\mbox{ is $n$-ary relation symbol}\\
0 & \mbox{ if }s\mbox{ is }\neg\\
-1 & \mbox{ if }s\mbox{ is }\ra\\
-1 & \mbox{ if }s\mbox{ is }\forall\\
-1 & \mbox{ if }s\mbox{ is left parenthesis}\\
1 & \mbox{ if }s\mbox{ is right parenthesis}\\
\end{cases}$$
Next we extend $K$ to more complicated words by
$$K(s_1...s_m) = K(s_1) +...+ K(s_m)$$
By recursion it follows that for every formula $\gamma$ of $\cL$ we have $K(\gamma) = 1$. Assume now that $\beta$ is a proper initial segment of a formula $\gamma$ of $\cL$. We show that $K(\beta) < 1$. By Lemma \ref{lemma:presentation_of_formulas} we have the following options.
\begin{enumerate}[label=\textbf{(\arabic*)}, leftmargin=3.0em]
\item $\gamma$ is an atomic formula. Then $\gamma$ is a concatenation of $\beta$ and (according to Lemma \ref{lemma:terminal_segment_of_a_term_is_a_concatenation_of_terms}) a concatenation of terms $w$. We have $K(w) \geq 1$ and hence $K(\beta) = K(\gamma)- K(w) = 1 - K(w) < 1$.
\item $\gamma$ is of the form $(\neg \phi)$ for some formula $\phi$ of $\cL$. Then $\beta$ is either an initial segment of $(\neg$ or a concatenation of $(\neg$ with some initial segment $\alpha$ of $\phi$. In the first case $K(\beta) = -1$. In the second case $K(\beta) = K(\gamma) - (K(\phi) - K(\alpha)) -1 = 1 - (1 - K(\alpha)) = K(\alpha)-1$ and it suffices to show that $K(\alpha) \leq 1$ for every initial segment $\alpha$ of $\phi$.
\item $\gamma$ is of the form $(\phi \ra \psi)$ for some formulas $\phi, \psi$ of $\cL$. Careful investigation shows that this option reduces to proving that $K(\alpha) \leq 1$ for every initial segment $\alpha$ of $\phi$ or $\psi$.
\item $\gamma$ is of the form $\forall x \phi$ for some variable $x$ and some formula $\phi$ of $\cL$. Careful investigation shows that this option reduces to proving that $K(\alpha) < 1$ for proper initial segment $\alpha$ of $\phi$.
\end{enumerate}
Options considered above imply that the assertion that for every proper initial segment $\beta$ of a formula $\gamma$ of $\cL$ we have $K(\beta) < 1$ follows by induction on $|\gamma|$. To prove the main assertion note that if $\gamma$ is simultaneously a formula of $\cL$ and a proper initial segment of some other formula of $\cL$, then we have that $K(\gamma) = 1$ and $K(\gamma) < 1$. This is a contradiction and the lemma is proved.
\end{proof}

\begin{proof}[Proof of the theorem]
Lemma \ref{lemma:presentation_of_formulas} implies that each formula of $\cL$ admits precisely one presentation of given type. It suffices to show uniqueness. The uniqueness part of description \textbf{(1)} follows from Lemma \ref{lemma:terminal_segment_of_a_term_is_a_concatenation_of_terms}. For \textbf{(2)} and \textbf{(3)} it is a consequence of Lemma \ref{lemma:no_formula_is_a_proper_initial_segment_of_any_other_formula}. For \textbf{(4)} the uniqueness is clear.
\end{proof}
\noindent
We define the function $\bd{Fr}$ that sends each formula of $\cL$ to some subset of $\bd{V}$. This definition is based on Theorem \ref{theorem:unique_readability_of_formulas}. Let $\gamma$ be a formula of $\cL$. If $\gamma$ is an atomic formula, then $\bd{Fr}(\gamma)$ is a set of all variables occurring in $\gamma$. If $\gamma$ is of the form $(\neg \phi)$, then we define $\bd{Fr}(\gamma) = \bd{Fr}(\phi)$. If $\gamma$ is of the form $(\phi \ra \psi)$, then $\bd{Fr}(\gamma) = \bd{Fr}(\phi)\cup \bd{Fr}(\psi)$. Finally if $\gamma$ is of the form $\forall x\phi$, then $\bd{Fr}(\gamma) = \bd{Fr}(\phi) \setminus \{x\}$.

\begin{definition}
For every formula $\phi$ of $\cL$ elements of the set $\bd{Fr}(\phi)$ are called \textit{free variables of $\phi$}.
\end{definition}

\begin{definition}
Let $\phi$ be a formula of $\cL$. If $\bd{Fr}(\phi)=\emptyset$, then $\phi$ is called \textit{a sentence of $\cL$}.
\end{definition}

\section{First order languages and their semantics}
\noindent
In this section we fix a first order language $\cL$ with the set of variables $\bd{V}$.

\begin{definition}
\textit{A model $\fU$ of $\cL$} consists of the following data.
\begin{enumerate}[label=\textbf{(\arabic*)}, leftmargin=3.0em]
\item A nonempty set $|\fU|$ called \textit{the universe of $\fU$} or \textit{the underlying set of $\fU$}.
\item For each $n$-ary relation symbol $r$ a relation $r^\fU \subseteq \underbrace{|\fU| \times ... \times |\fU|}_{n\,\mathrm{times}}$.
\item For each $n$-ary function symbol $f$ a function $f^\fU:\underbrace{|\fU| \times ... \times |\fU|}_{n\,\mathrm{times}} \ra |\fU|$.
\end{enumerate}
\end{definition}
  
\begin{definition}
Let $\fU$ be a model of $\cL$. \textit{An interpretation of $\cL$ in $\fU$} is a function $s:\bd{V}\ra |\fU|$.
\end{definition}
\noindent
Let $\fU$ be a model of $\cL$ and $s:\bd{V}\ra |\fU|$ be an interpretation. Then we define $s(t)$ for a term $t$ of $\cL$ by recursion on $|t|$. Namely suppose that $f$ is $n$-ary function symbol for some $n$ and $t_1, ..., t_n$ are terms such that $s(t_1),...,s(t_n)$ are defined. Then we define
$$s(ft_1...t_n)= f^{\fU}(s(t_1),...,s(t_n))$$
Now suppose that $\fU$ is a model of $\cL$. We define statements $\fU \vDash \gamma[s]$ for every formula $\gamma$ of $\cL$ simultaneously for all interpretations $s:\bd{V} \ra |\fU|$. The definition is recursion on the number of occurrances of $\forall, \neg, \ra$ in $\gamma$.
\begin{enumerate}[label=\textbf{(\arabic*)}, leftmargin=3.0em]
\item If $\gamma$ is of the form $rt_1 ...t_n$ for some $n$-ary relation symbol $r$ and terms $t_1,...,t_n$ of $\cL$, then $\fU \vDash \gamma[s]$ holds for some interpretation $s$ if and only if $\left(s(t_1),...,s(t_n)\right) \in r^\fU$ holds.
\item If $\gamma$ is of the form $(\neg \phi)$ for some formula $\phi$ of $\cL$, then $\fU \vDash \gamma[s]$ holds for some interpretation $s$ if and only if $\fU \vDash \phi[s]$ does not hold.
\item If $\gamma$ is of the form $(\phi \ra \psi)$ for some formulas $\phi, \psi$ of $\cL$, then $\fU \vDash \gamma[s]$ holds for some interpretation $s$ if and only if $\fU \vDash \psi[s]$ holds or $\fU \vDash \phi[s]$ does not hold.
\item If $\gamma$ is of the form $\forall x \phi$, then $\fU \vDash \gamma[s]$ holds if and only if for every interpretation $\tilde{s}$ such that $\tilde{s}_{\mid \bd{V}\setminus \{x\}} = s_{\mid \bd{V}\setminus \{x\}}$ statement $\fU \vDash \phi[\tilde{s}]$ holds.
\end{enumerate}
If $\fU \vDash \gamma[s]$ does not hold, then we write $\fU \nvDash \gamma[s]$.

\begin{definition}
Let $\Gamma$ be a set of formulas of $\cL$ and let $\phi$ be a formula of $\cL$. Suppose that for every model $\fU$ of $\cL$ and every interpretation $s:\bd{V} \ra |\fU|$ such that $\fU \vDash \gamma[s]$ for all $\gamma \in \Gamma$ we have $\fU \vDash \phi[s]$. Then we write $\Gamma \vDash \phi$ and say that \textit{$\Gamma$ logically implies $\phi$}.
\end{definition}

\begin{definition}
Let $\Gamma$ be a set of formulas of $\cL$. Let $\fU$ be a model of $\cL$ and let $s: \bd{V} \ra |\fU|$ be an interpretation. Suppose that for all $\phi$ in $\Gamma$ we have $\fU \vDash \phi[s]$. Then we say that $\fU$ together with $s$ is \textit{a model for $\Gamma$}.
\end{definition}

\begin{proposition}\label{proposition:only_free_variables_matter_for_interpretation}
Let $\gamma$ be a formula of $\cL$. Suppose that $\fU$ is a model of $\cL$ and $s_1, s_2: \bd{V} \ra |\fU|$ are interpretations such that $s_1(x) = s_2(x)$ for all $x\in \bd{Fr}(\gamma)$. Then $\fU \vDash \gamma[s_1]$ if and only if $\fU \vDash \gamma[s_2]$.
\end{proposition}
\begin{proof}
The proof goes on induction on the number of occurences of $\forall, \neg, \ra$ in $\gamma$. We prove the
statement for all pairs of interpretations $s_1,s_2: \bd{V} \ra |\fU|$ that agree on $\bd{Fr}(\gamma)$.
\begin{enumerate}[label=\textbf{(\arabic*)}, leftmargin=3.0em]
\item If $\gamma$ is of the form $rt_1 ...t_n$ for some $n$-ary relation symbol $r$ and terms $t_1,...,t_n$ of $\cL$, then all variables are free in $\gamma$ and hence $s_1(t_i) = s_2(t_i)$ for all $i$. Thus
$$\fU \vDash \gamma[s_1] \,\Leftrightarrow\, \left(s_1(t_1), ...,s_1(t_n)\right) \in r^\fU \,\Leftrightarrow\, \left(s_2(t_1),...,s_2(t_n)\right) \in r^\fU \,\Leftrightarrow\, \fU \vDash \gamma[s_2]$$
\item If $\gamma$ is of the form $(\neg \phi)$ for some formula $\phi$ of $\cL$, then
$$\fU \vDash \gamma[s] \,\Leftrightarrow\, \fU \nvDash \phi[s_1] \,\Leftrightarrow\, \fU \nvDash \phi[s_2] \,\Leftrightarrow\, \fU \vDash \gamma[s_2]$$
where the middle equivalence holds by induction.
\item $\gamma$ is of the form $(\phi \ra \psi)$ for some formulas $\phi, \psi$ of $\cL$. In this case we left the proof for the
reader.
\item Suppose that $\gamma$ is of the form $\forall x \phi$. Then $\fU \vDash \forall x \gamma[s_1]$ if and only if for every interpretation $\tilde{s}_1:\bd{V} \ra |\fU|$ such that $\tilde{s}_1$ and $s_1$ agree on $\bd{V} \setminus\{x\}$ we have $\fU \vDash \phi[\tilde{s}_1]$. Fix one such $\tilde{s}_1$. Now we define an interpretation $\tilde{s}_2:\bd{V} \ra |\fU|$ such that $\tilde{s}_1(x) = \tilde{s}_2(x)$ and $\tilde{s}_2$ agree with $s_2$ on $\bd{V} \setminus \{x\}$. By induction we have
$$\fU \vDash \phi[\tilde{s}_1] \,\Leftrightarrow\, \fU \vDash \phi[\tilde{s}_2]$$
Since $\fU \vDash \phi[\tilde{s}_1]$, we deduce that $\fU \vDash \phi[\tilde{s}_2]$. Note that the set of interpretations $\tilde{s}_2$ constructed in this manner is precisely the set of all interpretations that agree with $s_2$ on $\bd{V} \setminus \{x\}$. This proves that $\fU \vDash \gamma[s_2]$. Hence
$$\fU \vDash \gamma[s_1] \Rightarrow \fU \vDash \gamma[s_2]$$
By symmetry we deduce that
$$\fU \vDash \gamma[s_2] \,\Rightarrow\, \fU \vDash \gamma[s_1]$$
\end{enumerate}
\end{proof}

\begin{corollary}\label{corollary:sentence_satisfaction_depends_on_model}
Let $\Gamma$ be a set of sentences of $\cL$ and let $\fU$ be a model of $\cL$. Then the following assertions are
equivalent.
\begin{enumerate}[label=\emph{\textbf{(\roman*)}}, leftmargin=3.0em]
\item $\fU \vDash \phi[s]$ for some interpretation $s:\bd{V}\ra |\fU|$ and for every $\phi \in \Gamma$.
\item $\fU \vDash \phi[s]$ for every interpretation $s:\bd{V}\ra |\fU|$ and for every $\phi \in \Gamma$.
\end{enumerate}
\end{corollary}
\begin{proof}
This is immediate consequence of Proposition \ref{proposition:only_free_variables_matter_for_interpretation}.
\end{proof}

\begin{definition}
Let $\Gamma$ be a set of sentences of $\cL$ and let $\fU$ be a model of $\cL$. If conditions of Corollary \ref{corollary:sentence_satisfaction_depends_on_model} hold, then we say that $\fU$ is \textit{a model of $\Gamma$} and we write $\fU \vDash \Gamma$.
\end{definition}

\section{A deductive system}
\noindent
We fix a first order language $\cL$. We describe a deductive system $\vdash$ for the set of formulas of $\cL$. First we need to describe the set $\Lambda_\cL$ of its axioms.\\
For the first definition we assume that the reader is familiar with classical sentential logic.

\begin{definition}
Let $\Phi$ be a tautology of classical sentential logic (with connectives $\neg, \ra$) and let $p_1,...,p_n$ be all propositional variables occurring in $\Phi$. Let $\phi_1,...,\phi_n$ be formulas of $\cL$. \textit{A tautology of $\cL$} is a formula $\phi$ of $\cL$ obtained from $\Phi$ by replacing each occurrence of $p_i$ by $\phi_i$ for $i = 1, ..., n$.
\end{definition}
\noindent
Tautologies form a basis for the first group of axioms of $\vdash$. For the second group of axioms we need the notion of substitutabilty, which we now define.

\begin{definition}
Let $x$ be a variable and let $t$ be a term of $\cL$. We define \textit{substitutability of $t$ for a variable $x$ in a formula $\gamma$ of $\cL$} by means of Theorem \ref{theorem:unique_readability_of_formulas} as follows.
\begin{enumerate}[label=\textbf{(\arabic*)}, leftmargin=3.0em]
\item If $\gamma$ is an atomic formula, then $t$ is substitutable for $x$ in $\gamma$.
\item If $\gamma$ is of the form $(\neg \phi)$, then $t$ is substitutable for $x$ in $\gamma$ if and only if $t$ is substitutable for $x$ in $\phi$.
\item If $\gamma$ is of the form $(\phi \ra  \psi)$, then $t$ is substitutable for $x$ in $\gamma$ if and only if $t$ is substitutable for $x$ in $\phi$ and $t$ is substitutable for $x$ in $\psi$.
\item If $\gamma$ is of the form $\forall y \phi$ for $y \in \bd{V}\setminus \{x\}$, then $t$ is substitutable for $x$ in $\gamma$ if and only if $t$ is substitutable for $x$ in $\phi$ and $y$ does not occurre in $t$.
\item If $\gamma$ is of the form $\forall x \phi$, then $t$ is substitutable for $x$ in $\gamma$.
\end{enumerate}
\end{definition}
\noindent
Now we can define a substitution procedure.

\begin{definition}
Let $x$ be a variable and let $t$ be a term of $\cL$. Suppose that $t$ is substitutable for $x$ in $\gamma$. Then we define a formula $[\gamma]^x_t$ of $\cL$ by means of Theorem \ref{theorem:unique_readability_of_formulas} as follows.
\begin{enumerate}[label=\textbf{(\arabic*)}, leftmargin=3.0em]
\item If $\gamma$ is an atomic formula, then $[\gamma]^x_t$ is obtained from $\gamma$ by replacing each occurrence of variable $x$ by term $t$.
\item If $\gamma$ is of the form $(\neg \phi)$, then $[\gamma]^x_t$ is of the form $(\neg [\phi]^x_t)$.
\item If $\gamma$ is of the form $(\phi \ra \psi)$, then $[\gamma]^x_t$ is of the form $([\phi]^x_t \ra [\psi]^x_t)$.
\item If $\gamma$ is of the form $\forall y \phi$ for $y \in \bd{V} \setminus \{x\}$, then $[\gamma]^x_t$ is of the form $\forall y [\phi]^x_t$.
\item If $\gamma$ is of the form $\forall x\phi$, then $[\gamma]^x_t$ is of the form $\forall x\phi$.
\end{enumerate}
We say that $[\gamma]^x_t$ is obtained by \textit{substitution of $x$ by $t$ in $\gamma$}.
\end{definition}
\noindent
Lastly we need the notion of the generalization.

\begin{definition} 4.4. Let $\phi$ be a formula of $\cL$ and let $x_1,...,x_n$ be variables. Then a formula
$$\underbrace{\forall x_1 ...\forall x_n}_{n\,\mathrm{quantifiers}} \phi$$
is called \textit{a generalization of $\phi$}.
\end{definition}
\noindent
Now we are ready to describe axioms of $\vdash$.

\begin{definition}
We define a set $\Lambda_{\cL}$ consisting of generalization of the following formulas of $\cL$.
\begin{enumerate}[label=\textbf{(\arabic*)}, leftmargin=3.0em]
\item All tautologies of $\cL$.
\item
$$\left(\forall x\phi \ra [\phi]^x_t\right)$$
where $\phi$ is a formula of $\cL$, $t$ is a term of $\cL$, $x$ is a variable and $t$ is substitutable for $x$ in $\phi$.
\item
$$\big(\forall x(\phi \ra \psi) \ra (\forall x\phi \ra \forall x\psi)\big)$$
where $\phi, \psi$ are formulas of $\cL$ and $x$ is a variable.
\item
$$(\phi \ra \forall x\phi)$$
where $\phi$ is a formula and $x$ is a variable such that $x \not \in \bd{Fr}(\phi)$.
\end{enumerate}
The set $\Lambda_\cL$ is called \textit{the set of axioms for $\vdash$}.
\end{definition}
\noindent
Having defined axioms we need the notion of a deduction.

\begin{definition}
Let $\Gamma$ be a set of formulas of $\cL$ and let $\phi$ be a formula of $\cL$. \textit{A deduction of $\phi$ from $\Gamma$} is a sequence $(D_1 , ..., D_n)$ of formulas of $\cL$ such that $D_n$ is $\phi$, each $D_k$ is either an element of $\Lambda_\cL \cup \Gamma$ or there exist $i, j < k$ such that $D_j$ is of the form $(D_i \ra D_k)$. If there exists a deduction of $\phi$ from $\Gamma$, then we write $\Gamma \vdash \phi$ and say that \textit{$\phi$ is deducible from $\Gamma$}.
\end{definition}

\section{Soundness theorem}

\begin{theorem}[Soundness Theorem]\label{theorem:soundness_theorem}
Let $\Gamma$ be a set of formulas of some first order language $\cL$. If for some formula $\phi$ of $\cL$ we have $\Gamma \vdash \phi$, then also $\Gamma \vDash \phi$.
\end{theorem}

\begin{lemma}\label{lemma:classical_sentential_logic_holds_in_models}
Let $\Phi$ be a formula of classical sentential logic with connectives $\neg, \ra$ and let $p_1,...,p_n$ be all propositional variables occuring in $\Phi$. Suppose that $\phi_1,...,\phi_n$ are formulas of $\cL$, $\fU$ is a model of $\cL$ and $s$ is an interpretation of $\cL$ in $\fU$. Let $\phi$ be a formula of $\cL$ obtained from $\Phi$ by replacing each occurrence of $p_i$ by $\phi_i$ for $i = 1, ..., n$. Then the following are equivalent.
\begin{enumerate}[label=\textbf{\emph{(\roman*)}}, leftmargin=3.0em]
\item $\fU \vDash \phi[s]$
\item Consider a valuation $\nu:\{p_1,...,p_n\} \ra \{0, 1\}$ such that $\nu(p_i) = 1$ if and only if $\fU \vDash \phi_i[s]$. Then we have $\nu(\Phi) = 1$
\end{enumerate}
\end{lemma}
\begin{proof}[Proof of the lemma]
The proof goes on induction on the number of occurrances of $\neg, \ra$ in $\Phi$. The details are left to the reader.
\end{proof}

\begin{proof}[Proof of the theorem]
We denote by $\bd{V}$ the set of variables of $\cL$. In order to prove the theorem we need to check first that for every axiom $\gamma \in \Lambda_{\cL}$ we have $\vDash \gamma$.
\begin{itemize}
\item If $\gamma$ is a tautology of $\cL$, then the fact that $\vDash \gamma$ follows from Lemma \ref{lemma:classical_sentential_logic_holds_in_models}.
\item Suppose that $\gamma$ is of the form $\left(\forall x\phi \ra [\phi]^x_t\right)$, where $\phi$ is a formula of $\cL$, $x$ is a variable and $t$ is a term substitutable for $\phi$. Consider a model $\fU$ of $\cL$ and an interpretation $s:\bd{V} \ra |\fU|$. In order to prove that $\fU \vDash \gamma[s]$ it suffices to check
$$\fU \vDash \forall x\phi[s] \,\Rightarrow\,  \fU \vDash [\phi]^x_t[s]$$
Assume that $\fU \vDash \forall x\phi[s]$. Next pick an interpretation $\tilde{s}: \bd{V} \ra |\fU|$ such that $\tilde{s}_{\mid \bd{V}\setminus \{x\}} =s_{\mid \bd{V}\setminus \{x\}}$ and $\tilde{s}(x) = s(t)$. Since we have $\fU \vDash \forall x\phi[s]$, we deduce that $\fU \vDash \phi[\tilde{s}]$. Moreover, we have
$$\fU \vDash \phi[\tilde{s}] \,\Rightarrow\,\fU \vDash [\phi]^x_t[s]$$
Thus $\fU \vDash [\phi]^x_t[s]$. This proves implication
$$\fU \vDash \forall x\phi[s] \,\Rightarrow\, \fU \vDash [\phi]^x_t[s]$$
\item Suppose that $\gamma$ is of the form $\big(\forall x (\phi \ra \psi) \ra (\forall x\phi \ra \forall x\psi)\big)$ for $\phi, \psi$ formulas of $\cL$. Consider a model $\fU$ of $\cL$ and an interpretation $s:\bd{V} \ra |\fU|$. In order to prove that $\fU \vDash \gamma[s]$ it suffices to check
$$\big(\fU \vDash \forall x (\phi \ra \psi) [s]\mbox{ and }\fU \vDash \forall x\phi[s]\big) \,\Rightarrow\, \fU \vDash \forall x\psi[s]$$
Consider any interpretation $\tilde{s}:\bd{V} \ra |\fU|$ such that $\tilde{s}_{\mid \bd{V}\setminus \{x\}} = s_{\mid \bd{V}\setminus \{x\}}$. From $\fU \vDash \forall x (\phi \ra \psi) [s]$ we deduce that $\fU \vDash (\phi \ra \psi) [\tilde{s}]$ and from $\fU \vDash \forall x\phi[s]$ we deduce that $\fU \vDash \phi[\tilde{s}]$. Thus $\fU \vDash \psi[\tilde{s}]$. Since $\tilde{s}$ is arbitrary interpretation such that $\tilde{s}_{\mid \bd{V}\setminus \{x\}} = s_{\mid \bd{V}\setminus \{x\}}$, we deduce that $\fU \vDash \forall x\psi[s]$.
\item Suppose that $\gamma$ is of the form $\left(\phi \ra \forall x\phi\right)$, where $\phi$ is a formula of $\cL$ and $x \not \in \bd{Fr}(\phi)$. Consider a model $\fU$ of $\cL$ and an interpretation $s:\bd{V} \ra |\fU|$. In order to prove that $\fU \vDash \gamma[s]$ it suffices to check
$$\fU \vDash \phi[s] \,\Rightarrow\, \fU \vDash \forall x\phi[s]$$
Consider any interpretation $\tilde{s}:\bd{V} \ra |\fU|$ such that $\tilde{s}_{\mid \bd{V}\setminus \{x\}} = s_{\mid \bd{V}\setminus \{x\}}$. Since $x$ is not free in $\phi$, we derive by Proposition \ref{proposition:only_free_variables_matter_for_interpretation} that
$$\fU \vDash \phi[s] \,\Rightarrow\, \fU \vDash \phi[\tilde{s}]$$
Thus from $\fU \vDash \phi[s]$ we deduce that $\fU \vDash \phi[\tilde{s}]$. Since $\tilde{s}$ is arbitrary interpretation such that $\tilde{s}_{\mid \bd{V}\setminus \{x\}} = s_{\mid \bd{V}\setminus \{x\}}$, we deduce that $\fU \vDash \forall x\phi[s]$.
\item Suppose that $\gamma$ is of the form $\forall x\phi$, where $\phi$ is a formula of $\cL$ and $\vDash \phi$. Consider a model $\fU$ of $\cL$ and an interpretation $s:\bd{V} \ra |\fU|$. In order to prove that $\fU \vDash \gamma[s]$ it suffices to check
$$\fU \vDash \forall x\phi[s]$$
Consider any interpretation $\tilde{s}:\bd{V} \ra |\fU|$ such that $\tilde{s}_{\mid \bd{V}\setminus \{x\}} = s_{\mid \bd{V}\setminus \{x\}}$ . Then $\fU \vDash \phi[\tilde{s}]$. Indeed, this is a consequence of $\vDash \phi$. Thus $\fU \vDash \forall  x\phi[s]$.
\end{itemize}
This finishes the proof that $\vDash \gamma$ for every axiom of $\cL$. Now consider a set of formulas $\Gamma$ and a formula $\phi$ of $\cL$. Suppose that $\Gamma \vdash \phi$. Then there exists a deduction $(D_1, ..., D_n)$ of $\phi$ from $\Gamma$. Fix a model $\fU$ of $\cL$ and an interpretation $s:\bd{V} \ra |\fU|$. Suppose that for all formulas $\gamma \in \Gamma$ we have $\fU \vDash \gamma[s]$. We prove by induction that $\fU \vDash D_k[s]$ for all $k$.
\begin{enumerate}[label=\textbf{(\arabic*)}, leftmargin=3.0em]
\item If $D_k$ is an axiom of $\cL$ or element of $\Gamma$, then we have $\fU \vDash D_k[s]$ (by the argument above or
by assumption).
\item Suppose now that there are $i, j < k$ such that $D_j = (D_i \ra D_k)$. Assume also that $\fU \vDash D_i[s]$ and $\fU \vDash D_j[s]$. Then
$$\fU \vDash D_i[s]\mbox{ and }\big(\fU \vDash D_i[s] \,\Rightarrow\, \fU \vDash D_k[s]\big)$$
Hence $\fU \vDash D_k[s]$.
\end{enumerate}
Hence in particular, we have $\fU \vDash D_n[s]$. Since $D_n = \phi$, we derive that $\fU \vDash \phi[s]$. This shows that $\Gamma \vDash \phi$.
\end{proof}

\section{Some metatheorems}

\begin{proposition}[Monotonicity of $\vdash$]\label{proposition:monotonicity_of_deductive}
Let $\cL$ be a first order language. Let $\Gamma, \Delta$ be sets of formulas of $\cL$ and let $\phi$ be a formula of $\cL$. Suppose that $\Gamma \vdash \phi$ and $\Gamma \subseteq \Delta$. Then $\Delta \vdash \phi$.
\end{proposition}
\begin{proof}
Every deduction of $\phi$ from $\Gamma$ is a deduction of $\phi$ from $\Delta$.
\end{proof}

\begin{proposition}[Compactness of $\vdash$]\label{proposition:compactness_of_deductive}
Let $\cL$ be a first order language. Let $\Gamma$ be a set of formulas of $\cL$ and let $\phi$ be a formula of $\cL$. Suppose that $\Gamma \vdash \phi$. Then there exists finite subset $\{\phi_1,..., \phi_m\}$ of $\Gamma$ such that $\{\phi_1, ...\phi_m\} \vdash \phi$.
\end{proposition}
\begin{proof}
Every deduction of $\phi$ from $\Gamma$ involves only finitely many formulas in $\Gamma$.
\end{proof}

\begin{proposition}[Transitivity of $\vdash$]\label{proposition:transitivity_of_deductive}
Let $\cL$ be a first order language. Let $\Gamma, \Delta$ be sets of formulas of $\cL$ and let $\phi$ be a formula of $\cL$. Suppose that $\Gamma \vdash \psi$ for every formula $\psi \in \Delta$ and $\Delta \vdash \phi$. Then $\Gamma \vdash \phi$.
\end{proposition}
\begin{proof}
By Proposition \ref{proposition:compactness_of_deductive} there exist finitely many formulas $\psi_1,...,\psi_m$ of $\Delta$ such that
$$\{\psi_1,...,\psi_m\} \vdash \phi$$
Now for each $i$ there exists a deduction of $\psi_i$ from $\Gamma$. We concatenate these deductions for $i = 1,...,m$ and further concatenate them with a deduction of $\phi$ from $\{\psi_1,...,\psi_m\}$. This yields a deduction of $\phi$ from $\Gamma$.
\end{proof}

\begin{theorem}[Deduction Theorem]\label{theorem:deduction_theorem}
Let $\cL$ be a first order language. Let $\Gamma$ be a set of formulas of $\cL$ and let $\phi, \psi$ be formulas of $\cL$. Then $\Gamma \cup \{\phi\} \vdash \psi$ if and only if $\Gamma \vdash (\phi \ra \psi)$.
\end{theorem}
\begin{proof}
Note that $\big(\phi, (\phi \ra \psi) , \psi\big)$ is a deduction of $\psi$ from $\{\phi, (\phi \ra \psi)\}$. If $\Gamma \vdash (\phi \ra \psi)$, then by Proposition \ref{proposition:transitivity_of_deductive} we derive that $\Gamma \cup \{\phi\} \vdash \psi$.\\
Now suppose that $\Gamma \cup \{\phi\} \vdash \psi$. Let $(D_1 , ..., D_n)$ be a deduction of $\psi$ from $\Gamma \cup \{\phi\}$. We prove that $\Gamma \vdash (\phi \ra D_k)$ for each $k$. The proof goes on induction on $k$. We consider the following cases.
\begin{enumerate}[label=\textbf{(\arabic*)}, leftmargin=3.0em]
\item $D_k$ is an axiom of $\vdash$ or an element of $\Gamma$. Note that $\left(D_k \ra (\phi \ra D_k )\right)$ is a tautology of $\cL$. Hence
$$\big(D_k,\left((D_k \ra (\phi \ra D_k )\right),(\phi \ra D_k )\big)$$
is a deduction of $(\phi \ra D_k)$ from $\Gamma$.
\item $D_k$ is $\phi$. Then $(\phi \ra D_k)$ is a tautology of $\cL$ and hence
$$\left((\phi \ra D_k )\right)$$
is a deduction of $(\phi \ra D_k)$ from $\Gamma$.
\item There exist $i, j < k$ such that $D_j = (D_i \ra D_k)$. Note that
$$\bigg(\big(\phi \ra (D_i \ra D_k)\big) \ra \big((\phi \ra D_i ) \ra (\phi \ra D_k)\big)\bigg)$$
is a tautology of $\cL$ and hence one can write a deduction of $(\phi \ra D_k)$ from $\{(\phi \ra D_i),(\phi \ra D_j)\}$. By induction $\Gamma \vdash (\phi \ra D_i)$ and $\Gamma \vdash (\phi \ra D_j)$. Thus also $\Gamma \vdash (\phi \ra D_k)$ according to Proposition \ref{proposition:transitivity_of_deductive}.
\end{enumerate}
This proves that $\Gamma \vdash (\phi \ra D_k)$ for all $k$. Hence $\Gamma \vdash (\phi \ra D_n)$. Since $D_n = \psi$, we derive that $\Gamma \vdash (\phi \ra \psi)$.
\end{proof}


\begin{definition}
Let $\cL$ be a first order language and let $\bd{V}$ be the set of variables of $\cL$. Suppose that $I:F\cup R \cup \bd{V}\ra F \cup R \cup \bd{V}$ is a bijective map such that $I(\bd{V}) = \bd{V}$, $I(F_n) = F_n$ for all $n \in \NN$, $I(R_n) = R_n$ for all $n \in \NN$ and $n \geq 1$. Then $I$ is called \textit{an automorphism of $\cL$}.
\end{definition}
\noindent
Let $I$ be an automorphism of $\cL$. Then $I$ can be extended to a unique bijection on the set of expressions of $\cL$. There is no risk of confuction so we denote this extension by $I$.

\begin{theorem}[Invariance under automorphisms]\label{theorem:invariance_under_automorphisms}
Let $\cL$ be a first order language and let $\bd{V}$ be the set of variables of $\cL$. Suppose that $I$ is an automorphism of $\cL$. Then the following assertions hold.
\begin{enumerate}[label=\textbf{\emph{(\arabic*)}}, leftmargin=3.0em]
\item Set of terms of $\cL$, set of atomic formulas of $\cL$, set of formulas of $\cL$, set of tautologies of $\cL$ and set of
axioms $\Lambda_{\cL}$ of $\vdash$ are all invariant with respect to $I$.
\item Let $\Gamma$ be a set of formulas of $\cL$ and let $\phi$ be a formula of $\cL$. If $(D_1 , ..., D_n)$ is a deduction of $\phi$ from $\Gamma$, then $(I(D_1 ), ...I(D_n))$ is a deduction of $I(\phi)$ from $I(\Gamma)$.
\end{enumerate}
\end{theorem}
\begin{proof}
The proof of invariance of sets of terms and formulas of $\cL$ goes by the usual recursive method. From the fact that terms of $\cL$ are invariant under $I$ we deduce that also atomic formulas of $\cL$ are invariant under $I$. For tautologies and other types of axioms of $\vdash$ direct investigation works. The details of \textbf{(1)} are left to the reader.\\
We prove \textbf{(2)}. Let $(D_1 , ..., D_n)$ be a deduction of $\phi$ from $\Gamma$. Fix $k=1, 2, ..., n$. If $D_k$ is an axiom of $\vdash$, then $I(D_k)$ is an axiom of $\vdash$. If $D_k$ is an element of $\Gamma$, then $I(D_k) \in I(\Gamma)$. Finally suppose that there are $i, j < k$ such that $D_j = (D_i \ra D_k)$. Then $I(D_j) = \left(I(D_i) \ra I(D_k)\right)$. This shows that $(I(D_1), ..., I(D_n))$ is a deduction of $I(\phi)$ from $I(\Gamma)$.
\end{proof}

\begin{theorem}\label{theorem:generalization_rule}
Let $\cL$ be a first order language and let $\Gamma$ be a set of formulas of $\cL$. Assume that $x$ is a variable which does not occurre as a free variable in any formula in $\Gamma$. If $\Gamma \vdash \phi$ for some formula $\phi$ of $\cL$, then $\Gamma \vdash \forall x\phi$.
\end{theorem}
\begin{proof}
Let $(D_1,...,D_n)$ be a deduction of $\phi$ from $\Gamma$. We prove that for each $k$ there exists a deduction of $\forall xD_k$ from $\Gamma$. For this we consider the following cases.
\begin{enumerate}[label=\textbf{(\arabic*)}, leftmargin=3.0em]
\item $D_k$ is an axiom of $\vdash$. Since $\Lambda_{\cL}$ are closed under generalizations, we derive that $\forall xD_k$ is also an axiom of $\vdash$. Hence an one-element sequence $(\forall xD_k)$ is a deduction of $\forall x D_k$ from $\Gamma$.
\item $D_k$ is an element of $\Gamma$. Then
$$(D_k,(D_k \ra \forall x D_k),\forall x D_k)$$
is a deduction of $\forall xD_k$ from $\Gamma$. Note that here we use the assumption that there are no formulas in $\Gamma$ in which $x$ occurres as a free variable. Indeed, from this assumption it follows that $(D_k \ra \forall xD_k)$ is an axiom of $\vdash$.
\item There exist $i, j < k$ such that $D_j = (D_i \ra D_k)$. Consider the sequence of formulas
$$\bigg(\forall xD_i , \forall x (D_i \ra D_k ) , \big(\forall x (D_i \ra D_k ) \ra (\forall xD_i \ra \forall xD_k)\big) , (\forall xD_i \ra \forall xD_k) , \forall xD_k\bigg)$$
If there are deductions of $\forall x D_i$ and $\forall xD_j = \forall x (D_i \ra D_k)$ from $\Gamma$, then we add the sequence above to their concatenation in order to obtain a deduction of $\forall x D_k$ from $\Gamma$.
\end{enumerate}
From \textbf{(1)}, \textbf{(2)} and \textbf{(3)} we derive that the assertion follows by (finite) induction on $k$. In particular, there exists a deduction of $\forall x D_n = \forall x \phi$ from $\Gamma$.
\end{proof}

\section{Consistent sets of formulas}

\begin{definition}.
Let $\cL$ be a first order language and let $\Gamma$ be a set of formulas of $\cL$. We say that $\Gamma$ is \textit{consistent} if there exists a formula $\phi$ such that $\Gamma \nvdash \phi$. A set of formulas that is not consistent is called \textit{inconsistent}.
\end{definition}

\begin{proposition}\label{proposition:form_of_excluded_middle_for_consistent}
Let $\cL$ be a first order language and let $\Gamma$ be a set of formulas of $\cL$. If for some formula $\phi$ of $\cL$ we have $\Gamma \vdash \phi$ and $\Gamma \vdash (\neg \phi)$ then $\Gamma$ is inconsistent.
\end{proposition}
\begin{proof}
Left to the reader.
\end{proof}

\begin{proposition}
Let $\cL$ be a first order language, $\bd{V}$ be the set of its variables and let $\Gamma$ be a set of formulas of $\cL$. Suppose that $\fU$ is a model of $\cL$ and $s:\bd{V}\ra |\fU|$ is an interpretation of $\cL$ in $\fU$. If $\fU$ and $s$ form a model for $\Gamma$, then $\Gamma$ is consistent.
\end{proposition}
\begin{proof}
Consider a formula $\phi$ of $\cL$. We have either $\fU \vDash \phi[s]$ or $\fU \nvDash \phi[s]$. Thus we have either $\fU \nvDash \phi[s]$ or $\fU \nvDash (\neg \phi)[s]$. From Theorem \ref{theorem:soundness_theorem} and from the fact that $\fU$ with $s$ form a model of $\Gamma$ we deduce that either $\Gamma \nvdash \phi$ or $\Gamma \nvdash (\neg \phi)$.
\end{proof}

\begin{definition}
Let $\cL$ be a first order language and let $\Gamma$ be a set of formulas of $\cL$. We say that $\Gamma$ is \textit{maximally consistent} if there are no consistent sets of formulas of $\cL$ properly containing $\Gamma$.
\end{definition}

\begin{proposition}
Let $\cL$ be a first order language and let $\Gamma$ be a maximally consistent set of formulas of $\cL$. If $\Gamma \vdash \phi$ for some formula $\phi$ of $\cL$, then $\phi \in \Gamma$.
\end{proposition}
\begin{proof}
Left for the reader.
\end{proof}

\begin{proposition}
Let $\cL$ be a first order language and let $\Gamma$ be a maximally consistent set of formulas of $\cL$. Then for every formula $\phi$ of $\cL$ we have either $\phi \in \Gamma$ or $(\neg \phi)\in \Gamma$ but not both.
\end{proposition}
\begin{proof}
Left for the reader.
\end{proof}

\begin{proposition}[Lindenbaum lemma]\label{proposition:Lindenbaum_lemma}
Let $\Gamma$ be a set of formulas of a first order language $\cL$. If $\Gamma$ is consistent, then there exists maximally consistent set $\Delta$ of formulas of $\cL$ such that $\Gamma \subseteq \Delta$.
\end{proposition}
\begin{proof}
Consider a family
$$\cF = \big\{\Delta \,\big|\,\Delta\mbox{ is consistent set of formulas of }\cL\mbox{ and }\Gamma \subseteq \Delta \big\}$$
Suppose that $I$ is a linearly ordered set. Assume that $\{\Delta_i\}_{i\in I}$ is an order preserving map $I \ra \cF$ (we consider $\cF$ as a partially ordered set with respect to inclusion). We prove that $\bigcup_{i\in I}\Delta_i$ is consistent. Fix a formula $\phi$ of $\cL$ and assume that $\bigcup_{i\in I}\Delta_i$ is inconsistent. Then $\bigcup_{i\in I}\Delta_i\vdash \phi$ and $\bigcup_{i\in I}\Delta_i\vdash (\neg \phi)$. By compactness (Proposition \ref{proposition:compactness_of_deductive}) we derive that there exists $i_0\in I$ such that $\Delta_{i_0} \vdash \phi$ and $\Delta_{i_0} \vdash (\neg \phi)$. Proposition \ref{proposition:form_of_excluded_middle_for_consistent} implies that $\Delta_{i_0}$ is inconsistent. This is a contradiction. Thus $\bigcup_{i\in I}\Delta_i$ is consistent. Therefore, set $\cF$ partially ordered by inclusion has upper bounds for chains of its elements. Zorn’s lemma implies that $\cF$ admits a maximal element.
\end{proof}

\section{Consistency theorem}

\begin{theorem}[Henkin’s consistency theorem]\label{theorem:consistency_theorem}
Every consistent set of formulas of first order language
has a model.
\end{theorem}
\noindent
The proof of the theorem contains a lot of tedious details although the main idea is quite clear
and is encapsulated in Lemmas \ref{lemma:superconsistent_extension} and \ref{lemma:model_existence_lemma}. For the readers convenience we divide the whole argument
on three parts. First consist of initial technical preparations. In the second we prove interesting
conceptual results. The third part aggregates conceptual results into an elegant argument.
We start the first part of the proof by introducing a notion that is used in the proof.

\begin{definition}
Let $\cL$ be a first order language. Then a first order language $\cM$ is called \textit{a variable extension of $\cL$} if the signature of $\cM$ is the same as of $\cL$ and the set of variables of $\cL$ is contained in the set of variables of $\cM$.
\end{definition}
\noindent
Suppose that $\cL$ is a first order language. Then the deductive system of $\cL$ is denoted by $\vdash_{\cL}$ and the set of variables of $\cL$ is denoted by $\bd{V}_\cL$. This notation is important, since we consider different languages in our argument and hence we need to carefully distinguish between their deductive systems. Next for each expression $e$ of $\cL$ we denote by $\bd{Var}(e)$ the set of variables occurring in $e$. If $\Gamma$ is a set of formulas of $\cL$, then we define
$$\bd{Var}(\Gamma) = \bigcup_{\phi \in \Gamma} \bd{Var}(\phi)$$
Now we can prove the following results.

\begin{lemma}\label{lemma:deduction_in_variable_extensions_of_a_formula_that_belongs_to_extended_language}
Let $\Gamma$ be a set of formulas of $\cL$ and let $\cM$ be a variable extension of $\cL$. Suppose that $\phi$ is a formula of $\cL$. If $\Gamma \vdash_{\cM} \phi$, then also $\Gamma \vdash_{\cL} \phi$.
\end{lemma}
\begin{proof}[Proof of the lemma]
Consider a deduction $(D_1, ..., D_n)$ of $\phi$ from $\Gamma$ in $\cM$. Since the set
$$\bd{V}_{\cL} \setminus \big(\bd{Var}(D_1) \cup ... \cup \bd{Var}(D_n)\big)$$
is infinite, there exists an automorphism $I$ of $\cM$ that is identity on the signature of $\cM$, is identity on
$$\big(\bd{Var}(D_1) \cup ... \cup \bd{Var}(D_n)\big) \setminus \bd{V}_{\cM}$$
and sends variables
$$(\bd{Var}(D_1) \cup ... \cup \bd{Var}(D_n)\big) \cap \bd{V}_{\cM}$$
to variables in
$$\bd{V}_{\cL} \setminus \big(\bd{Var}(D_1) \cup ... \cup \bd{Var}(D_n)\big)$$
By Theorem \ref{theorem:invariance_under_automorphisms} we derive that $(I(D_1),...,I(D_n))$ is a deduction of $I(\phi)$ from
$$\big\{I(\gamma)\,\big|\,\gamma \in \Gamma \cap \{D_1,...,D_n\}\big\}$$
By definition of $I$ we have $I(\phi) = \phi$, $I(D_k) = D_k$ if $D_k \in \Gamma$ and $I(D_k)$ is a formula of $\cL$ for every $k$. Thus $\left(I(D_1),...,I(D_n)\right)$ is a deduction of $\phi$ from $\Gamma$ in $\cL$. Hence $\Gamma \vdash_{\cL} \phi$.
\end{proof}

\begin{lemma}\label{lemma:alpha_conversion}
Let $\phi$ be a formula of $\cL$ and let $x$ be a variable. Suppose that $z \not \in \bd{Var}(\phi) \cup \{x\}$ is a variable. Then
$$\forall z[\phi]^x_z \vdash_{\cL} \forall x\phi, \forall x\phi \vdash_{\cL} \forall z[\phi]^x_z$$
\end{lemma}
\begin{proof}[Proof of the lemma]
Note that $x$ is substitutable for $z$ in $[\phi]^x_z$ and $\phi = [[\phi]^x_z]^z_x$. Thus $(\forall z[\phi]^x_z \ra \phi)$ is an axiom of $\vdash_{\cL}$. Hence we deduce that $\forall z[\phi]^x_z \vdash_{\cL} \phi$. Note that $x$ does not occurre as a free variable in $\forall z[\phi]^x_z$ and hence by Theorem \ref{theorem:generalization_rule} we derive that $\forall z[\phi]^x_z  \vdash_{\cL} \forall x\phi$. The fact that $\forall x \phi \vdash_{\cL} \forall z[\phi]^x_z$ follows by symmetry. We left the details for the reader.
\end{proof}

\begin{lemma}\label{lemma:deduction_implies_quantified_deduction}
Suppose that $\phi \vdash_{\cL} \psi$ for some formulas $\phi, \psi$ of $\cL$. Then for every variable $x$ we have
$$\forall x\phi \vdash_{\cL} \forall x\psi$$
\end{lemma}
\begin{proof}[Proof of the lemma]
By Theorem \ref{theorem:deduction_theorem} we have $\vdash_{\cL} (\phi \ra \psi)$. Theorem \ref{theorem:generalization_rule} implies that $\vdash_{\cL} \forall x(\phi \ra \psi)$. Next
$$\big(\forall x(\phi \ra \psi) \ra (\forall x\phi \ra \forall x\psi)\big)$$
is an axiom of $\cL$. Thus $\vdash_{\cL} (\forall x\phi \ra \forall x\psi)$. Again by Theorem \ref{theorem:deduction_theorem} we derive that $\forall x\phi \vdash_{\cL} \forall x\psi$.
\end{proof}
\noindent
We prove now the following result often called \textit{lemma on alphabetic variants}.

\begin{lemma}\label{lemma:alphabetic_variants}
Let $\gamma$ be a formula of $\cL$, let $x$ be a variable and let $t$ be a term of $\cL$. Then there exists a formula $\alpha(\gamma)$ of $\cL$ such that the following assertions hold.
\begin{enumerate}[label=\emph{\textbf{(\arabic*)}}, leftmargin=3.0em]
\item $\gamma \vdash_{\cL} \alpha(\gamma)$ and $\alpha(\gamma) \vdash_{\cL} \gamma$.
\item $t$ is substitutable for $x$ in $\alpha(\gamma)$.
\item $\gamma$ and $\alpha(\gamma)$ have the same number of occurrences of symbols $\forall , \neg, \ra$.
\end{enumerate}
\end{lemma}
\begin{proof}[Proof of the lemma]
The construction of $\alpha(\gamma)$ goes on induction on $|\gamma|$. According to Theorem \ref{theorem:unique_readability_of_formulas} we consider the following cases.
\begin{enumerate}[label=\textbf{(\arabic*)}, leftmargin=3.0em]
\item $\gamma$ is an atomic formula. Then we define $\alpha(\gamma)$ as $\gamma$.
\item $\gamma$ is of the form $(\neg\phi)$ for some formula $\phi$ of $\cL$. We have $|\phi| < |\gamma|$. By induction $\alpha(\phi)$ is constructed. Now we define $\alpha(\gamma)$ as $\left(\neg\alpha(\phi)\right)$. Note that
$$\bigg(\big(\phi \ra \alpha(\phi)\big) \ra \big((\neg\alpha(\phi)) \ra (\neg \phi)\big)\bigg)$$
is a tautology of $\cL$ and from $\phi \vdash_{\cL} \alpha(\phi)$ we derive by Theorem \ref{theorem:deduction_theorem} that $\vdash_{\cL} (\phi \ra \alpha(\phi))$. Thus $\vdash_{\cL} ((\neg\alpha(\phi)) \ra (\neg\phi))$ and again by Theorem \ref{theorem:deduction_theorem} we derive that $(\neg\alpha(\phi)) \vdash_{\cL} (\neg\phi)$. Hence $\alpha(\gamma) \vdash_{\cL} \gamma$. Similarly one can show that $\gamma \vdash_{\cL} \alpha(\gamma)$. Moreover, $t$ is substitutable for
$x$ in $\alpha(\phi)$ and thus it substitutable for $x$ in $\alpha(\gamma)$.
\item $\gamma$ is of the form $(\phi \ra \psi)$ for some formulas $\phi, \psi$ of $\cL$. Then $|\phi|, |\psi| < |\gamma|$ and by induction $\alpha(\phi)$ and $\alpha(\psi)$ are constructed. We define $\alpha(\gamma)$ as $(\alpha(\phi) \ra \alpha(\psi))$. We left details to the reader as an exercise.
\item $\gamma$ is of the form $\forall y\phi$ and $y \neq x$. Then $|\phi| < |\gamma|$ and by induction $\alpha(\phi)$ is constructed. We define $\alpha(\gamma)$ as $\forall z[\alpha(\phi)]^y_z$, where $z \not \in \bd{Var}(\phi) \cup \bd{Var}(\alpha(\phi)) \cup \bd{Var}(t) \cup \{y\}$ is a variable. By Lemma \ref{lemma:alpha_conversion} we have $\forall z[\alpha(\phi)]^y_z \vdash_{\cL} \forall y \alpha(\phi)$. Since $\alpha(\phi) \vdash_{\cL} \phi$, we derive by Lemma \ref{lemma:deduction_implies_quantified_deduction} that $\forall y \alpha(\phi) \vdash_{\cL} \forall y\phi$. Thus $\forall z[\alpha(\phi)]^y_z \vdash_{\cL} \forall y\phi$. Similarly from $\phi \vdash_{\cL} \alpha(\phi)$ we derive that $\forall y\phi \vdash_{\cL} \forall z[\alpha(\phi)]^y_z$. For this part of the argument we used the assumption that $z \not \in \bd{Var}(\phi) \cup \bd{Var}(\alpha(\phi)) \cup \{y\}$. Next $t$ is substitutable for $x$ in $\alpha(\phi)$ and thus it is also substitutable (here we use the assumption that $z \not \in \bd{Var}(t)$) for $x$ in $\forall z[\alpha(\phi)]^y_z$.
\item $\gamma$ is of the form $\forall x\phi$. Then we define $\alpha(\gamma)$ as $\gamma$.
\end{enumerate}
This finishes the construction of $\alpha(\gamma)$ and proves the lemma.
\end{proof}
\noindent
This finishes the technical preparation. The following two lemmas are crucial and form the conceptual part of the proof.

\begin{lemma}\label{lemma:superconsistent_extension}
Let $\Gamma$ be a consistent set of formulas of $\cL$. Then there exist a variable extension $\cM$ of $\cL$ and a set $\Delta$ of formulas of $\cM$ such that the following assertions hold.
\begin{enumerate}[label=\emph{\textbf{(\arabic*)}}, leftmargin=3.0em]
\item $\Gamma \subseteq \Delta$.
\item If $(\neg\forall x\phi) \in \Gamma$, then there exists a variable $z$ in $\cM$ substitutable for $x$ in $\phi$ such that $(\neg[\phi]^x_z) \in \Delta$.
\item $\Delta$ is consistent in $\cM$.
\end{enumerate}
\end{lemma}
\begin{proof}[Proof of the lemma]
We define variable extension $\cM$ of $\cL$. We enlarge $\bd{V}_{\cL}$ to $\bd{V}_{\cM}$ by adding for each formula of the form $(\neg\forall x\phi) \in \Gamma$ a unique variable $z_{\phi,x}$. Hence
$$\bd{V}_{\cM} = \bd{V}_{\cL} \cup \big\{z_{\phi,x}\,\big|\,\mbox{ if }(\neg\forall x\phi) \in \Gamma\mbox{ for some }x \in \bd{V}_{\cL}\mbox{ and a formula }\phi\mbox{ of }\cL\big\}$$
We define
$$\Delta = \Gamma \cup \big\{(\neg[\phi]^x_{z_{\phi,x}})\,\big|\,\mbox{ if $(\neg\forall x\phi) \in \Gamma$}\big\}$$
Then $\Delta$ is a set of formulas of $\cM$ and \textbf{(1)} and \textbf{(2)} are satisfied. Suppose that $\Delta$ is inconsistent set of formulas of $\cM$. By compactness of $\vdash_{\cM}$ we derive that $\Delta$ has a finite inconsistent (with respect to $\vdash_{\cM}$) subset. By Lemma \ref{lemma:deduction_in_variable_extensions_of_a_formula_that_belongs_to_extended_language} we derive that $\Gamma$ is consistent set with respect to $\vdash_{\cM}$. Thus there exists a finite subset $$\bigg\{\left(\neg[\phi_1]^{x_1}_{z_{\phi_1,x_1}}\right),...,\left(\neg[\phi_n]^{x_n}_{z_{\phi_n,x_n}}\right),\left(\neg[\phi_{n+1}]^{x_{n+1}}_{z_{\phi_{n+1},x_{n+1}}}\right)\bigg\} \subseteq \bigg\{\left(\neg[\phi]^x_{z_{\phi,x}}\right)\,\bigg| \mbox{ if } \left(\neg\forall x\phi\right) \in \Gamma\bigg\}$$
such that its union with $\Gamma$ is inconsistent with respect to $\vdash_{\cM}$ and the union
$$\Xi = \Gamma \cup \bigg\{\left(\neg[\phi_1]^{x_1}_{z_{\phi_1,x_1}}\right),...,\left(\neg[\phi_n]^{x_n}_{z_{\phi_n,x_n}}\right)\bigg\}$$
is consistent with respect to $\vdash_{\cM}$. Write $\phi = \phi_{n+1}$ and $x = x_{n+1}$. We have that $\Xi \cup \bigg\{\left(\neg[\phi]^x_{z_{\phi,x}}\right)\bigg\}$ is inconsistent with respect to $\vdash_{\cM}$. We denote by $\perp$ any formula of $\cM$ that is a negation of a tautology of $\cM$. Then
$$\Xi \cup \bigg\{\left(\neg[\phi]^x_{z_{\phi,x}}\right)\bigg\} \vdash_{\cM} \perp$$
By Theorem \ref{theorem:deduction_theorem} we derive that
$$\Xi \vdash_{\cM} \bigg(\left(\neg[\phi]^x_{z_{\phi,x}}\right) \ra \perp\bigg)$$
Since formula
$$\bigg (\big((\neg[\phi]^x_{z_{\phi,x}}) \ra \perp \big) \ra [\phi]^x_{z_{\phi,x}}\bigg)$$
is a tautology of $\cM$, we deduce that $\Xi \vdash_{\cM} [\phi]^x_{z_{\phi,x}}$. By Theorem \ref{theorem:generalization_rule} we derive that $\Xi \vdash_{\cM} \forall z_{\phi,x} [\phi]^x_{z_{\phi,x}}$ (indeed, $z_{\phi,x}$ does not occurre as a free variable in any formula of $\Xi$). By Lemma \ref{lemma:alpha_conversion} we have $\forall z_{\phi,x} [\phi]^x_{z_{\phi,x}} \vdash_{\cM} \forall x\phi$ and hence by Theorem \ref{theorem:deduction_theorem} we have $\vdash_{\cM} (\forall z_{\phi,x} [\phi]^x_{z_{\phi,x}} \ra \forall x\phi)$. Thus $\Xi \vdash_{\cM} \forall x\phi$. On the other hand $(\neg\forall x\phi) \in \Gamma$ and hence this formula is an element of $\Xi$. Thus
$$\Xi \vdash_{\cM} \forall x\phi, \Xi \vdash_{\cM} (\neg\forall x\phi)$$
and therefore, $\Xi$ is inconsistent with respect to $\vdash_{\cM}$. This is a contradiction.
\end{proof}

\begin{lemma}\label{lemma:model_existence_lemma}
Let $\Gamma$ be a set of formulas of $\cL$. Assume that the following assertions hold.
\begin{enumerate}[label=\emph{\textbf{(\arabic*)}}, leftmargin=3.0em]
\item $\Gamma$ is maximally consistent.
\item If $(\neg\forall x\phi) \in \Gamma$, then there exists a variable $z$ substitutable for $x$ in $\phi$ such that $(\neg[\phi]^x_z) \in \Gamma$.
\end{enumerate}
Then $\Gamma$ has a model.
\end{lemma}
\begin{proof}[Proof of the lemma]
We construct a model $\fU$ for $\cL$. The universe $|\fU|$ is a disjoint copy
$$\big\{\ol{t}\,\big|\,t\mbox{ is a term of }\cL\big\}$$
of the set of terms of $\cL$. If $f$ is a $n$-ary function symbol of $\cL$ and $t_1,...,t_n$ are elements of $|\fU|$, then we define
$$f^{\fU}(\ol{t}_1,...,\ol{t}_n) = \ol{ft_1...t_n}$$
If $r$ is $n$-ary relation symbol of $\cL$ and $t_1$,...,$t_n$ are elements of $|\fU|$, then
$$\left(\ol{t}_1,...,\ol{t}_n\right)\in r^{\fU}\,\Leftrightarrow\,rt_1...t_n \in \Gamma$$
Moreover, for every variable $x$ in $\bd{V}_{\cL}$ we define $s(x) = \ol{x}$. This gives rise to an interpretation $s:\bd{V}_{\cL}\ra |\fU|$. We prove that for every formula $\gamma$ of $\cL$ the following holds
$$\fU \vDash \gamma[s] \,\Leftrightarrow\, \gamma \in \Gamma$$
The proof goes by induction on the number of symbols $\forall , \neg, \ra$ occurring in $\gamma$.
\begin{enumerate}[label=\textbf{(\arabic*)}, leftmargin=3.0em]
\item By definition of $\fU$ the assertion holds if $\gamma$ is an atomic formula.
\item $\gamma$ is of the form $(\neg\phi)$ for some formula $\phi$ of $\cL$. By induction we have $\fU \vDash \phi[s]$ if and only if $\phi \in \Gamma$. From this we deduce that $\fU \vDash (\neg\phi)[s]$ if and only if $(\neg\phi) \in \Gamma$.
\item $\gamma$ is of the form $(\phi \ra \psi)$ for some formulas $\phi, \psi$ of $\cL$. Then apply the same argument as in \textbf{(2)}. Details are left for the reader.
\item $\gamma$ is of the form $\forall x\phi$ for some formula $\phi$ of $\cL$ and a variable $x$. This is the most complicated
case.
\begin{itemize}
\item Suppose first that $\fU \nvDash \forall x\phi[s]$. Then there exists interpretation $\tilde{s}:\bd{V}_{\cL} \ra |\fU|$ and a term $t$ of $\cL$ such that $\tilde{s}(y) = s(y)$ for every variable $y \neq x$, $\tilde{s}(x) = \ol{t}$ and $\fU \nvDash \phi[\tilde{s}]$. We pick formula $\alpha(\phi)$ for $\phi, x$ and $t$ as in Lemma \ref{lemma:alphabetic_variants}. Then $\alpha(\phi) \vdash_{\cL} \phi$ and $t$ is substitutable for $x$ in $\alpha(\phi)$. Theorem \ref{theorem:soundness_theorem} implies that $\alpha(\phi) \vDash \phi$. Since
$$\fU \nvDash \phi[\tilde{s}]$$
we deduce that
$$\fU \nvDash \alpha(\phi)[\tilde{s}]$$
and hence
$$\fU \nvDash [\alpha(\phi)]^x_t[s]$$
Since according to Lemma \ref{lemma:alphabetic_variants} numbers of symbols $\forall , \neg, \ra$ occurring in $\alpha(\phi)$ and $\phi$ are the same, we deduce that $[\alpha(\phi)]^x_t$ has smaller number of occurrances of $\forall , \neg, \ra$ than $\gamma$. Thus by induction and the fact that $\fU \nvDash [\alpha(\phi)]^x_t[s]$, we derive that $[\alpha(\phi)]^x_t$ is not contained in $\Gamma$. If $\forall x \phi$ is contained in $\Gamma$, then by Lemma \ref{lemma:deduction_implies_quantified_deduction} it follows that $\Gamma \vdash_{\cL} \forall x\alpha(\phi)$ and then since $(\forall x\alpha(\phi) \ra [\alpha(\phi)]^x_t)$ is an axiom of $\vdash_{\cL}$, we obtain that $\Gamma \vdash_{\cL} [\alpha(\phi)]^x_t$. Since $\Gamma$ is maximally consistent, this would mean that $[\alpha(\phi)]^x_t \in \Gamma$. This is a contradiction. We deduce $\Gamma \nvdash \forall x\phi$ and hence $\forall x\phi \not \in \Gamma$.
\item Suppose conversely that $\forall x\phi \not \in  \Gamma$. Then $\Gamma \vdash_{\cL} (\neg\forall x\phi)$. Hence there exists a variable $z$ substitutable for $x$ in $\phi$ such that $\Gamma \vdash_{\cL} (\neg[\phi]^x_z)$. Since $\Gamma$ is maximally consistent, this implies that $[\phi]^x_z$ is not in $\Gamma$. Moreover, $[\phi]^x_z$ has smaller number of occurrences of $\forall , \neg, \ra$ than $\forall x\phi$. By induction and the fact that $[\phi]^x_z \not \in  \Gamma$, we obtain that $\fU \nvDash [\phi]^x_z[s]$ and hence also $\fU \nvDash \phi[\tilde{s}]$, where $\tilde{s}$ coincides with $s$ for all variables except $x$ and $\tilde{s}(x) =
s(z)$. This implies that $\fU \nvDash \forall x\phi[s]$.
\end{itemize}
Thus we have
$$\fU \nvDash \forall x\phi[s] \,\Leftrightarrow\, \forall x\phi \not \in \Gamma$$
and this finishes the proof of \textbf{(4)}.
\end{enumerate}
From the fact that
$$\fU \vDash \gamma[s] \,\Leftrightarrow\, \gamma \in \Gamma$$
for every formula $\gamma$ of $\cL$, we deduce that $\fU$ together with $s$ is a model for $\Gamma$.
\end{proof}
\noindent
Now we are ready to combine our results in final argument.

\begin{proof}[Proof of the theorem]
Let $\Gamma$ be a consistent set of formulas of some first order language $\cL$. We may assume by Proposition \ref{proposition:Lindenbaum_lemma} that $\Gamma$ is maximally consistent. We write $\Delta_0 = \Gamma_0 = \Gamma$ and $\cL_0 = \cL$. Suppose that $\cL_n$ is a first order language and $\Delta_n$ is a maximally consistent set of formulas of $\cL_n$ for some $n\in \NN$. Then by Lemma \ref{lemma:superconsistent_extension} there exists a variable extension $\cL_{n+1}$ of $\cL_n$ and a consistent set $\Gamma_{n+1}$ of formulas of $\cL_{n+1}$ containing $\Delta_n$ such that for every formula $\phi$ in $\cL_n$ if $(\neg\forall x\phi) \in \Delta_n$ for some variable $x$, then there exists a variable $z$ in $\cL_{n+1}$ substitutable for $x$ in $\phi$ with the property that $(\neg[\phi]^x_z) \in \Gamma_{n+1}$. By Proposition \ref{proposition:Lindenbaum_lemma} there exists a maximally consistent set $\Delta_{n+1}$ of formulas of $\cL_{n+1}$ that contains $\Gamma_{n+1}$. Note that
$$\Delta_n \subseteq \Gamma_{n+1} \subseteq \Delta_{n+1}$$
Now let $\cM$ be a first order language with the same signature as $\cL$ and with set of variables equal to the union of all sets of variables of $\cL_n$ for all $n\in \NN$. We define
$$\Delta = \bigcup_{n\in \NN} \Delta_n = \bigcup_{n\in \NN} \Gamma_n$$
Then $\Delta$ is maximally consistent set of formulas of $\cM$ and if $\phi$ is a formula of $\cM$ such that $(\neg\forall x\phi) \in
\Delta$ for some variable $x$ in $\cM$, then there exists a variable $z$ in $\cM$ substitutable for $x$ in $\phi$ such that $(\neg[\phi]^x_z) \in \Delta$. Now according to Lemma \ref{lemma:model_existence_lemma} we derive that $\Delta$ has a model. Since $\Gamma$ is a subset of $\Delta$, model of $\Delta$ with respect to $\cM$ is a model of $\Gamma$ with respect to $\cL$. Indeed, this holds because $\cM$ is a variable extension of $\cL$.
\end{proof}


 



























\small
\bibliographystyle{apalike}
\bibliography{../zzz}


\end{document}