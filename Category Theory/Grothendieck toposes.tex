\input pree.tex

\begin{document}

\title{Grothendieck toposes}
\date{}
\maketitle

\section{Introduction}
\noindent
In this notes we study Grothendieck topologies and toposes. For prerequisites we assume familiarity with \cite{Presheaves}. As usual we work in Tarski-Grothendieck set theory and we choose the base universe. We do not assume that underlying categories of our sites are small. This leads to some difficulties concerning sizes, but it pays off as the theory obtained is more general. This is especially important as possible applications we have in mind are in algebraic geometry, where so called \textit{gross sites} are used.\\
To deal with size difficulties in some of arguments we make now a few remarks of technical nature. Let $\cC$ be a category. There exists a Grothendieck universe $V$ containing the base universe $U$ such that $\cC$ is locally $V$-small (c.f. {\cite[section 1]{Presheaves}}). We have full and faithful embedding
$$\widehat{\cC} = \Fun\left(\cC^{\mathrm{op}},\Set\right) =  \Fun\left(\cC^{\mathrm{op}},\Set_U\right)\subseteq  \Fun\left(\cC^{\mathrm{op}},\Set_V\right)$$
and the inclusion preserves all limits. Objects of $\Fun\left(\cC^{\mathrm{op}},\Set_V\right)$ are called big presheaves on $\cC$. In particular, every presheaf on $\cC$ (an object $\widehat{\cC}$) is a big presheaf on $\cC$. Since $\cC$ is locally $V$-small, for every object $X$ in $\cC$ there exists big representable presheaf $h_X$ on $\cC$ and if $f:X\ra Y$ is a morphism in $\cC$, then $h_f:h_X\ra h_Y$ is a morphism of big presheaves on $\cC$.

\section{Sites and sheaves}
\noindent
In this section we fix a category $\cC$.

\begin{definition}
Let $X$ be an object of $\cC$. \textit{A sieve on $X$} is a family $S$ of arrows of $\cC$ with $X$ as a target such that for every $f:Y\ra X$ in $S$ and every morphisms $g:Z\ra Y$ their composition $f\cdot g$ is in $S$.
\end{definition}
\noindent
Every sieve $S$ on object $X$ of $\cC$ corresponds to a big subpresheaf of a big presheaf $h_X$ given by
$$\cC\ni Y\mapsto \{f:Y\ra X\,\big|\,f\in S\}\in \Set_V$$
This identifies the collection of sieves on $X$ with the collection of subpresheaves of $h_X$.

\begin{fact}\label{fact:propertiesofsieves}
Let $X$ be an object of $\cC$. The class-theoretic intersection and union of a collection of sieves on $X$ is a sieve on $X$.
\end{fact}
\begin{proof}
Left to the reader.
\end{proof}

\begin{definition}
Let $\cF$ be a collection of morphisms of $\cC$ with codomain in $X$. Then the intersection of all sieves on $X$ containing $\cF$ is called \textit{the sieve generated by $\cF$}.
\end{definition}
\noindent
One can directly describe the sieve on $X$ generated by $\cF = \{f_i:X_i\ra X\}_{i\in I}$ as a class of arrows $f:Y\ra X$ in $\cC$ such that $f$ factors through $f_i$ for some $i\in I$. 

\begin{definition}
Let $S$ be a sieve on $X$ and $f:Y\ra X$ be a morphism, then we define a sieve on $Y$ by formula
$$f^*S=\big\{g\in \bd{Mor}(\cC)\,\big|\,\mbox{ target of }g\mbox{ is }Y\mbox{ and }f\cdot g\in S\big\}$$
We call $f^*S$ \textit{the pullback of $S$ along $f$}.
\end{definition}

\begin{definition}
For every object $X$ in $\cC$ the family 
$$\big\{f\in \bd{Mor}(\cC)\,\big|\,\mbox{ target of }f\mbox{ is }X\big\}$$
is a sieve on $X$. We call it \textit{the maximal sieve on $X$}.
\end{definition}
 
\begin{definition}
\textit{A Grothendieck topology on $\cC$} is a collection $\cJ=\{\cJ(X)\}_{X\in \cC}$ such that $\cJ(X)$ is a class of sieves on $X$ and the following conditions are satisfied. 
\begin{enumerate}[label=\textbf{(\arabic*)}, leftmargin=1.5em]
\item The maximal sieve on $X$ is in $\cJ(X)$.
\item If $S\in \cJ(X)$ and $f:Y\ra X$, then $f^*S\in \cJ(Y)$.
\item Suppose that $S\in \cJ(X)$, $R$ is a sieve on $X$ and $f^*R\in \cJ(\mathrm{dom}(f))$ for every $f\in S$. Then $R\in \cJ(X)$.
\end{enumerate}
Sieves in class
$$\bigcup_{X\in \cC}\cJ(X)$$
are called covering sieves. A pair $(\cC,\cJ)$ consisting of a category $\cC$ and a Grothendieck topology $\cJ$ is called \textit{a site}.
\end{definition}

\begin{proposition}\label{proposition:coveringsievesproperties}
Let $\cJ$ be a Grothendieck topology on $\cC$ and $X$ be an object of $\cC$. Then the following assertions hold.
\begin{enumerate}[label=\emph{\textbf{(\arabic*)}}, leftmargin=1.5em]
\item Class $\cJ(X)$ is closed under finite intersections.
\item If $S\in \cJ(X)$ and $R$ is a sieve on $X$ such that $S\subseteq R$, then $R\in \cJ(X)$.
\end{enumerate}
\end{proposition}
\begin{proof}
We prove \textbf{(1)}. For this assume that $S$ and $T$ are covering sieves on $X$. Then $S\cap T$ is a sieve. Next pick $f:Y\ra X$ in $S$. Note that $f^*(S\cap T)=f^*T\in \cJ(Y)$. This implies that $S\cap T\in \cJ(X)$.\\
We prove now \textbf{(2)}. Fix $f:Y\ra X$ in $S$. Then $f^*R$ is the maximal sieve on $Y$ due to $S\subseteq R$. Hence $f^*R\in \cJ(Y)$. Since $S\in \cJ(X)$, we deduce that $R\in \cJ(X)$.
\end{proof}

\begin{fact}\label{fact:compositionofcoveringsieves}
Let $\cJ$ be a Grothendieck topology on $\cC$ and $X$ be an object of $\cC$. Suppose that $S$ is a covering sieve on $X$ and for each $f:Y\ra X$ in $S$ pick a covering sieve $R_f$ on $Y$. Then a family
$$R=\bigcup_{f\in S}f\cdot R_f$$ 
is a covering sieve on $X$.
\end{fact}
\begin{proof}
For every $f:Y\ra X$ in $S$ we have $R_f\subseteq f^*R$. By Proposition \ref{proposition:coveringsievesproperties} and since $R_f$ is in $\cJ(Y)$, we deduce that $f^*R\in \cJ(Y)$. Hence $f^*R$ is a covering sieve for every $f\in S$. This implies that $R\in \cJ(X)$.
\end{proof}

\begin{definition}
Let $F$ be a presheaf on $\cC$. Suppose that $X$ is an object of $\cC$ and $S$ is a sieve on $X$. We say that a family $\{x_f\}_{f\in S}$ such that $x_f\in F(\mathrm{dom}(f))$ is \textit{a matching family for $S$ of elements of $F$} if for every $f:Y\ra X$ in $S$ and $g:Z\ra Y$ in $\cC$ we have
$$F(g)(x_f)=x_{f\cdot g}$$
We say that an element $x\in F(X)$ is \textit{an amalgamation for the matching family} $\{x_f\}_{f\in S}$ if for every $f\in S$ we have $F(f)(x)=x_f$.
\end{definition}
\noindent
Let $S$ be an arbitrary sieve on object $X$ in $\cC$ and $F$ be a presheaf on $\cC$. In this notes we denote by $F(S)$ the class of matching families for $S$ of elements of $F$.\\
Note that if $S$ is a sieve on $X$ viewed as a big subpresheaf of $h_X$, then a matching family for $S$ of elements of $F$ can be viewed as a morphisms of big presheaves $S\ra F$. This identifies the collection of matching families for $S$ of elements of $F$ with a collection of morphisms $S\ra F$ of big presheaves. Next suppose that $\{x_f\}_{f\in S}$ is a matching family for $S$ of elements of $F$. Then amalgamations of $\{x_f\}_{f\in S}$ can be identified by means of Yoneda lemma {\cite[Theorem 3.3]{Presheaves}} with morphisms $h_X\ra F$ making the following triangle
\begin{center}
\begin{tikzpicture}
[description/.style={fill=white,inner sep=2pt}]
\matrix (m) [matrix of math nodes, row sep=3em, column sep=3em,text height=1.5ex, text depth=0.25ex] 
{ h_X&       F  \\
   S &          \\} ;
\path[densely dotted, ->,line width=0.8pt,font=\scriptsize]
(m-1-1) edge node[above] {$  $} (m-1-2);
\path[left hook ->,line width=0.8pt,font=\scriptsize]
(m-2-1) edge node[left] {$ $} (m-1-1);
\path[->,line width=0.8pt,font=\scriptsize]
(m-2-1) edge node[below = 3pt, right = 1pt] {$ \{x_f\}_{f\in S}$} (m-1-2);
\end{tikzpicture}
\end{center}
commutative.

\begin{definition}
Let $\cJ$ be a Grothendieck topology on $\cC$ and $F$ be a presheaf on $\cC$. We say that $F$ is \textit{a separated presheaf with respect to $\cJ$} if for any object $X$ in $\cC$, covering sieve $S\in \cJ(X)$ and for every matching family $\{x_f\}_{f\in S}$ for $S$ of elements of $F$ there exists at most one amalgamation $x\in F(X)$.
\end{definition} 

\begin{definition}
Let $\cJ$ be a Grothendieck topology on $\cC$ and $F$ be a presheaf on $\cC$. We say that $F$ is \textit{a sheaf with respect to $\cJ$} if for any object $X$ in $\cC$, covering sieve $S\in \cJ(X)$ and for every matching family $\{x_f\}_{f\in S}$ for $S$ of elements of $F$ there exists a unique amalgamation $x\in F(X)$.
\end{definition}
\noindent
In other words $F\in \widehat{\cC}$ is a separated presheaf (sheaf) with respect to a Grothendieck topology $\cJ$ on $\cC$ if for any $X\in \cC$, sieve $S\in \cJ(X)$ and morphism $S\ra F$ of big presheaves there exists at most one (a unique) morphism $h_X\ra F$ of big presheaves making the triangle 
\begin{center}
\begin{tikzpicture}
[description/.style={fill=white,inner sep=2pt}]
\matrix (m) [matrix of math nodes, row sep=3em, column sep=3em,text height=1.5ex, text depth=0.25ex] 
{ h_X&       F  \\
   S &          \\} ;
\path[densely dotted, ->,line width=0.8pt,font=\scriptsize]
(m-1-1) edge node[above] {$  $} (m-1-2);
\path[left hook ->,line width=0.8pt,font=\scriptsize]
(m-2-1) edge node[left] {$ $} (m-1-1);
\path[->,line width=0.8pt,font=\scriptsize]
(m-2-1) edge node[below = 5pt, right = -2pt] {$ \{x_f\}_{f\in S} $} (m-1-2);
\end{tikzpicture}
\end{center}
commutative.\\
Let $\cJ$ be a Grothendieck topology on $\cC$. We denote by $\PrSh_s(\cC,\cJ)$, $\Sh(\cC,\cJ)$  full subcategories of $\widehat{\cC}$ consisting of separated presheaves and sheaves with respect to $\cJ$, respectively.

\begin{theorem}\label{theorem:limitsinsheavesandseparatedpresheaves}
Let $\cJ$ be a Grothendieck topology on $\cC$. Then inclusion functors $\PrSh_s(\cC,\cJ)\ra \widehat{\cC}$, $\Sh(\cC,\cJ)\ra \widehat{\cC}$ create limits.
\end{theorem}
\begin{proof}
Let $D:I\ra \PrSh_s(\cC,\cJ)$ be a functor and assume that $\left(F,\big\{f_i:F\ra D(i)\big\}_{i\in I}\right)$ is a limiting cone over the composition of the functor $D:I\ra \PrSh_s(\cC,\cJ)$ with the inclusion $\PrSh_s(\cC,\cJ)\ra \widehat{\cC}$. We show that $F$ is a separated presheaf with respect to $\cJ$. Suppose that $S$ is a covering sieve on $X$ and $m:S\ra F$ is a morphism of big presheaves that represents certain matching family for $S$ of elements of $F$. Let $u:S\ra h_X$ be the inclusion. Suppose that morphism $p:h_X\ra F$ of big presheaves is an amalgamation for $m$. We need to show that this amalgamation is unique. For this it suffices to observe that from equality $p\cdot u = m$ we derive that $(f_i\cdot p)\cdot u = (f_i\cdot m)$ for $i\in I$. Hence for every $i\in I$ morphism $f_i\cdot p$ is an amalgamation of $f_i\cdot m$. Since $D(i)$ is a separated presheaf for every $i\in I$, this makes $f_i\cdot p$ uniquely determined for $i\in I$. Thus $p$ is uniquely determined itself according to the fact that the cone $\left(F,\big\{f_i\big\}_{i\in I}\right)$ is limiting in $\widehat{\cC}$ and the inclusion of presheaves in big presheaves preserves limits. Therefore, $F$ is a separated presheaf with respect to $\cJ$ and hence $\left(F,\big\{f_i\big\}_{i\in I}\right)$ is a limiting cone for $D$ in the category of separated presheaves.\\
Now assume that $D:I\ra \Sh(\cC,\cJ)$ is a functor and $\left(F,\big\{f_i:F\ra D(i)\big\}_{i\in I}\right)$ is a limiting cone over the composition of the functor $D:I\ra \Sh(\cC,\cJ)$ with the inclusion $\Sh(\cC,\cJ)\ra \widehat{\cC}$. We show that $F$ is sheaf with respect to $\cJ$. From what we prove above we know that $F$ is a separated presheaf with respect to $\cJ$. Suppose that $S$ is a covering sieve on $X$ and $m:S\ra F$ is a morphism of big presheaves that represents certain matching family for $S$ of elements of $F$. Let $u:S\ra h_X$ be the inclusion. It suffices to construct an amalgamation $p:h_X\ra F$ for $m$. We define $m_i = f_i \cdot m$ for $i\in I$. Now fix $i\in I$ for a moment. Then $m_i:S\ra D(i)$ is a matching family for $S$ of elements of a sheaf $D(i)$. Hence there exists a unique morphism $p_i:h_X\ra D(i)$ such that the triangle
\begin{center}
\begin{tikzpicture}
[description/.style={fill=white,inner sep=2pt}]
\matrix (m) [matrix of math nodes, row sep=3em, column sep=3em,text height=1.5ex, text depth=0.25ex] 
{ h_X&       D(i)  \\
   S &          \\} ;
\path[densely dotted, ->,line width=0.8pt,font=\scriptsize]
(m-1-1) edge node[above] {$ p_i $} (m-1-2);
\path[left hook ->,line width=0.8pt,font=\scriptsize]
(m-2-1) edge node[left] {$u$} (m-1-1);
\path[->,line width=0.8pt,font=\scriptsize]
(m-2-1) edge node[below = 5pt, right = -2pt] {$ m_i $} (m-1-2);
\end{tikzpicture}
\end{center}
is commutative. Now pick a morphism $\alpha:i\ra j$ in $I$. Then 
$$D(\alpha)\cdot p_i\cdot u = D(\alpha)\cdot m_i = m_j = p_j\cdot u$$
According to uniqueness of $p_j$ we deduce that $D(\alpha)\cdot p_i = p_j$. Hence $\left(h_X,\big\{p_i\big\}_{i\in I}\right)$ is a cone over $D$ in the category of big presheaves. Therefore, there exists a unique morphism $p:h_X\ra F$ of big presheaves such that $f_i\cdot p = p_i$ for every $i\in I$. Hence 
$$f_i\cdot p\cdot u=p_i\cdot u= m_i = f_i\cdot m$$
for every $i\in I$. Thus $p\cdot u = m$ because the cone $\left(F,\big\{f_i\big\}_{i\in I}\right)$ is limiting. Therefore, matching family $m$ for $S$ of elements of $F$ admits an amalgamation $p$ and hence $\left(F,\big\{f_i\big\}_{i\in I}\right)$ is a limiting cone for $D$ in the category of sheaves. 
\end{proof}
\noindent
The remaining part of this section contains some technical facts that we use in further developing material in this notes.

\begin{definition}
Let $F$ be a presheaf on $\cC$ and let $\cF = \big\{f_i:X_i\ra X\big\}_{i\in I}$ be a collection of morphisms in $\cC$ with codomain $X$. Assume that $\{x_i\}_{i\in I}$ is a collection such that $x_i\in F(X_i)$ for every $i\in I$ and
$$F(g_i)(x_i) = F(g_j)(x_j)$$
for any morphisms $g_i:Y\ra X_i$, $g_j:Y\ra X_j$ in $\cC$ satisfying $f_i\cdot g_i = f_j\cdot g_j$ for every pair $i,j\in I$. Then $\{x_i\}_{i\in I}$ is called \textit{a matching family for $\cF$ of elements of $F$}.
\end{definition}
\noindent
If $F$ is a presheaf on $\cC$ and $\cF = \big\{f_i:X_i\ra X\big\}_{i\in I}$ is a collection of morphisms in $\cC$ with codomain $X$, then we denote the class of matching families for $\cF$ of elements of $F$ by $F(\cF)$. Suppose that $S$ is a sieve generated by $\cF$. We have canonical injective map $\mathrm{can}_{\cF}:F(\cF)\ra \prod_{i\in I}F(X_i)$ and we denote by $\mathrm{res}_{S,\cF}:F(S)\ra F(\cF)$ a map that sends $\{x_f\}_{f\in S}$ to $\{x_{f_i}\}_{i\in I}$.

\begin{proposition}\label{proposition:matchingfamiliesaskernels}
Fix a presheaf $F$ on $\cC$ and a collection $\cF = \{f_i:X_i\ra X\}_{i\in I}$ of arrows in $\cC$ with codomain in $X$. Let $S$ be a sieve generated by this family. Then $\mathrm{res}_{S,\cF}$ is bijective. Moreover, if $\cC$ admits fiber products, then 
\begin{center}
\begin{tikzpicture}
[description/.style={fill=white,inner sep=2pt}]
\matrix (m) [matrix of math nodes, row sep=3em, column sep=6em,text height=1.5ex, text depth=0.25ex] 
{F(\cF) &   \prod_{i\in I}F(X_i)&  \prod_{(i,j)\in I\times I} F(X_i\times_XX_j)  \\} ;
\path[->,line width=0.8pt,font=\scriptsize]
(m-1-1) edge node[above] {$ \mathrm{can}_{\cF} $} (m-1-2)
(m-1-2) edge[transform canvas={yshift=0.5ex}] node[above] {$ \langle F(f'_{ij}) \cdot pr_i\rangle_{(i,j)}$} (m-1-3)
(m-1-2) edge[transform canvas={yshift=-0.5ex}] node[below] {$ \langle F(f''_{ij}) \cdot pr_j\rangle_{(i,j)}$} (m-1-3);
\end{tikzpicture}
\end{center}
is a kernel of a pair of arrows, where for every $(i,j)\in I\times I$ morphisms $f'_{ij}$ and $f'_{ji}$ form a cartesian square
\begin{center}
\begin{tikzpicture}
[description/.style={fill=white,inner sep=2pt}]
\matrix (m) [matrix of math nodes, row sep=3em, column sep=2em,text height=1.5ex, text depth=0.25ex] 
{X_i\times_XX_j &  &   X_j   \\
 X_i&   & X   \\} ;
\path[->,line width=0.8pt,font=\scriptsize]
(m-1-1) edge node[above] {$ f''_{ij}$} (m-1-3)
(m-2-1) edge node[below] {$ f_i $} (m-2-3)
(m-1-1) edge node[left] {$ f'_{ij} $} (m-2-1)
(m-1-3) edge node[right] {$ f_j  $} (m-2-3);
\end{tikzpicture}
\end{center}
\end{proposition}
\begin{proof}
Let $\{x_i\}_{i\in I}$ be a matching family for $\cF$ of elements of $F$. For every $f:Y\ra X$ in $S$ there exists $i\in I$ such that $f = f_i\cdot g_i$ for some $g_i:Y\ra X_i$. Indeed, this follows from the fact that $\cF$ generates $S$. We define $x_f = F(g_i)(x_i)$. Since $\{x_i\}_{i\in I}$ is a matching family for $\cF$ of elements of $F$, we derive that $x_f$ does not depend on the choice of $i\in I$ and factorization $f = f_i\cdot g_i$. This implies that $\{x_f\}_{f\in S}$ is a matching family for $S$ of elements of $F$. Now correspondence $\{x_i\}_{i\in I}\mapsto \{x_f\}_{f\in S}$ is the inverse of $\mathrm{res}_{S,\cF}$. This proves the first part of the statement.\\
Let $\left(x_i\right)_{i\in I}$ be an element of $\prod_{i\in I}F(X_i)$ such that $F(f'_{ij})(x_i)=F(f''_{ij})(x_j)$ for every pair $(i,j)\in I\times I$. Assume that for some $f:Y\ra X$ in $S$ we can write $f = f_i\cdot g_i$ for some $i\in I$ and $g_i:Y\ra X_i$ and similarly $f = f_j\cdot g_j$ for some $j\in I$ and $g_j:Y\ra X_j$. Then there exist a unique $g:Y\ra X_i\times_XX_j$ such that $g_i = f'_{ij}\cdot g$ and $g_j = f''_{ij}\cdot g$. We have
$$F(g_i)(x_i)=F(f'_{ij}\cdot g)(x_i)=F(g)\left(F(f'_{ij})(x_i)\right)=F(g)\left(F(f''_{ij})(x_j)\right)=F(f''_{ij}\cdot g)(x_j)=F(g_j)(x_j)$$
It follows that $\{x_i\}_{i\in I}$ is a matching family for $\cF$ of elements of $F$ and $\mathrm{can}_{\cF}\left(\{x_i\}_{i\in I}\right)=\left(x_i\right)_{i\in I}$. This proves that $\mathrm{can}_{\cF}$ is a bijection between $F(\cF)$ and the class of elements $\left(x_i\right)_{i\in I}\in \prod_{i\in I}F(X_i)$ such that $F(f'_{ij})(x_i)=F(f''_{ij})(x_j)$ for every pair $(i,j)\in I\times I$. This finishes the proof of the second part of the statement.
\end{proof}
\noindent
Next if $S\subseteq R$ are sieves on $X$ and $F$ is a presheaf on $\cC$, then we denote by $\mathrm{res}_{R,S}:F(R)\ra F(S)$ a map given by $\mathrm{res}_{R,S}(\{x_f\}_{f\in R}) = \{x_f\}_{f\in S}$. The next result is a useful technical tool.

\begin{proposition}\label{proposition:separatdpresheavesinduceinjections}
Let $\cJ$ be a Grothendieck topology on $\cC$ and $F$ be a separated presheaf with respect to $\cJ$. Pick $X$ in $\cC$. If $R$, $S$ in $\cJ(X)$ satisfy $S\subseteq R$, then $\mathrm{res}_{R,S}:F(R)\ra F(S)$ is injective.
\end{proposition}
\begin{proof}
Let $\mathrm{res}_{R,S}(\{x_f\}_{f\in R})=\{x_f\}_{f\in S}$. We show that $\{x_f\}_{f\in R}$ is uniquely determined by $\{x_f\}_{f\in S}$. For this pick $g\in R$ and consider $\{x_{g\cdot f}\}_{f\in g^*S}$. This is a subfamily of $\{x_f\}_{f\in S}$. For every $f\in g^*S$ we have $F(f)(x_g)=x_{g\cdot f}$ and hence $x_g$ is an amalgamation for a matching family $\{x_{g\cdot f}\}_{f\in g^*S}$ for $g^*S$ of elements of $F$. Since $F$ is a separated presheaf with respect to $\cJ$, we deduce that $x_g$ is uniquely determined with $\{x_{g\cdot f}\}_{f\in g^*S}$ and hence it is uniquely determined by $\{x_f\}_{f\in S}$. Arrow $g$ is an arbitrary element of $R$. Thus $\mathrm{res}_{R,S}$ is injective.
\end{proof}

\section{Grothendieck pretopologies}
\noindent
Let $\cC$ be a category with fiber products.

\begin{definition}
For every $X$ in $\cC$ let $\cK(X)$ be a class of collections $\{f_i:X_i\ra X\}_{i\in I}$ of arrows in $\cC$ with codomain in $X$. Assume that $\cK=\{\cK(X)\}_{X\in \cC}$ satisfies the following assertions.
\begin{enumerate}[label=\textbf{(\arabic*)}, leftmargin=1.5em]
\item $\{1_X:X\ra X\}\in \cK(X)$ for every object $X$ in $\cC$. 
\item If $\{f_i:X_i\ra X\}_{i\in I}\in \cK(X)$ for some $X$ in $\cC$ and $f:Y\ra X$ is a morphism, then $\{f'_i:X_i\times_XY\ra Y\}_{i\in I}\in \cK(Y)$ where $f'_i$ are defined by cartesian squares
\begin{center}
\begin{tikzpicture}
[description/.style={fill=white,inner sep=2pt}]
\matrix (m) [matrix of math nodes, row sep=3em, column sep=2em,text height=1.5ex, text depth=0.25ex] 
{X_i\times_XY &  &   X_i   \\
 Y&   & X   \\} ;
\path[->,line width= 0.8pt,font=\scriptsize]
(m-1-1) edge node[above] {$ $} (m-1-3)
(m-2-1) edge node[below] {$ f $} (m-2-3)
(m-1-1) edge node[left] {$ f'_i$} (m-2-1)
(m-1-3) edge node[right] {$f_i $} (m-2-3);
\end{tikzpicture}
\end{center}
\item Suppose that $\{f_i:X_i\ra X\}_{i\in I} \in \cK(X)$ and $\{f_{ij}:X_{ij}\ra X_i\}_{j\in J_i}\in \cK(X_i)$ for every $i\in I$. Then $\{f_i\cdot f_{ij}:X_{ij}\ra X\}_{i\in I,j\in J_i}\in \cK(X)$.
\end{enumerate}
Then we say that $\cK=\{\cK(X)\}_{x\in \cC}$ is \textit{a Grothendieck pretopology on $\cC$}.
\end{definition}

\begin{proposition}
Suppose that $\cK=\{\cK(X)\}_{X\in \cC}$ is a Grothendieck pretopology on $\cC$. For every $X$ in $\cC$ define 
$$\cJ(X)=\big\{S\,\big|\,S\mbox{ is a sieve on }X\mbox{ and }S\mbox{ contains some collection in }\cK(X)\big\}$$
Then $\cJ=\{\cJ(X)\}_{X\in \cC}$ is a Grothendieck topology on $\cC$.
\end{proposition}
\begin{proof}
Note that for every object $X$ in $\cC$ we have
$$\big\{f\in \bd{Mor}(\cC)\,\big|\,\mbox{ codomain of }f\mbox{ is }X\big\}=\mbox{a sieve on }X\mbox{ that contains }1_X$$
According to $\{1_X:X\ra X\}\in \cK(X)$, we derive that family $\cJ(X)$ contains the maximal sieve on $X$.\\
Now suppose that $S\in \cJ(X)$ and $f:Y\ra X$. There exists $\{f_i:X_i\ra X\}_{i\in I}\in \cK(X)$ that is contained in $S$. Then $f^*S$ contains $\{f'_i:X_i\times_XY\ra Y\}_{i\in I}$ where $f'_i$ are defined by cartesian squares
\begin{center}
\begin{tikzpicture}
[description/.style={fill=white,inner sep=2pt}]
\matrix (m) [matrix of math nodes, row sep=3em, column sep=2em,text height=1.5ex, text depth=0.25ex] 
{X_i\times_XY &  &   X_i   \\
 Y&   & X   \\} ;
\path[->,line width=0.8pt,font=\scriptsize]
(m-1-1) edge node[above] {$ $} (m-1-3)
(m-2-1) edge node[below] {$ f $} (m-2-3)
(m-1-1) edge node[left] {$ f'_i$} (m-2-1)
(m-1-3) edge node[right] {$f_i $} (m-2-3);
\end{tikzpicture}
\end{center}
Since we have $\{f'_i:X_i\times_XY\ra Y\}_{i\in I}\in \cK(Y)$, we deduce that $f^*S\in \cJ(Y)$.\\
Finally assume that $R$ is a sieve on $X$, $S\in \cJ(X)$ and for every $f\in S$ we have $f^*R\in \cJ(\mathrm{dom}(f))$. By definition there exists $\{f_i:X_i\ra X\}_{i\in I}\in \cK(X)$ contained in $S$ and for every $i\in I$ there exists $\{f_{ij}:X_{ij}\ra X\}_{j\in J_i}\in \cK(X_i)$ contained in $f_i^*R$. Thus $R$ contains $\{f_i\cdot f_{ij}:X_{ij}\ra X\}_{i\in I,j\in J_i}$ and this is a family in $\cK(X)$. Hence $R\in \cJ(X)$.
\end{proof}

\begin{definition}
Let $\cK$ be a Grothendieck pretopology on $\cC$ and $\cJ$ be a Grothendieck topology on $\cC$ given by 
$$\cJ(X)=\big\{S\,\big|\,S\mbox{ is a sieve on }X\mbox{ and }S\mbox{ contains some collection in }\cK(X)\big\}$$
then we say that $\cJ$ is \textit{a Grothendieck topology generated by $\cK$}.
\end{definition}

\begin{definition}
Let $\cJ$ be a Grothendieck topology on $\cC$ and $\cK$ be a Grothendieck pretopology on $\cC$ that generates $\cJ$. Then we say that $\cK$ is \textit{a basis of the Grothendieck topology $\cJ$}.
\end{definition}
\noindent
The next result characterizes sheaves on sites for which Grothendieck topology is generated by some Grothendieck pretopology.

\begin{theorem}\label{theorem:sheavesintermsofkernels}
Let $\cK$ be a Grothendieck pretopology on $\cC$ and $\cJ$ be a topology generated by $\cK$. Then a presheaf $F$ on $\cC$ is a sheaf on with respect to $\cJ$ if and only if for every $\{f_i:X_i\ra X\}_{i\in I}\in \cK(X)$ the diagram
\begin{center}
\begin{tikzpicture}
[description/.style={fill=white,inner sep=2pt}]
\matrix (m) [matrix of math nodes, row sep=3em, column sep=6em,text height=1.5ex, text depth=0.25ex] 
{F(X) &   \prod_{i\in I}F(X_i)& \prod_{(i,j)\in I\times I} F(X_i\times_XX_j)  \\} ;
\path[->,line width=0.8pt,font=\scriptsize]
(m-1-1) edge node[above] {$\langle F(f_i)\rangle_{i\in I} $} (m-1-2)
(m-1-2) edge[transform canvas={yshift=0.5ex}] node[above] {$ \langle F(f'_{ij}) \cdot pr_i\rangle_{(i,j)}$} (m-1-3)
(m-1-2) edge[transform canvas={yshift=-0.5ex}] node[below] {$ \langle F(f''_{ij}) \cdot pr_j\rangle_{(i,j)}$} (m-1-3);
\end{tikzpicture}
\end{center}
is a kernel of a pair of arrows, where for every $(i,j)\in I\times I$ morphisms $f'_{ij}$ and $f'_{ji}$ form a cartesian square
\begin{center}
\begin{tikzpicture}
[description/.style={fill=white,inner sep=2pt}]
\matrix (m) [matrix of math nodes, row sep=3em, column sep=2em,text height=1.5ex, text depth=0.25ex] 
{X_i\times_XX_j &  &   X_j   \\
 X_i&   & X   \\} ;
\path[->,line width=0.8pt,font=\scriptsize]
(m-1-1) edge node[above] {$ f''_{ij}$} (m-1-3)
(m-2-1) edge node[below] {$ f_i $} (m-2-3)
(m-1-1) edge node[left] {$ f'_{ij} $} (m-2-1)
(m-1-3) edge node[right] {$ f_j  $} (m-2-3);
\end{tikzpicture}
\end{center}
\end{theorem}
\begin{proof}
Suppose that $F$ is a sheaf with respect to $\cJ$ and $\cF = \{f_i:X_i\ra X\}_{i\in I}$ be a collection in $\cK(X)$. Let $S$ be a sieve generated by $\{f_i\}_{i\in I}$. Then according to Proposition \ref{proposition:matchingfamiliesaskernels} we deduce that the diagram
\begin{center}
\begin{tikzpicture}
[description/.style={fill=white,inner sep=2pt}]
\matrix (m) [matrix of math nodes, row sep=3em, column sep=6em,text height=1.5ex, text depth=0.25ex] 
{F(S) &   \prod_{i\in I}F(X_i)&  \prod_{(i,j)\in I\times I} F(X_i\times_XX_j)  \\} ;
\path[->,line width=0.8pt,font=\scriptsize]
(m-1-1) edge node[above] {$\mathrm{can}_{\cF} \cdot \mathrm{res}_{S,\cF}^{-1} $} (m-1-2)
(m-1-2) edge[transform canvas={yshift=0.5ex}] node[above] {$ \langle F(f'_{ij}) \cdot pr_i\rangle_{(i,j)}$} (m-1-3)
(m-1-2) edge[transform canvas={yshift=-0.5ex}] node[below] {$ \langle F(f''_{ij}) \cdot pr_j\rangle_{(i,j)}$} (m-1-3);
\end{tikzpicture}
\end{center}
is a kernel diagram. Since $F$ is sheaf in $\cJ$ and $S\in \cJ(X)$, we derive that the map $\mathrm{res}_S:F(X)\ra F(S)$ that sends $x\in F(X)$ to $\{F(f)(x)\}_{f\in S}$ is a bijection. Hence 
$$\langle F(f_i)\rangle_{i\in I} = \mathrm{can}_{\cF}\cdot \mathrm{res}_{S,\cF}^{-1}\cdot \mathrm{res}_S:F(X)\ra \prod_{i\in I}F(X_i)$$
is a kernel of a pair consisting of $\langle F(f'_{ij}) \cdot pr_i\rangle_{(i,j)}$ and $\langle F(f''_{ij}) \cdot pr_j\rangle_{(i,j)}$.\\
Now assume that $F$ is a presheaf on $\cC$ and for every collection $\{f_i:X_i\ra X\}_{i\in I}$ in $\cK(X)$ the diagram 
\begin{center}
\begin{tikzpicture}
[description/.style={fill=white,inner sep=2pt}]
\matrix (m) [matrix of math nodes, row sep=3em, column sep=6em,text height=1.5ex, text depth=0.25ex] 
{F(X) &   \prod_{i\in I}F(X_i)& \prod_{(i,j)\in I\times I} F(X_i\times_XX_j)  \\} ;
\path[->,line width=0.8pt,font=\scriptsize]
(m-1-1) edge node[above] {$\langle F(f_i)\rangle_{i\in I} $} (m-1-2)
(m-1-2) edge[transform canvas={yshift=0.5ex}] node[above] {$ \langle F(f'_{ij}) \cdot pr_i\rangle_{(i,j)}$} (m-1-3)
(m-1-2) edge[transform canvas={yshift=-0.5ex}] node[below] {$ \langle F(f''_{ij}) \cdot pr_j\rangle_{(i,j)}$} (m-1-3);
\end{tikzpicture}
\end{center}
is a kernel pair. Now Proposition \ref{proposition:matchingfamiliesaskernels} implies that for any object $X$ and sieve $S$ generated by a collection in $\cK(X)$ every matching family for $S$ of elements of $F$ admits a unique amalgamation. In other words for every sieve $S$ on $X$ generated by some collection in $\cK(X)$ the map $\mathrm{res}_S:F(X)\ra F(S)$ that sends $x\in F(X)$ to $\{F(f)(x)\}_{f\in S}$ is bijective. Consider now any sieve $R$ in $\cJ(X)$. Then there exists a sieve $S$ on $X$ generated by some collection of $\cK(X)$ such that $S\subseteq R$. Consider a commutative triangle
\begin{center}
\begin{tikzpicture}
[description/.style={fill=white,inner sep=2pt}]
\matrix (m) [matrix of math nodes, row sep=2em, column sep=1em,text height=1.5ex, text depth=0.25ex] 
{ F(R)&      &  F(S)  \\
      &F(X)&          \\} ;
\path[->,line width=0.8pt,font=\scriptsize]
(m-1-1) edge node[above] {$ \mathrm{res}_{R,S} $} (m-1-3)
(m-2-2) edge node[below = 3pt, left = 1pt] {$ \mathrm{res}_{R} $} (m-1-1)
(m-2-2) edge node[below = 3pt, right = 1pt] {$ \mathrm{res}_{S} $} (m-1-3);
\end{tikzpicture}
\end{center}
where $\mathrm{res}_{R,S}\left(\{x_f\}_{f\in R}\right)=\{x_f\}_{f\in S}$, $\mathrm{res}_{R}(x)=\{F(f)(x)\}_{f\in R}$ and $\mathrm{res}_{S}(x)=\{F(f)(x)\}_{f\in S}$. By what we prove above, we deduce that $\mathrm{res}_S$ is a bijection. Hence $\mathrm{res}_R$ is injective. Thus $F$ is a separated presheaf with respect to $\cJ$. By Proposition \ref{proposition:separatdpresheavesinduceinjections} the map $\mathrm{res}_{R,S}$ is injective. Therefore, $\mathrm{res}_{R,S}$, $\mathrm{res}_R$ are injective and $\mathrm{res}_S$ is bijective and they form a commutative triangle. Hence they are all bijective maps of classes. In particular, $\mathrm{res}_R$ is bijective. We deduce that $F$ is a sheaf with respect to $\cJ$.
\end{proof}

\section{Dense subsites}

\begin{proposition}\label{proposition:densesubcategoriescharacterization}
Let $(\cC,\cJ)$ be a site and $\cK$ be its full subcategory. Then the following are equivalent.
\begin{enumerate}[label=\emph{\textbf{(\roman*)}}, leftmargin=1.5em]
\item For every object $X$ of $\cC$ and every $S$ covering sieve in $\cJ(X)$ there exists a sieve $R$ in $\cJ(X)$ generated by a collection of morphisms with domains in $\cK$ and contained in $S$.
\item For every object $X$ of $\cC$ there exists a covering sieve $S$ of $X$ generated by a collection of morphisms in $\cC$ with domains in $\cK$.
\end{enumerate}
\end{proposition}
\begin{proof}
The implication $\textbf{(i)}\Rightarrow \textbf{(ii)}$ is obvious. We prove $\textbf{(ii)}\Rightarrow \textbf{(i)}$. Let $f:Y\ra X$ be a morphism in $S$. Since $\cK$ is dense subcategory of the site $(\cC,\cJ)$, we derive that there exists a covering sieve $R_f$ in $\cJ(Y)$ generated by a collection of morphisms with domains in $\cK$. Now a collection
$$R = \bigcup_{f\in S}f\cdot R_f$$
is a covering sieve on $X$ by Fact \ref{fact:compositionofcoveringsieves}. It is also contained in $S$ and is generated by morphisms with domains in $\cK$.
\end{proof}

\begin{definition}
Let $(\cC, \cJ)$ be a site and $\cK$ be a full subcategory of $\cC$ satisfying equivalent condition of Proposition \ref{proposition:densesubcategoriescharacterization}. Then $\cK$ is called \textit{a dense subcategory of a site $(\cC,\cJ)$}.
\end{definition}

\begin{corollary}\label{corollary:sievesondense}
Let $(\cC,\cJ)$ be a site and $\cK$ be its dense subcategory. Fix an object $X$ of $\cK$ and a sieve $T$ in $\cJ(X)$. Then $T\cap \cK$ generates a sieve in $\cC$ contained in $\cJ(X)$.
\end{corollary}
\begin{proof}
By Proposition \ref{proposition:densesubcategoriescharacterization} we derive that there exists a sieve $R$ in $\cJ(X)$ contained in $T$ and generated by morphisms in $\cK$. Now a sieve in $\cC$ generated by $T\cap \cK$ contains $R$ and hence is an element of $\cJ(X)$ according to Proposition \ref{proposition:coveringsievesproperties}.
\end{proof}

\begin{corollary}\label{corollary:topologygeneratedondense}
Let $(\cC, \cJ)$ be a site and $\cK$ be its dense subcategory. For an object $X$ of $\cK$ we define $\cJ_{\cK}(X)$ as a collection of all sieves on $X$ of the form $T\cap \cK$ for $T$ in $\cJ(X)$. Then $\cJ_{\cK}$ is a Grothendieck topology on $\cK$.
\end{corollary}
\begin{proof}
Let $X$ be an object of $\cK$. The maximal sieve on $X$ in $\cK$ is the intersection of the maximal sieve on $X$ in $\cC$ and $\cK$. Hence the former is an element of $\cJ_{\cK}(X)$.\\
Suppose next that $T$ is a sieve in $\cJ(X)$ for some object $X$ of $\cK$ and let $f:Y\ra X$ be a morphism in $\cK$. Then $f^*T\in \cJ(Y)$ and since we have $f^*(T\cap \cK) \subseteq f^*T\cap \cK$, we deduce that $f^*(T\cap \cK)\in \cJ_{\cK}(Y)$. Thus pullback of an element of $\cJ_{\cK}(X)$ by $f$ is in $\cJ_{\cK}(Y)$.\\
Finally suppose that $X$ is an object of $\cK$ and $S$, $R$ are sieves on $X$ in $\cK$. Assume that $S\in \cJ_{\cK}(X)$ and $f^*R\in \cJ_{\cK}(\mathrm{dom}(f))$ for every $f\in S$. Let $T$ be a sieve in $\cC$ generated by $R$. Then for every $f\in S$ we have $f^*R\subseteq f^*T$. Since $f^*R\in \cJ_{\cK}(\mathrm{dom}(f))$, we deduce by Corollary \ref{corollary:sievesondense} that sieve in $\cC$ generated by $f^*R$ is in $\cJ(\mathrm{dom}(f))$. This also shows $f^*T \in \cJ(\mathrm{dom}(f))$. Therefore, $f^*T$ is a covering sieve in $\cC$ for every $f\in S$. Since $S$ generates a covering sieve in $\cC$ by Corollary \ref{corollary:sievesondense}, we deduce that $T\in \cJ(X)$. Note that $R = T\cap \cK$ and hence $R\in \cJ_{\cK}(X)$.
\end{proof}

\begin{definition}
Let $(\cC,\cJ)$ be a site and $\cK$ be its dense subcategory. Then the Grothendieck topology $\cJ_{\cK}$ on $\cK$ described in Corollary \ref{corollary:topologygeneratedondense} is called \textit{the induced topology on $\cK$} and a pair $(\cK,\cJ_{\cK})$ is called \textit{a dense subsite of $(\cC,\cJ)$}.
\end{definition}

\begin{theorem}\label{theorem:densesubcategoriesandsheaves}
Let $\left(\cC,\cJ\right)$ be a site and $\cK\subseteq \cC$ be a dense subcategory. Then the embedding $\cK \hookrightarrow \cC$ induces a full and faithful functor
$$\Sh(\cC,\cJ)\ra \Sh(\cK,\cJ_{\cK})$$
Moreover, if for every object $X$ of $\cC$ there exists a covering sieve $S$ in $\cJ(X)$ generated by a set of morphisms with domains in $\cK$, then this functor is an equivalence of categories.
\end{theorem}
\noindent
The proof is a bit technical and for clarity we divide it into lemmas that encapsulate main steps of the argument. First we need to introduce some notation. The functor $\Sh(\cC,\cJ)\ra \Sh(\cK,\cJ_{\cK})$ is the restriction of the functor $\widehat{\cC}\ra \widehat{\cK}$ induced by the inclusion $\cK\hookrightarrow \cC$. We denote values of $\widehat{\cC}\ra \widehat{\cK}$ by $(-)_{\mid \cK}$, where placeholder $-$ stands either for a presheaf in $\cC$ or for a morphism of such presheaves.

\begin{lemma}\label{lemma:fullandfatiful}
The functor
$$\Sh(\cC,\cJ)\ra \Sh(\cK,\cJ_{\cK})$$
induced by the embedding $\cK \hookrightarrow \cC$ is full and faitful.
\end{lemma}
\begin{proof}[Proof of the lemma]
For an object $X$ of $\cC$ and a covering sieve $S$ in $\cJ(X)$ we denote by $S_{\mid \cK}$ the class of all morphisms in $S$ with domains in $\cK$. Suppose that $S$ is generated by $S_{\mid \cK}$. Let $F, G$ be sheaves on $(\cC,\cJ)$ and let $\tau:F_{\mid \cK}\ra G_{\mid \cK}$ be a morphism of presheaves. Then $\tau$ induces map $\tau_S:F(S_{\mid \cK})\ra G(S_{\mid \cK})$. Indeed, if $S_{\mid \cK} = \big\{f_i:X_i\ra X\big\}_{i\in I}$, then
\begin{center}
\begin{tikzpicture}
[description/.style={fill=white,inner sep=2pt}]
\matrix (m) [matrix of math nodes, row sep=3em, column sep=4em,text height=1.5ex, text depth=0.25ex] 
{\prod_{i\in I}F(X_i)          & \prod_{i\in I}G(X_i)       \\
 F(S_{\mid \cK})                      & G(S_{\mid \cK})                   \\} ;
\path[->,line width= 0.8pt,font=\scriptsize]
(m-1-1) edge node[above] {$ \prod_{i\in I}\tau_{X_i} $} (m-1-2)
(m-2-1) edge node[below] {$\tau_S $} (m-2-2);
\path[right hook ->,line width= 0.8pt,font=\scriptsize]
(m-2-1) edge node[left] {$ $} (m-1-1)
(m-2-2) edge node[right] {$ $} (m-1-2);
\end{tikzpicture}
\end{center}
where vertical injections are canonical. Since $F$, $G$ are sheaves on $(\cC,\cJ)$, we derive by Proposition \ref{proposition:matchingfamiliesaskernels} that $F(S_{\mid \cK})$ and $G(S_{\mid \cK})$ can be identified with $F(X)$ and $G(X)$, respectively. Thus $\tau$ induces a map $\tau_X:F(X)\ra G(X)$. Morphism $\tau_X$ does not depend on a covering sieve $S$ on $X$ generated by $S_{\mid \cK}$. This is a consequence of Proposition \ref{proposition:densesubcategoriescharacterization}. Thus the meaning of the symbol $\tau_X$ for an object $X$ already in $\cK$ is unambiguous, which follows from applying the definiton above to the maximal sieve on $X$. Now suppose that $f:X\ra Y$ is a morphism in $\cC$ and $S, R$ are covering sieves in $\cC$ on $X$, $Y$, respectively. Assume that $S, R$ are generated by $S_{\mid \cK}, R_{\mid \cK}$, respectively and $S\subseteq f^*R$. We have a commutative diagram
\begin{center}
\begin{tikzpicture}
[description/.style={fill=white,inner sep=2pt}]
\matrix (m) [matrix of math nodes, row sep=3em, column sep=5em,text height=1.5ex, text depth=0.25ex] 
{F(R_{\mid \cK})                                        &    G(R_{\mid \cK})                                             \\
 F\left(\left(f^*R\right)_{\mid \cK}\right)             &    G\left(\left(f^*R\right)_{\mid \cK}\right)                   \\
 F(S_{\mid \cK})                                        &    G(S_{\mid \cK})                                               \\} ;
\path[->,line width= 0.8pt,font=\scriptsize]
(m-1-1) edge node[above] {$\tau_R $} (m-1-2)
(m-2-1) edge node[below] {$\mbox{induced by }\tau $} (m-2-2)
(m-3-1) edge node[below] {$\tau_S $} (m-3-2)
(m-1-1) edge node[left] {$ $} (m-2-1)
(m-1-2) edge node[right] {$ $} (m-2-2);
\path[right hook ->,line width= 0.8pt,font=\scriptsize]
(m-3-2) edge node[right] {$\mbox{inclusion} $} (m-2-2)
(m-3-1) edge node[left] {$\mbox{inclusion} $} (m-2-1);
\end{tikzpicture}
\end{center}
in which left hand side vertical arrow with no labels is given by
$$F(R_{\mid \cK})\ni \{x_h\}_{h\in R_{\mid \cK}}\mapsto \{y_g\}_{g\in \left(f^*R\right)_{\mid \cK}} \in F\left(\left(f^*R\right)_{\mid \cK}\right)$$
where $y_g = x_{f\cdot g}$ for $g\in \left(f^*R\right)_{\mid \cK}$ and analogically we define right hand side vertical arrow with no label in the diagram. This diagram implies that $\tau_X\cdot F(f) = F(f)\cdot \tau_Y$ by Proposition \ref{proposition:matchingfamiliesaskernels}. This extends $\tau:F_{\mid \cK}\ra G_{\mid \cK}$ to a morphism of sheaves $F\ra G$. This extension is unique by its definition.
\end{proof}
\noindent
Suppose that $X$ is an object of $\cC$. Since a sieve $S$ on $X$ in $\cC$ can be identified with a big subpresheaf of big presheaf $h_X$, the restriction $S_{\mid \cK}$ makes sense (it also has the same meaning as the one introduced in the proof of Lemma \ref{lemma:fullandfatiful}). Note also that if $X$ is an object of $\cK$, then $\left(h_X\right)_{\mid \cK} = k_X$, where $k_X$ denotes big representable presheaf on $\cK$. 

\begin{lemma}\label{lemma:extensionofmorphismsforsheavesinsubcategory}
Let $F$ be a sheaf on $(\cK,\cJ_{\cK})$. Consider an object $X$ of $\cC$ and a covering sieve $S$ in $\cJ(X)$. Then for every morphism $\sigma:S_{\mid \cK}\ra F$ there exists a unique morphism $\tilde{\sigma}:\left(h_X\right)_{\mid \cK}\ra F$ making the following diagram commutative.
\begin{center}
\begin{tikzpicture}
[description/.style={fill=white,inner sep=2pt}]
\matrix (m) [matrix of math nodes, row sep=3em, column sep=3em,text height=1.5ex, text depth=0.25ex] 
{ \left(h_X\right)_{\mid \cK}&       F  \\
   S_{\mid \cK} &          \\} ;
\path[densely dotted, ->,line width=0.8pt,font=\scriptsize]
(m-1-1) edge node[above] {$ \tilde{\sigma} $} (m-1-2);
\path[left hook ->,line width=0.8pt,font=\scriptsize]
(m-2-1) edge node[left] {$ $} (m-1-1);
\path[->,line width=0.8pt,font=\scriptsize]
(m-2-1) edge node[below = 3pt, right = 1pt] {$ \sigma $} (m-1-2);
\end{tikzpicture}
\end{center}
\end{lemma}
\begin{proof}[Proof of the lemma]
Fix $\sigma:S_{\mid \cK}\ra F$. Next fix a morphism $f:Y\ra X$ in $\cC$ with $Y$ in $\cK$. Consider a sieve $R_f = f^*S\cap \cK$ in $\cK$ as a presheaf on $\cK$. Then  we have a morphism
$$R_f\ni g\mapsto f\cdot g\in S_{\mid \cK}$$
of presheaves on $\cK$. We denote the composition of this morphism with $\sigma$ by $\sigma_f$. Thus $\sigma_f:R_f\ra F$ is a morphism of presheaves. Since $F$ is a sheaf on $(\cK,\cJ_{\cK})$ and $R_f\in \cJ_{\cK}(Y)$, we derive that there exists a unique morphism $\tilde{\sigma}_f:k_{Y}\ra F$ such that
the triangle
\begin{center}
\begin{tikzpicture}
[description/.style={fill=white,inner sep=2pt}]
\matrix (m) [matrix of math nodes, row sep=3em, column sep=3em,text height=1.5ex, text depth=0.25ex] 
{ k_{Y} &       F  \\
  R_f &          \\} ;
\path[densely dotted, ->,line width=0.8pt,font=\scriptsize]
(m-1-1) edge node[above] {$ \tilde{\sigma}_f $} (m-1-2);
\path[left hook ->,line width=0.8pt,font=\scriptsize]
(m-2-1) edge node[left] {$ $} (m-1-1);
\path[->,line width=0.8pt,font=\scriptsize]
(m-2-1) edge node[below = 3pt, right = 1pt] {$ \sigma_f $} (m-1-2);
\end{tikzpicture}
\end{center}
is commutative. We define $\tilde{\sigma}:\left(h_X\right)_{\mid \cK}\ra F$ by formula $\tilde{\sigma}(f) = \tilde{\sigma}_f(1_Y)$. Then $\tilde{\sigma}$ satisfies conditions in the statement.
\end{proof}
\noindent
Let $F$ be a presheaf on $\cK$. Suppose that for every object $X$ in $\cC$ the class $\Mor_{\cK}\left(\left(h_X\right)_{\mid \cK},F\right)$ (of morphisms of big presheaves) is a set. Then we denote the presheaf $X\mapsto \Mor_{\cK}\left(\left(h_X\right)_{\mid \cK},F\right)$ on $\cC$ by $\tilde{F}$. If $X$ is in $\cK$, then we have a bijection
$$\Mor_{\cK}\left(\left(h_X\right)_{\mid \cK},F\right) = \Mor_{\cK}\left(k_X , F\right)\ni \tau \mapsto \tau(1_X)\in F(X)$$
This bijection is natural in object $X$ of $\cK$ and hence $\tilde{F}_{\mid \cK}$ can be identified with $F$. This defines an isomorphism $\xi_F:\tilde{F}_{\mid \cK}\ra F$.

\begin{lemma}\label{lemma:counitandsection}
Let $F$ be a presheaf on $\cK$ and let $X$ be an object of $\cK$. Suppose that for every object $X$ in $\cC$ the class $\Mor_{\cK}\left(\left(h_X\right)_{\mid \cK},F\right)$ is a set. Fix a morphism $\phi:k_X\ra \tilde{F}_{\mid \cK}$ of big presheaves. Then $\xi_F\cdot \phi = \phi(1_X)$.
\end{lemma}
\begin{proof}[Proof of the lemma]
We denote $\phi(1_X)$ by $\tau$. Then for every morphism $f:Y\ra X$ in $\cK$ we have
$$\left(\xi_{F}\cdot \phi\right)(f) = \xi_F\left(\phi(f)\right) = \xi_F\left(\tau\cdot \left(h_f\right)_{\mid \cK}\right) = \left(\tau\cdot \left(h_f\right)_{\mid \cK}\right)(1_Y) = \tau(f)$$
and hence $\xi_F\cdot \phi = \tau$.
\end{proof}

\begin{lemma}\label{lemma:extensionofsheafissheaf}
Let $F$ be a sheaf on $(\cK,\cJ_{\cK})$. Suppose that for every object $X$ in $\cC$ the class $\Mor_{\cK}\left(\left(h_X\right)_{\mid \cK},F\right)$ is a set. Then the presheaf $\tilde{F}$ on $\cC$ is a sheaf on $(\cC,\cJ)$.
\end{lemma}
\begin{proof}[Proof of the lemma]
Fix object $X$ in $\cC$ and a sieve $S$ in $\cJ(X)$. Assume that $S$ is generated by morphisms with domain in $\cK$. Consider a morphism $\sigma:S\ra \tilde{F}$. Then $\sigma_{\mid \cK}:S_{\mid \cK}\ra \tilde{F}_{\mid \cK}$ composed with the identification $\xi_F:\tilde{F}_{\mid \cK}\ra F$ gives rise to a morphism $\tau:S_{\mid \cK}\ra F$. According to Lemma \ref{lemma:extensionofmorphismsforsheavesinsubcategory} we derive that there exists $\tilde{\tau}:\left(h_X\right)_{\mid \cK}\ra F$ that extends $\tau$. Now suppose that $f:Y\ra X$ is a morphism in $S$ with $Y$ being object of $\cK$. Let $i_f:k_Y\ra S_{\mid \cK}$ be a morphism determined by $f$ and let $i:S\hookrightarrow h_X$ be the inclusion. Note that $\left(h_f\right)_{\mid \cK} = i_{\mid \cK}\cdot i_f$. We have (one step in the chain below requires Lemma \ref{lemma:counitandsection})
$$\tilde{F}(f)\left(\tilde{\tau}\right) =\tilde{\tau}\cdot \left(h_f\right)_{\mid \cK} = \tilde{\tau}\cdot i_{\mid \cK}\cdot i_{f} = \tau\cdot i_{f} = \xi_F\cdot \sigma_{\mid \cK}\cdot i_f = \left( \sigma_{\mid \cK}\cdot i_f \right)(1_Y) = \sigma_{\mid \cK}(f) = \sigma(f)$$
This equality implies that $\tilde{\tau}$ is an amalgamation of a matching family $\sigma$ and moreover, it shows that it is a unique amalgamation.\\
Now we proceed as in the proof of Theorem \ref{theorem:sheavesintermsofkernels}. Consider any sieve $R$ in $\cJ(X)$. Then by Proposition \ref{proposition:densesubcategoriescharacterization} there exists a sieve $S$ on $X$ generated by morphisms with domains in $\cK$ such that $S\subseteq R$. Consider a commutative triangle
\begin{center}
\begin{tikzpicture}
[description/.style={fill=white,inner sep=2pt}]
\matrix (m) [matrix of math nodes, row sep=2em, column sep=1em,text height=1.5ex, text depth=0.25ex] 
{ \tilde{F}(R)&      &  \tilde{F}(S)  \\
      &\tilde{F}(X)&          \\} ;
\path[->,line width=0.8pt,font=\scriptsize]
(m-1-1) edge node[above] {$ \mathrm{res}_{R,S} $} (m-1-3)
(m-2-2) edge node[below = 3pt, left = 1pt] {$ \mathrm{res}_{R} $} (m-1-1)
(m-2-2) edge node[below = 3pt, right = 1pt] {$ \mathrm{res}_{S} $} (m-1-3);
\end{tikzpicture}
\end{center}
where $\mathrm{res}_{R,S}$ is the restriction of matching families, $\mathrm{res}_{R}$ and $\mathrm{res}_{S}$ send elements of $\tilde{F}(X)$ to matching families on $R$ and $S$, respectively. By what we prove above we deduce that $\mathrm{res}_S$ is a bijection. Hence $\mathrm{res}_R$ is injective. Thus $\tilde{F}$ is a separated presheaf with respect to $\cJ$. By Proposition \ref{proposition:separatdpresheavesinduceinjections} the map $\mathrm{res}_{R,S}$ is injective. Therefore, $\mathrm{res}_{R,S}$, $\mathrm{res}_R$ are injective and $\mathrm{res}_S$ is bijective and they form a commutative triangle. Hence they are all bijective maps of classes. In particular, $\mathrm{res}_R$ is bijective. We deduce that $\tilde{F}$ is a sheaf with respect to $\cJ$.
\end{proof}

\begin{proof}[Proof of the theorem]
Lemma \ref{lemma:fullandfatiful} implies that the functor in question is full and faithful. Now suppose that every object $X$ in $\cC$ admits a covering sieve $S$ in $\cJ(X)$ generated by a set of morphisms with domains in $\cK$. Fix $X$ in $\cC$ and a sheaf $F$ on $\left(\cK,\cJ_{\cK}\right)$. Let $\cF = \big\{f_i:X_i\ra X\big\}_{i\in I}$ be a set of morphisms generating some covering sieve $S\in \cJ(X)$. Then $\Mor_{\cK}\left(S_{\mid \cK},F\right)$ is a set and by Lemma \ref{lemma:extensionofmorphismsforsheavesinsubcategory} we deduce that $\Mor_{\cK}\left(\left(h_X\right)_{\mid \cK},F\right)$ is a set. Therefore, for every sheaf $F$ on $\left(\cK,\cJ_{\cK}\right)$ presheaf $\tilde{F}$ exists. By Lemma \ref{lemma:extensionofsheafissheaf} it is a sheaf on $\left(\cC,\cJ\right)$ such that $\xi_{F}:\tilde{F}_{\mid \cK}\ra F$ is an isomorphism. Thus the functor
$$\Sh(\cC,\cJ)\ra \Sh(\cK,\cJ_{\cK})$$
is essentially surjective. Since it is full and faithful as it was previously established, we deduce that it is an equivalence of categories.
\end{proof}

\section{Sheaf associated to a presheaf}

\begin{definition}
Let $\cC$ be a category and let $\cJ$ be a Grothendieck topology on $\cC$. Suppose that for every object $X$ of $\cC$ there exists a set $\cS$ of covering sieves on $X$ such that for every covering sieve $R$ in $\cJ(X)$ there exists $S$ in $\cS$ contained in $R$ and every $S$ in $\cS$ is generated by some set of morphisms. Then $\cJ$ is called \textit{a small Grothendieck topology on $\cC$}.
\end{definition}

\begin{fact}
Let $\cC$ be a category and $\cK$ be a Grothendieck pretopology on $\cC$. Suppose that for every $X$ in $\cC$ the class $\cK(X)$ is a set and every member of $\cK(X)$ consists of a set of arrows. Then the Grothendieck topology generated by $\cK$ is small.
\end{fact}
\begin{proof}
Left to the reader.
\end{proof}

\begin{theorem}\label{theorem:associatedsheaf}
Let $F$ be a presheaf on a Grothendieck site $(\cC,\cJ)$ and assume that $\cJ$ is a small Grothendieck topology. There exists a sheaf $a(F)$ and a morphism $\eta_F:F \ra a(F)$ of presheaves such that for every sheaf $G$ and every morphism of presheaves $p:F \ra G$ there exists a unique morphism $r:a(F)\ra G$ making the diagram
\begin{center}
\begin{tikzpicture}
[description/.style={fill=white,inner sep=2pt}]
\matrix (m) [matrix of math nodes, row sep=3em, column sep=3em,text height=1.5ex, text depth=0.25ex] 
{ a(F)& G   \\
  F   &    \\} ;
\path[->,line width=0.8pt, font=\scriptsize]  
(m-1-1) edge node[above] {$r $} (m-1-2)
(m-2-1) edge node[left]  {$\eta_F $} (m-1-1) 
(m-2-1) edge node[below = 8pt, right = -5pt]  {$p $} (m-1-2);
\end{tikzpicture}
\end{center}
commutative.
\end{theorem}
\noindent
First we construct a separated presheaf $F^+$ out of $F$. Fix an object $X$ of $\cC$. Suppose that $S$ is a covering sieve on $X$. Denote by $F(S)$ the set of all matching families for $S$ of elements of $F$. If $S_1\subseteq S_2$ are covering sieves on $X$, then we have a function $F(S_2)\ra F(S_1)$ given by restriction. Thus $\{F(S)\}_{S\in \cJ(X)}$ is a diagram indexed by a directed set $\cJ(X)$ and we define
$$F^+(X)=\mathrm{colim}_{S\in \cJ(X)}F(S)$$
Note that for every morphism $f:X_1\ra X_2$ in $\cC$ and for every sieve $S\in \cJ(X_2)$ we have a function $F(S)\ra F(f^*S)$ given by $F(S)\ni \{s_g\}_{g\in S}\mapsto \{s_{f\cdot g}\}_{g\in f^*S}\in F(f^*S)$. These functions for all $S\in \cJ(X_2)$ induce a map
$$F^+(X_2)\ra F^+(X_1)$$
and this defines a presheaf $F^+$ according to the fact that $\cJ$ is small. We also have a morphism of presheaves $i_F^+:F\ra F^+$ that sends $x\in F(X)$ to a class in $F^+(X)$ represented by a matching family of the form $\{F(f)(x)\}_{f\in S}$ for every covering sieve $S$ on $X$.
\begin{lemma}\label{lemma:plusconstruction}
The following assertions hold.
\begin{enumerate}[label=\emph{\textbf{(\arabic*)}}, leftmargin=1.5em]
\item $F^+$ is a separated presheaf.
\item If $F$ is separated presheaf, then $F^+$ is a sheaf.
\end{enumerate} 
\end{lemma}
\begin{proof}[Proof of the lemma]
We prove \textbf{(1)}. Fix an object $X\in \cC$ and a covering sieve $S$ on $X$. Suppose that $\{x_f\}_{f\in S}$ is a matching family for $S$ of elements of $F^+$. Assume that $y$, $z\in F^+(X)$ are amalgamations of $\{x_f\}_{f\in S}$. Then there exists a covering sieve $T$ on $X$ such that $y$ is represented by some matching family $\{s_f\}_{f\in T}$ for $T$ of elements of $F$ and $z$ is represented by some matching family $\{t_f\}_{f\in T}$ for $T$ of elements of $F$. Fix a morphism $f:Y\ra X$ in $S$. Then $F^+(f)(y)$ is represented by $\{s_{f\cdot g}\}_{g\in f^*T}$ and $F^+(f)(z)$ is represented by $\{t_{f\cdot g}\}_{g\in f^*T}$. Moreover, $F^+(f)(y)=x_f=F^+(f)(z)$ and hence there exists a covering sieve $R_f$ on $Y$ such that $R_f\subseteq f^*T$ and $s_{f\cdot g}=t_{f\cdot g}$ for every $g\in R_f$. Now we know that 
$$R=\bigcup_{f\in T}f\cdot R_f\subseteq S$$
is a covering sieve on $X$ and matching families $\{s_f\}_{f\in R}$, $\{t_f\}_{f\in R}$ for $R$ of elements of $F$ represent respectively $y$ and $z$. Since these families are equal, we derive that $y=z$. This implies that $F^+$ is separated.\\
Let us prove \textbf{(2)}. Fix an object $X\in \cC$ and a covering sieve $S$ on $X$. Suppose that $\{x_f\}_{f\in S}$ is a matching family for $S$ of elements of $F^+$. For every $f:Y\ra X$ in $S$ there exists a covering sieve $R_f$ on $Y$ and a matching family $\{s(f)_g\}_{g\in R_f}$ for $R_f$ of elements of $F$ that represents $x_f$. Formula
$$R=\bigcup_{f\in S}f\cdot R_f$$
defines a covering sieve on $X$ contained in $S$. We set $r_{f\cdot g}=s(f)_g$ for every $f\in S$ and $g\in R_f$. We check now that this definition is independent of choices of $f\in S$ and $g\in R_f$. For this suppose that $f_1$, $f_2\in S$ and $g_1\in R_{f_1}$, $g_2\in R_{f_2}$ satisfy $f_1\cdot g_1=f_2\cdot g_2$. Let $Z\in \cC$ denote a common domain of morphisms $g_1$, $g_2$. Now $F^+(g_1)(x_{f_1})$ is represented by a matching family $\{s(f_1)_{g_1\cdot g}\}_{\mathrm{cod}(g)=Z}$ and $F^+(g_2)(x_{f_2})$ is represented by a matching family $\{s(f_2)_{g_2\cdot g}\}_{\mathrm{cod}(g)=Z}$. According to equality
$$F^+(g_1)(x_{f_1})=x_{f_1\cdot g_1}=x_{f_2\cdot g_2}=F^+(g_2)(x_{f_2})$$
these families represent the same element of $F^+(Z)$. Hence we deduce that there exists a covering sieve $T$ on $Z$ such that $\{s(f_1)_{g_1\cdot g}\}_{g\in T}=\{s(f_2)_{g_2\cdot g}\}_{g\in T}$. Next $s(f_1)_{g_1}$ is an amalgamation for $\{s(f_1)_{g_1\cdot g}\}_{g\in T}$ and $s(f_2)_{g_2}$ is an amalgamation for $\{s(f_2)_{g_2\cdot g}\}_{g\in T}$. By separatedness of $F$, we derive that $s(f_1)_{g_1}=s(f_2)_{g_2}$. Thus family $\{r_f\}_{f\in R}$ is well defined. By definition it is a matching family for $R$ of elements of $F$. Hence it defines an element of $F(R)$ and this element represents some $x\in F^+(X)$. Fix now $f\in S$. By definition of $F^+$ we deduce that $F^+(f)(x)$ is represented by $\{r_{f\cdot g}\}_{g\in f^*R}$. This family contains $\{r_{f\cdot g}\}_{g\in R_f}=\{s(f)_{g}\}_{g\in R_f}$ and thus $F^+(f)(x)=x_f$. This proves that $\{x_f\}_{f\in S}$ admits an amalgamation. By \textbf{(1)} presheaf $F$ is separated. Hence amalgamation of $\{x_f\}_{f\in S}$ is unique.
\end{proof}

\begin{lemma}\label{lemma:universalpropertyofplusconstruction}
Let $p:F\ra G$ be a morphism of presheaves and assume that $G$ is a sheaf. Then there exists a unique morphism $q:F^+\ra G$ such that the diagram
\begin{center}
\begin{tikzpicture}
[description/.style={fill=white,inner sep=2pt}]
\matrix (m) [matrix of math nodes, row sep=3em, column sep=3em,text height=1.5ex, text depth=0.25ex] 
{ F^+& G   \\
  F   &    \\} ;
\path[->,line width=0.8pt, font=\scriptsize]  
(m-1-1) edge node[above] {$q $} (m-1-2)
(m-2-1) edge node[left]  {$i^+_F $} (m-1-1) 
(m-2-1) edge node[below = 8pt, right = -5pt]  {$p $} (m-1-2);
\end{tikzpicture}
\end{center}
is commutative.
\end{lemma}
\begin{proof}[Proof of the lemma]
Fix $X\in \cC$ and $x\in F^+(X)$. Then there exists a covering sieve $S$ on $X$ and a matching family $\{s_f\}_{f\in S}$ for $S$ of elements of $F$ that represents $x$. By definitions of $F^+$ and $i_F^+$ we have matching family $\{i_F^+(s_f)\}_{f\in S}$ for $S$ of elements of $F^+$ with $x$ as its amalgamation.\\
Assume that $q:F^+\ra G$ is a morphism such that $p=q\cdot i^+_F$.  We have $p(s_f)=q(i_F^+(s_f))$ for every $f\in S$. Therefore, $q(x)$ must be an amalgamation of a matching family $\{p(s_f)\}_{f\in S}=\{q(i^+_F(s_f))\}_{f\in S}$ for $S$ of elements of $G$. Since $G$ is a separated presheaf, there exists at most one such amalgamation. This proves uniqueness of $q$.\\
The existence of such a $q$ is also evident. As $G$ is a sheaf, one picks $q(x)$ to be the amalgamation of a matching family $\{p(s_f)\}_{f\in S}$ for $S$ of elements of $G$. Verification that uses definitions of $F^+$ and $i^+_F$ shows that this gives rise to a morphism $q:F^+\ra G$ which satisfies $p=q\cdot i^+_F$.
\end{proof}

\begin{proof}[Proof of the theorem]
We define $a(F)=\left(F^+\right)^+$ and $\eta_F=i^+_{F^+}\cdot i^+_F$. By Lemma \ref{lemma:plusconstruction} presheaf $a(F)$ is a sheaf. Now suppose that $p:F\ra G$ is a morphism of presheaves and $G$ is a sheaf. We apply Lemma \ref{lemma:universalpropertyofplusconstruction} twice to obtain a unique morphism $r:a(F)\ra G$ such that $p=r\cdot \eta_F$.
\end{proof}







\small
\bibliographystyle{alpha}
\bibliography{zzz}


\end{document}