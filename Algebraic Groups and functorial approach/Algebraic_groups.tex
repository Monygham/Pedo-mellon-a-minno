\input ../pree.tex

\begin{document}

\title{Algebraic group schemes over field}
\date{}
\maketitle

\section{Introduction}
\noindent
In these notes we group schemes over fields. For background we refer to \cite{kfunctors} and \cite{Monoid_k_functors}.\\
Throughout these notes $k$ is a fixed field.



\section{Simple criterion for separatedness}

\begin{proposition}\label{proposition:separatedness_criterion}
Let $\bd{G}$ be a group scheme over $k$ and let $e_{\bd{G}}:\Spec k\ra \bd{G}$ be its unit. Then the following are equivalent.
\begin{enumerate}[label=\emph{\textbf{(\roman*)}}, leftmargin=3.0em]
\item $e_{\bd{G}}$ is a closed immersion.
\item $\bd{G}$ is separated.
\end{enumerate}
\end{proposition}
\begin{proof}
Suppose that \textbf{(i)} holds. Consider morphism $f:\bd{G}\times_k\bd{G}\ra \bd{G}$ given on $A$-points by formula
$$f(g_1,g_2) = g_1\cdot g_2^{-1}$$
where $A$ is a $k$-algebra. Note that we have a cartesian square
\begin{center}
\begin{tikzpicture}
[description/.style={fill=white,inner sep=2pt}]
\matrix (m) [matrix of math nodes, row sep=3em, column sep=4em,text height=1.5ex, text depth=0.25ex] 
{   \bd{G}                & \Spec k \\
    \bd{G}\times_k\bd{G}  & \bd{G}  \\} ;
\path[->,line width=1.0pt,font=\scriptsize]
(m-1-1) edge node[above] {$\pi $} (m-1-2)
(m-2-1) edge node[below] {$f $} (m-2-2)
(m-1-1) edge node[left] {$\delta_{\bd{G}} $} (m-2-1)
(m-1-2) edge node[right] {$e_{\bd{G}}  $} (m-2-2);
\end{tikzpicture}
\end{center}
where $\delta_{\bd{G}}$ is a diagonal of $\bd{G}$ and the top horizontal arrow is the structure morphism. Since base change of a closed immersion is a closed immersion, we derive that $\delta_{\bd{G}}$ is a closed immersion and hence $\bd{G}$ is separated. This is \textbf{(ii)}.\\
Suppose now that \textbf{(ii)} holds. Let $\pi:\bd{G}\ra \Spec k$ be the structural morphism. Then $\pi\cdot e_{\bd{G}} = 1_{\bd{G}}$. Since $\pi$ is a separated morphism, we derive that (by cancellation) $e_{\bd{G}}$ is closed immersion. This is \textbf{(i)}.
\end{proof}

\begin{definition}
Let $\bd{G}$ be a group scheme over $k$. If $\bd{G}$ is (locally) of finite type over $k$, then we say that $\bd{G}$ is \textit{an (a locally) algebraic group over $k$}.
\end{definition}

\begin{corollary}\label{corollary:locally_algebraic_groups_are_separated}
Let $\bd{G}$ be a locally algebraic group over $k$. Then $\bd{G}$ is separated. 
\end{corollary}
\begin{proof}
Consider the unit $e_{\bd{G}}:\Spec k \ra \bd{G}$. Since $\bd{G}$ is locally of finite type, we derive that each $k$-point is closed in $\bd{G}$. Thus $e_{\bd{G}}$ is a closed immersion. By Proposition \ref{proposition:separatedness_criterion} we derive that $\bd{G}$ is separated.
\end{proof}

\section{Abelian Varieties}
\noindent
We start this section with the following general result.

\begin{theorem}[Rigidity]\label{theorem:rigidity_result}
Let $\pi:X\ra Y$ be a proper morphism of schemes such that $\pi^{\#}:\cO_Y\ra \pi_*\cO_X$ is an isomorphism of sheaves. Let $g:X\ra Z$ be a morphism of schemes. Suppose that for some point $y$ in $Y$ there is a point $z$ of $Z$ such that $\pi^{-1}(y)\subseteq g^{-1}(z)$. Then there exist an affine neighborhood $V$ of $y$ and an affine neighborhood $W$ of $z$ such that $\pi^{-1}(V) \subseteq g^{-1}(W)$. Moreover, there exists a morphism $h:V\ra W$ making the diagram
\begin{center}
\begin{tikzpicture}
[description/.style={fill=white,inner sep=2pt}]
\matrix (m) [matrix of math nodes, row sep=3em, column sep=4em,text height=1.5ex, text depth=0.25ex] 
{  \pi^{-1}(V) & W  \\
    V  &  \\} ;
\path[->,line width=1.0pt,font=\scriptsize]
(m-1-1) edge node[above] {$ \mathrm{res.\,of}\,g $} (m-1-2)
(m-1-1) edge node[left] {$ \mathrm{restriction\,of}\,\pi $} (m-2-1)
(m-2-1) edge node[below = 7pt, right = -6pt] {$ h $} (m-1-2);
\end{tikzpicture}
\end{center}
commutative, where horizontal arrow is the restriction of $g$.
\end{theorem}
\begin{proof}
Consider an affine open neighborhood of $W$ of $z$. Since $\pi$ is proper and $\pi^{-1}(y)) = g^{-1}(z)$, we derive that $\pi\big(X\setminus g^{-1}(W)\big)$ is a closed subset of $Y$ that does not contain $y$. Pick an open affine neighborhood $V$ of $y$ in $Y$ that does not intersect with $\pi\big(X\setminus g^{-1}(W)\big)$. Then $\pi^{-1}(V) \subseteq g^{-1}(W)$. Since $\pi^{\#}$ is an isomorphism we have the composition
\begin{center}
\begin{tikzpicture}
[description/.style={fill=white,inner sep=2pt}]
\matrix (m) [matrix of math nodes, row sep=3em, column sep=4em,text height=1.5ex, text depth=0.25ex] 
{  \cO_Z(W) & \Gamma\big(g^{-1}(W),\cO_X\big)  & \Gamma\big(\pi^{-1}(V),\cO_X\big)  & \cO_Y(V)\\} ;
\path[->,line width=1.0pt,font=\scriptsize]  
(m-1-1) edge node[above] {$ g^{\#}_W  $} (m-1-2)
(m-1-2) edge node[above] {$ (-)_{\mid \pi^{-1}(V)} $} (m-1-3)
(m-1-3) edge node[above] {$ \left(\pi^{\#}_V\right)^{-1} $} (m-1-4);
\end{tikzpicture}
\end{center}
This composition induces a morphism of affine schemes $h:V\ra W$. Since a morphism from a scheme to an affine scheme is determined by the morphism on global sections of structure sheaves, we derive that $h$ makes the triangle in the statement commutative.
\end{proof}
\noindent
Now we can apply this result to study complete algebraic groups over $k$. For this we need the following definition.

\begin{definition}
Let $\bd{A}$ be a geometrically integral, complete algebraic group over $k$. Then we say that $\bd{A}$ is \textit{an abelian variety over $k$}.
\end{definition}
\noindent
Now we prove the following interesting result.

\begin{theorem}\label{theorem:morphisms_from_abelian_varieties}
Let $\bd{A}$ be an abelian variety over $k$, let $\bd{G}$ be a separated group scheme over $k$ and let $f:\bd{A}\ra \bd{G}$ be a morphism of schemes over $k$. Suppose that the diagram
\begin{center}
\begin{tikzpicture}
[description/.style={fill=white,inner sep=2pt}]
\matrix (m) [matrix of math nodes, row sep=3em, column sep=1em,text height=1.5ex, text depth=0.25ex] 
{  \bd{A} &        & \bd{G}  \\
          &\Spec k &  \\} ;
\path[->,line width=1.0pt,font=\scriptsize]
(m-1-1) edge node[above] {$ f $} (m-1-3)
(m-2-2) edge node[auto] {$ e_{\bd{A}} $} (m-1-1)
(m-2-2) edge node[below = 6pt, right = 1pt] {$ e_{\bd{G}} $} (m-1-3);
\end{tikzpicture}
\end{center}
is commutative. Then $f$ is a morphism of groups schemes over $k$.
\end{theorem}
\begin{proof}
We define a morphism $g:\bd{A}\times_k \bd{A}\ra \bd{G}$ given by
$$(x_1,x_2) \mapsto f(x_1)\cdot f(x_2)\cdot f(x_1\cdot x_2)^{-1}$$
where $A$ is a $k$-algebra and $x_1,x_2$ are $A$-points of $\bd{A}$. It suffices to show that $g$ factors through $\Spec k(e_{\bd{G}})$. For this we may change base to an algebraic closure of $k$ by faitfully flat descent. So we may assume that the field $k$ is algebraically closed and $\bd{A}$ is connected. Then the projection onto second factor $\pi:\bd{A}\times_k \bd{A}\ra \bd{A}$ is proper and $k = \Gamma\big(\bd{A},\cO_{\bd{A}}\big)$ implies that $\pi^{\#}$ is an isomorphism of sheaves on $\bd{A}$. Moreover, note that $\pi^{-1}(e_{\bd{A}})\subseteq g^{-1}(e_{\bd{G}})$. Indeed, this follows from the assumption that $f(e_{\bd{A}}) = e_{\bd{G}}$. By Theorem \ref{theorem:rigidity_result} we deduce that there exist an affine neighborhood $V$ of $e_{\bd{A}}$, an affine neighborhood $W$ of $e_{\bd{G}}$ and a morphism $h:\Spec k\ra W$ such that $\pi^{-1}(V) \subseteq g^{-1}(W)$ and the diagram
\begin{center}
\begin{tikzpicture}
[description/.style={fill=white,inner sep=2pt}]
\matrix (m) [matrix of math nodes, row sep=3em, column sep=4em,text height=1.5ex, text depth=0.25ex] 
{  \bd{A}\times_kV & W  \\
    V  &  \\} ;
\path[->,line width=1.0pt,font=\scriptsize]
(m-1-1) edge node[above] {$ \mathrm{res.\,of}\,g $} (m-1-2)
(m-1-1) edge node[left] {$ \mathrm{projection} $} (m-2-1)
(m-2-1) edge node[below = 7pt, right = -6pt] {$ h $} (m-1-2);
\end{tikzpicture}
\end{center}
is commutative. Hence for every $k$-point $v$ of $V$ we have the restiction $g_{\mid \bd{A} \times_k\Spec k(v)}$ factors through $\Spec k\left(h(v)\right)$. Since $g(v,e_{\bd{A}}) = e_{\bd{G}}$, we derive that $h(v) = e_{\bd{G}}$ and thus $g_{\mid \bd{A}\times_k\Spec k(v)}$ factors through $\Spec k(e_{\bd{G}})$. This holds for any $k$-point of $V$. Therefore, $g_{\mid \bd{A}\times_kV}$ factors through $\Spec k(e_{\bd{G}})$. Consider the kernel $i:Z\hookrightarrow \bd{A}\times_k\bd{A}$ of a pair consisting of $g$ and a morphism $\bd{A}\times_k\bd{A}\ra \bd{G}$ that factorizes through $\Spec k(e_{\bd{G}})$. Since $\bd{G}$ is separated, we derive that $i$ is a closed immersion. Moreover, $i$ dominates $\bd{A}\times_kV$. Since $\bd{A}\times_kV$ is schematically dense open subset of $\bd{A}\times_k\bd{A}$ (because $\bd{A}\times_k\bd{A}$ is integral), we derive that $i$ is an isomorphism and hence $g$ factors through $\Spec k(e_{\bd{G}})$.
\end{proof}

\begin{corollary}\label{corollary:abelian_varieties_are_commutative}
Let $\bd{A}$ be an abelian variety over $k$. Then $\bd{A}$ is a commutative group scheme over $k$.
\end{corollary}
\begin{proof}
Consider the morphism $f:\bd{A}\ra \bd{A}$ given on $A$-points of $\bd{A}$ by
$$f(x) = x^{-1}$$
where $A$ is a $k$-algebra. By Theorem \ref{theorem:morphisms_from_abelian_varieties} we derive that $f$ is a morphism of group schemes over $k$. Hence $\bd{A}$ is a commutative group scheme.
\end{proof}

\section{Representability of fixed points}

\begin{definition}
Let $\fG$ be a monoid $k$-functor and let $\alpha:\fG\times \fX\ra \fX$ be an action of $\fG$ on a $k$-functor. Then we define a $k$-subfunctor $\fX^{\fG}$ of $\fX$ by
$$\fX^{\fG}(A) = \big\{x\in \fX(A)\,\big|\,\mbox{ for any }A\mbox{-algebra }f:A\ra B\mbox{ and }g\in \fG(B)\mbox{ we have }\alpha\big(g,\fX(f)(x)\big) = \fX(f)(x)\big\}$$
for every $k$-algebra $A$. Then $\fX^{\fG}$ is called \textit{the fixed point $k$-functor}.
\end{definition}

\begin{theorem}\label{theorem:fixed_points_existence}
Let $\bd{G}$ be a group scheme over $k$ and let $a:\bd{G}\times_kX\ra X$ be an action of $\bd{G}$ on a $k$-scheme $X$. Suppose that one of the following assertions hold.
\begin{enumerate}[label=\emph{\textbf{(\roman*)}}, leftmargin=3.0em]
\item $X$ is separated.
\item $\bd{G}$ is a geometrically connected, locally algebraic group.
\end{enumerate}
\end{theorem}
\noindent
The following result is based on {\cite[Theorem 6.2]{kfunctors}} and plays the fundamental role in the proof.

\begin{lemma}\label{lemma:representability_of_fixed_point_functor}
Let $X,Y$ be $k$-schemes and let $a:Y\times_kX\ra X$ be a morphism of $k$-schemes. Suppose that one of the following assertions hold.
\begin{enumerate}[label=\emph{\textbf{(\arabic*)}}, leftmargin=3.0em]
\item $X$ is separated.
\item For every open subscheme $U$ of $X$ we have $a\left(Y\times_kU\right)\subseteq U$
\end{enumerate}
Consider a $k$-functor given by formula
$$A \mapsto \big\{f:\Spec A\ra X\,\big|\,a\cdot \left(1_Y\times_k f\right) = \mathrm{pr}_X\cdot \left(1_Y\times_kf\right)\big\}$$
where $A$ is a  $k$-algebra and $\mathrm{pr}_X:Y\times_kX \ra X$ is the projection. Then this $k$-functor is representable by a closed subscheme of $X$.
\end{lemma}
\begin{proof}[Proof of the lemma]
Assume first that $X$ is separated. Consider a morphism $\langle a, \mathrm{pr}_X\rangle:Y\times_kX\ra X\times_k X$. By {\cite[Corollary 4.6]{Monoid_k_functors}} we deduce that $\fP_{\langle a, \mathrm{pr}_X\rangle}$ corresponds to a morphism $\sigma:\fP_X\ra \iMor_k\left(\fP_Y, \fP_X\times \fP_X\right)$ of $k$-functors. Since $X$ is separated, the diagonal $\delta_X:X\ra X\times_k X$ is a closed immersion. This implies that $\fP_{\delta_X}$ is a closed immersion of $k$-functors. The fact that $Y$ is locally free over $k$ and {\cite[Theorem 6.2]{kfunctors}} imply that
$$\iMor_k\left(1_{\fP_Y}, \fP_{\delta_X} \right):\iMor_k\left( \fP_{Y},\fP_{X} \right) \hookrightarrow \iMor_k\left( \fP_{Y}, \fP_{X} \times \fP_{X} \right)$$
is a closed immersion of $k$-functors. Consider now a cartesian square
\begin{center}
\begin{tikzpicture}
[description/.style={fill=white,inner sep=2pt}]
\matrix (m) [matrix of math nodes, row sep=3em, column sep=4em,text height=1.5ex, text depth=0.25ex] 
{    \fX      & \iMor_k\left( \fP_{Y},\fP_{X} \right)               \\
  \fP_X      & \iMor_k\left( \fP_{Y}, \fP_{X} \times \fP_{X} \right)  \\} ;
\path[->,line width=1.0pt,font=\scriptsize]
(m-1-1) edge node[above] {$ j $} (m-1-2)
(m-2-1) edge node[below] {$ \sigma $} (m-2-2)
(m-1-1) edge node[left]  {$  $} (m-2-1)
(m-1-2) edge node[right] {$ \iMor_k\left(1_{\fP_Y}, \fP_{\delta_X} \right) $} (m-2-2);
\end{tikzpicture}
\end{center}
of $k$-functors. By base change $j:\fX\ra \fP_X$ is a closed immersion of $k$-functors. Thus we derive that $\fX$ is representable by a closed subscheme of $\fX$. It suffices to observe that $\fX$ is precisely the $k$-functor described in the statement. This proves the statement under the assumption \textbf{(1)}.\\
Now suppose that $a\left(Y\times_kU\right) \subseteq U$ for every open subscheme $U$ of $X$. For every open subscheme denote by $a_U:Y\times_kU\ra U$ the restriction of $a$. Let $\cU$ be an open affine cover of $X$. Then functors
$$\bigg\{\Alg_k\ni A\mapsto \big\{f:\Spec A\ra U\,\big|\,a\cdot \left(1_Y\times_k f\right) = \mathrm{pr}_X\cdot \left(1_Y\times_k f\right)\big\}\in \Set \bigg\}_{U\in \cU}$$
form an open cover ({\cite[Definition 4.5]{kfunctors}}) of the $k$-functor in the statemtent. Moreover, since each $U$ in $\cU$ is affine and hence separated, we derive by the first part of the proof that each $k$-functor in the family is representable. Now {\cite[Theorem 4.6]{kfunctors}} imply that the functor in the statement is representable. This finishes the proof in case \textbf{(2)}.
\end{proof}

\begin{lemma}\label{lemma:infinitesimal_isomorphisms}
Let $f:\bd{H}\ra \bd{G}$ be a morphism of locally algebraic groups over $k$. Suppose that the following assertions hold.
\begin{enumerate}[label=\emph{\textbf{(\arabic*)}}, leftmargin=3.0em]
\item The morphism
$$\widehat{\cO_{\bd{G},e_{\bd{G}}}} \ra  \widehat{\cO_{\bd{H},e_{\bd{H}}}}$$
induced by $f^{\#}$ is an isomorphism.
\item $f$ is a monomorphism of $k$-schemes.
\end{enumerate}
Then $f$ is an open immersion.
\end{lemma}
\begin{proof}[Proof of the lemma]
The assertion \textbf{(1)} implies that $f$ is {\'e}tale in $e_{\bd{H}}$. Let $K$ be an algebraic closure of $k$. Consider the {\'e}tale locus $U$ of $f_k = 1_K\otimes_kf:\bd{H}_K\ra \bd{G}_K$. Then $U$ is an open subscheme of $\bd{H}_K$ containing the unit. Moreover, for every $K$-point $h$ of $\bd{H}_K$ we have a commutative square
\begin{center}
\begin{tikzpicture}
[description/.style={fill=white,inner sep=2pt}]
\matrix (m) [matrix of math nodes, row sep=3em, column sep=4em,text height=1.5ex, text depth=0.25ex]
{ \bd{H}_K  & \bd{G}_K          \\
  \bd{H}_K  & \bd{G}_K  \\};
\path[->,line width=1.0pt,font=\scriptsize]
(m-1-1) edge node[above] {$ f_K $} (m-1-2)
(m-2-1) edge node[below] {$ f_K $} (m-2-2)
(m-1-1) edge node[left]  {$ h\cdot (-)  $} (m-2-1)
(m-1-2) edge node[right] {$ f_K(h)\cdot (-)  $} (m-2-2);
\end{tikzpicture}
\end{center}
where $h\cdot (-)$ and $f_K(h)\cdot (-)$ are isomorphisms of $K$-schemes. This proves that $h\cdot U\subseteq U$. Hence $U$ contains all $K$-rational points of $\bd{H}_K$. Therefore, the complement of $U$ in $\bd{H}_K$ is empty. Hence $U = \bd{H}_K$. This shows that $f_K$ is {\'e}tale and by faithfully flat descent also $f$ is {\'e}tale. Since {\'e}tale monomorphisms are open immersions, we derive that $f$ is an open immersion.
\end{proof}

\begin{proof}[Proof of the theorem]
If \textbf{(1)} holds, then the statement follows directly from Lemma \ref{lemma:representability_of_fixed_point_functor}.\\
Suppose now that \textbf{(2)} holds that is $\bd{G}$ is an algebraic group. For each $n\in \NN$ we define
$$\bd{G}_n = \Spec \cO_{\bd{G},e_{\bd{G}}}/\ideal{m}_{e_{\bd{G}}}^{n+1}$$
where $e$ is the unit of $\bd{G}$. Then $\bd{G}_n$ is the $n$-th infinitesimal neighborhood of $e$ in $\bd{G}$. Denote by $p_n:\bd{G}_n\times_kX\ra X$ the projection on the second factor. Let $a_n:\bd{G}_n\times_kX\ra X$ be the morphism induced by $a$. Note that for every open subscheme $U$ of $X$ we have $a_n\left(\bd{G}_n\times_kU\right)\subseteq U$. By Lemma \ref{lemma:representability_of_fixed_point_functor} it follows that the $k$-functor given by
$$\Alg_k \ni A \mapsto \big\{f:\Spec A\ra X\,\big|\,a_n\cdot \left(1_{\bd{G}_n}\times_k f\right) = \mathrm{pr}_n\cdot \left(1_{\bd{G}_n}\times_kf\right)\big\} \in \Set$$
is representable by a closed subscheme $Z_n$ of $X$. Consider now the quasi-coherent ideal $\cI_n$ of $Z_n$ inside $X$. Define
$$\cI = \sum_{n\in \NN}\cI_n$$
Let $i:Z\hookrightarrow X$ be a closed subscheme of $X$ determined by $\cI$. This means that $Z$ is the scheme-theoretic intersection inside $X$ of closed subschemes $Z_n$ for $n\in \NN$. We show that $Z$ represents the fixed point functor. For this assume that $A$ is a $k$-algebra and $f:\Spec A\ra X$ is a morphism of $k$-schemes such that $f$ is an $A$-point of the fixed point functor. This is equivalent with
$$a\cdot \left(1_{\bd{G}}\times_k f\right) = \mathrm{pr}_X\cdot \left(1_{\bd{G}}\times_kf\right)$$
From this equality we deduce that
$$a_n\cdot \left(1_{\bd{G}_n}\times_k f\right) = \mathrm{pr}_n\cdot \left(1_{\bd{G}_n}\times_kf\right)$$
for every $n\in \NN$ and hence $f$ factors through $Z_n$ for every $n\in \NN$. We derive that $f$ factors through $Z$. This proves that the fixed point functor is a $k$-subfunctor of the functor of points of $Z$. It suffices to prove that $Z$ is invariant with respect to $\bd{G}$-action. For this consider the morphism $b:\bd{G}\times_kZ\ra X$ induced by $a$. By {\cite[Corollary 4.6]{Monoid_k_functors}} morphism $b$ corresponds to a morphism $\sigma:\fP_{\bd{G}} \ra \iMor_k\left(\fP_{Z},\fP_X\right)$ of $k$-functors. Consider the cartesian square
\begin{center}
\begin{tikzpicture}
[description/.style={fill=white,inner sep=2pt}]
\matrix (m) [matrix of math nodes, row sep=3em, column sep=4em,text height=1.5ex, text depth=0.25ex] 
{    \fH      & \iMor_k\left( \fP_{Z},\fP_{Z} \right)               \\
  \fP_{\bd{G}}      & \iMor_k\left( \fP_{Z},\fP_{X}\right)  \\} ;
\path[->,line width=1.0pt,font=\scriptsize]
(m-1-1) edge node[above] {$ j $} (m-1-2)
(m-2-1) edge node[below] {$ \sigma $} (m-2-2)
(m-1-1) edge node[left]  {$  $} (m-2-1)
(m-1-2) edge node[right] {$ \iMor_k\left(1_{\fP_Z}, \fP_{i} \right) $} (m-2-2);
\end{tikzpicture}
\end{center}
The fact that $Z$ is locally free over $k$ and {\cite[Theorem 6.2]{kfunctors}} imply that $\iMor_k\left(\fP_{Z},\fP_{i}\right)$ is a closed immersion of $k$-functors. Hence $j:\fH \hookrightarrow \fP_{\bd{G}}$ is a closed immersion. Moreover, $\fH$ is a subgroup $k$-functor of $\fP_{\bd{G}}$. Thus we deduce that $j$ is induced by a closed immersion of an algebraic groups $f:\bd{H} \hookrightarrow \bd{G}$. By definition of $i:Z\hookrightarrow X$, we derive that morphism of local $k$-algebras
$$\widehat{\cO_{\bd{G},e_{\bd{G}}}} \ra  \widehat{\cO_{\bd{H},e_{\bd{H}}}}$$
induced by $f^{\#}$ is an isomorphism. Hence by Lemma \ref{lemma:infinitesimal_isomorphisms} $f$ is an open immersion of locally algebraic groups. Since $\bd{G}$ is geometrically connected, we deduce that $f$ is an isomorphism. Thus $j$ is an isomorphism and this means that $b:\bd{G}\times_k Z\ra X$ factors through $i:Z\hookrightarrow X$.
\end{proof}

\section{Transporters}

\begin{definition}
Let $\fG$ be a monoid $k$-functor and let $\alpha:\fG\times \fX\ra \fX$ be an action of $\fG$ on a $k$-functor $\fX$. Suppose that $\fY_1,\fY_2$ are $k$-subfunctors of $\fX$. For every $k$-algebra $A$ we define
$$\mathrm{Transp}_{\fG}\left(\fY_1,\fY_2\right)(A) = \big\{g\in \fG(A)\,\big|\,\alpha_g\left(\fY_1(A)\right)\subseteq \fY_2(A)\big\}$$
where as usual $\alpha_g$ is a slice of $\alpha$ along $g$. Then $\mathrm{Transp}_{\fG}\left(\fY_1,\fY_2\right)$ is a $k$-subfunctor of $\fG$. It is called \textit{the tranporter of $\fY_1$ into $\fY_2$ with respect to $\alpha$}. 
\end{definition}

\section{Morphisms of locally algebraic groups}

\begin{theorem}\label{theorem:monomorphisms_of_algebraic_groups}
Let $f:\bd{H}\ra \bd{G}$ be a morphism of algebraic groups over $k$. Then the following are equivalent.
\begin{enumerate}[label=\emph{\textbf{(\roman*)}}, leftmargin=3.0em]
\item $f$ is a monomorphism of $k$-schemes.
\item $f$ is a locally closed immersion of $k$-schemes.
\item $f$ is a closed immersion of $k$-schemes.
\end{enumerate}
\end{theorem}

\begin{theorem}\label{theorem:monomorphisms_of_algebraic_schemes}
Let $f:X\ra Y$ be a monomorphism of finite type with $X,Y$ noetherian. Then there exists open dense subscheme $V$ of $Y$ such that the morphism $f^{-1}(V)\ra V$ induced by $f$ is a locally closed immersion.
\end{theorem}
The proof is based on a sequence of results.

\begin{lemma}\label{lemma:monomorphisms_of_fields_spectra}
Let $K$ and $L$ be a fields. If $\Spec L\hookrightarrow \Spec K$ is a monomorphism of schemes, then it is an isomorphism. 
\end{lemma}
\begin{proof}[Proof of the lemma]
Since the diagonal of a monomorphism is an isomorphism, we deduce that the multiplication map $L\otimes_KL\ra L$ is an isomorphism. This implies that $\mathrm{dim}_k(L) = \mathrm{dim}_L\left(L\otimes_kL\right) = 1$. Hence $k\hookrightarrow L$ is an isomorphism of fields.
\end{proof}

\begin{lemma}\label{lemma:local_embeddings}
There exists an open dense subset $U$ of $X$ such that $f_{\mid U}$ is a locally closed immersion.
\end{lemma}
\begin{proof}[Proof of the lemma]
Suppose that $x$ is a generic point of an irreducible component of $X$. Let $y = f(x)$. Pick $f_{y}:X_{y}\ra \Spec k(y)$. Then $f_{y}$ is a monomorphism. Since $\Spec k(x)$ is a closed subscheme of $X_y$, we derive that the composition of $\Spec k(x)\hookrightarrow X_y$ and $f_y$ is a monomorphism $\Spec k(x)\hookrightarrow \Spec k(y)$ of schemes. By Lemma \ref{lemma:monomorphisms_of_fields_spectra} we deduce that is an isomorphism. We derive that $f_{y}$ is a retraction. A retraction that is a monomorphism is an isomorphism. Hence $f_y$ is an isomorphism. This shows that $k(y)\cong \cO_{X,x}/\ideal{m}_y\cO_{X,x}$. Thus we have
$$\cO_{X,x} = f^{\#}\left(\cO_{Y,y}\right)+\ideal{m}_{y}\cO_{X,x}$$
Since $x$ is a generic point of an irreducible component of $X$, we derive that $\cO_{X,x}$ is artinian. Thus $\ideal{m}_y\cO_{X,x} \subseteq \ideal{m}_x$ is a nilpotent ideal. Thus $f^{\#}:\cO_{Y,y}\ra \cO_{X,x}$ is onto.
\end{proof}

\begin{lemma}\label{lemma:local_surjections_give_local_closed_immersions}
Let $f:A\ra B$ be a morphism of finite type between noetherian rings. Suppose that $\ideal{p}\in \Spec A$ and $\ideal{q}\in \Spec B$ are prime ideals such that $f^{-1}(\ideal{q})=\ideal{p}$. Assume that $f$ induces a surjective morphism $A_{\ideal{p}}\ra B_{\ideal{q}}$. Then there exists $s\in B\setminus \ideal{q}$ such that $f$ induces a surjective morphism $A\ra B_s$.
\end{lemma}
\begin{proof}[Proof of the Lemma]
First assume that $A_{\ideal{p}}\ra B_{\ideal{q}}$ is bijective and let $\phi:B_{\ideal{q}}\ra A_{\ideal{p}}$. Since $f:A\ra B$ is morphism of finite type between noetherian rings, we have $B \cong A[x_1,...,x_n]/I$ for some finitely generated ideal $I\subseteq A[x_1,..,x_n]$ and free variables $x_1,...,x_n$. Let $\ol{x}_i = x_i\,\mathrm{mod}\,I$ for $1\leq i\leq n$. Suppose that $a_1,...,a_n$ are elements in $A$ such that
$$\frac{a_i}{1} = \phi\left(\frac{\ol{x}_i}{1}\right)$$
for $1\leq i\leq n$.
\end{proof}




















































\small
\bibliographystyle{alpha}
\bibliography{../zzz}




\end{document}
