\input ../pree.tex

\begin{document}

\title{Algebraic group schemes over field}
\date{}
\maketitle

\section{Introduction}
\noindent
In these notes we group schemes over fields. For background we refer to \cite{kfunctors} and \cite{Monoid_k_functors}. Let $k$ be a field..

\section{Regular functions on products of quasi-compact and semi-separated schemes}
\noindent
We start with the following notion.

\begin{definition}
Let $Y$ be a scheme and let $X$ be a $Y$-scheme. If the diagonal $X\ra X\times_Y X$ is affine, then we say that $X$ is \textit{semi-separated} over $Y$.
\end{definition}

\begin{remark}\label{remark:separated_is_semi-separated}
Let $Y$ be a scheme. Every separated $Y$-scheme is semi-separated.
\end{remark}

\begin{example}[Semi-separated scheme that is not separated]
Let $o$ be the origin of the affine line $\mathbb{A}^1_k$. Consider the following pushout diagram in the category of $k$-schemes.
\begin{center}
\begin{tikzpicture}
[description/.style={fill=white,inner sep=2pt}]
\matrix (m) [matrix of math nodes, row sep=3em, column sep=3em,text height=1.5ex, text depth=0.25ex] 
{   \mathbb{A}^1_k\setminus \{o\} & \mathbb{A}^1_k \\
    \mathbb{A}^1_k                & X              \\} ;
\path[right hook->,line width=1.0pt,font=\scriptsize]
(m-1-1) edge node[above] {$ $} (m-1-2)
(m-1-1) edge node[left] {$ $} (m-2-1);
\path[->,line width=1.0pt,font=\scriptsize]
(m-2-1) edge node[below] {$ $} (m-2-2)
(m-1-2) edge node[right] {$ $} (m-2-2);
\end{tikzpicture}
\end{center}
Then $X$ is an affine line with double origin. Then the diagonal $X\ra X\times_k X$ is affine but not a closed immersion. Hence $X$ is semi-separated but not separated.
\end{example}

\begin{theorem}\label{theorem:sections_of_qc_sheaves_on_products}
Let $X,Y$ be quasi-compact and semi-separated $k$-schemes. Denote by $\pi_X$ and $\pi_Y$ projections from $X\times_kY$ to $X$ and $Y$, respectively. Let $\cF$ and $\cG$ be quasi-coherent sheaves on $X$ and $Y$, respectively. Then the canonical morphism
$$\Gamma(X,\cF)\otimes_k\Gamma(Y,\cG)\ni s\otimes t \mapsto \pi_X^*s \otimes \pi_Y^{*}t \in \Gamma\left(X\times_kY,\pi_X^*\cF\otimes_{\cO_{X\times_kY}}\pi^*_Y\cG\right)$$
is an isomorphism.
\end{theorem}
\noindent
The theorem follows from the following result.

\begin{lemma}\label{lemma:sections_and_open_covers_of_a_factor}
Let $X,Y$ be $k$-schemes and let $\{V_i\}_{i=1}^n$ be a finite open cover of $Y$. Suppose that the canonical morphism
$$\Gamma(X,\cF)\otimes_k\Gamma\left(V_i\cap V_j,\cG \right) \ra \Gamma\left( X \times_k \left(V_i\cap V_j\right),\pi_X^*\cF\otimes_{\cO_{X\times_kY}}\pi^*_Y\cG\right)$$
is an isomorphism for any $i,j\in \{1,...,n\}$. Then the canonical morphism
$$\Gamma(X,\cF)\otimes_k\Gamma(Y,\cG)\ra \Gamma\left(X\times_kY,\pi_X^*\cF\otimes_{\cO_{X\times_kY}}\pi^*_Y\cG\right)$$
is an isomorphism.
\end{lemma}
\begin{proof}[Proof of the lemma]
For each $i\in \{1,...,n\}$ we have the restriction
$$r_i:\Gamma\left(X \times_kY, \pi^*_X\cF\otimes_{\cO_{X \times_kY}} \pi^*_Y\cG \right) \ra \Gamma\left(X\times_kV_i, \pi^*_X\cF\otimes_{\cO_{X \times_kY}} \pi^*_Y\cG \right)$$
and we denote by $p_i$ the restriction $\Gamma\left( Y,\cG \right) \ra \Gamma\left(V_i,\cG \right)$ tensored with $\Gamma(X,\cO_X)$. For $i, j \in \{1,...,n\}$ we have the restriction
$$r_{i,j}:\Gamma\left(X \times_kV_i,\pi^*_X\cF\otimes_{\cO_{X \times_kY}} \pi^*_Y\cG\right)\ra \Gamma\left(X\times_k\left(V_i\cap V_j\right),\pi^*_X\cF\otimes_{\cO_{X \times_kY}} \pi^*_Y\cG\right)$$
and we denote by $p_{i,j}$ the restriction $\Gamma\left(V_i,\cG\right)\ra \Gamma\left(V_i\cap V_j ,\cG\right)$ tensored with $\Gamma(X,\cF)$. Consider the commutative diagram
\begin{center}
\hspace*{-2.8cm}
\begin{tikzpicture}
[description/.style={fill=white,inner sep=2pt}]
\matrix (m) [matrix of math nodes, row sep=3em, column sep=4em,text height=1.5ex, text depth=0.25ex] 
{ \Gamma\left(X\times_k Y,\pi^*_X\cF\otimes_{\cO_{X \times_kY}} \pi^*_Y\cG \right) & \Oplus_{i=1}^n\Gamma\left(X\times_k V_i,\pi^*_X\cF\otimes_{\cO_{X \times_kY}} \pi^*_Y\cG\right)   &  \Oplus_{1\leq i < j\leq n}\Gamma\left(X\times_k \left(V_i\cap V_j\right),\pi^*_X\cF\otimes_{\cO_{X \times_kY}} \pi^*_Y\cG\right) \\
 \Gamma\left(X,\cF\right)\otimes_k\Gamma\left(Y,\cG\right) &  \Oplus_{i=1}^n\Gamma\left(X,\cF\right) \otimes_k \Gamma\left(V_i,\cG\right)   & \Oplus_{1\leq i <j \leq n} \Gamma\left(X,\cF\right)\otimes_k \Gamma\left(V_i\cap V_j,\cG\right)  \\} ;
\path[->,line width=0.8pt,font=\scriptsize]
(m-1-1) edge node[above] {$\langle r_i\rangle_{i=1}^n  $} (m-1-2)
(m-1-2) edge[transform canvas={yshift=0.6ex}] node[above]  {$\langle r_{i,j}\cdot \mathrm{pr}_i\rangle_{1\leq i<j\leq n}  $} (m-1-3)
(m-1-2) edge[transform canvas={yshift=-0.6ex}] node[below] {$\langle r_{j,i}\cdot \mathrm{pr}_j\rangle_{1\leq i<j\leq n}  $} (m-1-3)
(m-2-1) edge node[above] {$\langle p_i \rangle_{i=1}^n  $} (m-2-2)
(m-2-2) edge[transform canvas={yshift=0.6ex}] node[above] {$ \langle p_{i,j}\cdot \mathrm{pr}_i\rangle_{1\leq i<j\leq n}  $} (m-2-3)
(m-2-2) edge[transform canvas={yshift=-0.6ex}] node[below] {$\langle p_{j,i}\cdot \mathrm{pr}_j\rangle_{1\leq i<j\leq n}  $} (m-2-3)
(m-2-1) edge node[left] {$   $} (m-1-1)
(m-2-2) edge node[left] {$ \cong $} (m-1-2)
(m-2-3) edge node[right] {$ \cong $} (m-1-3);
\end{tikzpicture}
\end{center}
in which vertical arrows are canonically defined. Moreover, by assumptions right and middle vertical arrows are isomorphisms. Note also that both rows are kernel diagrams. Indeed, for the top row this follows from the sheaf property of $\pi^*_X\cF\otimes_{\cO_{X \times_kY}} \pi^*_Y\cG$ and for the bottom row this follows from the fact that $\Gamma(X,\cF)$ is flat over $k$ ($k$ is a field) together with the sheaf property of $\cG$. These imply that the left vertical arrow is an isomorphism and this completes the proof.
\end{proof}

\begin{proof}[Proof of the theorem]
The statement holds, if $X,Y$ are affine. Note that semi-separatedness of a scheme over commutative ring is equivalent to the fact that intersection of every pair of its open affine subschemes is affine. Now Lemma \ref{lemma:sections_and_open_covers_of_a_factor} implies that the result holds if $X$ is affine and $Y$ is quasi-compact and semi-separated over $k$. Next by symmetry in Lemma \ref{lemma:sections_and_open_covers_of_a_factor}, we derive that the result holds if $X,Y$ are quasi-compact and semi-separated over $k$.
\end{proof}

\begin{corollary}\label{corollary:regular_functions_on_products}
Let $X,Y$ be quasi-compact and semi-separated $k$-schemes. Denote by $\pi_X$ and $\pi_Y$ projections from $X\times_kY$ to $X$ and $Y$, respectively. Then the canonical morphism
$$\Gamma(X,\cO_X)\otimes_k\Gamma(Y,\cO_Y)\ni f\otimes_k g\mapsto \pi_X^{\#}(f)\cdot \pi_Y^{\#}(g)\in \Gamma\left(X\times_kY,\cO_{X\times_kY}\right)$$
is an isomorphism.
\end{corollary}

\begin{corollary}\label{corollary:functorial_reformulation_of_regular_functions_on_products}
Let $\Sch_k^{\mathrm{qc,ss}}$ be the category of quasi-compact and semi-separated schemes over $k$. Then the contravariant functor
$$\Sch_k^{\mathrm{qc},\mathrm{ss}}\ni X \mapsto \Gamma(X,\cO_X)\in \left(\Alg_k\right)^{\mathrm{op}}$$
preserves products.
\end{corollary}
\begin{proof}
This is a reformulation of Corollary \ref{corollary:regular_functions_on_products}.
\end{proof}

\section{General properties of groups schemes over a field}

\begin{proposition}\label{proposition:group_schemes_over_a_field_are_separated}
Let $\bd{G}$ be a group scheme over $k$. Then $\bd{G}$ is separated $k$-scheme.
\end{proposition}
\begin{proof}
Consider a morphism $f:\bd{G}\times_k\bd{G}\ra \bd{G}$ given on $A$-points by formula
$$f(g_1,g_2) = g_1\cdot g_2^{-1}$$
where $A$ is a $k$-algebra. Note that we have a cartesian square
\begin{center}
\begin{tikzpicture}
[description/.style={fill=white,inner sep=2pt}]
\matrix (m) [matrix of math nodes, row sep=3em, column sep=4em,text height=1.5ex, text depth=0.25ex] 
{   \bd{G}                & \Spec k \\
    \bd{G}\times_k\bd{G}  & \bd{G}  \\} ;
\path[->,line width=1.0pt,font=\scriptsize]
(m-1-1) edge node[above] {$\pi $} (m-1-2)
(m-2-1) edge node[below] {$f $} (m-2-2)
(m-1-1) edge node[left] {$\delta_{\bd{G}} $} (m-2-1)
(m-1-2) edge node[right] {$e_{\bd{G}}  $} (m-2-2);
\end{tikzpicture}
\end{center}
where $\delta_{\bd{G}}$ is the diagonal of $\bd{G}$, the top horizontal arrow is the structure morphism and $e_{\bd{G}}:\Spec k\ra \bd{G}$ is the unit. Since base change of a closed immersion is a closed immersion, we derive that $\delta_{\bd{G}}$ is a closed immersion if and only if $e_{\bd{G}}$ is a closed immersion. Since $\bd{G}$ is a $k$-scheme and $k$ is a field, it follows that every morphism $\Spec k\ra \bd{G}$ of $k$-schemes is a closed immersion. In particular, $e_{\bd{G}}$ is a closed immersion and hence $\bd{G}$ is separated.
\end{proof}

\begin{remark}\label{remark:action_is_isomorphic_with_projection}
Let $\fG$ be a group $k$-functor and let $a:\fG \times \fX \ra \fX$ be an action of $\fG$ on $\fX$. Consider an isomorphism $\phi:\fG\times \fX\ra \fG\times \fX$ given by
$$\fG(A)\times \fX(A)\ni (g,x)\mapsto (g,g^{-1}x)\in \fG(A)\times \fX(A)$$
for every $k$-algebra. Then the triangle
\begin{center}
\begin{tikzpicture}
[description/.style={fill=white,inner sep=2pt}]
\matrix (m) [matrix of math nodes, row sep=3em, column sep=2em,text height=1.5ex, text depth=0.25ex] 
{ \fG\times \fX  &     & \fG\times \fX   \\
                 &\fX  &  \\} ;
\path[->,line width=1.0pt,font=\scriptsize]
(m-1-1) edge node[above] {$ \phi $} (m-1-2)
(m-1-1) edge node[below = 2pt, left = 2pt] {$ \pi_{\fX}  $} (m-2-2)
(m-1-2) edge node[below = 2pt, right = 2pt] {$ a $} (m-2-2);
\end{tikzpicture}
\end{center}
is commutative.
\end{remark}

\begin{corollary}\label{corollary:action_isomorphic_with_projection}
Let $\bd{G}$ be a group scheme over $k$ and let $a:\bd{G}\times_k X\ra X$ be an action of $\bd{G}$ on $k$-scheme $X$. Then $a$ is isomorphism with the projection $\pi_X:\bd{G}\times_k X\ra X$.
\end{corollary}
\begin{proof}
This is a reformulation of Remark \ref{remark:action_is_isomorphic_with_projection}.
\end{proof}

\begin{corollary}\label{corollary:action_is_flat}
Let $\bd{G}$ be a group scheme over $k$ and let $a:\bd{G}\times_k X\ra X$ be an action of $\bd{G}$ on $k$-scheme $X$. Then $a$ is faithfuly flat.
\end{corollary}
\begin{proof}
This is a direct consequence of Corollary \ref{corollary:action_isomorphic_with_projection} and the fact that each group scheme $\bd{G}$ over a field $k$ is faithfuly flat.
\end{proof}

\begin{theorem}\label{theorem:base_locus_of_certain_classes_of_morphisms_are_invariant}
Let $\bd{G}$ be a group scheme over $k$, let $X,Y$ be $k$-schemes with $\bd{G}$-actions and let $f:X\ra Y$ be a $\bd{G}$-equivariant morphism. Suppose that $\bd{P}$ is a property of morphisms of $k$-schemes such that the following assertions hold.
\begin{enumerate}[label=\emph{\textbf{(\arabic*)}}, leftmargin=3.0em]
\item $\bd{P}$ is local on the base.
\item $\bd{P}$ is closed under base change.
\item $\bd{P}$ descends along faithfuly flat base change.
\end{enumerate}
Then there exists the largest open subset of $Y$ such that the restriction $f^{-1}(V)\ra V$ of $f$ is in $\bd{P}$ and it is $\bd{G}$-invariant.
\end{theorem}
\begin{proof}
Note that the existence of $V$ follows from \textbf{(1)}. We denote by $\tilde{f}$ the restriction of $f$ to $f^{-1}(V)\ra V$ and we denote by $\bd{G}\cdot V$, $\bd{G}\cdot f^{-1}(V)$ translations with respect to $\bd{G}$-actions of $V$, $f^{-1}(V)$, respectively. We also denote by $\hat{f}:\bd{G}\cdot f^{-1}(V)\ra \bd{G}\cdot V$ the restriction of $f$. Note that the square
\begin{center}
\begin{tikzpicture}
[description/.style={fill=white,inner sep=2pt}]
\matrix (m) [matrix of math nodes, row sep=3em, column sep=4em,text height=1.5ex, text depth=0.25ex] 
{   \bd{G}\times_k f^{-1}(V) & \bd{G}\cdot f^{-1}(V) \\
    \bd{G}\times_k V         & \bd{G}\cdot V  \\} ;
\path[->,line width=1.0pt,font=\scriptsize]
(m-1-1) edge node[above] {$ \bd{G}-\mathrm{action} $} (m-1-2)
(m-2-1) edge node[below] {$ \bd{G}-\mathrm{action} $} (m-2-2)
(m-1-1) edge node[left] {$1_{\bd{G}}\times_k\tilde{f}  $} (m-2-1)
(m-1-2) edge node[right] {$ \hat{f} $} (m-2-2);
\end{tikzpicture}
\end{center}
is cartesian. The assumption \textbf{(2)} implies that $1_{\bd{G}}\times_k\tilde{f}$ is in $\bd{P}$. Since bottom horizontal morphism is faithfuly flat by Corollary \ref{corollary:action_is_flat}, we deduce by \textbf{(3)} that $\hat{f}$ is in $\bd{P}$. Since $V$ is the largest open subset of $Y$ such that the restriction $f^{-1}(V)\ra V$ of $f$ is in $\bd{P}$ and
$$f^{-1}\left(\bd{G}\cdot V\right) = \bd{G}\cdot f^{-1}(V)$$
we derive that $\bd{G}\cdot V \subseteq V$. Hence $V = \bd{G}\cdot V$, which means that $V$ is $\bd{G}$-invariant.
\end{proof}

\section{Action on regular functions}

\begin{proposition}\label{proposition:regular_functions_on_k_qc_semiseparated_k_schemes}
Let $X$ be a quasi-compact and semi-separated $k$-scheme. Then $\iMor_k\left(\fP_X,\fO_k\right)$ exists and for every $k$-algebra $A$ there is an identification
$$\iMor_k\left(\fP_X,\fO_k\right)(A) =  A\otimes_k\Gamma(X,\cO_X)$$
natural in $A$.
\end{proposition}
\begin{proof}
Since $k$-functor $\fO_k$ is representable by $\mathbb{A}^1_k$ {\cite[Fact 3.2]{Monoid_k_functors}} and morphisms $\Spec A\times X\ra \mathbb{A}^1_A$ of $A$-schemes corresponds to regular functions on $\Spec A \times_k X$, we derive that there is an identification
$$\Mor_A\left((\fP_X)_A,\fO_A\right) = \Gamma\left(\Spec A\times_k X,\cO_{\Spec A\times_k X}\right)$$
natural in $k$-algebra $A$. Now by Corollary \ref{corollary:regular_functions_on_products} and assumptions on $X$ we deduce that for every $k$-algebra there is an identification
$$\iMor_k\left(\fP_X,\fO_k\right)(A) =  A\otimes_k\Gamma(X,\cO_X)$$
natural in $A$.
\end{proof}
\noindent
Recall that if $\fG$ is a monoid $k$-functor that acts on a $k$-functor $\fX$, then by {\cite[Proposition 10.2]{Monoid_k_functors}} there exists canonically defined action of $\fG^{\mathrm{op}}$ on $\iMor_k\left(\fX,\fO_k\right)$ assuming that the latter exists.

\begin{example}\label{example:k_monoid_action_on_regular_functions}
Let $\bd{M}$ be a quasi-compact and semi-separated monoid $k$-scheme and let $a:\bd{M}\times_kX\ra X$ be an action of $\bd{M}$ on quasi-compact and semi-separated $k$-scheme $X$. Then by the remark above and by Proposition \ref{proposition:regular_functions_on_k_qc_semiseparated_k_schemes} we deduce that $\Gamma(X,\cO_X)$ carries the canonical structure of a linear representation of $\bd{M}^{\mathrm{op}}$. Let us explain how this representation is defined. For this consider $k$-algebra $A$ and let $f:\Spec A\ra \bd{M}$ be a morphism of $k$-schemes. Since $\bd{M} = \bd{M}^{\mathrm{op}}$ as $k$-schemes, morphism $f$ is an $A$-point of $\bd{M}^{\mathrm{op}}$. Now the morphism
$$a\cdot \left(f\times_k 1_X\right):\Spec A\times_k X\ra X$$
defines a morphism
$$\rho(f):\Gamma\left(X,\cO_X\right)\ra A\otimes_k \Gamma\left(X,\cO_X\right)$$
on global sections. Thus we have
$$\fP_{\bd{M}^{\mathrm{op}}}(A)\ni f\mapsto \rho(f) \in \Hom_k\big(\Gamma\left(X,\cO_X\right), A\otimes_k \Gamma\left(X,\cO_X\right)\big)$$
and this determines $\Gamma\left(X,\cO_X\right)$ as the linear representation of $\bd{M}^{\mathrm{op}}$.
\end{example}

\begin{theorem}\label{theorem:rationality_of_regular_functions}
Let $\bd{M}$ be a a quasi-compact and semi-separated monoid $k$-scheme and let $a:\bd{M}\times_kX\ra X$ be an action of $\bd{M}$ on quasi-compact and semi-separated $k$-scheme $X$. Consider the canonical linear representation $\Gamma\left(X,\cO_X\right)$ of $\bd{M}^{\mathrm{op}}$. Then every finite dimensional subspace $V\subseteq \Gamma\left(X,\cO_X\right)$ is contained in some finite dimensional $\bd{M}^{\mathrm{op}}$-subrepresentation.
\end{theorem}
\begin{proof}
Let $\mu:\bd{M}\times_k\bd{M}\ra \bd{M}$ be the multiplication. Since $a$ is an action, the following square is commutative
\begin{center}
\begin{tikzpicture}
[description/.style={fill=white,inner sep=2pt}]
\matrix (m) [matrix of math nodes, row sep=3em, column sep=4em,text height=1.5ex, text depth=0.25ex] 
{   \bd{M}\times_k \bd{M}\times_k X & \bd{M}\times_k X \\
    \bd{M}\times_k X                & X  \\} ;
\path[->,line width=1.0pt,font=\scriptsize]
(m-1-1) edge node[above] {$ 1_{\bd{M}}\times_k a $} (m-1-2)
(m-2-1) edge node[below] {$ a $} (m-2-2)
(m-1-1) edge node[left] {$ \mu\times_k1_X $} (m-2-1)
(m-1-2) edge node[right] {$ a $} (m-2-2);
\end{tikzpicture}
\end{center}
It implies that the square
\begin{center}
\begin{tikzpicture}
[description/.style={fill=white,inner sep=2pt}]
\matrix (m) [matrix of math nodes, row sep=4em, column sep=8em,text height=1.5ex, text depth=0.25ex] 
{   \Gamma\left(\bd{M}\times_k \bd{M}\times_k X, \cO_{\bd{M}\times_k\bd{M}\times_kX}\right) & \Gamma\left(\bd{M}\times_k X, \cO_{\bd{M}\times_kX}\right) \\
    \Gamma\left(\bd{M}\times_k X, \cO_{\bd{M}\times_kX}\right)                              & \Gamma\left(X,\cO_X\right)  \\} ;
\path[->,line width=1.0pt,font=\scriptsize]
(m-1-2) edge node[above] {$ \Gamma\left(\bd{M}\times_kX, \left(1_{\bd{M}}\times_k a\right)^{\#}\right) $} (m-1-1)
(m-2-2) edge node[below] {$\Gamma\left(X,a^{\#}\right)  $} (m-2-1)
(m-2-1) edge node[left] {$ \Gamma\left(\bd{M}\times_kX, \left(\mu\times_k1_X\right)^{\#}\right) $} (m-1-1)
(m-2-2) edge node[right] {$ \Gamma\left(X,a^{\#}\right) $} (m-1-2);
\end{tikzpicture}
\end{center}
is commutative. Since $\bd{M}$, $X$ are quasi-compact and semi-separated, canonical identifications of Corollary \ref{corollary:regular_functions_on_products} can be used. This implies that the square
\begin{center}
\begin{tikzpicture}
[description/.style={fill=white,inner sep=2pt}]
\matrix (m) [matrix of math nodes, row sep=3em, column sep=8em,text height=1.5ex, text depth=0.25ex] 
{   \Gamma\left(\bd{M},\cO_{\bd{M}}\right)\otimes_k\Gamma\left(\bd{M},\cO_{\bd{M}}\right) \otimes_k \Gamma\left(X,\cO_{X}\right) & \Gamma\left(\bd{M},\cO_{\bd{M}}\right)\otimes_k\Gamma\left(X,\cO_X\right) \\
    \Gamma\left(\bd{M},\cO_{\bd{M}}\right)\otimes_k\Gamma\left(X,\cO_X\right)               & \Gamma\left(X,\cO_X\right)  \\} ;
\path[->,line width=1.0pt,font=\scriptsize]
(m-1-2) edge node[above] {$ 1_{\Gamma\left(\bd{M},\cO_{\bd{M}}\right)} \otimes_k \Gamma\left(X,a^{\#}\right) $} (m-1-1)
(m-2-2) edge node[below] {$ \Gamma\left(X,a^{\#}\right) $} (m-2-1)
(m-2-1) edge node[left] {$ \Gamma\left(\bd{M},\mu^{\#}\right) \otimes_k 1_{\Gamma\left(X,\cO_X\right)} $} (m-1-1)
(m-2-2) edge node[right] {$ \Gamma\left(X,a^{\#}\right) $} (m-1-2);
\end{tikzpicture}
\end{center}
is commutative. Fix a $k$-basis $\{r_i\}_{i\in I}$ of $\Gamma\left(X,\cO_X)\right)$. Pick a regular function $r$ on $X$ and write
$$a^{\#}_X(r) = \sum_{i\in F}\phi_i\otimes_kr_i$$
where $F$ is a finite subset of $I$ and $\phi_i$ are regular functions on $\bd{M}$ for $i\in F$. Let $W$ be a $k$-subspace of $\Gamma(X,\cO_X)$ spanned by $\{r_i\}_{i\in F}$ and $r$. Then from the commutative square we infer the equality
$$\sum_{i\in F}\phi_i\otimes_ka^{\#}(r_i) = \sum_{i\in F}\mu^{\#}(\phi_i)\otimes_kr_i \in \Gamma\left(\bd{M},\cO_{\bd{M}}\right)\otimes_k \Gamma\left(\bd{M},\cO_{\bd{M}}\right) \otimes_k W$$
Therefore, we have $a^{\#}(r_i)\in \Gamma\left(\bd{M},\cO_{\bd{M}}\right)\otimes_kW$ for every $i\in F$. We deduce that
$$a^{\#}(W) \subseteq \Gamma\left(\bd{M},\cO_{\bd{M}}\right)\otimes_kW$$
Now for each $k$-algebra $A$ and $A$-point $f$ of $k$-scheme $\bd{M}^{\mathrm{op}} = \bd{M}$ Corollary \ref{corollary:regular_functions_on_products} and description of $\rho$ in Example \ref{example:k_monoid_action_on_regular_functions} imply
$$\rho(f) =  \big(\Gamma\left(\bd{M},f^{\#}\right)\otimes_k 1_{\Gamma(X,\cO_X)}\big)\cdot \Gamma\left(X,a^{\#}\right)$$
Thus for every $A$-point $f$ of $\bd{M}^{\mathrm{op}}$ we have
$$\rho(f)\left(W\right) \subseteq A\otimes_kW$$
Thus $W$ is a linear $\bd{M}^{\mathrm{op}}$-subrepresentation of $\Gamma(X,\cO_X)$ and $r\in W$. This proves that each element of $\Gamma(X,\cO_X)$ is contained in a finite dimensional subrepresentation of $\bd{M}^{\mathrm{op}}$. This means that the statement holds for one dimensional vector subspaces of $\Gamma(X,\cO_X)$. If the statement holds for two finite dimensional vector subspaces of $\Gamma(X,\cO_X)$, then it also holds for their sum. Thus the assertion holds for arbitrary finite dimensional vector subspace of $\Gamma(X,\cO_X)$.
\end{proof}

\section{Morphisms of locally algebraic groups}
\noindent
We introduce fundamental notions.

\begin{definition}
Let $X$ be a scheme over $k$. If $X$ is (locally) of finite type over $k$, then we say that $X$ is \textit{an (a locally) algebraic scheme over $k$}.
\end{definition}

\begin{definition}
Let $\bd{M}$ be a monoid (group) scheme over $k$. If $\bd{M}$ is (locally) algebraic over $k$, then we say that $\bd{M}$ is \textit{an (a locally) algebraic monoid (group) over $k$}.
\end{definition}
We start with the following fundamental result.

\begin{theorem}\label{theorem:images_of_locally_algebraic_groups_homomorphisms}
Let $f:\bd{H}\ra \bd{G}$ be a morphism of locally algebraic groups over $k$. Suppose that $f$ is of finite type. Let $i:\bd{K}\ra \bd{G}$ be the scheme-theoretic image of $f$ and let $g:\bd{H}\ra \bd{K}$ be the unique morphism of schemes such that $f = i\cdot g$. Then the following assertions hold.
\begin{enumerate}[label=\emph{\textbf{(\arabic*)}}, leftmargin=3.0em]
\item $\bd{K}$ is a closed subgroup $k$-scheme of $\bd{G}$.
\item $g$ is a surjective morphism of group schemes over $k$.
\end{enumerate}
\end{theorem}
\begin{proof}
Since $f$ is quasi-compact, we deduce that $i:\bd{K} \hookrightarrow \bd{G}$ is a closed immersion determined by the kernel of $f^{\#}:\cO_{\bd{G}}\ra f_*\cO_{\bd{H}}$ and $g^{\#}:\cO_{\bd{K}}\ra g_*\cO_{\bd{H}}$ is injective morphism of sheaves. Moreover, $g$ is quasi-compact. We derive by Corollary \ref{corollary:regular_functions_on_products} and Proposition \ref{proposition:group_schemes_over_a_field_are_separated} that the square
\begin{center}
\begin{tikzpicture}
[description/.style={fill=white,inner sep=2pt}]
\matrix (m) [matrix of math nodes, row sep=3em, column sep=4em,text height=1.5ex, text depth=0.25ex] 
{ \cO_{\bd{K}\times_k\bd{K}}(V\times_k W)  & \cO_{\bd{H}\times_k \bd{H}}\left(g^{-1}(V)\times_k g^{-1}(W)\right) \\
  \cO_{\bd{K}}(V)\otimes_k \cO_{\bd{K}}(W) & \cO_{\bd{H}}\left(g^{-1}(V)\right)\otimes_k \cO_{\bd{H}}\left(g^{-1}(W)\right)  \\} ;
\path[->,line width=1.0pt,font=\scriptsize]
(m-1-1) edge node[above] {$ \left(g\times_k g\right)^{\#}_{V\times_k W} $} (m-1-2)
(m-2-1) edge node[below] {$ g^{\#}_V\otimes_k g^{\#}_W $} (m-2-2)
(m-2-1) edge node[left]  {$ \cong $} (m-1-1)
(m-2-2) edge node[right] {$ \cong $} (m-1-2);
\end{tikzpicture}
\end{center}
is commutative, where vertical arrows are canonical isomorphisms. This implies that the morphism $\left(g\times_k g\right)^{\#}$ is injective. Consider the commutative diagram
\begin{center}
\begin{tikzpicture}
[description/.style={fill=white,inner sep=2pt}]
\matrix (m) [matrix of math nodes, row sep=3em, column sep=4em,text height=1.5ex, text depth=0.25ex] 
{   \bd{H}\times_k \bd{H} & \bd{K}\times_k \bd{K} & \bd{G}\times_k\bd{G} \\
    \bd{H}                & \bd{K}                & \bd{G}\\} ;
\path[->,line width=1.0pt,font=\scriptsize]
(m-1-1) edge node[above] {$g\times_k g  $} (m-1-2)
(m-2-1) edge node[below] {$g  $} (m-2-2)
(m-1-1) edge node[left] {$ \nu_{\bd{H}} $} (m-2-1)
(m-1-3) edge node[right] {$ \nu_{\bd{G}} $} (m-2-3);
\path[right hook->,line width=1.0pt,font=\scriptsize]
(m-1-2) edge node[above] {$ i\times_k i $} (m-1-3)
(m-2-2) edge node[below] {$ i  $} (m-2-3);
\end{tikzpicture}
\end{center}
where $\nu_{\bd{G}}$ and $\nu_{\bd{H}}$ are morphisms determined by formula $(x_1,x_2)\mapsto x_1^{-1}\cdot x_2$ on functors of points. Commutativity of the diagram implies that we have equality
$$\big(\left(\nu_{\bd{G}}\right)_*\left(i\times_k i\right)_*\left(g\times_k g\right)^{\#}\big) \cdot \big(\left(\nu_{\bd{G}}\right)_*\left(i\times_k i\right)^{\#} \big) \cdot \nu_{\bd{G}}^{\#} = \big(i_*g_* \left(\nu_{\bd{H}}\right)^{\#} \big) \cdot \big(i_*g^{\#} \big) \cdot i^{\#}$$
This equality together with injectivity of $\left(g\times_k g\right)^{\#}$ implies that the kernel of
$$\big(\nu_{\bd{G}}\cdot \left(i\times_k i\right)\big)^{\#} = \big(\left(\nu_{\bd{G}}\right)_*\left(i\times_k i\right)^{\#} \big) \cdot \nu_{\bd{G}}^{\#}$$
contains $\Ker(i^{\#})$. Thus $\nu_{\bd{G}}\cdot \left(i\times_k i\right)$ factors through $i$. Hence there exists a unique morphism $\nu$ such that the square
\begin{center}
\begin{tikzpicture}
[description/.style={fill=white,inner sep=2pt}]
\matrix (m) [matrix of math nodes, row sep=3em, column sep=4em,text height=1.5ex, text depth=0.25ex] 
{   \bd{K}\times_k \bd{K} & \bd{G}\times_k\bd{G} \\
    \bd{K}                & \bd{G}\\} ;
\path[->,line width=1.0pt,font=\scriptsize]
(m-1-1) edge node[left] {$ \nu $} (m-2-1)
(m-1-2) edge node[right] {$ \nu_{\bd{G}} $} (m-2-2);
\path[right hook->,line width=1.0pt,font=\scriptsize]
(m-1-1) edge node[above] {$ i \times_k i $} (m-1-2)
(m-2-1) edge node[below] {$ i  $} (m-2-2);
\end{tikzpicture}
\end{center}
is commutative. This implies that $i:\bd{K} \hookrightarrow \bd{G}$ is a closed subgroup $k$-scheme of $\bd{G}$. Since $i$ is a monomorphism and
$$i\cdot \nu \cdot \left(g\times_k g\right) = \nu_{\bd{G}}\cdot \left(i\times_k i\right) \cdot \left(g\times_k g\right) = i \cdot g \cdot \nu_{\bd{H}}$$
we derive that $\nu \cdot \left(g\times_k g\right) = g\cdot \nu_{\bd{H}}$. Hence $g$ is a morphism of group schemes over $k$. It remains to prove that $g:\bd{H}\ra \bd{K}$ is surjective. Recall that $g$ is of finite type and $g^{\#}$ is injective. Note that these properties are preserved under base change to an algebraic closure of $k$. Moreover, the surjectivity of morphism descends along faithfuly flat base change. Thus we may assume that $k$ is algebraically closed. By {\cite[Theorem 3.4]{Constructibleandlocallyconstructiblesets}} and the fact that $g$ is of finite type, we deduce that $g\left(\bd{H}\right)$ is locally constructible in $\bd{K}$. Since $g^{\#}$ is injective, we derive that set-theoretic image $g\left(\bd{H}\right)\subseteq \bd{K}$ is dense. Thus $g\left(\bd{H}\right)$ is dense and locally constructible. Hence there exists an open and dense subset $V$ of $\bd{K}$ such that $V\subseteq g\left(\bd{H}\right)$. Since $k$ is algebraically closed and $\bd{K}$ is locally algebraic, we may pick a $k$-point $v$ in $V$. Since $V\subseteq g\left(\bd{H}\right)$, we deduce that $v\in g\left(\bd{H}\right)$ and thus $v^{-1}\in g\left(\bd{H}\right)$. Hence
$$W = v^{-1}\cdot V \subseteq g\left(\bd{H}\right)\cdot g\left(\bd{H}\right) \subseteq g\left(\bd{H}\right)$$
Thus $W$ is open neighborhood of the unit, dense in $\bd{K}$ and contained in $g\left(\bd{H}\right)$. Next
$$g\left(\bd{H}\right)\cdot W \subseteq g\left(\bd{H}\right)\cdot g\left(\bd{H}\right)\subseteq g\left(\bd{H}\right)$$
Thus $g(\bd{H})$ is open in $\bd{K}$. Now if $u\in \bd{K}\setminus g\left(\bd{H}\right)$ is a $k$-point, then
$$u\cdot g\left(\bd{H}\right)\cap g\left(\bd{H}\right) = \emptyset$$
as two distinct left cosets of an open subgroup $g\left(\bd{H}\right)$ in $\bd{K}$ are disjoint. This is contradiction, because $u\cdot g\left(\bd{H}\right)$ is open neighborhood of $u$ and $g\left(\bd{H}\right)$ is dense in $\bd{K}$. Therefore, $g\left(\bd{H}\right)$ is an open subset of $\bd{K}$ that contains all its $k$-points. Since $k$ is algebraically closed and $\bd{K}$ is locally algebraic, this implies that the closed subset $\bd{K}\setminus g\left(\bd{H}\right)$ is empty. Thus $g$ is surjective.
\end{proof}
\noindent
Now we study equivariant monomorphisms of actions of locally algebraic groups. We start by proving certain results that hold for monomorphisms locally of finite type in locally noetherian case.

\begin{proposition}\label{proposition:local_surjection_is_locally_a_locally_closed_immersion}
Let $f:X\ra Y$ be a morphism locally of finite type with $Y$ locally noetherian. Suppose that for some point $x$ in $X$ morphism $f^{\#}:\cO_{Y,f(x)}\ra \cO_{X,x}$ is surjective. Then there exist an open neighborhood $U$ of $x$ and an open neighborhood $V$ of $f(x)$ such that $f(U)\subseteq V$ and the restriction $U\ra V$ induced by $f$ is a closed immersion.
\end{proposition}
\begin{proof}
Consider an open affine neighborhood $W$ of $x$ and an open affine neighborhood $V$ of $f(x)$ such that $f(W)\subseteq V$. Let $V = \Spec A$, $W = \Spec B$ and $f = \Spec g$. Then $g$ induces a surjective map $A_{\ideal{p}}\ra B_{\ideal{q}}$ with $x$ corresponding to $\ideal{q}\in \Spec B$ and $f(x)$ corresponding to $\ideal{p}\in \Spec A$. Since $B$ is a finitely generated $A$-algebra ($g$ is of finite type), there exist $b_1,...,b_n$ generators of $B$ over $A$. Next the fact that $A_{\ideal{p}}\ra B_{\ideal{q}}$ is surjective, implies that there exist $a_1,...,a_n\in A$, $s_1,...,s_n \in A\setminus \ideal{p}$ and $t_1,...,t_n\in B\setminus \ideal{q}$ such that
$$t_i\left(g(a_i) - g(s_i)b_i\right) = 0$$
for $i=1,2,...,n$. Suppose that $\tilde{s} = s_1...s_n$ and $t = t_1...t_n g(s)$. Then
$$\frac{g(a_i)}{g(s_i)} = \frac{b_i}{1}$$
in $B_t$. Moreover, there exists a polynomial $F\in A[x_1,...,x_n]$ such that $t = F(b_1,...,b_n)$. This shows that
$$\frac{t}{1} = F\left(\frac{g(a_1)}{g(s_1)},...,\frac{g(a_n)}{g(s_n)}\right)$$
as an element of $B_t$. Thus in $B_t$ we have equality
$$\frac{t}{1} = \frac{g(a)}{g(\tilde{s})^m}$$
for some element $a\in A$ and $m\in \NN$. Note that $t\not \in \ideal{q}$ and hence $a \not \in \ideal{p}$. Since $B_t$ is generated by $\frac{b_1}{1},...,\frac{b_n}{1},\frac{1}{t}$ over $A$, we derive that the morphism $A_{\tilde{s}\cdot a}\ra B_t$ induced by $g$ is surjective. Since $\tilde{s}\not \in \ideal{p}$, we deduce that for $s = \tilde{s}\cdot a$ the restriction of $f$ to $W_t \ra V_s$ induces a closed immersion.
\end{proof}

\begin{remark}\label{remark:monomorphisms_of_fields_spectra}
Let $K \hookrightarrow L$ be a fields such that $\Spec L\hookrightarrow \Spec K$ is a monomorphism of schemes. Since the diagonal of a monomorphism is an isomorphism, we deduce that the multiplication map $L\otimes_KL\ra L$ is an isomorphism. This implies that $\mathrm{dim}_K(L) = \mathrm{dim}_L\left(L\otimes_KL\right) = 1$. Hence $K\hookrightarrow L$ is an isomorphism of fields. Thus every monomorphism of spectra of fields is an isomorphism.
\end{remark}

\begin{theorem}\label{theorem:monomorphisms_are_locally_densely_on_base_closed_immersions}
Let $f:X\ra Y$ be a monomorphism of finite type and let $Y$ be locally noetherian. Then there exists open dense subscheme $V$ of $Y$ such that the morphism $f^{-1}(V)\ra V$ induced by $f$ is a closed immersion.
\end{theorem}
The proof is based on a sequence of results.

\begin{lemma}\label{lemma:monomorphisms_over_artinian_local_rings}
Let $(A,\ideal{m})$ be an artinian local ring and $f:X\ra \Spec A$ be a monomorphism locally of finite type. Then $f$ is a closed immersion.
\end{lemma}
\begin{proof}[Proof of the lemma]
Let $K$ be a residue field of $A$ i.e. $K = A/\ideal{m}$. Since $f$ is locally of finite type, we derive that $X$ is a locally noetherian scheme. Moreover, $f$ is injective. Hence $X$ consists of a single point. Thus $X = \Spec B$ for some local artinian ring $(B,\ideal{n})$. Note that the composition $\Spec B/\ideal{n}\hookrightarrow \Spec B/\ideal{m}B \hookrightarrow \Spec K$ is a monomorphism of spectra of fields. Hence by Remark \ref{remark:monomorphisms_of_fields_spectra} we derive that it is an isomorphism. This implies that $\Spec B/\ideal{m}B\hookrightarrow \Spec K$ is simultaneously a monomorphism and a retraction of schemes. Thus $\Spec B/\ideal{m}B\hookrightarrow \Spec K$ is an isomorphism of schemes. We deduce that
$$B = f^{\#}(A) + \ideal{m}B$$
Since $\ideal{m}B \subseteq \ideal{n}$ and $\ideal{n}$ is nilpotent, this last equality shows that $B = f^{\#}(A)$. Hence $f^{\#}$ is surjective. This means that $f$ is a closed immersion.
\end{proof}

\begin{proof}[Proof of the theorem]
Let $y$ be a generic point of an irreducible component of $Y$. There are two cases.
\begin{enumerate}[label=\textbf{(\arabic*)}, leftmargin=3.0em]
\item Suppose that there exists $x$ in $X$ such that $f(x) = y$. Note that $\cO_{Y,y}$ is artinian local ring. By Lemma \ref{lemma:monomorphisms_over_artinian_local_rings} we derive that $f^{\#}:\cO_{Y,y}\ra \cO_{X,x}$ is surjective. Next by Proposition \ref{proposition:local_surjection_is_locally_a_locally_closed_immersion} we deduce that there exists an open neighborhood $U$ of $x$ and an open neighborhood $V$ of $y$ such that the restriction $U\ra V$ of $f$ is a closed immersion. Shrinking $V$ we may assume that it is affine and does not intersect any irreducible component of $Y$ other than $\bd{cl}\left(\{y\}\right)$. By Chevalley's theorem on images {\cite[Theorem 3.4]{Constructibleandlocallyconstructiblesets}} we derive that $f(U)$ is locally constructible in $V$ and contains $y$. Hence there exists open neighborhood $V_y$ of $y$ contained in $f(U)$. Since $f$ is injective, we infer that $f$ induces a closed immersion $f^{-1}(V_y)\ra V_y$.
\item Suppose that $y\not \in f(X)$. Since $f$ is of finite type and $Y$ is locally noetherian, we deduce by {\cite[Theorem 3.4]{Constructibleandlocallyconstructiblesets}} that $f(X)$ is locally constructible and does not contain $y$. This implies that there exists an open affine neighborhood $V_y$ of $y$ such that $V_y\cap f(X) = \emptyset$. Clearly $f^{-1}(V_y)\ra V_y$ is a closed immersion (inclusion of an empty closed subscheme into $V_y$).
\end{enumerate}
Since closed immersions are local on the target, we derive that $f^{-1}(V)\ra V$ is a closed immersion, where
$$V = \bigcup_{y\in Y_o}V_y$$
and $Y_o$ denotes the set of generic points of irreducible components of $Y$. By construction $V$ is dense in $Y$.
\end{proof}

\begin{theorem}\label{theorem:monomorphisms_are_locally_on_domain_closed_immersions}
Let $f:X\ra Y$ be a monomorphism locally of finite type with $Y$ locally noetherian. Then for each generic point $x$ of an irreducible component of $X$ there exist an open neighborhood $U$ of $x$ and an open neighborhood $V$ of $f(x)$ such that $f(U)\subseteq V$ and the restriction $U\ra V$ induced by $f$ is a closed immersion.
\end{theorem}
\begin{proof}
The proof is similar to the proof of Lemma \ref{lemma:monomorphisms_over_artinian_local_rings}. Note that $\cO_{X,x}$ is an artinian local ring and $f(x) = y$. Next consider the fiber $f_y:X_y\ra \Spec k(y)$ of $f$. This is a monomorphism. Hence the composition of closed immersion $\Spec k(x) \hookrightarrow X_y$ with $f_y$ is a monomorphism of spectra of fields. Thus by Remark \ref{remark:monomorphisms_of_fields_spectra} we derive that it is an isomorphism. Therefore, $f_y$ is simultaneously a monomorphism and a retraction. We deduce that $f_y$ is an isomorphism. This means that
$$\cO_{X,x} = f^{\#}(\cO_{Y,y})+\ideal{m}_y\cO_{X,x}$$
Since $\ideal{m}_y\subseteq \ideal{m}_x$ and this last ideal is nilpotent as $\cO_{X,x}$ is artinian, we infer that $\cO_{X,x} = f^{\#}(\cO_{Y,y})$. We conclude by Proposition \ref{proposition:local_surjection_is_locally_a_locally_closed_immersion}.
\end{proof}

\begin{corollary}\label{corollary:equivariant_monomorphisms_are_local_on_the_base_equivariant_closed_immersion}
Let $\bd{G}$ be a group scheme over $k$, let $X,Y$ be $k$-schemes with $\bd{G}$-actions and let $f:X\ra Y$ be a $\bd{G}$-equivariant morphism. If $f$ is a monomorphism of finite type and $Y$ is locally algebraic scheme over $k$, then there exists an open dense and $\bd{G}$-invariant subset $V$ such that the restriction $f^{-1}(V)\ra V$ of $f$ is a closed immersion.
\end{corollary}
\begin{proof}
By Theorem \ref{theorem:monomorphisms_are_locally_densely_on_base_closed_immersions} there exists an open dense subset $V$ of $Y$ such that the restriction $f^{-1}(V)\ra V$ is a closed immersion. Note that the class of closed immersions satisfies conditions \textbf{(1)}, \textbf{(2)}, \textbf{(3)} of Theorem \ref{theorem:base_locus_of_certain_classes_of_morphisms_are_invariant}. Thus we may pick $V$ that is open dense in $Y$ and $\bd{G}$-invariant. 
\end{proof}

\begin{corollary}\label{corollary:monomorphisms_of_algebraic_groups}
Let $f:\bd{H}\ra \bd{G}$ be a morphisms of finite type of locally algebraic group schemes over $k$. Then the following are equivalent.
\begin{enumerate}[label=\emph{\textbf{(\roman*)}}, leftmargin=3.0em]
\item $f$ is a monomorphism.
\item $f$ is a closed immersion.
\end{enumerate}
\end{corollary}
\begin{proof}
Assume \textbf{(i)}. By Theorem \ref{theorem:images_of_locally_algebraic_groups_homomorphisms} we may assume that $f$ is a surjective monomorphism. We view $f$ as a $\bd{H}$-equivariant morphism with respect to action of $\bd{H}$ on $\bd{G}$ given by $f$. Now Corollary \ref{corollary:equivariant_monomorphisms_are_local_on_the_base_equivariant_closed_immersion} implies that there exists an open, dense and $\bd{H}$-invariant subset $V$ of $\bd{G}$ such that the morphism $f^{-1}(V)\ra V$ induced by $f$ is a closed immersion. Every $\bd{H}$-invariant open subset of $\bd{G}$ is $\bd{G}$-invariant (this follows from the fact that $f$ is surjective). Thus $V = \bd{G}$ and we deduce \textbf{(ii)}.\\
The implication $\textbf{(ii)}\Rightarrow \textbf{(i)}$ is obvious.
\end{proof}

\section{Abelian Varieties}
\noindent
We start this section with the following general result.


\begin{theorem}[Rigidity]\label{theorem:rigidity_result}
Let $\pi:X\ra Y$ be a proper morphism of schemes such that $\pi^{\#}:\cO_Y\ra \pi_*\cO_X$ is an isomorphism of sheaves. Let $g:X\ra Z$ be a morphism of schemes. Suppose that for some point $y$ in $Y$ there is a point $z$ of $Z$ such that $\pi^{-1}(y)\subseteq g^{-1}(z)$. Then there exist an affine neighborhood $V$ of $y$ and an affine neighborhood $W$ of $z$ such that $\pi^{-1}(V) \subseteq g^{-1}(W)$. Moreover, there exists a morphism $h:V\ra W$ making the diagram
\begin{center}
\begin{tikzpicture}
[description/.style={fill=white,inner sep=2pt}]
\matrix (m) [matrix of math nodes, row sep=3em, column sep=4em,text height=1.5ex, text depth=0.25ex] 
{  \pi^{-1}(V) & W  \\
    V  &  \\} ;
\path[->,line width=1.0pt,font=\scriptsize]
(m-1-1) edge node[above] {$ \mathrm{res.\,of}\,g $} (m-1-2)
(m-1-1) edge node[left] {$ \mathrm{restriction\,of}\,\pi $} (m-2-1)
(m-2-1) edge node[below = 7pt, right = -6pt] {$ h $} (m-1-2);
\end{tikzpicture}
\end{center}
commutative, where horizontal arrow is the restriction of $g$.
\end{theorem}
\begin{proof}
Consider an affine open neighborhood of $W$ of $z$. Since $\pi$ is proper and $\pi^{-1}(y)) = g^{-1}(z)$, we derive that $\pi\big(X\setminus g^{-1}(W)\big)$ is a closed subset of $Y$ that does not contain $y$. Pick an open affine neighborhood $V$ of $y$ in $Y$ that does not intersect with $\pi\big(X\setminus g^{-1}(W)\big)$. Then $\pi^{-1}(V) \subseteq g^{-1}(W)$. Since $\pi^{\#}$ is an isomorphism we have the composition
\begin{center}
\begin{tikzpicture}
[description/.style={fill=white,inner sep=2pt}]
\matrix (m) [matrix of math nodes, row sep=3em, column sep=4em,text height=1.5ex, text depth=0.25ex] 
{  \cO_Z(W) & \Gamma\big(g^{-1}(W),\cO_X\big)  & \Gamma\big(\pi^{-1}(V),\cO_X\big)  & \cO_Y(V)\\} ;
\path[->,line width=1.0pt,font=\scriptsize]  
(m-1-1) edge node[above] {$ g^{\#}_W  $} (m-1-2)
(m-1-2) edge node[above] {$ (-)_{\mid \pi^{-1}(V)} $} (m-1-3)
(m-1-3) edge node[above] {$ \left(\pi^{\#}_V\right)^{-1} $} (m-1-4);
\end{tikzpicture}
\end{center}
This composition induces a morphism of affine schemes $h:V\ra W$. Since a morphism from a scheme to an affine scheme is determined by the morphism on global sections of structure sheaves, we derive that $h$ makes the triangle in the statement commutative.
\end{proof}
\noindent
Now we can apply this result to study complete algebraic groups over $k$. For this we need the following definition.

\begin{definition}
Let $\bd{A}$ be a geometrically integral, complete algebraic group over $k$. Then we say that $\bd{A}$ is \textit{an abelian variety over $k$}.
\end{definition}
\noindent
Now we prove the following interesting result.

\begin{theorem}\label{theorem:morphisms_from_abelian_varieties}
Let $\bd{A}$ be an abelian variety over $k$, let $\bd{G}$ be a separated group scheme over $k$ and let $f:\bd{A}\ra \bd{G}$ be a morphism of schemes over $k$. Suppose that the diagram
\begin{center}
\begin{tikzpicture}
[description/.style={fill=white,inner sep=2pt}]
\matrix (m) [matrix of math nodes, row sep=3em, column sep=1em,text height=1.5ex, text depth=0.25ex] 
{  \bd{A} &        & \bd{G}  \\
          &\Spec k &  \\} ;
\path[->,line width=1.0pt,font=\scriptsize]
(m-1-1) edge node[above] {$ f $} (m-1-3)
(m-2-2) edge node[auto] {$ e_{\bd{A}} $} (m-1-1)
(m-2-2) edge node[below = 6pt, right = 1pt] {$ e_{\bd{G}} $} (m-1-3);
\end{tikzpicture}
\end{center}
is commutative. Then $f$ is a morphism of groups schemes over $k$.
\end{theorem}
\begin{proof}
We define a morphism $g:\bd{A}\times_k \bd{A}\ra \bd{G}$ given by
$$(x_1,x_2) \mapsto f(x_1)\cdot f(x_2)\cdot f(x_1\cdot x_2)^{-1}$$
where $A$ is a $k$-algebra and $x_1,x_2$ are $A$-points of $\bd{A}$. It suffices to show that $g$ factors through $\Spec k(e_{\bd{G}})$. For this we may change base to an algebraic closure of $k$ by faitfully flat descent. So we may assume that the field $k$ is algebraically closed and $\bd{A}$ is connected. Then the projection onto second factor $\pi:\bd{A}\times_k \bd{A}\ra \bd{A}$ is proper and $k = \Gamma\big(\bd{A},\cO_{\bd{A}}\big)$ implies that $\pi^{\#}$ is an isomorphism of sheaves on $\bd{A}$. Moreover, note that $\pi^{-1}(e_{\bd{A}})\subseteq g^{-1}(e_{\bd{G}})$. Indeed, this follows from the assumption that $f(e_{\bd{A}}) = e_{\bd{G}}$. By Theorem \ref{theorem:rigidity_result} we deduce that there exist an affine neighborhood $V$ of $e_{\bd{A}}$, an affine neighborhood $W$ of $e_{\bd{G}}$ and a morphism $h:\Spec k\ra W$ such that $\pi^{-1}(V) \subseteq g^{-1}(W)$ and the diagram
\begin{center}
\begin{tikzpicture}
[description/.style={fill=white,inner sep=2pt}]
\matrix (m) [matrix of math nodes, row sep=3em, column sep=4em,text height=1.5ex, text depth=0.25ex] 
{  \bd{A}\times_kV & W  \\
    V  &  \\} ;
\path[->,line width=1.0pt,font=\scriptsize]
(m-1-1) edge node[above] {$ \mathrm{res.\,of}\,g $} (m-1-2)
(m-1-1) edge node[left] {$ \mathrm{projection} $} (m-2-1)
(m-2-1) edge node[below = 7pt, right = -6pt] {$ h $} (m-1-2);
\end{tikzpicture}
\end{center}
is commutative. Hence for every $k$-point $v$ of $V$ we have the restiction $g_{\mid \bd{A} \times_k\Spec k(v)}$ factors through $\Spec k\left(h(v)\right)$. Since $g(v,e_{\bd{A}}) = e_{\bd{G}}$, we derive that $h(v) = e_{\bd{G}}$ and thus $g_{\mid \bd{A}\times_k\Spec k(v)}$ factors through $\Spec k(e_{\bd{G}})$. This holds for any $k$-point of $V$. Therefore, $g_{\mid \bd{A}\times_kV}$ factors through $\Spec k(e_{\bd{G}})$. Consider the kernel $i:Z\hookrightarrow \bd{A}\times_k\bd{A}$ of a pair consisting of $g$ and a morphism $\bd{A}\times_k\bd{A}\ra \bd{G}$ that factorizes through $\Spec k(e_{\bd{G}})$. Since $\bd{G}$ is separated, we derive that $i$ is a closed immersion. Moreover, $i$ dominates $\bd{A}\times_kV$. Since $\bd{A}\times_kV$ is schematically dense open subset of $\bd{A}\times_k\bd{A}$ (because $\bd{A}\times_k\bd{A}$ is integral), we derive that $i$ is an isomorphism and hence $g$ factors through $\Spec k(e_{\bd{G}})$.
\end{proof}

\begin{corollary}\label{corollary:abelian_varieties_are_commutative}
Let $\bd{A}$ be an abelian variety over $k$. Then $\bd{A}$ is a commutative group scheme over $k$.
\end{corollary}
\begin{proof}
Consider the morphism $f:\bd{A}\ra \bd{A}$ given on $A$-points of $\bd{A}$ by
$$f(x) = x^{-1}$$
where $A$ is a $k$-algebra. By Theorem \ref{theorem:morphisms_from_abelian_varieties} we derive that $f$ is a morphism of group schemes over $k$. Hence $\bd{A}$ is a commutative group scheme.
\end{proof}

\section{Representability of fixed points}

\begin{definition}
Let $\fG$ be a monoid $k$-functor and let $\alpha:\fG\times \fX\ra \fX$ be an action of $\fG$ on a $k$-functor. Then we define a $k$-subfunctor $\fX^{\fG}$ of $\fX$ by
$$\fX^{\fG}(A) = \big\{x\in \fX(A)\,\big|\,\mbox{ for every morphism }f:A\ra B\mbox{ of }k\mbox{-algebras and }g\in \fG(B)\mbox{ we have }\alpha\big(g,\fX(f)(x)\big) = \fX(f)(x)\big\}$$
for every $k$-algebra $A$. Then $\fX^{\fG}$ is called \textit{the fixed point $k$-functor}.
\end{definition}

\begin{theorem}\label{theorem:fixed_points_existence}
Let $\bd{G}$ be a group scheme over $k$ and let $a:\bd{G}\times_kX\ra X$ be an action of $\bd{G}$ on a $k$-scheme $X$. Suppose that one of the following assertions hold.
\begin{enumerate}[label=\emph{\textbf{(\roman*)}}, leftmargin=3.0em]
\item $X$ is separated.
\item $\bd{G}$ is a geometrically connected, locally algebraic group.
\end{enumerate}
\end{theorem}
\noindent
The following result is based on {\cite[Theorem 7.2]{kfunctors}} and plays the fundamental role in the proof.

\begin{lemma}\label{lemma:representability_of_fixed_point_functor}
Let $X,Y$ be $k$-schemes and let $a:Y\times_kX\ra X$ be a morphism of $k$-schemes. Suppose that one of the following assertions hold.
\begin{enumerate}[label=\emph{\textbf{(\arabic*)}}, leftmargin=3.0em]
\item $X$ is separated.
\item For every open subscheme $U$ of $X$ we have $a\left(Y\times_kU\right)\subseteq U$
\end{enumerate}
Consider a $k$-functor given by formula
$$A \mapsto \big\{f:\Spec A\ra X\,\big|\,a\cdot \left(1_Y\times_k f\right) = \mathrm{pr}_X\cdot \left(1_Y\times_kf\right)\big\}$$
where $A$ is a  $k$-algebra and $\mathrm{pr}_X:Y\times_kX \ra X$ is the projection. Then this $k$-functor is representable by a closed subscheme of $X$.
\end{lemma}
\begin{proof}[Proof of the lemma]
Assume first that $X$ is separated. Consider a morphism $\langle a, \mathrm{pr}_X\rangle:Y\times_kX\ra X\times_k X$. By {\cite[Corollary 4.6]{Monoid_k_functors}} we deduce that $\fP_{\langle a, \mathrm{pr}_X\rangle}$ corresponds to a morphism $\sigma:\fP_X\ra \iMor_k\left(\fP_Y, \fP_X\times \fP_X\right)$ of $k$-functors. Since $X$ is separated, the diagonal $\delta_X:X\ra X\times_k X$ is a closed immersion. This implies that $\fP_{\delta_X}$ is a closed immersion of $k$-functors. The fact that $Y$ is locally free over $k$ and {\cite[Theorem 7.2]{kfunctors}} imply that
$$\iMor_k\left(1_{\fP_Y}, \fP_{\delta_X} \right):\iMor_k\left( \fP_{Y},\fP_{X} \right) \hookrightarrow \iMor_k\left( \fP_{Y}, \fP_{X} \times \fP_{X} \right)$$
is a closed immersion of $k$-functors. Consider now a cartesian square
\begin{center}
\begin{tikzpicture}
[description/.style={fill=white,inner sep=2pt}]
\matrix (m) [matrix of math nodes, row sep=3em, column sep=4em,text height=1.5ex, text depth=0.25ex] 
{    \fX      & \iMor_k\left( \fP_{Y},\fP_{X} \right)               \\
  \fP_X      & \iMor_k\left( \fP_{Y}, \fP_{X} \times \fP_{X} \right)  \\} ;
\path[->,line width=1.0pt,font=\scriptsize]
(m-1-1) edge node[above] {$ j $} (m-1-2)
(m-2-1) edge node[below] {$ \sigma $} (m-2-2)
(m-1-1) edge node[left]  {$  $} (m-2-1)
(m-1-2) edge node[right] {$ \iMor_k\left(1_{\fP_Y}, \fP_{\delta_X} \right) $} (m-2-2);
\end{tikzpicture}
\end{center}
of $k$-functors. By base change $j:\fX\ra \fP_X$ is a closed immersion of $k$-functors. Thus we derive that $\fX$ is representable by a closed subscheme of $\fX$. It suffices to observe that $\fX$ is precisely the $k$-functor described in the statement. This proves the statement under the assumption \textbf{(1)}.\\
Now suppose that $a\left(Y\times_kU\right) \subseteq U$ for every open subscheme $U$ of $X$. For every open subscheme denote by $a_U:Y\times_kU\ra U$ the restriction of $a$. Let $\cU$ be an open affine cover of $X$. Then functors
$$\bigg\{\Alg_k\ni A\mapsto \big\{f:\Spec A\ra U\,\big|\,a\cdot \left(1_Y\times_k f\right) = \mathrm{pr}_X\cdot \left(1_Y\times_k f\right)\big\}\in \Set \bigg\}_{U\in \cU}$$
form an open cover ({\cite[Definition 4.5]{kfunctors}}) of the $k$-functor in the statemtent. Moreover, since each $U$ in $\cU$ is affine and hence separated, we derive by the first part of the proof that each $k$-functor in the family is representable. Now {\cite[Theorem 4.6]{kfunctors}} imply that the functor in the statement is representable. This finishes the proof in case \textbf{(2)}.
\end{proof}

\begin{lemma}\label{lemma:infinitesimal_isomorphisms}
Let $f:\bd{H}\ra \bd{G}$ be a morphism of locally algebraic groups over $k$. Suppose that the following assertions hold.
\begin{enumerate}[label=\emph{\textbf{(\arabic*)}}, leftmargin=3.0em]
\item The morphism
$$\widehat{\cO_{\bd{G},e_{\bd{G}}}} \ra  \widehat{\cO_{\bd{H},e_{\bd{H}}}}$$
induced by $f^{\#}$ is an isomorphism.
\item $f$ is a monomorphism of $k$-schemes.
\end{enumerate}
Then $f$ is an open immersion.
\end{lemma}
\begin{proof}[Proof of the lemma]
The assertion \textbf{(1)} implies that $f$ is {\'e}tale in $e_{\bd{H}}$. Let $K$ be an algebraic closure of $k$. Consider the {\'e}tale locus $U$ of $f_k = 1_K\otimes_kf:\bd{H}_K\ra \bd{G}_K$. Then $U$ is an open subscheme of $\bd{H}_K$ containing the unit. Moreover, for every $K$-point $h$ of $\bd{H}_K$ we have a commutative square
\begin{center}
\begin{tikzpicture}
[description/.style={fill=white,inner sep=2pt}]
\matrix (m) [matrix of math nodes, row sep=3em, column sep=4em,text height=1.5ex, text depth=0.25ex]
{ \bd{H}_K  & \bd{G}_K          \\
  \bd{H}_K  & \bd{G}_K  \\};
\path[->,line width=1.0pt,font=\scriptsize]
(m-1-1) edge node[above] {$ f_K $} (m-1-2)
(m-2-1) edge node[below] {$ f_K $} (m-2-2)
(m-1-1) edge node[left]  {$ h\cdot (-)  $} (m-2-1)
(m-1-2) edge node[right] {$ f_K(h)\cdot (-)  $} (m-2-2);
\end{tikzpicture}
\end{center}
where $h\cdot (-)$ and $f_K(h)\cdot (-)$ are isomorphisms of $K$-schemes. This proves that $h\cdot U\subseteq U$. Hence $U$ contains all $K$-rational points of $\bd{H}_K$. Therefore, the complement of $U$ in $\bd{H}_K$ is empty. Hence $U = \bd{H}_K$. This shows that $f_K$ is {\'e}tale and by faithfully flat descent also $f$ is {\'e}tale. Since {\'e}tale monomorphisms are open immersions, we derive that $f$ is an open immersion.
\end{proof}

\begin{proof}[Proof of the theorem]
If \textbf{(1)} holds, then the statement follows directly from Lemma \ref{lemma:representability_of_fixed_point_functor}.\\
Suppose now that \textbf{(2)} holds that is $\bd{G}$ is an algebraic group. For each $n\in \NN$ we define
$$\bd{G}_n = \Spec \cO_{\bd{G},e_{\bd{G}}}/\ideal{m}_{e_{\bd{G}}}^{n+1}$$
where $e$ is the unit of $\bd{G}$. Then $\bd{G}_n$ is the $n$-th infinitesimal neighborhood of $e$ in $\bd{G}$. Denote by $p_n:\bd{G}_n\times_kX\ra X$ the projection on the second factor. Let $a_n:\bd{G}_n\times_kX\ra X$ be the morphism induced by $a$. Note that for every open subscheme $U$ of $X$ we have $a_n\left(\bd{G}_n\times_kU\right)\subseteq U$. By Lemma \ref{lemma:representability_of_fixed_point_functor} it follows that the $k$-functor given by
$$\Alg_k \ni A \mapsto \big\{f:\Spec A\ra X\,\big|\,a_n\cdot \left(1_{\bd{G}_n}\times_k f\right) = \mathrm{pr}_n\cdot \left(1_{\bd{G}_n}\times_kf\right)\big\} \in \Set$$
is representable by a closed subscheme $Z_n$ of $X$. Consider now the quasi-coherent ideal $\cI_n$ of $Z_n$ inside $X$. Define
$$\cI = \sum_{n\in \NN}\cI_n$$
Let $i:Z\hookrightarrow X$ be a closed subscheme of $X$ determined by $\cI$. This means that $Z$ is the scheme-theoretic intersection inside $X$ of closed subschemes $Z_n$ for $n\in \NN$. We show that $Z$ represents the fixed point functor. For this assume that $A$ is a $k$-algebra and $f:\Spec A\ra X$ is a morphism of $k$-schemes such that $f$ is an $A$-point of the fixed point functor. This is equivalent with
$$a\cdot \left(1_{\bd{G}}\times_k f\right) = \mathrm{pr}_X\cdot \left(1_{\bd{G}}\times_kf\right)$$
From this equality we deduce that
$$a_n\cdot \left(1_{\bd{G}_n}\times_k f\right) = \mathrm{pr}_n\cdot \left(1_{\bd{G}_n}\times_kf\right)$$
for every $n\in \NN$ and hence $f$ factors through $Z_n$ for every $n\in \NN$. We derive that $f$ factors through $Z$. This proves that the fixed point functor is a $k$-subfunctor of the functor of points of $Z$. It suffices to prove that $Z$ is invariant with respect to $\bd{G}$-action. For this consider the morphism $b:\bd{G}\times_kZ\ra X$ induced by $a$. By {\cite[Corollary 4.6]{Monoid_k_functors}} morphism $b$ corresponds to a morphism $\sigma:\fP_{\bd{G}} \ra \iMor_k\left(\fP_{Z},\fP_X\right)$ of $k$-functors. Consider the cartesian square
\begin{center}
\begin{tikzpicture}
[description/.style={fill=white,inner sep=2pt}]
\matrix (m) [matrix of math nodes, row sep=3em, column sep=4em,text height=1.5ex, text depth=0.25ex] 
{    \fH      & \iMor_k\left( \fP_{Z},\fP_{Z} \right)               \\
  \fP_{\bd{G}}      & \iMor_k\left( \fP_{Z},\fP_{X}\right)  \\} ;
\path[->,line width=1.0pt,font=\scriptsize]
(m-1-1) edge node[above] {$ j $} (m-1-2)
(m-2-1) edge node[below] {$ \sigma $} (m-2-2)
(m-1-1) edge node[left]  {$  $} (m-2-1)
(m-1-2) edge node[right] {$ \iMor_k\left(1_{\fP_Z}, \fP_{i} \right) $} (m-2-2);
\end{tikzpicture}
\end{center}
The fact that $Z$ is locally free over $k$ and {\cite[Theorem 7.2]{kfunctors}} imply that $\iMor_k\left(\fP_{Z},\fP_{i}\right)$ is a closed immersion of $k$-functors. Hence $j:\fH \hookrightarrow \fP_{\bd{G}}$ is a closed immersion. Moreover, $\fH$ is a subgroup $k$-functor of $\fP_{\bd{G}}$. Thus we deduce that $j$ is induced by a closed immersion of an algebraic groups $f:\bd{H} \hookrightarrow \bd{G}$. By definition of $i:Z\hookrightarrow X$, we derive that morphism of local $k$-algebras
$$\widehat{\cO_{\bd{G},e_{\bd{G}}}} \ra  \widehat{\cO_{\bd{H},e_{\bd{H}}}}$$
induced by $f^{\#}$ is an isomorphism. Hence by Lemma \ref{lemma:infinitesimal_isomorphisms} $f$ is an open immersion of locally algebraic groups. Since $\bd{G}$ is geometrically connected, we deduce that $f$ is an isomorphism. Thus $j$ is an isomorphism and this means that $b:\bd{G}\times_k Z\ra X$ factors through $i:Z\hookrightarrow X$.
\end{proof}

\section{Absolutely irreducible representations}
\noindent
If $K$ is a field over $k$ and $\bd{G}$ is a group scheme over $k$, then we denote $\Spec K\times_k \bd{G}$ by $\bd{G}_K$.

\begin{definition}
Let $V$ be a linear representation of a monoid $k$-functor $\fG$. Then $V$ is \textit{irreducible} if it has no nontrivial quotient representations.
\end{definition}
\noindent
The next results explains when irreducibility of a linear representation of a group $k$-scheme is preserved by change of a base field.

\begin{theorem}\label{theorem:absolutely_irreducible}
Let $\bd{G}$ be a group scheme over $k$. Suppose that $V$ is a finite dimensional representation of $\bd{G}$. Then the following assertions are equivalent.
\begin{enumerate}[label=\emph{\textbf{(\roman*)}}, leftmargin=3.0em]
\item For every field extension $K$ of $k$ the representation $K\otimes_kV$ of $\bd{G}_K$ is irreducible.
\item For every finite field extension $K$ of $k$ the representation $K\otimes_kV$ of $\bd{G}_K$ is irreducible.
\item For some algebraically closed extension $K$ of $k$ the representation $K\otimes_kV$ of $\bd{G}_K$ is irreducible.
\end{enumerate}
\end{theorem}
\begin{proof}
In order to prove the theorem we introduce certain action of $\bd{G}$ associated with the representation $V$. For this denote by $\rho:\fP_{\bd{G}}\ra \cL_V$ the morphism of monoid $k$-functors that makes $V$ into a $\bd{G}$-representation. Then for every positive integer $n$ morphism $\rho$ induces an action of $\fP_{\bd{G}}$ on grassmannian $k$-functor $\mathrm{Grass}_{V,n}$ (for the definition c.f. {\cite[Section 6]{kfunctors}}). Indeed, for each $k$-algebra $A$ we have an $A$-module automorphism $\rho(g):A\otimes_kV \ra A\otimes_kV$ and hence $\rho(g)$ permutes projective quotients of rank $n$ of $A\otimes_kV$. Since by {\cite[Theorem 6.3]{kfunctors}} and {\cite[Theorem 6.5]{kfunctors}} the $k$-functor $\mathrm{Grass}_{V,n}$ is representable by a separated $k$-scheme $\bd{Gr}(V,n)$ locally of finite type over $k$, we derive that there exists the corresponding action of $\bd{G}$ on $\bd{Gr}(V,n)$. Theorem \ref{theorem:fixed_points_existence} implies that the fixed points $k$-scheme $\bd{Gr}(V,n)^{\bd{G}}$ exists and is a closed subscheme of $\bd{Gr}(V,n)$. In particular, $\bd{Gr}(V,n)^{\bd{G}}$ is locally of finite type. Next if $A$ is a $k$-algebra, then $A$-points of $\bd{Gr}(V,n)^{\bd{G}}$ can be identified with quotients $\phi:A\otimes_kV\twoheadrightarrow U$ such that for every morphism of $k$-algebras $f:A\ra B$ and $g\in \fP_{\bd{G}}(B)$ we have
$$\rho(g)\left(\Ker(\phi)\right) = \Ker(\phi)$$
Now let $K$ be a field extension of $k$. Then from remarks above we deduce that $\bd{Gr}(V,n)^{\bd{G}}$ has a $K$-point for some $0 < n <\mathrm{dim}_k(V)$ if and only if $K\otimes_kV$ is not an irreducible linear representation of $\bd{G}_K$. Since for every $0 < n < \mathrm{dim}_k(V)$ scheme $\bd{Gr}(V,n)^{\bd{G}}$ is locally of finite type, we derive that
$$\bd{Gr}(V,n)^{\bd{G}}= \emptyset\,\Leftrightarrow\,\mbox{there are no $K$-points of $\bd{Gr}(V,n)^{\bd{G}}$ for finite extensions $K$ of $k$}$$
$$\Leftrightarrow\,\mbox{there are no $K$-points of $\bd{Gr}(V,n)^{\bd{G}}$ for some algebraically closed extension $K$ of $k$}$$
Thus we deduce that for every field extension $K$ of $k$ representation $K\otimes_kV$ of $\bd{G}_K$ is irreducible if and only if for every finite extension $K$ of $k$ representation $K\otimes_kV$ of $\bd{G}_K$ is irreducible if and only if for some algebraically closed  extension $K$ of $k$ representation $K\otimes_kV$ of $\bd{G}_K$ is irreducible.
\end{proof}

\begin{definition}
Let $V$ be a linear representation of a group $k$-scheme $\bd{G}$. Then $V$ is \textit{absolutely irreducible} if equivalent conditions in Theorem \ref{theorem:absolutely_irreducible} are satisfied.
\end{definition}

\section{Linearly reductive group schemes over fields}
\noindent
In this section we introduce important class of affine group $k$-schemes. Let $\bd{G}$ be an affine group $k$-scheme and let $\Delta_{\bd{G}}:k[\bd{G}]\ra k[\bd{G}]\otimes_kk[\bd{G}]$ be the corresponding comultiplication. If $K$ is a field over $k$ and $\bd{G}$ is a group scheme over $k$, then we denote $\Spec K\times_k \bd{G}$ by $\bd{G}_K$.

\begin{proposition}[{\cite[3.4 and 3.7]{jantzen2003representations}}]\label{proposition:forgetful_functor_to_vector_space_admits_a_right_adjoint}
Let $\bd{G}$ be an affine group $k$-scheme. Then a functor
$$\Vect_k \ni V\mapsto \left(k[\bd{G}]\otimes_kV,\Delta_{\bd{G}}\otimes_k V \right)\in \bd{Rep}(\bd{G})$$
is right adjoint to the forgetful functor $\bd{Rep}(\bd{G})\ra \Vect_k$.
\end{proposition}

\begin{corollary}\label{corollary:canonical_injective_objects_in_representations_of_groups}
Let $\bd{G}$ be an affine group $k$-scheme. For every vector $k$-space the $\bd{G}$-representation
$$\left(k[\bd{G}]\otimes_kV,\Delta_{\bd{G}}\otimes_k1_V\right)$$
is an injective object of $\bd{Rep}(\bd{G})$.
\end{corollary}
\begin{proof}
The functor
$$\Vect_k \ni V\mapsto \left(k[\bd{G}]\otimes_kV,\Delta_{\bd{G}}\otimes_k V \right)\in \bd{Rep}(\bd{G})$$
is right adjoint to the forgetful functor $\bd{Rep}(\bd{G})\ra \Vect_k$ by Proposition \ref{proposition:forgetful_functor_to_vector_space_admits_a_right_adjoint}. Moreover, the forgetful functor $\bd{Rep}(\bd{G})\ra \Vect_k$ is exact. Now right adjoint to an exact functor between abelian categories sends injective objects to injective objects. Since every vector space over $k$ is injective, we derive that $\left(k[\bd{G}]\otimes_kV,\Delta_{\bd{G}}\otimes_k1_V\right)$ is an injective object in $\bd{Rep}(\bd{G})$ for every vector space $V$ over $k$.
\end{proof}

\begin{proposition}[{\cite[3.7]{jantzen2003representations}}]\label{proposition:tensor_products_of_representations_and_its_trivial_version_with_regular_representation_are_isomorphic}
Let $V$ be a representation of an affine group $k$-scheme $\bd{G}$ with respect to coaction $c:V\ra k[\bd{G}]\otimes_kV$. Denote by $V_{tr}$ the trivial representation of $\bd{G}$ with $V$ as the underlying vector $k$-space. Then
$$k[\bd{G}]\otimes_kV \ni f\otimes v \mapsto f\cdot c(f) \in k[\bd{G}]\otimes_kV$$
is an isomorphism of tensor products of $\bd{G}$-representations.
\end{proposition}

\begin{corollary}\label{corollary:canonical_injective_objects_in_representations_of_groups_enhanced}
Let $\bd{G}$ be an affine group $k$-scheme and let $V$ be a $\bd{G}$-representation. Then the tensor product $k[\bd{G}]\otimes_kV$ of $\bd{G}$-representations is an injective object in $\bd{Rep}(\bd{G})$.
\end{corollary}
\begin{proof}
This is a consequence of Corollary \ref{corollary:canonical_injective_objects_in_representations_of_groups} and Proposition \ref{proposition:tensor_products_of_representations_and_its_trivial_version_with_regular_representation_are_isomorphic}.
\end{proof}

\begin{theorem}\label{theorem:characterization_of_fully_reducible_groups}
Let $\bd{G}$ be an affine group $k$-scheme. Then the following are equvalent.
\begin{enumerate}[label=\emph{\textbf{(\roman*)}}, leftmargin=3.0em]
\item $\bd{Rep}(\bd{G})$ is a semisimple abelian category.
\item Let $k$ be a trivial $\bd{G}$-representation and let $i:k\hookrightarrow k[\bd{G}]$ be the inclusion of constant functions. Then $i$ is a split monomorphism of $\bd{G}$-representations.
\end{enumerate}
\end{theorem}
\begin{proof}
In order to prove $\textbf{(i)}\Rightarrow \textbf{(ii)}$ it suffices to note that in semisimple abelian categories all monomorphisms split.\\
Suppose that \textbf{(ii)} holds. Pick a $\bd{G}$-representation $V$. Since $i:k\hookrightarrow k[\bd{G}]$ is a split monomorphism, we derive that the tensor product
$$i\otimes_kV:k\otimes_kV\hookrightarrow k[\bd{G}]\otimes_kV$$
is a split monomorphism. By Corollary \ref{corollary:canonical_injective_objects_in_representations_of_groups_enhanced} we deduce that $k[\bd{G}]\otimes_kV$ is an injective object of $\bd{Rep}(\bd{G})$. Hence $V \cong k\otimes_kV$ is an injective object of $\bd{Rep}(\bd{G})$, because it is a direct summand of $[\bd{G}]\otimes_kV$. Thus every object of $\bd{Rep}(\bd{G})$ is injective and hence every object of $\bd{Rep}(\bd{G})$ is completely reducible. This completes the proof of the implication $\textbf{(ii}\Rightarrow \textbf{(i)}$.
\end{proof}

\begin{proposition}\label{proposition:invariants_are_stable_under_change_of_base_field}
Let $\bd{G}$ be an affine group $k$-scheme and let $V$ be a representation of $\bd{G}$. Then for every field $K$ over $k$ the natural map of vector spaces over $K$
$$K\otimes_kV^{\bd{G}} \ra \left(K\otimes_kV\right)^{\bd{G}_K}$$
is an isomorphism.
\end{proposition}
\begin{proof}
Note that we have a left exact sequence of $k$-vector spaces defining invariants
\begin{center}
\begin{tikzpicture}
[description/.style={fill=white,inner sep=2pt}]
\matrix (m) [matrix of math nodes, row sep=3em, column sep=2.5em,text height=1.5ex, text depth=0.25ex] 
{0 &  V^{\bd{G}} &   V& k[\bd{G}]\otimes_{k}V    \\} ;
\path[->,font=\scriptsize]  
(m-1-1) edge node[above] {$ $} (m-1-2)
(m-1-2) edge node[above] {$ $} (m-1-3)
(m-1-3) edge node[above] {$c-p $} (m-1-4);
\end{tikzpicture}
\end{center}
where $c:V\ra k[\bd{G}] \otimes_{k}V$ is the coaction and $p:V\ra k[\bd{G}] \otimes_{k}V$ is the trivial coaction defined by formula $p(v)=1\otimes v$ for every $v \in V$. Now tensoring the sequence with $K$ over $k$ yields a left exact sequence
\begin{center}
\begin{tikzpicture}
[description/.style={fill=white,inner sep=2pt}]
\matrix (m) [matrix of math nodes, row sep=3em, column sep=2.5em,text height=1.5ex, text depth=0.25ex] 
{0 &  K\otimes_kV^{\bd{G}} &   K\otimes_kV & k[\bd{G}] \otimes_{K}\left(K\otimes_kV\right)   \\} ;
\path[->,font=\scriptsize]  
(m-1-1) edge node[above] {$ $} (m-1-2)
(m-1-2) edge node[above] {$ $} (m-1-3)
(m-1-3) edge node[above] {$c_K - p_K $} (m-1-4);
\end{tikzpicture}
\end{center}
where $c_K$ is the coaction on $K\otimes_kV$ induced by $c$ and $c_K$ is the trivial coaction on $K\otimes_kV$. This shows that $K\otimes_kV^{\bd{G}} \ra \left(K\otimes_kV\right)^{\bd{G}_K}$ is an isomorphism.
\end{proof}

\begin{theorem}\label{theorem:fully_reducible_groups_and_field_extension}
Let $\bd{G}$ be an affine group $k$-scheme. Then the following are equvalent.
\begin{enumerate}[label=\emph{\textbf{(\roman*)}}, leftmargin=3.0em]
\item $\bd{Rep}(\bd{G})$ is semisimple abelian category.
\item The category $\bd{Rep}(\bd{G}_K)$ is semisimple abelian for every field extension $K$ of $k$.
\item The category $\bd{Rep}(\bd{G}_K)$ is semisimple abelian for some field extension $K$ of $k$.
\item The category $\bd{Rep}(\bd{G}_K)$ is semisimple abelian for some algebraically closed field extension $K$ of $k$.
\end{enumerate}
\end{theorem}
\begin{proof}
Implications $\textbf{(i)}\Rightarrow \textbf{(ii)},\,\textbf{(ii)}\Rightarrow \textbf{(iii)}$ and $\textbf{(iii)}\Rightarrow \textbf{(iv)}$ follow from Theorem \ref{theorem:characterization_of_fully_reducible_groups}, since split monomorphisms are preserved by the flat base change.\\
Suppose now that \textbf{(iv)} holds. Fix a finite dimensional representation $V$ of $\bd{G}$. Let $W$ be an arbitrary subrepresentation of $V$ and denote by $j:W\hookrightarrow V$ the inclusion. If $\Hom_{k}(V,W)$ and $\Hom_{k}(W,W)$ are equipped with the usual structures of representations of $\bd{G}$, then $\Hom_{k}(j,1_W)$ is a surjective morphism of representations. This means that 
$$\Hom_{K}(1_K\otimes_kj,1_K\otimes_k1_W):\Hom_{K}(K\otimes_kV,K\otimes_KW) \twoheadrightarrow  \Hom_{K}(K\otimes_kW,K\otimes_kW)$$
is surjective morphism of representations of $\bd{G}_K$. Since $\bd{Rep}(\bd{G}_K)$ is semisimple, we derive that the functor of invariants $(-)^{\bd{G}_K}$ is exact. Hence the induced morphism
$$\Hom_{K}(1_K\otimes_kj,1_K\otimes_k1_W)^{\bd{G}_K}:\Hom_{K}(K\otimes_kV,K\otimes_KW)^{\bd{G}_K} \twoheadrightarrow \Hom_{K}(K\otimes_kW,K\otimes_kW)^{\bd{G}_K}$$
of $K$-vector spaces is surjective. According to Proposition , we deduce that the morphism 
$$K\otimes_k\Hom_{k}(V,W)^{\bd{G}} \twoheadrightarrow K\otimes_k \Hom_{k}(W,W)^{\bd{G}}$$
induced by $j$ is surjective. By faithfully flat descent, we derive that 
$$\Hom_{\bd{G}}(j,1_W):\Hom_{\bd{G}}(V,W)= \Hom_{k}(V,W)^{\bd{G}}\twoheadrightarrow \Hom_{k}(W,W)^{\bd{G}}=\Hom_{\bd{G}}(W,W)$$
is surjective. This means that there exists a morphism of representations $r:V\ra W$ such that $r\cdot j = 1_W$. Hence $j$ splits. Therefore, every representation of $\bd{G}$ of finite dimension over $k$ is completely reducible. Since every representation of $\bd{G}$ is a sum of its finitely dimensional subrepresentations, we derive that every representation of $\bd{G}$ is a sum of its completely reducible subrepresentations. Hence every representation of $\bd{G}$ is completely reducible and this means that $\bd{Rep}(\bd{G})$ is semisimple abelian category.
\end{proof}

\begin{definition}
Let $\bd{G}$ be an affine group $k$-scheme and suppose that equivalent conditions of Theorem \ref{theorem:fully_reducible_groups_and_field_extension} are satisfied for $\bd{G}$. Then we say that $\bd{G}$ is \textit{fully reducible group over $k$}.
\end{definition}

\begin{definition}
Let $\bd{G}$ be a fully reducible group $k$-scheme. Suppose that $\bd{G}$ is smooth and algebraic over $k$. Then we say that $\bd{G}$ is \textit{linearly reductive group over $k$}.
\end{definition}































































































\small
\bibliographystyle{alpha}
\bibliography{../zzz}




\end{document}
