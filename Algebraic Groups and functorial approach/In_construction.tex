\input ../pree.tex

\begin{document}

\title{In construction}
\date{}
\maketitle

\section{Introduction}
\noindent
In this notes we study affine monoid $k$-schemes by using their linear representation.


\section{Comodules}
\begin{theorem}\label{theorem:limits_creation_forgetful_from_coalgebras_to_modules}
Let $C$ be a $k$-coalgebra such that $C$ is a flat $k$-module. Then the forgetful functor $\bd{coMod}(C)\ra \Mod(k)$ creates finite limits.
\end{theorem}
\begin{proof}
Let $\Delta,\xi$ be the comultiplication and the counit of $C$, respectively. Suppose that $I\ni i\mapsto (V_i,d_i)\in \bd{coMod}(C)$ is a diagram of $C$-comodules indexed by some finite category $I$. Let $V$ together with $u_i:V\ra V_i$ for $i\in I$ be a limit of the diagram $I\ni i\mapsto V_i\in \Mod(k)$. Then $C\otimes_{k}V$ together with $1_C\otimes_{k}u_i:C\otimes_{k}V\ra C\otimes_{k}V_i$ for $i\in I$ is a limit of the diagram $I\ni i\mapsto C\otimes_{k}V_i\in \Mod(k)$. By the universal property of limit we deduce that there exists a unique morphism $d:V\ra C\otimes_{k}V$ such that diagrams 
\begin{center}
\begin{tikzpicture}
[description/.style={fill=white,inner sep=2pt}]
\matrix (m) [matrix of math nodes, row sep=3em, column sep=2em,text height=1.5ex, text depth=0.25ex] 
{V&  &       V_i    \\
C\otimes_{k}V &   &   C\otimes_{k}V_i  \\} ;
\path[->,line width=1.0pt,font=\scriptsize]
(m-1-1) edge node[above] {$ p_i$} (m-1-3)
(m-1-3) edge node[right] {$d_i $} (m-2-3)
(m-1-1) edge node[left] {$d $} (m-2-1)
(m-2-1) edge node[below] {$ 1_{C}\otimes_{k}p_i$} (m-2-3);
\end{tikzpicture}
\end{center} 
are commutative for every $i\in I$. Note that $C\otimes_{k}C\otimes_{k}V$ together with morphisms $1_C\otimes_{k}1_C\otimes_{k}p_i:C\otimes_{k}C\otimes_{k}V\ra C\otimes_{k}C\otimes_{k}V_i$ for every $i\in I$ is a limit of the diagram given by $I\ni i\mapsto C\otimes_{k}C\otimes_{k}V_i\in \Mod(k)$. In order to show that $(V,d)$ is a $C$-comodule note that for every $i\in I$ we have
$$\left(1_C\otimes_{k}1_C\otimes_{k}p_i\right)\cdot \left(1_C\otimes_{k}d\right)\cdot d=\left(1_C\otimes_{k}d_i\right)\cdot d_i\cdot p_i=$$
$$=\left(\Delta\otimes_{k}1_{V_i}\right)\cdot d_i\cdot p_i=\left(1_C\otimes_{k}1_C\otimes_{k}p_i\right)\cdot \left(\Delta\otimes_{k}1\right)\cdot d$$
and
$$\left(1_{k}\otimes_{k}p_i\right)\cdot \left(\xi\otimes_{k}1\right)\cdot d= \left(\xi\otimes_{k}1_{V_i}\right)\cdot d_i\cdot p_i=j_{V_i}\cdot p_i=\left(1_{k}\otimes_{k}p_i\right)\cdot j_V$$
where $j_W:W\ra k\otimes_{k}W$ is the natural isomorphism for every $k$-module $W$. Thus by the universality of a limit we deduce that $d$ is a coaction of $C$ on $V$. Suppose now that $(W,e)$ is a $C$-comodule and $r_i:W\ra V_i$ for $i\in I$ is a family of $C$-comodule morphisms compatible with the diagram $I\ni i\mapsto (V_i,d_i)\in \bd{coMod}(C)$. Then there exists a unique morphism of $k$-modules $f:W\ra V$ such that $p_i\cdot f=r_i$ for every $i$. Note that
$$(1_C\otimes_{k}p_i)\cdot d\cdot f= d_i\cdot p_i\cdot f=d_i\cdot r_i=\left(1_C\otimes_{k}r_i\right)\cdot e=\left(1_C\otimes_{k}p_i\right)\cdot \left(1_C\otimes_{k}f\right)\cdot e$$ 
for every $i\in I$. Hence $d\cdot f=\left(1_C\otimes_{k}f\right)\cdot e$. Thus $f$ is a morphism of left comodules over $C$. Thus $(V,d)$ and together with family $\{p_i:(V,d)\ra (V_i,d_i)\}_{i\in I}$ is a limit of the diagram $I\ni i\mapsto (V_i,d_i)\in \bd{coMod}(C)$ of $C$-comodules. This implies that the forgetful functor $\bd{coMod}(C)\ra \Mod(k)$ creates finite limits.
\end{proof}

\section{Motivation -- linear representations of compact topological groups}
\noindent
In this section we fix a compact topological group $\bd{G}$.
Assume that $\rho:\bd{G}\ra \mathrm{GL}_n(\CC)$ is a continuous homomorphism i.e. a complex, $n$-dimensional linear representation of $\bd{G}$. For every $g\in \bd{G}$ we get a matrix
$$\rho(g)=\left[c_{ij}(g)\right]_{1\leq i, j \leq n}$$
For $i$, $j$ function $c_{ij}:\bd{G}\ra \CC$ is a continuous complex valued function. Alternatively suppose that $\{e_1,e_2,...,e_n\}$ is the standard basis of $\CC^n$ on which $\mathrm{GL}_n(\CC)$ act. Then $c_{ij}$ is equal to a function 
$$\bd{G}\ni g \mapsto \langle g\cdot e_i, e_j\rangle \in \CC$$
Fix now $g_1$, $g_2\in \bd{G}$ and note that
$$\left[c_{ij}(g_2\cdot g_1)\right]_{1\leq i,j\leq n}  = \rho(g_2\cdot g_1) = \rho(g_2)\cdot \rho(g_1) = \left[\sum_{k=1}^nc_{ik}(g_2)\cdot c_{kj}(g_1)\right]_{1\leq i,j\leq n}$$
Hence 
$$c_{ij}(g_2\cdot g_1) = \sum_{k=1}^nc_{ik}(g_2)\cdot c_{kj}(g_1)$$
for every $1\leq i,j\leq n$. This implies that $\sum_{1\leq i,j\leq n}\CC\cdot c_{ij}\subseteq \cL^2(\bd{G},\CC)$ is a linear $\bd{G}\times \bd{G}^{\mathrm{op}}$-subrepresentation of the regular representation $\cL^2(\bd{G},\CC)$. We call it \textit{the matrix coefficients of $\rho$}. 


\section{The category of linear representations}
\noindent
In this section we fix a monoid $k$-functor $\fG$. Note that there exists the forgetful functor $\bd{Rep}(\fG)\ra \Mod(k)$ that sends each linear representation to its underlying $k$-module.

\begin{theorem}\label{theorem:forgetful_functor_from_reps_creates_colmits_and_flimits}
The forgetful functor
\begin{center}
\begin{tikzpicture}
[description/.style={fill=white,inner sep=2pt}]
\matrix (m) [matrix of math nodes, row sep=3em, column sep=3em,text height=1.5ex, text depth=0.25ex] 
{ \bd{Rep}(\fG)  & \Mod(k) \\};
\path[->,line width=1.0pt,font=\scriptsize]  
(m-1-1) edge node[auto] {$ $} (m-1-2);
\end{tikzpicture}
\end{center}
creates small colimits.
\end{theorem}
\begin{proof}
Suppose that $I\ni i\mapsto (V_i,\rho_i)\in \bd{Rep}(\fG)$ is a diagram of linear representations of $\fG$ indexed by some category $I$. Let $V$ together with $u_i:V_i\ra V$ for $i\in I$ be a colimit of the diagram $I\ni i\mapsto V_i\in \Mod(k)$.\\
Assume first that $(V,\rho)$ is a structure of the linear representation of $\fG$ on $V$ such that $u_i:V_i\ra V$ for $i\in I$ becomes a cocone over the diagram $I\ni i\mapsto (V_i,\rho_i)\in \bd{Rep}(\fG)$. For every $k$-algebra $A$ the functor $A\otimes_k(-)$ preserves colimits and hence $1_A\otimes_ku_i$ for $i\in I$ is a colimit of the diagram $I\ni i \mapsto 1_A\otimes_kV_i\in \Mod(A)$. For each $i\in I$ we have an action $\rho_i^A:\fG(A)\ra \Hom_A\left(A\otimes_kV_i, A\otimes_kV_i\right)$ of $\fG(A)$ on $A\otimes_kV_i$ and we may view the diagram $I\ni i \mapsto 1_A\otimes_kV_i\in \Mod(A)$ as a diagram in the category of $A$-modules equipped with $\fG(A)$-actions given by $A$-module morphisms. We refer to this category as to category of $A$-linear $\fG(A)$-actions. Now the forgetful functor
\begin{center}
\begin{tikzpicture}
[description/.style={fill=white,inner sep=2pt}]
\matrix (m) [matrix of math nodes, row sep=3em, column sep=4em,text height=1.5ex, text depth=0.25ex] 
{ \bigg\{\mbox{the category of $A$-linear $\fG(A)$-actions}\bigg\} & \Mod(A) \\};
\path[->,line width=1.0pt,font=\scriptsize]  
(m-1-1) edge node[auto] {$ $} (m-1-2);
\end{tikzpicture}
\end{center}
creates small limits. Indeed, the category on the right hand side is isomorphic with the category $\Mod\left(A[\fG(A)]\right)$ of left modules over the monoid $A$-algebra $A[\fG(A)]$ and the forgetful functor
\begin{center}
\begin{tikzpicture}
[description/.style={fill=white,inner sep=2pt}]
\matrix (m) [matrix of math nodes, row sep=3em, column sep=4em,text height=1.5ex, text depth=0.25ex] 
{ \Mod\left(A[\fG(A)]\right) & \Mod(A) \\};
\path[->,line width=1.0pt,font=\scriptsize]  
(m-1-1) edge node[auto] {$ $} (m-1-2);
\end{tikzpicture}
\end{center}
creates small colimits. This implies that $\rho^A:\fG(A) \ra \Hom_A(A\otimes_kV,A\otimes_kV)$ must be a unique morphism of monoids such that $1_A\otimes_ku_i$ for every $i\in I$ is a morphism of $A$-modules with $A$-linear action of $\fG(A)$. This implies that $\rho$ is unique and hence $(V,\rho)$ is unique lift of $\left(V,\{u_i\}_{i\in I}\right)$ to $\bd{Rep}(\fG)$. This shows the uniqueness of a lift.\\
For the existence assume for given $k$-algebra $A$ that $\rho^A:\fG(A) \ra \Hom_A(A\otimes_kV,A\otimes_kV)$ is a unique morphism of monoids such that $1_A\otimes_ku_i$ for every $i\in I$ is a morphism of $A$-modules with $A$-linear action of $\fG(A)$. Note that $\rho^A$ exists because the forgetful functor
\begin{center}
\begin{tikzpicture}
[description/.style={fill=white,inner sep=2pt}]
\matrix (m) [matrix of math nodes, row sep=3em, column sep=4em,text height=1.5ex, text depth=0.25ex] 
{ \bigg\{\mbox{the category of $A$-linear $\fG(A)$-actions}\bigg\} & \Mod(A) \\};
\path[->,line width=1.0pt,font=\scriptsize]  
(m-1-1) edge node[auto] {$ $} (m-1-2);
\end{tikzpicture}
\end{center}
creates small colimits. Denote $\rho = \{\rho^A\}_{A\in \Alg_k}$. We verify that $\rho$ is a morphism of $k$-functors $\rho:\fG\ra \cL_V$. For this consider morphism $f:A\ra B$ of $k$-algebras and the commutative square
\begin{center}
\begin{tikzpicture}
[description/.style={fill=white,inner sep=2pt}]
\matrix (m) [matrix of math nodes, row sep=3em, column sep=5em,text height=1.5ex, text depth=0.25ex] 
{  A\otimes_k V_i   & A\otimes_kV \\
   B\otimes_k V_i   & B\otimes_kV \\} ;
\path[->,line width=1.0pt,font=\scriptsize]  
(m-1-1) edge node[above] {$ 1_A\otimes_ku_i $} (m-1-2)
(m-2-1) edge node[below] {$ 1_B\otimes_ku_i$} (m-2-2)
(m-1-1) edge node[left] {$ f\otimes_k1_{V_i} $} (m-2-1)
(m-1-2) edge node[right] {$ f\otimes_k1_V $} (m-2-2);
\end{tikzpicture}
\end{center}
defined for every $i\in I$. Note that the top row of the square is a morphism of $A$-modules with $A$-linear $\fG(A)$-actions. Similarly interpreting $B\otimes_kV_i$ and $B\otimes_kV$ as $A$-modules with $A$-linear actions of $\fG(A)$ given by $\rho^B_i\cdot \fG(f)$ and $\rho^B\cdot \fG(f)$, respectively, we derive that the square consists of $A$-modules with $A$-linear actions of $\fG(A)$ and all maps preserve actions except possibly $f\otimes_k1_V$. Since $A\otimes_kV$ together with $1_A\otimes_ku_i$ for $i\in I$ is a colimit of $I\ni i \mapsto 1_A\otimes_kV_i\in \Mod(A)$ in the category of $A$-modules, we deduce that $f\otimes_k1_V$ is the only morphism of $A$-modules making squares commutative for all $i\in I$. Since $A\otimes_kV$ with $\rho^A$ and $1_A\otimes_k u_i$ for $i\in I$ is a colimit of the same diagram, but interpreted as a diagram of $A$-modules with $A$-linear action of $\fG(A)$-modules, we derive from uniqueness of $f\otimes_k1_V$ that it must also preserve $\fG(A)$-action. Hence $\left(f\otimes_k1_V\right)\cdot \rho^A = \rho^B\cdot \fG(f)$. Thus $\rho$ is a morphism of $k$-functors. By definition of $\rho^A$ for each $k$-algebra $A$, we derive that it is a morphism of monoid $k$-functors. Hence $(V,\rho)$ is a linear representation of $\fG$ and again by componentwise definition of $\rho$ we deduce that $(V,\rho)$ is a colimit of the diagram $I\ni i\mapsto (V_i,\rho_i)\in \bd{Rep}(\fG)$. 
\end{proof}

\begin{theorem}
Let $A$ be a commutative ring. The following assertions are equivalent.
\begin{enumerate}[label=\emph{\textbf{(\roman*)}}, leftmargin=3.0em]
\item $\Spec A$ is a Hausdorff space.
\item Every prime ideal of $A$ is maximal.
\item Every $A/\cN$-module is flat, where $\cN$ is a nilradical of $A$.
\item Every finitely generated ideal of $A$ is generated by an idempotent.
\end{enumerate}
\end{theorem}

\begin{lemma}\label{lemma:flatness_in_stalks}
Let $A$ be a commutative ring and $M$ be an $A$-module. Then $M$ is flat if and only if $M_{\ideal{p}}$ is flat for all $\ideal{p}\in \Spec A$.
\end{lemma}
\begin{proof}[Proof of the lemma]
For every $\ideal{p}\in \Spec A$ we have a natural isomorphism
$$M_{\ideal{p}}\otimes_A(-) \cong \left(M\otimes_A(-)\right)_{\ideal{p}}$$
Now the statement follows from the fact that a chain complex of $A$-modules is exact if and only if it is exact after localization in every prime ideal $\ideal{p}\in \Spec A$
\end{proof}

\begin{lemma}\label{lemma:absolutely_flat_local_ring}
Let $A$ be a local ring such that each $A$-module is flat. Then $A$ is a field.
\end{lemma}
\begin{proof}[Proof of the lemma]
Let $\ideal{m}$ be a maximal ideal of $A$ and $k$ be a residue field. Pick finitely generated ideal $\ideal{a}\subseteq \ideal{m}$. Consider the canonical exact sequence
\begin{center}
\begin{tikzpicture}
[description/.style={fill=white,inner sep=2pt}]
\matrix (m) [matrix of math nodes, row sep=2em, column sep=5em,text height=1.5ex, text depth=0.25ex] 
{0 & \ideal{a} &  A    & A/\ideal{a} & 0             \\} ;
\path[->,line width=1.0pt,font=\scriptsize]  
(m-1-1) edge node[auto] {$ $} (m-1-2)
(m-1-2) edge node[auto] {$ $} (m-1-3)
(m-1-3) edge node[auto] {$a \mapsto a\,\mathrm{mod}\,\ideal{a} $} (m-1-4)
(m-1-4) edge node[auto] {$ $} (m-1-5);
\end{tikzpicture}
\end{center}
Since $k$ is a flat $A$-module, we derive that the sequence
\begin{center}
\begin{tikzpicture}
[description/.style={fill=white,inner sep=2pt}]
\matrix (m) [matrix of math nodes, row sep=2em, column sep=5em,text height=1.5ex, text depth=0.25ex] 
{0 & k\otimes_A\ideal{a} &  k    & k/\ideal{a}k & 0             \\} ;
\path[->,line width=1.0pt,font=\scriptsize]  
(m-1-1) edge node[auto] {$ $} (m-1-2)
(m-1-2) edge node[auto] {$ $} (m-1-3)
(m-1-3) edge node[auto] {$\alpha \mapsto \alpha\,\mathrm{mod}\,\ideal{a}k $} (m-1-4)
(m-1-4) edge node[auto] {$ $} (m-1-5);
\end{tikzpicture}
\end{center}
is exact. Since $\ideal{a}k = 0$ because $\ideal{a}\subseteq \ideal{m}$, we deduce from the short exact sequence that $k\otimes_A\ideal{a} = 0$. By Nakayama lemma this implies that $\ideal{a} = 0$ ($\ideal{a}$ is finitely generated over $A$). Thus every finitely generated $A$-submodule of $\ideal{m}$ is trivial. Thus $\ideal{m} = 0$ and hence $A$ is a field.
\end{proof}

\section{Results on affine monoids}

\begin{definition}
Let $\fG$ be a monoid $k$-functor. We say that $\fG$ is \textit{a monoid $k$-functor with zero} if there exists a $k$-point $\bd{o}$ of $\fG$ such that the following two morphisms
\begin{center}   
\begin{tikzpicture}
[description/.style={fill=white,inner sep=2pt}]
\matrix (m) [matrix of math nodes, row sep=3em, column sep=3em,text height=1.5ex, text depth=0.25ex] 
{\bd{1}\times \fG & \fG\times\fG & \fG & \fG \times \bd{1} & \fG\times \fG & \fG  \\};
\path[->,line width=1.0pt,font=\scriptsize]    
(m-1-1) edge node[auto]  {$ \bd{o}\times 1_{\fG} $} (m-1-2)
(m-1-2) edge node[auto]  {$ \textrm{mul} $} (m-1-3)
(m-1-4) edge node[auto]  {$ 1_{\fG} \times \bd{o} $} (m-1-5)
(m-1-5) edge node[auto]  {$ \textrm{mul} $} (m-1-6);
\end{tikzpicture}
\end{center}
where $\textrm{mul}: \fG \times \fG \ra \fG$ is the multiplication on $\fG$, factor through $\bd{o}$. If this is the case, then $\bd{o}$ is called \textit{the zero of $\fG$}.
\end{definition}

\begin{definition}
Let $\fG$ be a monoid $k$-functor. For each $k$-algebra $A$ we denote by $\fG^*(A)$ the group of units of $\fG(A)$. This gives rise to a subgroup $k$-functor $\fG^*$ of $\fG$. We call $\fG^*$ \textit{the group of units of $\fG$}.
\end{definition}
\noindent
Now we describe the universal property of the group of units. Let $\fG$ be a monoid $k$-functor and let $\fG$ be a group $k$-functor. Suppose that $\sigma:\fG\ra \fG$ is a morphism of monoid $k$-functors. Then $\sigma$ factors through $\fG^*$.

\begin{proposition}\label{proposition:integral_monoids_groups_of_units_are_schematically_dense}
Let $\bd{M}$ be an affine $k$-monoid scheme and denote by $\fG$ the $k$-monoid functor that represents $\bd{M}$. Then $\fG^*$ is representable by an affine $k$-group scheme. Moreover, if $\bd{M}$ is an affine integral $k$-monoid scheme of finite type over $k$, then $\fG^*$ is an open $k$-subfunctor of $\fG$.
\end{proposition}



\section{Diagonalisable monoid $k$-schemes}
\noindent
Consider an abstract commutative monoid $\Gamma$. Consider the monoid $k$-algebra $k[\Gamma]$. Recall that $k[\Gamma]$ as a free $k$-vector space over $k$ and its elements can be uniquely written as
$$\sum_{\gamma\in \Gamma}k_{\gamma}\cdot \gamma$$
where almost all $k_{\gamma}$ are zero for $\gamma \in \Gamma$. Next the $k$-algebra $k[\Gamma]$ admits a structure of a commutative bialgebra with a comultiplication given by
$$k[\Gamma] \ni \sum_{\gamma\in \Gamma} k_{\gamma}\cdot \gamma \ra \sum_{\gamma\in \Gamma}k_{\gamma}\cdot \left(\gamma\otimes \gamma\right)\in  k[\Gamma]\otimes_kk[\Gamma]$$
and a counit
$$k[\Gamma]\ni \sum_{\gamma \in \Gamma}k_{\gamma}\cdot \gamma \mapsto \sum_{\gamma\in \Gamma}k_{\gamma}\in k$$
This makes $\Spec k[\Gamma]$ into a monoid $k$-scheme. We denote this monoid $k$-scheme by $\bd{D}_{\Gamma}$. For an alternative description note that we have identifications
$$\fP_{\bd{D}_{\Gamma}}(A) \cong \Mor_k\left(k[\Gamma],A\right) \cong \Mon\left(\Gamma,A^{\times}\right)$$
natural in $k$-algebra $A$, where the right hand side denotes the set of morphisms of monoids from $\Gamma$ to the multiplicative monoid $A^{\times}$ of $A$. The $k$-functor
$$\Alg_k\ni A\mapsto \Mon\left(\Gamma,A^{\times}\right)\in \Set$$
is a monoid $k$-functor with respect to multiplication of monoid homomorphisms in $\Mon\left(\Gamma,A^{\times}\right)$ for every $k$-algebra $A$. Hence the identification above makes the functor of points $\fP_{\bd{D}_{\Gamma}}$ into the monoid $k$-functor and induces precisely the bialgebra structure on $k[\Gamma]$ described above.\\
Note that if $g:\Gamma_1\ra \Gamma_2$ is a morphism of commutative monoids, then $k[g]:k[\Gamma_1]\ra k[\Gamma_2]$ is a morphism of bialgebras (with respect to the structure described above). We denote $\Spec k[g]$ by $\bd{D}_{g}$.

\begin{definition}
Let $\bd{M}$ be a monoid $k$-scheme. We say that $\bd{M}$ is \textit{diagonalisable} if there exists an abstract commutative monoid $\Gamma$ such that $\bd{M}$ is visomorphic to $\bd{D}_{\Gamma}$ as a monoid $k$-scheme.
\end{definition}
\noindent
Now we prove the following important result.

\begin{theorem}\label{theorem:commutative_monoids_and_diagonalisable_monoid_k_schemes}
Suppose that $k$ is commutative ring such that $\Spec k$ is connected (i.e. $k$ has no nontrivial idempotents). Consider the functor 
\begin{center}
\begin{tikzpicture}
[description/.style={fill=white,inner sep=2pt}]
\matrix (m) [matrix of math nodes, row sep=1em, column sep=2em,text height=1.5ex, text depth=0.25ex] 
{ \Gamma_1  &  & \bd{D}_{\Gamma_1}                           \\
    {}      &  & {}          \\
   \Gamma_2 &  & \bd{D}_{\Gamma_2}                  \\} ;
\path[->, line width=1.0pt, font=\scriptsize]  
(m-3-3) edge node[right] {$\bd{D}_{g} $} (m-1-3)
(m-1-1) edge node[left] {$g $} (m-3-1);
\path[|->, shorten >= 0.4cm, shorten <= 0.4cm, line width=1.0pt, font=\scriptsize]  
(m-2-1) edge node[right] {$ $} (m-2-3);
\end{tikzpicture}
\end{center}
defined on the category of commutative monoids and with values in the category of monoid schemes over $k$. This functor preserves finite products and induces an equivalence of categories between abstract commutative monoids and diagonalisable monoid schemes over $k$.
\end{theorem}
\begin{proof}
Suppose that $\Gamma_1, \Gamma_2$ are commutative monoids and $f:k[\Gamma_1]\ra k[\Gamma_2]$ is a morphism of bialgebras over $k$. Let $\Delta_1,\xi_1$ and $\Delta_2,\xi_2$ be comultiplications and counits for $k[\Gamma_1],k[\Gamma_2]$, respectively. Fix $\gamma\in \Gamma_1$ and suppose that $f(\gamma) = \sum_{\gamma'\in \Gamma_2}k_{\gamma'}\cdot \gamma'$. The fact that $f$ is a morphism of bialgebras over $k$ implies that
$$\Delta_2\left(f\left( \gamma\right)\right) = \left(f\otimes_k f\right)\left(\Delta_1(\gamma)\right) = \left(f\otimes_kf\right)(\gamma\otimes_k\gamma) = f(\gamma)\otimes_kf(\gamma)$$
Substituting $\sum_{\gamma'\in \Gamma_2}k_{\gamma'}\cdot \gamma'$ for $f(\gamma)$ we deduce that
$$\sum_{\gamma'\in \Gamma_2}k_{\gamma'}\cdot \left(\gamma'\otimes \gamma'\right) = \sum_{\gamma'\in \Gamma_2}\sum_{\gamma''\in \Gamma_2}k_{\gamma'}\cdot k_{\gamma''}\cdot \left(\gamma'\otimes \gamma''\right)$$
Thus we derive that
$$k_{\gamma'}\cdot k_{\gamma''} = \begin{cases} 0 & \mbox{ if }\gamma'\neq \gamma''\\
k_{\gamma'} & \mbox{ if }\gamma'=\gamma''
\end{cases}$$
Since there are no nontrivial idempotents in $k$, this implies that $k_{\gamma'} = 0,1$ for each $\gamma'\in \Gamma_2$. Again by the fact that $f$ is a morphism of $k$-bialgebras, we derive that
$$\xi_1(\gamma) = \xi_2\left(f(\gamma)\right)$$
Substituting $\sum_{\gamma'\in \Gamma_2}k_{\gamma'}\cdot \gamma'$ for $f(\gamma)$ yields that
$$\sum_{\gamma'\in \Gamma_2}k_{\gamma'} = 1$$
Combining this with previously established fact that $k_{\gamma'}=0,1$ for each $\gamma'\in \Gamma_2$ we deduce that there exists precisely one $\gamma'\in \Gamma_2$ such that $f(\gamma) = \gamma'$. This proves that $f(\Gamma_1)\subseteq \Gamma_2$. Since $f$ preserves multiplication and unit, we deduce that $f = k[g]$ for some homomorphism of abstract monoids $g:\Gamma_1\ra \Gamma_2$. Thus the functor described in the statement is full.\\
It is also clearly faithful. Indeed, for two distinct morphisms of monoids $g_1,g_2:\Gamma_1\ra \Gamma_2$ we have $k[g_1]\neq k[g_2]$ and hence $\Spec k[g_1] \neq \Spec k[g_2]$.\\
By definition of diagonalisable monoid the image of the functor is an essential subcategory of the category of diagonalisable $k$-schemes.\\
Finally, consider commutative monoids $\Gamma_1,\Gamma_2$ and note that isomorphism
$$k[\Gamma_1\times \Gamma_2] \ni \sum_{(\gamma_1,\gamma_2)\in \Gamma_1\times \Gamma_2}k_{(\gamma_1,\gamma_2)}\cdot (\gamma_1,\gamma_2) \mapsto \sum_{(\gamma_1,\gamma_2)\in \Gamma_1\times \Gamma_2}k_{(\gamma_1,\gamma_2)}\cdot \gamma_1\otimes \gamma_2 \in k[\Gamma_1]\otimes_kk[\Gamma_2]$$
is a morphism of $k$-bialgebras. This implies that the functor described in the statement preserves binary products. The functor preserves terminal objects, since $k$ is a monoid $k$-algebra for trivial (zero) commutative monoid.
\end{proof}

\section{Representations of diagonalisable monoid $k$-schemes}

\begin{definition}
Let $\Gamma$ be a commutative monoid and let $\bd{D}_{\Gamma}$ be the corresponding monoid $k$-scheme. Suppose that $V$ is a representation of $\bd{D}_{\Gamma}$ with respect to a morphism of monoid $k$-functors given by
$$\fP_{\bd{D}_{\Gamma}}(A) = \Mod\left(\Gamma,A^{\times}\right) \ni f\mapsto f(\gamma)\cdot (-) \in \cL_{V}(A)$$
where $\gamma$ is a fixed element of $\Gamma$. Then $V$ is called \textit{a representation of $\bd{D}_{\Gamma}$ of weight $\gamma$}.
\end{definition}

\begin{fact}\label{fact:weight_representation_as_comodules}
Let $\Gamma$ be a commutative monoid and let $\gamma$ be its element. Suppose that $V$ is a representation of $\bd{D}_{\Gamma}$ of weight $\gamma$. Then $V$ can be equivalently described as a comodule over $k[\Gamma]$ with respect to the following coaction
$$V_{\gamma}\ni v\mapsto \gamma\otimes v\in k[\Gamma]\otimes_kV_{\gamma}$$
\end{fact}
\begin{proof}
Denote by $\rho:\fP_{\bd{D}_{\Gamma}}\ra \cL_{V}$ the morphism of monoid $k$-functors that makes a $V$ into a representation of $\bd{D}_{\Gamma}$. Then $\rho\left(1_{\bd{D}_{\Gamma}}\right)$ is a morphism of $k[\Gamma]$-modules
$$k[\Gamma]\otimes_kV \ni 1\otimes v \mapsto \gamma \otimes v \in k[\Gamma]\otimes_k V$$
We obtain the coaction of $k[\Gamma]$ on $V$ corresponding to $\rho$ by transforming morphism $\rho\left(1_{\bd{D}_{\Gamma}}\right)$ via the canonical isomorphism
$$\Hom_{k[\Gamma]}\left(k[\Gamma]\otimes_kV,k[\Gamma]\otimes_kV\right)\cong \Hom_k\left(V,k[\Gamma]\otimes_kV\right)$$
Thus this coaction is given by formula
$$V \ni v\mapsto \gamma \otimes v\ni k[\Gamma]\otimes_kV$$
\end{proof}

\begin{fact}\label{fact:othogonality_of_weights_representations}
Let $\Gamma$ be a commutative monoid and let $\bd{D}_{\Gamma}$ be the corresponding monoid $k$-scheme. Suppose that $V_1,V_2$ are representations of $\bd{D}_{\Gamma}$ and assume that $V_1,V_2$ have weights $\gamma_1, \gamma_2$ with $\gamma_1\neq \gamma_2$. Then
$$\Hom_{\bd{D}_{\Gamma}}\left(V_1,V_2\right) = 0$$
\end{fact}
\begin{proof}
This follows from Fact \ref{fact:weight_representation_as_comodules}.
\end{proof}
\noindent
Let $\Gamma$ be a commutative monoid and let $\bd{D}_{\Gamma}$ be the corresponding monoid $k$-scheme. For every representation $V$ of $\bd{D}_{\Gamma}$ and fixed $\gamma$ in $\Gamma$  define
$$V[\gamma] = \big\{v\in V\,\big|\,d(v) = \gamma \otimes v\big\}$$
where $d:V\ra k[\Gamma]\otimes_kV$ is the coaction. Then $V[\gamma]$ is a subrepresentation of $V$. Note that according to Fact \ref{fact:weight_representation_as_comodules} $V[\gamma]$ is a subrepresentation of $V$ of weight $\gamma$.

\begin{proposition}\label{proposition:weight_decomposition}
Let $\Gamma$ be a commutative monoid and let $\bd{D}_{\Gamma}$ be the corresponding monoid $k$-scheme. For every representation $V$ of $\bd{D}_{\Gamma}$ we have a direct sum
$$V = \bigoplus_{\gamma \in \Gamma}V[\gamma]$$
\end{proposition}
\begin{proof}
Let $\Delta,\xi$ be the comultiplication and the counit of $k[\Gamma]$, respectively. Let $d:V\ra k[\Gamma]\otimes_k V$ be a coaction. Fix $v\in V$. Then we have a unique decomposition $d(v) = \sum_{\gamma \in \Gamma}\gamma\otimes v_{\gamma}$. Then
$$\sum_{\gamma\in \Gamma}\gamma \otimes \gamma \otimes v_{\gamma}  = \left(\Delta\otimes_k1_V\right)\left(d(v)\right) =  \left(1_{k[\Gamma]}\otimes_kd\right)\left(d(v)\right) = \sum_{\gamma \in \Gamma}\gamma \otimes d(v_{\gamma})$$
This implies that $d(v_{\gamma}) = \gamma \otimes v_{\gamma}$ and hence $v_{\gamma}\in V[\gamma]$. On the other hand we have
$$v =  \xi\left(d(v)\right) = \sum_{\gamma\in \Gamma}v_{\gamma}$$
Thus
$$v \in \sum_{\gamma\in \Gamma}V[\gamma]$$
Hence
$$V =  \sum_{\gamma\in \Gamma}V[\gamma]$$
Moreover, suppose that $\sum_{\gamma\in \Gamma}v_{\gamma} = \sum_{\gamma\in \Gamma}v'_{\gamma}$ for some $v_{\gamma},v'_{\gamma}\in V[\gamma]$. Then
$$\sum_{\gamma\in \Gamma}\gamma\otimes v_{\gamma} = d\left(\sum_{\gamma\in \Gamma}v_{\gamma}\right) = d\left(\sum_{\gamma\in \Gamma}v'_{\gamma}\right) =  \sum_{\gamma\in \Gamma}\gamma \otimes v'_{\gamma}$$
and hence $v_{\gamma} = v'_{\gamma}$ for each $\gamma \in \Gamma$. This proves the direct decomposition of $V$ as we claimed.
\end{proof}

\begin{corollary}\label{corollary:representations_of_diagonalisable_monoids}
Let $k$ be a field. Suppose that $\Gamma$ is a commutative monoid and let $\bd{D}_{\Gamma}$ be the corresponding monoid $k$-scheme. Then the category $\bd{Rep}\left(\bd{D}_{\Gamma}\right)$ is semisimple. Moreover, each irreducible representation of $\bd{D}_{\Gamma}$ is isomorphic to one-dimensional representation of weight $\gamma$ for a unique $\gamma \in \Gamma$.
\end{corollary}
\begin{proof}
This is a consequence of Fact \ref{fact:othogonality_of_weights_representations} and Proposition \ref{proposition:weight_decomposition}.
\end{proof}

\section{Diagonalisable group $k$-schemes}
\noindent
Let $\Gamma$ be an abstract commutative group. Then in addition to $k$-bialgebra structure the $k$-algebra $k[\Gamma]$ admits an antipode map
$$k[\Gamma]\ni \sum_{\gamma\in \Gamma}k_{\gamma}\cdot \gamma \mapsto \sum_{\gamma\in \Gamma}k_{\gamma}\cdot \gamma^{-1}\in k[\Gamma]$$
That makes $k[\Gamma]$ into a commutative Hopf $k$-algebra. Thus $\bd{D}_{\Gamma}$ is a group $k$-scheme in this case.\\
The forgetful functor $|-|:\Ab\ra \bd{CMon}$ sending commutative (abelian) group to its underlying commutative monoid admits left adjoint $(-)_{\Grp}:\bd{CMon}\ra \Ab$. Hence for every commutative monoid $\Gamma$ there exists a universal commutative group $\Gamma_{\Grp}$ generated by $\Gamma$. This is used in the following result.

\begin{proposition}\label{proposition:diagonalisable_monoid_k_schemes_its_group_of_units}
Let $\Gamma$ be a commutative monoid. Then the canonical morphism $\Gamma \ra \Gamma_{\Grp}$ induces a monomorphism of monoid $k$-schemes
$$\bd{D}_{\Gamma_{\Grp}} \hookrightarrow \bd{D}_{\Gamma}$$
that identifies $\bd{D}_{\Gamma_{\Grp}}$ with $\left( \bd{D}_{\Gamma} \right)^*$.
\end{proposition}
\begin{proof}
For every $k$-algebra we have an isomorphism of groups
$$\Mon\left(\Gamma,A^{\times}\right)^* \cong \Mon\left(\Gamma,A^*\right) \cong \Mon\left(\Gamma_{\Grp},A^*\right) \cong \Mon\left(\Gamma_{\Grp},A^{\times}\right)$$
natural in $A$. Note that this natural isomorphisms identifies $\fP_{\bd{D}_{\Gamma}}^*$ with $\fP_{\bd{D}_{\Gamma_{\Grp}}}$ by morphism induced by the unit $\Gamma\ra \Gamma_{\Grp}$ of the adjunction $|-|\vdash (-)_{\Grp}$.
\end{proof}

\begin{corollary}\label{corollary:diagonalisable_group_schemes}
Let $\bd{G}$ be a group $k$-scheme. Suppose that  $G$ is isomorphic to $\bd{D}_{\Gamma}$ as a monoid $k$-scheme for some commutative monoid $\Gamma$. Then $\Gamma$ is a group.
\end{corollary}
\begin{proof}
Suppose that $\bd{G}\cong \bd{D}_{\Gamma}$ as a monoid $k$-schemes. We derive that $\bd{D}_{\Gamma}$ is a group $k$-scheme. Hence $\bd{D}_{\Gamma_{\Grp}} \hookrightarrow \bd{D}_{\Gamma}$ is an isomorphism of monoid $k$-schemes. This implies that $\Gamma = \Gamma_{\Grp}$ and thus $\Gamma$ is an abstract group.
\end{proof}

\begin{definition}
Let $\bd{G}$ be a group $k$-scheme. We say that $\bd{G}$ is \textit{diagonalisable group $k$-scheme} if it is diagonalisable as a monoid scheme over $k$.
\end{definition}

\begin{example}\label{example:multiplicative_group}
Let $\mathbb{Z}$ be a commutative group of additive integers. We denote by $\mathbb{G}_{m}$ the monoid $k$-scheme $\bd{D}_{\ZZ}$. Note that $\mathbb{G}_{m}$ represents the group $k$-functor
$$\Alg_k\ni A\mapsto A^{*}\in \Ab$$
We call $\mathbb{G}_{m}$ \textit{the multiplicative group over $k$}.
\end{example}

\begin{definition}
Let $\fG$ be a monoid $k$-functor. Then the morphisms $\fG\ra \fP_{\mathbb{G}_m}$ of monoid $k$-functors are called \textit{characters of $\fG$}. They form a group $\cX(\fG)$ called \textit{the group of characters of $\fG$}.
\end{definition}

\begin{corollary}\label{corollary:characters_and_diagonalisable_groups}
Suppose that $k$ is commutative ring such that $\Spec k$ is connected (i.e. $k$ has no nontrivial idempotents). Functors 
\begin{center}
\begin{tikzpicture}
[description/.style={fill=white,inner sep=2pt}]
\matrix (m) [matrix of math nodes, row sep=1em, column sep=2em,text height=1.5ex, text depth=0.25ex] 
{ \Gamma_1  &  & \bd{D}_{\Gamma_1} & & \bd{G}_1  &  & \cX(\bd{G}_1)   \\
    {}      &  & {}                & & {}        &  & {}              \\
   \Gamma_2 &  & \bd{D}_{\Gamma_2} & &  \bd{G}_2 &  & \cX(\bd{G}_2)                 \\} ;
\path[->, line width=1.0pt, font=\scriptsize]  
(m-3-3) edge node[right] {$\bd{D}_{g} $} (m-1-3)
(m-1-1) edge node[left] {$g $} (m-3-1)
(m-3-7) edge node[right] {$\cX(f) $} (m-1-7)
(m-1-5) edge node[left] {$f $} (m-3-5);
\path[|->, shorten >= 0.4cm, shorten <= 0.4cm, line width=1.0pt, font=\scriptsize]  
(m-2-1) edge node[right] {$ $} (m-2-3)
(m-2-5) edge node[right] {$ $} (m-2-7);
\end{tikzpicture}
\end{center}
induce an equivalence between categories of abstract commutative groups and diagonalisable group schemes over $k$..
\end{corollary}
\begin{proof}
This is a consequence of Theorem \ref{theorem:commutative_monoids_and_diagonalisable_monoid_k_schemes}.
\end{proof}


\subsection{Results on linear representations}

\begin{proposition}\label{proposition:invariants_are_stable_under_tensoring_with_k_algebra}
Let $\bd{M}$ be an affine monoid $k$-scheme and let $V$ be a representation of $\bd{M}$. Then for every $k$-algebra $A$ the natural morphism of $A$-modules
$$V^{\bd{M}}\otimes_{k}A \ra \left(A\otimes_kV\right)^{\bd{M}_A}$$
is an isomorphism.
\end{proposition}
\begin{proof}
Note that we have a left exact sequence of $k$-vector spaces defining invariants
\begin{center}
\begin{tikzpicture}
[description/.style={fill=white,inner sep=2pt}]
\matrix (m) [matrix of math nodes, row sep=3em, column sep=3em,text height=1.5ex, text depth=0.25ex] 
{0 &  V^{\bd{M}} &   V& \Gamma(\bd{M},\cO_{\bd{M}}) \otimes_{k}V    \\} ;
\path[->,line width=1.0pt,font=\scriptsize]  
(m-1-1) edge node[above] {$ $} (m-1-2)
(m-1-2) edge node[above] {$ $} (m-1-3)
(m-1-3) edge node[above] {$\Delta-p $} (m-1-4);
\end{tikzpicture}
\end{center}
where $\Delta:V\ra \Gamma(\bd{M},\cO_{\bd{M}})\otimes_{k} V$ is the coaction and $p:V\ra \Gamma(\bd{M},\cO_{\bd{M}})\otimes_{k}V$ is the trivial coaction defined by formula $p(v)= 1\otimes v$ for every $v$ in $V$. Now tensoring the sequence with $k$-algebra $A$ yields a left exact sequence
\begin{center}
\begin{tikzpicture}
[description/.style={fill=white,inner sep=2pt}]
\matrix (m) [matrix of math nodes, row sep=3em, column sep=3em,text height=1.5ex, text depth=0.25ex] 
{0 &  V^{\bd{M}}\otimes_k A &   A \otimes_kV &\Gamma(\bd{M}_A,\cO_{\bd{M}_A})\otimes_{A}\left(A\otimes_kV\right)   \\} ;
\path[->,line width=1.0pt,font=\scriptsize]  
(m-1-1) edge node[above] {$ $} (m-1-2)
(m-1-2) edge node[above] {$ $} (m-1-3)
(m-1-3) edge node[above] {$\Delta_A-p_A $} (m-1-4);
\end{tikzpicture}
\end{center}
where $\Delta_A$ is the coaction on $A\otimes_kV$ induced by $\Delta$ and $p_A$ is the trivial coaction on $A\otimes_kV$. This shows that $V^{\bd{M}}\otimes_{k}A \ra \left(A\otimes_kV\right)^{\bd{M}_A}$ is an isomorphism.
\end{proof}

\begin{proposition}\label{proposition:base_change_for_hom_of_representations_of_affine_group_schemes}
Let $\bd{G}$ be an affine group $k$-scheme and let $V,W$ be representations of $\bd{G}$. If $V$ is finite dimensional, then for every $k$-algebra $A$ the canonical morphism
\begin{center}
\begin{tikzpicture}
[description/.style={fill=white,inner sep=2pt}]
\matrix (m) [matrix of math nodes, row sep=3em, column sep=3em,text height=1.5ex, text depth=0.25ex] 
{A\otimes_k\Hom_{\bd{G}}(V,W) & \Hom_{\bd{G}_A}\left(A\otimes_kV,A\otimes_kW\right)\\} ;
\path[->,line width=1.0pt,font=\scriptsize]  
(m-1-1) edge node[above] {$ $} (m-1-2);
\end{tikzpicture}
\end{center}
is an isomorphism of $A$-modules.
\end{proposition}
\begin{proof}
Fix a $k$-algebra $A$. Since $V$ is finite dimensional, for every $k$-algebra $B$ there exists an isomorphism $B\otimes_k\Hom_{k}(V,W) \ra \Hom_{B}\left(B\otimes_kV,B \otimes_kW\right)$ of $B$-modules natural in $B$. This implies that $\Hom_k(V,W)$ is a representation of $\bd{G}$ via the action given by formula
$$\left(g\cdot f\right)(v) = g\cdot f(g^{-1}\cdot v)$$
where $f\in \Hom_{B}\left(B\otimes_kV,B \otimes_kW\right)$, $v\in B\otimes_kV$ and $g\in \fP_{\bd{G}}(B)$. Similarly $\Hom_A(A\otimes_kV,A\otimes_kW)$ is a representation of $\bd{G}_K$ and the canonical isomorphism $A\otimes_k\Hom_{k}(V,W) \ra \Hom_{A}\left(A\otimes_kV,A \otimes_kW\right)$ of $A$-modules is $\bd{G}_A$-equivariant. Now we apply Proposition \ref{proposition:invariants_are_stable_under_tensoring_with_k_algebra} to derive a chain of isomorphisms
$$\Hom_{A}\left(A\otimes_kV,A\otimes_kW\right)^{\bd{G}_A} \cong \left(A\otimes_k\Hom_{k}(V,W)\right)^{\bd{G}_A} \cong A\otimes_k\Hom_k(V,W)^{\bd{G}}$$
of $A$-modules. Since we have identifications
$$\Hom_{\bd{G}_A}\left(A\otimes_kV,A\otimes_kW\right) \cong \Hom_{A}\left(A\otimes_kV,A\otimes_kW\right)^{\bd{G}_A} ,\,\Hom_{\bd{G}}\left(V,W\right)  \cong \Hom_k\left(V,W\right)^{\bd{G}}$$
we deduce the statement.
\end{proof}

\begin{proposition}\label{proposition:trivial_homs_stay_trivial}
Let $\bd{G}$ be an affine group scheme over $k$ and let $V, W$ be $\bd{G}$-representation such that $\Hom_{\bd{G}}(U,W) = 0$ for every finite dimensional $\bd{G}$-subrepresentation of $V$. Then for every $k$-algebra $A$ we have
$$\Hom_{\bd{G}_A}\left(A\otimes_kV,A\otimes_kW\right)=0$$
\end{proposition}
\begin{proof}
Let $\cF$ be a set of all finite dimensional $\bd{G}$-subrepresentations of $V$. Since $V$ is a $\bd{G}$-representation and $\bd{G}$ is an affine group $k$-scheme, we have
$$V= \mathrm{colim}_{U\in \cF}\,U$$
Fix $k$-algebra $A$ then we have identifications of $A$-modules
$$\Hom_{\bd{G}_A}\left(A\otimes_kV,A\otimes_kW\right) = \Hom_{\bd{G}_A}\left(A\otimes_k\mathrm{colim}_{U\in \cF}U,A\otimes_kW\right) =$$
$$ = \Hom_{\bd{G}_A}\left(\mathrm{colim}_{U\in \cF}A \otimes_k U,A \otimes_k W\right) = \lim_{U\in \cF} \Hom_{\bd{G}_A}\left(A \otimes_k U,A \otimes_k W\right) = $$
$$ = \lim_{U\in \cF} \big(A\otimes_k\Hom_{\bd{G}}\left(U,W\right)\big) = 0$$
where we apply Proposition \ref{proposition:base_change_for_hom_of_representations_of_affine_group_schemes}.
\end{proof}


\subsection{Linear algebraic monoids}

\begin{proposition}
Let $\bd{M}$ be a monoid $k$-scheme. Then the $k$-functor of units $\fP_{\bd{M}}^*$ of $\fP_{\bd{M}}$ is representable by a group $k$-scheme $\bd{M}^*$. Moreover, if $\bd{M}$ is affine and of finite type over $k$, then $\bd{M}^*$ is an open subscheme of $\bd{M}$.
\end{proposition}
\begin{proof}
Note that $\fP_{\bd{M}}^*$ fits into a cartesian square
\begin{center}
\begin{tikzpicture}
[description/.style={fill=white,inner sep=2pt}]
\matrix (m) [matrix of math nodes, row sep=3em, column sep=2em,text height=1.5ex, text depth=0.25ex] 
{\fP_{\bd{M}}^*                  &   \bd{1}      \\
 \fP_{\bd{M}}\times \fP_{\bd{M}} & \fP_{\bd{M}}  \\} ;
\path[->,line width=1.0pt,font=\scriptsize]  
(m-1-1) edge node[above] {$ $} (m-1-2)
(m-2-1) edge node[below] {$\fP_{m} $} (m-2-2)
(m-1-1) edge node[above] {$ $} (m-2-1)
(m-1-2) edge node[right] {$\fP_e $} (m-2-2);
\end{tikzpicture}
\end{center}
where $m:\bd{M}\times \bd{M}\ra \bd{M}$ is the multiplication and $e:\Spec k\ra \bd{M}$ is the unit. Since the functor
\begin{center}
\begin{tikzpicture}
[description/.style={fill=white,inner sep=2pt}]
\matrix (m) [matrix of math nodes, row sep=3em, column sep=3em,text height=1.5ex, text depth=0.25ex] 
{ \widehat{\Sch_k}  & \mbox{\emph{the category of $k$-functors}} \\};
\path[->,line width=1.0pt,font=\scriptsize]  
(m-1-1) edge node[auto] {$ $} (m-1-2);
\end{tikzpicture}
\end{center}
preserves fiber products, we derive that $\fP_{\bd{M}}^*$ is isomorphic to $\fP_{\bd{M}^*}$, where $\bd{M}^*$ is a $k$-scheme defined by the cartesian diagram
\begin{center}
\begin{tikzpicture}
[description/.style={fill=white,inner sep=2pt}]
\matrix (m) [matrix of math nodes, row sep=3em, column sep=2em,text height=1.5ex, text depth=0.25ex] 
{ \bd{M}^*             &      \Spec k      \\
  \bd{M}\times \bd{M} &       \bd{M}  \\} ;
\path[->,line width=1.0pt,font=\scriptsize]  
(m-1-1) edge node[above] {$ $} (m-1-2)
(m-2-1) edge node[below] {$ m $} (m-2-2)
(m-1-1) edge node[above] {$ $} (m-2-1)
(m-1-2) edge node[right] {$ e $} (m-2-2);
\end{tikzpicture}
\end{center}
Since $\fP_{\bd{M}^*}\cong \fP_{\bd{M}}^*$, we deduce that $\bd{M}^*$ admits a structure of a group $k$-scheme.\\
Now suppose that $\bd{M}$ is affine monoid $k$-scheme of finite type over $k$. Then there exist a finite dimensional vector space $V$ over $k$ and a closed immersion $i:\bd{M}\ra L(V)$ of monoid $k$-schemes.
\end{proof}

\begin{definition}
Let $\bd{M}$ be an affine monoid $k$-scheme. Suppose that the group $\bd{G}$ of units of $\bd{M}$ is an algebraic group over $k$ and that the open immersion $\bd{G}\hookrightarrow \bd{M}$ is schematically dense. Then $\bd{M}$ is \textit{a linear algebraic monoid over $k$}.
\end{definition}

\begin{definition}
Let $\bd{M}$ be a linear algebraic monoid over $k$. Suppose that the group $\bd{G}$ of units of $\bd{M}$ is (linearly) reductive. Then $\bd{M}$ is \textit{a (linearly) reductive monoid over $k$}.
\end{definition}







































\small
\bibliographystyle{alpha}
\bibliography{../zzz}

\end{document}