\input ../pree.tex

\begin{document}

\title{Bia{\l}ynicki-Birula functors}
\date{}
\maketitle

\section{Introduction}
\noindent
In this notes we study Bia{\l}ynicki-Birula functors. In the first section we prove some results concerning the forgetful functor $\bd{Rep}(\bd{M})\ra \bd{Rep}(\bd{G})$, where $\bd{M}$ is an affine monoid $k$-scheme and $\bd{G}$ is its group of units (we assume that $\bd{G}$ is open and schematically dense in $\bd{M}$). These results will be used in the following sections.\\
We assume that $k$ is a field. In these notes we use the following notational convention.

\begin{remark}\label{remark:notational_convention_identyfication_of_schemes_and_representable_preheaves}
Since the Yoneda embedding $\Sch_k\hookrightarrow \widehat{\Sch_k}$ is full and faithful, we identify $\Sch_k$ with the subcategory of $\widehat{\Sch_k}$ consisting of representable presheaves on $\Sch_k$. In particular, if $X$ is a $k$-scheme, then we denote by the same symbol the presheaf representable by $X$.
\end{remark}

\section{Tannakian formalism for quotient stacks}
\noindent
In this section we discuss an application of the main result of \cite{hall2019}. For this we need to briefly discuss \textit{algebraic stacks}, although for our purposes there is no need to use any technical details of this language. We refer the interested reader to the excellent exposition \cite{olsson2016algebraic} of this subject. We note the following facts.
\begin{enumerate}[label=\textbf{(\arabic*)}, leftmargin=3.0em]
\item \textit{An algebraic stack} is a category fibered over $\Sch_k$ satisfying certain extra conditions described in {\cite[Definition 4.6.1]{olsson2016algebraic}} and {\cite[Definition 8.1.4]{olsson2016algebraic}}. By  {\cite[Definition 8.2.1, Example 8.2.3]{olsson2016algebraic}} there are well defined notions of \textit{locally noetherian, noetherian and excellent algebraic stacks}. 
\item \textit{A morphism of algebraic stacks} is a morphism of fibered categories over $\Sch_k$. If $\cX$ and $\cY$ are algebraic stack, then we denote by $\Mor\left(\cX,\cY\right)$ the corresponding category of morphisms.
\item For every locally noetherian algebraic stack $\cX$ there exists an abelian monoidal category $\Coh(\cX)$ of coherent sheaves on $\cX$ ({\cite[Definition 9.1.14]{olsson2016algebraic}}). If $\cX$ and $\cY$ are locally noetherian algebraic stacks, then we denote by $\Hom_{r,\otimes,\cong}\left(\Coh(\cX),\Coh(\cY)\right)$ the category of right exact, monoidal functors $\Coh(\cX)\ra \Coh(\cY)$ with natural isomorphism as morphisms.
\item If $f:\cX\ra \cY$ is a morphism of locally noetherian algebraic stacks, then $f$ induces the functor $f^*:\Coh(\cY)\ra \Coh(\cX)$ such that $f^*\in \Hom_{r,\otimes,\cong}\left(\Coh(\cX),\Coh(\cY)\right)$.
\item Let $\bd{G}$ be a locally algebraic group $k$-scheme and let $X$ be a $k$-scheme equipped with an action of $\bd{G}$. We consider $\Sch_k$ as a Grothendieck site with respect to \textit{fppf topology} ({\cite[Example 2.1.14]{olsson2016algebraic}}). Next the quotient fibered category $[X/\bd{G}]$ ({\cite[Definition 9.5]{Fibered_categories_and_equivariant_objects}}) with respect to this topology is an algebraic stack by {\cite[Example 8.1.12]{olsson2016algebraic}}.
\item In \textbf{(5)} if $k$-scheme $X$ is locally noetherian (noetherian, excellent), then $[X/\bd{G}]$ is a locally noetherian (noetherian, excellent) by {\cite[Definition 8.2.1, Example 8.2.3]{olsson2016algebraic}} and {\cite[Example 8.1.12]{olsson2016algebraic}}.
\item In \textbf{(5)} if $k$-scheme $X$ is locally noetherian, then there exists an equivalence of monoidal categories $\Coh\left([X/\bd{G}]\right) \cong \Coh_{\bd{G}}(X)$ ({\cite[Exercise 9.H]{olsson2016algebraic}}). Moreover, this equivalence is functorial with respect to $\bd{G}$-equivariant morphism. That is if $Y$ is another locally noetherian $k$-scheme with action of $\bd{G}$ and $f:X\ra Y$ is a $\bd{G}$-equivariant morphism, then $f$ induces a morphism $[f/\bd{G}]:[X/\bd{G}]\ra [Y\bd{G}]$ by {\cite[Theorem 9.7]{Fibered_categories_and_equivariant_objects}} and the square
\begin{center}
\begin{tikzpicture}
[description/.style={fill=white,inner sep=2pt}]
\matrix (m) [matrix of math nodes, row sep=3em, column sep=3em,text height=1.5ex, text depth=0.25ex] 
{ \Coh([Y/\bd{G}]) &  \Coh([X/\bd{G}])    \\
  \Coh_{\bd{G}}(Y) &  \Coh_{\bd{G}}(X)           \\} ;
\path[->,line width=1.0pt,font=\scriptsize]
(m-1-1) edge node[above] {$ [f/\bd{G}]^*  $} (m-1-2)
(m-2-1) edge node[below] {$ f^*  $} (m-2-2)
(m-1-1) edge node[left] {$ \cong $} (m-2-1)
(m-1-2) edge node[right] {$ \cong  $} (m-2-2);
\end{tikzpicture}
\end{center}
of categories and functors is commutative.
\item If $\bd{G}$ is smooth and affine over $k$, then $[X/\bd{G}]$ has \textit{affine stabilizers}.
\end{enumerate}

\begin{remark}\label{remark:coherent_sheaves_on_principal_bundles_are_finite_dim_representations}
Let $\Spec k$ be a point equipped with the trivial action of a smooth and affine group $\bd{G}$. Then \textbf{(7)} together with {\cite[Example 4.7]{Algebraization}} impy that $\Coh([\Spec k/\bd{G}])$ can be identified with the category $\bd{Repf}_{\bd{G}}$ of finite dimensional representations of $\bd{G}$.
\end{remark}
\noindent
Let us state the main result of \cite{hall2019}.

\begin{theorem}[{\cite[Theorem 1.1]{hall2019}}]\label{theorem:hall_rydh_main_theorem}
Let $\cX$ be a noetherian algebraic stack with affine stabilizers. For every locally excellent algebraic stack $\cT$ the functor
\begin{center}
\begin{tikzpicture}
[description/.style={fill=white,inner sep=2pt}]
\matrix (m) [matrix of math nodes, row sep=5em, column sep=3em,text height=1.5ex, text depth=0.25ex] 
{ \Mor\left(\cX,\cT\right) &  \Hom_{r,\otimes,\cong}\left(\Coh(\cT),\Coh(\cX)\right)     \\} ;
\path[->,line width=1.0pt,font=\scriptsize]
(m-1-1) edge node[above] {$f \mapsto f^* $} (m-1-2);
\end{tikzpicture}
\end{center}
is an equivalence of categories.
\end{theorem}
\noindent
Keeping our previous remarks in mind we deduce the following result.

\begin{corollary}\label{corollary:tannakian_formalism_after_Rydh}
Let $\bd{G}$ be an smooth affine group $k$-scheme and let $X,Z$ be $k$-schemes equipped with an action of $\bd{G}$. Suppose that $Z$ is noetherian and $X$ is locally of finite type over $k$. Then
\begin{center}
\begin{tikzpicture}
[description/.style={fill=white,inner sep=2pt}]
\matrix (m) [matrix of math nodes, row sep=5em, column sep=3em,text height=1.5ex, text depth=0.25ex] 
{ \Mor\left([Z/\bd{G}],[X/\bd{G}]\right) &  \Hom_{r,\otimes,\cong}\bigg(\Coh\big([X/\bd{G}]\big),\Coh\big([Z/\bd{G}]\big)\bigg)     \\} ;
\path[->,line width=1.0pt,font=\scriptsize]
(m-1-1) edge node[above] {$f \mapsto f^* $} (m-1-2);
\end{tikzpicture}
\end{center}
is an equivalence of categories.
\end{corollary}
\begin{proof}
Note that $[Z/\bd{G}]$ is a noetherian algebraic stack according to \textbf{(5)} and \textbf{(6)}. It has affine stabilizers according to \textbf{(8)}. Similary by \textbf{(5)} $[X/\bd{G}]$ is an algebraic stack. Moreover, it is locally excellent according to the fact that $X$ is locally excellent (it is locally of finite type over $k$ and $k$ is a field) and \textbf{(6)}. Then by Theorem \ref{theorem:hall_rydh_main_theorem} we derive that the functor in the statement is an equivalence of categories. 
\end{proof}

\begin{corollary}\label{corollary:equivariant_morphisms_from_categories_Rydh_version}
Let $\bd{G}$ be an smooth affine group $k$-scheme and let $X,Z$ be $k$-schemes equipped with an action of $\bd{G}$. Suppose that $Z$ is noetherian and $X$ is locally of finite type over $k$. Then we have a bijection
\begin{center}
\begin{tikzpicture}
[description/.style={fill=white,inner sep=2pt}]
\matrix (m) [matrix of math nodes, row sep=5em, column sep=3em,text height=1.5ex, text depth=0.25ex] 
{ \big\{f:Z\ra X\,\big|\,f\mbox{ is }\bd{G}\mbox{-equivariant}\big\}  & \big\{F\in \Hom_{r,\otimes,\cong}\left(\Coh_{\bd{G}}(X),\Coh_{\bd{G}}(Z)\right)\,\big|\,F\cdot p_X^* = p_Z^*\big\}   \\} ;
\path[->,line width=1.0pt,font=\scriptsize]
(m-1-1) edge node[above] {$f\mapsto f^* $} (m-1-2);
\end{tikzpicture}
\end{center}
where $p_X^*:\bd{Repf}(\bd{G})\ra \Coh_{\bd{G}}(X)$ and $p_Z^*:\bd{Repf}(\bd{G})\ra \Coh_{\bd{G}}(Z)$ are functors induced by $\bd{G}$-equivariant morphisms $p_X:X\ra \Spec k$ and $p_Z:Z\ra \Spec k$, respectively.
\end{corollary}
\begin{proof}
Since fppf topology is subcanonical, {\cite[Theorem 9.7]{Fibered_categories_and_equivariant_objects}} shows that there exists a bijection
\begin{center}
\begin{tikzpicture}
[description/.style={fill=white,inner sep=2pt}]
\matrix (m) [matrix of math nodes, row sep=5em, column sep=4em,text height=1.5ex, text depth=0.25ex] 
{ \big\{f:Z\ra X\,\big|\,f\mbox{ is }\bd{G}\mbox{-equivariant}\big\} &  \big\{h:[Z/\bd{G}]\ra [X/\bd{G}]\,\big|\,[p_X/\bd{G}] \cdot h = [p_Y/\bd{G}]\big\}    \\} ;
\path[->,line width=1.0pt,font=\scriptsize]
(m-1-1) edge node[above] {$f \mapsto [f/\bd{G}] $} (m-1-2);
\end{tikzpicture}
\end{center}
Corollary \ref{corollary:tannakian_formalism_after_Rydh} implies that there exists a bijection
\begin{center}
\hspace*{-2.0cm}
\begin{tikzpicture}
[description/.style={fill=white,inner sep=2pt}]
\matrix (m) [matrix of math nodes, row sep=5em, column sep=3em,text height=1.5ex, text depth=0.25ex] 
{ \big\{h:[Z/\bd{G}]\ra [X/\bd{G}]\,\big|\,[p_X/\bd{G}] \cdot h = [p_Y/\bd{G}]\big\} & \big\{F\in \Hom_{r,\otimes,\cong}\left(\Coh([X/\bd{G}]),\Coh([Z/\bd{G}])\right)\,\big|\,F\cdot [p_X/\bd{G}]^* = [p_Z,\bd{G}]^*\big\}     \\} ;
\path[->,line width=1.0pt,font=\scriptsize]
(m-1-1) edge node[above] {$h \mapsto h^* $} (m-1-2);
\end{tikzpicture}
\end{center}
Next \textbf{(7)} implies that there exists a bijection
\begin{center}
\hspace*{-2.0cm}
\begin{tikzpicture}
[description/.style={fill=white,inner sep=2pt}]
\matrix (m) [matrix of math nodes, row sep=5em, column sep=3em,text height=1.5ex, text depth=0.25ex] 
{ \big\{F\in \Hom_{r,\otimes,\cong}\left(\Coh([X/\bd{G}]),\Coh([Z/\bd{G}])\right)\,\big|\,F\cdot [p_X/\bd{G}]^* = [p_Z,\bd{G}]^*\big\} & \big\{F\in \Hom_{r,\otimes,\cong}\left(\Coh_{\bd{G}}(X),\Coh_{\bd{G}}(Z)\right)\,\big|\,F\cdot p_X^* = p_Z^*\big\}    \\} ;
\path[->,line width=1.0pt,font=\scriptsize]
(m-1-1) edge node[above] {$ $} (m-1-2);
\end{tikzpicture}
\end{center}
and for a $\bd{G}$-equivariant morphism $f:Z\ra X$ the image of $[f/\bd{G}]^*:\Coh([X/\bd{G}])\ra \Coh([Z/\bd{G}])$ under this bijection is $f^*:\Coh_{\bd{G}}(X)\ra \Coh_{\bd{G}}(Z)$. These imply that the map of classes
\begin{center}
\begin{tikzpicture}
[description/.style={fill=white,inner sep=2pt}]
\matrix (m) [matrix of math nodes, row sep=5em, column sep=3em,text height=1.5ex, text depth=0.25ex] 
{ \big\{f:Z\ra X\,\big|\,f\mbox{ is }\bd{G}\mbox{-equivariant}\big\}  & \big\{F\in \Hom_{r,\otimes,\cong}\left(\Coh_{\bd{G}}(X),\Coh_{\bd{G}}(Z)\right)\,\big|\,F\cdot p_X^* = p_Z^*\big\}   \\} ;
\path[->,line width=1.0pt,font=\scriptsize]
(m-1-1) edge node[above] {$f\mapsto f^* $} (m-1-2);
\end{tikzpicture}
\end{center}
is a bijection.
\end{proof}
\noindent
Note that Corollary \ref{corollary:equivariant_morphisms_from_categories_Rydh_version} relies on some asumptions regarding $\bd{G}$, $X$ and $Z$. It is worth noting that Joachim Jelisiejew and the author were able to obtain a slightly more general (yet unpublished) result.

\begin{theorem}[{\cite[Theorem A.2]{jelisiejew2020bialynicki}}]\label{theorem:equivariant_morphisms_from_categories_Jelisiejew_et_al_version}
Let $\bd{G}$ be an affine algebraic group over $k$. Let $Z,X$ be $k$-schemes equipped with an action of $\bd{G}$ and assume that $X$ is quasi-compact and quasi-separated. Suppose that $F:\Qcoh_{\bd{G}}(X)\ra \Qcoh_{\bd{G}}(Z)$ is a cocontinuous, monoidal functor such that $F\cdot p_X^* = p_Z^*$. Then there exists a unique $\bd{G}$-equivariant morphism $f:Z\ra X$ such that $f^* = F$.
\end{theorem}

\section{Relations between representations of a monoid and its group of units}
\noindent
In this section we study the relation between the category $\bd{Rep}(\bd{M})$ of representations of an affine monoid $k$-scheme $\bd{M}$ and the category $\bd{Rep}(\bd{G})$ of representations of its group of units $\bd{G}$. Let $i:k[\bd{M}]\ra k[\bd{G}]$ be the morphism of $k$-bialgebras induced by $\bd{G}\hookrightarrow \bd{M}$. Let us first note the following elementary result.

\begin{fact}\label{fact:monoid_bialgebra_injects_into_group_bialgebra_if_units_are_open_and_schematically_dense}
Assume that $\bd{G}$ is open and schematically dense in $\bd{M}$. Then $i$ is an injective morphism of $k$-algebras.
\end{fact}
\begin{proof}
This follows from {\cite[Proposition 9.19]{gortz2010algebraic}}.
\end{proof}

\begin{fact}\label{fact:forgetful_from_reps_of_monoid_to_reps_of_units_creates_colimits_and_finite_limits}
The forgetful functor $\bd{Rep}(\bd{M})\ra \bd{Rep}(\bd{G})$ creates colimits and finite limits.
\end{fact}
\begin{proof}
This follows from {\cite[Theorem 14.3, Theorem 14.4]{Monoid_k_functors}} and the commutative triangle
\begin{center}
\begin{tikzpicture}
[description/.style={fill=white,inner sep=2pt}]
\matrix (m) [matrix of math nodes, row sep=2em, column sep=1em,text height=1.5ex, text depth=0.25ex] 
{  \bd{Rep}(\bd{M}) &        & \bd{Rep}(\bd{G})  \\
          &\Vect_k &  \\} ;
\path[->,line width=1.0pt,font=\scriptsize]
(m-1-1) edge node[above] {$  $} (m-1-3)
(m-1-1) edge node[swap] {$  $} (m-2-2)
(m-1-3) edge node[below = 6pt, right = 1pt] {$ $} (m-2-2);
\end{tikzpicture}
\end{center}
of functors.
\end{proof}
\noindent
The theorem below characterizes representations of $\bd{G}$ which are contained in the image of the forgetful functor $\bd{Rep}(\bd{M})\ra \bd{Rep}(\bd{G})$. 

\begin{theorem}\label{theorem:characterization_of_monoid_representations}
Assume that $\bd{G}$ is open and schematically dense in $\bd{M}$. Let $V$ be a $\bd{G}$-representation. Then the following are equivalent.
\begin{enumerate}[label=\emph{\textbf{(\roman*)}}, leftmargin=3.0em]
\item $V$ is in the image of the forgetful functor $\bd{Rep}(\bd{M}) \ra \bd{Rep}(\bd{G})$.
\item The coaction $d:V \ra k[\bd{G}]\otimes_kV$ factors through $i\otimes_k1_V:k[\bd{M}]\otimes_kV\hookrightarrow k[\bd{G}]\otimes_kV$.
\end{enumerate}
\end{theorem}
\begin{proof}
In the proof we denote by $\Delta_{\bd{M}}$ and $\Delta_{\bd{G}}$ comultiplications of $k[\bd{M}]$ and $k[\bd{G}]$, respectively. We also denote by $\xi_{\bd{M}}$ and $\xi_{\bd{G}}$ counits of $k[\bd{M}]$ and $k[\bd{G}]$, respectively. According to Fact \ref{fact:monoid_bialgebra_injects_into_group_bialgebra_if_units_are_open_and_schematically_dense} $i$ is an injective morphism of $k$-algebras.\\
Clearly $\textbf{(i)}\Rightarrow \textbf{(ii)}$. We prove the converse. Suppose that \textbf{(ii)} holds. Let $c:V\ra k[\bd{M}]\otimes_kV$ be a unique morphism such that $d = (i\otimes_k1_V) \cdot c$. It suffices to prove that $c$ is the coaction of the bialgebra $k[\bd{M}]$ on $V$. Observe that
$$\left(i\otimes_k i \otimes_k 1_V\right)\cdot \left(1_{k[\bd{M}]} \otimes_k c\right)\cdot c = \left(i\otimes_k d \right)\cdot c = \left(1_{k[\bd{G}]}\otimes_k d\right)\cdot d = \left(\Delta_{\bd{G}}\otimes_k1_V\right) \cdot d =$$
$$=\left(\Delta_{\bd{G}}\otimes_k1_V\right) \cdot \big(\left(i\otimes_k 1_V\right) \cdot c\big) = \left((\Delta_{\bd{G}}\cdot i )\otimes_k1_V\right) \cdot c = \left(i \otimes_k i \otimes_k 1_V\right) \cdot \left(\Delta_{\bd{M}}\otimes_k 1_V\right)\cdot c$$
Since $i\otimes_k i \otimes_k 1_V$ is a monomorphism, we deduce that $\left(1_{k[\bd{M}]} \otimes_k c\right)\cdot c =  \left(\Delta_{\bd{M}}\otimes_k 1_V\right)\cdot c$. Moreover, we have
$$(\xi_{\bd{G}}\otimes_k 1_V)\cdot d = (\xi_{\bd{G}}\otimes_k 1_V)\cdot \left((i \otimes_k 1_V) \cdot c\right) = (\xi_{\bd{M}}\otimes_k1_V)\cdot c$$
and hence $(\xi_{\bd{M}}\otimes_k1_V)\cdot c$ is the canonical isomorphism $V\cong k\otimes_kV$. Thus $c$ is the coaction of $k[\bd{M}]$ and $d = (i\otimes_k1_V) \cdot c$. Therefore, $V$ is in the image of $\bd{Rep}(\bd{M})\ra \bd{Rep}(\bd{G})$.
\end{proof}

\begin{theorem}\label{theorem:full_subcategory_closed_under_subobjects_and_quotients}
Assume that $\bd{G}$ is open and schematically dense in $\bd{M}$. Then $\bd{Rep}(\bd{M})$ is a full subcategory of $\bd{Rep}(\bd{G})$ closed under subobjects and quotients.
\end{theorem}
\begin{proof}
In the proof we denote by $\Delta_{\bd{M}}$ and $\Delta_{\bd{G}}$ comultiplications of $k[\bd{M}]$ and $k[\bd{G}]$, respectively. We also denote by $\xi_{\bd{M}}$ and $\xi_{\bd{G}}$ counits of $k[\bd{M}]$ and $k[\bd{G}]$, respectively. According to Fact \ref{fact:monoid_bialgebra_injects_into_group_bialgebra_if_units_are_open_and_schematically_dense} $i$ is an injective morphism of $k$-algebras.\\
We first prove that $\bd{Rep}(\bd{M})$ is a full subcategory of $\bd{Rep}(\bd{G})$. For this consider $\bd{M}$-representations $V,W$ and a their morphism $f:V\ra W$ as $\bd{G}$-representations. Let $c_V$ and $c_W$ be coactions of $k[\bd{M}]$ on $V$ and $W$, respectively. Our goal is to prove that $f$ is a morphism of $\bd{M}$-representations. Consider the diagram
\begin{center}
\begin{tikzpicture}
[description/.style={fill=white,inner sep=2pt}]
\matrix (m) [matrix of math nodes, row sep=3em, column sep=4em,text height=1.5ex, text depth=0.25ex] 
{ k[\bd{G}]\otimes_kV    & k[\bd{G}]\otimes_kW     \\
  k[\bd{M}]\otimes_kV    & k[\bd{M}]\otimes_kW            \\
  V                      & W                       \\} ;
\path[->,line width=1.0pt,font=\scriptsize]
(m-1-1) edge node[above] {$ 1_{k[\bd{G}]} \otimes_k f  $} (m-1-2)
(m-3-1) edge node[below] {$ f  $} (m-3-2)

(m-3-1) edge node[left] {$ c_V  $} (m-2-1)
(m-2-1) edge node[left] {$ i\otimes_k1_V  $} (m-1-1)

(m-3-2) edge node[right] {$ c_W  $} (m-2-2)
(m-2-2) edge node[right] {$ i\otimes_k1_W  $} (m-1-2);
\path[densely dotted,->,line width=1.0pt,font=\scriptsize]
(m-2-1) edge node[above] {$ 1_{k[\bd{M}]}\otimes_k f $} (m-2-2);
\end{tikzpicture}
\end{center}
in which the outer square is commutative. Our goal is to prove that the bottom square is commutative. We have
$$\left(i\otimes_k1_W\right) \cdot c_W\cdot f = \left(1_{k[\bd{G}]} \otimes_k f\right)\cdot \left(i\otimes_k1_V\right) \cdot c_V = \left(i\otimes_k1_W\right) \cdot \left(1_{k[\bd{M}]} \otimes_k f\right) \cdot  c_V$$
Since $i\otimes_k1_W$ is a monomorphism, we deduce that $c_W\cdot f = \left(1_{k[\bd{M}]} \otimes_k f\right) \cdot c_V$. Hence $f$ is a morphism of $\bd{M}$-representations.\\
Next we prove that $\bd{Rep}(\bd{M})$ is a subcategory of $\bd{Rep}(\bd{G})$ that is closed under subquotients. Consider an $\bd{M}$-representation $V$ and its quotient $\bd{G}$-representations $q:V\twoheadrightarrow W$. We show that $W$ is a quotient $\bd{M}$-representation of $V$. Let $c_V$ be the coaction of $\bd{M}$ on $V$ and let $d_W$ be the coaction of $\bd{G}$ on $W$. We have a commutative diagram
\begin{center}
\begin{tikzpicture}
[description/.style={fill=white,inner sep=2pt}]
\matrix (m) [matrix of math nodes, row sep=3em, column sep=3em,text height=1.5ex, text depth=0.25ex] 
{ k[\bd{G}]\otimes_kV    & k[\bd{G}]\otimes_kW     \\
  k[\bd{M}]\otimes_kV    &             \\
  V                      & W                       \\} ;
\path[->>,line width=1.0pt,font=\scriptsize]
(m-1-1) edge node[above] {$ 1_{k[\bd{G}]} \otimes_k q  $} (m-1-2)
(m-3-1) edge node[below] {$ q  $} (m-3-2);
\path[->,line width=1.0pt,font=\scriptsize]
(m-3-1) edge node[left] {$ c_V  $} (m-2-1)
(m-2-1) edge node[left] {$ i\otimes_k1_V  $} (m-1-1)

(m-3-2) edge node[right] {$ d_W  $} (m-1-2);
\end{tikzpicture}
\end{center}
and hence $d_W(W)\subseteq k[\bd{M}]\otimes_kW$. Thus Theorem \ref{theorem:characterization_of_monoid_representations} implies that $W$ is a representation of $\bd{M}$ and $q$ is a morphism of $\bd{M}$-representations. This shows that $\bd{Rep}(\bd{M})$ is a subcategory of $\bd{Rep}(\bd{G})$ closed under quotients. Next let $j:U\hookrightarrow V$ be a $\bd{G}$-subrepresentation of a $\bd{M}$-representation $V$. By what we proved above the cokernel $q:V\twoheadrightarrow W$ of $j$ in $\bd{Rep}(\bd{G})$ is contained in $\bd{Rep}(\bd{M})$. Since both $\bd{Rep}(\bd{M})$ and $\bd{Rep}(\bd{G})$ are abelian and the forgetful functor $\bd{Rep}(\bd{M})\ra \bd{Rep}(\bd{G})$ is exact, we derive that the kernel of $q$ in $\bd{Rep}(\bd{M})$ coincides with its kernel in $\bd{Rep}(\bd{G})$. Thus $U$ is a $\bd{M}$-representation and $j:U\hookrightarrow V$ is a morphism of $\bd{M}$-representations. Hence $\bd{Rep}(\bd{M})$ is the category of $\bd{Rep}(\bd{G})$ closed under subobjects.
\end{proof}

\begin{theorem}\label{theorem:monoid_representations_are_reflective_subcategory_in_representations_of_units_group}
Assume that $\bd{G}$ is open and schematically dense in $\bd{M}$. Let $V$ be a $\bd{G}$-representation of $\bd{G}$. There exists an $\bd{M}$-representation $W$ and a surjective morpism $q:V\twoheadrightarrow W$ of $\bd{G}$-representations such that for every $\bd{M}$-representation $U$ and a morphism $f:V\ra U$ of $\bd{G}$-representations there exists a unique morphism $\tilde{f}:W\ra U$ of $\bd{M}$-representations making the triangle
\begin{center}
\begin{tikzpicture}
[description/.style={fill=white,inner sep=2pt}]
\matrix (m) [matrix of math nodes, row sep=3em, column sep=4em,text height=1.5ex, text depth=0.25ex] 
{ V    &  U    \\
  W    &             \\} ;
\path[->>,line width=1.0pt,font=\scriptsize]
(m-1-1) edge node[left] {$ q  $} (m-2-1);
\path[->,line width=1.0pt,font=\scriptsize]
(m-1-1) edge node[above] {$ f $} (m-1-2)
(m-2-1) edge node[below = 3pt, right = 1pt] {$ \tilde{f} $} (m-1-2);
\end{tikzpicture}
\end{center}
commutative.
\end{theorem}
\begin{proof}
Assume first that $V$ is finite dimensional. Let $\cK$ be a set of $\bd{G}$-subrepresentations of $V$ that consists of all $K \subseteq V$ such that $V/K$ carries a structure of $\bd{M}$-representation. Clearly $\cK = \emptyset$ because $\{0\}\in \cK$. Since $V$ is finite dimensional, there exists a finite subset $\{K_1,...,K_n\}\subseteq \cK$ such that
$$\bigcap_{i=1}^nK_i = \bigcap_{K\in \cK}K$$
Then a morphism
$$V/\left( \bigcap_{K \in \cK}K \right)\ni v \mapsto \big(v\,\mathrm{mod}\,K_i\big)_{1\leq i\leq n} \in \bigoplus_{i=1}^nV/K_i$$
is a monomorphism and hence by Theorem \ref{theorem:full_subcategory_closed_under_subobjects_and_quotients} the quotient $W = V/\left( \bigcap_{K \in \cK}K \right)$ is an $\bd{M}$-representation. Let $q:V \twoheadrightarrow W$ be the canonical epimorphism. Consider now a morphism $f:V\ra U$ of $\bd{G}$-representations, where $U$ is an $\bd{M}$-representation. Then $\Image(f)$ is a $\bd{G}$-subrepresentation of $U$ and by Theorem \ref{theorem:full_subcategory_closed_under_subobjects_and_quotients} we derive that $\Image(f)$ is an $\bd{M}$-representation. This implies that $\Ker(f)$ is in $\cK$. Hence $f$ factors through $q$. Thus there exists a unique morphism $\tilde{f}:W\ra U$ of $\bd{G}$-representations such that $\tilde{f}\cdot q = f$. This completes the proof in case when $V$ is finite dimensional.\\
Now consider the general $V$. Let $\cF$ be the set of all finite dimensional $\bd{G}$-representations of $V$. According to {\cite[Corollary 15.2]{Monoid_k_functors}} we deduce that $V = \mathrm{colim}_{F\in \cF}F$. By the case considered above we deduce that for every $F$ in $\cF$ there exists a universal morphism $q_F:F\ra W_F$ of $\bd{G}$-representations into an $\bd{M}$-representation $W_F$. Note that if $F_1\subseteq F_2$ are two elements of $\cF$, then
\begin{center}
\begin{tikzpicture}
[description/.style={fill=white,inner sep=2pt}]
\matrix (m) [matrix of math nodes, row sep=3em, column sep=4em,text height=1.5ex, text depth=0.25ex] 
{ F_1    &  W_{F_1}    \\
  F_2    &  W_{F_2}           \\} ;
\path[->>,line width=1.0pt,font=\scriptsize]
(m-1-1) edge node[above] {$ q_{F_1}  $} (m-1-2)
(m-2-1) edge node[below] {$ q_{F_2}  $} (m-2-2);
\path[right hook->,line width=1.0pt,font=\scriptsize]
(m-1-1) edge node[left] {$  $} (m-2-1);
\path[->,line width=1.0pt,font=\scriptsize]
(m-1-2) edge node[right] {$  $} (m-2-2);
\end{tikzpicture}
\end{center}
Thus $\{W_F\}_{F\in \cF}$ together with morphisms $W_{F_1}\ra W_{F_2}$ for $F_1\subseteq F_2$ in $\cF$ form a diagram parametrized by the poset $\cF$. The category $\bd{Rep}(\bd{M})$ has small colimits ({\cite[Corollary 14.5]{Monoid_k_functors}}) and we define $W = \mathrm{colim}_{F\in \cF}W_F$. This is also a colimit of this diagram in the category $\bd{Rep}(\bd{G})$ by Fact \ref{fact:forgetful_from_reps_of_monoid_to_reps_of_units_creates_colimits_and_finite_limits}. We also define $q = \mathrm{colim}_{F\in \cF}q_F:V = \mathrm{colim}_{F\in \cF}F \ra W$.  Since a colimit of a family of epimorphisms is an epimorphism, we derive that $q$ is an epimorphism of $\bd{G}$-representations. Suppose now that $f:V\ra U$ is a morphism of $\bd{G}$-representations and $U$ is an $\bd{M}$-representation. Then $f_{\mid F}$ uniquely factors through $q_F$ for every $F$ in $\cF$. Hence by universal property of colimits we derive that $f$ factors through $q$ in a unique way. This completes the proof.
\end{proof}

\section{Bia{\l}ynicki-Birula functors}
\noindent
In this section we fix an affine group $k$-scheme $\bd{G}$. Let $\bd{M}$ be an affine monoid $k$-scheme with zero $\bd{o}$ such that $\bd{G}$ is its group of units.

\begin{definition}
Let $X$ be a $k$-scheme equipped with an action of $\bd{G}$. For every $k$-scheme $Y$ (considered as $\bd{G}$-scheme with the trivial $\bd{G}$-action) we define
$$\cD_X(Y) = \big\{\gamma:\bd{M}\times_kY\ra X\,\big|\,\gamma\mbox{ is }\bd{G}\mbox{-equivariant}\big\}$$
This gives gives rise to a subfunctor $\cD_X$ of $\Mor_{k}\left(\bd{M}\times_k(-),X\right):\Sch_k^{\mathrm{op}}\ra \Set$. We call it \textit{the Bia{\l}ynicki-Birula functor of $X$}.
\end{definition}

\begin{fact}\label{fact:bb_is_zariski_sheaf}
Let $X$ be a scheme equipped with an action of $\bd{G}$. Then $\cD_X$ is a Zariski sheaf.
\end{fact}
\begin{proof}
This is a consequence of the fact that $\Mor_{k}\left(\bd{M}\times_k(-),X\right)$ is a Zariski sheaf and if we glue $\bd{G}$-equivariant morphisms, then the result is $\bd{G}$-equivariant. Indeed, this shows that $\cD_X$ is a Zariski subsheaf of $\Mor_{k}\left(\bd{M}\times_k(-),X\right)$.
\end{proof}

\begin{remark}\label{remark:bb_diagram}
Let $X$ be a $k$-scheme equipped with an action of $\bd{G}$. Then there are canonical morphism of functors
\begin{center}
\begin{tikzpicture}
[description/.style={fill=white,inner sep=2pt}]
\matrix (m) [matrix of math nodes, row sep=4em, column sep=4em,text height=1.5ex, text depth=0.25ex] 
{ \cD_X        &  X    \\
  X^{\bd{G}}   &             \\} ;
\path[->,line width=1.0pt,font=\scriptsize]
(m-1-1) edge node[above] {$ i_X  $} (m-1-2)
(m-1-1) edge node[right] {$ r_X $} (m-2-1);
\path[->,bend left,line width=1.0pt,font=\scriptsize]
(m-2-1) edge node[left] {$ s_X $} (m-1-1);
\end{tikzpicture}
\end{center}
which we define now. First let us explain that in the diagram $X$ stands for the presheaf representable by the $k$-scheme $X$ (Remark \ref{remark:notational_convention_identyfication_of_schemes_and_representable_preheaves}) and $X^{\bd{G}}$ denotes the functor of fixed points of $X$ ({\cite[Definition 7.1]{Group_schemes_over_field}}). Now fix $k$-scheme $Y$ and $\gamma\in \cD_X(Y)$, then we define
$$i_X(\gamma) = \gamma_{\mid \{e\}\times_kX} = \gamma \cdot \langle e, 1_X\rangle,\,r_X(\gamma) = \gamma_{\mid \{o\}\times_kX} = \gamma\cdot \langle o, 1_X\rangle$$
where $e:\Spec k\ra \bd{M}$ is the unit of $\bd{M}$ and $\bd{o}:\Spec k\ra \bd{M}$ is the zero. Next if $f:Y\ra X$ is a morphism in $X^{\bd{G}}(Y)$, then we define
$$s_X(f) = f\cdot pr_Y$$
where $pr_Y:\bd{M}\times_kY\ra Y$ is the projection. Finally note that $r_X\cdot s_X = 1_{X^{\bd{G}}}$.
\end{remark}

\begin{remark}\label{remark:action_of_a_monoid_on_bb_functor}
Let $X$ be a $k$-scheme equipped with an action of $\bd{G}$. Then $\bd{M}$ (actually the presheaf of monoids represented by $\bd{M}$) acts on $\cD_X$. Indeed, fix $k$-scheme $Y$, $\gamma \in \cD_X(Y)$ and $m:Y\ra \bd{M}$. Then we define the product
$$m\gamma = \gamma\cdot \langle m, 1_Y\rangle$$
and this determines an action of $\bd{M}$ on $\cD_X$. Moreover, with respect to this action $i_X$ is $\bd{G}$-equivariant and $r_X,s_X$ are $\bd{M}$-equivariant ($X^{\bd{G}}$ is equipped with trivial action of $\bd{M}$).
\end{remark}

\begin{remark}\label{remark:functoriality_of_bb_functor}
Let $X,Y$ be $k$-schemes equipped with actions of $\bd{G}$ and let $f:X\ra Y$ be a $\bd{G}$-equivariant morphism, then there exists a morphism of functors $\cD_f:\cD_X\ra \cD_Y$ given by
$$\cD_f(\gamma) = f\cdot \gamma$$
for every element $\gamma$ of the functor $\cD_X$. Moreover, $\cD_f$ preserves the action of $\bd{M}$ described in Remark \ref{remark:action_of_a_monoid_on_bb_functor} above.
\end{remark}
\noindent
Let $X$ be a $k$-scheme equipped with an action of $\bd{G}$. It is useful to discuss subfunctors of $\cD_X$ defined by closed $\bd{G}$-stable subschemes of $X$.

\begin{theorem}\label{theorem:closed_subfunctors_induced_by_closed_stable_subschemes}
Let $X$ be a $k$-scheme equipped with an action of the group $\bd{G}$. Suppose that $\bd{G}$ is open and schematically dense in $\bd{M}$. If $j:Z\hookrightarrow X$ is a closed $\bd{G}$-stable subscheme of $X$, then the square
\begin{center}
\begin{tikzpicture}
[description/.style={fill=white,inner sep=2pt}]
\matrix (m) [matrix of math nodes, row sep=3em, column sep=3em,text height=1.5ex, text depth=0.25ex] 
{ \cD_Z &  \cD_X    \\
   Z    &   X           \\} ;
\path[->,line width=1.0pt,font=\scriptsize]
(m-1-1) edge node[above] {$ \cD_j  $} (m-1-2)
(m-2-1) edge node[below] {$ j  $} (m-2-2)
(m-1-1) edge node[left] {$ i_Z $} (m-2-1)
(m-1-2) edge node[right] {$ i_X  $} (m-2-2);
\end{tikzpicture}
\end{center}
is cartesian in the category of presheaves on $\Sch_k$.
\end{theorem}
\begin{proof}
The fact that the square is commutative follows by examination of definitions in Remarks \ref{remark:bb_diagram} and \ref{remark:functoriality_of_bb_functor}. Pick $k$-scheme $Y$, $f:Y\ra Z$ and $\gamma \in \cD_X(Y)$ such that $j\cdot f = i_X(\gamma)$. This is depicted in the diagram
\begin{center}
\begin{tikzpicture}
[description/.style={fill=white,inner sep=2pt}]
\matrix (m) [matrix of math nodes, row sep=2em, column sep=2em,text height=1.5ex, text depth=0.25ex] 
{         &   \gamma   \\
  f       &    j\cdot f = \gamma_{\mid \{e\}\times_kX}          \\} ;
\path[|->,line width=1.0pt,font=\scriptsize]
(m-2-1) edge node[below] {$ j  $} (m-2-2)
(m-1-2) edge node[right] {$ i_X  $} (m-2-2);
\end{tikzpicture}
\end{center}
Our goal is to show that there exists a unique $\bd{G}$-equivariant morphism $\eta:\bd{M}\times_kY\ra U$ such that $\cD_j(\eta) = \gamma$ and $i_Z(\eta) = f$. This is depicted by the diagram
\begin{center}
\begin{tikzpicture}
[description/.style={fill=white,inner sep=2pt}]
\matrix (m) [matrix of math nodes, row sep=2em, column sep=2em,text height=1.5ex, text depth=0.25ex] 
{ \eta                            &  \gamma = j\cdot \eta    \\
  f = \eta_{\mid \{e\}\times_k X} &              \\} ;
\path[|->,line width=1.0pt,font=\scriptsize]
(m-1-1) edge node[above] {$ \cD_j  $} (m-1-2)
(m-1-1) edge node[left] {$ r_U $} (m-2-1);
\end{tikzpicture}
\end{center}
It suffices to prove that $\gamma$ factors through $j$. First note that the assumption $\gamma_{\mid \{e\}\times_kY} = j\cdot f$ implies that
$$\gamma_{\mid \bd{G}\times_kY} = j\cdot f\cdot pr_{Y}$$
where $pr_Y:\bd{G}\times_kY\ra Y$ is the projection. This implies that $\gamma_{\mid \bd{G}\times_k}$ factors through $j$. Consider scheme-theoretic preimage $\gamma^{-1}(Z)$. Then $\gamma^{-1}(Z)$ is a closed $\bd{G}$-stable (as an inverse image of a $\bd{G}$-stable closed subscheme under the $\bd{G}$-equivariant morphism) subscheme of $\bd{M}\times_kY$, which contains $\bd{G} \times_k Y$. Since $\bd{G}$ is open, schematically dense in $\bd{M}$ and $k$ is a field, we derive that $\bd{G}\times_kY$ is open and schematically dense in $\bd{M}\times_kY$. Thus $\gamma^{-1}(Z) = \bd{M}\times_kY$ and hence $\gamma$ factors through $j$. 
\end{proof}
\noindent
In order to prove interesting result in the spirit of Theorem \ref{theorem:closed_subfunctors_induced_by_closed_stable_subschemes} which concerns open $\bd{G}$-stable subschemes, we need to assume that $\bd{M}$ is a Kempf monoid.

\begin{theorem}\label{theorem:open_subfunctors_induced_by_open_stable_subschemes}
Let $X$ be a $k$-scheme equipped with an action of the group $\bd{G}$ of units of a Kempf monoid $\bd{M}$. If $j:U\hookrightarrow X$ is an open $\bd{G}$-stable subscheme of $X$, then the square
\begin{center}
\begin{tikzpicture}
[description/.style={fill=white,inner sep=2pt}]
\matrix (m) [matrix of math nodes, row sep=3em, column sep=3em,text height=1.5ex, text depth=0.25ex] 
{ \cD_U         &  \cD_X    \\
  U^{\bd{G}}    &   X^{\bd{G}}           \\} ;
\path[->,line width=1.0pt,font=\scriptsize]
(m-1-1) edge node[above] {$ \cD_j  $} (m-1-2)
(m-2-1) edge node[below] {$ j^{\bd{G}}  $} (m-2-2)
(m-1-1) edge node[left] {$ r_U $} (m-2-1)
(m-1-2) edge node[right] {$ r_X  $} (m-2-2);
\end{tikzpicture}
\end{center}
is cartesian in the category of presheaves on $\Sch_k$.
\end{theorem}
\begin{proof}
The fact that the square is commutative follows by examination of definitions in Remarks \ref{remark:bb_diagram} and \ref{remark:functoriality_of_bb_functor}. Pick $k$-scheme $Y$, $f\in U^{\bd{G}}(Y)$ and $\gamma \in \cD_X(Y)$ such that $j^{\bd{G}}(f) = r_X(\gamma)$. This is depicted in the diagram
\begin{center}
\begin{tikzpicture}
[description/.style={fill=white,inner sep=2pt}]
\matrix (m) [matrix of math nodes, row sep=2em, column sep=2em,text height=1.5ex, text depth=0.25ex] 
{         &   \gamma   \\
  f       &    j\cdot f = \gamma_{\mid \{\bd{o}\}\times_kY}          \\} ;
\path[|->,line width=1.0pt,font=\scriptsize]
(m-2-1) edge node[below] {$ j^{\bd{G}}  $} (m-2-2)
(m-1-2) edge node[right] {$ r_X  $} (m-2-2);
\end{tikzpicture}
\end{center}
Our goal is to show that there exists a unique $\bd{G}$-equivariant morphism $\eta:\bd{M}\times_kY\ra U$ such that $\cD_j(\eta) = \gamma$ and $r_U(\eta) = f$. This is depicted by the diagram
\begin{center}
\begin{tikzpicture}
[description/.style={fill=white,inner sep=2pt}]
\matrix (m) [matrix of math nodes, row sep=2em, column sep=2em,text height=1.5ex, text depth=0.25ex] 
{ \eta                            &  \gamma = j\cdot \eta    \\
  f = \eta_{\mid \{\bd{o}\}\times_kY} &              \\} ;
\path[|->,line width=1.0pt,font=\scriptsize]
(m-1-1) edge node[above] {$ \cD_j  $} (m-1-2)
(m-1-1) edge node[left] {$ r_U $} (m-2-1);
\end{tikzpicture}
\end{center}
Fir this it suffices to prove that $\gamma$ factors through $j$. Consider $W = \gamma^{-1}(U)$. Note that $W$ is an open $\bd{G}$-stable (as an inverse image of a $\bd{G}$-stable open subscheme under the $\bd{G}$-equivariant morphism) subscheme of $\bd{M}\times_kY$, which contains $\{\bd{o}\} \times_k Y$. {\cite[Theorem 3.8]{Algebraic_monoids}} asserts that for every geometric point $\ol{y}$ of $Y$ we have $W_{\ol{y}} = \bd{M}_{k(\ol{y})}$, where $W_{\ol{y}}$ is the fiber over $\ol{y}$ of the projection $\bd{M}\times_kY\ra Y$ restricted to $W$. Since $W$ is an open subscheme of $\bd{M}\times_kY$, this implies that $W = \bd{M}\times_kY$ and hence $\gamma$ factors through $j$.
\end{proof}
\noindent
As we shall see below both Theorems are extremely useful properties of Bia{\l}ynicki-Birula functors.

\section{Formal Bia{\l}ynicki-Birula functors}
\noindent
We introduce a formal version of the Bia{\l}ynicki-Birula functor. We fix an affine group $k$-scheme $\bd{G}$. Let $\bd{M}$ be an affine monoid $k$-scheme with zero $\bd{o}$ such that $\bd{G}$ is its group of units.

\begin{definition}
Let $\bd{M}$ be an affine monoid $k$-scheme with zero $\bd{o}$ and let $\bd{G}$ be its group of units. For every $n\in \NN$ let $\bd{M}_n\hookrightarrow \bd{M}$ be an $n$-th infinitesimal neighborhood of $\bd{o}$ in $\bd{M}$. Let $X$ be a $k$-scheme equipped with an action of $\bd{G}$. For every $k$-scheme $Y$ (considered as $\bd{G}$-scheme with the trivial $\bd{G}$-action) we define
$$\widehat{\cD}_X(Y) = \big\{\{\gamma_n:\bd{M}_n\times_kY\ra X\}_{n\in \NN}\,\big|\,\forall_{n\in \NN}\,\gamma_n\mbox{ is }\bd{G}\mbox{-equivariant and }{\gamma_{n+1}}_{\mid \bd{M}_n\times_kY} = \gamma_n\big\}$$
This gives gives rise to a functor $\widehat{\cD}_X$. We call it \textit{the formal Bia{\l}ynicki-Birula functor of $X$}.
\end{definition}

\begin{remark}\label{remark:functoriality_of_formal_bb_functor}
Let $X,Y$ be $k$-schemes equipped with actions of $\bd{G}$ and let $f:X\ra Y$ be a $\bd{G}$-equivariant morphism, then there exists a morphism of functors $\widehat{\cD}_f:\widehat{\cD}_X\ra \widehat{\cD}_Y$ given by
$$\widehat{\cD}_f\big(\{\gamma_n\}_{n\in \NN}\big) = \{f\cdot \gamma_n\}_{n\in \NN}$$
for every element $\gamma$ of the functor $\widehat{\cD}_X$. 
\end{remark}

\begin{remark}\label{remark:comparison_between_formal_and_algebraic_bb_functors}
Let $\bd{M}$ be an affine monoid $k$-scheme with zero $\bd{o}$ and let $\bd{G}$ be its group of units. Let $X$ be a $k$-scheme equipped with an action of $\bd{G}$. Then there exists a canonical morphism of functors $\cD_X\ra \widehat{\cD}_X$ given by
$$\gamma \mapsto \{\gamma_{\mid \bd{M}_n\times_kY}\}_{n\in \NN}$$
for every $\gamma \in \cD_X(Y)$ and every $k$-scheme $Y$.
\end{remark}

\begin{remark}\label{remark:formal_bb_diagram}
Let $X$ be a $k$-scheme equipped with an action of $\bd{G}$. We define a morphism $\widehat{r}_X:\widehat{\cD}_X\ra X^{\bd{G}}$ by formula
$$\widehat{\cD}_X(Y)\ni \{\gamma_n\}_{n\in \NN} \mapsto \gamma_0 \in X^{\bd{G}}(Y)$$
for every $k$-scheme $Y$. This morphism fits into a commutative triangle
\begin{center}
\begin{tikzpicture}
[description/.style={fill=white,inner sep=2pt}]
\matrix (m) [matrix of math nodes, row sep=3em, column sep=2em,text height=1.5ex, text depth=0.25ex] 
{  \cD_X &            & \widehat{\cD}_X    \\
         & X^{\bd{G}} &                    \\} ;
\path[->,line width=1.0pt,font=\scriptsize]
(m-1-1) edge node[above] {$   $} (m-1-3)
(m-1-1) edge node[left = 4pt, below = -1pt] {$ r_X $} (m-2-2)
(m-1-3) edge node[right = 4pt, below = -1pt] {$ \widehat{r}_X $} (m-2-2);
\end{tikzpicture}
\end{center}
where horizontal morphism is described in Remark \ref{remark:comparison_between_formal_and_algebraic_bb_functors}. 
\end{remark}
\noindent
The next result is analogous to Theorem \ref{theorem:open_subfunctors_induced_by_open_stable_subschemes}, although its proof is much simpler.

\begin{proposition}\label{proposition:open_subfunctors_induced_by_open_stable_subschemes_for_formal_bb}
Let $X$ be a $k$-scheme equipped with an action of the group $\bd{G}$. If $j:U\hookrightarrow X$ is an open $\bd{G}$-stable subscheme of $X$, then the square
\begin{center}
\begin{tikzpicture}
[description/.style={fill=white,inner sep=2pt}]
\matrix (m) [matrix of math nodes, row sep=3em, column sep=3em,text height=1.5ex, text depth=0.25ex] 
{ \widehat{\cD}_U         &  \widehat{\cD}_X    \\
  U^{\bd{G}}    &   X^{\bd{G}}           \\} ;
\path[->,line width=1.0pt,font=\scriptsize]
(m-1-1) edge node[above] {$ \cD_j  $} (m-1-2)
(m-2-1) edge node[below] {$ j^{\bd{G}}  $} (m-2-2)
(m-1-1) edge node[left] {$ \widehat{r}_U $} (m-2-1)
(m-1-2) edge node[right] {$ \widehat{r}_X  $} (m-2-2);
\end{tikzpicture}
\end{center}
is cartesian in the category of presheaves on $\Sch_k$.
\end{proposition}
\begin{proof}
The fact that the square is commutative follows by examination of definitions in Remarks \ref{remark:functoriality_of_formal_bb_functor} and \ref{remark:functoriality_of_formal_bb_functor}. Pick $k$-scheme $Y$, $f\in U^{\bd{G}}(Y)$ and $\{\gamma_n\}_{n\in \NN} \in \widehat{\cD}_X(Y)$ such that $j^{\bd{G}}(f) = \widehat{r}_X\big(\{\gamma_n\}_{n\in \NN}\big))$. This is depicted in the diagram
\begin{center}
\begin{tikzpicture}
[description/.style={fill=white,inner sep=2pt}]
\matrix (m) [matrix of math nodes, row sep=2em, column sep=2em,text height=1.5ex, text depth=0.25ex] 
{         &   \{\gamma_n\}_{n\in \NN}   \\
  f       &    j\cdot f = \gamma_0          \\} ;
\path[|->,line width=1.0pt,font=\scriptsize]
(m-2-1) edge node[below] {$ j^{\bd{G}}  $} (m-2-2)
(m-1-2) edge node[right] {$ \widehat{r}_X  $} (m-2-2);
\end{tikzpicture}
\end{center}
Our goal is to show that there exists a unique family of $\bd{G}$-equivariant morphism $\eta_n:\bd{M}_n\times_kY\ra U$ for $n\in \NN$ such that $\widehat{\cD}_j\big(\{\eta_n\}_{n\in \NN}\big) = \{\gamma_n\}_{n\in \NN}$ and $\widehat{r}_U\big(\{\eta_n\}_{n\in \NN}) = f$. This is depicted by the diagram
\begin{center}
\begin{tikzpicture}
[description/.style={fill=white,inner sep=2pt}]
\matrix (m) [matrix of math nodes, row sep=2em, column sep=2em,text height=1.5ex, text depth=0.25ex] 
{ \{\eta_n\}_{n\in \NN} &  \{\gamma_n\}_{n\in \NN} = \{j\cdot \eta_n\}_{n\in \NN}    \\
  f = \eta_0            &              \\} ;
\path[|->,line width=1.0pt,font=\scriptsize]
(m-1-1) edge node[above] {$ \cD_j  $} (m-1-2)
(m-1-1) edge node[left] {$ r_U $} (m-2-1);
\end{tikzpicture}
\end{center}
Fir this it suffices to prove that $\gamma_n$ factors through $j$ for every $n\in \NN$. Note that all maps $\{\gamma_n\}_{n\in \NN}$ are equal set-theoretically and $\gamma_0 = j\cdot  f$ factors through $j$. Thus $\gamma_n$ factors through $j$ for every $n\in \NN$.
\end{proof}

\begin{theorem}\label{theorem:comparison_between_formal_and_algebraic_bb_functor_is_injective}
Let $\bd{G}$ be a group $k$-scheme and $\bd{M}$ be a Kempf monoid having $\bd{G}$ as a group of units. Suppose that $X$ is a $k$-scheme equipped with an action of $\bd{G}$. Then the canonical morphism $\cD_X\ra \widehat{\cD}_X$ is a monomorphism of functors.
\end{theorem}
\noindent
For the proof it is useful to make the following observation (essentially the same observation was made in the proof of Theorem \ref{theorem:closed_subfunctors_induced_by_closed_stable_subschemes}).

\begin{lemma}\label{lemma:extending_monoid_actions_on_closed_group_stable_subschemes}
Let $X$ be a $k$-scheme equipped with an action of a monoid $k$-scheme $\bd{M}$. Suppose that $j:Z\hookrightarrow X$ is closed $\bd{G}$-equivariant immersion, where $\bd{G}$ is a group of units of $\bd{M}$. If $\bd{G}$ is schematically dense in $\bd{M}$, then the action of $\bd{G}$ on $Z$ extends to the action of $\bd{M}$ in such a way that $j$ becomes $\bd{M}$-equivariant.
\end{lemma}
\begin{proof}[Proof of the lemma]
Let $a:\bd{M}\times_kX\ra X$ be the action of $\bd{M}$ on $X$. Since $j$ is $\bd{G}$-equivariant, we derive that $\bd{G}\times_kZ\subseteq a^{-1}(Z)$. Moreover, $\bd{G}\times_kZ$ is open and schematically dense in $\bd{M}\times_kZ$. Hence $\bd{M}\times_kZ\subseteq a^{-1}(Z)$ and thus $a_{\mid \bd{M}\times_kZ}$ factors through $j:Z\hookrightarrow X$.
\end{proof}

\begin{proof}[Proof of the theorem]
Let $Y$ be a $k$-scheme and let $\gamma,\eta:\bd{M}\times_kY\ra X$ be $\bd{G}$-equivariant morphisms. Suppose that $\gamma_{\mid \bd{M}_n\times_kY} = \eta_{\mid \bd{M}_n\times_kY}$ for every $n\in \NN$. Consider the kernel (equalizer) $j:E\hookrightarrow \bd{M}\times_kY$ of the pair $(\gamma,\eta)$. Then $E$ admits an action of $\bd{G}$ such that $i$ is $\bd{G}$-equivariant locally closed immersion and $\bd{M}_n\times_kY\subseteq E$ for every $n\in \NN$. Fix a point $y$ in $Y$. Let $\bd{M}_y$ and $E_y$ be fibers of the projection $\mathrm{pr}:\bd{M}\times_kY\ra Y$ and $\mathrm{pr}\cdot j$, respectively. Then $E_y\subseteq \bd{M}_y$ is a locally closed $\bd{G}_y$-equivariant subscheme, where $\bd{G}_y = \bd{G}\times_k \Spec k(y)$. Since $\bd{M}_y = \bd{M}\times_k\Spec k(y)$ is a Kempf monoid over $k(y)$ with group of units $\bd{G}_y$ and moreover, $E_y$ contains all infinitesimal neighborhoods of the zero in $\bd{M}_y$, we deduce by {\cite[Theorem 3.8]{Algebraic_monoids}} that $E_y = \bd{M}_y$. This implies that a locally closed immersion $j:E\hookrightarrow \bd{M}\times_kY$ is bijective. Hence it is a closed immersion. Now Lemma \ref{lemma:extending_monoid_actions_on_closed_group_stable_subschemes} implies that $E$ is a locally linear $\bd{M}$-scheme and $j$ is $\bd{M}$-equivariant. Note that $j$ induces an isomorphism $\widehat{E}\cong \reallywidehat{\bd{M}\times_kY}$ of formal $\bd{M}$-schemes. Hence according to {\cite[Corollary 7.4]{Algebraization}} we infer that $j$ is an isomorphism. This proves that $\gamma = \eta$. Therefore, the map
$$\cD_X(Y)\ra \widehat{\cD}_X(Y)$$
is injective. As $Y$ is arbitrary we infer that the canonical morphism $\cD_X\ra \widehat{\cD}_X$ of Remark \ref{remark:comparison_between_formal_and_algebraic_bb_functors} is a monomorphism of functors.
\end{proof}

\section{Representability of Bia{\l}ynicki-Birula functor for Kempf monoids}
\noindent
In this section we prove various results concerning representability of Bia{\l}ynicki-Birula functors.

\begin{theorem}\label{theorem:representability_of_bb_for_affine}
Let $\bd{M}$ be an affine monoid $k$-scheme with open and schematically dense group of units $\bd{G}$. Suppose that $X$ is an affine $k$-scheme equipped with an ation of $\bd{G}$. Then $\cD_X$ is representable and $i_X$ is a closed immersion of $k$-schemes.
\end{theorem}
\begin{proof}
Since $X$ is an affine $k$-scheme, the action of $\bd{G}$ on $X$ corresponds to the coaction of $k[\bd{G}]$ by $c:\Gamma(X,\cO_X)\ra k[\bd{G}]\otimes_k\Gamma(X,\cO_X)$. Note that $c$ is a morphism of $k$-algebras. By Theorem \ref{theorem:monoid_representations_are_reflective_subcategory_in_representations_of_units_group} there exists a universal morphism $q:\Gamma(X,\cO_X)\twoheadrightarrow W$ of $\bd{G}$-representations into a $\bd{M}$-representation. Let $I\subseteq \Gamma(X,\cO_X)$ be the ideal generated by $\Ker(q)$. Fix $f$ in $I$. Then
$$f = \sum_{i=1}^ng_i\cdot f_i$$
where $g_i \in k[\bd{G}]$ and $f_i\in \Ker(q)$ for $1\leq i\leq n$. Then
$$c(f) = c\bigg(\sum_{i=1}^n g_i\cdot f_i\bigg) = \sum_{i=1}^nc(g_i)\cdot c(f_i) \subseteq \bigg(k[\bd{G}]\otimes_k \Gamma(X,\cO_X)\bigg) \cdot \bigg(k[\bd{G}]\otimes_k \Ker(q) \bigg) \subseteq k[\bd{G}]\otimes_k I$$
Thus $c(I) \subseteq k[\bd{G}]\otimes_kI$ and hence $I$ is a $\bd{G}$-representation. Consider
\begin{center}
\begin{tikzpicture}
[description/.style={fill=white,inner sep=2pt}]
\matrix (m) [matrix of math nodes, row sep=4em, column sep=3em,text height=1.5ex, text depth=0.25ex] 
{  X^+ = V(I) = \Spec \Gamma(X,\cO_X)/I       &  X    \\} ;
\path[right hook->,line width=1.0pt,font=\scriptsize]
(m-1-1) edge node[above] {$ $} (m-1-2);
\end{tikzpicture}
\end{center}
Since $\Gamma(X,\cO_X)/I$ is the quotient $\bd{G}$-representation of $W$, we deduce by Theorem \ref{theorem:monoid_representations_are_reflective_subcategory_in_representations_of_units_group} that $\Gamma(X,\cO_X)/I$ is a $\bd{M}$-representation. Hence $X^+$ is a $k$-scheme equipped with action of $\bd{M}$ and $X^+\hookrightarrow X$ is $\bd{G}$-equivariant. Suppose now that $Y$ is an affine $k$-scheme. Then $\bd{M}\times_kY$ is a $\bd{M}$-scheme with respect to the left-hand side action of $\bd{M}$ and hence $\Gamma(\bd{M}\times_kY,\cO_{\bd{M}\times_kY})$ is a $\bd{M}$-representation. Now Theorem \ref{theorem:monoid_representations_are_reflective_subcategory_in_representations_of_units_group} implies that if $\gamma:\bd{M}\times_kY\ra X$ is a $\bd{G}$-equivariant morphism, then a morphism $\gamma^{\#}:\Gamma(X,\cO_X)\ra \Gamma(\bd{M}\times_kY,\cO_{\bd{M}}\times_kY)$ of $k$-algebras and $\bd{G}$-representations factors through $q:\Gamma(X,\cO_X)\twoheadrightarrow W$ and thus by construction of $I$ we have
\begin{center}
\begin{tikzpicture}
[description/.style={fill=white,inner sep=2pt}]
\matrix (m) [matrix of math nodes, row sep=4em, column sep=3em,text height=1.5ex, text depth=0.25ex] 
{  \Gamma(X,\cO_X)       &  \Gamma(X,\cO_X)/I &  \Gamma(\bd{M}\times_kY,\cO_{\bd{M}}\times_kY) \\} ;
\path[->>,line width=1.0pt,font=\scriptsize]
(m-1-1) edge node[above] {$    $} (m-1-2);
\path[->,line width=1.0pt,font=\scriptsize]
(m-1-2) edge node[above] {$ f  $} (m-1-3);
\path[->,bend left,line width=1.0pt,font=\scriptsize]
(m-1-1) edge node[above] {$ \gamma^{\#}  $} (m-1-3);
\end{tikzpicture}
\end{center}
for some morphism $f$ of $k$-algebras and $\bd{G}$-representations. Since both $\Gamma(X,\cO_X)/I$ and $\Gamma(\bd{M}\times_kY,\cO_{\bd{M}}\times_kY)$ are $\bd{M}$-representations and by Theorem \ref{theorem:full_subcategory_closed_under_subobjects_and_quotients} the subcategory $\bd{Rep}(\bd{M})\subseteq \bd{Rep}(\bd{G})$ is full, we derive that $f$ is a morphism of $\bd{M}$-representations. Thus $f$ corresponds to a unique $\bd{M}$-equivariant morphism $\eta:\bd{M}\times_kY\ra X^+$ such that the diagram
\begin{center}
\begin{tikzpicture}
[description/.style={fill=white,inner sep=2pt}]
\matrix (m) [matrix of math nodes, row sep=3em, column sep=3em,text height=1.5ex, text depth=0.25ex] 
{     &   X^+    \\
  \bd{M}\times_kY    &   X          \\} ;
\path[->,line width=1.0pt,font=\scriptsize]
(m-2-1) edge node[below] {$ \gamma  $} (m-2-2);
\path[->,bend left,line width=1.0pt,font=\scriptsize]
(m-2-1) edge node[auto] {$ \eta  $} (m-1-2);
\path[right hook->,line width=1.0pt,font=\scriptsize]
(m-1-2) edge node[above] {$  $} (m-2-2);
\end{tikzpicture}
\end{center}
is commutative. Now this result can be extended to an arbitrary $k$-scheme $Y$, since $\Mor_k(\bd{M}\times_k(-),X^+)$ is a Zariski sheaf and a morphism that is $\bd{M}$-equivariant locally on the domain is $\bd{M}$-equivariant. Thus for every $k$-scheme $Y$ we have a bijection
$$\cD_X(Y)\ni \gamma \mapsto \eta \in \big\{\bd{M}\mbox{-equivariant morphisms }\bd{M}\times_kY\ra X^+\big\}$$
Since we also have a bijection
$$\big\{\bd{M}\mbox{-equivariant morphisms }\bd{M}\times_kY\ra X^+\big\}\ni \eta \mapsto \eta\cdot \langle e, 1_{X^+}\rangle \in \Mor_k(Y,X^+)$$
and both this bijections are natural, we derive that $\cD_X$ is represented by $X^+$ and moreover, $i_X:\cD_X\ra X$ is a closed immersion $X^+\hookrightarrow X$.
\end{proof}

\begin{corollary}\label{corollary:bb_is_representable_on_schemes_with_affine_cover_of_fixed_points}
Let $\bd{G}$ be a group $k$-scheme and $\bd{M}$ be a Kempf monoid having $\bd{G}$ as a group of units. Suppose that $X$ is a $k$-scheme equipped with an action of $\bd{G}$ such that there exists a family $\cU$ of open affine $\bd{G}$-stable open subschemes of $X$ such that functors $\{U^{\bd{G}}\}_{U \in \cU}$ form an open cover of $X^{\bd{G}}$. Then $\cD_X$ is representable.
\end{corollary}
\begin{proof}
Note that $\bd{G}$ is affine group $k$-scheme as a unit group of an affine monoid $\bd{M}$ ({\cite[Proposition 12.4]{Monoid_k_functors}}). Moreover, $\bd{M}$ is a Kempf monoid and hence $\bd{G}$ is open and schematically dense in $\bd{M}$. By Theorem \ref{theorem:representability_of_bb_for_affine} each $\cD_U$ is representable by a $k$-scheme. On the other hand by Theorem \ref{theorem:open_subfunctors_induced_by_open_stable_subschemes} for each $U\in \cU$ we have a cartesian square
\begin{center}
\begin{tikzpicture}
[description/.style={fill=white,inner sep=2pt}]
\matrix (m) [matrix of math nodes, row sep=3em, column sep=3em,text height=1.5ex, text depth=0.25ex] 
{ \cD_U         &  \cD_X    \\
  U^{\bd{G}}    &   X^{\bd{G}}           \\} ;
\path[->,line width=1.0pt,font=\scriptsize]
(m-1-1) edge node[above] {$   $} (m-1-2)
(m-2-1) edge node[below] {$   $} (m-2-2)
(m-1-1) edge node[left] {$ r_U $} (m-2-1)
(m-1-2) edge node[right] {$ r_X  $} (m-2-2);
\end{tikzpicture}
\end{center}
of functors. This implies that $\{\cD_U\hookrightarrow \cD_X\}_{U\in \cU}$ is an open cover of $\cD_X$ as a pullback of an open cover $\{U^{\bd{G}}\hookrightarrow X^{\bd{G}}\}_{U\in \cU}$. Hence Fact \ref{fact:bb_is_zariski_sheaf} and {\cite[Theorem 8.9]{gortz2010algebraic}} (or if you like {\cite[Theorem 4.6]{kfunctors}}) imply that $\cD_X$ is representable. 
\end{proof}

\begin{corollary}\label{corollary:bb_is_representable_on_locally_linear_schemes}
Let $\bd{G}$ be group $k$-scheme and $\bd{M}$ be a Kempf monoid having $\bd{G}$ as a group of units. Suppose that $X$ is a locally linear $\bd{G}$-scheme. Then $\cD_X$ is representable.
\end{corollary}
\begin{proof}
This is a consequence of Corollary \ref{corollary:bb_is_representable_on_schemes_with_affine_cover_of_fixed_points}. Indeed, $X$ admits a cover $\cU$ by open $\bd{G}$-stable affine subschemes. Then $\{U^{\bd{G}}\}_{U\in \cU}$ is an open cover of $X^{\bd{G}}$.
\end{proof}
\noindent
Now we prove our main result.

\begin{theorem}\label{theorem:representability_of_bb_functors}
Let $\bd{G}$ be a group $k$-scheme and $\bd{M}$ be a Kempf monoid having $\bd{G}$ as a group of units. Suppose that $X$ is a $k$-scheme equipped with an action of $\bd{G}$. Then the following results hold.
\begin{enumerate}[label=\emph{\textbf{(\arabic*)}}, leftmargin=3.0em]
\item $\widehat{\cD}_X$ is representable. Moreover, the morphism $\widehat{r}_X:\widehat{\cD}_X\ra X^{\bd{G}}$ is affine and if $X$ is locally noetherian, then it is of finite type.
\item If $X$ is of finite type over $k$, then the canonical morphism $\cD_X\ra \widehat{\cD}_X$ is an isomorphism of functors.
\end{enumerate}
\end{theorem}
\begin{proof}
Consider the ideal $\cI$ in $\cO_X$ corresponding to a closed subscheme $X^{\bd{G}}$ of $X$. We define $X_n$ as a closed subscheme of $X$ determined by the ideal $\cI^n$ and we denote by $\cI_{n}$ the ideal of $X_0$ in $X_n$. Then $\widehat{X} = \{X_n\}_{n\in \NN}$ is a formal $\bd{G}$-scheme. Moreover, by {\cite[Corollary 5.1]{Algebraization}} each $X_n$ is a locally linear $\bd{G}$-scheme and hence by Corollary \ref{corollary:bb_is_representable_on_locally_linear_schemes} there exists a $k$-scheme $Z_n$ equipped with $\bd{M}$-action that represents $\cD_{X_n}$. Note that the square
\begin{center}
\begin{tikzpicture}
[description/.style={fill=white,inner sep=2pt}]
\matrix (m) [matrix of math nodes, row sep=3em, column sep=3em,text height=1.5ex, text depth=0.25ex] 
{  Z_n    &  Z_{n+1}    \\
   X_n    & X_{n+1}           \\} ;
\path[right hook->,line width=1.0pt,font=\scriptsize]
(m-1-1) edge node[above] {$   $} (m-1-2)
(m-2-1) edge node[below] {$   $} (m-2-2);
\path[->,line width=1.0pt,font=\scriptsize]
(m-1-1) edge node[left] {$ i_n $} (m-2-1)
(m-1-2) edge node[right] {$ i_{n+1}  $} (m-2-2);
\end{tikzpicture}
\end{center}
is cartesian according to Theorem \ref{theorem:closed_subfunctors_induced_by_closed_stable_subschemes} for each $n\in \NN$. This implies that the vanishing closed subscheme of $i_{n+1}^{-1}\cI^n_{n+1}\cdot \cO_{Z_{n+1}}$ in $Z_{n+1}$ is $Z_n$. Since the square
\begin{center}
\begin{tikzpicture}
[description/.style={fill=white,inner sep=2pt}]
\matrix (m) [matrix of math nodes, row sep=3em, column sep=3em,text height=1.5ex, text depth=0.25ex] 
{  Z_0    &  Z_{n+1}    \\
   X_0    & X_{n+1}           \\} ;
\path[right hook->,line width=1.0pt,font=\scriptsize]
(m-1-1) edge node[above] {$   $} (m-1-2)
(m-2-1) edge node[below] {$   $} (m-2-2);
\path[->,line width=1.0pt,font=\scriptsize]
(m-1-1) edge node[left] {$ i_0 $} (m-2-1)
(m-1-2) edge node[right] {$ i_{n+1}  $} (m-2-2);
\end{tikzpicture}
\end{center}
is cartesian as a combination of cartesian squares, we derive that the vanishing closed subscheme of $i_{n+1}^{-1}\cI_{n+1}\cdot \cO_{Z_{n+1}}$ in $Z_{n+1}$ is $Z_0$. Note that $$\big(i_{n+1}\cI_{n+1}\cdot \cO_{Z_{n+1}}\big)^n = i_{n+1}^{-1}\cI_{n+1}^n\cO_{Z_{n+1}}$$
Thus $\cZ = \{Z_n\}_{n\in \NN}$ is a formal $\bd{G}$-scheme. According to Remarks \ref{remark:action_of_a_monoid_on_bb_functor} and \ref{remark:functoriality_of_bb_functor}, we derive that it is a formal $\bd{M}$-scheme. Now the commutative diagram
\begin{center}
\begin{tikzpicture}
[description/.style={fill=white,inner sep=2pt}]
\matrix (m) [matrix of math nodes, row sep=3em, column sep=4em,text height=1.5ex, text depth=0.25ex] 
{Z_0 &  ... &  Z_n &  Z_{n+1} &  ...         \\
 X_0 & ... &  X_n  & X_{n+1}  & ...  & X\\} ;
\path[right hook->,line width=1.0pt,font=\scriptsize] 
(m-1-1) edge node[above] {$ $} (m-1-2)
(m-1-2) edge node[above] {$ $} (m-1-3)
(m-1-3) edge node[above] {$ $} (m-1-4)
(m-1-4) edge node[above] {$ $} (m-1-5)
(m-2-1) edge node[above] {$ $} (m-2-2)
(m-2-2) edge node[above] {$ $} (m-2-3)
(m-2-3) edge node[above] {$ $} (m-2-4)
(m-2-4) edge node[above] {$ $} (m-2-5)
(m-2-5) edge node[above] {$ $} (m-2-6);
\path[->,line width=1.0pt,font=\scriptsize]
(m-1-1) edge node[left] {$ i_0$} (m-2-1)
(m-1-3) edge node[left] {$ i_n$} (m-2-3)
(m-1-4) edge node[left] {$ i_{n+1}$} (m-2-4);
\end{tikzpicture}
\end{center}
shows that $\{i_n\}_{n\in \NN}$ is a morphism of formal $\bd{G}$-schemes. Since $\bd{M}$ is a Kempf monoid, {\cite[Theorem 7.1]{Algebraization}} implies that there exists a locally linear $\bd{M}$-scheme $Z$ such that $\widehat{Z} = \{Z_n\}_{n\in \NN}$. Here our argument ramifies. We first provide the proof of \textbf{(1)} and later deal with \textbf{(2)}.
\begin{itemize}
\item Consider a $k$-scheme $Y$ and a family $\{\gamma_n:\bd{M}_n\times_kY\ra X\}_{n\in \NN}\in \widehat{\cD}_X(Y)$. Note that $\gamma_n$ uniquely factors through $X_n$ and hence there exists a unique $\bd{M}$-equivariant morphism $\delta_n:\bd{M}_n\times_kY\ra Z_n$. Hence the family $\{\delta_n\}_{n\in \NN}$ is a morphism
\begin{center}
\begin{tikzpicture}
[description/.style={fill=white,inner sep=2pt}]
\matrix (m) [matrix of math nodes, row sep=3em, column sep=4em,text height=1.5ex, text depth=0.25ex] 
{\bd{M}_0\times_kY &  ... & \bd{M}_n\times_kY &  \bd{M}_{n+1}\times_kY &  ...          \\
 Z_0 & ... &  Z_n  & Z_{n+1}  & ...  \\} ;
\path[right hook->,line width=1.0pt,font=\scriptsize] 
(m-1-1) edge node[above] {$ $} (m-1-2)
(m-1-2) edge node[above] {$ $} (m-1-3)
(m-1-3) edge node[above] {$ $} (m-1-4)
(m-1-4) edge node[above] {$ $} (m-1-5)
(m-2-1) edge node[above] {$ $} (m-2-2)
(m-2-2) edge node[above] {$ $} (m-2-3)
(m-2-3) edge node[above] {$ $} (m-2-4)
(m-2-4) edge node[above] {$ $} (m-2-5);
\path[->,line width=1.0pt,font=\scriptsize]
(m-1-1) edge node[left] {$ \delta_0$} (m-2-1)
(m-1-3) edge node[left] {$ \delta_n$} (m-2-3)
(m-1-4) edge node[left] {$ \delta_{n+1}$} (m-2-4);
\end{tikzpicture}
\end{center}
of a formal $\bd{M}$-schemes. According to {\cite[Example 7.3]{Algebraization}} and {\cite[Corollary 7.4]{Algebraization}} there exists a unique $\bd{M}$-equivariant morphism $\delta:\bd{M}\times_kY\ra Z$ such that $\delta_{\mid \bd{M}_n\times_kY}$ induces $\delta_n:\bd{M}_n\times_kY\ra Z_n$ for every $n\in \NN$. Note that $\delta$ as a $\bd{M}$-equivariant morphism is uniquely determined by a morphism $\eta = \delta \cdot \langle e,1_Y \rangle$ of $k$-schemes, where $e:\Spec k\ra \bd{M}$ is the unit of $\bd{M}$. This proves that
$$\widehat{\cD}_X(Y) \ni \{\gamma_n:\bd{M}_n\times_kY\ra X\}_{n\in \NN} \mapsto \eta \in \Mor_k(Y,Z)$$
is a bijection natural in $Y$. Thus $\widehat{\cD}_X$ is representable by $Z$. Note that $\widehat{r}_X:\widehat{\cD}_X\ra X^{\bd{G}}$ is representable by the canonical retraction $r_Z:Z\ra Z^{\bd{M}} = X^{\bd{G}}$. Hence $\widehat{r}_X$ is affine and if $X$ is locally noetherian, then $\widehat{Z} = \cZ$ is a locally noetherian formal $\bd{M}$-scheme and hence by {\cite[Theorem 7.5]{Algebraization}} we derive that $\widehat{r}_X$ is of finite type.
\item Assume That $X$ is of finite type over $k$. Then $\cZ$ is locally noetherian formal $\bd{M}$-scheme and {\cite[Theorem 7.5]{Algebraization}} implies that the canonical retraction ({\cite[Proposition 5.2]{Algebraization}}) $r:Z\ra Z^{\bd{M}} = X^{\bd{G}}$ is of finite type. Since $X^{\bd{G}}$ is closed subscheme of $X$, we dervie that $Z$ is of finite type over $k$. Next {\cite[Theorem 7.6]{Algebraization}} implies that the comparison functor $\Coh_{\bd{G}}(Z)\ra \Coh_{\bd{G}}(\cZ)$ is an equivalence of categories. Therefore, we derive that there exists a unique monoidal and finitely cocontinuous functor $F:\Coh_{\bd{G}}(X)\ra \Coh_{\bd{G}}(Z)$ such that for every $n\in \NN$ we have the commutative square of monoidal, finitely cocontinuous functors
\begin{center}
\begin{tikzpicture}
[description/.style={fill=white,inner sep=2pt}]
\matrix (m) [matrix of math nodes, row sep=3em, column sep=3em,text height=1.5ex, text depth=0.25ex] 
{  \Coh_{\bd{G}}(Z)    & \Coh_{\bd{G}}(Z_n)     \\
   \Coh_{\bd{G}}(X)    & \Coh_{\bd{G}}(X_n)           \\} ;
\path[->,line width=1.0pt,font=\scriptsize]
(m-1-1) edge node[above] {$   $} (m-1-2)
(m-2-1) edge node[below] {$   $} (m-2-2)
(m-2-1) edge node[left] {$ F $} (m-1-1)
(m-2-2) edge node[right] {$ i_n^*  $} (m-1-2);
\end{tikzpicture}
\end{center}
where horizontal functors are pullbacks along closed immersions $X_n\hookrightarrow X$ and $Z_n\hookrightarrow Z$. In particular, it follows that $F\cdot p^*_X = p^*_Z$, where $p_X:X\ra \Spec k,\,p_Z:Z\ra \Spec k$ are structural morphism and $p_X^*:\bd{Repf}(\bd{G})\ra \Coh_{\bd{G}}(X),p_Z^*:\bd{Repf}(\bd{G})\ra \Coh_{\bd{G}}(Z)$ are pullbacks of coherent $\bd{G}$-sheaves (i.e. finite dimensional $\bd{G}$-representations by {\cite[Example 4.7]{Algebraization}}) from $\Spec k$ (eqipped with the trivial $\bd{G}$-action). Corollary \ref{corollary:equivariant_morphisms_from_categories_Rydh_version} implies that there exists a unique $\bd{G}$-equivariant morphism $f:Z\ra X$ such that for every $n\in \NN$ we have a commutative square
\begin{center}
\begin{tikzpicture}
[description/.style={fill=white,inner sep=2pt}]
\matrix (m) [matrix of math nodes, row sep=3em, column sep=3em,text height=1.5ex, text depth=0.25ex] 
{  Z_n    & Z     \\
   X_n    & X           \\} ;
\path[right hook->,line width=1.0pt,font=\scriptsize]
(m-1-1) edge node[above] {$   $} (m-1-2)
(m-2-1) edge node[below] {$   $} (m-2-2);
\path[->,line width=1.0pt,font=\scriptsize]
(m-1-1) edge node[left] {$ i_n $} (m-2-1)
(m-1-2) edge node[right] {$ f  $} (m-2-2);
\end{tikzpicture}
\end{center}
Consider a $k$-scheme $Y$ and a family $\{\gamma_n:\bd{M}_n\times_kY\ra X\}_{n\in \NN}\in \widehat{\cD}_X(Y)$. Then $\gamma_n$ factors through the composition of $i_n:Z_n\ra X_n$ and the closed immersion $X_n\hookrightarrow X$ for every $n\in \NN$. Thus a family $\{\gamma_n\}_{n\in \NN}$ determines and is determined by a unique family $\{\delta_n:\bd{M}_n\times_kY\ra Z_n\}_{n\in \NN}$ of $\bd{M}$-equivariant morphisms. As above {\cite[Example 7.3]{Algebraization}} and {\cite[Corollary 7.4]{Algebraization}} show that there is a $\bd{M}$-equivariant morphism $\delta:\bd{M}\times_kY\ra Z$ such that $\delta_{\mid \bd{M}_n\times_kY}$ induces $\delta_n$ for every $n\in \NN$. Define $\gamma = f\cdot \delta$. Then $\gamma:\bd{M}\times_kY\ra X$ is a $\bd{G}$-equivariant morphism and $\gamma_{\mid \bd{M}_n\times_kY} = \gamma_n$ for every $n\in \NN$. This shows that the map
$$\cD_X(Y)\ra \widehat{\cD}_X(Y)$$
is surjective for every $k$-scheme $Y$. By Theorem \ref{theorem:comparison_between_formal_and_algebraic_bb_functor_is_injective} we derive that it is injective and hence the canonical morphism $\cD_X\ra \widehat{\cD}_X$ is an isomorphism.
\end{itemize}
\end{proof}
\noindent
It is easy to strengthen \textbf{(2)} in Theorem \ref{theorem:representability_of_bb_functors}.

\begin{corollary}\label{corollary:formal_and_algebraic_bb_functor_are_isomorphic_for_k_schemes_locally_of_finite_type}
Let $\bd{G}$ be a group $k$-scheme and $\bd{M}$ be a Kempf monoid having $\bd{G}$ as a group of units. Suppose that $X$ is a scheme locally of finite type over $k$ equipped with an action of $\bd{G}$. Then the canonical morphism $\cD_X\ra \widehat{\cD}_X$ is an isomorphism. In particular, $\cD_X$ is representable and $r_X:\cD_X\ra X^{\bd{G}}$ is affine and of finite type.
\end{corollary}
\begin{proof}
Let $a:\bd{G}\times_kX\ra X$ be an action of $\bd{G}$ on $X$. Consider an open affine subscheme $V$ of $X$. Then $a$ induces a surjective morphism $a_V:\bd{G}\times_k V\twoheadrightarrow a(\bd{G}\times_k V) = \bd{G}\cdot V$. Since $\bd{G}\times_k V$ is affine $k$-scheme, it is quasi-compact. The image of a quasi-compact topological space under a continuous map is quasi-compact. Thus $\bd{G}\cdot V$ is quasi-compact. Since $X$ is locally of finite type over $k$, we derive that $\bd{G}\cdot V$ is of finite type over $k$. It is also $\bd{G}$-stable. This proves that $X$ admits an open cover $\cU$ by an open $\bd{G}$-stable subschemes of finite type over $k$. By Remark \ref{remark:formal_bb_diagram} we have a commutative triangle
\begin{center}
\begin{tikzpicture}
[description/.style={fill=white,inner sep=2pt}]
\matrix (m) [matrix of math nodes, row sep=3em, column sep=2em,text height=1.5ex, text depth=0.25ex] 
{  \cD_X &            & \widehat{\cD}_X    \\
         & X^{\bd{G}} &                    \\} ;
\path[->,line width=1.0pt,font=\scriptsize]
(m-1-1) edge node[above] {$   $} (m-1-3)
(m-1-1) edge node[left = 4pt, below = -1pt] {$ r_X $} (m-2-2)
(m-1-3) edge node[right = 4pt, below = -1pt] {$ \widehat{r}_X $} (m-2-2);
\end{tikzpicture}
\end{center}
and according to Theorem \ref{theorem:open_subfunctors_induced_by_open_stable_subschemes} and Proposition \ref{proposition:open_subfunctors_induced_by_open_stable_subschemes_for_formal_bb} for every $U$ in $\cU$ base change of the triangle above along open immersion $U^{\bd{G}}\hookrightarrow X^{\bd{G}}$ yields a triangle
\begin{center}
\begin{tikzpicture}
[description/.style={fill=white,inner sep=2pt}]
\matrix (m) [matrix of math nodes, row sep=3em, column sep=2em,text height=1.5ex, text depth=0.25ex] 
{  \cD_U &            & \widehat{\cD}_U    \\
         & U^{\bd{G}} &                    \\} ;
\path[->,line width=1.0pt,font=\scriptsize]
(m-1-1) edge node[above] {$   $} (m-1-3)
(m-1-1) edge node[left = 4pt, below = -1pt] {$ r_U $} (m-2-2)
(m-1-3) edge node[right = 4pt, below = -1pt] {$ \widehat{r}_U $} (m-2-2);
\end{tikzpicture}
\end{center}
in which the horizontal morphism $\cD_U\ra \widehat{\cD}_U$ is an isomorphism by \textbf{(2)} in Theorem \ref{theorem:representability_of_bb_functors} and the fact that $U$ is $\bd{G}$-scheme of finite type over $k$. Since $\widehat{\cD}_X$ is representable by \textbf{(1)} in Theorem \ref{theorem:representability_of_bb_functors}, it follows that $\cD_X$ is representable and the canonical morphism $\cD_X\ra \widehat{\cD}_X$ is an isomorphism of functors. Thus $r_X$ and $\widehat{r}_X$ are isomorphic and this completes the proof.
\end{proof}






















\small
\bibliographystyle{apalike}
\bibliography{../zzz}

\end{document}
