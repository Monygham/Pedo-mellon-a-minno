\input ../pree.tex

\usetikzlibrary{matrix,arrows, cd, calc}
\newcommand{\Zhat}{\widehat{Z}}
\newcommand{\What}{\widehat{W}}
\newcommand{\Zformal}{\cZ}
\newcommand{\Wformal}{\cW}
\newcommand{\Group}{\mathbf{G}}%
\newcommand{\Groupop}{\mathbf{G}^{\mathrm{op}}}%
\newcommand{\Gprod}{\Group\times \Groupop}%
\newcommand{\Groupconn}{\mathbf{G}^{\circ}}%
\newcommand{\Nroup}{\mathbf{N}}%
\newcommand{\Nred}{\Nroup_{\mathrm{red}}^{\circ}}%
\newcommand{\Gbar}{\overline{\mathbf{G}}}%
\newcommand{\Nbar}{\overline{\mathbf{N}}}%
\newcommand{\Fbar}{\overline{\mathbf{F}}}%
\newcommand{\Ghat}{\widehat{\mathbf{G}}}%

\newcommand{\Gmult}{\mathbb{G}_m}%
%% Glowna rozmaitosc
\newcommand{\kk}{k}%
\newcommand{\kkbar}{\overline{k}}%
\newcommand{\varX}{X}%
%% Kategoria schematów lokalnie skończonego typu nad ciałem
\newcommand{\kSch}{\Sch_{\kk}}%
%% Funktor zbieznych deformacji (z argumentem i bez)
\newcommand{\Dfunctor}[1]{\mathcal{D}_{#1}}%
\newcommand{\DX}{\Dfunctor{\varX}}%
\newcommand{\Xplus}{\varX^+}%
\newcommand{\Yplus}{Y^+}%
\newcommand{\fplus}{f^+}%
%% Funktor punktow stalych (z argumentem i bez)
\newcommand{\Ffunctor}[1]{{#1}^{\Group}}%
\newcommand{\FX}{\Ffunctor{\varX}}%
\newcommand{\MorX}{\Mor_{\kk}(-, \varX)}%
%% Funktor formalnych kompaktyfikacji (z argumentem i bez)
\newcommand{\Hfunctor}[1]{\widehat{\mathcal{D}}_{#1}}%
\newcommand{\HX}{\Hfunctor{\varX}}%
%% Wlozenie ``wlokna nad 1''
\newcommand{\ione}[1]{i_{#1}}%
\newcommand{\ioneX}{\ione{\varX}}
%% Wlozenie ``wlokna nad \infty''
\newcommand{\iinfty}[1]{\pi_{#1}}%
\newcommand{\iinftyX}{\iinfty{\varX}}%
\newcommand{\isection}[1]{s_{#1}}%
\newcommand{\isectionX}{\isection{\varX}}%
%% wlozenie puntkow stalych w X^+
\newcommand{\emb}{e}%
%% ABB, dla skrotu
\newcommand{\BBname}{Bia{\l}ynicki-Birula}%
\newcommand{\Tname}{locally $\Group$-affine}%
%% Stogi
\DeclareMathOperator{\Maps}{Maps}
\DeclareMathOperator{\Irr}{Irr}
\newcommand{\MapsStack}{\underline{\Maps}}%
\DeclareMathOperator{\colim}{colim}


\DeclareMathOperator{\charr}{char}%
\DeclareMathOperator{\id}{id}%
\DeclareMathOperator{\Hilb}{Hilb}%
\newcommand{\into}{\hookrightarrow}%
\newcommand{\onto}{\twoheadrightarrow}%
\newcommand{\sigmabar}{\overline{\sigma}}%

\newcommand{\GroupGlobalSects}{H^0(\Group, \cO_{\Group})}%
\newcommand{\Hbar}{\overline{\mathbf{H}}}%
\newcommand{\Hroup}{\mathbf{H}}%
\newcommand{\mubar}{\overline{\mu}}%

\begin{document}

\section{Formal functors and representability}

        

        \begin{example}[Formal schemes from algebraic ones]\label{algebraicformalgroupschemes}
            Let $Z$ be a $\Group$-scheme and $\cI$ be the ideal of
            $Z^{\Group}$. Then $Z_n=V(\cI^{n+1})$ is a closed $\Group$-stable
            subscheme of $Z$ for every $n\in \NN$ and this yields to a formal
            $\Group$-scheme $\Zformal=\{Z_n\}_{n\in \NN}$. We denote this formal
            $\Group$-scheme by $\Zhat$.
        \end{example}


        Now we define morphisms of formal $\Group$-schemes.
        \begin{definition}
            Let $\Zformal =\{Z_n\}$ and $\Wformal =\{W_n\}$ be formal $\Group$-schemes.
            A \emph{morphism $\varphi:\Wformal\to \Zformal$ of formal
            $\Group$-schemes} is a family of $\Group$-equivariant morphisms
            $\varphi=\{\varphi_n:W_n\ra Z_n\}$ such that for every $n\in \NN$
            we have a commutative square
            \begin{center}
                \begin{tikzpicture}
                    [description/.style={fill=white,inner sep=2pt}]
                    \matrix (m) [matrix of math nodes, row sep=3em, column sep=2em,text height=1.5ex, text depth=0.25ex] 
                    {W_{n+1}&  &    Z_{n+1}          \\
                W_n&   & Z_n         \\} ;
                \path[->,font=\scriptsize]  
                (m-1-1) edge node[above] {$\varphi_{n+1}$} (m-1-3)
                (m-2-1) edge node[below] {$\varphi_n$} (m-2-3);
                \path[right hook->,font=\scriptsize]
                (m-2-3) edge node[right] {$ $} (m-1-3)
                (m-2-1) edge node[right] {$ $} (m-1-1);
            \end{tikzpicture}
        \end{center}
    \end{definition}
    \begin{remark}[Morphisms of formal $\Gbar$-schemes are $\Gbar$-equivariant]
        Let $\Wformal$ and $\Zformal$ be formal $\Gbar$-schemes and consider
        their morphism $\varphi:\Wformal \to \Zformal$ (as formal
        $\Group$-schemes). Then for every $n\in \NN$ the morphism
        $\varphi_n\colon W_n\ra Z_n$ is $\Gbar$-equivariant. To see this,
        consider Diagram~\eqref{eq:Gbarformaldiagram}.
        \begin{equation}\label{eq:Gbarformaldiagram}
            \begin{tikzpicture}
                [description/.style={fill=white,inner sep=2pt}]
                \matrix (m) [matrix of math nodes, row sep=3em, column sep=2em,text height=1.5ex, text depth=0.25ex]
                {W_{n}&  &  W_0\times_{Z_0}Z_n  & &Z_n          \\
                     &   & W_0           & &Z_0     \\} ;
                \path[->,font=\scriptsize] 
                (m-1-1) edge[bend left = 30] node[above] {$ \varphi_n $} (m-1-5) 
                (m-1-1) edge node[below] {$r_n $} (m-1-3)
                (m-1-3) edge node[right] {$p_n $} (m-2-3)
                (m-1-1) edge node[left = 3pt, below = 3pt] {$\iinfty{W_n} $} (m-2-3)
                (m-1-3) edge node[below] {$q_n $} (m-1-5)
                (m-2-3) edge node[below] {$\varphi_0 $} (m-2-5)
                (m-1-5) edge node[right] {$\iinfty{Z_n} $} (m-2-5);
            \end{tikzpicture}
        \end{equation}
        Since $W_0$ and $Z_0$ are equipped with trivial $\Gbar$-actions, also the
        pullback $W_0\times_{Z_0}Z_n$ is a $\Gbar$-scheme and $q_n$ is
        $\Gbar$-equivariant. Recall that $\iinfty{Z_n}$, $\iinfty{W_n}$ are
        affine morphisms. Therefore, $p_n$ is affine. Hence $r_n$ is a
        $\Group$-equivariant morphism between $\Gbar$-schemes separated (even
        affine) over $W_0$. Thus $r_n$ is $\Gbar$-equivariant.
    \end{remark}

    \begin{definition}\label{ref:locallylinear:def}
        A \emph{locally linear $\Gbar$-scheme} is a $\Gbar$-scheme which
        admits an open cover by affine $\Gbar$-stable subschemes. The category
        of locally linear $\Gbar$-schemes consists of those schemes and
        $\Gbar$-equivariant morphisms.
    \end{definition}
    Let $Z$ be a locally linear $\Gbar$-scheme. By Proposition~\ref{ref:isomorphism:prop}, the
    map $\Dfunctor{Z}\to Z$ is an isomorphism. In particular, there is a
    canonical morphism $\iinfty{Z}\colon Z\to Z^{\Group}$, which is the
    multiplication by zero. For an affine open $\Gbar$-stable cover
    $\{V_i\}_i$
    of $Z$, we have $V_i = \iinfty{Z}^{-1}(\iinfty{Z}(V_i))$ by
    Proposition~\ref{ref:openimmersionscartesian:prop}, hence the canonical morphism $\iinfty{Z}\colon Z\to Z^{\Group}$ is affine.

\begin{definition}\label{ref:algebraization:def}
Let $\Zformal$ be a formal $\Gbar$-scheme. An \emph{algebraization} of $\Zformal$ is a $\Gbar$-scheme $Z$ such that
\begin{enumerate}
            \item\label{it:algebraization:one} $Z$ is a locally linear
                $\Gbar$-scheme.
            \item\label{it:algebraization:two} $\Zformal$ and $\Zhat$ are isomorphic formal $\Gbar$-schemes.
        \end{enumerate}
    \end{definition}
    By the above discussion, the morphism
    $\iinfty{Z}\colon Z\to Z^{\Group}$ is affine for any algebraization $Z$.

    \begin{theorem}[Algebraization of a formal $\Gbar$-scheme]\label{ref:algebraizationOfFormalSchemes:thm}
        Let $\Zformal =\{Z_n\}$ be a formal $\Gbar$-scheme. Then there exists
        a colimit
        \[
            Z = \colim_{n} Z_n
        \]
        in the category of locally linear $\Gbar$-schemes and $Z$ is the
        unique algebraization of $\Zformal$.
        If in addition $\Zformal$ is locally Noetherian, then $\iinfty{Z}$ is of finite type. If
        $\Zformal$ is locally Noetherian and $Z_0$ is of finite type, then also $Z$ is of
        finite type.
    \end{theorem}

Now we spell out the main idea of the proof: the $\Gbar$-scheme $Z$
required in Theorem~\ref{ref:algebraizationOfFormalSchemes:thm} is equal to $\Spec_{Z_0} \cA$, where
$\cA$ is the limit of $\cA_n$ \emph{in the category of
$\Gbar$-algebras}; in other words each isotypic component of $\cA$ is the
limit of isotypic components of $\cA_n$.
Our first goal is to prove a stabilization result.
We denote by $\Irr(\Group)$ the set of isomorphism types of irreducible
$\Group$-representations and by $\Irr(\Gbar) \subset \Irr(\Group)$ the
subset of $\Gbar$-representations. For $\lambda\in \Irr(\Group)$ and
a quasi-coherent $\Gbar$-module $\cC$ on $Z_0$ we denote by $\cC[\lambda]
\subset \cC$ the $\Gbar$-submodule such that $H^0(U, \cC[\lambda]) \subset H^0(U, \cC)$
is the union of all $\Group$-subrepresentations of $H^0(U, \cC)$ isomorphic to
$\lambda$ (i.e., the isotypic component of $\lambda$).


\begin{lemma}[stabilization on an isotypic component]\label{ref:stability:lem}
Let $\lambda\in \Irr(\Gbar)$. Then there exists a number $n_{\lambda}\in \NN$
such that the following holds. Let $\Zformal=\{Z_n\}$ be a formal $\Gbar$-scheme
and $\{\cA_{n+1} \onto \cA_n\}$ be the associated sequence of quasi-coherent
$\Gbar$-algebras. Then for every $n > n_{\lambda}$ the surjection
\[
    \cA_{n}[\lambda] \onto \cA_{n-1}[\lambda]
\]
is an isomorphism. If $\lambda_0\in \Irr(\Gbar)$ is the
trivial representation, then we may take $n_{\lambda_0}=0$.
\end{lemma}
\begin{proof}[Proof of Lemma~\ref{ref:stability:lem}]
    The claims are preserved under field extension, so we may assume our field
    is algebraically closed (hence perfect) so we may use the Kempf's torus.
    Fix a grading on
    $\kk[\Gbar]$ induced by a Kempf's torus for $\kk$ as in
    Corollary~\ref{ref:KempfTorus:cor}. Denote by $A_{\lambda}\subseteq \NN$
    the set of weights which appear in
    $\kk[\Group]_{\lambda}$. Since $\dim_{\kk}\kk[\Group]_{\lambda}$
    is finite by Proposition~\ref{ref:isotypiccomponents:prop}, the set
    $A_{\lambda}$ is finite. Put
    \[
        n_{\lambda}=\sup A_{\lambda}.
    \]
    Fix $n> n_{\lambda}$ and let
    $\cI_n = \ker(\cA_n\to \cA_0)$. Then we have a decomposition
    with respect to the chosen torus
    \[
        \cA_n=\bigoplus_{i\geq 0}(\cA_n)[i],
    \]
    By Corollary~\ref{ref:KempfTorus:cor}, we have $\cI_n =
    \bigoplus_{i\geq 1}(\cA_n)[i]$. Since $n > n_{\lambda}$ we have
    \[
        \cI^{n}_n \subset \bigoplus_{i\geq
        n}(\cA_n)[i]\subseteq \bigoplus_{i \not \in
            A_{\lambda}}(\cA_n)[i]
    \]
Hence, $\cI^{n}_n[\lambda] = 0$. But
$\cI^{n}_n[\lambda] = \ker(\cA_{n}[\lambda] \to
\cA_{n-1}[\lambda])$, thus $\cA_{n}[\lambda] \to \cA_{n-1}[\lambda]$ is an
isomorphism.
Finally note that $A_{\lambda_0}=\{0\}$. This implies that $n_{\lambda_0}=0$.
\end{proof}

\begin{proof}[Proof of Theorem~\ref{ref:algebraizationOfFormalSchemes:thm}]
    Let $\cA_n$ be the quasi-coherent $\Gbar$-algebras as
    in~\eqref{eq:Andefinition}. For $\lambda\in \Irr(\Gbar)$ we define
    $\cA[\lambda] := \cA_n[\lambda]$, where $n\geq n_{\lambda}$ as in
    Lemma~\ref{ref:stability:lem}.
    \[
        \cA=\bigoplus_{\lambda\in
            \Irr(\Gbar)}\cA[\lambda]=\bigoplus_{\lambda\in
                \Irr(\Gbar)}\cA_{n_{\lambda}}[\lambda].
    \]
    Clearly $\cA[\lambda_0] = \cA_0 = \cO_{Z_0}$ canonically (where
    $\lambda_0$ is the trivial representation), hence $\cA$ is an
    $\cO_{Z_0}$-module.
    Actually $\cA=\lim_{n}\cA_n$ in the category of quasi-coherent
    $\Gbar$-modules on $Z_0$.
    We construct the algebra structure on $\cA$. For this
    pick $\eta_1, \eta_2\in \Irr(\Gbar)$. Fix the finite set
    $\{\lambda_1, \ldots ,\lambda_s\}\subseteq \Irr(\Gbar)$ of representations
    which appear in $\kk[\Gbar]_{\eta_1}\otimes_{\kk}\kk[\Gbar]_{\eta_2}$.
    Then, for every $n\in \NN$, we have the multiplication
$$\cA_n[\eta_1]\otimes_{\kk} \cA_n[\eta_2]\ra \cA_n[\eta_1]\cdot \cA_n[\eta_2]\subseteq \bigoplus_{i=1}^s\cA_n[\lambda_i]$$
and by Lemma \ref{ref:stability:lem} these morphisms can be identified for $n\geq \sup \{n_{\eta_1},n_{\eta_2},n_{\lambda_1},...,n_{\lambda_s}\}$. We define
$$\cA[\eta_1]\otimes_{\kk} \cA[\eta_2]\ra  \bigoplus_{i=1}^s\cA[\lambda_i]\subseteq \cA$$
as a morphism induced by the multiplication morphism for any $n\geq \sup
\{n_{\eta_1},n_{\eta_2},n_{\lambda_1}, \ldots ,n_{\lambda_s}\}$. This gives an
$\cO_{Z_0}$-algebra structure on $\cA$, so $\cA$ is in fact the limit of
$\cA_n$ is the category of $\Gbar$-algebras. Note that from the description of
$\cA$ it follows that for every $n\in \NN$ we have a surjective morphism
$p_n:\cA\onto \cA_n$ of $\Gbar$-algebras. We denote its kernel
by $\cJ_n$ and we put $\cJ:=\cJ_0$. The natural injection $\cO_{Z_0} = \cA_0 \to \cA$ is a section
of $p_0$, so that we have
\[
    \cJ=\bigoplus_{\lambda \in \Irr(\Gbar)\setminus
        \{\lambda_0\}}\cA[\lambda].
\]
We also denote by $\cI_n$ the kernel of $\cA_n\twoheadrightarrow
\cA_0=\cO_{Z_0}$ for $n\in \NN$. Then $\cI_n=\cJ/\cJ_n$.
%Since $\Zformal$ is a
%formal $\Gbar$-scheme, the ideal $\cI_n^{n+1}$ cuts $Z_n$ out of $Z_n$, hence
%$\cI_n^{n+1} = 0$. Consequently, $\cJ^{n+1} \subset
%\cJ_n$\jjtodo{to wydaje się niepotrzebne, bo potem i tak pokazujemy równość}.
Fix $m\in \NN$ and consider $n\in \NN$
such that $n\geq m$. Since $\Zformal$ is a formal $\Gbar$-scheme, the sheaf
$\cI_n^{m+1}$ is the kernel of the morphism $\cA_n\twoheadrightarrow \cA_m$.
Thus
\[
\cJ_m/\cJ_n=\cI_n^{m+1}=(\cJ^{m+1}+\cJ_n)/\cJ_n.
\]
Both $\cJ_m$ and $\cJ^{m+1}$ are $\Irr(\Gbar)$-graded and for given
$\lambda\in \Irr(\Gbar)$ and $n\gg 0$ the isotypic component
$\cJ_n[\lambda]$ is zero by Lemma~\ref{ref:stability:lem}. Hence $\cJ_m=\cJ^{m+1}$ for every $m \in \NN$.
We define
\[
    Z=\Spec_{Z_0}(\cA)
\]
and we denote by $\pi:Z\to Z_0$ the structural morphism. The scheme $Z$
inherits a $\Gbar$-action from $\cA$. For
every $n\in \NN$ the zero-set of $\cJ^{n+1}\subseteq \cA$ is a $\Gbar$-scheme
isomorphic to $Z_n$. Hence $\Zformal$ is isomorphic to $\Zhat$.
Thus $Z$ is an algebraization of $\Zformal$. Since $\cA=\lim \cA_n$, we
have $Z = \colim Z_n$ in the category of locally linear $\Gbar$-schemes.

It remains to prove uniqueness of algebraization. Let $Z' = \Spec_{Z_0} \cA'$
be an algebraization of $\Zformal = \{Z_n\}$. Then $Z_n \into Z'$, so by the
universal property of colimit, we obtain a $\Gbar$-morphism $Z\to Z'$,
corresponding to $\cA' \to \cA$. It induces epimorphisms $\cA' \onto \cA_n$
for all $n$. For each $\lambda\in \Irr(\Gbar)$, the composition
\[
    \cA'[\lambda]\to \cA[\lambda]  \simeq \cA_{n_\lambda}[\lambda]
\]
is an epimorphism, hence $\cA'\to \cA$ is an epimorphism. The kernel of
$\cA'\to \cA$ is equal to
\[
    \bigcap_n \ker(\cA'\to \cA_n) = \bigcap_n \ker(\cA' \to \cA_0)^n.
\]
To prove that this kernel is zero, we may enlarge the field to an
algebraically closed field, so the result follows from
Corollary~\ref{ref:KempfTorus:cor}.

Assume that each scheme $Z_n$ is locally Noetherian over $\kk$. Then $\cI_n$
is a coherent $\cA_n$-module, thus $\cI_n^i/\cI^{i+1}$ is a coherent
$\cA_0$-module for all $i$. The series
\[
    0 = \cI_n^{n+1} \subset \cI^n \subset  \ldots \subset \cI \subset \cA_n
\]
has coherent subquotients, hence $\cA_n$ is a coherent $\cO_{Z_n}$-algebra.
Thus $\cA[\lambda]$ is a coherent $\cO_{Z_0}$-module for every
$\lambda\in \Irr(\Gbar)$. The claim that $\pi$ is of finite type is local on
$Z^{\Group}$, hence we may
assume that $Z^{\Group}$ is quasi-compact.
The sheaf $\cJ/\cJ^2\subseteq \cA_1$ is coherent so there exists a finite set
$\lambda_1, \ldots, \lambda_r\in \Irr(\Gbar)\setminus \{\lambda_0\}$ such that the morphism
\[
    \bigoplus_{i=1}^r\cA[\lambda_i]\ra \cJ/\cJ^2
\]
induced by $\cA\twoheadrightarrow \cA_2$ is surjective. Let $\cB \subset \cA$
be the quasi-coherent $\cO_{Z_0}$-subalgebra generated by the coherent
subsheaf $\cM := \bigoplus_{i=1}^r\cA[\lambda_i]\subseteq \cA$.
Let $\kkbar$ be an algebraic closure of $\kk$ and let $\cA' = \cA \otimes
\kkbar$. Fix a Kempf's torus over
$\kkbar$ and the associated grading $\cA' = \bigoplus_{i\geq 0}
\cA'[i]$ as in
Corollary~\ref{ref:KempfTorus:cor}.
Then $\cJ = \bigoplus_{i\geq 1} \cA'[i]$ is a graded ideal and $\cJ/\cJ^2$ is
generated by the graded (coherent) subsheaf $\cM' = \bigoplus_{i=1}^r\cA'[\lambda_i]$. By
graded Nakayama's lemma, the ideal $\cJ$ itself is generated by (the elements
of) $\cM'$. Then by induction on the degree, $\cA'$ is generated by $\cM'$ as
an algebra. In other words, $\cA' = \cB\otimes \kkbar$. Thus also $\cA = \cB$ and so $\cA$ is of
finite type over $\cO_{Z_0}$.
\end{proof}

\newcommand{\varphihat}{\widehat{\varphi}}%
With the proof of Theorem~\ref{ref:algebraizationOfFormalSchemes:thm} in hand,
we can easily algebraize also equivariant mappings between formal schemes.

\begin{proposition}[Algebraization of morphisms of formal
    $\Gbar$-schemes]\label{ref:algebraizationOfMaps:prop}
    Let $\Wformal = \{W_n\}$ and $\Zformal = \{Z_n\}$ be formal $\Gbar$-schemes. Let $W$ and $Z$ be
    algebraizations of $\Wformal$ and $\Zformal$ respectively (see
    Theorem~\ref{ref:algebraizationOfFormalSchemes:thm}). Then every
    $\Gbar$-morphism $\varphihat\colon\Wformal\to \Zformal$  is the formalization of a unique
    $\Gbar$-equivariant morphism $\varphi\colon W\to Z$.
\end{proposition}
\begin{proof}
    The map $\varphihat$ induces maps $W_n \to Z_n \into Z$. By
    Theorem~\ref{ref:algebraizationOfFormalSchemes:thm}, the scheme $W$ is a
    colimit of $W_n$ in the category of locally linear $\Gbar$-schemes. By the universal
    property of the colimit, we obtain a unique $\Gbar$-equivariant morphism $W\to Z$.
\end{proof}

\section{Formal $\bd{M}$-schemes}
\noindent
Let $\bd{M}$ be a $k$-monoid scheme.

\begin{definition}
Let $X$ be a $\bd{M}$-scheme. We say that $X$ is \textit{locally linear $\bd{M}$-scheme} if there exists an open cover of $X$ consisting of affine and $\bd{M}$-stable subchemes of $X$.
\end{definition}

\begin{definition}
\textit{A formal $\bd{M}$-scheme} consists of a sequence $\cZ = \{Z_n\}_{n\in \NN}$ of $\bd{M}$-schemes together with $\bd{M}$-equivariant closed immersions
\begin{center}
\begin{tikzpicture}
[description/.style={fill=white,inner sep=2pt}]
\matrix (m) [matrix of math nodes, row sep=3em, column sep=3em,text height=1.5ex, text depth=0.25ex] 
{ Z_0 &  Z_1 & ... & Z_n & Z_{n+1} & ... \\} ;
\path[right hook->,line width=1.0pt,font=\scriptsize]  
(m-1-1) edge node[above] {$ $} (m-1-2)
(m-1-2) edge node[above] {$ $} (m-1-3)
(m-1-3) edge node[above] {$ $} (m-1-4)
(m-1-4) edge node[above] {$ $} (m-1-5)
(m-1-5) edge node[above] {$ $} (m-1-6);
\end{tikzpicture}
\end{center}
\begin{enumerate}[label=\textbf{(\arabic*)}, leftmargin=1.5em]
\item $\bd{M}$-scheme $Z_0$ is locally linear.
\item Let $\cI_n$ be an ideal of $\cO_{Z_n}$ defining $Z_0$. Then for every $m\leq n$ the subscheme $Z_m \subset Z_n$ is defined by $\cI_n^{m+1}$.
\end{enumerate}
\end{definition}

\section{Thick subcategories}

\begin{definition}
Let $\cC$ be an abelian category and let $\cS$ be its full subcategory. Suppose that for every exact sequence in $\cC$
\begin{center}
\begin{tikzpicture}
[description/.style={fill=white,inner sep=2pt}]
\matrix (m) [matrix of math nodes, row sep=3em, column sep=3em,text height=1.5ex, text depth=0.25ex] 
{ 0 & X' & X & X'' & 0 \\} ;
\path[->,line width=1.0pt,font=\scriptsize]  
(m-1-1) edge node[above] {$ $} (m-1-2)
(m-1-2) edge node[above] {$ $} (m-1-3)
(m-1-3) edge node[above] {$ $} (m-1-4)
(m-1-4) edge node[above] {$ $} (m-1-5);
\end{tikzpicture}
\end{center}
we have $X\in \cS$ if and only if $X',X''\in \cS$. Then $\cS$ is called \textit{a thick subcategory of $\cC$}.
\end{definition}

\begin{definition}
A category $\cC$ is called \textit{well-powered} if the class of subobjects of $X$ is a set for every object $X$ in $\cC$.
\end{definition}

\begin{proposition}\label{proposition:largest_subobject_in_thick_subcategory}
Let $\cC$ be an $\bd{Ab}3$-category and $\cS$ be a thick subcategory. Assume that $\cS$ is closed under small direct sums. Then for every object $X$ in $\cC$ there exists a unique subobject $S(X)$ such that for every morphism $f:Y\ra X$ in $\cC$ with $Y$ in $\cS$ we have $f(Y)\subseteq S(X)$.
\end{proposition}
\begin{proof}
One can prove the result by invoking appropriate adjoint functor theorems {\cite[Chapter V, Sections 5 and 6]{Maclane}}. For self-containment we present the complete proof below.\\
Fix an object $X$ of $\cC$. Since $\cC$ is well-powered, the class $\{Y_i\}_{i\in I}$ of subobjects of $X$ that belong to $\cS$ is a set. Since $\cS$ is closed under small direct sums we derive that $\sum_{i\in I}Y_i\subseteq X$ is in $\cS$. Indeed, this is the image of the canonical morphism
\begin{center}
\begin{tikzpicture}
[description/.style={fill=white,inner sep=2pt}]
\matrix (m) [matrix of math nodes, row sep=3em, column sep=3em,text height=1.5ex, text depth=0.25ex] 
{ \bigoplus_{i\in I}Y_i & X \\} ;
\path[->,line width=1.0pt,font=\scriptsize]  
(m-1-1) edge node[above] {$ $} (m-1-2);
\end{tikzpicture}
\end{center}
and since $\cS$ is a thick subcategory closed under small direct sums, we deduce that this image is an object of $\cS$. Thus $S(X) = \sum_{i\in I}Y_i$ is the largest subobject of $X$ contained in $\cS$. This implies the statement.
\end{proof}

\begin{definition}
Let $\cC$ be an $\bd{Ab}3$-category and $\cS$ be a thick subcategory. Assume that $\cS$ is closed under small direct sums. Then 
\end{definition}


\small
\bibliographystyle{apalike}
\bibliography{../zzz}

\end{document}
