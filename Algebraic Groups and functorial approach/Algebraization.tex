\input ../pree.tex

\usepackage{todonotes}
\newcommand{\lstodo}[1]{\todo[color=green!40,bordercolor=green,size=\footnotesize]{\textbf{LS TODO: }#1}}

\begin{document}

\title{Algebraization of formal $\bd{M}$-schemes}
\date{}
\maketitle

\section{Introduction}
\noindent
In these notes we prove some results concerning algebraization of formal schemes in equivariant setting.

\section{Some $2$-categorical limits}
\noindent
Consider a category $\cC$ and its endofunctor $T:\cC\ra \cC$. Our goal is to construct certain $2$-categorical limit associated with a pair $(\cC,T)$. Consider pairs $\left(X,u\right)$ consisting of an object $X$ of $\cC$ and an isomorphism $u:T(X)\ra X$ in $\cC$. If $\left(X,u\right)$ and $\left(Y,w\right)$ are two such pairs, then a morphism $f:(X,u) \ra (Y,u)$ is a morphism $f:X\ra Y$ in $\cC$ such that the following square
\begin{center}
\begin{tikzpicture}
[description/.style={fill=white,inner sep=2pt}]
\matrix (m) [matrix of math nodes, row sep=3em, column sep=3em,text height=1.5ex, text depth=0.25ex] 
{ T(X) &  X    \\
  T(Y) &  Y           \\} ;
\path[->,line width=1.0pt,font=\scriptsize]  
(m-1-1) edge node[above] {$ u  $} (m-1-2)
(m-2-1) edge node[below] {$ w $} (m-2-2)
(m-1-1) edge node[left] {$ T(f) $} (m-2-1)
(m-1-2) edge node[right] {$ f  $} (m-2-2);
\end{tikzpicture}
\end{center}
is commutative. This data give rise to a category $\cC(T)$. There exists a forgetful functor $\pi:\cC(T)\ra \cC$ that sends a morphism $f:(X,u)\ra (Y,w)$ to $f:X\ra Y$. Moreover, there exists a natural isomorphism $\sigma:T\cdot \pi \Rightarrow \pi$ such that the component of $\sigma$ on an object $(X,u)$ of $\cC(T)$ is $u$. The next result states that the data above form a certain $2$-categorical limit.

\begin{theorem}\label{theorem:endoscope_2_limits}
Let $(\cC,T)$ be a pair consiting of a category and its endofunctor $T:\cC\ra \cC$. Suppose that $\cD$ is a category, $P:\cD\ra \cC$ is a functor and $\tau:T\cdot P \Rightarrow P$ is a natural isomorphisms. Then there exists a unique functor $F:\cD\ra \cC(T)$ such that $P = \pi\cdot F$ and $\sigma_F = \tau$.
\end{theorem}
\begin{proof}
Suppose that $F:\cD\ra \cC(T)$ is a functor such that $P = \pi\cdot F$ and $\sigma_F = \tau$. Pick an object $X$ of $\cD$. Then we have $\pi\cdot F(X) = P(X)$ and $\sigma_{F(X)} = \tau_X$. This implies that
$$F(X) = \left(P(X),\tau_X:T(P(X))\ra P(X)\right)$$
Next if $f:X\ra Y$ is a morphism in $\cD$, then we derive that $\pi(F(f)) = P(f)$. Hence $F(f) = P(f)$. This implies that there exists at most one functor $F$ satisfying the properties above. Note also that formulas
$$F(X) = \left(P(X),\tau_X:T(P(X))\ra P(X)\right),\,F(f) = P(f)$$
for an object $X$ in $\cD$ and a morphism $f:X\ra Y$ in $\cD$, give rise to a functor that satisfy $P = \pi\cdot F$ and $\sigma_F = \tau$. This establishes existence and the uniqueness of $F$.
\end{proof}
\noindent
Assume now that the pair $(\cC,T)$ consists of a monoidal category $\cC$ and a monoidal endofunctor $T$. Then there exists a canonical monoidal structure on $\cC(T)$. We define $(-)\otimes_{\cC(T)}(-)$ by formula
$$(X,u) \otimes_{\cC(T)}(Y,w) = \left(X\otimes_{\cC}Y,\left(u \otimes_{\cC} w\right)\cdot m_{X,Y}\right)$$
where
$$m_{X,Y}:T\left(X\otimes_{\cC} Y\right) \ra T(X)\otimes_{\cC}T(Y)$$
is the tensor preserving isomorphism of $T$. We also define the unit
$$I_{\cC(T)} = \left(I, T(I)\cong I\right)$$
where isomorphism $T(I)\cong I$ is precisely the unit preserving isomorphism of the monoidal functor $T$. The associativity natural isomorphism for $(-)\otimes_{\cC(T)}(-)$ and right, left units for $I_{\cC(T)}$ in $\cC(T)$ are associavity natural isomorphism and right, left units for $\cC$, respectively. The structure makes a functor $\pi:\cC(T)\ra \cC$ strict monoidal and $\sigma$ a monoidal natural isomorphism. The next result states that the data with these extra monoidal structure form a $2$-categorical limit in the $2$-category of monoidal categories.

\begin{theorem}\label{theorem:endoscope_monoidal_2_limit}
Let $(\cC,T)$ be a pair consiting of a monoidal category and its monoidal endofunctor $T:\cC\ra \cC$. Suppose that $\cD$ is a monoidal category, $P:\cD\ra \cC$ is a monoidal functor and $\tau:T\cdot P \Rightarrow P$ is a monoidal natural isomorphisms. Then there exists a unique monoidal functor $F:\cD\ra \cC(T)$ such that $P = \pi\cdot F$ and $\sigma_F = \tau$ as monoidal functors and monoidal transformations.
\end{theorem}
\begin{proof}
Note that $F$ must be defined as it was described in the proof of Theorem \ref{theorem:endoscope_2_limits}. Namely we must have
$$F(X) = \left(P(X),\tau_X:T(P(X))\ra P(X)\right),\,F(f) = P(f)$$
for an object $X$ in $\cC$ and a morphism $f:X\ra Y$ in $\cC$.\\
Suppose now that $F$ admits a structure of a monoidal functor such that $P = \pi\cdot F$ as monoidal functors. Let
$$\big\{m^F_{X,Y}:F(X\otimes_{\cD}Y)\ra F(X)\otimes_{\cC(T)}F(Y)\big\}_{X,Y\in \cC},\,\phi^F:F(I_{\cD})\ra I_{\cC(T)}$$
be the data forming that structure. Since $\pi$ is a strict monoidal functor and $P = \pi\cdot F$ as monoidal functors, we derive that for any objects $X,Y$ of $\cC$
$$\pi(m^F_{X,Y}):P(X\otimes_{\cD}Y) \ra P(X)\otimes_{\cC}P(Y)$$
is the tensor preserving isomorphism $m^{P}_{X,Y}:P(X\otimes_{\cD}Y) \ra P(X)\otimes_{\cC}P(Y)$ of the monoidal functor $P$. By the same argument
$$\pi(\phi_F):P(I_{\cD})\ra I_{\cC(T)}$$
is the unit preserving isomorphism $\phi^{P}:P(I_{\cD})\ra I_{\cC(T)}$ of $P$. Thus we deduce that for any objects $X,Y$ of $\cC$ we have $m^F_{X,Y} = m^{P}_{X,Y}$ and $\phi^F = \phi^{P}$. This implies that there exists at most one monoidal functor $F$ such that $P = \pi\cdot F$ as monoidal functors.\\
On the other hand define $m^F_{X,Y} = m^{P}_{X,Y}$ for objects $X,Y$ in $\cC$ and $\phi^F = \phi^{P}$. We check now that $F$ equipped with these data is a monoidal functor. Fix objects $X,Y$ in $\cC$. The square
\begin{center}
\begin{tikzpicture}
[description/.style={fill=white,inner sep=2pt}]
\matrix (m) [matrix of math nodes, row sep=4em, column sep=8em,text height=1.5ex, text depth=0.25ex] 
{T\left(P\left(X\otimes_{\cD}Y\right)\right)  & P\left(X\otimes_{\cC}Y\right)     \\
 T\left(P(X)\otimes_{\cC}P(Y)\right)  & P(X)\otimes_{\cC}P(Y)  \\} ;
\path[->,line width=1.0pt,font=\scriptsize]  
(m-1-1) edge node[above] {$ \tau_{X\otimes_{\cC}Y}  $} (m-1-2)
(m-2-1) edge node[below] {$ \left(\tau_X\otimes_{\cC}\tau_Y\right)\cdot m^{T}_{P(X),P(Y)} $} (m-2-2)
(m-1-1) edge node[left] {$ T\left(m^{P}_{X,Y}\right) $} (m-2-1)
(m-1-2) edge node[right] {$  m^{P}_{X,Y} $} (m-2-2);
\end{tikzpicture}
\end{center}
is commutative due to the fact that $\tau:T\cdot P \Rightarrow P$ is a monoidal natural isomorphisms. This implies that $m^F_{X,Y}$ is a morphism in $\cC(T)$. It follows that $m^F_{X,Y}$ is a natural isomorphism and due to the definition of associativity in $\cC(T)$, we derive its compatibility with $m^F_{X,Y}$. Similarly, since the square
\begin{center}
\begin{tikzpicture}
[description/.style={fill=white,inner sep=2pt}]
\matrix (m) [matrix of math nodes, row sep=3em, column sep=3em,text height=1.5ex, text depth=0.25ex] 
{T\left( P\left(I_{\cD}\right) \right)  & P\left(I_{\cD}\right)     \\
 T\left( I_{\cC} \right)  & I_{\cC}  \\} ;
\path[->,line width=1.0pt,font=\scriptsize]  
(m-1-1) edge node[above] {$ \tau_{I_{\cD}}  $} (m-1-2)
(m-2-1) edge node[below] {$ \phi^{T}  $} (m-2-2)
(m-1-1) edge node[left] {$ T\left(\phi^{P}\right)  $} (m-2-1)
(m-1-2) edge node[right] {$ \phi^{P}  $} (m-2-2);
\end{tikzpicture}
\end{center}
is commutative, we deduce that $\phi^F$ is a morphism in $\cC(T)$. By definition of left and right unit in $\cC(T)$, we derive their compatibility with $\phi^F$. This finishes the verification of the fact that $F$ with $\{m^F_{X,Y}\}_{X,Y\in \cC}$ and $\phi^F$ is a monoidal functor. Definitions of $\{m^F_{X,Y}\}_{X,Y\in \cC}$ and $\phi^F$ show that the identities $P = \pi\cdot F$ holds on the level of monoidal structures. Since the $2$-forgetful functor from $2$-category of monoidal categories into $2$-category of categories is faithful on $2$-cells, the identity $\sigma_F = \tau$ of natural isomorphisms is also the identity of monoidal natural isomorphisms.
\end{proof}

\begin{theorem}\label{theorem:endoscope_colimits}
Let $(\cC,T)$ be a pair consiting of a category and its endofunctor $T:\cC\ra \cC$. Assume that $T$ preserves colomits. Then the following assertions hold.
\begin{enumerate}[label=\textbf{\emph{(\arabic*)}}, leftmargin=3.0em]
\item $\pi:\cC(T)\ra \cC$ creates colimits.
\item Suppose that $\cD$ is a category, $P:\cD\ra \cC$ a functor preserving small colimits and $\tau:T\cdot P \Rightarrow P$ a natural isomorphisms. Then the unique functor $F:\cD \ra \cC(T)$ such that $P = \pi\cdot F$ and $\sigma_F = \tau$ preserves small colimits.
\end{enumerate}
\end{theorem}
\begin{proof}
Let $I$ be a small category and $D:I\ra \cC(T)$ be a diagram such that the composition $\pi\cdot D:I\ra \cC$ admits a colimit given by cocone $(X,\{g_i\}_{i\in I})$. Since $T$ preserves colimits, we derive that $\left(T(X), \{T(u_i)\}_{i\in I}\right)$ is a colimit of $T\cdot \pi \cdot D:I\ra \cC$. Now $\sigma_D:T\cdot \pi\cdot D\ra \pi\cdot D$ is a natural isomorphism. Hence there exists a unique arrow $u:T(X)\ra X$ such that $u \cdot T(g_i) = g_i\cdot \sigma_{D(i)}$ for $i\in I$. Clearly $u$ is an isomorphism and hence $(X,u)$ is an object of $\cC(T)$. Moreover, the family $\{g_i\}_{i\in I}$ together with $(X,u)$ is a colimiting cocone over $D$. This proves \textbf{(1)}. Now \textbf{(2)} is a consequence of \textbf{(1)}.
\end{proof}
\noindent
Now we apply the results above to certain more general diagrams of categories.

\begin{definition}
A diagram
\begin{center}   
\begin{tikzpicture}
[description/.style={fill=white,inner sep=2pt}]
\matrix (m) [matrix of math nodes, row sep=3em, column sep=2em,text height=1.5ex, text depth=0.25ex] 
{... &  \cC_{n+1}  &  \cC_n & ... & \cC_2 & \cC_1 & \cC_0  \\};
\path[->,line width=1.0pt,font=\scriptsize]    
(m-1-1) edge node[auto]  {$F_{n+1}  $} (m-1-2)
(m-1-2) edge node[auto]  {$F_n  $} (m-1-3)
(m-1-3) edge node[auto]  {$F_{n-1}  $} (m-1-4)
(m-1-4) edge node[auto]  {$F_2  $} (m-1-5)
(m-1-5) edge node[auto]  {$F_1  $} (m-1-6)
(m-1-6) edge node[auto]  {$F_0  $} (m-1-7);
\end{tikzpicture}
\end{center}
of categories and functors is called \textit{a telescope of categories}.
\end{definition}

\begin{definition}
Let 
\begin{center}   
\begin{tikzpicture}
[description/.style={fill=white,inner sep=2pt}]
\matrix (m) [matrix of math nodes, row sep=3em, column sep=2em,text height=1.5ex, text depth=0.25ex] 
{... &  \cC_{n+1}  &  \cC_n & ... & \cC_2 & \cC_1 & \cC_0  \\};
\path[->,line width=1.0pt,font=\scriptsize]    
(m-1-1) edge node[auto]  {$F_{n+1}  $} (m-1-2)
(m-1-2) edge node[auto]  {$F_n  $} (m-1-3)
(m-1-3) edge node[auto]  {$F_{n-1}  $} (m-1-4)
(m-1-4) edge node[auto]  {$F_2  $} (m-1-5)
(m-1-5) edge node[auto]  {$F_1  $} (m-1-6)
(m-1-6) edge node[auto]  {$F_0  $} (m-1-7);
\end{tikzpicture}
\end{center}
be a telescope of monoidal categories and monoidal (finitely) cocontinuous functors. Then \textit{a $2$-categorical limit of the telescope} consists of a monoidal category $\cC$, a family of monoidal (finitely) cocontinuous functors $\{\pi_n:\cC\ra \cC_n\}_{n\in \NN}$ and a family of monoidal natural isomorphisms $\{\sigma_n:F_{n+1} \cdot \pi_{n+1}\Rightarrow \pi_n\}_{n\in \NN}$ such that the following universal property holds. For any monoidal category $\cD$, family $\{P_n:\cD\ra \cC_n\}_{n\in \NN}$ of (finitely) cocontinuous monoidal functors and a family $\{\tau_n:F_nP_{n+1}\Rightarrow P_{n}\}_{n\in \NN}$ of monoidal natural isomorphisms there exists a unique monoidal (finitely) cocontinuous functor $F:\cD\ra \cC$ satisfying $P_n = \pi_n \cdot F$ and $\left(\sigma_n\right)_F = \tau_n$ for every $n\in \NN$.
\end{definition}

\begin{corollary}\label{corollary:telescope_2_limits}
Let 
\begin{center}   
\begin{tikzpicture}
[description/.style={fill=white,inner sep=2pt}]
\matrix (m) [matrix of math nodes, row sep=3em, column sep=2em,text height=1.5ex, text depth=0.25ex] 
{... &  \cC_{n+1}  &  \cC_n & ... & \cC_2 & \cC_1 & \cC_0  \\};
\path[->,line width=1.0pt,font=\scriptsize]    
(m-1-1) edge node[auto]  {$F_{n+1}  $} (m-1-2)
(m-1-2) edge node[auto]  {$F_n  $} (m-1-3)
(m-1-3) edge node[auto]  {$F_{n-1}  $} (m-1-4)
(m-1-4) edge node[auto]  {$F_2  $} (m-1-5)
(m-1-5) edge node[auto]  {$F_1  $} (m-1-6)
(m-1-6) edge node[auto]  {$F_0  $} (m-1-7);
\end{tikzpicture}
\end{center}
be a telescope of monoidal categories and monoidal (finitely) cocontinuous functors. Then its $2$-limit exists.
\end{corollary}
\begin{proof}
We decompose the task of constructing its $2$-limit as follows. First note that one may form a product $\cC = \prod_{n\in \NN}\cC_n$. Next the functors $\{F_n\}_{n\in \NN}$ induce an endofunctor $T = \prod_{n\in \NN}F_n\times t$, where $\bd{1}$ is the terminal category (it has single object and single identity arrow) and $t:\cC_0\rightarrow \bd{1}$ is the unique functor. Consider the category $\cC(T)$. We define $\{\pi_n:\cC(T)\ra \cC_n\}_{n\in \NN}$ to be a family of functors given by coordinates of $\pi:\cC(T)\ra \cC$ and $\{\sigma_n:F_n\cdot \pi_{n+1}\Rightarrow \pi_n\}_{n\in \NN}$ to be a family of natural isomorphisms given by coordinates of $\sigma:\pi\cdot T\Rightarrow \pi$. Now this data form a $2$-limit of the telescope by compilation of Theorem \ref{theorem:endoscope_monoidal_2_limit} and Theorem \ref{theorem:endoscope_colimits}.
\end{proof}

\section{Formal $\bd{M}$-schemes}
\noindent
This section is devoted to introducing some notions from formal geometry that play a fundamental role in these notes. 

\begin{definition}
Let $\bd{M}$ be a monoid $k$-scheme. \textit{A formal $\bd{M}$-scheme} consists of a sequence $\cZ = \{Z_n\}_{n\in \NN}$ of $\bd{M}$-schemes together with $\bd{M}$-equivariant closed immersions
\begin{center}
\begin{tikzpicture}
[description/.style={fill=white,inner sep=2pt}]
\matrix (m) [matrix of math nodes, row sep=3em, column sep=3em,text height=1.5ex, text depth=0.25ex] 
{ Z_0 &  Z_1 & ... & Z_n & Z_{n+1} & ... \\} ;
\path[right hook->,line width=1.0pt,font=\scriptsize]  
(m-1-1) edge node[above] {$ $} (m-1-2)
(m-1-2) edge node[above] {$ $} (m-1-3)
(m-1-3) edge node[above] {$ $} (m-1-4)
(m-1-4) edge node[above] {$ $} (m-1-5)
(m-1-5) edge node[above] {$ $} (m-1-6);
\end{tikzpicture}
\end{center}
satisfying the following assertions.
\begin{enumerate}[label=\textbf{(\arabic*)}, leftmargin=3.0em]
\item We have $Z_0 = Z_n^{\bd{M}}$ scheme-theoretically for every $n\in \NN$.
\item Let $\cI_n$ be an ideal of $\cO_{Z_n}$ defining $Z_0$. Then for every $m\leq n$ the subscheme $Z_m \subset Z_n$ is defined by $\cI_n^{m+1}$.
\end{enumerate}
\end{definition}

\begin{example}\label{example:formal_neighborhood_of_fixed_pts}
Let $\bd{M}$ be a monoid $k$-scheme and let $Z$ be a $\bd{M}$-scheme. Consider a quasi-coherent ideal $\cI$ of fixed point subscheme $Z^{\bd{M}}$ of $Z$. Then for every $n\in \NN$ ideal $\cI^n$ is $\bd{M}$-equivariant and hence
\begin{center}
\begin{tikzpicture}
[description/.style={fill=white,inner sep=2pt}]
\matrix (m) [matrix of math nodes, row sep=3em, column sep=3em,text height=1.5ex, text depth=0.25ex] 
{ V(\cI) &  V(\cI^2) & ... & V(\cI^n) & ... \\} ;
\path[right hook->,line width=1.0pt,font=\scriptsize]  
(m-1-1) edge node[above] {$ $} (m-1-2)
(m-1-2) edge node[above] {$ $} (m-1-3)
(m-1-3) edge node[above] {$ $} (m-1-4)
(m-1-4) edge node[above] {$ $} (m-1-5);
\end{tikzpicture}
\end{center}
is a formal $\bd{M}$-scheme. We denote it by $\widehat{Z}$.
\end{example}

\begin{definition}
Let $\bd{M}$ be a monoid $k$-scheme and let $\cZ = \{Z_n\}_{n\in \NN}$ be a formal $\bd{M}$-scheme. We say that $\cZ$ is \textit{locally noetherian} if for all $n\in \NN$ scheme $Z_n$ is locally Noetherian.
\end{definition}

\begin{definition}
Let $\bd{M}$ be a monoid $k$-scheme. Suppose that $\cZ = \{Z_n\}_{n\in \NN}$ and $\cW = \{W_n\}_{n\in \NN}$ are formal $\bd{M}$-schemes. Then \textit{a morphism $f:\cZ\ra \cW$ of formal $\bd{M}$-schemes} consists of a family of $\bd{M}$-equivariant morphisms $f = \big\{f_n:Z_n\ra W_n\}_{n\in \NN}$ such that the diagram
\begin{center}
\begin{tikzpicture}
[description/.style={fill=white,inner sep=2pt}]
\matrix (m) [matrix of math nodes, row sep=3em, column sep=3em,text height=1.5ex, text depth=0.25ex] 
{ Z_0 &  Z_1 & ... & Z_n & Z_{n+1} & ... \\
 W_0 &  W_1 & ... & W_n & W_{n+1} & ... \\} ;
\path[right hook->,line width=1.0pt,font=\scriptsize]  
(m-1-1) edge node[above] {$ $} (m-1-2)
(m-1-2) edge node[above] {$ $} (m-1-3)
(m-1-3) edge node[above] {$ $} (m-1-4)
(m-1-4) edge node[above] {$ $} (m-1-5)
(m-1-5) edge node[above] {$ $} (m-1-6)
(m-2-1) edge node[above] {$ $} (m-2-2)
(m-2-2) edge node[above] {$ $} (m-2-3)
(m-2-3) edge node[above] {$ $} (m-2-4)
(m-2-4) edge node[above] {$ $} (m-2-5)
(m-2-5) edge node[above] {$ $} (m-2-6);
\path[->,line width=1.0pt,font=\scriptsize]
(m-1-1) edge node[left] {$f_0 $} (m-2-1)
(m-1-2) edge node[left] {$f_1 $} (m-2-2)
(m-1-4) edge node[left] {$f_n $} (m-2-4)
(m-1-5) edge node[left] {$f_{n+1} $} (m-2-5);
\end{tikzpicture}
\end{center}
is commutative.
\end{definition}

\begin{definition}
Let $\bd{M}$ be a monoid $k$-scheme. Let $\cZ = \{Z_n\}_{n\in \mathbb{N}}$ be locally noetherian a formal $\bd{M}$-scheme. Then we have the corresponding telescope of monoidal categories
\begin{center}   
\begin{tikzpicture}
[description/.style={fill=white,inner sep=2pt}]
\matrix (m) [matrix of math nodes, row sep=3em, column sep=2em,text height=1.5ex, text depth=0.25ex] 
{... &  \Coh_{\bd{M}}(Z_{n+1})  &  \Coh_{\bd{M}}(Z_n) & ... & \Coh_{\bd{M}}(Z_2) & \Coh_{\bd{M}}(Z_1) & \Coh_{\bd{M}}(Z_0)  \\};
\path[->,line width=1.0pt,font=\scriptsize]    
(m-1-1) edge node[auto]  {$  $} (m-1-2)
(m-1-2) edge node[auto]  {$  $} (m-1-3)
(m-1-3) edge node[auto]  {$  $} (m-1-4)
(m-1-4) edge node[auto]  {$  $} (m-1-5)
(m-1-5) edge node[auto]  {$  $} (m-1-6)
(m-1-6) edge node[auto]  {$  $} (m-1-7);
\end{tikzpicture}
\end{center}
and finitely cocontinuous monoidal functors given by restricting $\bd{M}$-equivariant coherent sheaves to closed $\bd{M}$-subschemes. Then we define \textit{a category $\Coh_{\bd{M}}(\cZ)$ of coherent $\bd{M}$-equivariant sheaves on $\cZ$} as a monoidal category which is a $2$-limit of the telescope above. This category is defined uniquely up to a monoidal equivalence.
\end{definition}
\noindent
Fix now a monoid $k$-scheme $\bd{M}$. Let $Z$ be a locally noetherian $\bd{M}$-scheme and suppose that $Z^{\bd{M}}$ exists. Suppose that $\cI$ is a coherent ideal of $Z^{\bd{M}}$. We have a commutative diagram
\begin{center}
\begin{tikzpicture}
[description/.style={fill=white,inner sep=2pt}]
\matrix (m) [matrix of math nodes, row sep=3em, column sep=3em,text height=1.5ex, text depth=0.25ex] 
{ V(\cI) &  V(\cI^2) & ... & V(\cI^n) & ... \\
         &           & Z   &          &      \\} ;
\path[right hook->,line width=1.0pt,font=\scriptsize]  
(m-1-1) edge node[above] {$ $} (m-1-2)
(m-1-2) edge node[above] {$ $} (m-1-3)
(m-1-3) edge node[above] {$ $} (m-1-4)
(m-1-4) edge node[above] {$ $} (m-1-5)
(m-1-2) edge node[above] {$ $} (m-2-3);
\path[right hook->,bend right, line width=1.0pt,font=\scriptsize]  
(m-1-1) edge node[above] {$ $} (m-2-3);
\path[right hook->, line width=1.0pt,font=\scriptsize]  
(m-1-4) edge node[above] {$ $} (m-2-3);
\end{tikzpicture}
\end{center}
in the category of $\bd{M}$-schemes. Thus restriction functors $\Coh_{\bd{M}}(Z) \ra \Coh_{\bd{M}}(V(\cI^n))$ for $n\in \NN$ induce a unique finitely cocontinuous monoidal functor $\Coh_{\bd{M}}(Z)\ra \Coh_{\bd{M}}(\widehat{Z})$.

\begin{definition}
Let $Z$ be a locally noetherian $\bd{M}$-scheme such that $Z^{\bd{M}}$ exists. Then a unique finitely cocontinuous monoidal functor $\Coh_{\bd{M}}(Z)\ra \Coh_{\bd{M}}(\widehat{Z})$ is called \textit{the comparison functor}.
\end{definition}
\noindent
Since group $k$-scheme is also a monoid $k$-scheme, definitions above can be applied to group $k$-schemes.

\begin{definition}
Let $\bd{M}$ be a monoid $k$-scheme with group of units $\bd{G}$. Let $\cZ = \{Z_n\}_{n\in \mathbb{N}}$ be a locally noetherian formal $\bd{M}$-scheme. A locally noetherian $\bd{M}$-scheme $Z$ is called \textit{an algebraization of $\cZ$} if the following two conditions are satisfied.
\begin{enumerate}[label=\textbf{(\arabic*)}, leftmargin=3.0em]
\item $\cZ$ is isomorphic to $\widehat{Z}$ in the category of formal $\bd{M}$-schemes.
\item The comparison functor $\Coh_{\bd{G}}(Z)\ra \Coh_{\bd{G}}\left(\widehat{Z}\right)$ is an equivalence of monoidal categories.
\end{enumerate}
\end{definition}

\section{Locally linear $\bd{M}$-schemes}

\begin{definition}
Let $\bd{M}$ be a monoid $k$-scheme and let $X$ be a $\bd{M}$-scheme. Suppose that each point of $X$ admits an open affine $\bd{M}$-stable neighborhood. Then we say that $X$ is \textit{a locally linear $\bd{M}$-scheme}.
\end{definition}

\begin{proposition}\label{proposition:monoid_open_stable_correspondence}
Let $\bd{M}$ be a monoid $k$-scheme and let $X$ be a $\bd{M}$-scheme. Suppose that $Z$ is a closed $\bd{M}$-stable subscheme of $X$ defined by the ideal with nilpotent sections. Consider an open subset $U$ of $X$. Then the following are equivalent.
\begin{enumerate}[label=\emph{\textbf{(\arabic*)}}, leftmargin=3.0em]
\item $U$ is $\bd{M}$-stable.
\item Scheme-theoretic intersection $U\cap Z$ is $\bd{M}$-stable.
\end{enumerate}
\end{proposition}
\begin{proof}
Let $\alpha:\bd{M} \times_k X\ra X$ be the action of $\bd{M}$ on $X$. Fix open subset $U$ of $X$. If $U$ is $\bd{M}$-stable, then $U\cap Z$ is $\bd{M}$-stable. So suppose that $U\cap Z$ is $\bd{M}$-stable. Since ideal of $Z$ has nilpotent sections and $\bd{M}$ is affine, we derive that closed immersions $U\cap Z\hookrightarrow U$ and $\bd{M}\times_k \left(U\cap Z\right) \hookrightarrow \bd{M}\times_k U$ induce homeomorphisms on topological spaces. Consider the commutative diagram
\begin{center}
\begin{tikzpicture}
[description/.style={fill=white,inner sep=2pt}]
\matrix (m) [matrix of math nodes, row sep=3em, column sep=3em,text height=1.5ex, text depth=0.25ex] 
{ \bd{M}\times_k U & X     \\
  \bd{M}\times_k \left(U\cap Z\right) & U\cap Z   \\} ;
\path[->,line width=1.0pt,font=\scriptsize]  
(m-1-1) edge node[above] {$ \alpha_{\mid U\cap Z}  $} (m-1-2)
(m-2-1) edge node[below] {$   $} (m-2-2);
\path[right hook->,line width=1.0pt,font=\scriptsize]  
(m-2-1) edge node[left] {$  $} (m-1-1)
(m-2-2) edge node[right] {$ $} (m-1-2);
\end{tikzpicture}
\end{center}
where the bottom horizontal arrow is the induced action on $U\cap Z$ and vertical morphisms are homeomorphisms. The commutativity of the diagram implies that $\alpha\left(\bd{M}\times_k U\right)$ is contained set-theoretically in $U$. Since $U$ is open in $X$, we derive that morphism of schemes $\alpha_{\mid \bd{M}\times_k U}$ factors through $U$. Hence $U$ is $\bd{M}$-stable.
\end{proof}

\begin{corollary}\label{corollary:monoid_stable_open_affine_correspondence}
Let $\bd{M}$ be a monoid $k$-scheme and let $X$ be a $\bd{M}$-scheme. Suppose that $Z$ is a closed $\bd{M}$-stable subscheme of $X$ defined by the nilpotent ideal. Consider an open subset $U$ of $X$. Then the following are equivalent.
\begin{enumerate}[label=\emph{\textbf{(\arabic*)}}, leftmargin=3.0em]
\item $U$ is $\bd{M}$-stable and affine.
\item $U\cap Z$ is $\bd{M}$-stable and affine.
\end{enumerate}
\end{corollary}
\begin{proof}
Since ideal of $Z$ is nilpotent, we derive that $U$ is affine if and only if $U\cap Z$ is affine. Combining this with Proposition \ref{proposition:monoid_open_stable_correspondence}, we deduce the result.
\end{proof}

\begin{corollary}\label{corollary:locally_linear_are_stable_under_thickenings}
Let $\bd{M}$ be a monoid $k$-scheme and let $X$ be a $\bd{M}$-scheme. Suppose that $Z$ is a closed $\bd{M}$-stable subscheme of $X$ defined by the nilpotent ideal. Then $X$ is locally linear $\bd{M}$-scheme if and only if $Z$ is locally linear $\bd{M}$-scheme.
\end{corollary}
\begin{proof}
This is a consequence of Corollary \ref{corollary:monoid_stable_open_affine_correspondence}.
\end{proof}

\section{Some results on formal $\bd{M}$-schemes}

\begin{corollary}\label{corollary:each_formal_scheme_consists_of_locally_linear_schemes_if_group_is_affine}
Let $\bd{M}$ be an affine monoid $k$-scheme and let $\cZ = \{Z_n\}_{n\in \NN}$ be a formal $\bd{G}$-scheme. Then $Z_n$ is locally linear $\bd{G}$-scheme for every $n\in \NN$.
\end{corollary}
\begin{proof}
Let $\cI_n$ be an ideal defining $Z_0$ in $Z_n$. Since $\cZ$ is a formal $\bd{M}$-scheme, we derive that $\cI_n^{n+1} = 0$ and $Z_0$ is locally linear $\bd{M}$-scheme. Thus we apply Corollary \ref{corollary:locally_linear_are_stable_under_thickenings} and derive that $Z_n$ is locally linear $\bd{M}$-scheme.
\end{proof}
\noindent
We are particularly interested in formal $\bd{M}$-schemes for monoid $\bd{M}$ with zero. For this we need the following elementary result.

\begin{proposition}\label{proposition:retraction_for_monoids_with_zero}
Let $\bd{M}$ be a monoid $k$-scheme with zero $\bd{o}$ and let $X$ be a $\bd{M}$-scheme. Then the following results hold.
\begin{enumerate}[label=\emph{\textbf{(\arabic*)}}, leftmargin=3.0em]
\item The multiplication by zero $\bd{o}\cdot(-):X\ra X$ factors through $X^{\bd{M}}$ inducing a $\bd{M}$-equivariant retraction $\pi_{\bd{M}}:X\twoheadrightarrow X^{\bd{M}}$.
\item If $\bd{N}$ is a submonoid $k$-scheme of $\bd{M}$ and $\bd{o}$ is a $k$-point of $\bd{N}$, then $\pi_{\bd{M}} = \pi_{\bd{N}}$.
\item If $\bd{M}$ is affine and $X$ is locally linear $\bd{M}$-scheme, then $\pi_{\bd{M}}$ is affine.
\end{enumerate}
\end{proposition}
\begin{proof}
The multiplication $\bd{o}\cdot (-):\fP_X\ra \fP_X$ factors as an $\fP_{\bd{M}}$-equivariant epimorphism $\fP_X\twoheadrightarrow \fP_{X^{\bd{M}}}$ composed with a closed immersion $\fP_{X^{\bd{M}}}\hookrightarrow \fP_X$. The $\fP_{\bd{M}}$-equivariant epimorphism $\fP_X\ra \fP_{X^{\bd{M}}}$ corresponds to a $\bd{M}$-equivariant morphism $\pi_{\bd{M}}:X\ra X^{\bd{M}}$ of $k$-schemes such that $\pi_{\bd{M}}$ restricted to $X^{\bd{M}}$ is the identity $1_{X^{\bd{M}}}$. This proves \textbf{(1)}.\\
For the proof of \textbf{(2)} note that $\bd{o}\cdot (-):\fP_X\ra \fP_X$ is defined similarly for $\bd{M}$ and $\bd{N}$ (provided that $\bd{o}$ is a $k$-point of $\bd{N}$). Thus $\pi_{\bd{M}} = \pi_{\bd{N}}$.\\
Suppose now that $\bd{M}$ is affine and $X$ is locally linear $\bd{M}$-scheme. Consider the action $\alpha:\bd{M}\times_k X\ra X$ of $\bd{M}$ on $X$. Since $X$ is locally linear and $\bd{M}$ is affine, we derive that $\alpha$ is an affine morphism of $k$-schemes. Now $\bd{o}\cdot (-):X\ra X$ is given as a composition 
\begin{center}
\begin{tikzpicture}
[description/.style={fill=white,inner sep=2pt}]
\matrix (m) [matrix of math nodes, row sep=3em, column sep=3em,text height=1.5ex, text depth=0.25ex] 
{X & \bd{o}\times_k X & \bd{M}\times_k X & X \\} ;
\path[->,line width=1.0pt,font=\scriptsize]  
(m-1-1) edge node[above] {$\cong$} (m-1-2)
(m-1-3) edge node[above] {$\alpha$} (m-1-4);
\path[right hook->,line width=1.0pt,font=\scriptsize]  
(m-1-2) edge node[above] {$ $} (m-1-3);
\end{tikzpicture}
\end{center}
The morphism above is affine (as a composition of affine morphisms). Since the composition of $\pi$ with a closed immersion $X^{\bd{M}}\hookrightarrow X$ is $\bd{o}\times_k(-)$ and hence an affine morphism, we derive that $\pi$ is affine. This proves \textbf{(3)}.
\end{proof}
\noindent
Let us note the immediate consequence of this result.

\begin{corollary}\label{corollary:restraction_for_formal_schemes_and_pointed_submonoids}
Let $\bd{M}$ be an affine monoid $k$-scheme with zero and let $\cZ = \{Z_n\}_{n\in \NN}$ be a formal $\bd{M}$-scheme. Then $\cZ$ is a part of the commutative diagram
\begin{center}
\begin{tikzpicture}
[description/.style={fill=white,inner sep=2pt}]
\matrix (m) [matrix of math nodes, row sep=2em, column sep=3em,text height=1.5ex, text depth=0.25ex] 
{ Z_0 &  Z_1 & ... & Z_n & ... \\
      &      & Z_0 &     &  \\} ;
\path[right hook->,line width=1.0pt,font=\scriptsize]  
(m-1-1) edge node[above] {$ $} (m-1-2)
(m-1-2) edge node[above] {$ $} (m-1-3)
(m-1-3) edge node[above] {$ $} (m-1-4)
(m-1-4) edge node[above] {$ $} (m-1-5);
\path[->>,line width=1.0pt,font=\scriptsize]
(m-1-1) edge[bend right = 20] node[below] {$\pi_0 = 1_{Z_0} $} (m-2-3)
(m-1-2) edge node[above] {$ \pi_1 $} (m-2-3)
(m-1-4) edge[bend left = 20] node[above] {$ \pi_n $} (m-2-3);
\end{tikzpicture}
\end{center}
in which vertical morphisms $\pi_n:Z_n\twoheadrightarrow Z_0$ are affine $\bd{M}$-equivariant morphisms such that ${\pi_n}_{\mid Z_0} = 1_{Z_0}$. Moreover, if $\bd{N}$ is a submonoid $k$-scheme of $\bd{M}$ containing the zero of $\bd{M}$, then $\cZ$ is a formal $\bd{N}$-scheme.
\end{corollary}
\begin{proof}
This is an immediate consequence of Corollary \ref{corollary:each_formal_scheme_consists_of_locally_linear_schemes_if_group_is_affine} and Proposition \ref{proposition:retraction_for_monoids_with_zero}.
\end{proof}

\section{Toruses and toric monoid $k$-schemes}

\begin{definition}
Let $T$ be an affine algebraic group over $k$. Suppose that there exists $n\in \NN$ such that for every algebraically closed extension $K$ of $k$ there exists an isomorphism
$$T_K \cong  \Spec K \times_k \underbrace{\mathbb{G}_{m}\times_k \mathbb{G}_{m}\times_k ...\times_k \mathbb{G}_{m}}_{n\,\mathrm{times}} $$
of group schemes over $K$. Then $T$ is called \textit{a torus over $k$}.
\end{definition}

\begin{example}\label{example:split_torus}
If $T \cong \underbrace{\mathbb{G}_{m}\times_k \mathbb{G}_{m}\times_k ...\times_k \mathbb{G}_{m}}_{n\,\mathrm{times}}$, then $T$ is a torus. We call toruses $T$ of this form \textit{split toruses}.
\end{example}

\begin{example}\label{example:non_split_torus}
Define
$$\bd{S}^1 = \Spec k[x,y]/(x^2+y^2-1)$$
a scheme over $k$ and let $\fP_{\bd{S}^1}$ be its functor of points. Then for every $k$-algebra $A$ we have
$$\fP_{\bd{S}^1}(A) = \big\{(u,v)\in A\times_k A\,\big|\,u^2+v^2=1\big\}$$
There is also a morphism $\fP_{\bd{S}^1}\times_k \fP_{\bd{S}^1}\ra \fP_{\bd{S}^1}$ of $k$-functors given by
$$\fP_{\bd{S}^1}(A)\times_k \fP_{\bd{S}^1}(A)\ra \fP_{\bd{S}^1}\ni \left((u_1,v_1),(u_2,v_2)\right)\mapsto (u_1u_2-v_1v_2,u_1v_2+u_2v_1)\in \fP_{\bd{S}^1}(A)$$
for every $k$-algebra $A$. This makes $\fP_{\bd{S}^1}$ into a group $k$-functor. Thus $\bd{S}^1$ with the group structure described above is an affine algebraic group over $k$. We call it \textit{the circle group over $k$}.\\
Now suppose that $\mathrm{char}(k) \neq 2$ and $K$ is an algebraically closed extension of $k$. Consider an element $i\in K$ such that $i^2 = -1$. For every $K$-algebra $A$ we have a map
$$\fP_{\bd{S}^1}(A)\ni (u,v)\mapsto u+iv\in A^*$$
First note that this map is bijective. Indeed, its inverse is given by
$$A^*\ni a \mapsto \left(\frac{1}{2}(a+a^{-1}),\frac{1}{2i}(a-a^{-1})\right) \in \fP_{\bd{S}^1}(A)$$
Moreover, the map $\fP_{\bd{S}^1}(A)\ra A^*$ is a homomorphism of abstract groups. Thus $\fP_{\bd{S}^1}$ resricted to the category $\Alg_K$ of $K$-algebras is isomorphic with $\fP_{\Spec K\times_k \mathbb{G}_{m}}$ as a group $k$-functor. Hence
$$\bd{S}^1_K \cong \Spec K\times_k \mathbb{G}_{m}$$
as algebraic group schemes over $K$. Hence $\bd{S}^1$ is a torus over $k$.\\
Now assume that $k = \RR$. Then abstract groups
$$\fP_{\bd{S}^1}(\RR) = \big\{z\in \CC\,\big|\,|z|=1\big\} \subseteq \CC^*,\,\RR^*$$
are not isomorphic. Indeed, the left hand side group has infinite torsion subgroup and the right hand side group has torsion subgroup equal to $\{-1,1\}$. This implies that over $\RR$ algebraic groups $\bd{S}^1$ and $\mathbb{G}_{m}$ are not isomorphic. Hence $\bd{S}^1$ is not a split torus over $\RR$.
\end{example}

\begin{corollary}\label{corollary:toruses_are_linearly_reductive}
Let $T$ be a torus over $k$. Then $T$ is a linearly reductive algebraic group.
\end{corollary}

\begin{definition}
Let $T$ be a torus over $k$ and let $\ol{T}$ be a linearly reductive monoid having $T$ as the group of units. Then $\ol{T}$ is \textit{a toric monoid over $k$}
\end{definition}

\begin{theorem}\label{theorem:toric_monoids_properties_Kempf_torus}
Let $\ol{T}$ be a toric monoid over $k$ with group of units $T$ and let $K$ be an algebraically closed extension of $k$. Suppose that $N$ is a dimension of $T$.
\begin{enumerate}[label=\emph{\textbf{(\arabic*)}}, leftmargin=3.0em]
\item The group of characters of $T_K$ is isomorphic to $\ZZ^N$ and there exists an abstract submonoid $S$ of $\ZZ^N$ such that the open immersion
$$T_K = \Spec\left(\bigoplus_{m\in \ZZ^N}K\cdot \chi^m\right) \hookrightarrow \Spec\left(\bigoplus_{m\in S}K\cdot \chi^m\right) = \ol{T}_K$$
is induced by the inclusion $S\hookrightarrow \ZZ^N$.
\item Let $\{V_{\lambda}\}_{\lambda\in \bd{Irr}(T)}$ be a set of irreducible representation of $T$ such that $V_{\lambda}$ is in isomorphism class $\lambda$. For every $\lambda$ there exists a finite subset $A_{\lambda}$ of $\ZZ^N$ such that
$$K\otimes_kV_{\lambda} = \bigoplus_{m\in A_{\lambda}}K\cdot \chi^m$$
If $\lambda$ is in  $\bd{Irr}(\ol{T})$, then $A_{\lambda}$ is a subset of $S$. Moreover, we have
$$\ZZ^N = \coprod_{\lambda\in \bd{Irr}(T)}A_{\lambda}$$
and $A_{\lambda_0} = \{0\}$, where $\lambda_0$ is the class of the trivial representation of $T$.
\item If $\ol{T}$ has a zero, then there exists a homomorphism $f:\ZZ^N\ra \ZZ$ of abelian groups such that $f_{\mid S\setminus \{0\}}>0$. In particular, $f$ induces a closed immersion
$$\Spec K\times_k \mathbb{G}_{m} = \Spec K[\ZZ]\hookrightarrow \Spec \left(\bigoplus_{m\in \ZZ^N}K\cdot \chi^m\right) = T_K$$
of group $K$-schemes that extends to a zero preserving closed immersion $\mathbb{A}^1_K\hookrightarrow \ol{T}_K$ of monoid $K$-schemes.
\end{enumerate}
\end{theorem}
\begin{proof}
Since $T$ is a torus, we derive that
$$T_K = \Spec K \times_k \underbrace{\mathbb{G}_{m}\times_k \mathbb{G}_{m}\times_k ...\times_k \mathbb{G}_{m}}_{N\,\mathrm{times}} = \Spec \left(\bigoplus_{m\in \ZZ^N}K\cdot \chi^m\right)$$
and hence
$$\ol{T}_K = \Spec \left(\bigoplus_{s\in S}K\cdot \chi^s\right)$$
for some abstract submonoid $S$ of $\ZZ^N$. Moreover, the open immersion $T_K\hookrightarrow \ol{T}_K$ is induced by the inclusion $S \hookrightarrow \ZZ^N$. This proves \textbf{(1)}.\\
We have identification
$$k[T] = \bigoplus_{\lambda\in \bd{Irr}(T)}V_{\lambda}^{n_{\lambda}}$$
of $T$-representations, where $n_{\lambda}\in \NN\setminus \{0\}$ for each $\lambda$. Thus
$$\bigoplus_{m\in \ZZ^N}K\cdot \chi^m = K\otimes_kk[T] = \bigoplus_{\lambda\in \bd{Irr}(T)}\left(K\otimes_kV_{\lambda}\right)^{n_{\lambda}}$$
This implies that $n_{\lambda} = 1$ for every $\lambda$ and moreover, we derive that
$$K\otimes_kV_{\lambda} = \bigoplus_{m\in A_{\lambda}}K\cdot \chi^m$$
for some finite set $A_{\lambda}\subseteq \ZZ^N$. We also have $A_{\lambda_0} = \{0\}$ and $A_{\lambda}\subseteq S\setminus \{0\}$ for $\lambda \in \bd{Irr}(\ol{T})$. This proves \textbf{(2)}.\\
Since $\ol{T}$ admits a zero, we derive that
$$\ideal{m} = \bigoplus_{m \in S\setminus\{0\}}K\cdot \chi^s \subseteq \bigoplus_{m\in \ZZ^N}K\cdot \chi^m$$
is an ideal. This implies that $S\setminus \{0\}$ is closed under addition. In particular, there exists a homomorphism of abelian groups $f:\ZZ^N\ra \ZZ$ such that $f_{\mid S\setminus \{0\}}>0$. This implies \textbf{(3)}.
\end{proof}

\section{Commuting actions}

\begin{proposition}\label{proposition:commuting_action_preserves_isotypic_decomposition}
Let $\fG$ and $\fH$ be monoid $k$-functors. Denote by $\Lambda$ the set of isomorphism classes of irreducible $\fH$-representations. Suppose that $V$ is a representation of both $\fG$ and $\fH$ and assume that their actions on $V$ commute. Assume that $V$ is completely reducible as a $\fH$-representation and consider the decomposition
$$V = \bigoplus_{\lambda\in \Lambda}V[\lambda]$$
onto isotypic components with respect to the action of $\fH$. Then for every $\lambda$ in $\Lambda$ the subspace $V[\lambda]$ is a $\fG$-subrepresentation of $V$.
\end{proposition}
\begin{proof}
Consider morphisms $\rho:\fG\ra \cL_V$ and $\delta:\fH\ra \cL_V$ determining the structure of $V$ as the $\fG$-representation and $\fH$-representation, respectively. Fix $k$-algebra $A$ and $g\in \fG(A)$. Consider $A\otimes_k V$ as a tensor product of $\fH$-representation $V$ with $A$ as a trivial $\fH$-representation. We claim that $\rho(g):A\otimes_kV \ra A\otimes_kV$ is an endomorphism of this $\fH$-representation. For this consider $k$-algebra $B$ and $h\in \fH(B)$. Since actions of $\fG$ and $\fH$ on $V$ commute, we derive that
$$\big(1_B\otimes_k\rho(g)\big)\cdot \big(1_A\otimes_k\delta(h)\big) = \big(1_A\otimes_k\delta(h)\big)\cdot \big(1_B\otimes_k\rho(g)\big)$$
Sonce this holds for every $k$-algebra $B$ and every $h\in \fH(B)$, we deduce that indeed $\rho(g)$ is a $\fH$-endomorphism of $A\otimes_kV$. Next we have
$$\left(A\otimes_kV\right)[\lambda] = A\otimes_k V[\lambda]$$
for every $\lambda \in \Lambda$. Thus
$$\rho(g)\left(A \otimes_k V[\lambda] \right)\subseteq A \otimes_kV[\lambda]$$
for every $\lambda$ in $\Lambda$. This holds for every $k$-algebra $A$ and $g\in \fG(A)$. Hence $V[\lambda]$ is a $\fG$-subrepresentation of $V$.
\end{proof}

\section{Algebraization of formal $\bd{M}$-schemes}
\noindent
This section proves some results in equivariant formal geometry. 

\begin{theorem}\label{theorem:every_formal_over_kempf_monoid_is_formal_neighborhood}
Let $\bd{M}$ be a Kempf monoid and let $\cZ = \{Z_n\}_{n\in \NN}$ be a formal $\bd{M}$-scheme. Then there exists a locally linear $\bd{M}$-scheme $Z$ equipped with an action of $\bd{M}$ such that $\widehat{Z}$ is isomorphic to $\cZ$.
\end{theorem}

\begin{proof}[Setup]
Monoid $\bd{M}$ is affine and admits zero $\bd{o}$. Hence by Corollary \ref{corollary:restraction_for_formal_schemes_and_pointed_submonoids} formal $\bd{M}$-scheme $\cZ$ corresponds to a sequence of surjections
\begin{center}
\begin{tikzpicture}
[description/.style={fill=white,inner sep=2pt}]
\matrix (m) [matrix of math nodes, row sep=3em, column sep=2em,text height=1.5ex, text depth=0.25ex] 
{ ... &  \cA_{n+1} & \cA_n & ...& \cA_1 & \cA_0 = \cO_{Z_0} \\} ;
\path[->>,line width=1.0pt,font=\scriptsize]  
(m-1-1) edge node[above] {$ $} (m-1-2)
(m-1-2) edge node[above] {$ $} (m-1-3)
(m-1-3) edge node[above] {$ $} (m-1-4)
(m-1-4) edge node[above] {$ $} (m-1-5)
(m-1-5) edge node[above] {$ $} (m-1-6);
\end{tikzpicture}
\end{center}
of quasi-coherent $\bd{M}$-algebras on $Z_0$ such that $\cA_n^{\bd{M}} = \cA_0$ for every $n\in \NN$ and if $\cI_n$ is the kernel of $\cA_n\twoheadrightarrow \cA_0$ in $\cA_n$, then $\cI_n^{m+1}$ is the kernel of $\cA_n\twoheadrightarrow \cA_m$ for $m\leq n$ and $n\in \NN$. Since $\bd{M}$ is a Kempf monoid, there exists a closed subgroup $T$ of the center $Z(\bd{G})$ of the unit group $\bd{G}$ of $\bd{M}$ such that $T$ is a torus and the scheme-theoretic closure $\ol{T}$ of $T$ in $\bd{M}$ contains the zero $\bd{o}$ of $\bd{M}$. We derive by Corollary \ref{corollary:restraction_for_formal_schemes_and_pointed_submonoids} that $\cA_n^{\bd{\ol{T}}} = \cA_0$ for every $n\in \NN$. By definition $\ol{T}$ is a toric monoid $k$-scheme with $T$ as a group of units. Let $\{V_{\lambda}\}_{\lambda\in \bd{Irr}(T)}$ be a set of irreducible representations of $T$ such that $V_{\lambda}$ is contained in $\lambda$.
\end{proof}

\begin{lemma}\label{lemma:stablization_for_representations}
Let $\lambda$ be in $\bd{Irr}(T)$. Then there exists $n_{\lambda}\in \NN$ such that for each $n > n_{\lambda}$ and any $\lambda_1,...,\lambda_n\in \bd{Irr}(\ol{T})\setminus \{\lambda_0\}$ the representation
$$\bigotimes_{i=1}^nV_{\lambda_i}$$ 
has trivial isotypic component of type $\lambda$. We have $n_{\lambda_0} = 0$, where $\lambda_0$ is an isomorphism type of the trivial representation of $T$.
\end{lemma}
\begin{proof}[Proof of the lemma]
Let $K$ be an algebraically closed extension of $k$. Pick $A_{\lambda}$ and $f$ as in Theorem \ref{theorem:toric_monoids_properties_Kempf_torus} and define
$$n_{\lambda} = \sup_{m\in A_{\lambda}}f(m)$$
We have
$$K\otimes_kV_{\lambda_1}\otimes_k...\otimes_kV_{\lambda_n} = \bigoplus_{(m_1,...,m_n)\in A_{\lambda_1}\times_k ...\times_k A_{\lambda_n}}K\cdot \chi^{m_1+...+m_n}$$
and since $m_1,...m_n\in A_{\lambda_1}\cup ...\cup A_{\lambda_n}\subseteq S\setminus \{0\}$ we derive that
$$f(m_1+...+m_n) = f(m_1) + ... + f(m_n) \geq n > n_{\lambda} = \sup_{m\in A_{\lambda}}f(m)$$
This implies that isotypic component of $V_{\lambda_1}\otimes_k...\otimes_kV_{\lambda_n}$ corresponding to $\lambda$ is trivial.
\end{proof}

\begin{lemma}\label{lemma:stabilization_for_formal_schemes}
Fix $\lambda$ in $\bd{Irr}(T)$. Then $\cA_{n+1}[\lambda]\twoheadrightarrow \cA_n[\lambda]$ is an isomorphism for $n \geq  n_{\lambda}$.
\end{lemma}
\begin{proof}[Proof of the lemma]
For $\lambda \not \in \bd{Irr}(\ol{T})\setminus \{\lambda_0\}$ we have $\cA_{n+1}[\lambda] = \cA_n[\lambda] = 0$, because $\cA_{n+1}$ and $\cA_n$ are quasi-coherent $\ol{T}$-algebras. Fix $\lambda \in \bd{Irr}(\ol{T})$. Consider an affine open subset $U$ of $Z_0$. By Lemma \ref{lemma:stablization_for_representations} we derive that
$$\underbrace{\bigg(\Gamma\left(U,\cI_{n+1}\right) \otimes_{k} \Gamma\left(U, \cI_{n+1}\right) \otimes_{k}...\otimes_{k}\Gamma\left(U,\cI_{n+1}\right)\bigg)}_{n+1\,\mathrm{times}}[\lambda] = 0$$
for every $n \geq n_{\lambda}$. We have canonical surjection
\begin{center}
\begin{tikzpicture}
[description/.style={fill=white,inner sep=2pt}]
\matrix (m) [matrix of math nodes, row sep=3em, column sep=2em,text height=1.5ex, text depth=0.25ex]
{\underbrace{\bigg(\Gamma\left(U,\cI_{n+1}\right) \otimes_{k} \Gamma\left(U, \cI_{n+1}\right) \otimes_{k}...\otimes_{k}\Gamma\left(U,\cI_{n+1}\right)\bigg)}_{n+1\,\mathrm{times}}  & \Gamma\bigg(U, \underbrace{\left(\cI_{n+1} \otimes_{\cO_{Z_0}} \cI_{n+1} \otimes_{\cO_{Z_0}}...\otimes_{\cO_{Z_0}} \cI_{n+1}\right)}_{n+1\,\mathrm{times}}\bigg)\\};
\path[->>,line width=1.0pt,font=\scriptsize]
(m-1-1) edge node[above] {$ $} (m-1-2);;
\end{tikzpicture}
\end{center}
of $T$-representations. This implies that
$$\underbrace{\left(\cI_{n+1} \otimes_{\cO_{Z_0}} \cI_{n+1} \otimes_{\cO_{Z_0}}...\otimes_{\cO_{Z_0}} \cI_{n+1}\right)}_{n+1\,\mathrm{times}}[\lambda] = 0$$
for every $n \geq n_{\lambda}$. Next the multiplication
\begin{center}
\begin{tikzpicture}
[description/.style={fill=white,inner sep=2pt}]
\matrix (m) [matrix of math nodes, row sep=3em, column sep=2em,text height=1.5ex, text depth=0.25ex]
{ \underbrace{\left(\cI_{n+1} \otimes_{\cO_{Z_0}} \cI_{n+1} \otimes_{\cO_{Z_0}}...\otimes_{\cO_{Z_0}} \cI_{n+1}\right)}_{n+1\,\mathrm{times}}  & \cA_{n+1} \\};
\path[->,line width=1.0pt,font=\scriptsize]
(m-1-1) edge node[above] {$ $} (m-1-2);;
\end{tikzpicture}
\end{center}
is an morphism of quasi-coherent $T$-sheaves with image $\cI_{n+1}^{n+1}$. Thus we derive that $\cI_{n+1}^{n+1}[\lambda]=0$ for $n \geq n_{\lambda}$. Hence the kernel of $\cA_{n+1}[\lambda]\twoheadrightarrow \cA_n[\lambda]$ is trivial.
\end{proof}

\begin{proof}[Proof of Theorem]
According to Proposition \ref{proposition:commuting_action_preserves_isotypic_decomposition} and the fact that $T$ is central in $\bd{M}$ we derive that $\cA_n[\lambda]$ is a quasi-coherent $\bd{M}$-sheaf. For $\lambda\in \bd{Irr}(T)$ we define
$$\cA[\lambda] = \cA_n[\lambda]$$
where $n\geq n_{\lambda}$ as in Lemma \ref{lemma:stabilization_for_formal_schemes}. Note that $\cA[\lambda] = 0$ for $\lambda \not \in \bd{Irr}(\ol{T})$. We set
$$\cA=\bigoplus_{\lambda\in \bd{Irr}(\ol{T})}\cA[\lambda]$$
Clearly $\cA[\lambda_0] = \cA_0 = \cO_{Z_0}$ canonically (where $\lambda_0$ is the trivial $T$-representation), hence $\cA$ is a quasi-coherent $\bd{M}$-sheaf on $Z_0$. Actually $\cA=\lim_{n\in \NN}\cA_n$ in the category of quasi-coherent $\bd{M}$-sheaves on $Z_0$, but this will not be used in argument. We construct the $\cO_{Z_0}$-algebra structure on $\cA$. For this pick $\lambda_1, \lambda_2\in \bd{Irr}(\ol{T})$. Consider the irreducible representations $V_{\lambda_1}$ and $V_{\lambda_1}$ in classes $\lambda_1$ and $\lambda_2$, respectively. Suppose that $\eta_1,...,\eta_s$ are finitely many classes in $\bd{Irr}(\ol{T})$ such that $V_{\lambda_1}\otimes_k V_{\lambda_2}$ can be completely decomposed onto irreducible representation in these classes. Since the image of the multiplication $\cA_n[\lambda_1]\otimes_{\cO_{Z_0}}\cA_n[\lambda_2]\ra \cA_n$ on $\cA_n$ is also the image of a morphism
\begin{center}
\begin{tikzpicture}
[description/.style={fill=white,inner sep=2pt}]
\matrix (m) [matrix of math nodes, row sep=3em, column sep=2em,text height=1.5ex, text depth=0.25ex] 
{\cA_n[\lambda_1]\otimes_{k}\cA_n[\lambda_2]  &\cA_n[\lambda_1]\otimes_{\cO_{Z_0}}\cA_n[\lambda_2]  & \cA_n\\} ;
\path[->>,line width=1.0pt,font=\scriptsize]  
(m-1-1) edge node[above] {$ $} (m-1-2);
\path[->,line width=1.0pt,font=\scriptsize]  
(m-1-2) edge node[above] {$ $} (m-1-3);
\end{tikzpicture}
\end{center}
we deduce that it is contained in $\bigoplus_{i=1}^s\cA_n[\eta_i]$. By Lemma \ref{lemma:stabilization_for_formal_schemes} all these multiplications for $n\geq \sup \{n_{\lambda_1},n_{\lambda_2},n_{\eta_1},...,n_{\eta_s}\}$ can be identified. Now we define
$$\cA[\lambda_1]\otimes_{\cO_{Z_0}} \cA[\lambda_2]\ra  \bigoplus_{i=1}^s\cA[\eta_i]\subseteq \cA$$
as a morphism induced by the multiplication morphism for any $n\geq \sup\{n_{\lambda_1},n_{\lambda_2},n_{\eta_1},...,n_{\eta_s}\}$. This gives an $\cO_{Z_0}$-algebra structure on $\cA$ (so $\cA$ is in fact the limit of $\{\cA_n\}_{n\in \NN}$ is the category of quasi-coherent $\bd{M}$-algebras on $Z_0$). Note that from the description of $\cA$ it follows that for every $n\in \NN$ we have a surjective morphism $p_n:\cA\twoheadrightarrow \cA_n$ of algebras. We denote its kernel by $\cJ_n$ and we put $\cJ = \cJ_0$. We have
$$\cJ=\bigoplus_{\lambda \in \bd{Irr}(\ol{T})\setminus \{\lambda_0\}}\cA[\lambda]$$
Recall that we denote by $\cI_n$ the kernel of $\cA_n\twoheadrightarrow \cA_0=\cO_{Z_0}$ for $n\in \NN$. Then $\cI_n=\cJ/\cJ_n$. Fix $m\in \NN$ and consider $n\in \NN$ such that $n\geq m$. Since $\cZ$ is a formal $\bd{M}$-scheme, the sheaf $\cI_n^{m+1}$ is the kernel of the morphism $\cA_n\twoheadrightarrow \cA_m$. Thus
$$\cJ_m/\cJ_n=\cI_n^{m+1}=(\cJ^{m+1}+\cJ_n)/\cJ_n$$
Both $\cJ_m$ and $\cJ^{m+1}$ are $\bd{Irr}(\ol{T})$-graded by their isotypic $\ol{T}$-components and for given $\lambda\in \bd{Irr}(\ol{T})$ and for $n \geq n_{\lambda}$ the isotypic component $\cJ_n[\lambda]$ is zero by Lemma \ref{lemma:stabilization_for_formal_schemes}. Hence $\cJ_m=\cJ^{m+1}$ for every $m \in \NN$.
We define
$$Z=\Spec_{Z_0}\cA$$
and we denote by $\pi:Z\to Z_0$ the structural morphism. The scheme $Z$ inherits a $\bd{M}$-action from $\cA$. For every $n\in \NN$ the zero-set of $\cJ^{n+1}$ in $\cA$ is a $\bd{M}$-scheme isomorphic to $Z_n = \Spec_{Z_0}\cA_n$. Hence $\cZ$ is isomorphic to $\widehat{Z}$ and this proves the theorem.\\
\end{proof}

\begin{theorem}
Let $\bd{M}$ be a Kempf monoid and let $Z$ be a locally linear $\bd{M}$-scheme. Suppose that $\pi:Z\ra Z^{\bd{M}}$ is the canonical retraction. If the formal $\bd{M}$-scheme $\widehat{Z}$ is locally noetherian, then $\pi:Z\ra Z^{\bd{M}}$ is of finite type.
\end{theorem}
\begin{proof}
Since $\pi$ is affine (Proposition \ref{proposition:retraction_for_monoids_with_zero}), we derive that $\cA = \pi_*\cO_Z$ is a quasi-coherent $\bd{M}$-algebra on $Z^{\bd{M}}$. We denote by $\cJ$ the ideal of $\cA$ that corresponds to the closed immersion $Z^{\bd{M}}\hookrightarrow Z$. We know that the formal $\bd{M}$-scheme
\begin{center}
\begin{tikzpicture}
[description/.style={fill=white,inner sep=2pt}]
\matrix (m) [matrix of math nodes, row sep=3em, column sep=3em,text height=1.5ex, text depth=0.25ex] 
{ Z^{\bd{M}} = \Spec_{Z^{\bd{M}}}\cA/\cJ &  ... &  \Spec_{Z^{\bd{M}}}\cA/\cJ^{n+1} &  \Spec_{Z^{\bd{M}}}\cA/\cJ^{n+2} & ... \\} ;
\path[right hook->,line width=1.0pt,font=\scriptsize]  
(m-1-1) edge node[above] {$ $} (m-1-2)
(m-1-2) edge node[above] {$ $} (m-1-3)
(m-1-3) edge node[above] {$ $} (m-1-4)
(m-1-4) edge node[above] {$ $} (m-1-5);
\end{tikzpicture}
\end{center}
is locally noetherian. Hence $\cJ/\cJ^{n+1}$ is $\cA/\cJ^{n+1}$-module of finite type. Thus $\{\cJ^i/\cJ^{i+1}\}_{1\leq i\leq n}$ are finite type $\cA/\cJ$-modules. The series
$$0 \subseteq \cJ^n/\cJ^{n+1} \subseteq  ... \subseteq \cJ/\cJ^{n+1} \subseteq \cA/\cJ^{n+1}$$
has subquotients that are of finite type over $\cO_{Z^{\bd{M}}} = \cA/\cJ$. This implies that $\cA/\cJ^{n+1}$ is a coherent $\cO_{Z^{\bd{M}}}$-algebra for every $n\in \NN$. The claim that $\pi$ is of finite type is local on $Z^{\bd{M}}$, hence we may assume that $Z^{\bd{M}}$ is quasi-compact. This reduces the question to the noetherian $Z^{\bd{M}}$. The sheaf $\cJ/\cJ^2\subseteq \cA/\cJ^2$ is coherent over $\cO_{Z^{\bd{M}}}$. Since $Z^{\bd{M}}$ is noetherian, there exists coherent $\cO_{Z^{\bd{M}}}$-subsheaf $\cM\subseteq \cJ$ such that the morphism $\cM\twoheadrightarrow \cJ/\cJ^2$ is surjective. Fix an algebraically closed extension $K$ of $k$ and denote
$$\cA_K = K\otimes_k\cA,\cJ_K = K\otimes_k\cJ,\cM_K = K\otimes_k\cM$$
Since $\bd{M}$ is a Kempf monoid and by \textbf{(3)} Theorem \ref{theorem:toric_monoids_properties_Kempf_torus} there exists a closed immersion $\mathbb{A}^1_K\hookrightarrow \bd{M}_K$ of monoid $K$-schemes that preserve zero. This implies that we have $\NN$-grading $\cA_K = \bigoplus_{i\geq 0}\cA_K[i]$ that gives rise to the action of $\mathbb{A}^1_K$. Moreover, by Propostion \ref{proposition:retraction_for_monoids_with_zero} we deduce that
$$\Spec K\times_k Z^{\bd{M}} = \left(\Spec K \times_k Z\right)^{\bd{M}_K} = \left(\Spec K \times_k Z\right)^{\mathbb{A}^1_K}$$
as $K$-schemes. This shows that $\cJ_K = K\otimes_k\cJ = \bigoplus_{i\geq 1} \cA_K[i]$ is an ideal with positive grading. We have surjection $\cM_K\twoheadrightarrow \cJ_K/\cJ_K^2$. By graded version of Nakayama's lemma, the ideal $\cJ_K$  is generated by $\cM_K$. Then by induction on degrees we deduce that $\cA_K$ is generated by $\cM_K$ as a $K\otimes_k\cO_{Z^{\bd{M}}}$-algebra. Thus $1_{\Spec K}\times_k\pi$ is of finite type and by faitfully flat descent also $\pi$ is of finite type.
\end{proof}

\begin{theorem}
Let $\bd{M}$ be a Kempf monoid with group of unit $\bd{G}$ and let $Z$ be a locally linear $\bd{M}$-scheme. Suppose that $\pi:Z\ra Z^{\bd{M}}$ is the canonical retraction. If $Z$ is locally noetherian, then the comparison functor
$$\Coh_{\bd{G}}(Z)\ra \Coh_{\bd{G}}(\widehat{Z})$$
is an equivalence of monoidal categories.
\end{theorem}
\begin{proof}[Setup]
Since $\bd{M}$ is a Kempf torus, there exists a central closed torus $T$ in $\bd{G}$ such that the scheme-theoretic closure $\ol{T}$ of $T$ in $\bd{M}$ contains the zero. As above we note that $\pi$ is affine (Proposition \ref{proposition:retraction_for_monoids_with_zero}) and we pick a quasi-coherent $\bd{M}$--algebra $\cA = \pi_*\cO_Z$ on $Z^{\bd{M}}$. We denote by $\cJ$ the ideal of $\cA$ that corresponds to the closed immersion $Z^{\bd{M}}\hookrightarrow Z$. Then $\cO_{Z^{\bd{M}}} = \cA/\cJ$ and since $\pi$ is a retraction, we derive that $\cA = \cO_{Z^{\bd{M}}}\oplus \cJ$. Next $\widehat{Z}$ is locally noetherian (this follows from the fact that $Z$ is locally noetherian). Hence an object of $\Coh_{\bd{G}}(\widehat{Z})$ corresponds to a sequence of surjections
\begin{center}
\begin{tikzpicture}
[description/.style={fill=white,inner sep=2pt}]
\matrix (m) [matrix of math nodes, row sep=3em, column sep=2em,text height=1.5ex, text depth=0.25ex] 
{ ... &  \cM_{n+1} & \cM_n & ...& \cM_1 & \cM_0 \\} ;
\path[->>,line width=1.0pt,font=\scriptsize]  
(m-1-1) edge node[above] {$ $} (m-1-2)
(m-1-2) edge node[above] {$ $} (m-1-3)
(m-1-3) edge node[above] {$ $} (m-1-4)
(m-1-4) edge node[above] {$ $} (m-1-5)
(m-1-5) edge node[above] {$ $} (m-1-6);
\end{tikzpicture}
\end{center}
of coherent $\bd{G}$-modules on $Z^{\bd{M}}$ such that the following assertions hold.
\begin{enumerate}[label=\textbf{(\arabic*)}, leftmargin=3.0em]
\item For each $n\in \NN$ sheaf $\cM_n$ is a module over $\cA/\cJ^{n+1}$.
\item For each $n\in \NN$ the kernel of the surjection $\cM_{n+1}\twoheadrightarrow \cM_n$ is $\cJ^{n+1}\cM_{n+1}$.
\end{enumerate}
We fix an algebraically closed field $K$ containing $k$. By \textbf{(3)} of Theorem \ref{theorem:toric_monoids_properties_Kempf_torus} there exists a closed immersion $\Spec K\times_k \mathbb{G}_m\hookrightarrow T_K$ of group $K$-schemes that induces zero preserving closed immersion $\mathbb{A}^1_K\hookrightarrow \ol{T}_K$ of monoid $K$-schemes. By Proposition \ref{proposition:retraction_for_monoids_with_zero} we have
$$\Spec K\times_k Z^{\bd{M}} = \left(\Spec K\times_k Z\right)^{\bd{M}_K} = \left(\Spec K\times_k Z\right)^{\mathbb{A}^1_K}$$
This implies that
$$\cA_K = K\otimes_k\cA = \bigoplus_{i\geq 0}\cA_K[i],\,\cJ_K = K\otimes_k\cJ = \bigoplus_{i\geq 1}\cA_K[i]$$
where gradation is induced by the action of $\mathbb{A}^1_K$. For every $n\in \NN$ the action of $\Spec K\times_k \mathbb{G}_m$ on $K\otimes_k\cM_n$ induced by the closed immersion $\Spec K\times_k \mathbb{G}_m \hookrightarrow \ol{T}_K\hookrightarrow \bd{M}_K$ of group $K$-schemes gives rise to a gradation
$$K\otimes_k\cM_n = \bigoplus_{i\in \ZZ}\left(K\otimes_k\cM_n\right)[i]$$
\end{proof}

\begin{lemma}\label{lemma:boundedness_of_gradation_and_two_stabilizations}
The following assertions hold.
\begin{enumerate}[label=\emph{\textbf{(\arabic*)}}, leftmargin=3.0em]
\item There exists $i_0\in \ZZ$ such that for every $n\in \NN$ we have $\left(K\otimes_k\cM_n\right)[i] = 0$ for $i< i_0$.
\item For every $i\in \ZZ$ there exists $n_i\in \NN$ such that for all $n\geq n_i$ the surjection $\left(K\otimes_k\cM_{n+1}\right)[i]\twoheadrightarrow \left(K\otimes_k\cM_n\right)[i]$ is an isomorphisms.
\item For every $\lambda$ in $\bd{Irr}(T)$ there exists $n_{\lambda}\in \NN$ such that for all $n\geq n_{\lambda}$ the surjection $\cM_{n+1}[\lambda]\twoheadrightarrow \cM_n[\lambda]$ is an isomorphisms.
\end{enumerate}
\end{lemma}
\begin{proof}[Proof of the lemma]
Fix $n\in \NN$ and consider the decomposition $K\otimes_k\cM_n = \bigoplus_{i\in \ZZ}\left(K\otimes_k\cM_n\right)[i]$. Since $K\otimes_k\cM_n$ is a coherent $K\otimes_k \cO_{Z^{\bd{M}}}$-module and the decomposition consists of modules over $K\otimes_k\cO_{Z^{\bd{M}}}$, we derive that there are only finitely many $i\in \ZZ$ such that $\left(K\otimes_k\cM_n\right)[i]\neq 0$. Hence we may write $K\otimes_k\cM_n = \bigoplus_{i\geq i_n}\left(K\otimes_k\cM_n\right)[i]$ for some $i_n\in \ZZ$ such that $\left(K\otimes_k\cM_n\right)[i_n] \neq 0$. Moreover, we know that the kernel of the surjection
$$K\otimes_k\cM_{n+1} = \bigoplus_{i\geq i_{n+1}}\left(K\otimes_k\cM_{n+1}\right)[i] \twoheadrightarrow \bigoplus_{i\geq i_n}\left(K\otimes_k\cM_n\right)[i] = K\otimes_k\cM_n$$
is $\cJ_K^{n+1}\left(K\otimes_k\cM_{n+1}\right)$ and hence is contained in $\bigoplus_{i\geq (i_{n+1}+n+1)}\left(K\otimes_k\cM_{n+1}\right)[i]$
This implies that $\left(K\otimes_k\cM_{n}\right)[i]  = \left(K\otimes_k\cM_{n+1}\right)[i]$ for $i_{n+1}\leq i \leq i_{n+1}+n$. In particular, we have $\left(K\otimes_k\cM_{n}\right)[i_{n+1}]  = \left(K\otimes_k\cM_{n+1}\right)[i_{n+1}] \neq 0$ and thus $i_{n+1} \geq i_n$. This shows that $i_n\geq i_0$ for every $n\in \NN$ and \textbf{(1)} is proved. Now the surjection
$$K\otimes_k\cM_{n+1} = \bigoplus_{i\geq i_{0}}\left(K\otimes_k\cM_{n+1}\right)[i] \twoheadrightarrow \bigoplus_{i\geq i_0}\left(K\otimes_k\cM_n\right)[i] = K\otimes_k\cM_n$$
induces an isomorphism for $i$-th graded component, where $i_0 \leq i\leq i_0+n$. Hence for fixed $i\in \ZZ$ there exists $n_i\in \NN$ such that for all $n\geq n_i$ the surjection $\left(K\otimes_k\cM_{n+1}\right)[i] \twoheadrightarrow \left(K\otimes_k\cM_n\right)[i]$ is an isomorphism. Thus we proved \textbf{(2)}. Fix now $\lambda$ in $\bd{Irr}(T)$ and let $V_{\lambda}$ be an irreducible representation in class $\lambda$. There exists finite subset $B_{\lambda}\subseteq \ZZ$ such that for $\left(K\otimes_kV_{\lambda}\right)[i] \neq 0$ if $i\in B_{\lambda}$. Now define $n_{\lambda} = \sup_{i\in B_{\lambda}}n_i$ the surjection $K\otimes_k\cM_{n+1}\twoheadrightarrow K\otimes_k\cM_n$ induces an isomorphism $\left(K\otimes_k\cM_{n+1}\right)[i] \cong \left(K\otimes_k\cM_n\right)[i]$ for every $i$ in $B_{\lambda}$. Thus for $n\geq n_{\lambda}$ the surjection $\cM_{n+1}\twoheadrightarrow \cM_n$ induces an isomorphism $\cM_{n+1}[\lambda]\cong \cM_n[\lambda]$. This completes the proof of \textbf{(3)}.
\end{proof}

\begin{proof}[Proof of the theorem]
For fixed $\lambda$ in $\bd{Irr}(T)$ we define $\cM[\lambda] = \cM_n[\lambda]$ for any $n\geq n_{\lambda}$, where $n_{\lambda}\in \NN$ is as in \textbf{(3)} of Lemma \ref{lemma:boundedness_of_gradation_and_two_stabilizations} (in particular, $\cM[\lambda]$ does not depend on $n\geq n_{\lambda}$). Next we define
$$\cM = \bigoplus_{\lambda\in \bd{Irr}}\cM[\lambda]$$
Since by Proposition \ref{proposition:commuting_action_preserves_isotypic_decomposition} for every $n\in \NN$ and $\lambda \in \bd{Irr}(T)$ sheaf $\cM_n[\lambda]$ admits structure of a $\bd{G}$-sheaf. Therefore, $\cM$ is a quasi-coherent $\bd{G}$-sheaf of $\cO_{Z^{\bd{M}}}$-modules. We now show that $\cM$ admits a canonical structure of $\cA$-module. For this pick $\lambda_1$ and $\lambda_2$ in $\bd{Irr}(T)$. Consider the irreducible representations $V_{\lambda_1}$ and $V_{\lambda_1}$ in classes $\lambda_1$ and $\lambda_2$, respectively. Suppose that $\eta_1,...,\eta_s$ are finitely many classes in $\bd{Irr}(T)$ such that $V_{\lambda_1}\otimes_k V_{\lambda_2}$ can be completely decomposed into irreducible representations contained in classes $\eta_1,...,\eta_s$. Since the image of the multiplication $\cA[\lambda_1]\otimes_{\cO_{Z^{\bd{M}}}}\cM_n[\lambda_2]\ra \cM_n$ is also the image of a morphism
\begin{center}
\begin{tikzpicture}
[description/.style={fill=white,inner sep=2pt}]
\matrix (m) [matrix of math nodes, row sep=3em, column sep=2em,text height=1.5ex, text depth=0.25ex] 
{\cA[\lambda_1]\otimes_{k}\cM_n[\lambda_2]  & \cA[\lambda_1]\otimes_{\cO_{Z^{\bd{M}}}}\cM_n[\lambda_2]  & \cM_n\\} ;
\path[->>,line width=1.0pt,font=\scriptsize]  
(m-1-1) edge node[above] {$ $} (m-1-2);
\path[->,line width=1.0pt,font=\scriptsize]  
(m-1-2) edge node[above] {$ $} (m-1-3);
\end{tikzpicture}
\end{center}
we deduce that it is contained in $\bigoplus_{i=1}^s\cM_n[\eta_i]$. By \textbf{(3)} of Lemma \ref{lemma:boundedness_of_gradation_and_two_stabilizations} all these multiplications for $n\geq \sup \{n_{\lambda_1},n_{\lambda_2},n_{\eta_1},...,n_{\eta_s}\}$ can be identified. Now we define
$$\cA[\lambda_1]\otimes_{\cO_{Z^{\bd{M}}}} \cM[\lambda_2]\ra  \bigoplus_{i=1}^s\cM[\eta_i]\subseteq \cM$$
as a morphism induced by the multiplication morphism for any $n\geq \sup\{n_{\lambda_1},n_{\lambda_2},n_{\eta_1},...,n_{\eta_s}\}$. This gives an $\cA$-module structure on $\cM$. Next we prove that $\cM$ is $\cA$-module of finite type. Denote $K\otimes_k\cM$ by $\cM_K$. Note that the combination of \textbf{(2)} and \textbf{(3)} of Lemma \ref{lemma:boundedness_of_gradation_and_two_stabilizations} show that
$$\cM_K[i] = \left(K\otimes_k\cM_n\right)[i]$$
for $n\geq n_i$. Hence by \textbf{(1)} of Lemma \ref{lemma:boundedness_of_gradation_and_two_stabilizations} we have
$$\bigoplus_{\lambda\in \bd{Irr}(T)}\cM[\lambda]_K = \cM_K = \bigoplus_{i\geq i_0}\cM_K[i]$$
Since each $\cM_n$ is a coherent $\cO_{Z^{\bd{M}}}$-module, we derive that $\cM_K[i]$ is a coherent $K\otimes_k\cO_{Z^{\bd{M}}}$-module for every $i\in \ZZ$. Now we may pick $\lambda_1,...,\lambda_r$ in $\bd{Irr}(T)$ such that we have a surjection
$$\bigoplus_{j=1}^r\cM[\lambda_j]_K \twoheadrightarrow \bigoplus_{i_0\leq i\leq 1}\cM_K[i]$$
induced by the projection $\cM_K = \bigoplus_{i\geq i_0}\cM_K[i]\twoheadrightarrow \bigoplus_{i_0\leq i\leq 1}\cM_K[i]$. Let
$$\cG = \bigoplus_{j=1}^r\cM[\lambda_j]$$
be a $\cO_{Z^{\bd{M}}}$-submodule of $\cM$. Clearly each $\cM[\lambda]$ is a coherent $\cO_{Z^{\bd{M}}}$-module. Hence $\cG$ is a coherent $\cO_{Z^{\bd{M}}}$-module. Since $\cJ_K = \bigoplus_{i\geq 1}\cA_K[i]$, we derive that
$$\cM_K = \sum_{j\geq 1}\cJ_K^j\cdot \cG_K$$
and hence $\cG_K$ generates $\cM_K$ as an $\cA_K$-module. By faithfully flat descent we deduce that $\cG$ generates $\cM$ as an $\cA$-module. Since $\cG$ is a coherent $\cO_{Z^{\bd{M}}}$-module, we derive that $\cM$ is $\cA$-module of finite type. Moreover, by construction of $\cM$ we have $\cM/\cJ^{n+1}\cM = \cM_{n}$ for every $n\in \NN$.\\
All these facts imply that $\cM$ corresponds to a coherent $\bd{G}$-sheaf on $Z$ such that its image under the comparison functor $\Coh_{\bd{G}}(Z)\ra \Coh_{\bd{G}}(\widehat{Z})$ is a coherent $\bd{G}$-sheaf on $\widehat{Z}$ with $\bd{G}$-structure described by $\{\cM_n\}_{n\in \NN}$. Hence the comparison functor is essentially surjective. We now prove that it is full and faithful. For this let
\begin{center}
\begin{tikzpicture}
[description/.style={fill=white,inner sep=2pt}]
\matrix (m) [matrix of math nodes, row sep=3em, column sep=2em,text height=1.5ex, text depth=0.25ex] 
{ ... &  \cN_{n+1} & \cN_n & ...& \cN_1 & \cN_0 \\} ;
\path[->>,line width=1.0pt,font=\scriptsize]  
(m-1-1) edge node[above] {$ $} (m-1-2)
(m-1-2) edge node[above] {$ $} (m-1-3)
(m-1-3) edge node[above] {$ $} (m-1-4)
(m-1-4) edge node[above] {$ $} (m-1-5)
(m-1-5) edge node[above] {$ $} (m-1-6);
\end{tikzpicture}
\end{center}
represents some other object of $\Coh_{\bd{G}}(\widehat{Z})$. As for $\{\cM_n\}_{n\in \NN}$ we can construct finite type $\cA$-module $\cN$ with $\bd{G}$-linearization such that $\cN/\cJ^{n+1}\cN = \cN_n$ for every $n\in \NN$. Pick a morphism $f:\cM\ra \cN$ of $\cA$-modules with $\bd{G}$-linearization. For every $\lambda$ in $\bd{Irr}(T)$ morphism $f[\lambda]:\cM[\lambda]\ra \cN[\lambda]$ is equal (by virtue of constructions of $\cN$ and $\cM$) to a morphism $\left(1_{\cA/\cJ^{n+1}}\otimes_{\cA}f\right)[\lambda]$ for sufficiently large $n\in \NN$. This implies that the comparison functor is full and faithful.
\end{proof}





\small
\bibliographystyle{apalike}
\bibliography{../zzz}

\end{document}
