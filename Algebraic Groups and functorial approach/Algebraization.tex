\input ../pree.tex

\usepackage{todonotes}
\newcommand{\lstodo}[1]{\todo[color=green!40,bordercolor=green,size=\footnotesize]{\textbf{LS TODO: }#1}}

\usetikzlibrary{matrix,arrows, cd, calc}
\newcommand{\Zhat}{\widehat{Z}}
\newcommand{\What}{\widehat{W}}
\newcommand{\Zformal}{\cZ}
\newcommand{\Wformal}{\cW}
\newcommand{\Group}{\mathbf{G}}%
\newcommand{\Groupop}{\mathbf{G}^{\mathrm{op}}}%
\newcommand{\Gprod}{\Group\times \Groupop}%
\newcommand{\Groupconn}{\mathbf{G}^{\circ}}%
\newcommand{\Nroup}{\mathbf{N}}%
\newcommand{\Nred}{\Nroup_{\mathrm{red}}^{\circ}}%
\newcommand{\Gbar}{\overline{\mathbf{G}}}%
\newcommand{\Nbar}{\overline{\mathbf{N}}}%
\newcommand{\Fbar}{\overline{\mathbf{F}}}%
\newcommand{\Ghat}{\widehat{\mathbf{G}}}%

\newcommand{\Gmult}{\mathbb{G}_m}%
%% Glowna rozmaitosc
\newcommand{\kk}{k}%
\newcommand{\kkbar}{\overline{k}}%
\newcommand{\varX}{X}%
%% Kategoria schematów lokalnie skończonego typu nad ciałem
\newcommand{\kSch}{\Sch_{\kk}}%
%% Funktor zbieznych deformacji (z argumentem i bez)
\newcommand{\Dfunctor}[1]{\mathcal{D}_{#1}}%
\newcommand{\DX}{\Dfunctor{\varX}}%
\newcommand{\Xplus}{\varX^+}%
\newcommand{\Yplus}{Y^+}%
\newcommand{\fplus}{f^+}%
%% Funktor punktow stalych (z argumentem i bez)
\newcommand{\Ffunctor}[1]{{#1}^{\Group}}%
\newcommand{\FX}{\Ffunctor{\varX}}%
\newcommand{\MorX}{\Mor_{\kk}(-, \varX)}%
%% Funktor formalnych kompaktyfikacji (z argumentem i bez)
\newcommand{\Hfunctor}[1]{\widehat{\mathcal{D}}_{#1}}%
\newcommand{\HX}{\Hfunctor{\varX}}%
%% Wlozenie ``wlokna nad 1''
\newcommand{\ione}[1]{i_{#1}}%
\newcommand{\ioneX}{\ione{\varX}}
%% Wlozenie ``wlokna nad \infty''
\newcommand{\iinfty}[1]{\pi_{#1}}%
\newcommand{\iinftyX}{\iinfty{\varX}}%
\newcommand{\isection}[1]{s_{#1}}%
\newcommand{\isectionX}{\isection{\varX}}%
%% wlozenie puntkow stalych w X^+
\newcommand{\emb}{e}%
%% ABB, dla skrotu
\newcommand{\BBname}{Bia{\l}ynicki-Birula}%
\newcommand{\Tname}{locally $\Group$-affine}%
%% Stogi
\DeclareMathOperator{\Maps}{Maps}
\DeclareMathOperator{\Irr}{Irr}
\newcommand{\MapsStack}{\underline{\Maps}}%
\DeclareMathOperator{\colim}{colim}


\DeclareMathOperator{\charr}{char}%
\DeclareMathOperator{\id}{id}%
\DeclareMathOperator{\Hilb}{Hilb}%
\newcommand{\into}{\hookrightarrow}%
\newcommand{\onto}{\twoheadrightarrow}%
\newcommand{\sigmabar}{\overline{\sigma}}%

\newcommand{\GroupGlobalSects}{H^0(\Group, \cO_{\Group})}%
\newcommand{\Hbar}{\overline{\mathbf{H}}}%
\newcommand{\Hroup}{\mathbf{H}}%
\newcommand{\mubar}{\overline{\mu}}%

\begin{document}

\title{Algebraization of formal $\bd{M}$-schemes}
\date{}
\maketitle

\section{Some $2$-categorical limits}
\noindent
Consider a category $\cC$ and its endofunctor $T:\cC\ra \cC$. Our goal is to construct certain $2$-categorical limit associated with a pair $(\cC,T)$. Consider pairs $\left(X,u\right)$ consisting of an object $X$ of $\cC$ and an isomorphism $u:T(X)\ra X$ in $\cC$. If $\left(X,u\right)$ and $\left(Y,w\right)$ are two such pairs, then a morphism $f:(X,u) \ra (Y,u)$ is a morphism $f:X\ra Y$ in $\cC$ such that the following square
\begin{center}
\begin{tikzpicture}
[description/.style={fill=white,inner sep=2pt}]
\matrix (m) [matrix of math nodes, row sep=3em, column sep=3em,text height=1.5ex, text depth=0.25ex] 
{ T(X) &  X    \\
  T(Y) &  Y           \\} ;
\path[->,line width=1.0pt,font=\scriptsize]  
(m-1-1) edge node[above] {$ u  $} (m-1-2)
(m-2-1) edge node[below] {$ w $} (m-2-2)
(m-1-1) edge node[left] {$ T(f) $} (m-2-1)
(m-1-2) edge node[right] {$ f  $} (m-2-2);
\end{tikzpicture}
\end{center}
is commutative. This data give rise to a category $\cC(T)$. There exists a forgetful functor $\pi:\cC(T)\ra \cC$ that sends a morphism $f:(X,u)\ra (Y,w)$ to $f:X\ra Y$. Moreover, there exists a natural isomorphism $\sigma:T\cdot \pi \Rightarrow \pi$ such that the component of $\sigma$ on an object $(X,u)$ of $\cC(T)$ is $u$. The next result states that the data above form a certain $2$-categorical limit.

\begin{theorem}\label{theorem:endoscope_2_limits}
Let $(\cC,T)$ be a pair consiting of a category and its endofunctor $T:\cC\ra \cC$. Suppose that $\cD$ is a category, $P:\cD\ra \cC$ is a functor and $\tau:T\cdot P \Rightarrow P$ is a natural isomorphisms. Then there exists a unique functor $F:\cD\ra \cC(T)$ such that $P = \pi\cdot F$ and $\sigma_F = \tau$.
\end{theorem}
\begin{proof}
Suppose that $F:\cD\ra \cC(T)$ is a functor such that $P = \pi\cdot F$ and $\sigma_F = \tau$. Pick an object $X$ of $\cD$. Then we have $\pi\cdot F(X) = P(X)$ and $\sigma_{F(X)} = \tau_X$. This implies that
$$F(X) = \left(P(X),\tau_X:T(P(X))\ra P(X)\right)$$
Next if $f:X\ra Y$ is a morphism in $\cD$, then we derive that $\pi(F(f)) = P(f)$. Hence $F(f) = P(f)$. This implies that there exists at most one functor $F$ satisfying the properties above. Note also that formulas
$$F(X) = \left(P(X),\tau_X:T(P(X))\ra P(X)\right),\,F(f) = P(f)$$
for an object $X$ in $\cD$ and a morphism $f:X\ra Y$ in $\cD$, give rise to a functor that satisfy $P = \pi\cdot F$ and $\sigma_F = \tau$. This establishes existence and the uniqueness of $F$.
\end{proof}
\noindent
Assume now that the pair $(\cC,T)$ consists of a monoidal category $\cC$ and a monoidal endofunctor $T$. Then there exists a canonical monoidal structure on $\cC(T)$. We define $(-)\otimes_{\cC(T)}(-)$ by formula
$$(X,u) \otimes_{\cC(T)}(Y,w) = \left(X\otimes_{\cC}Y,\left(u \otimes_{\cC} w\right)\cdot m_{X,Y}\right)$$
where
$$m_{X,Y}:T\left(X\otimes_{\cC} Y\right) \ra T(X)\otimes_{\cC}T(Y)$$
is the tensor preserving isomorphism of $T$. We also define the unit
$$I_{\cC(T)} = \left(I, T(I)\cong I\right)$$
where isomorphism $T(I)\cong I$ is precisely the unit preserving isomorphism of the monoidal functor $T$. The associativity natural isomorphism for $(-)\otimes_{\cC(T)}(-)$ and right, left units for $I_{\cC(T)}$ in $\cC(T)$ are associavity natural isomorphism and right, left units for $\cC$, respectively. The structure makes a functor $\pi:\cC(T)\ra \cC$ strict monoidal and $\sigma$ a monoidal natural isomorphism. The next result states that the data with these extra monoidal structure form a $2$-categorical limit in the $2$-category of monoidal categories.

\begin{theorem}\label{theorem:endoscope_monoidal_2_limit}
Let $(\cC,T)$ be a pair consiting of a monoidal category and its monoidal endofunctor $T:\cC\ra \cC$. Suppose that $\cD$ is a monoidal category, $P:\cD\ra \cC$ is a monoidal functor and $\tau:T\cdot P \Rightarrow P$ is a monoidal natural isomorphisms. Then there exists a unique monoidal functor $F:\cD\ra \cC(T)$ such that $P = \pi\cdot F$ and $\sigma_F = \tau$ as monoidal functors and monoidal transformations.
\end{theorem}
\begin{proof}
Note that $F$ must be defined as it was described in the proof of Theorem \ref{theorem:endoscope_2_limits}. Namely we must have
$$F(X) = \left(P(X),\tau_X:T(P(X))\ra P(X)\right),\,F(f) = P(f)$$
for an object $X$ in $\cC$ and a morphism $f:X\ra Y$ in $\cC$.\\
Suppose now that $F$ admits a structure of a monoidal functor such that $P = \pi\cdot F$ as monoidal functors. Let
$$\big\{m^F_{X,Y}:F(X\otimes_{\cD}Y)\ra F(X)\otimes_{\cC(T)}F(Y)\big\}_{X,Y\in \cC},\,\phi^F:F(I_{\cD})\ra I_{\cC(T)}$$
be the data forming that structure. Since $\pi$ is a strict monoidal functor and $P = \pi\cdot F$ as monoidal functors, we derive that for any objects $X,Y$ of $\cC$
$$\pi(m^F_{X,Y}):P(X\otimes_{\cD}Y) \ra P(X)\otimes_{\cC}P(Y)$$
is the tensor preserving isomorphism $m^{P}_{X,Y}:P(X\otimes_{\cD}Y) \ra P(X)\otimes_{\cC}P(Y)$ of the monoidal functor $P$. By the same argument
$$\pi(\phi_F):P(I_{\cD})\ra I_{\cC(T)}$$
is the unit preserving isomorphism $\phi^{P}:P(I_{\cD})\ra I_{\cC(T)}$ of $P$. Thus we deduce that for any objects $X,Y$ of $\cC$ we have $m^F_{X,Y} = m^{P}_{X,Y}$ and $\phi^F = \phi^{P}$. This implies that there exists at most one monoidal functor $F$ such that $P = \pi\cdot F$ as monoidal functors.\\
On the other hand define $m^F_{X,Y} = m^{P}_{X,Y}$ for objects $X,Y$ in $\cC$ and $\phi^F = \phi^{P}$. We check now that $F$ equipped with these data is a monoidal functor. Fix objects $X,Y$ in $\cC$. The square
\begin{center}
\begin{tikzpicture}
[description/.style={fill=white,inner sep=2pt}]
\matrix (m) [matrix of math nodes, row sep=4em, column sep=8em,text height=1.5ex, text depth=0.25ex] 
{T\left(P\left(X\otimes_{\cD}Y\right)\right)  & P\left(X\otimes_{\cC}Y\right)     \\
 T\left(P(X)\otimes_{\cC}P(Y)\right)  & P(X)\otimes_{\cC}P(Y)  \\} ;
\path[->,line width=1.0pt,font=\scriptsize]  
(m-1-1) edge node[above] {$ \tau_{X\otimes_{\cC}Y}  $} (m-1-2)
(m-2-1) edge node[below] {$ \left(\tau_X\otimes_{\cC}\tau_Y\right)\cdot m^{T}_{P(X),P(Y)} $} (m-2-2)
(m-1-1) edge node[left] {$ T\left(m^{P}_{X,Y}\right) $} (m-2-1)
(m-1-2) edge node[right] {$  m^{P}_{X,Y} $} (m-2-2);
\end{tikzpicture}
\end{center}
is commutative due to the fact that $\tau:T\cdot P \Rightarrow P$ is a monoidal natural isomorphisms. This implies that $m^F_{X,Y}$ is a morphism in $\cC(T)$. It follows that $m^F_{X,Y}$ is a natural isomorphism and due to the definition of associativity in $\cC(T)$, we derive its compatibility with $m^F_{X,Y}$. Similarly, since the square
\begin{center}
\begin{tikzpicture}
[description/.style={fill=white,inner sep=2pt}]
\matrix (m) [matrix of math nodes, row sep=3em, column sep=3em,text height=1.5ex, text depth=0.25ex] 
{T\left( P\left(I_{\cD}\right) \right)  & P\left(I_{\cD}\right)     \\
 T\left( I_{\cC} \right)  & I_{\cC}  \\} ;
\path[->,line width=1.0pt,font=\scriptsize]  
(m-1-1) edge node[above] {$ \tau_{I_{\cD}}  $} (m-1-2)
(m-2-1) edge node[below] {$ \phi^{T}  $} (m-2-2)
(m-1-1) edge node[left] {$ T\left(\phi^{P}\right)  $} (m-2-1)
(m-1-2) edge node[right] {$ \phi^{P}  $} (m-2-2);
\end{tikzpicture}
\end{center}
is commutative, we deduce that $\phi^F$ is a morphism in $\cC(T)$. By definition of left and right unit in $\cC(T)$, we derive their compatibility with $\phi^F$. This finishes the verification of the fact that $F$ with $\{m^F_{X,Y}\}_{X,Y\in \cC}$ and $\phi^F$ is a monoidal functor. Definitions of $\{m^F_{X,Y}\}_{X,Y\in \cC}$ and $\phi^F$ show that the identities $P = \pi\cdot F$ holds on the level of monoidal structures. Since the $2$-forgetful functor from $2$-category of monoidal categories into $2$-category of categories is faithful on $2$-cells, the identity $\sigma_F = \tau$ of natural isomorphisms is also the identity of monoidal natural isomorphisms.
\end{proof}

\begin{theorem}\label{theorem:endoscope_colimits}
Let $(\cC,T)$ be a pair consiting of a category and its endofunctor $T:\cC\ra \cC$. Assume that $T$ preserves colomits. Then the following assertions hold.
\begin{enumerate}[label=\textbf{\emph{(\arabic*)}}, leftmargin=1.5em]
\item $\pi:\cC(T)\ra \cC$ creates colimits.
\item Suppose that $\cD$ is a category, $P:\cD\ra \cC$ a functor preserving small colimits and $\tau:T\cdot P \Rightarrow P$ a natural isomorphisms. Then the unique functor $F:\cD \ra \cC(T)$ such that $P = \pi\cdot F$ and $\sigma_F = \tau$ preserves small colimits.
\end{enumerate}
\end{theorem}
\begin{proof}
Let $I$ be a small category and $D:I\ra \cC(T)$ be a diagram such that the composition $\pi\cdot D:I\ra \cC$ admits a colimit given by cocone $(X,\{g_i\}_{i\in I})$. Since $T$ preserves colimits, we derive that $\left(T(X), \{T(u_i)\}_{i\in I}\right)$ is a colimit of $T\cdot \pi \cdot D:I\ra \cC$. Now $\sigma_D:T\cdot \pi\cdot D\ra \pi\cdot D$ is a natural isomorphism. Hence there exists a unique arrow $u:T(X)\ra X$ such that $u \cdot T(g_i) = g_i\cdot \sigma_{D(i)}$ for $i\in I$. Clearly $u$ is an isomorphism and hence $(X,u)$ is an object of $\cC(T)$. Moreover, the family $\{g_i\}_{i\in I}$ together with $(X,u)$ is a colimiting cocone over $D$. This proves \textbf{(1)}. Now \textbf{(2)} is a consequence of \textbf{(1)}.
\end{proof}
\noindent
Now we apply the results above to certain more general diagrams of categories.

\begin{definition}
A diagram
\begin{center}   
\begin{tikzpicture}
[description/.style={fill=white,inner sep=2pt}]
\matrix (m) [matrix of math nodes, row sep=3em, column sep=2em,text height=1.5ex, text depth=0.25ex] 
{... &  \cC_{n+1}  &  \cC_n & ... & \cC_2 & \cC_1 & \cC_0  \\};
\path[->,line width=1.0pt,font=\scriptsize]    
(m-1-1) edge node[auto]  {$F_{n+1}  $} (m-1-2)
(m-1-2) edge node[auto]  {$F_n  $} (m-1-3)
(m-1-3) edge node[auto]  {$F_{n-1}  $} (m-1-4)
(m-1-4) edge node[auto]  {$F_2  $} (m-1-5)
(m-1-5) edge node[auto]  {$F_1  $} (m-1-6)
(m-1-6) edge node[auto]  {$F_0  $} (m-1-7);
\end{tikzpicture}
\end{center}
of categories and functors is called \textit{a telescope of categories}.
\end{definition}

\begin{definition}
Let 
\begin{center}   
\begin{tikzpicture}
[description/.style={fill=white,inner sep=2pt}]
\matrix (m) [matrix of math nodes, row sep=3em, column sep=2em,text height=1.5ex, text depth=0.25ex] 
{... &  \cC_{n+1}  &  \cC_n & ... & \cC_2 & \cC_1 & \cC_0  \\};
\path[->,line width=1.0pt,font=\scriptsize]    
(m-1-1) edge node[auto]  {$F_{n+1}  $} (m-1-2)
(m-1-2) edge node[auto]  {$F_n  $} (m-1-3)
(m-1-3) edge node[auto]  {$F_{n-1}  $} (m-1-4)
(m-1-4) edge node[auto]  {$F_2  $} (m-1-5)
(m-1-5) edge node[auto]  {$F_1  $} (m-1-6)
(m-1-6) edge node[auto]  {$F_0  $} (m-1-7);
\end{tikzpicture}
\end{center}
be a telescope of monoidal categories and monoidal cocontinuous functors. Then \textit{a $2$-categorical limit of the telescope} consists of a monoidal category $\cC$, a family of monoidal cocontinuous functors $\{\pi_n:\cC\ra \cC_n\}_{n\in \NN}$ and a family of monoidal natural isomorphisms $\{\sigma_n:F_{n+1} \cdot \pi_{n+1}\Rightarrow \pi_n\}_{n\in \NN}$ such that the following universal property holds. For any monoidal category $\cD$, family $\{P_n:\cD\ra \cC_n\}_{n\in \NN}$ of cocontinuous monoidal functors and a family $\{\tau_n:F_nP_{n+1}\Rightarrow P_{n}\}_{n\in \NN}$ of monoidal natural isomorphisms there exists a unique monoidal cocontinuous functor $F:\cD\ra \cC$ satisfying $P_n = \pi_n \cdot F$ and $\left(\sigma_n\right)_F = \tau_n$ for every $n\in \NN$.
\end{definition}

\begin{corollary}\label{corollary:telescope_2_limits}
Let 
\begin{center}   
\begin{tikzpicture}
[description/.style={fill=white,inner sep=2pt}]
\matrix (m) [matrix of math nodes, row sep=3em, column sep=2em,text height=1.5ex, text depth=0.25ex] 
{... &  \cC_{n+1}  &  \cC_n & ... & \cC_2 & \cC_1 & \cC_0  \\};
\path[->,line width=1.0pt,font=\scriptsize]    
(m-1-1) edge node[auto]  {$F_{n+1}  $} (m-1-2)
(m-1-2) edge node[auto]  {$F_n  $} (m-1-3)
(m-1-3) edge node[auto]  {$F_{n-1}  $} (m-1-4)
(m-1-4) edge node[auto]  {$F_2  $} (m-1-5)
(m-1-5) edge node[auto]  {$F_1  $} (m-1-6)
(m-1-6) edge node[auto]  {$F_0  $} (m-1-7);
\end{tikzpicture}
\end{center}
be a telescope of monoidal categories and monoidal cocontinuous functors. Then its $2$-limit exists.
\end{corollary}
\begin{proof}
We decompose the task of constructing its $2$-limit as follows. First note that one may form a product $\cC = \prod_{n\in \NN}\cC_n$. Next the functors $\{F_n\}_{n\in \NN}$ induce an endofunctor $T = \prod_{n\in \NN}F_n\times t$, where $\bd{1}$ is the terminal category (it has single object and single identity arrow) and $t:\cC_0\rightarrow \bd{1}$ is the unique functor. Consider the category $\cC(T)$. We define $\{\pi_n:\cC(T)\ra \cC_n\}_{n\in \NN}$ to be a family of functors given by coordinates of $\pi:\cC(T)\ra \cC$ and $\{\sigma_n:F_n\cdot \pi_{n+1}\Rightarrow \pi_n\}_{n\in \NN}$ to be a family of natural isomorphisms given by coordinates of $\sigma:\pi\cdot T\Rightarrow \pi$. Now this data form a $2$-limit of the telescope by compilation of Theorem \ref{theorem:endoscope_monoidal_2_limit} and Theorem \ref{theorem:endoscope_colimits}.
\end{proof}

\section{Formal $\bd{G}$-schemes}
\noindent
This section is devoted to introducing some notions from formal geometry that are central in this notes. We fix a group scheme $\bd{G}$ over $k$. 

\begin{definition}
\textit{A formal $\bd{G}$-scheme} consists of a sequence $\cZ = \{Z_n\}_{n\in \NN}$ of $\bd{G}$-schemes together with $\bd{G}$-equivariant closed immersions
\begin{center}
\begin{tikzpicture}
[description/.style={fill=white,inner sep=2pt}]
\matrix (m) [matrix of math nodes, row sep=3em, column sep=3em,text height=1.5ex, text depth=0.25ex] 
{ Z_0 &  Z_1 & ... & Z_n & Z_{n+1} & ... \\} ;
\path[right hook->,line width=1.0pt,font=\scriptsize]  
(m-1-1) edge node[above] {$ $} (m-1-2)
(m-1-2) edge node[above] {$ $} (m-1-3)
(m-1-3) edge node[above] {$ $} (m-1-4)
(m-1-4) edge node[above] {$ $} (m-1-5)
(m-1-5) edge node[above] {$ $} (m-1-6);
\end{tikzpicture}
\end{center}
satisfying the following assertions.
\begin{enumerate}[label=\textbf{(\arabic*)}, leftmargin=1.5em]
\item We have $Z_0 = Z_n^{\bd{G}}$ scheme-theoretically for every $n\in \NN$.
\item Let $\cI_n$ be an ideal of $\cO_{Z_n}$ defining $Z_0$. Then for every $m\leq n$ the subscheme $Z_m \subset Z_n$ is defined by $\cI_n^{m+1}$.
\end{enumerate}
\end{definition}

\begin{example}\label{example:formal_neighborhood_of_fixed_pts}
Let $Z$ be a $\bd{G}$-scheme. Consider a quasi-coherent ideal $\cI$ of fixed point subscheme $Z^{\bd{G}}$ of $Z$. Then for every $n\in \NN$ ideal $\cI^n$ is $\bd{G}$-equivariant and hence
\begin{center}
\begin{tikzpicture}
[description/.style={fill=white,inner sep=2pt}]
\matrix (m) [matrix of math nodes, row sep=3em, column sep=3em,text height=1.5ex, text depth=0.25ex] 
{ V(\cI) &  V(\cI^2) & ... & V(\cI^n) & ... \\} ;
\path[right hook->,line width=1.0pt,font=\scriptsize]  
(m-1-1) edge node[above] {$ $} (m-1-2)
(m-1-2) edge node[above] {$ $} (m-1-3)
(m-1-3) edge node[above] {$ $} (m-1-4)
(m-1-4) edge node[above] {$ $} (m-1-5);
\end{tikzpicture}
\end{center}
is a formal $\bd{G}$-scheme. We denote it by $\widehat{Z}$.
\end{example}

\begin{definition}
Let $\cZ = \{Z_n\}_{n\in \NN}$ and $\cW = \{W_n\}_{n\in \NN}$ are formal $\bd{G}$-schemes. Then \textit{a morphism $f:\cZ\ra \cW$ of formal $\bd{G}$-schemes} consists of a family of $\bd{G}$-equivariant morphisms $f = \big\{f_n:Z_n\ra W_n\}_{n\in \NN}$ such that the diagram
\begin{center}
\begin{tikzpicture}
[description/.style={fill=white,inner sep=2pt}]
\matrix (m) [matrix of math nodes, row sep=3em, column sep=3em,text height=1.5ex, text depth=0.25ex] 
{ Z_0 &  Z_1 & ... & Z_n & Z_{n+1} & ... \\
 W_0 &  W_1 & ... & W_n & W_{n+1} & ... \\} ;
\path[right hook->,line width=1.0pt,font=\scriptsize]  
(m-1-1) edge node[above] {$ $} (m-1-2)
(m-1-2) edge node[above] {$ $} (m-1-3)
(m-1-3) edge node[above] {$ $} (m-1-4)
(m-1-4) edge node[above] {$ $} (m-1-5)
(m-1-5) edge node[above] {$ $} (m-1-6)
(m-2-1) edge node[above] {$ $} (m-2-2)
(m-2-2) edge node[above] {$ $} (m-2-3)
(m-2-3) edge node[above] {$ $} (m-2-4)
(m-2-4) edge node[above] {$ $} (m-2-5)
(m-2-5) edge node[above] {$ $} (m-2-6);
\path[->,line width=1.0pt,font=\scriptsize]
(m-1-1) edge node[left] {$f_0 $} (m-2-1)
(m-1-2) edge node[left] {$f_1 $} (m-2-2)
(m-1-4) edge node[left] {$f_n $} (m-2-4)
(m-1-5) edge node[left] {$f_{n+1} $} (m-2-5);
\end{tikzpicture}
\end{center}
is commutative.
\end{definition}

\begin{definition}
Let $\cZ = \{Z_n\}_{n\in \mathbb{N}}$ be a formal $\bd{G}$-scheme. Then there we have the corresponding telescope of monoidal categories
\begin{center}   
\begin{tikzpicture}
[description/.style={fill=white,inner sep=2pt}]
\matrix (m) [matrix of math nodes, row sep=3em, column sep=2em,text height=1.5ex, text depth=0.25ex] 
{... &  \Qcoh_{\bd{G}}(Z_{n+1})  &  \Qcoh_{\bd{G}}(Z_n) & ... & \Qcoh_{\bd{G}}(Z_2) & \Qcoh_{\bd{G}}(Z_1) & \Qcoh_{\bd{G}}(Z_0)  \\};
\path[->,line width=1.0pt,font=\scriptsize]    
(m-1-1) edge node[auto]  {$  $} (m-1-2)
(m-1-2) edge node[auto]  {$  $} (m-1-3)
(m-1-3) edge node[auto]  {$  $} (m-1-4)
(m-1-4) edge node[auto]  {$  $} (m-1-5)
(m-1-5) edge node[auto]  {$  $} (m-1-6)
(m-1-6) edge node[auto]  {$  $} (m-1-7);
\end{tikzpicture}
\end{center}
and cocontinuous monoidal functors given by restricting $\bd{G}$-equivariant quasi-coherent sheaves to closed $\bd{G}$-subschemes. Then we define \textit{a category $\Qcoh(\cZ)$ of quasi-coherent sheaves on $\cZ$} as a monoidal category which is a $2$-limit of the telescope above. This category is defined uniquely up to a monoidal equivalence.
\end{definition}
\noindent
Let $Z$ be a $\bd{G}$-scheme and let $\cI$ be a quasi-coherent ideal of $Z^{\bd{G}}$. We have a commutative diagram
\begin{center}
\begin{tikzpicture}
[description/.style={fill=white,inner sep=2pt}]
\matrix (m) [matrix of math nodes, row sep=3em, column sep=3em,text height=1.5ex, text depth=0.25ex] 
{ V(\cI) &  V(\cI^2) & ... & V(\cI^n) & ... \\
         &           & Z   &          &      \\} ;
\path[right hook->,line width=1.0pt,font=\scriptsize]  
(m-1-1) edge node[above] {$ $} (m-1-2)
(m-1-2) edge node[above] {$ $} (m-1-3)
(m-1-3) edge node[above] {$ $} (m-1-4)
(m-1-4) edge node[above] {$ $} (m-1-5)
(m-1-2) edge node[above] {$ $} (m-2-3);
\path[right hook->,bend right, line width=1.0pt,font=\scriptsize]  
(m-1-1) edge node[above] {$ $} (m-2-3);
\path[right hook->, line width=1.0pt,font=\scriptsize]  
(m-1-4) edge node[above] {$ $} (m-2-3);
\end{tikzpicture}
\end{center}
in the category of $\bd{G}$-schemes. Thus restriction functors $\Qcoh_{\bd{G}}(Z) \ra \Qcoh_{\bd{G}}(V(\cI^n))$ for $n\in \NN$ induce a unique cocontinuous monoidal functor $\Qcoh_{\bd{G}}(Z)\ra \Qcoh(\widehat{Z})$.

\begin{definition}
Let $Z$ be a $\bd{G}$-scheme. Then a unique cocontinuous monoidal functor $\Qcoh_{\bd{G}}(Z)\ra \Qcoh(\widehat{Z})$ is called \textit{the comparison functor}.
\end{definition}

\begin{definition}
Let $\cZ = \{Z_n\}_{n\in \mathbb{N}}$ be a formal $\bd{G}$-scheme. A $\bd{G}$-scheme $Z$ is called \textit{an algebraization of $\cZ$} if the following two conditions are satisfied.
\begin{enumerate}[label=\textbf{(\arabic*)}, leftmargin=1.5em]
\item $\cZ$ is isomorphic to $\widehat{Z}$ in the category of formal $\bd{G}$-schemes.
\item The comparison functor $\Qcoh_{\bd{G}}(Z)\ra \Qcoh\left(\widehat{Z}\right)$ is an equivalence of monoidal categories.
\end{enumerate}
\end{definition}

\section{Preliminaries}
\lstodo{Większość z wyników, które tutaj są, powinna być w teoretyczym wstępie. Idea jest taka, by tutaj w zasadzie tylko przygotować notację do dowodu głównego twierdzenia.}
\subsection{Toruses}

\begin{definition}
Let $T$ be an affine algebraic group over $k$. Suppose that there exists $n\in \NN$ such that for every algebraically closed extension $K$ of $k$ there exists an isomorphism
$$T_K \cong  \underbrace{\mathbb{G}_{m,K}\times \mathbb{G}_{m,K}\times ...\times \mathbb{G}_{m,K}}_{n\,\mathrm{times}} $$
of group schemes over $k$. Then $T$ is called \textit{a torus over $k$}.
\end{definition}

\begin{example}\label{example:split_torus}
If $T \cong \underbrace{\mathbb{G}_{m,K}\times \mathbb{G}_{m,K}\times ...\times \mathbb{G}_{m,K}}_{n\,\mathrm{times}}$, then $T$ is a torus. We call toruses $T$ of this form \textit{split toruses}.
\end{example}

\begin{example}\label{example:non_split_torus}
Assume that the characteristic of $k$ is not $2$. Define
$$\bd{S}^1 = \Spec k[x,y]/(x^2+y^2-1)$$
a scheme over $k$ and let $\fP_{\bd{S}^1}$ be its functor of points. Then for every $k$-algebra $A$ we have
$$\fP_{\bd{S}^1}(A) = \big\{(u,v)\in A\times A\,\big|\,u^2+v^2=1\big\}$$
There is also a morphism $\fP_{\bd{S}^1}\times \fP_{\bd{S}^1}\ra \fP_{\bd{S}^1}$ of $k$-functors given by
$$\fP_{\bd{S}^1}(A)\times \fP_{\bd{S}^1}(A)\ra \fP_{\bd{S}^1}\ni \left((u_1,v_1),(u_2,v_2)\right)\mapsto (u_1u_2-v_1v_2,u_1v_2+u_2v_1)\in \fP_{\bd{S}^1}(A)$$
for every $k$-algebra $A$. This makes $\fP_{\bd{S}^1}$ into a group $k$-functor. Thus $\bd{S}^1$ with the group structure described above is an affine algebraic group over $k$. We call it \textit{the circle group over $k$}.\\
Now suppose that $K$ is an algebraically closed extension of $k$. Consider an element $i\in K$ such that $i^2 = -1$. For every $K$-algebra $A$ we have a map
$$\fP_{\bd{S}^1}(A)\ni (u,v)\mapsto u+iv\in A^*$$
First note that this map is bijective. Indeed, its inverse is given by
$$A^*\ni a \mapsto \left(\frac{1}{2}(a+a^{-1}),\frac{1}{2i}(a-a^{-1})\right) \in \fP_{\bd{S}^1}(A)$$
Moreover, the map $\fP_{\bd{S}^1}(A)\ra A^*$ is a homomorphism of abstract groups. Thus $\fP_{\bd{S}^1}$ resricted to the category $\Alg_K$ of $K$-algebras is isomorphic with $\fP_{\mathbb{G}_{m,K}}$ as a group $k$-functor. Hence
$$\bd{S}^1_K \cong \mathbb{G}_{m,K}$$
as algebraic group schemes over $K$. Hence $\bd{S}^1$ is a torus over $k$.\\
Now assume that $k = \RR$. Then abstract groups
$$\fP_{\bd{S}^1}(\RR) = \big\{z\in \CC\,\big|\,|z|=1\big\} \subseteq \CC^*,\,\RR^*$$
are not isomorphic. Indeed, the left hand side group has infinite torsion subgroup and the right hand side group has torsion subgroup equal to $\{-1,1\}$. This imlies that over $\RR$ algebraic groups $\bd{S}^1$ and $\mathbb{G}_{m}$ are not isomorphic. Hence $\bd{S}^1$ is not a split torus over $\RR$.
\end{example}

\subsection{Locally linear schemes}

\begin{definition}
Let $\bd{M}$ be a monoid $k$-scheme and let $X$ be a $\bd{M}$-scheme. Suppose that each point of $X$ admits an open affine $\bd{M}$-stable neighborhood. Then we say that $X$ is \textit{a locally linear $\bd{M}$-scheme}.
\end{definition}

\begin{proposition}\label{proposition:monoid_open_stable_correspondence}
Let $\bd{M}$ be an affine monoid $k$-scheme and let $X$ be a $\bd{M}$-scheme. Suppose that there exists a quasi-coherent $\bd{M}$-equivariant ideal $\cI$ on $X$ with nilpotent sections. Consider an open subset $U$ of $X$. Then the following are equivalent.
\begin{enumerate}[label=\emph{\textbf{(\arabic*)}}, leftmargin=1.5em]
\item $U$ is $\bd{M}$-stable.
\item $U\cap V(\cI)$ is $\bd{M}$-stable.
\end{enumerate}
\end{proposition}
\begin{proof}
Let $\alpha:\bd{M} \times X\ra X$ be the action of $\bd{M}$ on $X$. Fix open subset $U$ of $X$. If $U$ is $\bd{M}$-stable, then $U\cap V(\cI)$ is $\bd{M}$-stable. So suppose that $U\cap V(\cI)$ is $\bd{M}$-stable. Since $\cI$ has nilpotent sections and $\bd{M}$ is affine, we derive that closed immersions $U\cap V(\cI)\hookrightarrow U$ and $\bd{M}\times \left(U\cap V(\cI)\right) \hookrightarrow \bd{M}\times U$ induce homeomorphisms on topological spaces. Consider the commutative diagram
\begin{center}
\begin{tikzpicture}
[description/.style={fill=white,inner sep=2pt}]
\matrix (m) [matrix of math nodes, row sep=3em, column sep=3em,text height=1.5ex, text depth=0.25ex] 
{ \bd{M}\times U & X     \\
  \bd{M}\times \left(U\cap V(\cI)\right) & U\cap V(\cI)   \\} ;
\path[->,line width=1.0pt,font=\scriptsize]  
(m-1-1) edge node[above] {$ \alpha_{\mid U\cap V(\cI)}  $} (m-1-2)
(m-2-1) edge node[below] {$   $} (m-2-2);
\path[right hook->,line width=1.0pt,font=\scriptsize]  
(m-2-1) edge node[left] {$  $} (m-1-1)
(m-2-2) edge node[right] {$ $} (m-1-2);
\end{tikzpicture}
\end{center}
where the bottom horizontal arrow is the induced action on $U\cap V(\cI)$ and vertical morphisms are homeomorphisms. The commutativity of the diagram implies that $\alpha\left(\bd{M}\times U\right)$ is contained set-theoretically in $U$. Since $U$ is open in $X$, we derive that morphism of schemes $\alpha_{\mid \bd{M}\times U}$ factors through $U$. Hence $U$ is $\bd{M}$-stable.
\end{proof}

\begin{corollary}\label{corollary:monoid_stable_open_affine_correspondence}
Let $\bd{M}$ be an affine monoid $k$-scheme and let $X$ be a $\bd{M}$-scheme. Suppose that there exists a quasi-coherent $\bd{M}$-equivariant ideal $\cI$ on $X$ such that $\cI^n = 0$ for $n\in \NN$. Consider an open subset $U$ of $X$. Then the following are equivalent.
\begin{enumerate}[label=\emph{\textbf{(\arabic*)}}, leftmargin=1.5em]
\item $U$ is $\bd{M}$-stable and affine.
\item $U\cap V(\cI)$ is $\bd{M}$-stable and affine.
\end{enumerate}
\end{corollary}
\begin{proof}
Since $\cI^n = 0$, we derive that $U$ is affine if and only if $U\cap V(\cI)$ is affine. Combining this with Proposition \ref{proposition:monoid_open_stable_correspondence}, we deduce the result.
\end{proof}

\begin{corollary}\label{corollary:locally_linear_are_stable_under_thickenings}
Let $\bd{M}$ be an affine monoid $k$-scheme and let $X$ be a $\bd{M}$-scheme. Suppose that there exists a quasi-coherent $\bd{M}$-equivariant ideal $\cI$ on $X$ such that $\cI^n = 0$ for $n\in \NN$. Then $X$ is locally linear $\bd{M}$-scheme if and only if $V(\cI)$ is locally linear $\bd{M}$-scheme.
\end{corollary}
\begin{proof}
This is a consequence of Corollary \ref{corollary:monoid_stable_open_affine_correspondence}.
\end{proof}

\subsection{Affine monoid schemes with zero}

\begin{proposition}\label{proposition:retraction_for_monoids_with_zero}
Let $\bd{M}$ be an affine monoid $k$-scheme with zero and let $X$ be a locally linear $\bd{M}$-scheme. Then there exists an affine $\bd{M}$-equivariant morphism
\begin{center}
\begin{tikzpicture}
[description/.style={fill=white,inner sep=2pt}]
\matrix (m) [matrix of math nodes, row sep=3em, column sep=3em,text height=1.5ex, text depth=0.25ex] 
{X & X^{\bd{M}} \\} ;
\path[->,line width=1.0pt,font=\scriptsize]  
(m-1-1) edge node[above] {$r$} (m-1-2);
\end{tikzpicture}
\end{center}
such that $r_{\mid X^{\bd{M}}} = 1_{X^{\bd{M}}}$.
\end{proposition}
\begin{proof}
Consider the action $\alpha:\bd{M}\times X\ra X$ of $\bd{M}$ on $X$. Since $X$ is locally linear and $\bd{M}$ is affine, we derive that $\alpha$ is an affine morphism of $k$-schemes. Now if $\bd{o}$ is a zero of $\bd{M}$, then we define a morphism 
\begin{center}
\begin{tikzpicture}
[description/.style={fill=white,inner sep=2pt}]
\matrix (m) [matrix of math nodes, row sep=3em, column sep=3em,text height=1.5ex, text depth=0.25ex] 
{X & \bd{o}\times X & \bd{M}\times X & X \\} ;
\path[->,line width=1.0pt,font=\scriptsize]  
(m-1-1) edge node[above] {$\cong$} (m-1-2)
(m-1-3) edge node[above] {$\alpha$} (m-1-4);
\path[right hook->,line width=1.0pt,font=\scriptsize]  
(m-1-2) edge node[above] {$ $} (m-1-3);
\end{tikzpicture}
\end{center}
The morphism above is affine (as a composition of affine morphisms) and induces multiplication by $\bd{o}$ on functors of points $\bd{o}\cdot(-):\fP_X\ra \fP_X$. Now $\bd{o}\cdot (-):\fP_X\ra \fP_X$ factors as an $fP_{\bd{M}}$-equivariant epimorphism $\fP_X\twoheadrightarrow \fP_{X^{\bd{M}}}$ composed with a closed immersion $\fP_{X^{\bd{M}}}\hookrightarrow \fP_X$. The $\fP_{\bd{M}}$-equivariant epimorphism $\fP_X\ra \fP_{X^{\bd{M}}}$ corresponds to a $\bd{M}$-equivariant morphism $r:X\ra X^{\bd{M}}$ of $k$-schemes such that $r_{\mid X^{\bd{M}}} = 1_{X^{\bd{M}}}$. Moreover, the composition of $r$ with a closed immersion $X^{\bd{M}}\hookrightarrow X$ is an affine morphism. Thus $r$ is affine.
\end{proof}

\subsection{Results on linear representations}

\begin{proposition}\label{proposition:invariants_are_stable_under_tensoring_with_k_algebra}
Let $\bd{M}$ be an affine monoid $k$-scheme and let $V$ be a representation of $\bd{M}$. Then for every $k$-algebra $A$ the natural morphism of $A$-modules
$$V^{\bd{M}}\otimes_{k}A \ra \left(A\otimes_kV\right)^{\bd{M}_A}$$
is an isomorphism.
\end{proposition}
\begin{proof}
Note that we have a left exact sequence of $k$-vector spaces defining invariants
\begin{center}
\begin{tikzpicture}
[description/.style={fill=white,inner sep=2pt}]
\matrix (m) [matrix of math nodes, row sep=3em, column sep=3em,text height=1.5ex, text depth=0.25ex] 
{0 &  V^{\bd{M}} &   V& \Gamma(\bd{M},\cO_{\bd{M}}) \otimes_{k}V    \\} ;
\path[->,line width=1.0pt,font=\scriptsize]  
(m-1-1) edge node[above] {$ $} (m-1-2)
(m-1-2) edge node[above] {$ $} (m-1-3)
(m-1-3) edge node[above] {$\Delta-p $} (m-1-4);
\end{tikzpicture}
\end{center}
where $\Delta:V\ra \Gamma(\bd{M},\cO_{\bd{M}})\otimes_{k} V$ is the coaction and $p:V\ra \Gamma(\bd{M},\cO_{\bd{M}})\otimes_{k}V$ is the trivial coaction defined by formula $p(v)= 1\otimes v$ for every $v$ in $V$. Now tensoring the sequence with $k$-algebra $A$ yields a left exact sequence
\begin{center}
\begin{tikzpicture}
[description/.style={fill=white,inner sep=2pt}]
\matrix (m) [matrix of math nodes, row sep=3em, column sep=3em,text height=1.5ex, text depth=0.25ex] 
{0 &  V^{\bd{M}}\otimes_k A &   A \otimes_kV &\Gamma(\bd{M}_A,\cO_{\bd{M}_A})\otimes_{A}\left(A\otimes_kV\right)   \\} ;
\path[->,line width=1.0pt,font=\scriptsize]  
(m-1-1) edge node[above] {$ $} (m-1-2)
(m-1-2) edge node[above] {$ $} (m-1-3)
(m-1-3) edge node[above] {$\Delta_A-p_A $} (m-1-4);
\end{tikzpicture}
\end{center}
where $\Delta_A$ is the coaction on $A\otimes_kV$ induced by $\Delta$ and $p_A$ is the trivial coaction on $A\otimes_kV$. This shows that $V^{\bd{M}}\otimes_{k}A \ra \left(A\otimes_kV\right)^{\bd{M}_A}$ is an isomorphism.
\end{proof}

\begin{proposition}\label{proposition:base_change_for_hom_of_representations_of_affine_group_schemes}
Let $\bd{G}$ be an affine group $k$-scheme and let $V,W$ be representations of $\bd{G}$. If $V$ is finite dimensional, then for every $k$-algebra $A$ the canonical morphism
\begin{center}
\begin{tikzpicture}
[description/.style={fill=white,inner sep=2pt}]
\matrix (m) [matrix of math nodes, row sep=3em, column sep=3em,text height=1.5ex, text depth=0.25ex] 
{A\otimes_k\Hom_{\bd{G}}(V,W) & \Hom_{\bd{G}_A}\left(A\otimes_kV,A\otimes_kW\right)\\} ;
\path[->,line width=1.0pt,font=\scriptsize]  
(m-1-1) edge node[above] {$ $} (m-1-2);
\end{tikzpicture}
\end{center}
is an isomorphism of $A$-modules.
\end{proposition}
\begin{proof}
Fix a $k$-algebra $A$. Since $V$ is finite dimensional, for every $k$-algebra $B$ there exists an isomorphism $B\otimes_k\Hom_{k}(V,W) \ra \Hom_{B}\left(B\otimes_kV,B \otimes_kW\right)$ of $B$-modules natural in $B$. This implies that $\Hom_k(V,W)$ is a representation of $\bd{G}$ via the action given by formula
$$\left(g\cdot f\right)(v) = g\cdot f(g^{-1}\cdot v)$$
where $f\in \Hom_{B}\left(B\otimes_kV,B \otimes_kW\right)$, $v\in B\otimes_kV$ and $g\in \fP_{\bd{G}}(B)$. Similarly $\Hom_A(A\otimes_kV,A\otimes_kW)$ is a representation of $\bd{G}_K$ and the canonical isomorphism $A\otimes_k\Hom_{k}(V,W) \ra \Hom_{A}\left(A\otimes_kV,A \otimes_kW\right)$ of $A$-modules is $\bd{G}_A$-equivariant. Now we apply Proposition \ref{proposition:invariants_are_stable_under_tensoring_with_k_algebra} to derive a chain of isomorphisms
$$\Hom_{A}\left(A\otimes_kV,A\otimes_kW\right)^{\bd{G}_A} \cong \left(A\otimes_k\Hom_{k}(V,W)\right)^{\bd{G}_A} \cong A\otimes_k\Hom_k(V,W)^{\bd{G}}$$
of $A$-modules. Since we have identifications
$$\Hom_{\bd{G}_A}\left(A\otimes_kV,A\otimes_kW\right) \cong \Hom_{A}\left(A\otimes_kV,A\otimes_kW\right)^{\bd{G}_A} ,\,\Hom_{\bd{G}}\left(V,W\right)  \cong \Hom_k\left(V,W\right)^{\bd{G}}$$
we deduce the statement.
\end{proof}

\begin{proposition}\label{proposition:commuting_action_preserves_isotypic_decomposition}
Let $\bd{G}$ be an affine group scheme over $k$ and let $\fG$ be a monoid $k$-functor. Denote by $\Lambda$ the set of isomorphism classes of irreducible $\bd{G}$-representations. Suppose that $V$ is a representation of both $\bd{G}$ and $\fG$ and assume that their actions on $V$ commute. Assume that $V$ is completely reducible as a $\bd{G}$-representation and consider the decomposition
$$V = \bigoplus_{\lambda\in \Lambda}V_{\lambda}$$
onto isotypic components with respect to the action of $\bd{G}$. Then for every $\lambda$ in $\Lambda$ the subspace $V_{\lambda}$ is a $\fG$-subrepresentation of $V$.
\end{proposition}
\begin{proof}
\lstodo{Uzupełnić dowód w oparciu o Stwierdzenie \ref{proposition:base_change_for_hom_of_representations_of_affine_group_schemes}.}
\end{proof}

\subsection{$\bd{M}$-equivariant quasi-coherent sheaves}
\lstodo{Tu trzeba zdefiniować i następnie opisać przypadek schematu z trywialnym działaniem, bo on jest najważniejszy}

\subsection{Kempf monoids}

\begin{definition}
\lstodo{Tutaj trzeba zdefiniować monoidy Kempfa. Najpierw trzeba porządnie spisać dowód algebraizacji, żeby mieć poprawną definicję}
Let $\bd{M}$ be a monoid $k$-scheme. Suppose that the following conditions hold.
\begin{enumerate}[label=\textbf{(\arabic*)}, leftmargin=1.5em]
\item $\bd{M}$ is affine, geometrically connected and geometrically normal.
\item There exists zero $\bd{o}$ in $\bd{M}$.
\item There exists a torus $T$ over $k$ contained in the center of $\bd{M}$ such that the closure $\bd{cl}(T)$ of $T$ in $\bd{M}$ contains $\bd{o}$.
\end{enumerate}
Then $\bd{M}$ is called \textit{Kempf \lstodo{(or rather Jelisiejew :D)} monoid}.
\end{definition}
\noindent
Let $\bd{M}$ be a Kempf monoid and let $\bd{G}$ be its group of units. If $V$ is a representation of $\bd{G}$ and $\lambda$ is a class in $\Lambda$, then we denote by $V[\lambda]\subseteq V$ the sum of all irreducible $T$-subpresentations of $V$ of isomorphism type $\lambda$. Since $T$ is a central subgroup of $\bd{G}$, we derive by Proposition \ref{proposition:commuting_action_preserves_isotypic_decomposition} that $V[\lambda]$ is a $\bd{G}$-representation of $V$.\\
Suppose that $Z$ is a $k$-scheme with trivial action of $\bd{M}$. If $\cF$ is a quasi-coherent sheaf on $Z$ equipped with $\bd{G}$-action, then we denote by $\cF[\lambda]$ a sheaf given by
$$U\mapsto \cF(U)[\lambda]$$
for every open affine subset $U$ of $Z$. Then $\cF[\lambda]\subseteq \cF$ is a $\bd{G}$-quasi-coherent subsheaf of $\cF$.

\subsection{Formal $\bd{M}$-schemes}

\begin{definition}
Let $\bd{M}$ be a monoid $k$-scheme having $\bd{G}$ as the group of units. \textit{A formal $\bd{M}$-scheme} is a formal $\bd{G}$-scheme $\cZ = \{Z_n\}_{n\in \NN}$ such that for each $n\in \NN$ scheme $Z_n$ is $\bd{M}$-scheme and the sequence of closed immersions
\begin{center}
\begin{tikzpicture}
[description/.style={fill=white,inner sep=2pt}]
\matrix (m) [matrix of math nodes, row sep=3em, column sep=3em,text height=1.5ex, text depth=0.25ex] 
{ Z_0 &  Z_1 & ... & Z_n & Z_{n+1} & ... \\} ;
\path[right hook->,line width=1.0pt,font=\scriptsize]  
(m-1-1) edge node[above] {$ $} (m-1-2)
(m-1-2) edge node[above] {$ $} (m-1-3)
(m-1-3) edge node[above] {$ $} (m-1-4)
(m-1-4) edge node[above] {$ $} (m-1-5)
(m-1-5) edge node[above] {$ $} (m-1-6);
\end{tikzpicture}
\end{center}
consists of $\bd{M}$-equivariant morphisms.
\end{definition}
\noindent
Let $\bd{M}$ be an affine monoid $k$-scheme with zero and let $\cZ = \{Z_n\}_{n\in \NN}$ be a formal $\bd{M}$-scheme. Suppose that $\bd{M}$ is a monoid with zero. Then by Proposition \ref{proposition:retraction_for_monoids_with_zero}, we derive that $\cZ$ is a part of the commutative diagram
\begin{center}
\begin{tikzpicture}
[description/.style={fill=white,inner sep=2pt}]
\matrix (m) [matrix of math nodes, row sep=2em, column sep=3em,text height=1.5ex, text depth=0.25ex] 
{ Z_0 &  Z_1 & ... & Z_n & ... \\
      &      & Z_0 &     &  \\} ;
\path[right hook->,line width=1.0pt,font=\scriptsize]  
(m-1-1) edge node[above] {$ $} (m-1-2)
(m-1-2) edge node[above] {$ $} (m-1-3)
(m-1-3) edge node[above] {$ $} (m-1-4)
(m-1-4) edge node[above] {$ $} (m-1-5);
\path[->>,line width=1.0pt,font=\scriptsize]
(m-1-1) edge[bend right = 20] node[below] {$r_0 = 1_{Z_0} $} (m-2-3)
(m-1-2) edge node[above] {$r_1 $} (m-2-3)
(m-1-4) edge[bend left = 20] node[above] {$r_n $} (m-2-3);
\end{tikzpicture}
\end{center}
in which vertical morphisms $r_n:Z_n\twoheadrightarrow Z_0$ are affine morphisms such that ${r_n}_{\mid Z_0} = 1_{Z_0}$. This implies that $Z_n$ is affine over $Z_0$ for each $n\in \NN$ and hence we write $\Spec_{Z_0}\cA_n$ for $n\in \NN$, where $\cA_n$ is a quasi-coherent $Z_0$-algebra equipped with the action of $\bd{M}$. Moreover, we have $\cA_0 = \cO_{Z_0}$. The diagram above induces the following sequence of epimorphisms
\begin{center}
\begin{tikzpicture}
[description/.style={fill=white,inner sep=2pt}]
\matrix (m) [matrix of math nodes, row sep=3em, column sep=3em,text height=1.5ex, text depth=0.25ex] 
{ ... &  \cA_{n+1} & \cA_n & ...& \cA_1 & \cA_0 = \cO_{Z_0} \\} ;
\path[->>,line width=1.0pt,font=\scriptsize]  
(m-1-1) edge node[above] {$ $} (m-1-2)
(m-1-2) edge node[above] {$ $} (m-1-3)
(m-1-3) edge node[above] {$ $} (m-1-4)
(m-1-4) edge node[above] {$ $} (m-1-5)
(m-1-5) edge node[above] {$ $} (m-1-6);
\end{tikzpicture}
\end{center}
of quasi-coherent $\cO_{Z_0}$-algebras with $\bd{M}$-action. Denote by $\bd{G}$ the group of units of $\bd{M}$. If $\bd{G}$ is schematically dense in $\bd{M}$ (for instance if $\bd{M}$ is integral), then we have $Z_0 = Z_n^{\bd{G}} = Z_n^{\bd{M}}$ and hence $Z_0$ admits trivial $\bd{M}$-action. This alternative description of formal $\bd{M}$-schemes will be used in the proof of the main theorem.

\section{Formal functors and representability - OLD}

    \begin{theorem}[Algebraization of a formal $\Gbar$-scheme]\label{ref:algebraizationOfFormalSchemes:thm}
        Let $\Zformal =\{Z_n\}$ be a formal $\Gbar$-scheme. Then there exists
        a colimit
        \[
            Z = \colim_{n} Z_n
        \]
        in the category of locally linear $\Gbar$-schemes and $Z$ is the
        unique algebraization of $\Zformal$.
        If in addition $\Zformal$ is locally Noetherian, then $\iinfty{Z}$ is of finite type. If
        $\Zformal$ is locally Noetherian and $Z_0$ is of finite type, then also $Z$ is of
        finite type.
    \end{theorem}

Now we spell out the main idea of the proof: the $\Gbar$-scheme $Z$
required in Theorem~\ref{ref:algebraizationOfFormalSchemes:thm} is equal to $\Spec_{Z_0} \cA$, where
$\cA$ is the limit of $\cA_n$ \emph{in the category of
$\Gbar$-algebras}; in other words each isotypic component of $\cA$ is the
limit of isotypic components of $\cA_n$.
Our first goal is to prove a stabilization result.
We denote by $\Irr(\Group)$ the set of isomorphism types of irreducible
$\Group$-representations and by $\Irr(\Gbar) \subset \Irr(\Group)$ the
subset of $\Gbar$-representations. For $\lambda\in \Irr(\Group)$ and
a quasi-coherent $\Gbar$-module $\cC$ on $Z_0$ we denote by $\cC[\lambda]
\subset \cC$ the $\Gbar$-submodule such that $H^0(U, \cC[\lambda]) \subset H^0(U, \cC)$
is the union of all $\Group$-subrepresentations of $H^0(U, \cC)$ isomorphic to
$\lambda$ (i.e., the isotypic component of $\lambda$).


\begin{lemma}[stabilization on an isotypic component]\label{ref:stability:lem}
Let $\lambda\in \Irr(\Gbar)$. Then there exists a number $n_{\lambda}\in \NN$
such that the following holds. Let $\Zformal=\{Z_n\}$ be a formal $\Gbar$-scheme
and $\{\cA_{n+1} \onto \cA_n\}$ be the associated sequence of quasi-coherent
$\Gbar$-algebras. Then for every $n > n_{\lambda}$ the surjection
\[
    \cA_{n}[\lambda] \onto \cA_{n-1}[\lambda]
\]
is an isomorphism. If $\lambda_0\in \Irr(\Gbar)$ is the
trivial representation, then we may take $n_{\lambda_0}=0$.
\end{lemma}
\begin{proof}[Proof of Lemma~\ref{ref:stability:lem}]
    The claims are preserved under field extension, so we may assume our field
    is algebraically closed (hence perfect) so we may use the Kempf's torus.
    Fix a grading on
    $\kk[\Gbar]$ induced by a Kempf's torus for $\kk$ as in
    Corollary~\ref{ref:KempfTorus:cor}. Denote by $A_{\lambda}\subseteq \NN$
    the set of weights which appear in
    $\kk[\Group]_{\lambda}$. Since $\dim_{\kk}\kk[\Group]_{\lambda}$
    is finite by Proposition~\ref{ref:isotypiccomponents:prop}, the set
    $A_{\lambda}$ is finite. Put
    \[
        n_{\lambda}=\sup A_{\lambda}.
    \]
    Fix $n> n_{\lambda}$ and let
    $\cI_n = \ker(\cA_n\to \cA_0)$. Then we have a decomposition
    with respect to the chosen torus
    \[
        \cA_n=\bigoplus_{i\geq 0}(\cA_n)[i],
    \]
    By Corollary~\ref{ref:KempfTorus:cor}, we have $\cI_n =
    \bigoplus_{i\geq 1}(\cA_n)[i]$. Since $n > n_{\lambda}$ we have
    \[
        \cI^{n}_n \subset \bigoplus_{i\geq
        n}(\cA_n)[i]\subseteq \bigoplus_{i \not \in
            A_{\lambda}}(\cA_n)[i]
    \]
Hence, $\cI^{n}_n[\lambda] = 0$. But
$\cI^{n}_n[\lambda] = \ker(\cA_{n}[\lambda] \to
\cA_{n-1}[\lambda])$, thus $\cA_{n}[\lambda] \to \cA_{n-1}[\lambda]$ is an
isomorphism.
Finally note that $A_{\lambda_0}=\{0\}$. This implies that $n_{\lambda_0}=0$.
\end{proof}

\begin{proof}[Proof of Theorem~\ref{ref:algebraizationOfFormalSchemes:thm}]
    Let $\cA_n$ be the quasi-coherent $\Gbar$-algebras as
    in~\eqref{eq:Andefinition}. For $\lambda\in \Irr(\Gbar)$ we define
    $\cA[\lambda] := \cA_n[\lambda]$, where $n\geq n_{\lambda}$ as in
    Lemma~\ref{ref:stability:lem}.
    \[
        \cA=\bigoplus_{\lambda\in
            \Irr(\Gbar)}\cA[\lambda]=\bigoplus_{\lambda\in
                \Irr(\Gbar)}\cA_{n_{\lambda}}[\lambda].
    \]
    Clearly $\cA[\lambda_0] = \cA_0 = \cO_{Z_0}$ canonically (where
    $\lambda_0$ is the trivial representation), hence $\cA$ is an
    $\cO_{Z_0}$-module.
    Actually $\cA=\lim_{n}\cA_n$ in the category of quasi-coherent
    $\Gbar$-modules on $Z_0$.
    We construct the algebra structure on $\cA$. For this
    pick $\eta_1, \eta_2\in \Irr(\Gbar)$. Fix the finite set
    $\{\lambda_1, \ldots ,\lambda_s\}\subseteq \Irr(\Gbar)$ of representations
    which appear in $\kk[\Gbar]_{\eta_1}\otimes_{\kk}\kk[\Gbar]_{\eta_2}$.
    Then, for every $n\in \NN$, we have the multiplication
$$\cA_n[\eta_1]\otimes_{\kk} \cA_n[\eta_2]\ra \cA_n[\eta_1]\cdot \cA_n[\eta_2]\subseteq \bigoplus_{i=1}^s\cA_n[\lambda_i]$$
and by Lemma \ref{ref:stability:lem} these morphisms can be identified for $n\geq \sup \{n_{\eta_1},n_{\eta_2},n_{\lambda_1},...,n_{\lambda_s}\}$. We define
$$\cA[\eta_1]\otimes_{\kk} \cA[\eta_2]\ra  \bigoplus_{i=1}^s\cA[\lambda_i]\subseteq \cA$$
as a morphism induced by the multiplication morphism for any $n\geq \sup
\{n_{\eta_1},n_{\eta_2},n_{\lambda_1}, \ldots ,n_{\lambda_s}\}$. This gives an
$\cO_{Z_0}$-algebra structure on $\cA$, so $\cA$ is in fact the limit of
$\cA_n$ is the category of $\Gbar$-algebras. Note that from the description of
$\cA$ it follows that for every $n\in \NN$ we have a surjective morphism
$p_n:\cA\onto \cA_n$ of $\Gbar$-algebras. We denote its kernel
by $\cJ_n$ and we put $\cJ:=\cJ_0$. The natural injection $\cO_{Z_0} = \cA_0 \to \cA$ is a section
of $p_0$, so that we have
\[
    \cJ=\bigoplus_{\lambda \in \Irr(\Gbar)\setminus
        \{\lambda_0\}}\cA[\lambda].
\]
We also denote by $\cI_n$ the kernel of $\cA_n\twoheadrightarrow
\cA_0=\cO_{Z_0}$ for $n\in \NN$. Then $\cI_n=\cJ/\cJ_n$.
%Since $\Zformal$ is a
%formal $\Gbar$-scheme, the ideal $\cI_n^{n+1}$ cuts $Z_n$ out of $Z_n$, hence
%$\cI_n^{n+1} = 0$. Consequently, $\cJ^{n+1} \subset
%\cJ_n$\jjtodo{to wydaje się niepotrzebne, bo potem i tak pokazujemy równość}.
Fix $m\in \NN$ and consider $n\in \NN$
such that $n\geq m$. Since $\Zformal$ is a formal $\Gbar$-scheme, the sheaf
$\cI_n^{m+1}$ is the kernel of the morphism $\cA_n\twoheadrightarrow \cA_m$.
Thus
\[
\cJ_m/\cJ_n=\cI_n^{m+1}=(\cJ^{m+1}+\cJ_n)/\cJ_n.
\]
Both $\cJ_m$ and $\cJ^{m+1}$ are $\Irr(\Gbar)$-graded and for given
$\lambda\in \Irr(\Gbar)$ and $n\gg 0$ the isotypic component
$\cJ_n[\lambda]$ is zero by Lemma~\ref{ref:stability:lem}. Hence $\cJ_m=\cJ^{m+1}$ for every $m \in \NN$.
We define
\[
    Z=\Spec_{Z_0}(\cA)
\]
and we denote by $\pi:Z\to Z_0$ the structural morphism. The scheme $Z$
inherits a $\Gbar$-action from $\cA$. For
every $n\in \NN$ the zero-set of $\cJ^{n+1}\subseteq \cA$ is a $\Gbar$-scheme
isomorphic to $Z_n$. Hence $\Zformal$ is isomorphic to $\Zhat$.
Thus $Z$ is an algebraization of $\Zformal$. Since $\cA=\lim \cA_n$, we
have $Z = \colim Z_n$ in the category of locally linear $\Gbar$-schemes.

It remains to prove uniqueness of algebraization. Let $Z' = \Spec_{Z_0} \cA'$
be an algebraization of $\Zformal = \{Z_n\}$. Then $Z_n \into Z'$, so by the
universal property of colimit, we obtain a $\Gbar$-morphism $Z\to Z'$,
corresponding to $\cA' \to \cA$. It induces epimorphisms $\cA' \onto \cA_n$
for all $n$. For each $\lambda\in \Irr(\Gbar)$, the composition
\[
    \cA'[\lambda]\to \cA[\lambda]  \simeq \cA_{n_\lambda}[\lambda]
\]
is an epimorphism, hence $\cA'\to \cA$ is an epimorphism. The kernel of
$\cA'\to \cA$ is equal to
\[
    \bigcap_n \ker(\cA'\to \cA_n) = \bigcap_n \ker(\cA' \to \cA_0)^n.
\]
To prove that this kernel is zero, we may enlarge the field to an
algebraically closed field, so the result follows from
Corollary~\ref{ref:KempfTorus:cor}.

Assume that each scheme $Z_n$ is locally Noetherian over $\kk$. Then $\cI_n$
is a coherent $\cA_n$-module, thus $\cI_n^i/\cI^{i+1}$ is a coherent
$\cA_0$-module for all $i$. The series
\[
    0 = \cI_n^{n+1} \subset \cI^n \subset  \ldots \subset \cI \subset \cA_n
\]
has coherent subquotients, hence $\cA_n$ is a coherent $\cO_{Z_n}$-algebra.
Thus $\cA[\lambda]$ is a coherent $\cO_{Z_0}$-module for every
$\lambda\in \Irr(\Gbar)$. The claim that $\pi$ is of finite type is local on
$Z^{\Group}$, hence we may
assume that $Z^{\Group}$ is quasi-compact.
The sheaf $\cJ/\cJ^2\subseteq \cA_1$ is coherent so there exists a finite set
$\lambda_1, \ldots, \lambda_r\in \Irr(\Gbar)\setminus \{\lambda_0\}$ such that the morphism
\[
    \bigoplus_{i=1}^r\cA[\lambda_i]\ra \cJ/\cJ^2
\]
induced by $\cA\twoheadrightarrow \cA_2$ is surjective. Let $\cB \subset \cA$
be the quasi-coherent $\cO_{Z_0}$-subalgebra generated by the coherent
subsheaf $\cM := \bigoplus_{i=1}^r\cA[\lambda_i]\subseteq \cA$.
Let $\kkbar$ be an algebraic closure of $\kk$ and let $\cA' = \cA \otimes
\kkbar$. Fix a Kempf's torus over
$\kkbar$ and the associated grading $\cA' = \bigoplus_{i\geq 0}
\cA'[i]$ as in
Corollary~\ref{ref:KempfTorus:cor}.
Then $\cJ = \bigoplus_{i\geq 1} \cA'[i]$ is a graded ideal and $\cJ/\cJ^2$ is
generated by the graded (coherent) subsheaf $\cM' = \bigoplus_{i=1}^r\cA'[\lambda_i]$. By
graded Nakayama's lemma, the ideal $\cJ$ itself is generated by (the elements
of) $\cM'$. Then by induction on the degree, $\cA'$ is generated by $\cM'$ as
an algebra. In other words, $\cA' = \cB\otimes \kkbar$. Thus also $\cA = \cB$ and so $\cA$ is of
finite type over $\cO_{Z_0}$.
\end{proof}

\section{Algebraization of formal $\bd{M}$-schemes}
\noindent
Now we prove the main result.

\begin{theorem}\label{theorem:algebraization}
Let $\bd{M}$ be a Kempf monoid with unit group $\bd{G}$ and let $\cZ = \{Z_n\}_{n\in \NN}$ be a formal $\bd{M}$-scheme. Then there exists an algebraization $Z$ of $\cZ$.
Moreover, the following assertions hold.
\begin{enumerate}[label=\emph{\textbf{(\arabic*)}}, leftmargin=1.5em]
\item $Z$ is $\bd{M}$-scheme.
\item The canonical morphism $\pi:Z\ra Z_0$ is an affine morphism.
\end{enumerate}
Moreover, if $\cZ$ is locally noetherian, then $\pi$ is of finite type.
\end{theorem}
\noindent
Let $\bd{G}$ be the group of units of $\bd{M}$. According to the fact that $\bd{M}$ is integral, we derive that $\bd{G}$ is schematically dense in $\bd{M}$ and hence $Z_0$ admits trivial $\bd{M}$-action. Since $\bd{M}$ has zero, formal $\bd{M}$-scheme $\cZ$ corresponds to a sequence
\begin{center}
\begin{tikzpicture}
[description/.style={fill=white,inner sep=2pt}]
\matrix (m) [matrix of math nodes, row sep=3em, column sep=3em,text height=1.5ex, text depth=0.25ex] 
{ ... &  \cA_{n+1} & \cA_n & ...& \cA_1 & \cA_0 = \cO_{Z_0} \\} ;
\path[->>,line width=1.0pt,font=\scriptsize]  
(m-1-1) edge node[above] {$ $} (m-1-2)
(m-1-2) edge node[above] {$ $} (m-1-3)
(m-1-3) edge node[above] {$ $} (m-1-4)
(m-1-4) edge node[above] {$ $} (m-1-5)
(m-1-5) edge node[above] {$ $} (m-1-6);
\end{tikzpicture}
\end{center}
of $\bd{M}$-quasi-coherent $\cO_{Z_0}$-algebras such that $Z_n = \Spec_{Z_0}\cA_n$ for every $n\in \NN$. Next since $\bd{M}$ is a Kempf monoid, there exists a closed subgroup $T$ of the center $Z(\bd{G})$ such that $T$ is a torus and the scheme-theoretic closure $\ol{T}$ of $T$ in $\bd{M}$ contains the zero $\bd{o}$ of $\bd{M}$. Denote by $\Lambda$ the isomorphism classes of irreducible representations of $T$. The proof of the theorem is based on the following results.

\begin{lemma}\label{lemma:affine_toric_geometry_extracted}
Let $K$ be an algebraically closed field over $k$. Then the following assertions hold.
\begin{enumerate}[label=\emph{\textbf{(\arabic*)}}, leftmargin=1.5em]
\item $\ol{T}_K = \Spec K\times_{\Spec k}\ol{T}$ is an affine toric variety with torus $T_K$.
\item There exists an abstract submonoid $S$ of the character group $\cX(T_K)$ such that
$$\ol{T}_K = \Spec K[S]$$
as $T_K$-varieties, where $K[S]$ is monoid $K$-algebra of $S$.
\item $K$-subspace
$$\ideal{m} = \bigoplus_{\chi \in S\setminus \{0\}}K \cdot \chi\subseteq \bigoplus_{\chi \in S}K \cdot \chi = K[S]$$
is an ideal of $K[S]$ and for every finite subset $A$ of $\cX(T_K)$ there exists $n_A\in \NN$ such that for all $n\geq n_A$ we have
$$K\cdot \chi \not \in \ideal{m}^{n}$$
for every $\chi \in A$.
\end{enumerate}
\end{lemma}
\begin{proof}[Proof of the lemma]
Note that $T\hookrightarrow \ol{T}$ is schematically dense open $T$-equivariant immersion of affine $k$-schemes. Since $K$ is flat over $k$, we derive that $T_K\hookrightarrow \ol{T}_K$ is schematically dense open $T_K$-immersion. Thus $\ol{T}_K$ is integral affine $T_K$-scheme and hence $\ol{T}_K$ is a affine toric variety having $T_K$ as a torus. Moreover, zero of $\bd{M}$ is contained in $\ol{T}$. Thus $\ol{T}_K$ admits a fixed $K$-point with respect to $T_K$-action. Now the lemma follows by basic affine toric geometry. \lstodo{dokładniej ten dowód, lub odniesienie do KOKSA. Nie ma Kempfa HAHA!}
\end{proof}

\begin{lemma}[Stabilization]\label{lemma:stabilization_lemma}
Fix $\lambda$ in $\Lambda$. Then there exists a number $n_{\lambda}\in \NN$ such that the following holds. Let $\Zformal=\{Z_n\}$ be a formal $\bd{M}$-scheme and $\{\cA_{n+1} \onto \cA_n\}_{n\in \NN}$ be the associated sequence of quasi-coherent $\bd{M}$-algebras. Then for every $n > n_{\lambda}$ the surjection
\begin{center}
\begin{tikzpicture}
[description/.style={fill=white,inner sep=2pt}]
\matrix (m) [matrix of math nodes, row sep=3em, column sep=3em,text height=1.5ex, text depth=0.25ex] 
{\cA_{n+1}[\lambda] & \cA_n[\lambda]\\} ;
\path[->>,line width=1.0pt,font=\scriptsize]  
(m-1-1) edge node[above] {$ $} (m-1-2);
\end{tikzpicture}
\end{center}
is an isomorphism. If $\lambda_0$ is the isomorphism type of trivial representation of $\bd{G}$, then $n_{\lambda_0} = 0$.
\end{lemma}
\begin{proof}[Proof of the lemma]
Fix an algebraically closed field $K$ over $k$. Consider an irreducible representation $V$ of $T$ in $\lambda$. Then $V_K = K\otimes_kV$ is an irreducible representation of $T_K$. Since $V_K$ is finite dimensional over $K$, it follows that $V_K$ as a $T_K$-representation decomposes onto finitely many character spaces of $T_K$. Say
$$V_K = \bigoplus_{\chi \in A_{\lambda}}V_K[\chi]$$
for some finite subset $A_{\lambda}\subseteq \cX(T_K)$, where $\cX(T_K)$ is the character group of $T_K$. Next we use Lemma \ref{lemma:affine_toric_geometry_extracted}. We can write
$$\ol{T}_K = \Spec K[S],\ideal{m} = \bigoplus_{\chi \in S\setminus \{0\}}K \cdot \chi\subseteq \bigoplus_{\chi \in S}K \cdot \chi = K[S]$$
and moreover, there exists $n_{\lambda}\in \NN$ such that for every $n\geq n_{\lambda}$ we have $K\cdot \chi \not \in \ideal{m}^n$ for every $\chi\in A_{\lambda}$. Next $\cA_n$ is a quasi-coherent $\ol{T}$-sheaf on $Z_0$ for every $n\in \NN$. Hence
$$K\otimes_k\cA_n = \bigoplus_{\chi \in S}\left(K\otimes_k\cA_n\right)[\chi]$$
for every $n\in \NN$.
\end{proof}
















\small
\bibliographystyle{apalike}
\bibliography{../zzz}

\end{document}
