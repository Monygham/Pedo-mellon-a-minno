\input ../pree.tex

\usepackage{todonotes}
\newcommand{\lstodo}[1]{\todo[color=green!40,bordercolor=green,size=\footnotesize]{\textbf{LS TODO: }#1}}

\begin{document}

\title{Algebraization of formal $\bd{M}$-schemes}
\date{}
\maketitle

\section{Introduction}
\noindent
In these notes we prove results concerning algebraization of formal schemes in equivariant setting. First we describe $2$-limits of telescopes of categories. Then we introduce formal $\bd{M}$-schemes for a monoid $k$-scheme $\bd{M}$ and the notion of algebraization of such a formal object, which is an essential application of the notion of $2$-categorical limit for the telescope of categories of coherent sheaves associated with this formal scheme. In next sections we prepare ground to prove algebraization results. For this we discuss locally linear $\bd{M}$-schemes and relate them to formal $\bd{M}$-schemes. The last section of these notes is devoted to prove algebraization theorems for formal $\bd{M}$-schemes for Kempf monoid $k$-schemes. 

\section{Some $2$-categorical limits}
\noindent
Consider a category $\cC$ and an endofunctor $T:\cC\ra \cC$. Our goal is to construct certain $2$-categorical limit associated with a pair $(\cC,T)$. Consider pairs $\left(X,u\right)$ consisting of an object $X$ of $\cC$ and an isomorphism $u:T(X)\ra X$ in $\cC$. If $\left(X,u\right)$ and $\left(Y,w\right)$ are two such pairs, then a morphism $f:(X,u) \ra (Y,u)$ is a morphism $f:X\ra Y$ in $\cC$ such that the following square
\begin{center}
\begin{tikzpicture}
[description/.style={fill=white,inner sep=2pt}]
\matrix (m) [matrix of math nodes, row sep=3em, column sep=3em,text height=1.5ex, text depth=0.25ex] 
{ T(X) &  X    \\
  T(Y) &  Y           \\} ;
\path[->,line width=1.0pt,font=\scriptsize]  
(m-1-1) edge node[above] {$ u  $} (m-1-2)
(m-2-1) edge node[below] {$ w $} (m-2-2)
(m-1-1) edge node[left] {$ T(f) $} (m-2-1)
(m-1-2) edge node[right] {$ f  $} (m-2-2);
\end{tikzpicture}
\end{center}
is commutative. This data give rise to a category $\cC(T)$. There exists a forgetful functor $\pi:\cC(T)\ra \cC$ that sends a morphism $f:(X,u)\ra (Y,w)$ to $f:X\ra Y$. Moreover, there exists a natural isomorphism $\sigma:T\cdot \pi \Rightarrow \pi$ such that the component of $\sigma$ on an object $(X,u)$ of $\cC(T)$ is $u$. The next result states that the data above form a certain $2$-categorical limit.

\begin{theorem}\label{theorem:endoscope_2_limits}
Let $(\cC,T)$ be a pair consiting of a category and an endofunctor $T:\cC\ra \cC$. Suppose that $\cD$ is a category, $P:\cD\ra \cC$ is a functor and $\tau:T\cdot P \Rightarrow P$ is a natural isomorphisms. Then there exists a unique functor $F:\cD\ra \cC(T)$ such that $P = \pi\cdot F$ and $\sigma_F = \tau$.
\end{theorem}
\begin{proof}
Suppose that $F:\cD\ra \cC(T)$ is a functor such that $P = \pi\cdot F$ and $\sigma_F = \tau$. Pick an object $X$ of $\cD$. Then we have $\pi\left(F(X)\right) = P(X)$ and $\sigma_{F(X)} = v\tau_X$. This implies that
$$F(X) = \left(P(X),\tau_X:T(P(X))\ra P(X)\right)$$
Next if $f:X\ra Y$ is a morphism in $\cD$, then we derive that $\pi(F(f)) = P(f)$. Hence $F(f) = P(f)$. This implies that there exists at most one functor $F$ satisfying the properties above. Note also that formulas
$$F(X) = \left(P(X),\tau_X:T(P(X))\ra P(X)\right),\,F(f) = P(f)$$
for an object $X$ in $\cD$ and a morphism $f:X\ra Y$ in $\cD$, give rise to a functor that satisfy $P = \pi\cdot F$ and $\sigma_F = \tau$. This establishes existence and the uniqueness of $F$.
\end{proof}
\noindent
Assume now that the pair $(\cC,T)$ consists of a monoidal category $\cC$ and a monoidal endofunctor $T$. Then there exists a canonical monoidal structure on $\cC(T)$. We define $(-)\otimes_{\cC(T)}(-)$ by formula
$$(X,u) \otimes_{\cC(T)}(Y,w) = \left(X\otimes_{\cC}Y,\left(u \otimes_{\cC} w\right)\cdot m_{X,Y}\right)$$
where
$$m_{X,Y}:T\left(X\otimes_{\cC} Y\right) \ra T(X)\otimes_{\cC}T(Y)$$
is the tensor preserving isomorphism of $T$. We also define the unit
$$I_{\cC(T)} = \left(I, T(I)\cong I\right)$$
where isomorphism $T(I)\cong I$ is precisely the unit preserving isomorphism of the monoidal functor $T$. The associativity natural isomorphism for $(-)\otimes_{\cC(T)}(-)$ and right, left units for $I_{\cC(T)}$ in $\cC(T)$ are associavity natural isomorphism and right, left units for $\cC$, respectively. The structure makes a functor $\pi:\cC(T)\ra \cC$ strict monoidal and $\sigma$ a monoidal natural isomorphism. The next result states that the data with these extra monoidal structure form a $2$-categorical limit in the $2$-category of monoidal categories.

\begin{theorem}\label{theorem:endoscope_monoidal_2_limit}
Let $(\cC,T)$ be a pair consiting of a monoidal category and its monoidal endofunctor $T:\cC\ra \cC$. Suppose that $\cD$ is a monoidal category, $P:\cD\ra \cC$ is a monoidal functor and $\tau:T\cdot P \Rightarrow P$ is a monoidal natural isomorphisms. Then there exists a unique monoidal functor $F:\cD\ra \cC(T)$ such that $P = \pi\cdot F$ and $\sigma_F = \tau$ as monoidal functors and monoidal transformations.
\end{theorem}
\begin{proof}
Note that $F$ must be defined as it was described in the proof of Theorem \ref{theorem:endoscope_2_limits}. Namely we must have
$$F(X) = \left(P(X),\tau_X:T(P(X))\ra P(X)\right),\,F(f) = P(f)$$
for an object $X$ in $\cC$ and a morphism $f:X\ra Y$ in $\cC$. Suppose now that $F$ admits a structure of a monoidal functor such that $P = \pi\cdot F$ as monoidal functors. Let
$$\big\{m^F_{X,Y}:F(X\otimes_{\cD}Y)\ra F(X)\otimes_{\cC(T)}F(Y)\big\}_{X,Y\in \cC},\,\phi^F:F(I_{\cD})\ra I_{\cC(T)}$$
be the data forming that structure. Since $\pi$ is a strict monoidal functor and $P = \pi\cdot F$ as monoidal functors, we derive that for any objects $X,Y$ of $\cC$
$$\pi(m^F_{X,Y}):P(X\otimes_{\cD}Y) \ra P(X)\otimes_{\cC}P(Y)$$
is the tensor preserving isomorphism $m^{P}_{X,Y}:P(X\otimes_{\cD}Y) \ra P(X)\otimes_{\cC}P(Y)$ of the monoidal functor $P$. By the same argument
$$\pi(\phi_F):P(I_{\cD})\ra I_{\cC(T)}$$
is the unit preserving isomorphism $\phi^{P}:P(I_{\cD})\ra I_{\cC(T)}$ of $P$. Thus we deduce that for any objects $X,Y$ of $\cC$ we have $m^F_{X,Y} = m^{P}_{X,Y}$ and $\phi^F = \phi^{P}$. This implies that there exists at most one monoidal functor $F$ such that $P = \pi\cdot F$ as monoidal functors. On the other hand define $m^F_{X,Y} = m^{P}_{X,Y}$ for objects $X,Y$ in $\cC$ and $\phi^F = \phi^{P}$. We check now that $F$ equipped with these data is a monoidal functor. Fix objects $X,Y$ in $\cC$. The square
\begin{center}
\begin{tikzpicture}
[description/.style={fill=white,inner sep=2pt}]
\matrix (m) [matrix of math nodes, row sep=4em, column sep=8em,text height=1.5ex, text depth=0.25ex] 
{T\left(P\left(X\otimes_{\cD}Y\right)\right)  & P\left(X\otimes_{\cC}Y\right)     \\
 T\left(P(X)\otimes_{\cC}P(Y)\right)  & P(X)\otimes_{\cC}P(Y)  \\} ;
\path[->,line width=1.0pt,font=\scriptsize]  
(m-1-1) edge node[above] {$ \tau_{X\otimes_{\cC}Y}  $} (m-1-2)
(m-2-1) edge node[below] {$ \left(\tau_X\otimes_{\cC}\tau_Y\right)\cdot m^{T}_{P(X),P(Y)} $} (m-2-2)
(m-1-1) edge node[left] {$ T\left(m^{P}_{X,Y}\right) $} (m-2-1)
(m-1-2) edge node[right] {$  m^{P}_{X,Y} $} (m-2-2);
\end{tikzpicture}
\end{center}
is commutative due to the fact that $\tau:T\cdot P \Rightarrow P$ is a monoidal natural isomorphisms. This implies that $m^F_{X,Y}$ is a morphism in $\cC(T)$. It follows that $m^F_{X,Y}$ is a natural isomorphism and due to the definition of associativity in $\cC(T)$, we derive its compatibility with $m^F_{X,Y}$. Similarly, since the square
\begin{center}
\begin{tikzpicture}
[description/.style={fill=white,inner sep=2pt}]
\matrix (m) [matrix of math nodes, row sep=3em, column sep=3em,text height=1.5ex, text depth=0.25ex] 
{T\left( P\left(I_{\cD}\right) \right)  & P\left(I_{\cD}\right)     \\
 T\left( I_{\cC} \right)  & I_{\cC}  \\} ;
\path[->,line width=1.0pt,font=\scriptsize]  
(m-1-1) edge node[above] {$ \tau_{I_{\cD}}  $} (m-1-2)
(m-2-1) edge node[below] {$ \phi^{T}  $} (m-2-2)
(m-1-1) edge node[left] {$ T\left(\phi^{P}\right)  $} (m-2-1)
(m-1-2) edge node[right] {$ \phi^{P}  $} (m-2-2);
\end{tikzpicture}
\end{center}
is commutative, we deduce that $\phi^F$ is a morphism in $\cC(T)$. By definition of left and right unit in $\cC(T)$, we derive their compatibility with $\phi^F$. This finishes the verification of the fact that $F$ with $\{m^F_{X,Y}\}_{X,Y\in \cC}$ and $\phi^F$ is a monoidal functor. Definitions of $\{m^F_{X,Y}\}_{X,Y\in \cC}$ and $\phi^F$ show that the identities $P = \pi\cdot F$ holds on the level of monoidal structures. Since the $2$-forgetful functor from $2$-category of monoidal categories into $2$-category of categories is faithful on $2$-cells, the identity $\sigma_F = \tau$ of natural isomorphisms is also the identity of monoidal natural isomorphisms.
\end{proof}

\begin{theorem}\label{theorem:endoscope_colimits}
Let $(\cC,T)$ be a pair consiting of a category and its endofunctor $T:\cC\ra \cC$. Assume that $T$ preserves colomits. Then the following assertions hold.
\begin{enumerate}[label=\textbf{\emph{(\arabic*)}}, leftmargin=3.0em]
\item $\pi:\cC(T)\ra \cC$ creates colimits.
\item Suppose that $\cD$ is a category, $P:\cD\ra \cC$ a functor preserving small colimits and $\tau:T\cdot P \Rightarrow P$ a natural isomorphisms. Then the unique functor $F:\cD \ra \cC(T)$ such that $P = \pi\cdot F$ and $\sigma_F = \tau$ preserves small colimits.
\end{enumerate}
\end{theorem}
\begin{proof}
Let $I$ be a small category and $D:I\ra \cC(T)$ be a diagram such that the composition $\pi\cdot D:I\ra \cC$ admits a colimit given by cocone $(X,\{g_i\}_{i\in I})$. Since $T$ preserves colimits, we derive that $\left(T(X), \{T(u_i)\}_{i\in I}\right)$ is a colimit of $T\cdot \pi \cdot D:I\ra \cC$. Now $\sigma_D:T\cdot \pi\cdot D\ra \pi\cdot D$ is a natural isomorphism. Hence there exists a unique arrow $u:T(X)\ra X$ such that $u \cdot T(g_i) = g_i\cdot \sigma_{D(i)}$ for $i\in I$. Clearly $u$ is an isomorphism and hence $(X,u)$ is an object of $\cC(T)$. Moreover, the family $\{g_i\}_{i\in I}$ together with $(X,u)$ is a colimiting cocone over $D$. This proves \textbf{(1)}. Now \textbf{(2)} is a consequence of \textbf{(1)}.
\end{proof}
\noindent
Now we apply the results above to certain more general diagrams of categories.

\begin{definition}
A diagram
\begin{center}   
\begin{tikzpicture}
[description/.style={fill=white,inner sep=2pt}]
\matrix (m) [matrix of math nodes, row sep=3em, column sep=2em,text height=1.5ex, text depth=0.25ex] 
{... &  \cC_{n+1}  &  \cC_n & ... & \cC_2 & \cC_1 & \cC_0  \\};
\path[->,line width=1.0pt,font=\scriptsize]    
(m-1-1) edge node[auto]  {$F_{n+1}  $} (m-1-2)
(m-1-2) edge node[auto]  {$F_n  $} (m-1-3)
(m-1-3) edge node[auto]  {$F_{n-1}  $} (m-1-4)
(m-1-4) edge node[auto]  {$F_2  $} (m-1-5)
(m-1-5) edge node[auto]  {$F_1  $} (m-1-6)
(m-1-6) edge node[auto]  {$F_0  $} (m-1-7);
\end{tikzpicture}
\end{center}
of categories and functors is called \textit{a telescope of categories}.
\end{definition}

\begin{definition}
Let 
\begin{center}   
\begin{tikzpicture}
[description/.style={fill=white,inner sep=2pt}]
\matrix (m) [matrix of math nodes, row sep=3em, column sep=2em,text height=1.5ex, text depth=0.25ex] 
{... &  \cC_{n+1}  &  \cC_n & ... & \cC_2 & \cC_1 & \cC_0  \\};
\path[->,line width=1.0pt,font=\scriptsize]    
(m-1-1) edge node[auto]  {$F_{n+1}  $} (m-1-2)
(m-1-2) edge node[auto]  {$F_n  $} (m-1-3)
(m-1-3) edge node[auto]  {$F_{n-1}  $} (m-1-4)
(m-1-4) edge node[auto]  {$F_2  $} (m-1-5)
(m-1-5) edge node[auto]  {$F_1  $} (m-1-6)
(m-1-6) edge node[auto]  {$F_0  $} (m-1-7);
\end{tikzpicture}
\end{center}
be a telescope of monoidal categories and monoidal (finitely) cocontinuous functors. Then \textit{a $2$-categorical limit of the telescope} consists of a monoidal category $\cC$, a family of monoidal (finitely) cocontinuous functors $\{\pi_n:\cC\ra \cC_n\}_{n\in \NN}$ and a family of monoidal natural isomorphisms $\{\sigma_n:F_{n+1} \cdot \pi_{n+1}\Rightarrow \pi_n\}_{n\in \NN}$ such that the following universal property holds. For any monoidal category $\cD$, family $\{P_n:\cD\ra \cC_n\}_{n\in \NN}$ of (finitely) cocontinuous monoidal functors and a family $\{\tau_n:F_nP_{n+1}\Rightarrow P_{n}\}_{n\in \NN}$ of monoidal natural isomorphisms there exists a unique monoidal (finitely) cocontinuous functor $F:\cD\ra \cC$ satisfying $P_n = \pi_n \cdot F$ and $\left(\sigma_n\right)_F = \tau_n$ for every $n\in \NN$.
\end{definition}

\begin{corollary}\label{corollary:telescope_2_limits}
Let 
\begin{center}   
\begin{tikzpicture}
[description/.style={fill=white,inner sep=2pt}]
\matrix (m) [matrix of math nodes, row sep=3em, column sep=2em,text height=1.5ex, text depth=0.25ex] 
{... &  \cC_{n+1}  &  \cC_n & ... & \cC_2 & \cC_1 & \cC_0  \\};
\path[->,line width=1.0pt,font=\scriptsize]    
(m-1-1) edge node[auto]  {$F_{n+1}  $} (m-1-2)
(m-1-2) edge node[auto]  {$F_n  $} (m-1-3)
(m-1-3) edge node[auto]  {$F_{n-1}  $} (m-1-4)
(m-1-4) edge node[auto]  {$F_2  $} (m-1-5)
(m-1-5) edge node[auto]  {$F_1  $} (m-1-6)
(m-1-6) edge node[auto]  {$F_0  $} (m-1-7);
\end{tikzpicture}
\end{center}
be a telescope of monoidal categories and monoidal (finitely) cocontinuous functors. Then its $2$-limit exists.
\end{corollary}
\begin{proof}
We decompose the task of constructing its $2$-limit as follows. First note that one may form a product $\cC = \prod_{n\in \NN}\cC_n$. Next the functors $\{F_n\}_{n\in \NN}$ induce an endofunctor $T = \prod_{n\in \NN}F_n\times t$, where $\bd{1}$ is the terminal category (it has single object and single identity arrow) and $t:\cC_0\rightarrow \bd{1}$ is the unique functor. Consider the category $\cC(T)$. We define $\{\pi_n:\cC(T)\ra \cC_n\}_{n\in \NN}$ to be a family of functors given by coordinates of $\pi:\cC(T)\ra \cC$ and $\{\sigma_n:F_n\cdot \pi_{n+1}\Rightarrow \pi_n\}_{n\in \NN}$ to be a family of natural isomorphisms given by coordinates of $\sigma:\pi\cdot T\Rightarrow \pi$. Now this data form a $2$-limit of the telescope by compilation of Theorem \ref{theorem:endoscope_monoidal_2_limit} and Theorem \ref{theorem:endoscope_colimits}.
\end{proof}
\noindent
It is worth to extract from previous results a more concrete description of the $2$-limit of a telescopes of categories.

\begin{remark}[$2$-limit of a telescope]\label{remark:2_limit_of_a_telescope_description}
Consider a telescope
\begin{center}   
\begin{tikzpicture}
[description/.style={fill=white,inner sep=2pt}]
\matrix (m) [matrix of math nodes, row sep=3em, column sep=2em,text height=1.5ex, text depth=0.25ex] 
{... &  \cC_{n+1}  &  \cC_n & ... & \cC_2 & \cC_1 & \cC_0  \\};
\path[->,line width=1.0pt,font=\scriptsize]    
(m-1-1) edge node[auto]  {$F_{n+1}  $} (m-1-2)
(m-1-2) edge node[auto]  {$F_n  $} (m-1-3)
(m-1-3) edge node[auto]  {$F_{n-1}  $} (m-1-4)
(m-1-4) edge node[auto]  {$F_2  $} (m-1-5)
(m-1-5) edge node[auto]  {$F_1  $} (m-1-6)
(m-1-6) edge node[auto]  {$F_0  $} (m-1-7);
\end{tikzpicture}
\end{center}
of categories. Then its $2$-limit is the category that can be described as follows. Its objects are pairs $\left(\{X_n\}_{n\in \NN},\{u_n\}_{n\in \NN}\right)$ consisting of a sequence $\{X_n\}_{n\in \NN}$ such that $X_n$ is an object of $\cC_n$ for every $n\in \NN$ and a sequence $\{u_n\}_{n\in \NN}$ such that $u_n:F_n(X_{n+1})\ra X_{n}$ is an isomorphism in $\cC_n$ for every $n\in \NN$. Next if $\left(\{X_n\}_{n\in \NN},\{u_n\}_{n\in \NN}\right)$ and $\left(\{Y_n\}_{n\in \NN},\{w_n\}_{n\in \NN}\right)$ are two objects in the $2$-limit, then a morphism between them consists of a sequence $\{f_n\}_{n\in \NN}$ of morphisms such that $f_n:X_n\ra Y_n$ is a morphism in $\cC_n$ for every $n\in \NN$ and squares
\begin{center}
\begin{tikzpicture}
[description/.style={fill=white,inner sep=2pt}]
\matrix (m) [matrix of math nodes, row sep=3em, column sep=3em,text height=1.5ex, text depth=0.25ex] 
{ F_n(X_{n+1}) &  X_n    \\
  F_n(Y_{n+1}) &  Y_n           \\} ;
\path[->,line width=1.0pt,font=\scriptsize]  
(m-1-1) edge node[above] {$ u_n  $} (m-1-2)
(m-2-1) edge node[below] {$ w_n $} (m-2-2)
(m-1-1) edge node[left] {$ F_n(f_{n+1}) $} (m-2-1)
(m-1-2) edge node[right] {$ f_n  $} (m-2-2);
\end{tikzpicture}
\end{center}
are commutative for every $n\in \NN$.
\end{remark}

\section{Formal $\bd{M}$-schemes}
\noindent
This section is devoted to introducing some notions from formal geometry that play a fundamental role in these notes. 

\begin{definition}
Let $\bd{M}$ be a monoid $k$-scheme. \textit{A formal $\bd{M}$-scheme} consists of a sequence $\cZ = \{Z_n\}_{n\in \NN}$ of $\bd{M}$-schemes together with $\bd{M}$-equivariant closed immersions
\begin{center}
\begin{tikzpicture}
[description/.style={fill=white,inner sep=2pt}]
\matrix (m) [matrix of math nodes, row sep=3em, column sep=3em,text height=1.5ex, text depth=0.25ex] 
{ Z_0 &  Z_1 & ... & Z_n & Z_{n+1} & ... \\} ;
\path[right hook->,line width=1.0pt,font=\scriptsize]  
(m-1-1) edge node[above] {$ $} (m-1-2)
(m-1-2) edge node[above] {$ $} (m-1-3)
(m-1-3) edge node[above] {$ $} (m-1-4)
(m-1-4) edge node[above] {$ $} (m-1-5)
(m-1-5) edge node[above] {$ $} (m-1-6);
\end{tikzpicture}
\end{center}
satisfying the following assertions.
\begin{enumerate}[label=\textbf{(\arabic*)}, leftmargin=3.0em]
\item We have $Z_0 = Z_n^{\bd{M}}$ scheme-theoretically for every $n\in \NN$.
\item Let $\cI_n$ be an ideal of $\cO_{Z_n}$ defining $Z_0$. Then for every $m\leq n$ the subscheme $Z_m \subset Z_n$ is defined by $\cI_n^{m+1}$.
\end{enumerate}
\end{definition}

\begin{example}\label{example:formal_neighborhood_of_fixed_pts}
Let $\bd{M}$ be a monoid $k$-scheme and let $Z$ be a $\bd{M}$-scheme. Consider a quasi-coherent ideal $\cI$ of fixed point subscheme $Z^{\bd{M}}$ of $Z$. Then for every $n\in \NN$ ideal $\cI^n$ is quasi-coherent $\bd{M}$-ideal and hence
\begin{center}
\begin{tikzpicture}
[description/.style={fill=white,inner sep=2pt}]
\matrix (m) [matrix of math nodes, row sep=3em, column sep=3em,text height=1.5ex, text depth=0.25ex] 
{ V(\cI) &  V(\cI^2) & ... & V(\cI^n) & ... \\} ;
\path[right hook->,line width=1.0pt,font=\scriptsize]  
(m-1-1) edge node[above] {$ $} (m-1-2)
(m-1-2) edge node[above] {$ $} (m-1-3)
(m-1-3) edge node[above] {$ $} (m-1-4)
(m-1-4) edge node[above] {$ $} (m-1-5);
\end{tikzpicture}
\end{center}
is a formal $\bd{M}$-scheme. We denote it by $\widehat{Z}$.
\end{example}

\begin{definition}
Let $\bd{M}$ be a monoid $k$-scheme and let $\cZ = \{Z_n\}_{n\in \NN}$ be a formal $\bd{M}$-scheme. We say that $\cZ$ is \textit{locally noetherian} if for all $n\in \NN$ scheme $Z_n$ is locally noetherian.
\end{definition}

\begin{definition}
Let $\bd{M}$ be a monoid $k$-scheme. Suppose that $\cZ = \{Z_n\}_{n\in \NN}$ and $\cW = \{W_n\}_{n\in \NN}$ are formal $\bd{M}$-schemes. Then \textit{a morphism $f:\cZ\ra \cW$ of formal $\bd{M}$-schemes} consists of a family of $\bd{M}$-equivariant morphisms $f = \big\{f_n:Z_n\ra W_n\}_{n\in \NN}$ such that the diagram
\begin{center}
\begin{tikzpicture}
[description/.style={fill=white,inner sep=2pt}]
\matrix (m) [matrix of math nodes, row sep=3em, column sep=3em,text height=1.5ex, text depth=0.25ex] 
{ Z_0 &  Z_1 & ... & Z_n & Z_{n+1} & ... \\
 W_0 &  W_1 & ... & W_n & W_{n+1} & ... \\} ;
\path[right hook->,line width=1.0pt,font=\scriptsize]  
(m-1-1) edge node[above] {$ $} (m-1-2)
(m-1-2) edge node[above] {$ $} (m-1-3)
(m-1-3) edge node[above] {$ $} (m-1-4)
(m-1-4) edge node[above] {$ $} (m-1-5)
(m-1-5) edge node[above] {$ $} (m-1-6)
(m-2-1) edge node[above] {$ $} (m-2-2)
(m-2-2) edge node[above] {$ $} (m-2-3)
(m-2-3) edge node[above] {$ $} (m-2-4)
(m-2-4) edge node[above] {$ $} (m-2-5)
(m-2-5) edge node[above] {$ $} (m-2-6);
\path[->,line width=1.0pt,font=\scriptsize]
(m-1-1) edge node[left] {$f_0 $} (m-2-1)
(m-1-2) edge node[left] {$f_1 $} (m-2-2)
(m-1-4) edge node[left] {$f_n $} (m-2-4)
(m-1-5) edge node[left] {$f_{n+1} $} (m-2-5);
\end{tikzpicture}
\end{center}
is commutative.
\end{definition}

\begin{definition}
Let $\bd{M}$ be a monoid $k$-scheme and let $\bd{G}$ be its group of units. Let $\cZ = \{Z_n\}_{n\in \mathbb{N}}$ be a locally noetherian formal $\bd{M}$-scheme. Then we have the corresponding telescope of monoidal categories
\begin{center}   
\begin{tikzpicture}
[description/.style={fill=white,inner sep=2pt}]
\matrix (m) [matrix of math nodes, row sep=3em, column sep=2em,text height=1.5ex, text depth=0.25ex] 
{... &  \Coh_{\bd{G}}(Z_{n+1})  &  \Coh_{\bd{G}}(Z_n) & ... & \Coh_{\bd{G}}(Z_2) & \Coh_{\bd{G}}(Z_1) & \Coh_{\bd{G}}(Z_0)  \\};
\path[->,line width=1.0pt,font=\scriptsize]    
(m-1-1) edge node[auto]  {$  $} (m-1-2)
(m-1-2) edge node[auto]  {$  $} (m-1-3)
(m-1-3) edge node[auto]  {$  $} (m-1-4)
(m-1-4) edge node[auto]  {$  $} (m-1-5)
(m-1-5) edge node[auto]  {$  $} (m-1-6)
(m-1-6) edge node[auto]  {$  $} (m-1-7);
\end{tikzpicture}
\end{center}
and finitely cocontinuous monoidal functors given by restricting $\bd{G}$-equivariant coherent sheaves to closed $\bd{G}$-subschemes. Then we define \textit{a category $\Coh_{\bd{G}}(\cZ)$ of coherent $\bd{G}$-equivariant sheaves on $\cZ$} as a monoidal category which is a $2$-limit of the telescope above. This category is defined uniquely up to a monoidal equivalence.
\end{definition}
\noindent
Fix now a monoid $k$-scheme $\bd{M}$ with $\bd{G}$ as a group of units. Let $Z$ be a locally noetherian $\bd{M}$-scheme and suppose that $Z^{\bd{M}}$ exists. Suppose that $\cI$ is a coherent ideal of $Z^{\bd{M}}$. We have a commutative diagram
\begin{center}
\begin{tikzpicture}
[description/.style={fill=white,inner sep=2pt}]
\matrix (m) [matrix of math nodes, row sep=3em, column sep=3em,text height=1.5ex, text depth=0.25ex] 
{ V(\cI) &  V(\cI^2) & ... & V(\cI^n) & ... \\
         &           & Z   &          &      \\} ;
\path[right hook->,line width=1.0pt,font=\scriptsize]  
(m-1-1) edge node[above] {$ $} (m-1-2)
(m-1-2) edge node[above] {$ $} (m-1-3)
(m-1-3) edge node[above] {$ $} (m-1-4)
(m-1-4) edge node[above] {$ $} (m-1-5)
(m-1-2) edge node[above] {$ $} (m-2-3);
\path[right hook->,bend right, line width=1.0pt,font=\scriptsize]  
(m-1-1) edge node[above] {$ $} (m-2-3);
\path[right hook->, line width=1.0pt,font=\scriptsize]  
(m-1-4) edge node[above] {$ $} (m-2-3);
\end{tikzpicture}
\end{center}
in the category of $\bd{M}$-schemes. Thus restriction functors $\Coh_{\bd{G}}(Z) \ra \Coh_{\bd{G}}(V(\cI^n))$ for $n\in \NN$ induce a unique finitely cocontinuous monoidal functor $\Coh_{\bd{G}}(Z)\ra \Coh_{\bd{G}}(\widehat{Z})$.

\begin{definition}
Let $Z$ be a locally noetherian $\bd{M}$-scheme such that $Z^{\bd{M}}$ exists. Let $\bd{G}$ be a group of units of $\bd{M}$. Then a unique finitely cocontinuous monoidal functor $\Coh_{\bd{G}}(Z)\ra \Coh_{\bd{G}}(\widehat{Z})$ is called \textit{the comparison functor}.
\end{definition}

\section{Locally linear $\bd{M}$-schemes}

\begin{definition}
Let $\bd{M}$ be a monoid $k$-scheme and let $X$ be a $\bd{M}$-scheme. Suppose that each point of $X$ admits an open affine $\bd{M}$-stable neighborhood. Then we say that $X$ is \textit{a locally linear $\bd{M}$-scheme}.
\end{definition}

\begin{proposition}\label{proposition:monoid_open_stable_correspondence}
Let $\bd{M}$ be a monoid $k$-scheme and let $X$ be a $\bd{M}$-scheme. Suppose that $Z$ is a closed $\bd{M}$-stable subscheme of $X$ defined by the ideal with nilpotent sections. Consider an open subset $U$ of $X$. Then the following are equivalent.
\begin{enumerate}[label=\emph{\textbf{(\arabic*)}}, leftmargin=3.0em]
\item $U$ is $\bd{M}$-stable.
\item Scheme-theoretic intersection $U\cap Z$ is $\bd{M}$-stable.
\end{enumerate}
\end{proposition}
\begin{proof}
Let $\alpha:\bd{M} \times_k X\ra X$ be the action of $\bd{M}$ on $X$. Fix open subset $U$ of $X$. If $U$ is $\bd{M}$-stable, then $U\cap Z$ is $\bd{M}$-stable. So suppose that $U\cap Z$ is $\bd{M}$-stable. Since ideal of $Z$ has nilpotent sections and $\bd{M}$ is affine, we derive that closed immersions $U\cap Z\hookrightarrow U$ and $\bd{M}\times_k \left(U\cap Z\right) \hookrightarrow \bd{M}\times_k U$ induce homeomorphisms on topological spaces. Consider the commutative diagram
\begin{center}
\begin{tikzpicture}
[description/.style={fill=white,inner sep=2pt}]
\matrix (m) [matrix of math nodes, row sep=3em, column sep=3em,text height=1.5ex, text depth=0.25ex] 
{ \bd{M}\times_k U & X     \\
  \bd{M}\times_k \left(U\cap Z\right) & U\cap Z   \\} ;
\path[->,line width=1.0pt,font=\scriptsize]  
(m-1-1) edge node[above] {$ \alpha_{\mid U\cap Z}  $} (m-1-2)
(m-2-1) edge node[below] {$   $} (m-2-2);
\path[right hook->,line width=1.0pt,font=\scriptsize]  
(m-2-1) edge node[left] {$  $} (m-1-1)
(m-2-2) edge node[right] {$ $} (m-1-2);
\end{tikzpicture}
\end{center}
where the bottom horizontal arrow is the induced action on $U\cap Z$ and vertical morphisms are homeomorphisms. The commutativity of the diagram implies that $\alpha\left(\bd{M}\times_k U\right)$ is contained set-theoretically in $U$. Since $U$ is open in $X$, we derive that morphism of schemes $\alpha_{\mid \bd{M}\times_k U}$ factors through $U$. Hence $U$ is $\bd{M}$-stable.
\end{proof}

\begin{corollary}\label{corollary:monoid_stable_open_affine_correspondence}
Let $\bd{M}$ be a monoid $k$-scheme and let $X$ be a $\bd{M}$-scheme. Suppose that $Z$ is a closed $\bd{M}$-stable subscheme of $X$ defined by the nilpotent ideal. Consider an open subset $U$ of $X$. Then the following are equivalent.
\begin{enumerate}[label=\emph{\textbf{(\arabic*)}}, leftmargin=3.0em]
\item $U$ is $\bd{M}$-stable and affine.
\item $U\cap Z$ is $\bd{M}$-stable and affine.
\end{enumerate}
\end{corollary}
\begin{proof}
Since ideal of $Z$ is nilpotent, we derive that $U$ is affine if and only if $U\cap Z$ is affine. Combining this with Proposition \ref{proposition:monoid_open_stable_correspondence}, we deduce the result.
\end{proof}

\begin{corollary}\label{corollary:locally_linear_are_stable_under_thickenings}
Let $\bd{M}$ be a monoid $k$-scheme and let $X$ be a $\bd{M}$-scheme. Suppose that $Z$ is a closed $\bd{M}$-stable subscheme of $X$ defined by the nilpotent ideal. Then $X$ is locally linear $\bd{M}$-scheme if and only if $Z$ is locally linear $\bd{M}$-scheme.
\end{corollary}
\begin{proof}
This is a consequence of Corollary \ref{corollary:monoid_stable_open_affine_correspondence}.
\end{proof}
\noindent
Let $\bd{G}$ be an affine group $k$-scheme. We describe quasi-coherent $\bd{G}$-sheaves on locally linear $\bd{G}$-schemes.

\begin{theorem}\label{theorem:coactions_and_equivariance_for_locally_linear_schemes}
Let $\bd{G}$ be an affine group $k$-scheme and let $X$ be a $k$-scheme equipped with an action $a:\bd{G}\times X\ra X$ of $\bd{G}$ that makes $X$ a locally linear $\bd{G}$-scheme. Let $\pi:\bd{G}\times_kX\ra X$ be the projection. Suppose that $\cF$ is a quasi-coherent sheaf on $X$. Assume that $\gamma:\cF\ra a_*\pi^*\cF$ is a morphism of quasi-coherent sheaves on $X$. Then the following are equivalent.
\begin{enumerate}[label= \emph{\textbf{(\roman*)}}, leftmargin=3.0em]
\item For every $\bd{G}$-stable open affine subscheme $U$ of $X$ consider the morphism
$$\cF(U)\ra k[\bd{G}]\otimes_k\cF(U)$$
determined as the composition of $\Gamma\left(U,\gamma\right)$ with the identification $\Gamma(U,\pi^*\cF) = k[\bd{G}]\otimes_k\cF(U)$. Then this morphism is a coaction of $k[\bd{G}]$ on $\cF(U)$.
\item Let $\tau$ be the image of $\gamma$ under the adjunction bijection
\begin{center}
\begin{tikzpicture}
[description/.style={fill=white,inner sep=2pt}]
\matrix (m) [matrix of math nodes, row sep=1em, column sep=2em,text height=1.5ex, text depth=0.25ex] 
{\Hom_{\cO_X}\left(\cF, a_*\pi^*\cF\right)  & \Hom_{\cO_{\bd{M}\times_kX}}\left(a^*\cF, \pi^*\cF\right) \\} ;
\path[->,line width=1.0pt,font=\scriptsize]  
(m-1-1) edge node[above] {$ $} (m-1-2);
\end{tikzpicture}
\end{center}
for $a^*\dashv a_*$. Then $\tau$ is invertible and $(\cF,\tau^{-1})$ is a quasi-coherent $\bd{G}$-sheaf on $X$.
\end{enumerate}
\end{theorem}
\begin{proof}[Setup]
In the proof we denote by $p_{\bd{G}}$ the unique morphism $\bd{G}\ra \Spec k$. Let $\mu:\bd{G}\times_k\bd{G}\ra \bd{G}$ be the multiplication and $e:\Spec k\ra \bd{G}$ be the unit of the group $k$-scheme structure on $\bd{G}$. Moreover, we denote by $\pi_{23}:\bd{G}\times_k \bd{G}\times_k X\ra \bd{G}\times_kX$ the projection on the last two factors.
\end{proof}

\begin{lemma}\label{lemma:equivariant_formulas_for_group_imply_invertibility}
Let $\bd{G}$ be a group $k$-scheme and let $X$ be a $k$-scheme equipped with an action $a:\bd{G}\times X\ra X$ of $\bd{G}$. Let $\pi:\bd{G}\times_kX\ra X$ be the projection. Suppose that $\cF$ is a quasi-coherent sheaf on $X$ and $\tau:a^*\cF\ra \pi^*\cF$ is a morphisms of quasi-coherent sheaves on $\bd{G}\times_k X$. Then
$$\pi_{23}^*\tau\cdot \left(1_{\bd{G}}\times_k a\right)^*\tau= (\mu\times_k 1_X)^*\tau,\,\langle e,1_X \rangle^*\tau = 1_{\cF}$$
if and only if $\tau$ is an isomorphism and $(\cF,\tau^{-1})$ is a quasi-coherent $\bd{G}$-sheaf.
\end{lemma}
\begin{proof}[Proof of the lemma]
Suppose that the formulas
$$\pi_{23}^*\tau\cdot \left(1_{\bd{G}}\times_k a\right)^*\tau= (\mu\times_k 1_X)^*\tau,\,\langle e,1_X \rangle^*\tau = 1_{\cF}$$
hold. Since $\bd{G}$ is a group $k$-scheme, there exists a morphism $i:\bd{G}\ra \bd{G}$ of $k$-schemes such that
$$\mu\cdot \langle 1_{\bd{G}}, i \rangle = e\cdot p_{\bd{G}} = \mu\cdot \langle i, 1_{\bd{G}} \rangle$$
and $i\cdot i = 1_{\bd{G}}$. Then
$$ 1_{\pi^*\cF}  = \pi^*\langle e,1_X \rangle^*\tau  = (e\cdot p_{\bd{G}}\times_k 1_X)^*\tau = \big(\langle i, 1_{\bd{G}} \rangle \times_k 1_X \big)^*(\mu\times_k 1_X)^*\tau =$$
$$=\big(\langle i, 1_{\bd{G}} \rangle \times_k 1_X \big)^*\big(\pi_{23}^*\tau\cdot \left(1_{\bd{G}}\times_k a\right)^*\tau\big) = \big(\langle i, 1_{\bd{G}} \rangle \times_k 1_X \big)^*\pi_{23}^*\tau \cdot \big(\langle i, 1_{\bd{G}} \rangle \times_k 1_X \big)^*\left(1_{\bd{G}}\times_k a\right)^*\tau = $$
$$= \tau \cdot  \big(\langle i, 1_{\bd{G}} \rangle \times_k 1_X \big)^*\left(1_{\bd{G}}\times_k a\right)^*\tau$$
Therefore, $\tau$ is a retraction. Similarly we have
$$1_{a^*\cF} = a^*\langle e,1_X \rangle^*\tau  = \langle 1_{\bd{G}},a\rangle^*(e\cdot p_{\bd{G}}\times_k 1_X)^*\tau =\langle 1_{\bd{G}},a\rangle^*\big(\langle 1_{\bd{G}},i \rangle \times_k 1_X \big)^*(\mu\times_k 1_X)^*\tau  =$$
$$=\langle 1_{\bd{G}},a\rangle^*\big(\langle 1_{\bd{G}}, i \rangle \times_k 1_X \big)^*\big(\pi_{23}^*\tau\cdot \left(1_{\bd{G}}\times_k a\right)^*\tau\big) = \langle 1_{\bd{G}},a\rangle^*\big(\langle 1_{\bd{G}},i \rangle \times_k 1_X \big)^*\pi_{23}^*\tau \cdot \langle 1_{\bd{G}},a\rangle^*\big(\langle 1_{\bd{G}},i \rangle \times_k 1_X \big)^*\left(1_{\bd{G}}\times_k a\right)^*\tau =$$
$$=\langle 1_{\bd{G}},a\rangle^*\big(\langle 1_{\bd{G}},i \rangle \times_k 1_X \big)^*\pi_{23}^*\tau \cdot \tau$$
Thus $\tau$ is a coretraction. Therefore, if the formulas above hold, we deduce that $\tau$ is an isomorphism and
$$\left(1_{\bd{G}}\times_k a\right)^*\tau^{-1} \cdot \pi_{23}^*\tau^{-1} = (\mu\times_k 1_X)^*\tau^{-1},\,\langle e,1_X \rangle^*\tau^{-1} = 1_{\cF}$$
On the other hand if $\tau$ is an isomorphism and $(\cF,\tau^{-1})$ is a quasi-coherent $\bd{G}$-sheaf, then clearly
$$\pi_{23}^*\tau\cdot \left(1_{\bd{G}}\times_k a\right)^*\tau= (\mu\times_k 1_X)^*\tau,\,\langle e,1_X \rangle^*\tau = 1_{\cF}$$
\end{proof}

\begin{proof}[Proof of the theorem]
Let $\tau$ is the image of $\gamma$ under the adjunction bijection
\begin{center}
\begin{tikzpicture}
[description/.style={fill=white,inner sep=2pt}]
\matrix (m) [matrix of math nodes, row sep=1em, column sep=2em,text height=1.5ex, text depth=0.25ex] 
{\Hom_{\cO_X}\left(\cF, a_*\pi^*\cF\right)  & \Hom_{\cO_{\bd{M}\times_kX}}\left(a^*\cF, \pi^*\cF\right) \\} ;
\path[->,line width=1.0pt,font=\scriptsize]  
(m-1-1) edge node[above] {$ $} (m-1-2);
\end{tikzpicture}
\end{center}
for $a^*\dashv a_*$. Fix an open $\bd{G}$-stable affine subscheme $U$ of $X$. Let $c$ be the morphism
$$\cF(U)\ra k[\bd{G}]\otimes_k\cF(U)$$
determined as the composition of $\Gamma\left(U,\gamma\right)$ with the identification $\Gamma(U,\pi^*\cF) = k[\bd{G}]\otimes_k\cF(U)$. Next observe that $\gamma = a_*\tau \cdot \eta_{\cF}$, where $\eta_{\cF}:\cF\ra a_*a^*\cF$ is the unit of $a^*\dashv a_*$. Thus $c$ is the composition of $$\Gamma\left(\bd{G}\times_kU, \tau \right)\cdot \Gamma\left(U,\eta_{\cF}\right)$$
with the identification $\Gamma(U,\pi^*\cF) = k[\bd{G}]\otimes_k\cF(U)$. Note that $\Gamma\left(U,\eta_{\cF}\right)(s) = a^*s$ for every $s$ in $\cF(U)$. Fix now $s$ in $\cF(U)$. Suppose that
$$c(s) = \sum_{i=1}^na_i\otimes s_i$$
where $a_i \in k[\bd{M}]$ and $s_i \in \cF(U)$ for all $i$. Then
$$\big(1_{k[\bd{G}]}\otimes_k c\big)\big(c(s)\big) = \sum_{i=1}^na_i\otimes c(s_i) =  \sum_{i=1}^n\bigg( \Gamma\left(\bd{G}\times_k \bd{G} \times_kU, \pi_{23}^*\tau \right)\big(a_i \otimes a^*s_i\big) \bigg) =$$
$$= \Gamma\left(\bd{G}\times_k \bd{G} \times_k U, \pi_{23}^*\tau \right)\big(\left(1_{\bd{G}}\times_k a\right)^*c(s)\big) =$$
$$= \bigg(\Gamma\left(\bd{G}\times_k \bd{G} \times_kU, \pi_{23}^*\tau \right) \cdot \Gamma\left(\bd{G} \times_k \bd{G} \times_k U, \left(1_{\bd{G}}\times_k a\right)^*\tau \right)\bigg) \big(\left(1_{\bd{G}}\times_k a\right)^*a^*s)\big) = $$
$$= \Gamma\big(\bd{G}\times_k \bd{G} \times_kU, \pi_{23}^*\tau\cdot \left(1_{\bd{G}}\times_k a\right)^*\tau \big) \big(\left(1_{\bd{G}}\times_k a\right)^*a^*s)\big)$$
and
$$\big(\Delta_{\bd{G}}\otimes_k 1_{\cF(U)}\big)\big(c(s)\big) =\left(\mu\times_k 1_X\right)^*c(s)=\Gamma\left(\bd{G}\times_k\bd{G}\times_kU, (\mu\times_k 1_X)^*\tau \right)\big(\left(\mu\times_k 1_X\right)^*a^*s\big)$$
where $\Delta_{\bd{G}}$ is the comultiplication of $k[\bd{G}]$. Since $s$ is an arbitrary section of $\cF$ over $U$, we derive that
$$\big(1_{k[\bd{G}]}\otimes_k c\big) \cdot c = \big(\Delta_{\bd{G}}\otimes_k 1_{\cF(U)}\big)\cdot c$$
if and only if
$$\Gamma\big(\bd{G}\times_k \bd{G} \times_kU, \pi_{23}^*\tau\cdot \left(1_{\bd{G}}\times_k a\right)^*\tau \big) = \Gamma\left(\bd{G}\times_k\bd{G}\times_kU, (\mu\times_k 1_X)^*\tau \right)$$
Next suppose that $\xi_{\bd{G}}:k\ra k[\bd{G}]$ is the counit of $k[\bd{G}]$. Then
$$\sum_{i=1}^n \xi_{\bd{G}}(a_i) \cdot s_i = \langle e, 1_X\rangle^*c(s) = \Gamma\big(U, \langle e,1_X \rangle^*\tau \big)\big(\langle e, 1_X\rangle^*a^*s\big) = \Gamma\big(U, \langle e, 1_X \rangle^*\tau \big)(s)$$
Since $s$ is arbitrary, we derive that $\big(\xi_{\bd{G}} \otimes_k 1_{\cF(U)}\big)\cdot c$ is isomorphic with $1_{\cF(U)}$ if and only if
$$\Gamma\big(U, \langle e, 1_X \rangle^*\tau \big) = 1_{\cF(U)}$$
Thus $c$ is a coaction of $k[\bd{G}]$ if and only if
$$\Gamma\big(\bd{G}\times_k \bd{G} \times_kU, \pi_{23}^*\tau\cdot \left(1_{\bd{G}}\times_k a\right)^*\tau \big) = \Gamma\left(\bd{G}\times_k\bd{G}\times_kU, (\mu\times_k 1_X)^*\tau \right)$$
and
$$\Gamma\big(U, \langle e,1_X \rangle^*\tau \big) = 1_{\cF(U)}$$
Now $X$ is a locally linear $\bd{G}$-scheme. From this assumption we deduce that \textbf{(i)} is equivalent with the fact that formulas
$$\pi_{23}^*\tau\cdot \left(1_{\bd{G}}\times_k a\right)^*\tau= (\mu\times_k 1_X)^*\tau,\,\langle e,1_X \rangle^*\tau = 1_{\cF}$$
hold. By Lemma \ref{lemma:equivariant_formulas_for_group_imply_invertibility} it follows that these these formulas hold if and only if \textbf{(ii)} holds. Thus assertions \textbf{(i)} and \textbf{(ii)} are equivalent.
\end{proof}

\begin{remark}\label{remark:the_category_of_equivariant_sheaves_on_locally_linear_schemes_alternative_description}
Theorem \ref{theorem:coactions_and_equivariance_for_locally_linear_schemes} gives rise to the alternative description of the category $\Qcoh_{\bd{G}}(X)$, where $X$ is a $k$-scheme equipped with an action $a:\bd{G}\times_kX\ra X$ of affine group $k$-scheme $\bd{G}$ that makes it into a $\bd{G}$-linear scheme. We give now details of this description. Denote by $\pi:\bd{G}\times_kX\ra X$ the projection. Objects of $\Qcoh_{\bd{G}}(X)$ are pairs $(\cF,\gamma)$ consisting of a quasi-coherent sheaf $\cF$ on $X$ and a morphism $\gamma:\cF\ra a_*\pi^*\cF$ of quasi-coherent sheaves on $X$ such that for every open $\bd{G}$-stable affine subscheme $U$ of $X$ morphism
$$\Gamma(U,\gamma):\cF(U)\ra k[\bd{G}]\otimes_k\cF(U)$$
is a coaction of the bialgebra $k[\bd{G}]$. Now if $(\cF_1,\gamma_1)$ and $(\cF_2,\gamma_2)$ are two objects of $\Qcoh_{\bd{G}}(X)$, then a morphism $\phi:(\cF_1,\gamma_1)\ra (\cF_2,\gamma_2)$ is a morphism $\phi:\cF_1\ra \cF_2$ of quasi-coherent sheaves on $X$ such that the square
\begin{center}
\begin{tikzpicture}
[description/.style={fill=white,inner sep=2pt}]
\matrix (m) [matrix of math nodes, row sep=3em, column sep=3em,text height=1.5ex, text depth=0.25ex] 
{ \cF_1 & a_*\pi^*\cF_1     \\
  \cF_2 & a_*\pi^*\cF_2             \\} ;
\path[->,line width=1.0pt,font=\scriptsize]  
(m-1-1) edge node[above] {$ \gamma_1 $} (m-1-2)
(m-2-1) edge node[below] {$ \gamma_2 $} (m-2-2)
(m-1-1) edge node[left] {$ \phi $} (m-2-1)
(m-1-2) edge node[right] {$ a_*\pi^*\phi  $} (m-2-2);
\end{tikzpicture}
\end{center}
is commutative. Moreover, if $X$ is locally noetherian, then analogical description is valid for $\Coh_{\bd{G}}(X)$.
\end{remark}
\noindent
The next two examples are consequences of Remark \ref{remark:the_category_of_equivariant_sheaves_on_locally_linear_schemes_alternative_description}. 

\begin{example}\label{example:equivariant_objects_on_point}
Consider $\Spec k$ as a $k$-scheme with trivial action of an affine group $k$-scheme $\bd{G}$. Then $\Qcoh_{\bd{G}}(\Spec k)$ is isomorphic with $\bd{Rep}(\bd{G})$.
\end{example}
\noindent
The example above can be generalized.

\begin{example}\label{example:equivariant_sheaves_on_trivial_equivariant_schemes}
Let $\bd{G}$ be an affine group $k$-scheme and let $X$ be a $k$-scheme equipped with the trivial action of $\bd{G}$. Suppose that $\cF$ is a quasi-coherent sheaf on $X$. Then to give a structure of $\bd{G}$-sheaf on $\cF$ is the same as determining a morphism $\cF \ra k[\bd{G}]\otimes_k\cF$ of quasi-coherent sheaves on $X$ such that for every open affine subscheme $U$ the induced morphism
$$\cF(U) \ra k[\bd{G}]\otimes_k\cF(U)$$
is the coaction of $k[\bd{G}]$. In other words a structure of $\bd{G}$-sheaf on $\cF$ is the same as a structure of $\bd{G}$-representation on $\cF(U)$ for every open affine subscheme $U$ of $X$ such that the restriction morphism $\cF(U)\ra \cF(V)$ is a morphism of $\bd{G}$-representations for every pair $U\subseteq V$ of open affine subschemes of $X$. In this way we obtain a description of $\Qcoh_{\bd{G}}(X)$.
\end{example}
\noindent
Using Theorem \ref{theorem:coactions_and_equivariance_for_locally_linear_schemes} and Remark \ref{remark:the_category_of_equivariant_sheaves_on_locally_linear_schemes_alternative_description} we give yet another description of the category $\Coh_{\bd{G}}(X)$ on a $\bd{G}$-schemes $X$ which are affine over trivial $\bd{G}$-schemes. This description will be extremely robust as it enables to use representation theory of $\bd{G}$ in studying $\bd{G}$-sheaves. 

\begin{remark}\label{remark:equivariant_scheme_affine_over_trivial_scheme}
Let $\bd{G}$ be an affine group $k$-scheme and let $X$ be $k$-scheme equipped with an action $a:\bd{G}\times_k X\ra X$ of $\bd{G}$. Suppose that $r:X\ra Y$ is a $\bd{G}$-equivariant morphism into a trivial $\bd{G}$-scheme. Assume that $r$ is affine. Then $X = \Spec_Y\cA$, where $\cA$ is a quasi-coherent algebra on $Y$ and the action $a$ corresponds to the morphism $\cA\ra k[\bd{G}]\otimes_k\cA$ of algebras over $\cO_Y$ such that for every open affine subscheme $V$ of $Y$ its restriction
$$\cA(V) \ra k[\bd{G}]\otimes_k\cA(V)$$
to sections over $V$ is the coaction of $k[\bd{G}]$ on $\cA(V)$. Now suppose that $\cF$ is a quasi-coherent $\bd{G}$-sheaf on $X$ with respect to $\gamma:\cF\ra a_*\pi^*\cF$ (Remark \ref{remark:the_category_of_equivariant_sheaves_on_locally_linear_schemes_alternative_description}), where $\pi:\bd{G}\times_kZ\ra Z$ is the projection. Then $r_*\cF = \cM$ is a quasi-coherent sheaf on $Y$ which is an $\cA$-module and $r_*\gamma$ is the morphism $\cM\ra k[\bd{G}]\otimes_k\cM$ of quasi-coherent on $Y$ such that the following assertions hold.
\begin{enumerate}[label=\textbf{(\arabic*)}, leftmargin=3.0em]
\item For every open affine subscheme $V$ of $Y$ the restriction
$$\cM(V) \ra k[\bd{G}]\otimes_k\cM(V)$$
to sections over $V$ is the coaction of $k[\bd{G}]$ on $\cM(V)$.
\item $\cM \ra k[\bd{G}]\otimes_k\cM$ is the morphism of $\cA$-modules, where $k[\bd{G}]\otimes_k\cM$ carries the structure of an $\cA$-module induced by restriction of its $k[\bd{G}]\otimes_k\cA$-module structure along the morphism $\cA\ra k[\bd{G}]\otimes_k\cA$ that corresponds to $a$.
\end{enumerate}
The pair $(\cF,\gamma)$ is uniquely determined by $(r_*\cF,r_*\gamma)$.
\end{remark}

\section{Some results on formal $\bd{M}$-schemes}

\begin{corollary}\label{corollary:each_formal_scheme_consists_of_locally_linear_schemes_if_group_is_affine}
Let $\bd{M}$ be an affine monoid $k$-scheme and let $\cZ = \{Z_n\}_{n\in \NN}$ be a formal $\bd{G}$-scheme. Then $Z_n$ is locally linear $\bd{G}$-scheme for every $n\in \NN$.
\end{corollary}
\begin{proof}
Let $\cI_n$ be an ideal defining $Z_0$ in $Z_n$. Since $\cZ$ is a formal $\bd{M}$-scheme, we derive that $\cI_n^{n+1} = 0$ and $Z_0$ is locally linear $\bd{M}$-scheme. Thus we apply Corollary \ref{corollary:locally_linear_are_stable_under_thickenings} and derive that $Z_n$ is locally linear $\bd{M}$-scheme.
\end{proof}
\noindent
We are particularly interested in formal $\bd{M}$-schemes for monoid $\bd{M}$ with zero. For this we need the following elementary result.

\begin{proposition}\label{proposition:retraction_for_monoids_with_zero}
Let $\bd{M}$ be a monoid $k$-scheme with zero $\bd{o}$ and let $X$ be a $\bd{M}$-scheme. Then the following results hold.
\begin{enumerate}[label=\emph{\textbf{(\arabic*)}}, leftmargin=3.0em]
\item The multiplication by zero $\bd{o}\cdot(-):X\ra X$ factors through $X^{\bd{M}}$ inducing a $\bd{M}$-equivariant retraction $r_{\bd{M}}:X\twoheadrightarrow X^{\bd{M}}$.
\item If $\bd{N}$ is a submonoid $k$-scheme of $\bd{M}$ and $\bd{o}$ is a $k$-point of $\bd{N}$, then $r_{\bd{M}} = r_{\bd{N}}$.
\item If $\bd{M}$ is affine and $X$ is locally linear $\bd{M}$-scheme, then $r_{\bd{M}}$ is affine.
\item If $\bd{M}$ is affine, $X$ is both locally noetherian and locally linear $\bd{M}$-scheme and ideal of $X^{\bd{M}}$ in $X$ is nilpotent, then $r_{\bd{M}}$ is finite.
\end{enumerate}
\end{proposition}
\begin{proof}
The multiplication $\bd{o}\cdot (-):X\ra X$ factors as an $\bd{M}$-equivariant epimorphism $X \twoheadrightarrow X^{\bd{M}}$ composed with a closed immersion $X^{\bd{M}} \hookrightarrow X$. The $\bd{M}$-equivariant epimorphism $X\ra X^{\bd{M}}$ corresponds to a $\bd{M}$-equivariant morphism $r_{\bd{M}}:X\ra X^{\bd{M}}$ of $k$-schemes such that $r_{\bd{M}}$ restricted to $X^{\bd{M}}$ is the identity $1_{X^{\bd{M}}}$. This proves \textbf{(1)}.\\
For the proof of \textbf{(2)} note that $\bd{o}\cdot (-):X\ra X$ is defined similarly for $\bd{M}$ and $\bd{N}$ (provided that $\bd{o}$ is a $k$-point of $\bd{N}$). Thus $r_{\bd{M}} = r_{\bd{N}}$.\\
Suppose now that $\bd{M}$ is affine and $X$ is locally linear $\bd{M}$-scheme. Consider the action $\alpha:\bd{M}\times_k X\ra X$ of $\bd{M}$ on $X$. Since $X$ is locally linear $\bd{M}$-scheme and $\bd{M}$ is affine, we derive that $\alpha$ is an affine morphism of $k$-schemes. Now $\bd{o}\cdot (-):X\ra X$ is given as a composition 
\begin{center}
\begin{tikzpicture}
[description/.style={fill=white,inner sep=2pt}]
\matrix (m) [matrix of math nodes, row sep=3em, column sep=3em,text height=1.5ex, text depth=0.25ex] 
{X & \bd{o}\times_k X & \bd{M}\times_k X & X \\} ;
\path[->,line width=1.0pt,font=\scriptsize]  
(m-1-1) edge node[above] {$\cong$} (m-1-2)
(m-1-3) edge node[above] {$\alpha$} (m-1-4);
\path[right hook->,line width=1.0pt,font=\scriptsize]  
(m-1-2) edge node[above] {$ $} (m-1-3);
\end{tikzpicture}
\end{center}
The morphism above is affine (as a composition of affine morphisms). Since the composition of $r_{\bd{M}}$ with a closed immersion $X^{\bd{M}}\hookrightarrow X$ is $\bd{o}\times_k(-)$ and hence an affine morphism, we derive that $r_{\bd{M}}$ is affine. This proves \textbf{(3)}.\\
Now we prove \textbf{(4)}. From \textbf{(3)} we know that $r_{\bd{M}}$ is affine morphism. Hence $r_{\bd{M}}:X\twoheadrightarrow X^{\bd{M}}$ corresponds to some quasi-coherent algebra $\cA$ on $X^{\bd{M}}$. Moreover, the embedding $X^{\bd{M}}\hookrightarrow X$ corresponds to the surjection $\cA\twoheadrightarrow \cO_{X^{\bd{M}}}$ which ideal $\cI\subseteq \cA$ is nilpotent. Assume that $\cI^n = 0$. Then we have a filtration
$$0 = \cI^n \subseteq \cI^{n-1} \subseteq ... \subseteq \cI\subseteq \cA$$
with factors $\cI^k/\cI^{k+1}$ for $k=0,1,...,n-1$. Since $X$ is locally noetherian, we derive that each $\cI^k/\cI^{k+1}$ is a finite type $\cA$-module. Hence each factor is a finite type module over $\cA/\cI = \cO_{X^{\bd{M}}}$. Thus $\cA$ has finite filtrations which factors are coherent sheaves on $X^{\bd{M}}$. Therefore, $\cA$ is coherent algebra on $X^{\bd{M}}$ and this shows that $r_{\bd{M}}$ is finite.
\end{proof}
\noindent
Let us note the immediate consequence of this result.

\begin{corollary}\label{corollary:restraction_for_formal_schemes_and_pointed_submonoids}
Let $\bd{M}$ be an affine monoid $k$-scheme with zero and let $\cZ = \{Z_n\}_{n\in \NN}$ be a formal $\bd{M}$-scheme. Then $\cZ$ is a part of the commutative diagram
\begin{center}
\begin{tikzpicture}
[description/.style={fill=white,inner sep=2pt}]
\matrix (m) [matrix of math nodes, row sep=2em, column sep=3em,text height=1.5ex, text depth=0.25ex] 
{ Z_0 &  Z_1 & ... & Z_n & ... \\
      &      & Z_0 &     &  \\} ;
\path[right hook->,line width=1.0pt,font=\scriptsize]  
(m-1-1) edge node[above] {$ $} (m-1-2)
(m-1-2) edge node[above] {$ $} (m-1-3)
(m-1-3) edge node[above] {$ $} (m-1-4)
(m-1-4) edge node[above] {$ $} (m-1-5);
\path[->>,line width=1.0pt,font=\scriptsize]
(m-1-1) edge[bend right = 20] node[below] {$r_0 = 1_{Z_0} $} (m-2-3)
(m-1-2) edge node[above] {$ r_1 $} (m-2-3)
(m-1-4) edge[bend left = 20] node[above] {$ r_n $} (m-2-3);
\end{tikzpicture}
\end{center}
in which vertical morphisms $r_n:Z_n\twoheadrightarrow Z_0$ are affine $\bd{M}$-equivariant morphisms such that ${r_n}_{\mid Z_0} = 1_{Z_0}$. Moreover, the following assertions hold.
\begin{enumerate}[label=\emph{\textbf{(\arabic*)}}, leftmargin=3.0em]
\item If $\cZ$ is locally noetherian, then every $r_n$ is finite morphism.
\item If $\bd{N}$ is a submonoid $k$-scheme of $\bd{M}$ containing the zero of $\bd{M}$, then $\cZ$ is a formal $\bd{N}$-scheme.
\end{enumerate}
\end{corollary}
\begin{proof}
This is an immediate consequence of Corollary \ref{corollary:each_formal_scheme_consists_of_locally_linear_schemes_if_group_is_affine} and Proposition \ref{proposition:retraction_for_monoids_with_zero}.
\end{proof}

\section{Isotypic decompositions}
\noindent
The following result will be used in the next section.

\begin{proposition}\label{proposition:commuting_action_preserves_isotypic_decomposition}
Let $\fG$ and $\fH$ be monoid $k$-functors. Denote by $\Lambda$ the set of isomorphism classes of irreducible $\fH$-representations. Suppose that $V$ is a representation of both $\fG$ and $\fH$ and assume that their actions on $V$ commute. Assume that $V$ is completely reducible as a $\fH$-representation and consider the decomposition
$$V = \bigoplus_{\lambda\in \Lambda}V[\lambda]$$
onto isotypic components with respect to the action of $\fH$. Then for every $\lambda$ in $\Lambda$ the subspace $V[\lambda]$ is a $\fG$-subrepresentation of $V$.
\end{proposition}
\begin{proof}
Consider morphisms $\rho:\fG\ra \cL_V$ and $\delta:\fH\ra \cL_V$ determining the structure of $V$ as the $\fG$-representation and $\fH$-representation, respectively. Fix $k$-algebra $A$ and $g\in \fG(A)$. Consider $A\otimes_k V$ as a tensor product of $\fH$-representation $V$ with $A$ as a trivial $\fH$-representation. We claim that $\rho(g):A\otimes_kV \ra A\otimes_kV$ is an endomorphism of this $\fH$-representation. For this consider $k$-algebra $B$ and $h\in \fH(B)$. Since actions of $\fG$ and $\fH$ on $V$ commute, we derive that
$$\big(1_B\otimes_k\rho(g)\big)\cdot \big(1_A\otimes_k\delta(h)\big) = \big(1_A\otimes_k\delta(h)\big)\cdot \big(1_B\otimes_k\rho(g)\big)$$
Since this holds for every $k$-algebra $B$ and every $h\in \fH(B)$, we deduce that indeed $\rho(g)$ is a $\fH$-endomorphism of $A\otimes_kV$. Next we have
$$\left(A\otimes_kV\right)[\lambda] = A\otimes_k V[\lambda]$$
for every $\lambda \in \Lambda$. Thus
$$\rho(g)\left(A \otimes_k V[\lambda] \right)\subseteq A \otimes_k V[\lambda]$$
for every $\lambda$ in $\Lambda$. This holds for every $k$-algebra $A$ and $g\in \fG(A)$. Hence $V[\lambda]$ is a $\fG$-subrepresentation of $V$.
\end{proof}

\begin{definition}
Let $\bd{G}$ be an affine monoid $k$-scheme and let $X$ be a $k$-scheme equipped with the trivial action of $\bd{G}$. Fix $\lambda$ in $\bd{Irr}(\bd{G})$ and a quasi-coherent $\bd{G}$-sheaf $\cF$ on $X$. We define a quasi-coherent $\bd{G}$-subsheaf $\cF[\lambda]$ of $\cF$ by setting its module of sections over every open affine subscheme $U$ of $X$ to be an isotypic component $\cF(U)[\lambda]$ of the linear $\bd{G}$-representation $\cF(U)$ (Example \ref{example:equivariant_sheaves_on_trivial_equivariant_schemes}). We call it \textit{a $\lambda$-isotypic component of $\cF$}.
\end{definition}

\begin{fact}\label{fact:tensor_product_of_isotypic_components}
Let $\bd{G}$ be an affine group $k$-scheme and let $X$ be a $k$-scheme equipped with the trivial action of $\bd{G}$. Suppose that $\cF_1,\cF_2$ are quasi-coherent $\bd{G}$-sheaves on $X$. Fix now $\lambda_1,\lambda_2,\eta_1,...,\eta_n$ in $\bd{Irr}(\bd{G})$ such that
$$V_{\lambda_1}\otimes_k V_{\lambda_2} \cong \bigoplus_{i=1}^nV_{\eta_i}$$
as $\bd{G}$-representations, where by $V_{\lambda}$ we denote the irreducible representation in class $\lambda\in \bd{Irr}(\bd{G})$. Then
$$\big(\cF[\lambda_1]\otimes_{\cO_X}\cF_2[\lambda_2]\big)[\lambda] = 0$$
for $\lambda \neq \eta_i$ and $1\leq i\leq n$.
\end{fact}
\begin{proof}
Consider an open affine subscheme $U$ of $X$. The canonical surjection
\begin{center}
\begin{tikzpicture}
[description/.style={fill=white,inner sep=2pt}]
\matrix (m) [matrix of math nodes, row sep=3em, column sep=2em,text height=1.5ex, text depth=0.25ex]
{\Gamma(U,\cF_1)[\lambda_1] \otimes_k \Gamma(U,\cF_2)[\lambda_2] & \Gamma(U,\cF_1)[\lambda_1] \otimes_{\cO_X(U)} \Gamma(U,\cF_2)[\lambda_2]\\};
\path[->>,line width=1.0pt,font=\scriptsize]
(m-1-1) edge node[above] {$ $} (m-1-2);;
\end{tikzpicture}
\end{center}
is a morphism of $\bd{G}$-representations. Since $V_{\lambda_1}\otimes_k V_{\lambda_2} \cong \bigoplus_{i=1}^nV_{\eta_i}$, we derive that
$$\big(\Gamma(U,\cF_1)[\lambda_1] \otimes_k \Gamma(U,\cF_2)[\lambda_2]\big)[\lambda] = 0$$
for $\lambda \neq \eta_i$ and $1\leq i\leq n$. This implies that $\big(\Gamma(U,\cF_1)[\lambda_1] \otimes_{\cO_X(U)} \Gamma(U,\cF_2)[\lambda_2]\big)[\lambda] = 0$ for $\lambda \neq \eta_i$ and $1\leq i\leq n$. Since $U$ is an arbitrary affine open subscheme of $X$, we deduce that the statement holds.
\end{proof}

\section{Algebraization of formal $\bd{M}$-schemes}
\noindent
Now we are ready to prove certain results concerning \textit{algebraizations} of formal $\bd{M}$-schemes for Kempf monoids.

\begin{theorem}\label{theorem:every_formal_over_kempf_monoid_is_formal_neighborhood}
Let $\bd{M}$ be a Kempf monoid and let $\cZ = \{Z_n\}_{n\in \NN}$ be a formal $\bd{M}$-scheme. Then there exists a locally linear $\bd{M}$-scheme $Z$ such that $\widehat{Z}$ is isomorphic to $\cZ$. Moreover, we have that
$$Z = \mathrm{colim}_{n\in \NN}Z_n$$
in category of $\bd{M}$-schemes affine over $Z_0$.
\end{theorem}
\begin{proof}[Setup]
Monoid $\bd{M}$ is affine and admits zero $\bd{o}$. By Corollary \ref{corollary:restraction_for_formal_schemes_and_pointed_submonoids} a formal $\bd{M}$-scheme $\cZ = \{Z_n\}_{n\in \NN}$ corresponds to a sequence of surjections
\begin{center}
\begin{tikzpicture}v
[description/.style={fill=white,inner sep=2pt}]
\matrix (m) [matrix of math nodes, row sep=3em, column sep=2em,text height=1.5ex, text depth=0.25ex] 
{ ... &  \cA_{n+1} & \cA_n & ...& \cA_1 & \cA_0 = \cO_{Z_0} \\} ;
\path[->>,line width=1.0pt,font=\scriptsize]  
(m-1-1) edge node[above] {$ $} (m-1-2)
(m-1-2) edge node[above] {$ $} (m-1-3)
(m-1-3) edge node[above] {$ $} (m-1-4)
(m-1-4) edge node[above] {$ $} (m-1-5)
(m-1-5) edge node[above] {$ $} (m-1-6);
\end{tikzpicture}
\end{center}
of quasi-coherent algebras on $Z_0$ such that the following assertions hold.
\begin{enumerate}[label=\textbf{(\arabic*)}, leftmargin=3.0em]
\item For each $n\in \NN$ there exists a morphism $\cA_n\ra k[\bd{M}]\otimes_k\cA_n$ such that for every open affine neighborhood $U$ of $Z_0$ its restriction 
$$\cA_n(U)\ra k[\bd{M}]\otimes_k \cA_n(U)$$
to sections on $U$ is a coaction of $k[\bd{M}]$ on $\cA_n(U)$.
\item For every $n\in \NN$ the epimorphism $\cA_{n+1} \twoheadrightarrow \cA_n$ preserves coaction described in \textbf{(1)}.
\item $\cA_n\twoheadrightarrow \cA_0$ is the surjection on coinvariants of $\cA_n$ for every $n\in \NN$.
\item $\cA_n^{\bd{M}}\hookrightarrow \cA_n\twoheadrightarrow \cA_0$ is an isomorphism for every $n\in \NN$.
\item If $\cI_n$ is the kernel of $\cA_n\twoheadrightarrow \cA_0$ in $\cA_n$, then $\cI_n^{m+1}$ is the kernel of $\cA_n\twoheadrightarrow \cA_m$ for $m\leq n$ and $n\in \NN$.
\end{enumerate}
Since $\bd{M}$ is a Kempf monoid, there exists a closed subgroup $T$ of the center $Z(\bd{G})$ of the unit group $\bd{G}$ of $\bd{M}$ such that $T$ is a torus and the scheme-theoretic closure $\ol{T}$ of $T$ in $\bd{M}$ contains the zero $\bd{o}$ of $\bd{M}$. We derive by Corollary \ref{corollary:restraction_for_formal_schemes_and_pointed_submonoids} that $\cA_n^{\bd{\ol{T}}} = \cA_0$ for every $n\in \NN$. By definition $\ol{T}$ is a toric monoid $k$-scheme with $T$ as a group of units. Let $\{V_{\lambda}\}_{\lambda\in \bd{Irr}(T)}$ be a set of irreducible representations of $T$ such that $V_{\lambda}$ is contained in $\lambda$.
\end{proof}

\begin{lemma}\label{lemma:stablization_for_representations}
Let $\lambda$ be in $\bd{Irr}(T)$. Then there exists $n_{\lambda}\in \NN$ such that for each $n > n_{\lambda}$ and any $\lambda_1,...,\lambda_n\in \bd{Irr}(\ol{T})\setminus \{\lambda_0\}$ the representation
$$\bigotimes_{i=1}^nV_{\lambda_i}$$ 
has trivial isotypic component of type $\lambda$. We have $n_{\lambda_0} = 0$, where $\lambda_0$ is an isomorphism type of the trivial representation of $T$.
\end{lemma}
\begin{proof}[Proof of the lemma]
Let $K$ be an algebraically closed extension of $k$. Pick $A_{\lambda}$ and $f$ as in {\cite[Theorem 2.3]{Algebraic_monoids}} and define
$$n_{\lambda} = \sup_{m\in A_{\lambda}}f(m)$$
We have
$$K\otimes_kV_{\lambda_1}\otimes_k...\otimes_kV_{\lambda_n} = \bigoplus_{(m_1,...,m_n)\in A_{\lambda_1}\times_k ...\times_k A_{\lambda_n}}K\cdot \chi^{m_1+...+m_n}$$
and since $m_1,...m_n\in A_{\lambda_1}\cup ...\cup A_{\lambda_n}\subseteq S\setminus \{0\}$ we derive that
$$f(m_1+...+m_n) = f(m_1) + ... + f(m_n) \geq n > n_{\lambda} = \sup_{m\in A_{\lambda}}f(m)$$
This implies that isotypic component of $V_{\lambda_1}\otimes_k...\otimes_kV_{\lambda_n}$ corresponding to $\lambda$ is trivial.
\end{proof}

\begin{lemma}\label{lemma:stabilization_for_formal_schemes}
Fix $\lambda$ in $\bd{Irr}(T)$. Then $\cA_{n+1}[\lambda]\twoheadrightarrow \cA_n[\lambda]$ is an isomorphism for $n \geq  n_{\lambda}$.
\end{lemma}
\begin{proof}[Proof of the lemma]
For $\lambda \not \in \bd{Irr}(\ol{T})\setminus \{\lambda_0\}$ we have $\cA_{n+1}[\lambda] = \cA_n[\lambda] = 0$, because $\cA_{n+1}$ and $\cA_n$ are quasi-coherent $\ol{T}$-algebras. Fix $\lambda \in \bd{Irr}(\ol{T})$. Consider an affine open subset $U$ of $Z_0$. By Lemma \ref{lemma:stablization_for_representations} and Fact \ref{fact:tensor_product_of_isotypic_components} we derive that
$$\underbrace{\left(\cI_{n+1} \otimes_{\cO_{Z_0}} \cI_{n+1} \otimes_{\cO_{Z_0}}...\otimes_{\cO_{Z_0}} \cI_{n+1}\right)}_{n+1\,\mathrm{times}}[\lambda] = 0$$
for every $n \geq n_{\lambda}$. Next the multiplication
\begin{center}
\begin{tikzpicture}
[description/.style={fill=white,inner sep=2pt}]
\matrix (m) [matrix of math nodes, row sep=3em, column sep=2em,text height=1.5ex, text depth=0.25ex]
{ \underbrace{\left(\cI_{n+1} \otimes_{\cO_{Z_0}} \cI_{n+1} \otimes_{\cO_{Z_0}}...\otimes_{\cO_{Z_0}} \cI_{n+1}\right)}_{n+1\,\mathrm{times}}  & \cA_{n+1} \\};
\path[->,line width=1.0pt,font=\scriptsize]
(m-1-1) edge node[above] {$ $} (m-1-2);;
\end{tikzpicture}
\end{center}
is a morphism of quasi-coherent $T$-sheaves with image $\cI_{n+1}^{n+1}$. Thus we derive that $\cI_{n+1}^{n+1}[\lambda]=0$ for $n \geq n_{\lambda}$. Hence the kernel of $\cA_{n+1}[\lambda]\twoheadrightarrow \cA_n[\lambda]$ is trivial.
\end{proof}

\begin{proof}[Proof of Theorem]
According to Proposition \ref{proposition:commuting_action_preserves_isotypic_decomposition} and the fact that $T$ is central in $\bd{M}$ we derive that $\cA_n[\lambda]$ is a quasi-coherent $\bd{M}$-sheaf. For $\lambda\in \bd{Irr}(T)$ we define
$$\cA[\lambda] = \cA_n[\lambda]$$
where $n\geq n_{\lambda}$ as in Lemma \ref{lemma:stabilization_for_formal_schemes}. Note that $\cA[\lambda] = 0$ for $\lambda \not \in \bd{Irr}(\ol{T})$. We set
$$\cA=\bigoplus_{\lambda\in \bd{Irr}(\ol{T})}\cA[\lambda]$$
Clearly $\cA[\lambda_0] = \cA_0 = \cO_{Z_0}$ canonically (where $\lambda_0$ is the trivial $T$-representation), hence $\cA$ is a quasi-coherent $\bd{M}$-sheaf on $Z_0$. Actually $\cA=\lim_{n\in \NN}\cA_n$ in the category of quasi-coherent $\bd{M}$-sheaves on $Z_0$. We construct the $\cO_{Z_0}$-algebra structure on $\cA$. For this pick $\lambda_1, \lambda_2\in \bd{Irr}(\ol{T})$. Consider the irreducible representations $V_{\lambda_1}$ and $V_{\lambda_1}$ in classes $\lambda_1$ and $\lambda_2$, respectively. Suppose that $\eta_1,...,\eta_s$ are finitely many classes in $\bd{Irr}(\ol{T})$ such that $V_{\lambda_1}\otimes_k V_{\lambda_2}$ can be completely decomposed onto irreducible representation in these classes. According to Fact \ref{fact:tensor_product_of_isotypic_components} we deduce that the image of the multiplication
\begin{center}
\begin{tikzpicture}
[description/.style={fill=white,inner sep=2pt}]
\matrix (m) [matrix of math nodes, row sep=3em, column sep=2em,text height=1.5ex, text depth=0.25ex] 
{\cA_n[\lambda_1]\otimes_{\cO_{Z_0}}\cA_n[\lambda_2]  & \cA_n\\} ;
\path[->,line width=1.0pt,font=\scriptsize]  
(m-1-1) edge node[above] {$ $} (m-1-2);
\end{tikzpicture}
\end{center}
is contained in $\bigoplus_{i=1}^s\cA_n[\eta_i]$. By Lemma \ref{lemma:stabilization_for_formal_schemes} all these multiplications for $n\geq \sup \{n_{\lambda_1},n_{\lambda_2},n_{\eta_1},...,n_{\eta_s}\}$ can be identified. Now we define
$$\cA[\lambda_1]\otimes_{\cO_{Z_0}} \cA[\lambda_2]\ra  \bigoplus_{i=1}^s\cA[\eta_i]\subseteq \cA$$
as a morphism induced by the multiplication morphism for any $n\geq \sup\{n_{\lambda_1},n_{\lambda_2},n_{\eta_1},...,n_{\eta_s}\}$. This gives an $\cO_{Z_0}$-algebra structure on $\cA$. So $\cA$ is in fact the limit of $\{\cA_n\}_{n\in \NN}$ in the category of quasi-coherent $\bd{M}$-algebras on $Z_0$. This implies that
$$\Spec_{Z_0}\cA = \mathrm{colim}_{n\in \NN}Z_n$$
in the category of $\bd{M}$-schemes affine over $Z_0$. Note that from the description of $\cA$ it follows that for every $n\in \NN$ we have a surjective morphism $p_n:\cA\twoheadrightarrow \cA_n$ of algebras. We denote its kernel by $\cJ_n$ and we put $\cJ = \cJ_0$. We have
$$\cJ=\bigoplus_{\lambda \in \bd{Irr}(\ol{T})\setminus \{\lambda_0\}}\cA[\lambda]$$
Recall that we denote by $\cI_n$ the kernel of $\cA_n\twoheadrightarrow \cA_0=\cO_{Z_0}$ for $n\in \NN$. Then $\cI_n=\cJ/\cJ_n$. Fix $m\in \NN$ and consider $n\in \NN$ such that $n\geq m$. Since $\cZ$ is a formal $\bd{M}$-scheme, the sheaf $\cI_n^{m+1}$ is the kernel of the morphism $\cA_n\twoheadrightarrow \cA_m$. Thus
$$\cJ_m/\cJ_n=\cI_n^{m+1}=(\cJ^{m+1}+\cJ_n)/\cJ_n$$
Both $\cJ_m$ and $\cJ^{m+1}$ are $\bd{Irr}(\ol{T})$-graded by their isotypic $\ol{T}$-components and for given $\lambda\in \bd{Irr}(\ol{T})$ and for $n \geq n_{\lambda}$ the isotypic component $\cJ_n[\lambda]$ is zero by Lemma \ref{lemma:stabilization_for_formal_schemes}. Hence $\cJ_m=\cJ^{m+1}$ for every $m \in \NN$.
We define
$$Z=\Spec_{Z_0}\cA$$
and we denote by $r^Z:Z\to Z_0$ the structural morphism. The scheme $Z$ inherits a $\bd{M}$-action from $\cA$. For every $n\in \NN$ the zero-set of $\cJ^{n+1}$ in $\cA$ is a $\bd{M}$-scheme isomorphic to $Z_n = \Spec_{Z_0}\cA_n$. Hence $\cZ$ is isomorphic to $\widehat{Z}$ and the proof is complete.
\end{proof}

\begin{theorem}\label{theorem:uniqueness_of_algebraization_for_formal_schemes_over_Kampf_monoid}
Let $\bd{M}$ be a Kempf monoid. Suppose that $Z$ and $W$ are locally linear $\bd{M}$-schemes such that $\widehat{Z}$ and $ \widehat{W}$ are isomorphic as $\bd{M}$-formal schemes. Then $Z$ is $\bd{M}$-equivariantly isomorphic to $W$.
\end{theorem}
\begin{proof}
Pick a locally linear $\bd{M}$-scheme $W$. Let $r^W:W\ra W^{\bd{M}}$ be the affine retraction (Proposition \ref{proposition:retraction_for_monoids_with_zero}). According to Theorem \ref{theorem:every_formal_over_kempf_monoid_is_formal_neighborhood} there exists a locally linear $\bd{M}$-scheme $Z$ such that formal $\bd{M}$-schemes $\widehat{Z} = \{Z_n\}_{n\in \NN}$ and $\widehat{W} = \{W_n\}_{n\in \NN}$ are isomorphic. Moreover
$$Z = \mathrm{colim}_{n\in \NN}Z_n$$
in the category of $\bd{M}$-schemes affine over $Z^{\bd{M}}$, where $Z$ is affine over $Z^{\bd{M}}$ via the affine retraction $r^Z:Z\ra Z^{\bd{M}}$ (again by Proposition \ref{proposition:retraction_for_monoids_with_zero}). It suffices to prove that $Z$ is $\bd{M}$-equivariantly isomorphic with $W$. By the universal property of colimits there exists a $\bd{M}$-equivariant morphism $f:Z\ra W$ such that $r^W\cdot f = r^Z$ and $f_{\mid Z_n}$ is isomorphic to the closed immersion $W_n\hookrightarrow W$ for every $n\in \NN$. We consider now $Z$ and $W$ as $\bd{M}$-schemes affine over the same base $Z^{\bd{M}} = W^{\bd{M}}$ equipped with the trivial $\bd{M}$-action. Then $Z,W$ correspond to quasi-coherent $\bd{M}$-algebras $\cA,\cB$ on $Z^{\bd{M}}=W^{\bd{M}}$, respectively, and moreover, there are quasi-coherent $\bd{M}$-ideals $\cI\subseteq \cA$, $\cJ\subseteq \cB$ such that $$\cA/\cI = \cO_{Z^{\bd{M}}} = \cO_{W^{\bd{M}}} = \cB/\cJ$$
Then $f$ corresponds to a morphism $h:\cB\ra \cA$ of quasi-coherent $\bd{M}$-algebras such that $h(\cJ)\subseteq \cI$ and for every $n\in \NN$ morphism $h$ induces an isomorphism $$\cB/\cJ^{n+1}\cong \cA/\cI^{n+1}$$
of quasi-coherent $\bd{M}$-algebras. Pick now an algebraically closed extension $K$ of $k$ and a zero preserving closed immersion $\mathbb{A}^1_K\hookrightarrow \Spec K\times_k\bd{M}$ of monoid $K$-schemes (this follows from the fact that $\bd{M}$ is a Kempf monoid {\cite[Corollary 3.7]{Algebraic_monoids}}). Then we have induced $\NN$-gradings on
$$K\otimes_k\cA= \cA_K = \bigoplus_{i\in \NN}\cA_K[i],\,K\otimes_k\cB = \cB_K =  \bigoplus_{i\in \NN}\cB_K[i]$$
and $h_K = 1_K \otimes_k h$ is $\NN$-graded homomorphism of algebras. Since the closed immersion of monoid $K$-schemes considered above is zero preserving and according to Proposition \ref{proposition:retraction_for_monoids_with_zero}, we deduce that
$$\Spec K\times_k Z^{\bd{M}} = \left(\Spec K \times_k Z\right)^{\bd{M}_K} = \left(\Spec K \times_k Z\right)^{\mathbb{A}^1_K}$$
as $K$-schemes and hence
$$\cI_K = K\otimes_k\cI = \bigoplus_{i>0}\cA_K[i],\,\cJ_K = K\otimes_k\cJ = \bigoplus_{i>0}\cB_K[i]$$
Moreover, $h_K$ induces isomorphisms of $\NN$-graded algebras
$$\cB_K/\cJ_K^{n+1}\cong \cA_K/\cI_K^{n+1}$$
for every $n\in \NN$. These imply that for every $i\in \NN$ morphism $h_K[i]:\cB_K[i]\ra \cA_K[i]$ is an isomorphism and hence $h_K$ is an isomorphism. By faithfully flat descent we deduce that $h$ is an isomorphism of quasi-coherent algebras on $\Spec K\times_kZ^{\bd{M}} = \Spec K\times_k W^{\bd{M}}$. Thus $f$ is a $\bd{M}$-equivariant isomorphism.
\end{proof}

\begin{example}\label{example:formal_Kempf_monoid_and_its_algebraization}
Let $\bd{M}$ be a Kempf monoid and let $Y$ be a $k$-scheme. We consider $Y$ as a $\bd{M}$-scheme with the trivial $\bd{M}$-action. Since $\bd{M}$ is a Kempf monoid it admits the zero $\bd{o}$. For every $n\in \NN$ let $\bd{M}_n$ be the $n$-th infinitesimal neighborhood of $\bd{o}$ in $\bd{M}$. Note that $\bd{M}_n$ is a closed $\bd{M}$-stable subscheme of $\bd{M}$ for every $n\in \NN$. Hence we have a formal $\bd{M}$-scheme  
\begin{center}
\begin{tikzpicture}
[description/.style={fill=white,inner sep=2pt}]
\matrix (m) [matrix of math nodes, row sep=3em, column sep=3em,text height=1.5ex, text depth=0.25ex] 
{ \bd{M}_0 &  \bd{M}_1\times_kY & ... & \bd{M}_n\times_kY & \bd{M}_{n+1}\times_kY & ... \\} ;
\path[right hook->,line width=1.0pt,font=\scriptsize]  
(m-1-1) edge node[above] {$ $} (m-1-2)
(m-1-2) edge node[above] {$ $} (m-1-3)
(m-1-3) edge node[above] {$ $} (m-1-4)
(m-1-4) edge node[above] {$ $} (m-1-5)
(m-1-5) edge node[above] {$ $} (m-1-6);
\end{tikzpicture}
\end{center}
Observe that $\bd{M}\times_kY$ is a locally linear $\bd{M}$-scheme and it is the unique locally linear $\bd{M}$-scheme such that $\reallywidehat{\bd{M}\times_kY} = \big\{\bd{M}_n\times_kY\big\}_{n\in \NN}$.
\end{example}
\noindent
Let us note the following interesting consequence of previous theorems.

\begin{corollary}\label{corollary:morphisms_of_locally_linear_schemes_are_morphisms_of_formal_neighborhoods}
Let $\bd{M}$ be a Kempf monoid and let $Z,W$ be locally linear $\bd{M}$-schemes. Then the canonical map
\begin{center}
\begin{tikzpicture}
[description/.style={fill=white,inner sep=2pt}]
\matrix (m) [matrix of math nodes, row sep=3em, column sep=2em,text height=1.5ex, text depth=0.25ex] 
{\Mor_{\bd{M}}\big(Z,W\big)  & \Mor\big(\widehat{Z},\widehat{W}\big)\\} ;
\path[->,line width=1.0pt,font=\scriptsize]  
(m-1-1) edge node[above] {$ $} (m-1-2);
\end{tikzpicture}
\end{center}
is a bijection, where $\Mor_{\bd{M}}\big(Z,W\big)$ is the class of $\bd{M}$-equivariant morphisms $Z\ra W$ and $\Mor\big(\widehat{Z},\widehat{W}\big)$ is the class of morphisms of formal $\bd{M}$-schemes.
\end{corollary}
\begin{proof}
Suppose that $\widehat{Z} = \{Z_n\}_{n\in \NN}$ and $\widehat{W} = \{W_n\}_{n\in \NN}$. Consider a morphism $\{f_n:Z_n\ra W_n\}_{n\in \NN}$ of formal $\bd{M}$-schemes. Since the retraction on fixed points in Proposition \ref{proposition:retraction_for_monoids_with_zero} is given by the multiplication by zero of $\bd{M}$, we derive that the square
\begin{center}
\begin{tikzpicture}
[description/.style={fill=white,inner sep=2pt}]
\matrix (m) [matrix of math nodes, row sep=3em, column sep=3em,text height=1.5ex, text depth=0.25ex] 
{ Z_n & W_n    \\
  Z_0 & W_0                   \\} ;
\path[->,line width=1.0pt,font=\scriptsize]  
(m-1-1) edge node[above] {$f_n  $} (m-1-2)
(m-2-1) edge node[below] {$f_0  $} (m-2-2);
\path[->>,line width=1.0pt,font=\scriptsize]  
(m-1-1) edge node[left] {$ r^Z_n $} (m-2-1)
(m-1-2) edge node[right] {$ r^W_n  $} (m-2-2);
\end{tikzpicture}
\end{center}
is commutative for $n\in \NN$, where $r^W_n$ and $r^Z_n$ are canonical retractions. Hence for every $n\in \NN$ we have a diagram
\begin{center}
\begin{tikzpicture}
[description/.style={fill=white,inner sep=2pt}]
\matrix (m) [matrix of math nodes, row sep=4em, column sep=4em,text height=1.5ex, text depth=0.25ex] 
{ Z_n & W_n\times_{W_0}Z_0        & W_n    \\
      & Z_0                       & W_0           \\} ;
\path[->,bend left, line width=1.0pt,font=\scriptsize]
(m-1-1) edge node[above] {$ f_n $} (m-1-3);
\path[->,line width=1.0pt,font=\scriptsize]  
(m-1-1) edge node[above] {$ p_n $} (m-1-2)
(m-1-2) edge node[above] {$ q_n $} (m-1-3)
(m-2-2) edge node[below] {$ f_0 $} (m-2-3);
\path[->>,line width=1.0pt,font=\scriptsize]
(m-1-1) edge node[left = 3pt, below = 3pt] {$r^Z_n  $} (m-2-2)
(m-1-2) edge node[left] {$r_n  $} (m-2-2)
(m-1-3) edge node[left] {$r^W_n  $} (m-2-3);
\end{tikzpicture}
\end{center}
in which the rightmost square is cartesian and $p_n:Z_n\ra W_n\times_{W_0}Z_0$ is the unique morphism that makes the diagram commutative. Now since $r^W_n$ is affine by \textbf{(3)} of Proposition \ref{proposition:retraction_for_monoids_with_zero}, we derive that $r_n:W_n\times_{W_0}Z_0\ra Z_0$ is affine as its base change. Our goal is to show that there exists a unique morphism $p:Z\ra W\times_{W_0}Z_0$ such that the square
\begin{center}
\begin{tikzpicture}
[description/.style={fill=white,inner sep=2pt}]
\matrix (m) [matrix of math nodes, row sep=3em, column sep=3em,text height=1.5ex, text depth=0.25ex] 
{ Z   & W\times_{W_0}Z_0    \\
  Z_n & W_n\times_{W_0}Z_0                   \\} ;
\path[densely dotted,->,line width=1.0pt,font=\scriptsize]  
(m-1-1) edge node[above] {$p  $} (m-1-2);
\path[->,line width=1.0pt,font=\scriptsize]  
(m-2-1) edge node[below] {$p_n  $} (m-2-2);
\path[right hook->,line width=1.0pt,font=\scriptsize]  
(m-2-1) edge node[left] {$  $} (m-1-1)
(m-2-2) edge node[right] {$  $} (m-1-2);
\end{tikzpicture}
\end{center}
is commutative. Theorems \ref{theorem:uniqueness_of_algebraization_for_formal_schemes_over_Kampf_monoid} and \ref{theorem:every_formal_over_kempf_monoid_is_formal_neighborhood} imply that 
$$Z = \mathrm{colim}_{n\in \NN}Z_n$$
in the category of $\bd{M}$-schemes which are affine over $Z_0$. Thus by the universal property of colimit we deduce that $q$ exists. Now composing $q$ with the morphism $q:W\times_{W_0}Z_0\ra W$ coming from the cartesian square
\begin{center}
\begin{tikzpicture}
[description/.style={fill=white,inner sep=2pt}]
\matrix (m) [matrix of math nodes, row sep=3em, column sep=3em,text height=1.5ex, text depth=0.25ex] 
{ W\times_{W_0}Z_0 & W    \\
  Z_0 & W_0                   \\} ;
\path[->,line width=1.0pt,font=\scriptsize]  
(m-1-1) edge node[above] {$q  $} (m-1-2)
(m-2-1) edge node[below] {$f_0  $} (m-2-2);
\path[->>,line width=1.0pt,font=\scriptsize]  
(m-1-1) edge node[left] {$ r $} (m-2-1)
(m-1-2) edge node[right] {$ r^W  $} (m-2-2);
\end{tikzpicture}
\end{center}
we obtain $\bd{M}$-equivariant morphism $f = q\cdot p:Z\ra W$. By construction $f_{\mid Z_n}$ induces $f_n$ for every $n\in \NN$ and by uniqueness of $p$ we infer that $f$ is a unique $\bd{M}$-equivariant morphism with this property. This completes the proof.
\end{proof}

\begin{theorem}\label{theorem:if_formal_neighborhood_is_locally_noetherian_then_canonical_retraction_is_of_finite_type}
Let $\bd{M}$ be a Kempf monoid and let $Z$ be a locally linear $\bd{M}$-scheme. Suppose that $r:Z\ra Z^{\bd{M}}$ is the canonical retraction. If the formal $\bd{M}$-scheme $\widehat{Z}$ is locally noetherian, then $r$ is of finite type.
\end{theorem}
\begin{proof}
Since $r$ is affine (Proposition \ref{proposition:retraction_for_monoids_with_zero}), we derive that $\cA = r_*\cO_Z$ is a quasi-coherent $\bd{M}$-algebra on $Z^{\bd{M}}$. We denote by $\cJ$ the ideal of $\cA$ that corresponds to the closed immersion $Z^{\bd{M}}\hookrightarrow Z$. We know that the formal $\bd{M}$-scheme
\begin{center}
\begin{tikzpicture}
[description/.style={fill=white,inner sep=2pt}]
\matrix (m) [matrix of math nodes, row sep=3em, column sep=3em,text height=1.5ex, text depth=0.25ex] 
{ Z^{\bd{M}} = \Spec_{Z^{\bd{M}}}\cA/\cJ &  ... &  \Spec_{Z^{\bd{M}}}\cA/\cJ^{n+1} &  \Spec_{Z^{\bd{M}}}\cA/\cJ^{n+2} & ... \\} ;
\path[right hook->,line width=1.0pt,font=\scriptsize]  
(m-1-1) edge node[above] {$ $} (m-1-2)
(m-1-2) edge node[above] {$ $} (m-1-3)
(m-1-3) edge node[above] {$ $} (m-1-4)
(m-1-4) edge node[above] {$ $} (m-1-5);
\end{tikzpicture}
\end{center}
is locally noetherian. Hence $\cJ/\cJ^{n+1}$ is $\cA/\cJ^{n+1}$-module of finite type. Thus $\{\cJ^i/\cJ^{i+1}\}_{1\leq i\leq n}$ are finite type $\cA/\cJ$-modules. The series
$$0 \subseteq \cJ^n/\cJ^{n+1} \subseteq  ... \subseteq \cJ/\cJ^{n+1} \subseteq \cA/\cJ^{n+1}$$
has subquotients that are of finite type over $\cO_{Z^{\bd{M}}} = \cA/\cJ$. This implies that $\cA/\cJ^{n+1}$ is a coherent $\cO_{Z^{\bd{M}}}$-algebra for every $n\in \NN$. The claim that $r$ is of finite type is local on $Z^{\bd{M}}$, hence we may assume that $Z^{\bd{M}}$ is quasi-compact. This reduces the question to the noetherian $Z^{\bd{M}}$. The sheaf $\cJ/\cJ^2\subseteq \cA/\cJ^2$ is coherent over $\cO_{Z^{\bd{M}}}$. Since $Z^{\bd{M}}$ is noetherian, there exists coherent $\cO_{Z^{\bd{M}}}$-subsheaf $\cM\subseteq \cJ$ such that the morphism $\cM\twoheadrightarrow \cJ/\cJ^2$ is surjective. Fix an algebraically closed extension $K$ of $k$ and denote
$$\cA_K = K\otimes_k\cA,\cJ_K = K\otimes_k\cJ,\cM_K = K\otimes_k\cM$$
Since $\bd{M}$ is a Kempf monoid and by {\cite[Corollary 3.7]{Algebraic_monoids}} there exists a closed immersion $\mathbb{A}^1_K\hookrightarrow \bd{M}_K$ of monoid $K$-schemes that preserve zero. This implies that we have $\NN$-grading $\cA_K = \bigoplus_{i\geq 0}\cA_K[i]$ that gives rise to the action of $\mathbb{A}^1_K$. Moreover, by Proposition \ref{proposition:retraction_for_monoids_with_zero} we deduce that
$$\Spec K\times_k Z^{\bd{M}} = \left(\Spec K \times_k Z\right)^{\bd{M}_K} = \left(\Spec K \times_k Z\right)^{\mathbb{A}^1_K}$$
as $K$-schemes. This shows that $\cJ_K = K\otimes_k\cJ = \bigoplus_{i\geq 1} \cA_K[i]$ is an ideal with positive grading. We have surjection $\cM_K\twoheadrightarrow \cJ_K/\cJ_K^2$. By graded version of Nakayama's lemma, the ideal $\cJ_K$  is generated by $\cM_K$. Then by induction on degrees we deduce that $\cA_K$ is generated by $\cM_K$ as a $K\otimes_k\cO_{Z^{\bd{M}}}$-algebra. Thus $1_{\Spec K}\times_k r$ is of finite type and by faitfully flat descent also $r$ is of finite type.
\end{proof}

\begin{theorem}\label{theorem:comparison_functor_is_an_equivalence}
Let $\bd{M}$ be a Kempf monoid with group of unit $\bd{G}$ and let $Z$ be a locally linear $\bd{M}$-scheme. Suppose that $r:Z\ra Z^{\bd{M}}$ is the canonical retraction. If $Z$ is locally noetherian, then the comparison functor
$$\Coh_{\bd{G}}(Z)\ra \Coh_{\bd{G}}(\widehat{Z})$$
is an equivalence of monoidal categories.
\end{theorem}
\begin{proof}[Setup]
Since $\bd{M}$ is a Kempf torus, there exists a central closed torus $T$ in $\bd{G}$ such that the scheme-theoretic closure $\ol{T}$ of $T$ in $\bd{M}$ contains the zero. As above we note that $\pi$ is affine (Proposition \ref{proposition:retraction_for_monoids_with_zero}) and we pick a quasi-coherent $\bd{M}$--algebra $\cA = r_*\cO_Z$ on $Z^{\bd{M}}$. We denote by $\cJ$ the ideal of $\cA$ that corresponds to the closed immersion $Z^{\bd{M}}\hookrightarrow Z$. Then $\cO_{Z^{\bd{M}}} = \cA/\cJ$ and since $r$ is a retraction, we derive that $\cA = \cO_{Z^{\bd{M}}}\oplus \cJ$. Next $\widehat{Z}$ is locally noetherian (this follows from the fact that $Z$ is locally noetherian). By Remark \ref{remark:equivariant_scheme_affine_over_trivial_scheme} and Remark \ref{remark:2_limit_of_a_telescope_description} an object of $\Coh_{\bd{G}}(\widehat{Z})$ corresponds to a sequence of surjections
\begin{center}
\begin{tikzpicture}
[description/.style={fill=white,inner sep=2pt}]
\matrix (m) [matrix of math nodes, row sep=3em, column sep=2em,text height=1.5ex, text depth=0.25ex] 
{ ... &  \cM_{n+1} & \cM_n & ...& \cM_1 & \cM_0 \\} ;
\path[->>,line width=1.0pt,font=\scriptsize]  
(m-1-1) edge node[above] {$ $} (m-1-2)
(m-1-2) edge node[above] {$ $} (m-1-3)
(m-1-3) edge node[above] {$ $} (m-1-4)
(m-1-4) edge node[above] {$ $} (m-1-5)
(m-1-5) edge node[above] {$ $} (m-1-6);
\end{tikzpicture}
\end{center}
of coherent sheaves on $Z^{\bd{M}}$ such that the following assertions hold.
\begin{enumerate}[label=\textbf{(\arabic*)}, leftmargin=3.0em]
\item $\cM_n$ is a module over $\cA_n$ for every $n\in \NN$.
\item $\cJ^{n+1}\cM_{n+1}$ is the kernel of the epimorphism $\cM_{n+1}\twoheadrightarrow \cM_n$ for every $n\in \NN$.
\item For each $n\in \NN$ there exists a morphism $\cM_n\ra k[\bd{M}]\otimes_k\cM_n$ such that for every open affine neighborhood $U$ of $Z_0$ its restriction 
$$\cM_n(U)\ra k[\bd{G}]\otimes_k \cM_n(U)$$
to sections on $U$ is a coaction of $k[\bd{G}]$ on $\cM_n(U)$.
\item $\cM_n \ra k[\bd{G}]\otimes_k\cM_n$ is the morphism of $\cA$-modules, where $k[\bd{G}]\otimes_k\cM_n$ carries the structure of an $\cA$-module induced by restriction of its $k[\bd{G}]\otimes_k\cA_n$-module structure along the morphism $\cA_n\ra k[\bd{G}]\otimes_k\cA_n$ that corresponds to the action of $\bd{G}$ on $Z_n$.
\item For every $n\in \NN$ the epimorphism $\cM_{n+1} \twoheadrightarrow \cM_n$ preserves coaction described in \textbf{(3)}.
\end{enumerate}
We fix an algebraically closed field $K$ containing $k$. By {\cite[Corollary 3.7]{Algebraic_monoids}} there exists a closed immersion $\Spec K\times_k \mathbb{G}_m\hookrightarrow T_K$ of group $K$-schemes that induces zero preserving closed immersion $\mathbb{A}^1_K\hookrightarrow \ol{T}_K$ of monoid $K$-schemes. By Proposition \ref{proposition:retraction_for_monoids_with_zero} we have
$$\Spec K\times_k Z^{\bd{M}} = \left(\Spec K\times_k Z\right)^{\bd{M}_K} = \left(\Spec K\times_k Z\right)^{\mathbb{A}^1_K}$$
This implies that
$$\cA_K = K\otimes_k\cA = \bigoplus_{i\geq 0}\cA_K[i],\,\cJ_K = K\otimes_k\cJ = \bigoplus_{i\geq 1}\cA_K[i]$$
where gradation is induced by the action of $\mathbb{A}^1_K$. For every $n\in \NN$ the action of $\Spec K\times_k \mathbb{G}_m$ on $K\otimes_k\cM_n$ induced by the closed immersion $\Spec K\times_k \mathbb{G}_m \hookrightarrow \ol{T}_K\hookrightarrow \bd{M}_K$ of group $K$-schemes gives rise to a gradation
$$K\otimes_k\cM_n = \bigoplus_{i\in \ZZ}\left(K\otimes_k\cM_n\right)[i]$$
\end{proof}

\begin{lemma}\label{lemma:boundedness_of_gradation_and_two_stabilizations}
The following assertions hold.
\begin{enumerate}[label=\emph{\textbf{(\arabic*)}}, leftmargin=3.0em]
\item There exists $i_0\in \ZZ$ such that for every $n\in \NN$ we have $\left(K\otimes_k\cM_n\right)[i] = 0$ for $i \geq i_0$.
\item For every $i\in \ZZ$ there exists $n_i\in \NN$ such that for all $n\geq n_i$ the surjection $\left(K\otimes_k\cM_{n+1}\right)[i]\twoheadrightarrow \left(K\otimes_k\cM_n\right)[i]$ is an isomorphisms.
\item For every $\lambda$ in $\bd{Irr}(T)$ there exists a finite subset $B_{\lambda}\subseteq \ZZ$ such that
$$K\otimes_k V_{\lambda} = \bigoplus_{i\in B_{\lambda}}\left(K\otimes_k V_{\lambda}\right)[i]$$
Define $n_{\lambda} = \sup_{i\in B_{\lambda}}n_i \in \NN$. Then for all $n\geq n_{\lambda}$ the surjection $\cM_{n+1}[\lambda]\twoheadrightarrow \cM_n[\lambda]$ is an isomorphisms.
\end{enumerate}
\end{lemma}
\begin{proof}[Proof of the lemma]
Fix $n\in \NN$ and consider the decomposition $K\otimes_k\cM_n = \bigoplus_{i\in \ZZ}\left(K\otimes_k\cM_n\right)[i]$. Since $K\otimes_k\cM_n$ is a coherent $K\otimes_k \cO_{Z^{\bd{M}}}$-module and the decomposition consists of modules over $K\otimes_k\cO_{Z^{\bd{M}}}$, we derive that there are only finitely many $i\in \ZZ$ such that $\left(K\otimes_k\cM_n\right)[i]\neq 0$. Hence we may write $K\otimes_k\cM_n = \bigoplus_{i\geq i_n}\left(K\otimes_k\cM_n\right)[i]$ for some $i_n\in \ZZ$ such that $\left(K\otimes_k\cM_n\right)[i_n] \neq 0$. Moreover, we know that the kernel of the surjection
$$K\otimes_k\cM_{n+1} = \bigoplus_{i\geq i_{n+1}}\left(K\otimes_k\cM_{n+1}\right)[i] \twoheadrightarrow \bigoplus_{i\geq i_n}\left(K\otimes_k\cM_n\right)[i] = K\otimes_k\cM_n$$
is $\cJ_K^{n+1}\left(K\otimes_k\cM_{n+1}\right)$ and hence is contained in $\bigoplus_{i\geq (i_{n+1}+n+1)}\left(K\otimes_k\cM_{n+1}\right)[i]$
This implies that $\left(K\otimes_k\cM_{n}\right)[i]  = \left(K\otimes_k\cM_{n+1}\right)[i]$ for $i_{n+1}\leq i \leq i_{n+1}+n$. In particular, we have $\left(K\otimes_k\cM_{n}\right)[i_{n+1}]  = \left(K\otimes_k\cM_{n+1}\right)[i_{n+1}] \neq 0$ and thus $i_{n+1} \geq i_n$. This shows that $i_n\geq i_0$ for every $n\in \NN$ and \textbf{(1)} is proved. Now the surjection
$$K\otimes_k\cM_{n+1} = \bigoplus_{i\geq i_{0}}\left(K\otimes_k\cM_{n+1}\right)[i] \twoheadrightarrow \bigoplus_{i\geq i_0}\left(K\otimes_k\cM_n\right)[i] = K\otimes_k\cM_n$$
induces an isomorphism for $i$-th graded component, where $i_0 \leq i\leq i_0+n$. Hence for fixed $i\in \ZZ$ there exists $n_i\in \NN$ such that for all $n\geq n_i$ the surjection $\left(K\otimes_k\cM_{n+1}\right)[i] \twoheadrightarrow \left(K\otimes_k\cM_n\right)[i]$ is an isomorphism. Thus we proved \textbf{(2)}.\\
Fix now $\lambda$ in $\bd{Irr}(T)$ and let $V_{\lambda}$ be an irreducible representation in class $\lambda$. Since $K\otimes_kV_{\lambda}$ is a finite dimensional vector space over $K$, there exists a finite subset $B_{\lambda}\subseteq \ZZ$ such that for $\left(K\otimes_kV_{\lambda}\right)[i] \neq 0$ if $i\in B_{\lambda}$. Now define $n_{\lambda} = \sup_{i\in B_{\lambda}}n_i$ the surjection $K\otimes_k\cM_{n+1}\twoheadrightarrow K\otimes_k\cM_n$ induces an isomorphism $\left(K\otimes_k\cM_{n+1}\right)[i] \cong \left(K\otimes_k\cM_n\right)[i]$ for every $i$ in $B_{\lambda}$. Thus for $n\geq n_{\lambda}$ the surjection $\cM_{n+1}\twoheadrightarrow \cM_n$ induces an isomorphism $\cM_{n+1}[\lambda]\cong \cM_n[\lambda]$. This completes the proof of \textbf{(3)}.
\end{proof}

\begin{proof}[Proof of the theorem]
Fix a coherent $\bd{G}$-sheaf $\{\cM_n\}_{n\in \NN}$ on $\widehat{Z}$ described as in the setup above. For fixed $\lambda$ in $\bd{Irr}(T)$ we define $\cM[\lambda] = \cM_n[\lambda]$ for any $n\geq n_{\lambda}$, where $n_{\lambda}\in \NN$ is as in \textbf{(3)} of Lemma \ref{lemma:boundedness_of_gradation_and_two_stabilizations} (in particular, $\cM[\lambda]$ does not depend on $n\geq n_{\lambda}$). Next we define
$$\cM = \bigoplus_{\lambda\in \bd{Irr}}\cM[\lambda]$$
By Proposition \ref{proposition:commuting_action_preserves_isotypic_decomposition} for every $n\in \NN$ and $\lambda \in \bd{Irr}(T)$ sheaf $\cM_n[\lambda]$ admits a structure of a $\bd{G}$-sheaf. Therefore, $\cM$ is a quasi-coherent $\bd{G}$-sheaf of $\cO_{Z^{\bd{M}}}$-modules. We now show that $\cM$ admits a canonical structure of $\cA$-module. For this pick $\lambda_1$ and $\lambda_2$ in $\bd{Irr}(T)$. Consider the irreducible representations $V_{\lambda_1}$ and $V_{\lambda_1}$ in classes $\lambda_1$ and $\lambda_2$, respectively. Suppose that $\eta_1,...,\eta_s$ are finitely many classes in $\bd{Irr}(T)$ such that $V_{\lambda_1}\otimes_k V_{\lambda_2}$ can be completely decomposed into irreducible representations contained in classes $\eta_1,...,\eta_s$. According to Fact \ref{fact:tensor_product_of_isotypic_components} the image of the multiplication $\cA[\lambda_1]\otimes_{\cO_{Z^{\bd{M}}}}\cM_n[\lambda_2]\ra \cM_n$ is contained in $\bigoplus_{i=1}^s\cM_n[\eta_i]$. By \textbf{(3)} of Lemma \ref{lemma:boundedness_of_gradation_and_two_stabilizations} all these multiplications for $n\geq \sup \{n_{\lambda_1},n_{\lambda_2},n_{\eta_1},...,n_{\eta_s}\}$ can be identified. Now we define
$$\cA[\lambda_1]\otimes_{\cO_{Z^{\bd{M}}}} \cM[\lambda_2]\ra  \bigoplus_{i=1}^s\cM[\eta_i]\subseteq \cM$$
as a morphism induced by the multiplication morphism for any $n\geq \sup\{n_{\lambda_1},n_{\lambda_2},n_{\eta_1},...,n_{\eta_s}\}$. This gives an $\cA$-module structure on $\cM$. Next we prove that $\cM$ is an $\cA$-module of finite type. Denote $K\otimes_k\cM$ by $\cM_K$. Note that the combination of \textbf{(2)} and \textbf{(3)} of Lemma \ref{lemma:boundedness_of_gradation_and_two_stabilizations} show that
$$\cM_K[i] = \left(K\otimes_k\cM_n\right)[i]$$
for $n\geq n_i$ and $i\geq i_0$. Hence by \textbf{(1)} of Lemma \ref{lemma:boundedness_of_gradation_and_two_stabilizations} we have
$$\bigoplus_{\lambda\in \bd{Irr}(T)}\cM[\lambda]_K = \cM_K = \bigoplus_{i\geq i_0}\cM_K[i]$$
Since each $\cM_n$ is a coherent $\cO_{Z^{\bd{M}}}$-module, we derive that $\cM_K[i]$ is a coherent $K\otimes_k\cO_{Z^{\bd{M}}}$-module for every $i\in \ZZ$. Now we may pick $\lambda_1,...,\lambda_r$ in $\bd{Irr}(T)$ such that we have a surjection
$$\bigoplus_{j=1}^r\cM[\lambda_j]_K \twoheadrightarrow \bigoplus_{i_0\leq i\leq 1}\cM_K[i]$$
induced by the projection $\cM_K = \bigoplus_{i\geq i_0}\cM_K[i]\twoheadrightarrow \bigoplus_{i_0\leq i\leq 1}\cM_K[i]$. Let
$$\cG = \bigoplus_{j=1}^r\cM[\lambda_j]$$
be a $\cO_{Z^{\bd{M}}}$-submodule of $\cM$. Clearly each $\cM[\lambda]$ is a coherent $\cO_{Z^{\bd{M}}}$-module. Hence $\cG$ is a coherent $\cO_{Z^{\bd{M}}}$-module. Since $\cJ_K = \bigoplus_{i\geq 1}\cA_K[i]$, we derive that
$$\cM_K = \sum_{j\geq 1}\cJ_K^j\cdot \cG_K$$
and hence $\cG_K$ generates $\cM_K$ as an $\cA_K$-module. By faithfully flat descent we deduce that $\cG$ generates $\cM$ as an $\cA$-module. Since $\cG$ is a coherent $\cO_{Z^{\bd{M}}}$-module, we derive that $\cM$ is an $\cA$-module of finite type. We show that  $\cM/\cJ^{n+1}\cM = \cM_{n}$ for every $n\in \NN$. Fix $n\in \NN$. By faithfully flat descent it suffices to show that
$$\left(\cM_K/\cJ_K^{n+1}\cM_K\right)[i] = \left(K\otimes_k\cM_n\right)[i]$$
for every $i\in \ZZ$. Let us fix $i\in \ZZ$. Pick $m$ greater than $\sup_{i_0\leq j\leq i}n_j$ and $n$. Then
$$\cM_K[j] = \left(K\otimes_k\cM_m\right)[j],\,\cJ_K^{n+1}\cM_K[j] = \cJ_K^{n+1}\left(K\otimes_k \cM_m\right)[j]$$
for $i_0\leq j\leq i$. Since $\cM_m/\cJ^{n+1}\cM_m = \cM_n$ as $m \geq n$, we derive that
$$\left(\cM_K/\cJ_K^{n+1}\cM_K\right)[i] = \cM_K[i]/\cJ_K^{n+1}\cM_K[i] = \left(K\otimes_k\cM_m\right)[i]/\cJ_K^{n+1}\left(K\otimes_k \cM_m\right)[i] = \left(K\otimes_k\cM_n\right)[i]$$
and this completes the proof of our claim. All these facts imply that $\cM$ corresponds to a coherent $\bd{G}$-sheaf on $Z$ such that its image under the comparison functor $\Coh_{\bd{G}}(Z)\ra \Coh_{\bd{G}}(\widehat{Z})$ is a coherent $\bd{G}$-sheaf on $\widehat{Z}$ with $\bd{G}$-structure described by $\{\cM_n\}_{n\in \NN}$. Hence the comparison functor is essentialy surjective. Note also that
$$\cM = \mathrm{colim}_{n\in \NN}\cM_n$$
in the category of sheaves of $\cO_{Z^{\bd{M}}}$-modules. Now we are going to prove that $\Coh_{\bd{G}}(Z)\ra \Coh_{\bd{G}}(\widehat{Z})$ is full and faithful. For this consider a commutative diagram
\begin{center}
\begin{tikzpicture}
[description/.style={fill=white,inner sep=2pt}]
\matrix (m) [matrix of math nodes, row sep=3em, column sep=2em,text height=1.5ex, text depth=0.25ex]
{ ... &  \cM_{n+1} & \cM_n & ...& \cM_1 & \cM_0 \\
 ... &  \cN_{n+1} & \cN_n & ...& \cN_1 & \cN_0 \\} ;
\path[->>,line width=1.0pt,font=\scriptsize]  
(m-1-1) edge node[above] {$ $} (m-1-2)
(m-1-2) edge node[above] {$ $} (m-1-3)
(m-1-3) edge node[above] {$ $} (m-1-4)
(m-1-4) edge node[above] {$ $} (m-1-5)
(m-1-5) edge node[above] {$ $} (m-1-6)

(m-2-1) edge node[above] {$ $} (m-2-2)
(m-2-2) edge node[above] {$ $} (m-2-3)
(m-2-3) edge node[above] {$ $} (m-2-4)
(m-2-4) edge node[above] {$ $} (m-2-5)
(m-2-5) edge node[above] {$ $} (m-2-6);
\path[->,line width=1.0pt,font=\scriptsize]
(m-1-2) edge node[left] {$f_{n+1} $} (m-2-2)
(m-1-3) edge node[left] {$f_n $} (m-2-3)
(m-1-5) edge node[left] {$f_1 $} (m-2-5)
(m-1-6) edge node[left] {$f_0 $} (m-2-6);
\end{tikzpicture}
\end{center}
that represents the morphism in $\Coh_{\bd{G}}(\widehat{Z})$. This means that $f_n$ is a morphism of $\cA/\cJ^{n+1}$-modules and preserves the $k[\bd{G}]$-coactions for every $n\in \NN$. Next suppose that $\cN$ is an $\cA$-module with $k[\bd{G}]$-coaction that corresponds to an object of $\Coh_{\bd{G}}(Z)$ which image under the comparison functor yields $\{\cN_n\}_{n\in \NN}$. We define $f:\cM \ra \cN$ as follows. We pick $\lambda\in \bd{Irr}(T)$ and set $f[\lambda]:\cM[\lambda]\ra \cN[\lambda]$ to be $f_n[\lambda]:\cM_n[\lambda]\ra \cN_n[\lambda]$ for sufficiently large $n\in \NN$. By \textbf{(3)} of Lemma \ref{lemma:boundedness_of_gradation_and_two_stabilizations} this definition makes sense and by construction of $\cA$-module structure on $\cM$ and $\cN$ gives rise to a morphism of $\cA$-modules that preserves the $\bd{G}$-coactions. Moreover, we have
$$f = \mathrm{colim}_{n\in \NN}f_n$$
in the category of sheaves of $\cO_{Z^{\bd{M}}}$-modules. Thus $f$ is a unique morphism of sheaves of $\cO_{Z^{\bd{M}}}$ such that the square
\begin{center}
\begin{tikzpicture}
[description/.style={fill=white,inner sep=2pt}]
\matrix (m) [matrix of math nodes, row sep=3em, column sep=3em,text height=1.5ex, text depth=0.25ex] 
{ \cM & \cM_n     \\
  \cN & \cN_n           \\} ;
\path[->>,line width=1.0pt,font=\scriptsize]  
(m-1-1) edge node[above] {$   $} (m-1-2)
(m-2-1) edge node[below] {$   $} (m-2-2);
\path[->,line width=1.0pt,font=\scriptsize]  
(m-1-1) edge node[left] {$  f $} (m-2-1)
(m-1-2) edge node[right] {$ f_n  $} (m-2-2);
\end{tikzpicture}
\end{center}
is commutative for every $n\in \NN$. Next denote $K\otimes_k f = f_K$ and fix $i\in \ZZ$. Then by \textbf{(2)} and \textbf{(3)} of Lemma \ref{lemma:boundedness_of_gradation_and_two_stabilizations} we have
$$f_K[i] = (K\otimes_kf_n)[i]$$
for sufficiently large $n\in \NN$. Fix now $n\in \NN$. According to \textbf{(1)} of Lemma \ref{lemma:boundedness_of_gradation_and_two_stabilizations} for $i\in \ZZ$ we may pick $m\geq n$ such that
$$f_K[j] = (K\otimes_kf_m)[j]$$
for all $j \leq i$. Thus
$$f_K[i]\,\mathrm{mod}\,\cJ_K^{n+1}\cM_K[i] = (1_K\otimes_kf_m)[i]\,\mathrm{mod}\,\cJ_K^{n+1}(K\otimes_k\cM_m)[i] = (1_K\otimes_kf_n)[i]$$
Since $i\in \ZZ$ is aribtrary, we derive that
$$f_K\,\mathrm{mod}\,\cJ_K^{n+1}\cM_K = (1_K\otimes_kf_n)$$
By faithfully flat descent we deduce that $f_n = \left(1_{\cA/\cJ^{n+1}}\otimes_{\cA}f\right)$ for every $n\in \NN$. Therefore,  $f$ is a unique morphism in $\Coh_{\bd{G}}(Z)$ such that its image under the comparison functor is $\{f_n\}_{n\in \NN}$. This completes the proof that the comparison functor is full and faithfull. We proved that it is essentially surjective above. Thus the comparison functor is an equivalence of categories.
\end{proof}

\begin{definition}
Let $\bd{M}$ be a monoid $k$-scheme with group of units $\bd{G}$. Let $\cZ = \{Z_n\}_{n\in \mathbb{N}}$ be a locally noetherian formal $\bd{M}$-scheme. A locally noetherian $\bd{M}$-scheme $Z$ is called \textit{an algebraization of $\cZ$} if the following two conditions are satisfied.
\begin{enumerate}[label=\textbf{(\arabic*)}, leftmargin=3.0em]
\item $\cZ$ is isomorphic to $\widehat{Z}$ in the category of formal $\bd{M}$-schemes.
\item The comparison functor $\Coh_{\bd{G}}(Z)\ra \Coh_{\bd{G}}\left(\widehat{Z}\right)$ is an equivalence of monoidal categories.
\end{enumerate}
\end{definition}

\begin{corollary}\label{corollary:existence_of_algebraization}
Let $\bd{M}$ be a Kempf monoid with $\bd{G}$ as a group of units and let $\cZ = \{Z_n\}_{n\in \NN}$ be a locally noetherian formal $\bd{M}$-scheme. Then $\cZ$ admits an algebraization.
\end{corollary}
\begin{proof}
By Theorem \ref{theorem:every_formal_over_kempf_monoid_is_formal_neighborhood} there exists a locally linear $\bd{M}$-scheme $Z$ such that $\widehat{Z}$ is isomorphic with $\cZ$. By Theorem \ref{theorem:if_formal_neighborhood_is_locally_noetherian_then_canonical_retraction_is_of_finite_type} we deduce that the canonical retraction $r:Z\ra Z^{\bd{M}}$ is of finite type. Hence $Z$ is locally noetherian. Finally Theorem \ref{theorem:comparison_functor_is_an_equivalence} implies that the comparison functor $\Coh_{\bd{G}}(Z) \ra \Coh_{\bd{G}}(\widehat{Z})$ is an equivalence of categories. Thus $Z$ is an algebraization of $\cZ$.
\end{proof}



\small
\bibliographystyle{apalike}
\bibliography{../zzz}

\end{document}
