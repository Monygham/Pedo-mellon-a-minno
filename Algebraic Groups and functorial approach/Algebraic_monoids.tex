\input ../pree.tex

\begin{document}

\title{Algebraic monoids}
\date{}
\maketitle

\section{The unit group of an algebraic monoid}
\noindent
We start by proving result on generic finiteness.

\begin{theorem}\label{theorem:generically_finite_morphisms_are_finite_on_some_neighborhood}
Let $f:X\ra Y$ be a dominant morphism of finite type between irreducible schemes. Suppose that $\eta$ is a generic point and assume that the generic fiber $f^{-1}(\eta)$ is finite. Then there exists an open and nonempty subset $V$ of $Y$ such that the restriction $f^{-1}(V)\ra V$ of $V$ is finite.
\end{theorem}
\noindent
For the proof we need the following local version of the theorem.

\begin{lemma}\label{lemma:generically_finite_means_finite_on_some_neighborhood_local_version}
Let $A$ be a ring such that $\Spec A$ is irreducible and let $B$ be an $A$-algebra of finite type. Suppose that a unique minimal prime ideal $\ideal{p}$ of $A$ is nilpotent and $k(\ideal{p})\otimes_AB$ is finite over $k(\ideal{p})$, where $k(\ideal{p})$ denotes the residue field of $\ideal{p}$ in $A$. Then there exists nonzero $s$ in $A$ such that $B_s$ is a finite $A_s$-module.
\end{lemma}
\begin{proof}[Proof of the lemma]
Let $b_1,...,b_n$ be generators of $B$ as an $A$-algebra. Then
$$\ol{b_i} = b_i\,\mathrm{mod}\,\ideal{p}B$$
for $1 \leq i\leq n$ are generators of $B/\ideal{p}B$ as an $A/\ideal{p}$ algebra. Since $k(\ideal{p})\otimes_AB$ is finite over $k(\ideal{p})$ for each $i$ there exists positive integer $m_i$ and a polynomial
$$f_i(x) = s_{m_i}x^{m_i}+s_{m_i-1}x^{m_i-1}+...+s_0\in \left(A/\ideal{p}\right)[x]$$
such that $s_{m_i}\neq 0$ and $f_i(\ol{b_i}) = 0$. Let $s\in A$ be an element such that
$$\ol{s} = s\,\mathrm{mod}\,\ideal{p} = s_{m_1}\cdot s_{m_2}\cdot ...\cdot s_{m_n}$$
Clearly $s$ is nonzero and $B_s/\left(\ideal{p}B\right)_s = \left(B/\ideal{p}B\right)_s$ is a finite $A_s$-algebra. Hence there exists a finite $A_s$-submodule $M$ of $B_s$ such that
$$B_s = M + \left(\ideal{p}B\right)_s = M + \ideal{p}B_s$$
Since $\ideal{p}$ is nilpotent, there exists $N\in \mathbb{N}$ such that $\ideal{p}^N = 0$. Thus
$$B_s = M + \ideal{p}B_s = M + \ideal{p}M + \ideal{p}^2B_s = ... = M + \ideal{p}M + ... + \ideal{p}^{N-1}M + \ideal{p}^NB_s = M + \ideal{p}M + ... + \ideal{p}^{N-1}M$$
is a finite $A_s$-module.
\end{proof}

\begin{proof}[Proof of the theorem]
Pick an open, nonempty, affine neighborhood $W$ of $\eta$. Since $f$ is of finite type, we derive that
$$f^{-1}(W) = \bigcup_{i=1}^nU_i$$
where each $U_i$ is nonempty open affine subscheme of $X$ and moreover, the morphism $U_i\ra V$ induced by $f$ is of finite type. According to Lemma \ref{lemma:generically_finite_means_finite_on_some_neighborhood_local_version} for each $i$ there exists an open, affine and nonempty subscheme $W_i\subseteq W$ such that the morphism $f^{-1}(W_i)\cap U_i\ra W_i$ induced by $f$ is finite. Thus replacing $W$ by the intersection of $W_1,...,W_n$ we may assume that each $U_i\ra W$ is finite. Consider
$$F = f^{-1}(W) \setminus \left(\bigcap_{i=1}^nU_i\right)$$
Then $F$ is a closed subset of $f^{-1}(W)$ and it does not contain the generic point $\xi$ of $X$. Since each restriction $U_i\ra W$ of $f$ is finite, we derive that $f\big(U_i\cap F\big)$ is closed in $W$ for every $1\leq i\leq n$ and does not contain $\eta = f(\xi)$ ($f$ is dominant). Thus $f(F)$ is a closed subset of $W$ and $\eta \not \in f(F)$. Hence $V = W\setminus f(F)$ is an open neighborhood of $\eta$ and $f^{-1}(V) \subseteq \bigcap_{i=1}^nU_i$. Thus the restriction $f^{-1}(V)\ra V$ of $f$ is finite.
\end{proof}

\begin{theorem}\label{theorem:units_are_open_in_a_geometrically_integral_monoid}
Let $\bd{M}$ be a geometrically integral algebraic monoid $k$-scheme. Suppose that $\bd{G}$ is a group of units of $\bd{M}$ and $i:\bd{G}\hookrightarrow \bd{M}$ is the canonical monomorphism. Then $i$ is an open immersion.
\end{theorem}
\begin{proof}
Assume that $k$ is algebraically closed. Denote by $\mu:\bd{M}\times_k\bd{M}\ra \bd{M}$ and $e:\Spec k\ra \bd{M}$ the multiplication and the unit, respectively. Since $\bd{M}$ is integral and of finite type over $k$, we derive that $\bd{M}\times_k\bd{M}$ is integral and
$$\mathrm{dim}\left(\bd{M}\times_k\bd{M}\right) = 2\cdot \mathrm{dim}\left(\bd{M}\right)$$
Moreover, $\mu$ is surjective (which can be checked on $k$-functors of points). Pick any irreducible component $Z$ of $\mu^{-1}(e)$. By {\cite[Lemma 14.109]{gortz2010algebraic}} we deduce $$\mathrm{dim}\left(Z\right) \geq \mathrm{dim}\left(\mu^{-1}(\eta)\right)$$
where $\eta$ is the generic point of $\bd{M}$. Since 
$$\mathrm{dim}\left(\mu^{-1}(\eta)\right) = \mathrm{dim}\left(\bd{M}\times_k\bd{M}\right) - \mathrm{dim}\left(\bd{M}\right) =  2\cdot \mathrm{dim}\left(\bd{M}\right) - \mathrm{dim}\left(\bd{M}\right) = \mathrm{dim}\left(\bd{M}\right)$$
we deduce that $\mathrm{dim}\left(Z\right) \geq \mathrm{dim}(\bd{M})$. Moreover, we have $\bd{G} \cong \mu^{-1}(e)$ as $k$-schemes and this isomorphism is given by the restriction $\pi:\mu^{-1}(e)\ra \bd{G}$ to $\mu^{-1}(e)$ of the projection $\mathrm{pr}:\bd{M}\times_k\bd{M}\ra \bd{M}$ on the first factor (this can be checked on $k$-functors of points). Hence $\bd{G}$ is of finite type over $k$ as it is isomorphic with a closed subscheme of $\bd{M}\times_k\bd{M}$ and each irreducible component $Z$ of $\bd{G}$ is of dimension at least $ \mathrm{dim}(\bd{M})$. Now we fix an irreducible component $Z$ of $\bd{G}$ and consider it as a closed subscheme of $\bd{G}$ with reduced structure. Then the morphism $i_{\mid Z}:Z\hookrightarrow \bd{M}$ is a monomorphism of finite type and $\mathrm{dim}(Z) \geq \mathrm{dim}(\bd{M})$. Hence $i_{\mid Z}$ is dominant. Since $i$ is a monomorphism, this implies that $\bd{G}$ has only one irreducible component and $i:\bd{G}\hookrightarrow \bd{M}$ is dominant. By Theorem \ref{theorem:generically_finite_morphisms_are_finite_on_some_neighborhood} there exists an open and nonempty subset $V$ of $\bd{M}$ such that the morphism $i^{-1}(V)\hookrightarrow V$ induced by $i$ is finite. Finite monomorphisms are closed immersions and dominant, closed immersions with integral scheme as a codomain are isomorphisms. Thus $i^{-1}(V)\ra V$ is an isomorphism. Now pick a $k$-point $g$ of $\bd{G}$. Since $\bd{G}$ is a group $k$-scheme, we derive that $g\cdot (-):\bd{M}\ra \bd{M}$ is an automorphism of the $k$-scheme $\bd{M}$. This implies that $i^{-1}(g \cdot V)\ra g\cdot V$ is an isomorphism. This holds for every $k$-point of $\bd{G}$ and 
$$i(\bd{G})\subseteq \bigcup_{g\in \bd{G}(k)}g\cdot V$$
where $\bd{G}(k)$ is the set of $k$-points of $\bd{G}$. Therefore, $i$ is an open immersion.\\
If $k$ is not algebraically closed, then we pick an algebraically closed extension $K$ of $k$ and consider $1_{\Spec K}\times_k i$. This is an open immersion according to the case considered above. By faithfuly flat descent $i$ is an open immersion. 
\end{proof}
\noindent
The more general result for algebraically closed fields is {\cite[Theorem 1]{brion2014algebraic}}.

\begin{corollary}\label{corollary:unit_group_is_schematically_dense_and_open_in_geometrically_integral_algebraic_monoids}
Let $\bd{M}$ be a geometrically integral, algebraic monoid over $k$. Then the inclusion $\bd{G}\hookrightarrow \bd{M}$ of the group of units is schematically dense open immersion.
\end{corollary}

Let us also note the following theorem.

\begin{theorem}[{\cite[Chapitre 2, {\&}2, Corollaire 3.6]{demazure1970groupes}}]\label{theorem:units_are_open_in_an_affine_algebraic_monoid_and_it_is_a_closed_submonoid_of_a_matrix_monoid}
Let $\bd{M}$ be an affine, algebraic monoid $k$-scheme. Suppose that $\bd{G}$ is a group of units of $\bd{M}$. Then the following results holds.
\begin{enumerate}[label=\emph{\textbf{(\arabic*)}}, leftmargin=3.0em]
\item There exists a finite dimensional vector space $V$ over $k$ and a closed immersion
$$\bd{M}\hookrightarrow \Spec \Sym(V\otimes_kV^{\vee}) = \bd{L}(V)$$
of algebraic monoids.
\item There exists a regular function $f$ on $\bd{M}$ such that canonical morphism $\bd{G}\hookrightarrow \bd{M}$ is the inclusion of open subscheme of $\bd{M}$ on which $f$ is nonzero.
\end{enumerate}
\end{theorem}
\noindent
The converse is also true.

\begin{theorem}[{\cite[Theorem 2]{brion2014algebraic}}]\label{theorem:units_are_affine_then_monoid_is_affine}
Let $\bd{M}$ be a geometrically integral algebraic monoid over a field $k$ and let $\bd{G}$ be an group of units of $\bd{M}$. If $\bd{G}$ is affine, then $\bd{M}$ is affine.
\end{theorem}

\begin{definition}
Let $\bd{M}$ be a geometrically integral algebraic monoid over $k$ and let $\bd{G}$ be its group of units. If $\bd{G}$ is (linearly) reductive, then $\bd{M}$ is called \textit{a (linearly) reductive monoid over $k$}.
\end{definition}
\noindent
By definition every reductive group is affine. Hence using Theorem \ref{theorem:units_are_affine_then_monoid_is_affine} we deduce the following result.

\begin{corollary}\label{corollary:reductive_monoids_are_affine}
Let $\bd{M}$ be a reductive monoid over $k$. Then $\bd{M}$ is affine.
\end{corollary}

\section{Toric monoids}

\begin{definition}
Let $T$ be a torus over $k$ and let $\ol{T}$ be a geometrically integral, algebraic monoid having $T$ as the group of units. Then $\ol{T}$ is \textit{a toric monoid over $k$}.
\end{definition}

\begin{corollary}\label{corollary:every_toric_monoid_is_linearly_reductive}
Let $\ol{T}$ be a toric monoid over $k$. Then $\ol{T}$ is a linearly reductive monoid over $k$.
\end{corollary}
\begin{proof}
This follows from {\cite[Corollary 10.4]{Group_schemes_over_field}}
\end{proof}

\begin{theorem}\label{theorem:toric_monoids_properties_Kempf_torus}
Let $\ol{T}$ be a toric monoid over $k$ with group of units $T$ and let $K$ be an algebraically closed extension of $k$. Suppose that $N$ is a dimension of $T$.
\begin{enumerate}[label=\emph{\textbf{(\arabic*)}}, leftmargin=3.0em]
\item The group of characters of $T_K$ is isomorphic to $\ZZ^N$ and there exists an abstract submonoid $S$ of $\ZZ^N$ such that the open immersion
$$T_K = \Spec\left(\bigoplus_{m\in \ZZ^N}K\cdot \chi^m\right) \hookrightarrow \Spec\left(\bigoplus_{m\in S}K\cdot \chi^m\right) = \ol{T}_K$$
is induced by the inclusion $S\hookrightarrow \ZZ^N$.
\item Let $\{V_{\lambda}\}_{\lambda\in \bd{Irr}(T)}$ be a set of irreducible representation of $T$ such that $V_{\lambda}$ is in isomorphism class $\lambda$. For every $\lambda$ there exists a finite subset $A_{\lambda}$ of $\ZZ^N$ such that
$$K\otimes_kV_{\lambda} = \bigoplus_{m\in A_{\lambda}}K\cdot \chi^m$$
If $\lambda$ is in  $\bd{Irr}(\ol{T})$, then $A_{\lambda}$ is a subset of $S$. Moreover, we have
$$\ZZ^N = \coprod_{\lambda\in \bd{Irr}(T)}A_{\lambda}$$
and $A_{\lambda_0} = \{0\}$, where $\lambda_0$ is the class of the trivial representation of $T$.
\item If $\ol{T}$ has a zero, then there exists a homomorphism $f:\ZZ^N\ra \ZZ$ of abelian groups such that $f_{\mid S\setminus \{0\}}>0$. In particular, $f$ induces a closed immersion
$$\Spec K\times_k \mathbb{G}_{m} = \Spec K[\ZZ]\hookrightarrow \Spec \left(\bigoplus_{m\in \ZZ^N}K\cdot \chi^m\right) = T_K$$
of group $K$-schemes that extends to a zero preserving closed immersion $\mathbb{A}^1_K\hookrightarrow \ol{T}_K$ of monoid $K$-schemes.
\end{enumerate}
\end{theorem}
\begin{proof}
Since $T$ is a torus, we derive that
$$T_K = \Spec K \times_k \underbrace{\mathbb{G}_{m}\times_k \mathbb{G}_{m}\times_k ...\times_k \mathbb{G}_{m}}_{N\,\mathrm{times}} = \Spec \left(\bigoplus_{m\in \ZZ^N}K\cdot \chi^m\right)$$
and hence
$$\ol{T}_K = \Spec \left(\bigoplus_{s\in S}K\cdot \chi^s\right)$$
for some abstract submonoid $S$ of $\ZZ^N$. Moreover, the open immersion $T_K\hookrightarrow \ol{T}_K$ is induced by the inclusion $S \hookrightarrow \ZZ^N$. This proves \textbf{(1)}.\\
We have identification
$$k[T] = \bigoplus_{\lambda\in \bd{Irr}(T)}V_{\lambda}^{n_{\lambda}}$$
of $T$-representations, where $n_{\lambda}\in \NN\setminus \{0\}$ for each $\lambda$. Thus
$$\bigoplus_{m\in \ZZ^N}K\cdot \chi^m = K\otimes_kk[T] = \bigoplus_{\lambda\in \bd{Irr}(T)}\left(K\otimes_kV_{\lambda}\right)^{n_{\lambda}}$$
This implies that $n_{\lambda} = 1$ for every $\lambda$ and moreover, we derive that
$$K\otimes_kV_{\lambda} = \bigoplus_{m\in A_{\lambda}}K\cdot \chi^m$$
for some finite set $A_{\lambda}\subseteq \ZZ^N$. We also have $A_{\lambda_0} = \{0\}$ and $A_{\lambda}\subseteq S\setminus \{0\}$ for $\lambda \in \bd{Irr}(\ol{T})$. This proves \textbf{(2)}.\\
Since $\ol{T}$ admits a zero, we derive that
$$\ideal{m} = \bigoplus_{m \in S\setminus\{0\}}K\cdot \chi^s \subseteq \bigoplus_{m\in \ZZ^N}K\cdot \chi^m$$
is an ideal. This implies that $S\setminus \{0\}$ is closed under addition. In particular, there exists a homomorphism of abelian groups $f:\ZZ^N\ra \ZZ$ such that $f_{\mid S\setminus \{0\}}>0$. This implies \textbf{(3)}.
\end{proof}

\section{Kempf monoids}
\noindent
In this section we introduce important class of a monoid $k$-schemes, which contains all reductive monoids over $k$. We recall classical result concerning quotients with respect to actions of linearly reductive groups on affine algebraic schemes over $k$.

\begin{theorem}\label{theorem:affine_good_quotients_for_linearly_reductive_groups}[{\cite[Theorem 5.4 and discussion below its statement]{bialynicki2013algebraic}}]
Let $X$ be an affine $k$-scheme of finite type equipped with an action of a linearly reductive algebraic group $\bd{G}$. Consider the morphism $\pi:X\ra Y$ of affine $k$-schemes induced by the inclusion $\Gamma(X,\cO_X)^{\bd{G}}\hookrightarrow \Gamma(X,\cO_X)$. Then the following assertions hold.
\begin{enumerate}[label=\emph{\textbf{(\arabic*)}}, leftmargin=3.0em]
\item $Y$ is of finite type over $k$.
\item If $Z_1$ and $Z_2$ are disjoint, $\bd{G}$-stable and closed subschemes of $X$, then $\pi(Z_1)$ and $\pi(Z_2)$ are disjoint.
\item $\pi$ is surjective.
\item If we consider $Y$ as a $k$-scheme with trivial $\bd{G}$-action, then $\pi$ is $\bd{G}$-equivariant morphism.
\item If $p:X\ra W$ is a $\bd{G}$-equivariant morphism and $W$ is a $k$-scheme with trivial $\bd{G}$-action, then $p$ uniquely factors through $\pi$.
\end{enumerate}
\end{theorem}
\noindent
Now we are ready to prove the following result.

\begin{theorem}\label{theorem:reductive_monoids_are_Kempf}
Let $\bd{M}$ be a reductive algebraic monoid over $k$ and let $\bd{G}$ be a group of units of $\bd{M}$. Assume that $\bd{M}$ admits a zero $\bd{o}$. Then there exists a central torus $T$ in $\bd{G}$ such that $\bd{o}\in \bd{cl}(T)$.
\end{theorem}
\begin{proof}
By assumption $\bd{G}$ is a reductive group. According to {\cite[Corollary 17.62 and Notation 12.29]{milne2017algebraic}} its centre $Z(\bd{G})$ is an algebraic group of multiplicative type and the largest subtorus $T$ of $Z(\bd{G})$ is the solvable radical $R(\bd{G})$ of $\bd{G}$. In particular, the quotient group $\bd{G}/T$ has trivial solvable radical and hence it is a semisimple algebraic group. Now $T$ is linearly reductive {\cite[Corollary 10.4]{Group_schemes_over_field}}. Thus by Theorem \ref{theorem:affine_good_quotients_for_linearly_reductive_groups} we obtain a quotient $\pi:\bd{M}\twoheadrightarrow \bd{Q}$ of $\bd{M}$ by the action of $T$. Note also that $T$ is central in $\bd{M}$ as it is central in $\bd{G}$. Next the fact that $T$ is central in $\bd{M}$, the fact that $\bd{M}$ is geometrically integral and Theorem \ref{theorem:affine_good_quotients_for_linearly_reductive_groups} imply that $\bd{Q}$ is a geometrically integral, affine and algebraic monoid $k$-scheme with zero. Moreover, $\pi$ is a surjective morphism of algebraic monoids over $k$. According to Theorem \ref{theorem:units_are_open_in_a_geometrically_integral_monoid} we derive that the group of units $\bd{Q}^*$ is the open subscheme of $\bd{Q}$. From the fact that $\bd{G}\hookrightarrow \bd{M}$ is dominant we derive that the restriction $\pi_{\mid \bd{G}}:\bd{G}\ra \bd{Q}$ is dominant. Thus $\pi$ induces a dominant morphism of geometrically integral algebraic groups $\bd{G}\ra \bd{Q}^*$. Next {\cite[Theorem 5.3]{Group_schemes_over_field}} implies that $\pi(\bd{G}) = \bd{Q}^*$. Theorem \ref{theorem:units_are_open_in_an_affine_algebraic_monoid_and_it_is_a_closed_submonoid_of_a_matrix_monoid} implies that there exists a closed immersion of monoids $i:\bd{Q} \hookrightarrow \bd{L}(V)$ for some finite dimensional vector $k$-space $V$. Thus $i\cdot \pi_{\bd{G}}$ composed with the determinant $\mathrm{det}:\bd{L}(V) \ra \mathbb{G}_{m}$ is a character of $\bd{G}$ that factors through the quotient morphism $\bd{G}\twoheadrightarrow \bd{G}/T$, but $\bd{G}/T$ is a semisimple algebraic group and hence it has only trivial characters. Therefore, the character of $\bd{G}$ constructed above is trivial. Hence $i(\bd{Q}^*) = i\cdot \pi(\bd{G})$ is contained in the algebraic subgroup $\bd{SL}(V)$ of $\bd{L}(V)$. Next $i$ induces a morphism of algebraic groups $\bd{Q}^*\hookrightarrow \bd{SL}(V)$ and by {\cite[Theorem 5.3]{Group_schemes_over_field}} we infer that $i(\bd{Q}^*)$ is closed in $\bd{SL}(V)$. Since $\bd{SL}(V)$ is closed in $\bd{L}(V)$, we derive that $i(\bd{Q}^*)$ is closed in $\bd{L}(V)$ and hence it is also closed in $\bd{Q}$. On the other hand we proved that is open in $\bd{Q}$. $\bd{Q}$ is (geometrically) integral and hence it is connected. Thus $\bd{Q}^* = \bd{Q}$ which means that $\bd{Q}$ is a group $k$-scheme. Moreover, $\bd{Q}$ is a monoid $k$-scheme with zero. Thus is only possible if $\bd{Q}$ is $\Spec k$. Therefore, the categorical quotient $\pi:\bd{M}\ra \bd{Q}$ consists of a single $k$-rational point. Thus by \textbf{(2)} Theorem \ref{theorem:affine_good_quotients_for_linearly_reductive_groups} the closure of every orbit of $T$ in $\bd{M}$ contains the zero $\bd{o}$. In particular, $\bd{o}\in \bd{cl}(T)$.
\end{proof}
\noindent
This theorem motivates the following definition.

\begin{definition}
Let $\bd{M}$ be a geometrically integral, affine algebraic monoid over $k$. Assume that $\bd{M}$ admits a zero $\bd{o}$ and let $\bd{G}$ is a group of units of $\bd{M}$. Suppose that there exists a central subtorus $T$ of $\bd{G}$ such that its closure contains $\bd{o}$. Then we say that $\bd{M}$ is \textit{a Kempf monoid over $k$}.
\end{definition}
\noindent
Let us note for the future reference the following reformulation of Theorem \ref{theorem:reductive_monoids_are_Kempf}.

\begin{corollary}\label{corollary:reductive_monoids_are_Kempf}
Let $\bd{M}$ be a reductive monoid over $k$. Then $\bd{M}$ is a Kempf monoid.
\end{corollary}
\noindent
Now we give an example of a Kempf monoid which is not reductive.

\begin{example}[Kempf monoid with nonreductive group of units]\label{example:nonreductive_Kempf_monoid}
Let $n$ be a positive integer. Consider the algebraic group $\bd{B}_n$ of invertible upper triangular $n\times n$ matrices. Let $\ol{\bd{B}}_n$ be the closure of $\bd{B}_n$ in the algebraic monoid of all $n\times n$ matrices $\bd{M}_n$. Then $\ol{\bd{B}}_n$ is an affine, geometrically integral algebraic monoid over $k$ with zero (it contains zero matrix). Actually $\ol{\bd{B}}_n$ (or better to say its $k$-functor of points) consists of all upper triangular $n\times n$ matrices. The group of units of $\ol{\bd{B}}_n$ is $\bd{B}_n$ and hence it is solvable. Moreover, the center of $\bd{B}_n$ contains the one-dimensional split torus $\mathbb{G}_{m}$ consisting of scalar matrices. The closure of this torus in $\ol{\bd{B}}_n$ contains zero matrix and hence $\ol{\bd{B}}_n$ is the Kempf monoid.
\end{example}
\noindent
Let us discuss some properties of Kempf monoids. We first note the following.

\begin{proposition}\label{proposition:central_toric_submonoid_of_a_Kempf_monoid}
Let $\bd{M}$ be a Kempf monoid over $k$ and let $T$ be a central torus of $\bd{M}$ such that $T$ contains $\bd{o}$. Then the closure $\ol{T}$ of $T$ in $\bd{M}$ with reduced subscheme structure is a closed toric submonoid $k$-scheme of $\bd{M}$ containing zero.
\end{proposition}
\begin{proof}
The multiplication $\mu$ on $\bd{M}$ induces a morphism $\mu_{\mid \ol{T}\times_k\ol{T}}:\ol{T}\times_k\ol{T}\ra \bd{M}$. Since scheme-theoretic image of $\mu(T\times_kT)$ is contained in $\ol{T}$ and $T\times_kT$ is open and schematically dense in $\ol{T}\times_k\ol{T}$, we deduce that $\mu_{\mid \ol{T}\times_k\ol{T}}$ factors through closed subscheme $\ol{T}$. Thus $\mu$ restricts to a multiplication $\nu:\ol{T}\times_k\ol{T}\ra \ol{T}$ and hence $\ol{T}\hookrightarrow \bd{M}$ is closed immersion of monoid $k$-schemes. Clearly $\ol{T}$ is geometrically integral as a scheme-theoretic closure of a geometrically integral scheme $T$. The fact that the zero $\bd{o}$ of $\bd{M}$ is contained in $\ol{T}$ follows by definition.
\end{proof}

\begin{corollary}\label{corollary:Kempf_one_parameter_subgroup}
Let $\bd{M}$ be a Kempf monoid over $k$. Fix an algebraically closed field $K$ over $k$. Then there exists a closed immersion
$$i:\mathbb{A}^1_K \hookrightarrow \Spec K\times_k\bd{M}$$
of monoid $K$-schemes sending the zero of $\mathbb{A}^1_K$ to the zero of $\bd{M}_K = \Spec K\times_k\bd{M}$.
\end{corollary}
\begin{proof}
This follows from Proposition \ref{proposition:central_toric_submonoid_of_a_Kempf_monoid} and \textbf{(3)} Theorem \ref{theorem:toric_monoids_properties_Kempf_torus}.
\end{proof}

\begin{theorem}\label{theorem:if_locally_closed_group_stable_subscheme_of_kempf_monoid_contains_infinitesimal_neighborhood_of_zero_then_it_is_whole_monoid}
Let $\bd{M}$ be a Kempf monoid over $k$ with group $\bd{G}$ of units and let $j:Z\hookrightarrow \bd{M}$ be a locally closed $\bd{G}$-stable subscheme of $\bd{M}$. Then the following are equivalent.
\begin{enumerate}[label=\textbf{\emph{(\roman*)}}, leftmargin=3.0em]
\item For every $n\in \NN$ the $n$-th infinitesimal neighborhood $\bd{M}_n$ of $\bd{o}$ in $\bd{M}$ is contained in $Z$.
\item $j$ is an isomorphism.
\end{enumerate}
\end{theorem}
\noindent
We first consider the following special case.

\begin{lemma}\label{lemma:if_open_group_stable_subscheme_of_Kempf_monoid_contains_zero_then_it_is_whole_monoid}
Let $U$ be an open $\bd{G}$-stable subscheme of $\bd{M}$. If $\bd{o}$ is a point of $U$, then $U = \bd{M}$.
\end{lemma}
\begin{proof}[Proof of the lemma]
Fix $i:\mathbb{A}^1_K \hookrightarrow \Spec K\times_k\bd{M}$ as in Corollary \ref{corollary:Kempf_one_parameter_subgroup}. Denote
$$\Spec K\times_k \bd{M},\,\Spec K\times_k \bd{G},\,\Spec K\times_kU$$
by $\bd{M}_K,\,\bd{G}_K,\,U_K$, respectively. Note that $i\left(\mathbb{G}_{m,K}\right)\subseteq \bd{G}_K$. Fix a field $L$ over $K$ and a morphism $j:\Spec L \hookrightarrow \bd{M}_K$. Next consider the composition
\begin{center}
\begin{tikzpicture}
[description/.style={fill=white,inner sep=2pt}]
\matrix (m) [matrix of math nodes, row sep=4em, column sep=3em,text height=1.5ex, text depth=0.25ex] 
{  \mathbb{A}^1_{L} = \mathbb{A}^1_K\times_K\Spec L       &  \bd{M}_K\times_k \bd{M}_K & \bd{M}_K    \\} ;
\path[right hook->,line width=1.0pt,font=\scriptsize]
(m-1-1) edge node[above] {$ i \times_K j   $} (m-1-2);
\path[right hook->,line width=1.0pt,font=\scriptsize]
(m-1-2) edge node[below] {$ \mu_K  $} (m-1-3);
\path[->,bend left,line width=1.0pt,font=\scriptsize]
(m-1-1) edge node[above] {$ f  $} (m-1-3);
\end{tikzpicture}
\end{center}
where the second morphism $\mu_K:\bd{M}_K\times_k \bd{M}_K\ra \bd{M}_K$ is the multiplication. Clearly $f$ is $\mathbb{G}_{m,L}$-equivariant. Hence $f^{-1}(U_K)$ is an open $\mathbb{G}_{m,L}$-stable subscheme of $\mathbb{A}^1_{L}$. It contains the zero of $\mathbb{A}^1_L$ because $\bd{o}_K \in U_K$ by assumption. Since the only open $\mathbb{G}_{m,L}$-stable subscheme of $\mathbb{A}^1_{L}$ containing the zero is $\mathbb{A}^1_{L}$, we derive that $f^{-1}(U_K) = \mathbb{A}^1_{L}$. Thus the image of $j$ is in $U_K$. Hence $U_K = \bd{M}_K$ because $j:\Spec L\ra \bd{M}_K$ and $L$ are arbitrary. By faithfuly flat descent, we derive that $U = \bd{M}$.
\end{proof}

\begin{proof}[Proof of the theorem]
Assume that \textbf{(i)} holds. Since $\bd{o}$ is a point in $Z$, we have a surjective morphism $j^{\#}:\cO_{\bd{M},\bd{o}}\twoheadrightarrow \cO_{Z,\bd{o}}$ of local rings. Both schemes $Z,\bd{M}$ are noetherian and hence we have a commutative square
\begin{center}
\begin{tikzpicture}
[description/.style={fill=white,inner sep=2pt}]
\matrix (m) [matrix of math nodes, row sep=3em, column sep=3em,text height=1.5ex, text depth=0.25ex] 
{ \widehat{\cO_{\bd{M},\bd{o}}}   & \widehat{\cO_{Z,\bd{o}}}  \\
  \cO_{\bd{M},\bd{o}}             & \cO_{Z,\bd{o}}            \\} ;
\path[->,line width=0.8pt,font=\scriptsize]
(m-1-1) edge node[above] {$ \widehat{j^{\#}}$} (m-1-2);
\path[->>,line width=0.8pt,font=\scriptsize]
(m-2-1) edge node[below] {$ j^{\#} $} (m-2-2);
\path[right hook->,line width=0.8pt,font=\scriptsize]
(m-2-1) edge node[left] {$ $} (m-1-1)
(m-2-2) edge node[right] {$ $} (m-1-2);
\end{tikzpicture}
\end{center}
where vertical morphisms are injective. Since $\bd{M}_n\subseteq Z$ for every $n\in \NN$, we derive that $\widehat{j^{\#}}$ is an isomorphism. Hence $j^{\#}$ is injective and thus it is an isomorphism. This implies that there exists an open neighborhood $V$ of $\bd{o}$ in $\bd{M}$ such that $V\subseteq Z$. Let $\bd{G}\cdot V$ be the open subscheme of $\bd{M}$ defined as the image of $\bd{G}\times_kV$ under the left action $\bd{G}\times_k\bd{M}\ra \bd{M}$. This is $\bd{G}$-stable open subscheme of $\bd{M}$. From the fact that $j$ is $\bd{G}$-equivariant, we deduce that $\bd{G}\cdot V\subseteq Z$. By Lemma \ref{lemma:if_open_group_stable_subscheme_of_Kempf_monoid_contains_zero_then_it_is_whole_monoid} we infer that $\bd{G}\cdot V = \bd{M}$ because $\bd{o}\in V\subseteq \bd{G}\cdot V$. This shows that $Z = \bd{M}$. Thus we have $\textbf{(i)}\Rightarrow \textbf{(ii)}$.\\
The implication $\textbf{(ii)}\Rightarrow \textbf{(i)}$ is obvious.
\end{proof}

































\small
\bibliographystyle{apalike}
\bibliography{../zzz}

\end{document}
