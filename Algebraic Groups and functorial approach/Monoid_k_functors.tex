\input ../pree.tex

\begin{document}

\title{Monoid $k$-functors}
\date{}
\maketitle


\section{Algebraic structures in the category of $k$-functors}
\noindent
In the sequel we assume that the reader is familiar with notions of a monoid, group etc. in arbitrary category with finite products. For definitions and some discussion related to these notions cf. {\cite[pages 2-5]{Maclane}}.

\begin{definition}
\textit{A monoid (group, abelian group, ring) $k$-functor} is a monoid (group, abelian group, ring) object in the category of $k$-functors.
\end{definition}

\begin{example}\label{example:endomorphisms_of_k_functor}
Let $\fX$ be a $k$-functor such that $\iMor_k(\fX,\fX)$ exists. Then $\iMor_k(\fX,\fX)$ is a monoid $k$-functor with respect to composition of morphisms.
\end{example}

\begin{example}\label{example:constant_ring_k_functor}
Basic example of a ring $k$-functor is a $k$-functor $\ideal{K}$ given by
$$\fK(A) = k,\,\fK(f) = 1_k$$
for any $k$-algebra $A$ and morphism $f$ of $k$-algebras. It can be described as a constant $k$-functor ({\cite[page 67]{Maclane}}) corresponding to $k$.
\end{example}

\begin{definition}
Let $\fA$ be a commutative ring $k$-functor. \textit{An $\fA$-algebra} is an $\fA$-algebra object in the category of $k$-functors.
\end{definition}

\begin{definition}
Let $\fR$ be a ring $k$-functor. Suppose that $\fM$ is an abelian group $k$-functor and there exists a morphism $\fR \times \fM\ra \fM$ of $k$-functors that for each $k$-algebra $A$ makes $\fM(A)$ into an $\fR(A)$-module. Then we say that $\fM$ is \textit{a module $k$-functor over $\fR$}.
\end{definition}

\begin{definition}
Let $\fR$ be an ring $k$-functor and let $\fM_1,\fM_2$ be module $k$-functors over $\fR$. Suppose that $\sigma:\fM_1\ra \fM_2$ is a morphism of abelian group $k$-functors such that the diagram
\begin{center}
\begin{tikzpicture}
[description/.style={fill=white,inner sep=2pt}]
\matrix (m) [matrix of math nodes, row sep=3em, column sep=5em,text height=1.5ex, text depth=0.25ex] 
{  \fR \times \fM_1   & \fR \times \fM_2  \\
   \fM_1              & \fM_2            \\} ;
\path[->,line width=1.0pt,font=\scriptsize]  
(m-1-1) edge node[above] {$ 1_{\fR} \times \sigma  $} (m-1-2)
(m-2-1) edge node[below] {$ \sigma $} (m-2-2)
(m-1-1) edge node[left] {$ \alpha_1 $} (m-2-1)
(m-1-2) edge node[right] {$ \alpha_2 $} (m-2-2);
\end{tikzpicture}
\end{center}
is commutative, where $\alpha_i:\fR\times \fM_i\ra \fM_i$ define $\fR$-module structure on $\fM_i$ for $i=1,2$. Then $\sigma$ is \textit{a morphism of modules over $\fR$}.
\end{definition}
\noindent
Let $\fM_1$ and $\fM_2$ be module $k$-functors over $\fR$. We denote by
$$\Hom_{\fR}\left(\fM_1,\fM_2\right)$$
as a class of all morphisms of modules $\fM_1\ra \fM_2$ over $\fR_A$.

\begin{definition}
Let $\fM_1$ and $\fM_2$ be module $k$-functors over $\fR$. Assume that $\Hom_{\fR_A}\left((\fM_1)_A,(\fM_2)_A\right)$ is a set for every $k$-algebra $A$. Then we define a $k$-subfunctor $\shHom_{\fR}(\fM_1,\fM_2)$ of internal hom of $\fM_1$ and $\fM_2$ by formula
$$\Alg_k\ni A \mapsto \Hom_{\fR_A}\left((\fM_1)_A,(\fM_2)_A\right) \in \Set$$
We call $\shHom_{\fR}(\fM_1,\fM_2)$ \textit{a $k$-functor of module morphisms of $\fM_1$ and $\fM_2$}.
\end{definition}
\noindent
If $\fM$ is a module $k$-functor over some ring $k$-functor $\fR$, then we denote (if it exists) $\shHom_{\fR}(\fM,\fM)$ by $\cE nd_{\fR}(\fM)$.

\begin{example}\label{example:endomorphisms_of_module_k_functor}
Let $\fM$ be a module over a ring $k$-functor $\fR$. Assume that $\cE nd_{\fR}(\fM)$ exists. Then $\cE nd_{\fR}(\fM)$ is a ring $k$-functor with respect to composition of morphisms of modules as the multiplication and canonically defined addition of module morphisms.\\
If $\fR$ is a commutative ring $k$-functor, then $\cE nd_{\fR}(\fM)$ admits additional strucutre of a $\fR$-algebra $k$-functor induced via a unique morphism $\fR\ra \cE nd_{\fR}(\fM)$ of ring $k$-functors that sends $1\mapsto 1_{\fM}$.
\end{example}

\section{Global regular functions on a $k$-functor}
\noindent
Recall the ring $k$-functor $\fK$ from Example \ref{example:constant_ring_k_functor}. Note that a $\fK$-algebra $\fA$ can be viewed as a functor $\fA:\Alg_k\ra \Alg_k$.

\begin{definition}
The $\fK$-algebra $\fO_k$ represented by the identity functor on $\Alg_k$ is called \textit{the structure $\fK$-algebra}.
\end{definition}
\noindent
Let $|-|:\Alg_k\ra \Set$ be the forgetful $k$-functor. Note that $|-|$ is the underlying $k$-functor of $\fK$-algebra $\fO_k$. Recall that the affine line $\mathbb{A}^1_k$ is an affine $k$-scheme having $k$-algebra of polynomials with one variable as a $k$-algebra of regular functions.

\begin{fact}\label{fact:affineline_as_forgetfulfunctor}
Let $|-|:\Alg_k\ra \Set$ be the forgetful $k$-functor. Then we have natural isomorphism
$$\fP_{\mathbb{A}^1_k} \cong |-|$$
\end{fact}
\begin{proof}
Let $B$ be a $k$-algebra. We have the following chain of identifications
$$\fP_{\mathbb{A}^1_k}(B) = \Mor_k\left(\Spec B, \mathbb{A}^1_k\right) = \Mor_k\left(\Spec B, \Spec k[x]\right) = \Mor_k\left(k[x], B\right) = |B|$$
natural in $B$.
\end{proof}
\noindent
In particular, since $|-|$ carries the structure $\fK$-algebra $\fO_k$, we derive that $\fP_{\mathbb{A}^1_k}$ admits a structure of $\fK$-algebra isomorphic to $\fO_k$.\\
No we introduce regular functions on a $k$-functors.

\begin{definition}
Let $\fX$ be a $k$-functor and assume that $\Mor_k\left(\fX, \fO_k\right)$ is a set. Then $\Mor_k\left(\fX, \fO_k\right)$ is a $k$-algebra with respect to the structure induced by $\fO_k$. We call this $k$-algebra \textit{the $k$-algebra of global regular functions on $\fX$}. Its elements are called \textit{global regular functions on $\fX$}.
\end{definition}

\begin{definition}
Let $\fX$ be a $k$-functor. Suppose that $A$ is a $k$-algebra, $x\in \fX(A)$ and $f\in \Mor_k\left(\fX,\fO_k\right)$. The element $f(x) \in A$ is called \textit{the value of $f$ on point $x$}.
\end{definition}
\noindent
For given $k$-functor $\fX$ we describe $k$-algebra operations on $\Mor_k\left(\fX,\fO_k\right)$ in terms of values of its elements on points of $\fX$. For this consider $\alpha \in k$ and $f$, $g\in \Mor_k\left(\fX, \fO_k\right)$. We have formulas
$$\left(f+g\right)(x) = f(x)+g(x),\,\left(f\cdot g\right)(x) = f(x)\cdot g(x),\,\left(\alpha \cdot f\right)(x) = \alpha \cdot f(x)$$
in which right hand side are $k$-algebra operations in $A$.

\begin{example}\label{example:regular_functions_as_an_algebra_over_structure_algebra}
Let $\fX$ be a $k$-functor and assume that $\iMor_k(\fX,\fO_k)$ exists. Fix $k$-algebra $A$. Note that $\Mor_A(\fX_A,\fO_A)$ is an $A$-algebra of global regular functions on $\fX_A$. Moreover, if $B$ is an $A$-algebra, then
$$\Mor_A(\fX_A,\fO_A) \ni f \mapsto f_B\in \Mor_B(\fX_B,\fO_B)$$
is a morphism of $A$-algebras. This implies that $\iMor_k(\fX,\fO_k)$ admits a canonical structure of an $\fO_k$-algebra $k$-functor.
\end{example}

\section{Actions of monoid $k$-functors}\label{section:actions_of_monoid_k_functors}
\noindent
In the sequel we assume that the reader is familiar with notion of an action of a monoid object in arbitrary category with finite products. For definitions and some discussion related to these notions cf. {\cite[pages 5]{Maclane}}.\\
Let $\fG$ be a monoid $k$-functor and $\fX$ be a $k$-functor together with an action $\alpha:\fG\times \fX \ra \fX$. Next assume that $k$-functor $\iMor_k(\fX,\fX)$ exists. By Example \ref{example:endomorphisms_of_k_functor} it is a monoid $k$-functor. We define a morphism $\rho:\fG\ra \iMor_k(\fX,\fX)$ of $k$-functors by formula $\rho(x) = \alpha_x$. Note that by discussion preceding {\cite[Theorem 2.7]{kfunctors}} and by {\cite[Corollary 2.9]{kfunctors}}, we deduce that $\rho$ is a well defined morphism of $k$-functors. We show now that $\rho$ is a morphism of monoids. For this pick $k$-algebra $A$ and $x, y\in \fG(A)$. Since $\alpha$ is an action, we deduce that $\alpha_{x \cdot y} = \alpha_x \cdot \alpha_y$ and hence also
$$\rho(x\cdot y) = \alpha_{x \cdot y} = \alpha_x\cdot \alpha_y = \rho(x)\cdot \rho(y)$$
Therefore, $\rho$ is a morphism of monoid $k$-functors. This shows how to construct a morphism of monoid $k$-functors $\rho$ from an action $\alpha$ of $\fG$.

\begin{theorem}\label{theorem:actions_and_monoid_morphisms}
Let $\fG$ be a monoid $k$-functor and let $\fX$ be a $k$-functor such that $\cI so_{k}(\fX,\fX)$ exists. Suppose that
\begin{center}
\begin{tikzpicture}
[description/.style={fill=white,inner sep=2pt}]
\matrix (m) [matrix of math nodes, row sep=3em, column sep=4em,text height=1.5ex, text depth=0.25ex] 
{\bigg\{\mbox{actions of $\fG$ on $\fX$}\bigg\} & \bigg\{\mbox{Morphisms $\rho:\fG\ra \iMor_{k}(\fX,\fX)$ of monoid $k$-functors}\bigg\} \\};
\path[->,line width=1.0pt,font=\scriptsize]  
(m-1-1) edge node[auto] {$ $} (m-1-2);
\end{tikzpicture}
\end{center}
is a map of classes described above. Then it is bijection.
\end{theorem}
\begin{proof}
Our goal is to construct the inverse of the map. Recall {\cite[Theorem 2.7]{kfunctors}} and substitute in that Theorem $\fJ = \iMor_k(\fX,\fX)$. Consider maps
$$\Phi:\bigg\{\mbox{families }\fG\times \fX\ra \fX\mbox{ of morphisms}\bigg\} \ra  \Mor_{k}\left(\fG,\iMor_k(\fX,\fX)\right)$$
and
$$\Psi:\Mor_{k}\left(\fG, \iMor_k(\fX,\fX)\right) \ra \bigg\{\mbox{families }\fG\times \fX\ra \fX\mbox{ of morphisms}\bigg\}$$
in that Theorem. Then the map in the statement above is the restriction of $\Phi$ to $\fG$-actions on $\fX$ on the right and morphisms $\fG\ra \iMor_k(\fX,\fX)$ of monoid $k$-functors on the left. Since by {\cite[Theorem 2.7]{kfunctors}} maps $\Phi$ and $\Psi$ are mutually inverse, it suffices to check that $\Psi$ sends a morphism $\rho:\fG\ra \iMor_{k}(\fX,\fX)$ of monoids to an action of $\fG$ on $\fX$. For this denote $\Psi(\rho)$ by $\alpha$. Consider $k$-algebra $A$ and $A$-points $x,y\in \fG(A),\,z\in \fX(A)$. Then
$$\alpha\left(y, \alpha(x, z)\right) = \rho(y)\left(\rho(x)(z)\right) = \left(\rho(y)\cdot \rho(x)\right)(z) = \rho\left(x\cdot y\right)(z) = \alpha\left(x\cdot y, z\right)$$
Therefore, $\alpha$ is an action of $\fG$ on $\fX$.
\end{proof}

\begin{proposition}\label{proposition:morphism_of_monoid_actions}
Let $\fG$ be a monoid $k$-functor and let $\fX_1$, $\fX_2$ be $k$-functors such that $\iMor_k(\fX_1,\fX_1),\iMor_k(\fX_2,\fX_2)$ exist. Suppose that $\alpha_1:\fG\times \fX_1 \ra \fX_1,\,\alpha_2:\fG\times \fX_2 \ra \fX_2$ are actions of $\fG$, respectively. Suppose that $\sigma:\fX_1\ra \fX_2$ is a morphism of $k$-functors. Then the following assertions are equivalent.
\begin{enumerate}[label=\emph{\textbf{(\roman*)}}, leftmargin=1.5em]
\item The square
\begin{center}
\begin{tikzpicture}
[description/.style={fill=white,inner sep=2pt}]
\matrix (m) [matrix of math nodes, row sep=3em, column sep=5em,text height=1.5ex, text depth=0.25ex] 
{  \fG\times \fX_1   & \fG\times \fX_2 \\
   \fX_1             & \fX_2 \\} ;
\path[->,line width=1.0pt,font=\scriptsize]  
(m-1-1) edge node[above] {$ 1_{\fG}\times \sigma  $} (m-1-2)
(m-2-1) edge node[below] {$ \sigma $} (m-2-2)
(m-1-1) edge node[left] {$ \alpha_1 $} (m-2-1)
(m-1-2) edge node[right] {$ \alpha_2 $} (m-2-2);
\end{tikzpicture}
\end{center}
is commutative.
\item For every $k$-algebra $A$ and $x\in \fG(A)$ we have
$$\sigma_A \cdot \rho_1(x) = \rho_2(x) \cdot \sigma_A$$
where $\rho_1:\fG\ra \iMor_k(\fX_1,\fX_1)$ and $\rho_2:\fG\ra \iMor_{k}(\fX_2,\fX_2)$ are morphism of monoid $k$-functors corresponding to $\alpha_1$ and $\alpha_2$, respectively.
\end{enumerate}
\end{proposition}
\begin{proof}
Conditions expressed in \textbf{(i)} and \textbf{(ii)} are directly translatable to each other by virtue of the bijection in Theorem \ref{theorem:actions_and_monoid_morphisms}. 
\end{proof}

\begin{definition}
Let $\fG$ be a monoid $k$-functor and let $(\fX_1,\alpha_1)$, $(\fX_2,\alpha_2)$ be $k$-functors with actions of $\fG$. Suppose that $\sigma:\fX_1\ra \fX_2$ is a morphism $k$-functors such that the square
\begin{center}
\begin{tikzpicture}
[description/.style={fill=white,inner sep=2pt}]
\matrix (m) [matrix of math nodes, row sep=3em, column sep=5em,text height=1.5ex, text depth=0.25ex] 
{  \fG\times \fX_1   & \fG\times \fX_2 \\
   \fX_1             & \fX_2 \\} ;
\path[->,line width=1.0pt,font=\scriptsize]  
(m-1-1) edge node[above] {$ 1_{\fG}\times \sigma  $} (m-1-2)
(m-2-1) edge node[below] {$ \sigma $} (m-2-2)
(m-1-1) edge node[left] {$ \alpha_1 $} (m-2-1)
(m-1-2) edge node[right] {$ \alpha_2 $} (m-2-2);
\end{tikzpicture}
\end{center}
is commutative. Then $\sigma$ is called \textit{an $\fG$-equivariant morphism}.
\end{definition}

\section{Modules over ring $k$-functor}
\noindent
Let $\fA$ be a commutative ring $k$-functor and let $\fR$ be a $\fA$-algebra $k$-functor. This means that there exists a morphism $\fA\ra \fR$ of ring $k$-functors and for every $k$-algebra $A$ induced morphism $\fA(A)\ra \fR(A)$ sends $\fA(A)$ to the center of a ring $\fR(A)$. Fix a module $\fM$ over $\fA$. Next assume that $k$-functor $\cE nd_{\fA}(\fM)$ exists. Recall that by Example \ref{example:endomorphisms_of_module_k_functor} it is a ring $k$-functor.

\begin{definition}
In the setting above suppose that $\alpha:\fR\times \fM\ra \fM$ is a morphism of $k$-functors. Suppose that $\alpha$ makes $\fM$ into $\fR$-module and moreover, for every $k$-algebra $A$ and for every point $x\in \fR(A)$ morphism $\alpha_x$ is a morphism of $\fA_A$-modules. Then $\alpha$ is called \textit{a $\fA$-linear $\fR$-action on $\fM$}.
\end{definition}
\noindent
We continue the discussion. We assume that we are given an $\fA$-linear $\fR$-action $\alpha:\fR\times \fM \ra \fM$ on $\fM$. We define a morphism $\rho:\fR\ra \cE nd_{\fA}(\fM)$ of $k$-functors by formula $\rho(x) = \alpha_x$. As in Section \ref{section:actions_of_monoid_k_functors} we can prove that $\rho$ is a morphism of ring $k$-functors. Now we have the following result.

\begin{theorem}\label{theorem:linear_morphisms_and_homomorphisms_of_rings}
Let $\fR$ be an algebra $k$-functor over commutative ring $\fA$ $k$-functor and let $\fM$ be a $\fA$-module such that $\cE nd_{\fA}(\fM)$ exists. Suppose that
\begin{center}
\begin{tikzpicture}
[description/.style={fill=white,inner sep=2pt}]
\matrix (m) [matrix of math nodes, row sep=3em, column sep=4em,text height=1.5ex, text depth=0.25ex] 
{\bigg\{\mbox{$\fA$ linear actions of $\fR$ on $\fM$}\bigg\} & \bigg\{\mbox{Morphisms $\rho:\fR\ra \cE nd_{\fO_k}(\fM)$ of ring $k$-functors}\bigg\} \\};
\path[->,line width=1.0pt,font=\scriptsize]  
(m-1-1) edge node[auto] {$ $} (m-1-2);
\end{tikzpicture}
\end{center}
is a map of classes described above. Then it is bijection.
\end{theorem}
\begin{proof}
The proof is similar to the proof of Theorem \ref{theorem:actions_and_monoid_morphisms}.
\end{proof}


\section{Monoid algebra $\fO_k[\fG]$ and its modules}

\begin{definition}
Let $\fG$ be a monoid $k$-functor. Then we construct an $\fO_k$-algebra $\fO_k[\fG]$ as follows. For every $k$-algebra $A$ we define
$$\fO_k[\fG](A) = A\big[\fG(A)\big]$$
where the right hand side is monoid $A$-algebra for the abstract monoid $\fG(A)$. The structure of monoid $k$-functor on $\fG$ and $\fK$-algebra $\fO_k$ makes $\fO_k[\fG]$ into a ring $k$-functor. Moreover, we have a morphism $\fO_k\ra \fO_k[\fG]$ which for every $k$-algebra $A$ is given by the canonical inclusion
$$A \hookrightarrow A\big[\fG(A)\big]$$
Thus $\fO_k[\fG]$ is $\fO_k$-algebra. We call $\fO_k[\fG]$ \textit{a monoid $\fO_k$-algebra over $\fG$}.
\end{definition}

\begin{fact}\label{fact:universal_property_of_monoid_algebra}
Let $\fG$ be a monoid $k$-functor and let $\fR$ be an $\fO_k$-algebra $k$-functor. Consider $\fR$ as a monoid $k$-functor with respect to its multiplicative structure. Then every morphism
$$\sigma:\fG\ra \fR$$
of monoid $k$-functors admits a unique extension
$$\tilde{\sigma}:\fO_k[\fG]\ra \fR$$
to a morphism of $\fO_k$-algebras.
\end{fact}
\begin{proof}
This follows from the analogical universal property of algebras over abstract monoids (monoid algebras in $\Set$).
\end{proof}

\begin{definition}
Let $\fG$ be a monoid $k$-functor and let $\fM$ be a module over $\fO_k$. Suppose that $\alpha:\fG\times \fM\ra \fM$ is an action of $\fG$ such that for any $k$-algebra $A$ and point $x\in \fG(A)$ morphism $\alpha_x:\fM_A\ra \fM_A$ is a morphism of $\fO_A$-modules. Then $\alpha$ is called \textit{a linear $\fG$-action on $\fM$}.
\end{definition}
\noindent
Suppose now that $\fG$ is a monoid $k$-functor and $\fM$ is a module $\fO_k$. Note that every linear $\fG$-action $\alpha:\fG\times \fM \ra \fM$ extends uniquely to a $\fO_k$-linear action $\fO_k[\fG]\times \fM\ra \fM$ of monoid $\fO_k$-algebra. This gives a bijection
\begin{center}
\begin{tikzpicture}
[description/.style={fill=white,inner sep=2pt}]
\matrix (m) [matrix of math nodes, row sep=3em, column sep=4em,text height=1.5ex, text depth=0.25ex] 
{\bigg\{\mbox{Linear actions of $\fG$ on $\fM$}\bigg\} & \bigg\{\mbox{$\fO_k$-linear actions $\fO_k[\fG]\times \fM\ra \fM$}\bigg\} \\};
\path[->,line width=1.0pt,font=\scriptsize]  
(m-1-1) edge node[auto] {$ $} (m-1-2);
\end{tikzpicture}
\end{center}
Next assume that $k$-functor $\cE nd_{\fO_k}(\fM)$ exists. By Example \ref{example:endomorphisms_of_module_k_functor} it is an $\fO_k$-algebra $k$-functor. Next by Theorem \ref{theorem:linear_morphisms_and_homomorphisms_of_rings} we have a bijection
\begin{center}
\begin{tikzpicture}
[description/.style={fill=white,inner sep=2pt}]
\matrix (m) [matrix of math nodes, row sep=3em, column sep=4em,text height=1.5ex, text depth=0.25ex] 
{\bigg\{\mbox{$\fO_k$-linear actions of $\fO_k[\fG]\times \fM\ra \fM$}\bigg\} & \bigg\{\mbox{Morphisms $\fO_k[\fG]\ra \cE nd_{\fO_k}(\fM)$ of $\fO_k$-algebras}\bigg\} \\};
\path[->,line width=1.0pt,font=\scriptsize]  
(m-1-1) edge node[auto] {$ $} (m-1-2);
\end{tikzpicture}
\end{center}
Finally Fact \ref{fact:universal_property_of_monoid_algebra} implies that we have a bijection
\begin{center}
\begin{tikzpicture}
[description/.style={fill=white,inner sep=2pt}]
\matrix (m) [matrix of math nodes, row sep=3em, column sep=4em,text height=1.5ex, text depth=0.25ex] 
{\bigg\{\mbox{Morphisms $\fO_k[\fG]\ra \cE nd_{\fO_k}(\fM)$ of $\fO_k$-algebras} \bigg\} & \bigg\{\mbox{Morphisms $\fG\ra \cE nd_{\fO_k}(\fM)$ of monoids}\bigg\} \\};
\path[->,line width=1.0pt,font=\scriptsize]  
(m-1-1) edge node[auto] {$ $} (m-1-2);
\end{tikzpicture}
\end{center}
This chain of bijections sends a linear action $\alpha:\fG\times \fM\ra \fM$ of $\fG$ to a morphism $\rho:\fG\ra \cE nd_{\fO_k}(\fM)$ of monoid $k$-functors given by $\rho(x) = \alpha_x$ for every $x\in \fG(A)$ and every $k$-algebra $A$. We proved the following result.

\begin{proposition}\label{proposition:decription_of_linear_monoid_actions_on_modules}
Let $\fG$ be a monoid $k$-functor and let $\fM$ be a $\fO_k$-module such that $\cE nd_{\fO_k}(\fM)$ exists. Then the following classes are in canonical bijections described above.
\begin{enumerate}[label=\emph{\textbf{(\arabic*)}}, leftmargin=1.5em]
\item Linear actions of $\fG$ on $\fM$.
\item $\fO_k$-linear actions $\fO_k[\fG]\times \fM\ra \fM$. These are precisely $\fO_k[\fG]$-modules.
\item Morphisms $\fO_k[\fG]\ra \cE nd_{\fO_k}(\fM)$ of $\fO_k$-algebras.
\item Morphisms $\fG\ra \cE nd_{\fO_k}(\fM)$ of monoids.
\end{enumerate}
Moreover, the bijection between class \emph{\textbf{(1)}} and \emph{\textbf{(2)}} does not require the existence of $\cE nd_{\fO_k}(\fM)$.
\end{proposition}
\noindent
Now in a similar manner we can describe morphisms.

\begin{proposition}\label{proposition:morphism_of_monoid_modules}
Let $\fG$ be a monoid $k$-functor and let $\fM_1$, $\fM_2$ be $k$-functors of $\fO_k$-modules such that $\cE nd_{\fO_k}(\fM_1),\cE nd_{\fO_k}(\fM_2)$ exist. Suppose that $\alpha_1:\fG\times \fM_1 \ra \fM_1,\,\alpha_2:\fG\times \fM_2 \ra \fM_2$ are linear actions of $\fG$, respectively. Suppose that $\sigma:\fM_1\ra \fM_2$ is a morphism of modules over $\fO_k$. Then the following assertions are equivalent.
\begin{enumerate}[label=\emph{\textbf{(\roman*)}}, leftmargin=1.5em]
\item The square
\begin{center}
\begin{tikzpicture}
[description/.style={fill=white,inner sep=2pt}]
\matrix (m) [matrix of math nodes, row sep=3em, column sep=5em,text height=1.5ex, text depth=0.25ex] 
{  \fG\times \fM_1   & \fG\times \fM_2 \\
   \fM_1             & \fM_2 \\} ;
\path[->,line width=1.0pt,font=\scriptsize]  
(m-1-1) edge node[above] {$ 1_{\fG}\times \sigma  $} (m-1-2)
(m-2-1) edge node[below] {$ \sigma $} (m-2-2)
(m-1-1) edge node[left] {$ \alpha_1 $} (m-2-1)
(m-1-2) edge node[right] {$ \alpha_2 $} (m-2-2);
\end{tikzpicture}
\end{center}
is commutative.
\item The square
\begin{center}
\begin{tikzpicture}
[description/.style={fill=white,inner sep=2pt}]
\matrix (m) [matrix of math nodes, row sep=3em, column sep=5em,text height=1.5ex, text depth=0.25ex] 
{  \fO_k[\fG] \times \fM_1   & \fO_k[\fG] \times \fM_2 \\
   \fM_1                     & \fM_2 \\} ;
\path[->,line width=1.0pt,font=\scriptsize]  
(m-1-1) edge node[above] {$ 1_{\fO_k[\fG]}\times \sigma  $} (m-1-2)
(m-2-1) edge node[below] {$ \sigma $} (m-2-2)
(m-1-1) edge node[left] {$ \tilde{\alpha_1} $} (m-2-1)
(m-1-2) edge node[right] {$ \tilde{\alpha_2} $} (m-2-2);
\end{tikzpicture}
\end{center}
is commutative, where $\tilde{\alpha_1}$ and $\tilde{\alpha_2}$ are $\fO_k$-linear actions of $\fO_k[\fG]$ corresponding to $\alpha_1$ and $\alpha_2$, respectively. This states that $\sigma$ is a morphism of $\fO_k[\fG]$-modules.
\item For every $k$-algebra $A$ and $x\in \fG(A)$ we have
$$\sigma_A \cdot \tilde{\rho}_1(x) = \tilde{\rho}_2(x) \cdot \sigma_A$$
where $\tilde{\rho}_1:\fO_k[\fG] \ra \cE nd_{\fO_k}(\fM_1)$ and $\tilde{\rho}_2:\fO_k[\fG] \ra \cE nd_{\fO_k}(\fM_2)$ are morphism of $\fO_k$-algebras corresponding to $\tilde{\alpha_1}$ and $\tilde{\alpha_2}$, respectively.
\item For every $k$-algebra $A$ and $x\in \fG(A)$ we have
$$\sigma_A \cdot \rho_1(x) = \rho_2(x) \cdot \sigma_A$$
where $\rho_1:\fO_k[\fG] \ra \cE nd_{\fO_k}(\fM_1)$ and $\rho_2:\fO_k[\fG] \ra \cE nd_{\fO_k}(\fM_2)$ are morphism of monoid $k$-functors restricting $\tilde{\rho_1}$ and $\tilde{\rho_2}$, respectively.
\end{enumerate}
The equivalence of \emph{\textbf{(1)}} and \emph{\textbf{(2)}} does not require the existence of $\cE nd_{\fO_k}(\fM_1)$ and $\cE nd_{\fO_k}(\fM_2)$.
\end{proposition}
\begin{proof}
Conditions expressed in \textbf{(i)}-\textbf{(iv)} are directly translatable to each other by virtue of bijections in Proposition \ref{proposition:decription_of_linear_monoid_actions_on_modules}. 
\end{proof}
\noindent
Let $\fG$ be a monoid $k$-functor. We denote by $\Mod\left(\fO_k[\fG]\right)$ the category of $\fO_k[\fG]$-modules.

\section{Regular functions as a module over monoid $k$-functor}
\noindent
Let $\fG$ be a monoid $k$-functor. In this section we discuss important example of a $\fO_k[\fG]$-module. Fix a $k$-functor $\fX$ for which $\iMor_k(\fX,\fO_k)$ exists. Let $\alpha:\fG\times \fX\ra \fX$ be an action of $\fG$ on $\fX$. According to {\cite[Corollary 2.12]{kfunctors}} we deduce that $\alpha$ corresponds to a unique morphism of $k$-functors $\rho:\fG\ra \cI so_k(\fX)$. For every $k$-algebra $A$ and $x\in \fG(A)$ we have $\rho(x) = \alpha_x$. Moreover, $\rho$ is a morphism of $k$-monoids (this is a consequence of the fact that $\alpha$ is an action). Next we have a map of sets
$$\Mor_A\left(\fX_A,(\fO_k)_A\right) \ni f\mapsto f\cdot \rho(x)\in \Mor_A\left(\fX_A,(\fO_k)_A\right)$$
For every $A$-algebra $B$ and every point $y\in \fX(B)$ we have
$$(f\cdot \rho(x))(y) = f\left(\rho(x)(y)\right)$$
From this description it follows that the map $f\mapsto f\cdot \rho(x)$ is a morphism of $A$-algebras.










\small
\bibliographystyle{alpha}
\bibliography{../zzz}


\end{document}

