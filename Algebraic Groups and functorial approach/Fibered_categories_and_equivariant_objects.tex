\input ../pree.tex

\begin{document}

\title{Fibered categories and equivariant objects}
\date{}
\maketitle

\section{Introduction}
\noindent
In these notes we often work with two distinct categories. In order to make our notation clear we denote by $h^{\cC}:\cC\ra \widehat{\cC}$ the Yoneda embedding for category $\cC$. In particular, if $X$ is an object of $\cC$, then $h^{\cC}_X$ is a presheaf associated with $X$.

\section{Fibered categories}
\noindent
We fix a functor $p:\cE\ra \cB$. We introduce now some convenient notation that will help clarifying our definitions. Consider a morphism $\phi:\xi\ra \eta$ of $\cE$ such that $p(\phi) = f$ and $f:X\ra Y$. We depict this situation by the square diagram
\begin{center}
\begin{tikzpicture}
[description/.style={fill=white,inner sep=2pt}]
\matrix (m) [matrix of math nodes, row sep=3em, column sep=3em,text height=1.5ex, text depth=0.25ex] 
{ \xi      &  \eta      \\
  X        &  Y         \\} ;
\path[->,line width=1.0pt,font=\scriptsize]
(m-1-1) edge node[above] {$ \phi $} (m-1-2)
(m-2-1) edge node[below] {$ f $} (m-2-2);
\path[|->,line width=1.0pt,font=\scriptsize]
(m-1-1) edge node[left]  {$  $} (m-2-1)
(m-1-2) edge node[right] {$  $} (m-2-2);
\end{tikzpicture}
\end{center}
Note that to every such square there corresponds a commutative square 
\begin{center}
\begin{tikzpicture}
[description/.style={fill=white,inner sep=2pt}]
\matrix (m) [matrix of math nodes, row sep=3em, column sep=3em,text height=1.5ex, text depth=0.25ex] 
{ h^{\cE}_\xi      &  h^{\cE}_\eta      \\
  h^{\cB}_X\cdot p        &  h^{\cB}_Y\cdot p         \\} ;
\path[->,line width=1.0pt,font=\scriptsize]
(m-1-1) edge node[above] {$ h^{\cE}_\phi $} (m-1-2)
(m-2-1) edge node[below] {$ \left(h^{\cB}_f\right)_p $} (m-2-2)
(m-1-1) edge node[left]  {$ p_{\mathrm{hom}} $} (m-2-1)
(m-1-2) edge node[right] {$ p_{\mathrm{hom}} $} (m-2-2);
\end{tikzpicture}
\end{center}
of presheaves on $\cE$.

\begin{definition}
Consider a square 
\begin{center}
\begin{tikzpicture}
[description/.style={fill=white,inner sep=2pt}]
\matrix (m) [matrix of math nodes, row sep=3em, column sep=3em,text height=1.5ex, text depth=0.25ex] 
{ \xi      &  \eta      \\
  X        &  Y         \\} ;
\path[->,line width=1.0pt,font=\scriptsize]
(m-1-1) edge node[above] {$ \phi $} (m-1-2)
(m-2-1) edge node[below] {$ f $} (m-2-2);
\path[|->,line width=1.0pt,font=\scriptsize]
(m-1-1) edge node[left]  {$  $} (m-2-1)
(m-1-2) edge node[right] {$  $} (m-2-2);
\end{tikzpicture}
\end{center}
We call the square \textit{cartesian} and $\phi$ \textit{a cartesian morphism with respect to $p$} if the corresponding square of presheaves on $\cE$ is cartesian in the category of presheaves.
\end{definition}
\noindent
One can rephrase definition above in terms of presheaves as follows. Morphism $\phi:\xi\ra \eta$ is cartesian with respect to $p$ if the square 
\begin{center}
\begin{tikzpicture}
[description/.style={fill=white,inner sep=2pt}]
\matrix (m) [matrix of math nodes, row sep=3em, column sep=7em,text height=1.5ex, text depth=0.25ex] 
{ \Mor_{\cE}(\zeta,\xi) &  \Mor_{\cE}(\zeta,\eta)    \\
  \Mor_{\cB}\big(p(\zeta),p(\xi)\big) &  \Mor_{\cB}\big(p(\zeta),p(\eta)\big)           \\} ;
\path[->,line width=1.0pt,font=\scriptsize]
(m-1-1) edge node[above] {$ \Mor_{\cE}\left(1_{\zeta},\phi\right)  $} (m-1-2)
(m-2-1) edge node[below] {$ \Mor_{\cB}\left(1_{p(\zeta)},p(\phi)\right) $} (m-2-2)
(m-1-1) edge node[left]  {$ p_{\mathrm{hom}} $} (m-2-1)
(m-1-2) edge node[right] {$ p_{\mathrm{hom}} $} (m-2-2);
\end{tikzpicture}
\end{center}
of classes is cartesian for every object $\zeta$ of $\cE$.

\begin{fact}\label{fact:uniqueness_of_pullbacks}
Let $p:\cE\ra \cB$ be a functor, let $f:X\ra Y$ be a morphism of $\cB$ and let $\eta$ be an object of $\cE$. Suppose that $\phi_1:\xi_1\ra \eta,\phi_2:\xi_2\ra \eta$ are morphisms of $\cE$ that are cartesian with respect to $p$ and assume that $p(\phi_1) = p(\phi_2)$. Then there exists a unique morphism $\theta:\xi_1\ra \xi_2$ such that $\phi_1 = \phi_2\cdot \theta$. Moreover, $\theta$ is an isomorphism.
\end{fact}
\begin{proof}
We use the presheaf reformulation of a definition of cartesian morphisms of $p$. It implies that there exists a unique natural transformation $\sigma:h^{\cE}_{\xi_1}\ra h^{\cE}_{\xi_2}$ such that $h^{\cE}_{\phi_1} = h^{\cE}_{\phi_2}\cdot \sigma$. Moreover, $\sigma$ is a natural isomorphism. Since $h^{\cE}:\cE\ra \widehat{\cE}$ is full and faithful, we derive that there exists a unique morphism $\theta:\xi_1 \ra \xi_2$ such that $h^{\cE}_{\theta} = \sigma$. Then $\theta$ satisfies the assertion.
\end{proof}

\begin{definition}
Let $p:\cE\ra \cB$ be a functor, let $f:X\ra Y$ be a morphism of $\cB$ and let $\eta$ be an object of $\cE$ such that $p(\eta) = Y$. A pair $(\xi,\phi)$ such that $\xi$ is an object of $\cE$ and $\phi:\xi\ra \eta$ is a morphism of $\cE$ is called \textit{a pullback of $\eta$ along $f$} if the following conditions are satisfied.
\begin{enumerate}[label=\textbf{(\arabic*)}, leftmargin=3.0em]
\item $p(\phi) = f$
\item $\phi$ is cartesian morphism of $p$.
\end{enumerate}
\end{definition}
\noindent
Note that Fact \ref{fact:uniqueness_of_pullbacks} implies that pullbacks are unique up to a unique isomorphism.

\begin{definition}
Let $p:\cE\ra \cB$ be a functor. Then $p$ is \textit{a fibered category} if and only if for every morphism $f:X\ra Y$ of $\cB$ and every object $\eta$ of $\cE$ such that $p(\eta) = Y$ there exists a pullback of $\eta$ along $f$. If $p:\cE\ra \cB$ is a fibered category, then we say that \textit{$\cE$ is fibered over $\cB$ with respect to $p$}.
\end{definition}
\noindent
Now we give some examples of fibered categories. The first is a prototypical for the notion of a cartesian category. It shows that any category $\cB$ with fiber products gives rise in a canonical way to a fibered category over $\cB$ with cartesian arrows as cartesian squares in $\cB$.

\begin{example}[the fibered category of arrows]\label{example:the_fibered_category_of_arrows}
Let $\cB$ be a category. We define the category $\mathrm{Arr}(\cB)$ of arrows of $\cB$ as follows. Objects of $\mathrm{Arr}(\cB)$ are morphisms $\pi:\tilde{X}\ra X$ of $\cB$. Now if $\pi:\tilde{X}\ra X$ and $\psi:\tilde{Y}\ra Y$ are objects of $\mathrm{Arr}(\cB)$, then a morphism $\pi\ra \psi$ is a pair $(f,\phi)$ such that $f:X\ra Y$ and $\phi:\tilde{X}\ra \tilde{Y}$ are morphisms in $\cB$ making the square
\begin{center}
\begin{tikzpicture}
[description/.style={fill=white,inner sep=2pt}]
\matrix (m) [matrix of math nodes, row sep=2em, column sep=2em,text height=1.5ex, text depth=0.25ex] 
{ \tilde{X} &  \tilde{Y}    \\
  X &  Y           \\} ;
\path[->,line width=1.0pt,font=\scriptsize]
(m-1-1) edge node[above] {$ \phi  $} (m-1-2)
(m-2-1) edge node[below] {$ f $} (m-2-2)
(m-1-1) edge node[left] {$\pi $} (m-2-1)
(m-1-2) edge node[right] {$  \psi $} (m-2-2);
\end{tikzpicture}
\end{center}
commutative. There exists a functor $p_{\mathrm{Arr}}:\mathrm{Arr}(\cB)\ra \cB$ given by formula $p_{\mathrm{Arr}}\big((f,\phi)\big) = f$. Suppose now that $f:X\ra Y$ and $\psi:\tilde{Y}\ra Y$ are morphisms of $\cB$ and there exists a commutative square
\begin{center}
\begin{tikzpicture}
[description/.style={fill=white,inner sep=2pt}]
\matrix (m) [matrix of math nodes, row sep=2em, column sep=2em,text height=1.5ex, text depth=0.25ex] 
{ \tilde{X} &  \tilde{Y}    \\
  X &  Y           \\} ;
\path[->,line width=1.0pt,font=\scriptsize]
(m-1-1) edge node[above] {$ \phi  $} (m-1-2)
(m-2-1) edge node[below] {$ f $} (m-2-2)
(m-1-1) edge node[left] {$\pi $} (m-2-1)
(m-1-2) edge node[right] {$  \psi $} (m-2-2);
\end{tikzpicture}
\end{center}
It is a direct consequence of the definition that $(f,\phi)$ is a cartesian morphisms of $p_{\mathrm{Arr}}$ if and only if the square above is cartesian. Thus $p_{\mathrm{Arr}}$ is a fibered category provided that $\cB$ admits fiber products.
\end{example}

\begin{definition}
Suppose that $p_1:\cE_1\ra \cB$ and $p_2:\cE_2\ra \cB$ are fibered categories. Then a functor $F:\cE_1\ra \cE_2$ is \textit{a morphism of fibered categories} if the following two assertions are satisfied.
\begin{enumerate}[label=\textbf{(\arabic*)}, leftmargin=3.0em]
\item $p_1 = F\cdot p_2$ or in other words $F$ is a functor over $\cB$.
\item Image under $F$ of a cartesian morphism of $p_1$ is a cartesian morphism of $p_2$.
\end{enumerate}
\end{definition}
\noindent
Next example is closely related to the previous one, but is of more topological flavour.

\begin{example}[the fibered category vector bundles]\label{example:the_fibered_category_of_vector_bundles}
Let $\Top$ be the category of topological spaces. We define a subcategory $\bd{VectBund}_{\RR}$ of $\mathrm{Arr}(\Top)$ of vector bundles as follows. Objects of $\bd{VectBund}_{\RR}$ are topological $\RR$-vector bundles $\pi:\cV \ra X$. Now if $\pi:\cV \ra X$ and $\psi:\cW \ra Y$ are topological $\RR$-vector bundles, then a morphism $\pi\ra \psi$ is a pair $(f,\phi)$ such that $f:X\ra Y$ is a continuous map and $\phi:\cV \ra \cW$ is a continuous making the square
\begin{center}
\begin{tikzpicture}
[description/.style={fill=white,inner sep=2pt}]
\matrix (m) [matrix of math nodes, row sep=2em, column sep=2em,text height=1.5ex, text depth=0.25ex] 
{ \cV &  \cW    \\
  X &  Y           \\} ;
\path[->,line width=1.0pt,font=\scriptsize]
(m-1-1) edge node[above] {$ \phi  $} (m-1-2)
(m-2-1) edge node[below] {$ f $} (m-2-2)
(m-1-1) edge node[left] {$\pi $} (m-2-1)
(m-1-2) edge node[right] {$  \psi $} (m-2-2);
\end{tikzpicture}
\end{center}
commutative and moreover, $\phi$ induces an $\RR$-linear map on fibers i.e. for each point $x$ in $X$ map $\phi$ induces an $\RR$-linear map $\pi^{-1}(x)\ra \psi^{-1}\left(f(x)\right)$. Since topological vector bundles are stable under continuous change of base, we obtain a fibered category $\bd{VectBund}_{\RR}\ra \Top$ as the restriction of $p_{\mathrm{Arr}}:\mathrm{Arr}(\Top)\ra \Top$. Thus we have a commutative triangle
\begin{center}
\begin{tikzpicture}
[description/.style={fill=white,inner sep=2pt}]
\matrix (m) [matrix of math nodes, row sep=2em, column sep=1em,text height=1.5ex, text depth=0.25ex] 
{  \bd{VectBund}_{\RR} &        & \mathrm{Arr}(\Top)  \\
          &\Top &  \\} ;
\path[right hook->,line width=1.0pt,font=\scriptsize]
(m-1-1) edge node[above] {$  $} (m-1-3);
\path[->,line width=1.0pt,font=\scriptsize]
(m-1-1) edge node[swap] {$  $} (m-2-2)
(m-1-3) edge node[below = 6pt, right = 1pt] {$ p_{\mathrm{Arr}} $} (m-2-2);
\end{tikzpicture}
\end{center}
According to Example \ref{example:the_fibered_category_of_arrows} the inclusion $\bd{VectBund}_{\RR}\hookrightarrow \mathrm{Arr}(\Top)$ is a morphism of fibered categories.
\end{example}

\section{Pseudo-functors and fibered categories of elements}
\noindent
Pseudo-functors are certain non-strict 2-functors. In this section we introduce a procedure that enables to construct a fibered category out of a pseudo-functor. We start by defining this notion.

\begin{definition}
Let $\cB$ be a category. Consider the tuple of collections
$$F = \big(\{F(X)\}_{X\in \mathrm{Ob}(\cB)},\{F(f)\}_{f\in \Mor(\cB)},\{\Theta^{f,g}\}_{f,g\in \Mor(\cB),\,\mathrm{cod}(f)=\mathrm{dom}(g)},\{\epsilon^X\}_{X\in \mathrm{Ob}(\cB)}\big)$$
of the following data.
\begin{enumerate}[label=\textbf{(\arabic*)}, leftmargin=3.0em]
\item For each object $X$ of $\cB$ a category $F(X)$.
\item For each arrow $f:X\ra Y$ a functor $F(f):F(Y)\ra F(X)$.
\item For each object $X$ of $\cB$ a natural isomorphism $\epsilon^X:1_{F(X)} \ra F(1_{X})$.
\item For any two composable morphisms $f:X\ra Y$ and $g:Y\ra Z$ of $\cB$ a natural isomorphism $\Theta^{g,f}: F(f)\cdot F(g) \ra F(g\cdot f)$
\end{enumerate}
Suppose that these data are subject to the following conditions.
\begin{enumerate}[label=\textbf{(\arabic*)}, leftmargin=3.0em]
\item For every arrow $f:X\ra Y$ in $\cB$ we have
$$1_{F(f)} = \Theta^{f,1_X} \cdot \epsilon^X_{F(f)},\,1_{F(f)} = \Theta^{1_Y,f} \cdot F(f)\left(\epsilon^Y\right)$$
\item For any three morphisms $f:X\ra Y,g:Y\ra Z,h:Z\ra W$ of $\cB$ the square of functors and natural isomorphisms
\begin{center}
\begin{tikzpicture}
[description/.style={fill=white,inner sep=2pt}]
\matrix (m) [matrix of math nodes, row sep=3em, column sep=3em,text height=1.5ex, text depth=0.25ex] 
{ F(f)\cdot F(g)\cdot F(h) & F(f)\cdot F\left(h\cdot g\right)      \\
  F\left(g\cdot f\right)\cdot F(h) & F\left(h\cdot g\cdot f\right) \\} ;
\path[->,line width=1.0pt,font=\scriptsize]
(m-1-1) edge node[above] {$  F(f)\big(\Theta^{h,g}\big)  $} (m-1-2)
(m-2-1) edge node[below] {$ \Theta^{h,g\cdot f} $} (m-2-2)
(m-1-1) edge node[left] {$  \Theta^{g,f}_{F(h)}  $} (m-2-1)
(m-1-2) edge node[right] {$ \Theta^{h\cdot g,f} $} (m-2-2);
\end{tikzpicture}
\end{center}
is commutative.
\end{enumerate}
Then $F$ is called \textit{a pseudo-functor on $\cB$}
\end{definition}
\noindent
Now we show how to construct a fibered category from a pseudo-functor. Suppose that $\cB$ is a category and
$$F = \big(\{F(X)\}_{X\in \mathrm{Ob}(\cB)},\{F(f)\}_{f\in \Mor(\cB)},\{\Theta^{f,g}\}_{f,g\in \Mor(\cB),\,\mathrm{cod}(f)=\mathrm{dom}(g)},\{\epsilon^X\}_{X\in \mathrm{Ob}(\cB)}\big)$$
is a pseudo-functor on $\cB$. We define a category $\bigint_{\cB}F$. Its objects are pairs $(X,\xi)$ such that $X$ is an object of $\cB$ and $\xi$ is an object of $F(X)$. If $(X,\xi)$ and $(Y,\eta)$ are objects of $\bigint_{\cB}F$, then a morphism between these objects is a pair $(f,\sigma)$ such that $f:X\ra Y$ is a morphism of $\cB$ and $\sigma:\xi\ra F(f)(\eta)$ is a morphism of $F(X)$. Now suppose that $(f,\sigma):(X,\xi)\ra (Y,\eta)$ and $(g,\tau):(Y,\eta)\ra (Z,\zeta)$ are morphisms of $\bigint_{\cB}F$. Then we define their composition by formula
$$(g,\tau)\cdot (f,\sigma) = \big(g\cdot f, \Theta^{g,f}_{\zeta}\cdot F(f)\left(\tau\right)\cdot \sigma\big)$$

\begin{fact}
$\bigint_{\cB}F$ is a well defined category.
\end{fact}
\begin{proof}
We first verify that the composition of morphisms in $\bigint_{\cB}F$ is associative. Suppose that $(f,\sigma):(X,\xi)\ra (Y,\eta),(g,\tau):(Y,\eta)\ra (Z,\zeta),(h,\rho):(Z,\zeta)\ra (W,\omega)$ are morphisms of $\bigint_{\cB}F$. Then
$$\big((h,\rho)\cdot (g,\tau)\big)\cdot (f,\sigma) = \big(h\cdot g,\Theta^{h,g}_{\omega}\cdot F(g)\left(\rho\right)\cdot \tau\big)\cdot (f,\sigma) = $$
$$ = \bigg(h\cdot g\cdot f,\Theta^{h\cdot g,f}_{\omega}\cdot F(f)\big(\Theta^{h,g}_{\omega}\cdot F(g)\left(\rho\right)\cdot \tau \big)\cdot \sigma\bigg) = \bigg(h\cdot g\cdot f,\Theta^{h\cdot g,f}_{\omega}\cdot F(f)\big(\Theta^{h,g}_{\omega}\big)\cdot F(f)\big(F(g)\left(\rho\right)\big)\cdot F(f)\big(\tau\big)\cdot \sigma \bigg)$$
and
$$(h,\rho)\cdot \big((g,\tau) \cdot (f,\sigma) \big) = (h, \rho)\cdot \bigg(g\cdot f, \Theta^{g,f}_{\zeta}\cdot F(f)\big(\tau\big)\cdot \sigma\bigg) = $$
$$ = \big(h\cdot g\cdot f, \Theta^{h, g\cdot f}_{\omega}\cdot F(g\cdot f)\big(\rho\big)\cdot \Theta^{g,f}_{\zeta}\cdot F(f)\big(\tau\big)\cdot \sigma \big) = \big(h\cdot g\cdot f, \Theta^{h, g\cdot f}_{\omega}\cdot  \Theta^{g,f}_{F(h)(\omega)} \cdot F(f)\big(F(g)\left(\rho\right)\big)\cdot F(f)\big(\tau\big)\cdot \sigma \big)$$
Since $\Theta^{h\cdot g,f}_{\omega}\cdot F(f)\big(\Theta^{h,g}_{\omega}\big) = \Theta^{h, g\cdot f}_{\omega}\cdot  \Theta^{g,f}_{F(h)(\omega)}$, we deduce that
$$\big((h,\rho)\cdot (g,\tau)\big)\cdot (f,\sigma) = (h,\rho)\cdot \big((g,\tau) \cdot (f,\sigma) \big)$$
and hence the composition in $\bigint_{\cB}F$ is associative. Next we prove that for each object $(X,\xi)$ of $\bigint_{\cB}F$ there exists an identity morphism. We claim that $(1_X,\epsilon^X_{\xi}):(X,\xi)\ra (X,\xi)$ is the identity. Indeed, for morphisms $(f,\sigma):(X,\xi)\ra (Y,\eta)$ and $(g,\tau):(Z,\zeta)\ra (X,\xi)$ we have
$$(f,\sigma) \cdot (1_X,\epsilon^X_{\xi}) = \big(f,\Theta^{f,1_X}_{\eta}\cdot F(1_X)\left(\sigma \right)\cdot \epsilon^X_{\xi}\big) =  \big(f,\Theta^{f,1_X}_{\eta}\cdot \epsilon^X_{F(f)(\eta)}\cdot \sigma\big) = (f,\sigma)$$
and
$$(1_X,\epsilon^X_{\xi}) \cdot (g,\tau) = \big(g,\Theta^{1_X,g}_{\xi} \cdot F(g)\left(\epsilon^X_{\xi}\right)\cdot \tau \big) = (g,\tau)$$
Therefore, $\bigint_{\cB}F$ is a category.
\end{proof}
\noindent
Next we define a functor $p_F:\bigint_{\cB}F\ra \cB$ by formula
$$p_F\bigg((f,\sigma):(X,\xi)\ra (Y,\tau)\bigg) = f:X\ra Y$$
This is clearly a well defined functor. Now we prove the following statement.
\begin{flushleft}
\textit{The functor $p_F:\bigint_{\cB}F\ra \cB$ is a fibered category.}
\end{flushleft}
\begin{proof}
Let $f:X\ra Y$ be a morphism in $\cB$ and $\eta$ be an object of $F(Y)$. Thus $(Y,\eta)$ is an object of $\bigint_{\cB}F$. It suffices to show that $(Y,\eta)$ admits a pullback along $f$. We claim that
$$\big(f,1_{F(f)(\eta)}\big):\big(X,F(f)(\eta)\big)\ra (Y,\eta)$$
is a cartesian morphism of $p_F$ that yields a pullback of $\eta$ along $f$. To prove the claim consider an object $(Z,\zeta)$ of $\bigint_{\cB}F$ and suppose that $(g,\tau):(Z,\zeta)\ra (Y,\eta)$ is a morphism of $\bigint_{\cB}F$ such that $g$ factors through $f$. Then there exists $h:Z\ra X$ such that $f\cdot h = g$. Note that $\tau:\zeta\ra F(g)(\eta)$. Since $g = f\cdot h$, we have
$$\tau = \Theta^{f,h}_{\eta}\cdot \left(\Theta^{f,h}_{\eta}\right)^{-1}\cdot \tau = \Theta^{f,h}_{\eta}\cdot F(h)\big(1_{F(f)(\eta)}\big) \cdot  \left(\Theta^{f,h}_{\eta}\right)^{-1}\cdot \tau$$
and hence
$$(g,\tau) = \big(f, 1_{F(f)(\eta)} \big) \cdot \bigg(h,\left(\Theta^{f,h}_{\eta}\right)^{-1}\cdot \tau \bigg)$$
Thus $(g,\tau)$ factors through $\big(f,1_{F(f)(\eta)}\big)$ and the formula above shows that this factorization is unique. Hence $\big(f,1_{F(f)(\eta)}\big)$ is a cartesian morphism of $p_F$.
\end{proof}

\begin{definition}
Let $\cB$ be a category and let $F$ be a pseudo-functor on $\cB$. A fibered category $p_F:\bigint_{\cB}F\ra \cB$ constructed above is called \textit{the fibered category of elements of the pseudo-functor $F$}.
\end{definition}
\noindent
It is possible to construct a pseudo-functor out of a fibered category. We will give a brief outline of this construction. For this we introduce notation that will be also used in other considerations.

\begin{definition}
Let $p:\cE\ra \cB$ be a fibered category. For every object $X$ of $\cB$ we denote by $p^{-1}(X)$ a subcategory of $\cE$ consisting of all morphisms $\phi:\xi\ra \eta$ such that $p(\phi) = 1_X$. Then $p^{-1}(X)$ is called \textit{the fiber of $p$ over $X$}.
\end{definition}
\noindent
Suppose now that $p:\cE\ra \cB$ is a fibered category. Let $f:X\ra Y$ be a morphism. For every object $\eta$ in $p^{-1}(Y)$ we pick its pullback $\tilde{f}_{\eta}:f^*\eta\ra \eta$ along $f$. By universal property of cartesian morphisms we deduce that this induces a functor $f^*:p^{-1}(Y)\ra p^{-1}(X)$. Universal property of cartesian morphisms implies also the following assertions.
\begin{enumerate}[label=\textbf{(\arabic*)}, leftmargin=3.0em]
\item For each object $X$ of $\cB$ we may choose $(1_X)^* = 1_{p^{-1}(X)}$.
\item For any two composable morphisms $f:X\ra Y$ and $g:Y\ra Z$ of $\cB$ there exists a unique natural isomorphism $\Theta^{g,f}: f^*g^* \ra (g\cdot f)^*$ of functors such that for every $\zeta$ in $p^{-1}(Z)$ we have commutative diagram
\begin{center}
\begin{tikzpicture}
[description/.style={fill=white,inner sep=2pt}]
\matrix (m) [matrix of math nodes, row sep=3em, column sep=3em,text height=1.5ex, text depth=0.25ex] 
{ f^*g^*\zeta       &  g^*\zeta & \zeta    \\
  (g\cdot f)^*\zeta &           & \zeta    \\} ;
\path[->,line width=1.0pt,font=\scriptsize]
(m-1-1) edge node[above] {$ \tilde{f}_{g^*{\zeta}}  $} (m-1-2)
(m-1-2) edge node[above] {$ \tilde{g}_{\zeta}  $} (m-1-3)
(m-1-1) edge node[left] {$ \Theta^{g,f}_{\zeta}  $} (m-2-1)
(m-1-3) edge node[right] {$ 1_{\zeta}  $} (m-2-3)
(m-2-1) edge node[below] {$ \widetilde{g\cdot f}_{\zeta}  $} (m-2-3);
\end{tikzpicture}
\end{center}
\end{enumerate}
From \textbf{(1)}, \textbf{(2)} and Fact \ref{fact:uniqueness_of_pullbacks} one can deduce that the collection
$$\big(\{p^{-1}(X)\}_{X\in \mathrm{Ob}(\cB)},\{f^*\}_{f\in \Mor(\cB)},\{\Theta^{f,g}\}_{f,g\in \Mor(\cB),\,\mathrm{cod}(f)=\mathrm{dom}(g)},\{1_{p^{-1}(X)}\}_{X\in \mathrm{Ob}(\cB)}\big)$$
is a pseudo-functor.

\begin{remark}\label{remark:fibered_categories_and_pseudo_functors}
The construction of the fibered category of elements is a part of $2$-equivalence between appropriately defined category of pseudo-functors on $\cB$ and the category of fibered categories over $\cB$.
\end{remark}

\section{Example: Quasi-coherent sheaves}
\noindent
Note that all examples of fibered categories given so far were fibered subcategories of the fibered category of arrows $p_{\mathrm{Arr}}:\mathrm{Arr}(\cB)\ra \cB$ for a given category $\cB$ with fibered-products. In this section we employ the procedure that produces a fibered category out of a pseudo-functor to obtain an important example of a category fibered over $\Sch_k$ (the category of schemes over a ring $k$), which is not of this type.\\
Let $f:X\ra Y$ be a morphism of $k$-schemes. We have an adjunction
\begin{center}   
\begin{tikzpicture}
[description/.style={fill=white,inner sep=2pt}]
\matrix (m) [matrix of math nodes, row sep=3em, column sep=1em,text height=1.5ex, text depth=0.25ex] 
{ \Qcoh(X)& \perp  &\Qcoh(Y)  \\};
\path[solid,->,line width=1.0pt,font=\scriptsize]
(m-1-1) edge [bend left=30] node[auto]  {$ f_* $} (m-1-3)
(m-1-3) edge [bend left=30] node[auto]  {$ f^* $} (m-1-1);
\end{tikzpicture}
\end{center}
It is determined by the bijection
\begin{center}
\begin{tikzpicture}
[description/.style={fill=white,inner sep=2pt}]
\matrix (m) [matrix of math nodes, row sep=3em, column sep=4em,text height=1.5ex, text depth=0.25ex] 
{ \Hom_{\cO_Y}\big(f^*\cG,\cF\big) & \Hom_{\cO_X}\big(\cG,f_*\cF\big) \\} ;
\path[->,line width=1.0pt,font=\scriptsize]
(m-1-1) edge node[above] {$ \Phi^f_{\cG,\cF}  $} (m-1-2);
\end{tikzpicture}
\end{center}
Suppose now that $f:X\ra Y$ and $g:Y\ra Z$ are morphisms of $k$-schemes. Since $(g\cdot f)_* = g_*\cdot f_*$, there exists a unique natural isomorphism $\Theta^{g,f}: f^*g^* \ra (g\cdot f)^*$ such that for every quasi-coherent sheaf $\cF$ in $\Qcoh(X)$ and every quasi-coherent sheaf $\cH$ in $\Qcoh(Z)$ we have
$$\Phi^{g\cdot f}_{\cH,\cF} = \Phi^g_{\cH,f_*\cF}\cdot \Phi^f_{g^*\cH,\cF}  \cdot \Hom_{\cO_X}\big(\Theta^{g,f}_{\cH},1_{\cF}\big)$$
Now we have the following result.

\begin{fact}\label{fact:quasi_coherent_sheaves_are_pseudo_functor_associativity}
Suppose that $f:X\ra Y$, $g:Y\ra Z$ and $h:Z\ra W$ are morphism of $k$-schemes. Then the square
\begin{center}
\begin{tikzpicture}
[description/.style={fill=white,inner sep=2pt}]
\matrix (m) [matrix of math nodes, row sep=3em, column sep=3em,text height=1.5ex, text depth=0.25ex] 
{ f^*g^*h^*                      & f^*\left(h\cdot g\right)^*       \\
  \left(g\cdot f\right)^*h^*     & \left(h\cdot g\cdot f\right)^*   \\} ;
\path[->,line width=1.0pt,font=\scriptsize]
(m-1-1) edge node[above] {$ f^*\Theta^{h,g}     $} (m-1-2)
(m-2-1) edge node[below] {$ \Theta^{h,g\cdot f} $} (m-2-2)
(m-1-1) edge node[left]  {$ \Theta^{g,f}_{h^*}  $} (m-2-1)
(m-1-2) edge node[right] {$ \Theta^{h\cdot g,f} $} (m-2-2);
\end{tikzpicture}
\end{center}
of functors and natural isomorphisms is commutative.
\end{fact}
\begin{proof}
Suppose that $\cF$ is an object of $\Qcoh(X)$ and $\cK$ is an object of $\Qcoh(W)$. Then
$$\Phi^h_{\cK,g_*f_*\cF}\cdot \Phi^g_{h^*\cK,f_*\cF}\cdot \Phi^f_{g^*h^*\cK,\cF} \cdot \Hom_{\cO_X}\big(\Theta^{g,f}_{h^*\cK},1_{\cF}\big) \cdot \Hom_{\cO_X}\big(\Theta^{h,g\cdot f}_{\cK},1_{\cF}\big)  = $$
$$= \Phi^h_{\cK,g_*f_*\cF}\cdot \Phi^{g\cdot f}_{h^*\cK,\cF} \cdot \Hom_{\cO_X}\big(\Theta^{g,f}_{h^*\cK},1_{\cF}\big) = \Phi^{h\cdot g\cdot f}_{\cK,\cF}$$

and
$$\Phi^h_{\cK,g_*f_*\cF}\cdot \Phi^g_{h^*\cK,f_*\cF}\cdot \Phi^f_{g^*h^*\cK,\cF}\cdot \Hom_{\cO_X}\big(f^*\Theta^{h,g}_{\cK},1_{\cF}\big)\cdot \Hom_{\cO_X}\big(\Theta^{h\cdot g,f}_{\cK},1_{\cF}\big) = $$
$$ = \Phi^h_{\cK,g_*f_*\cF}\cdot \Phi^g_{h^*\cK,f_*\cF}\cdot \Hom_{\cO_X}\big(\Theta^{h,g}_{\cK},1_{f_*\cF}\big) \cdot  \Phi^f_{(h\cdot g)^*\cK,\cF} \cdot \Hom_{\cO_X}\big(\Theta^{h\cdot g,f}_{\cK},1_{\cF}\big) = $$
$$ = \Phi^{h\cdot g}_{\cK,f_*\cF} \cdot  \Phi^f_{(h\cdot g)^*\cK,\cF} \cdot \Hom_{\cO_X}\big(\Theta^{h\cdot g,f}_{\cK},1_{\cF}\big) = \Phi^{h\cdot g\cdot f}_{\cK,\cF} $$
Therefore, we derive that
$$\Phi^h_{\cK,g_*f_*\cF}\cdot \Phi^g_{h^*\cK,f_*\cF}\cdot \Phi^f_{g^*h^*\cK,\cF}\cdot \Hom_{\cO_X}\big(\Theta^{g,f}_{h^*\cK},1_{\cF}\big)\cdot \Hom_{\cO_X}\big(\Theta^{h,g\cdot f}_{\cK},1_{\cF}\big) = $$
$$ = \Phi^h_{\cK,g_*f_*\cF}\cdot \Phi^g_{h^*\cK,f_*\cF}\cdot \Phi^f_{g^*h^*\cK,\cF}\cdot \Hom_{\cO_X}\big(f^*\Theta^{h,g}_{\cK},1_{\cF}\big)\cdot \Hom_{\cO_X}\big(\Theta^{h\cdot g,f}_{\cK},1_{\cF}\big)$$
and hence
$$\Hom_{\cO_X}\big(\Theta^{h,g\cdot f}_{\cK} \cdot \Theta^{g,f}_{h^*\cK},1_{\cF}\big) = \Hom_{\cO_X}\big(\Theta^{g,f}_{h^*\cK},1_{\cF}\big)\cdot \Hom_{\cO_X}\big(\Theta^{h,g\cdot f}_{\cK},1_{\cF}\big) =$$
$$= \Hom_{\cO_X}\big(f^*\Theta^{h,g}_{\cK},1_{\cF}\big) \cdot \Hom_{\cO_X}\big(\Theta^{h\cdot g,f}_{\cK},1_{\cF}\big) = \Hom_{\cO_X}\big( \Theta^{h\cdot g,f}_{\cK} \cdot f^*\Theta^{h,g}_{\cK},1_{\cF}\big)$$
Since this equality holds for every quasi-coherent sheaf $\cF$ on $X$, we deduce that
$$\Theta^{h,g\cdot f}_{\cK} \cdot \Theta^{g,f}_{h^*\cK} = \Theta^{h\cdot g,f}_{\cK} \cdot f^*\Theta^{h,g}_{\cK}$$
for every quasi-coherent sheaf $\cK$. This proves the assertion.
\end{proof}
\noindent
Note that for every $k$-scheme $X$ we may assume that $\left(1_X\right)_* = 1_{\Qcoh(X)} = \left(1_X\right)^*$ and $\Phi^{1_X}_{\cG,\cF} = \Hom_{\cO_X}\left(1_\cF,1_\cG\right)$.

\begin{fact}\label{fact:quasi_coherent_sheaves_are_pseudo_functor_unit}
Let $f:X\ra Y$ and $g:Z\ra X$ be morphisms of $k$-schemes. Then
$$\Theta^{f,1_X} = 1_{f^*},\,\Theta^{1_X,g} = 1_{g^*}$$
\end{fact}
\begin{proof}
Suppose that $\cF$ is an object of $\Qcoh(X)$ and $\cG$ is an object of $\Qcoh(Y)$. Then
$$\Phi^{f}_{\cG,\cF} = \Phi^{f\cdot 1_X}_{\cG,\cF} = \Phi^f_{\cG,\cF} \cdot \Phi^{1_X}_{f^*\cG,\cF} \cdot \Hom_{\cO_X}\left(\Theta^{f,1_X}_{\cG},1_{\cF}\right) = \Phi^f_{\cG,\cF} \cdot \Hom_{\cO_X}\left(\Theta^{f,1_X}_{\cG},1_{\cF}\right)$$
and thus $\Hom_{\cO_X}\left(\Theta^{f,1_X}_{\cG},1_{\cF}\right) = \Hom_{\cO_X}\left(1_{f^*\cG},1_{\cF}\right)$. Since this holds for every quasi-coherent sheaf $\cF$ on $X$, we derive that $\Theta^{f,1_X}_{\cG} = 1_{f^*\cG}$. Thus $\Theta^{f,1_X} = 1_{f^*}$.\\
Suppose that $\cH$ is an object of $\Qcoh(X)$ and $\cF$ is an object of $\Qcoh(Z)$. Then
$$\Phi^g_{\cH,\cF} = \Phi^{1_X\cdot g}_{\cH,\cF} = \Phi^{1_X}_{\cH,g_*\cF} \cdot \Phi^g_{\cH,\cF}\cdot \Hom_{\cO_Z}\left(\Theta^{1_X,g}_{\cH},1_{\cF}\right) = \Phi^g_{\cH,\cF}\cdot \Hom_{\cO_Z}\left(\Theta^{1_X,g}_{\cH},1_{\cF}\right)$$
and thus $\Hom_{\cO_Z}\left(\Theta^{1_X,g}_{\cH},1_{\cF}\right) = \Hom_{\cO_Z}\left(1_{g^*\cH} ,1_{\cF}\right)$. Since this holds for every quasi-coherent sheaf $\cF$ on $Z$, we derive that $\Theta^{1_X,g}_{\cH} = 1_{g^*\cH}$. Thus $\Theta^{1_X,g} = 1_{g^*}$.
\end{proof}
\noindent
Now Facts \ref{fact:quasi_coherent_sheaves_are_pseudo_functor_associativity} and \ref{fact:quasi_coherent_sheaves_are_pseudo_functor_unit} imply that the collection
$$\big(\{\Qcoh(X)\}_{X\in \Sch_k},\{f^*\}_{f\in \Mor(\Sch_k)},\{\Theta^{f,g}\}_{f,g\in \Mor(\Sch_k),\,\mathrm{cod}(f)=\mathrm{dom}(g)},\{1_{1_{\Qcoh(X)}}\}_{X\in \Sch_k}\big)$$
forms a pseudo-functor on $\Sch_k$.

\begin{definition}
\textit{The fibered category of quasi-coherent sheaves on $\Sch_k$} is the fibered category of elements of the pseudo-functor
$$\big(\{\Qcoh(X)\}_{X\in \Sch_k},\{f^*\}_{f\in \Mor(\Sch_k)},\{\Theta^{f,g}\}_{f,g\in \Mor(\Sch_k),\,\mathrm{cod}(f)=\mathrm{dom}(g)},\{1_{1_{\Qcoh(X)}}\}_{X\in \Sch_k}\big)$$
We denote it by $\Qcoh \ra \Sch_k$.
\end{definition}
\noindent
For every $k$-scheme $X$ we have a category $\Alg\left(\Qcoh(X)\right)$ of quasi-coherent $\cO_X$-algebras. If $f:X\ra Y$ is a morphism of $k$-schemes, then we have an adjuntion
\begin{center}   
\begin{tikzpicture}
[description/.style={fill=white,inner sep=2pt}]
\matrix (m) [matrix of math nodes, row sep=3em, column sep=1em,text height=1.5ex, text depth=0.25ex] 
{ \Alg\left(\Qcoh(X)\right)& \perp  & \Alg\left(\Qcoh(Y)\right)  \\};
\path[solid,->,line width=1.0pt,font=\scriptsize]
(m-1-1) edge [bend left=30] node[auto]  {$ f_* $} (m-1-3)
(m-1-3) edge [bend left=30] node[auto]  {$ f^* $} (m-1-1);
\end{tikzpicture}
\end{center}
Using similar argument as above one can show that there exists a canonical structure of a pseudo-functor on the collection
$$\big(\{\Alg\left(\Qcoh(X)\right)\}_{X\in \Sch_k},\{f^*\}_{f\in \Mor(\Sch_k)}\big)$$

\begin{definition}
\textit{The fibered category of quasi-coherent algebras on $\Sch_k$} is the fibered category of elements of the canonical pseudo-functor determined by the collection
$$\big(\{\Alg\left(\Qcoh(X)\right)\}_{X\in \Sch_k},\{f^*\}_{f\in \Mor(\Sch_k)}\big)$$
We denote it by $\Alg\left(\Qcoh\right)\ra \Sch_k$.
\end{definition}

\begin{remark}\label{remark:from_qc_algebras_to_qc_sheaves_over_schemes}
For every $k$-scheme we also have a canonical functor $|-|:\Alg\left(\Qcoh(X)\right)\ra \Qcoh(X)$ that forgets about an algebra structure. The collection of all these functors for all $k$-schemes gives rise to a morphism of fibered categories
\begin{center}
\begin{tikzpicture}
[description/.style={fill=white,inner sep=2pt}]
\matrix (m) [matrix of math nodes, row sep=2em, column sep=1em,text height=1.5ex, text depth=0.25ex] 
{  \Alg\left(\Qcoh\right) &        & \Qcoh  \\
          &\Sch_k &  \\} ;
\path[->,line width=1.0pt,font=\scriptsize]
(m-1-1) edge node[above] {$|-|  $} (m-1-3)
(m-1-1) edge node[swap] {$  $} (m-2-2)
(m-1-3) edge node[below = 6pt, right = 1pt] {$ $} (m-2-2);
\end{tikzpicture}
\end{center}
\end{remark}

\begin{remark}\label{remark:from_qc_algebras_to_the_category_of_arrows_over_schemes}
Note that $\mathrm{Arr}(\Sch_k)$ admits a fibered subcategory that consists of affine morphisms $\pi:\widetilde{X}\ra X$ of $k$-schemes. We denote this fibered category by $\bd{Aff}\left(\Sch_k\right)\ra \Sch_k$. For every $k$-scheme $X$ we have the relative affine spectrum functor $\Spec_X:\Alg\left(\Qcoh(X)\right)\ra \bd{Aff}_X$. It is an equivalence of categories. Moreover, if $f:X\ra Y$ is a morphism of $k$-schemes and $\cA$ is a quasi-coherent $\cO_Y$-algebra, then the canonical square
\begin{center}
\begin{tikzpicture}
[description/.style={fill=white,inner sep=2pt}]
\matrix (m) [matrix of math nodes, row sep=2em, column sep=2em,text height=1.5ex, text depth=0.25ex] 
{ \Spec_X f^*\cA & \Spec_Y\cA     \\
  X             & Y             \\} ;
\path[->,line width=1.0pt,font=\scriptsize]  
(m-1-1) edge node[above] {$  $} (m-1-2)
(m-2-1) edge node[below] {$ f $} (m-2-2)
(m-1-1) edge node[left] {$  $} (m-2-1)
(m-1-2) edge node[right] {$  $} (m-2-2);
\end{tikzpicture}
\end{center}
is cartesian. Thus the collection of all these functors for all $k$-schemes gives rise to a morphism of fibered categories
\begin{center}
\begin{tikzpicture}
[description/.style={fill=white,inner sep=2pt}]
\matrix (m) [matrix of math nodes, row sep=2em, column sep=1em,text height=1.5ex, text depth=0.25ex] 
{  \Alg\left(\Qcoh\right) &        & \bd{Aff}(\Sch_k)  \\
          &\Sch_k &  \\} ;
\path[->,line width=1.0pt,font=\scriptsize]
(m-1-1) edge node[above] {$ \Spec  $} (m-1-3)
(m-1-1) edge node[swap] {$  $} (m-2-2)
(m-1-3) edge node[below = 6pt, right = 1pt] {$ $} (m-2-2);
\end{tikzpicture}
\end{center}
\end{remark}

\section{Equivariant objects in fibered categories}
\noindent
Let $k$ be a commutative ring. The following notion is very useful for studying actions of algebraic groups and monoids.

\begin{definition}
Let $X$ be a $k$-scheme and let $\bd{M}$ be a monoid $k$-scheme with an action $a:\bd{M}\times_kX\ra X$ on $X$. We denote by $\pi:\bd{M}\times_kX\ra X$ the projection. Consider a pair $(\cF,\tau)$ consisting of a quasi-coherent sheaf $\cF$ on $X$ and an isomorphism $\tau:\pi^*\cF \ra a^* \cF$. We call it \textit{a quasi-coherent $\bd{M}$-sheaf on $(X,a)$} if the following equality
$$(\mu\times_k1_X)^*\phi = (1_{\bd{M}} \times_k a)^*\phi \cdot \pi_{2,3}^*\phi$$
holds, where $\mu: \bd{M} \times_k \bd{M} \ra \bd{M}$ is the multiplication on $\bd{M}$ and $\pi_{2,3}:\bd{M}\times_k \bd{M} \times_k X \ra \bd{M} \times_kX$ is the projection on last two factors.
\end{definition}

\begin{definition}
Let $X$ be a $k$-scheme and let $\bd{M}$ be a monoid $k$-scheme with an action $a:\bd{M}\times_kX\ra X$ on $X$. We denote by $\pi:\bd{M}\times_kX\ra X$ the projection. Let $(\cF_1,\tau_1)$ and $(\cF_2,\tau_2)$ be quasi-coherent $\bd{M}$-sheaves on $(X,a)$. Suppose that $\phi:\cF_1\ra \cF_2$ is a morphism of quasi-coherent sheaves on $X$ such that the square
\begin{center}
\begin{tikzpicture}
[description/.style={fill=white,inner sep=2pt}]
\matrix (m) [matrix of math nodes, row sep=3em, column sep=3em,text height=1.5ex, text depth=0.25ex] 
{ \pi^*\cF_1 & a^*\cF_1     \\
  \pi^*\cF_2 & a^*\cF_2             \\} ;
\path[->,line width=1.0pt,font=\scriptsize]  
(m-1-1) edge node[above] {$ \tau_1 $} (m-1-2)
(m-2-1) edge node[below] {$ \tau_2 $} (m-2-2)
(m-1-1) edge node[left] {$ \pi^*\phi $} (m-2-1)
(m-1-2) edge node[right] {$ a^*\phi  $} (m-2-2);
\end{tikzpicture}
\end{center}
is commutative. Then $\phi$ is \textit{a morphism of quasi-coherent $\bd{M}$-sheaves on $(X,a)$}. We denote by $\Qcoh_{\bd{M}}(X)$ the category of quasi-coherent $\bd{M}$-sheaves and call it \textit{the category of quasi-coherent $\bd{M}$-sheaves on $(X,a)$}.
\end{definition}
\noindent
Our goal in this section is to explain somewhat nonintuitive notion of quasi-coherent $\bd{M}$-sheaf on a $k$-scheme $X$ equipped with action of $\bd{M}$. For this we use the machinery of fibered categories. We fix a fibered category $p:\cE \ra \cB$ . If $f:X\ra Y$ and $\eta$ is an object of $p^{-1}(Y)$, then we denote by $\tilde{f}_{\eta}:f^*\eta\ra \eta$ a pullback of $\eta$. That is the square
\begin{center}
\begin{tikzpicture}
[description/.style={fill=white,inner sep=2pt}]
\matrix (m) [matrix of math nodes, row sep=3em, column sep=4em,text height=1.5ex, text depth=0.25ex] 
{ f^*\eta      &  \eta      \\
  X        &  Y         \\} ;
\path[->,line width=1.0pt,font=\scriptsize]
(m-1-1) edge node[above] {$ \widetilde{f}_{\eta} $} (m-1-2)
(m-2-1) edge node[below] {$ f $} (m-2-2);
\path[|->,line width=1.0pt,font=\scriptsize]
(m-1-1) edge node[left]  {$  $} (m-2-1)
(m-1-2) edge node[right] {$  $} (m-2-2);
\end{tikzpicture}
\end{center}
is cartesian. Using some choice of pullback we obtain a functor $f^*:p^{-1}(Y)\ra p^{-1}(X)$. We start with the following observation.

\begin{remark}\label{remark:unique_identification}
Consider morphisms $f_1,f_2,g_1,g_2$ in $\cB$ such that $g_1\cdot f_1 = g_2\cdot f_2$ with $\mathrm{cod}(g_1) = Y = \mathrm{cod}(g_2)$. For every object $\eta$ in $p^{-1}(Y)$ we have a unique identification $f_1^*g_1^*\eta \cong f_2^*g_2^*\eta$ such that the square
\begin{center}
\begin{tikzpicture}
[description/.style={fill=white,inner sep=2pt}]
\matrix (m) [matrix of math nodes, row sep=3em, column sep=4em,text height=1.5ex, text depth=0.25ex] 
{ f_2^*g_2^*\eta \cong f_1^*g_1^*\eta &  g_1^*\eta                \\
  g_2^*\eta                       &  \eta           \\} ;
\path[->,line width=1.0pt,font=\scriptsize]
(m-1-1) edge node[above] {$ \widetilde{f_1}_{g_1^*\eta} $} (m-1-2)
(m-2-1) edge node[below] {$ \widetilde{g_2}_{\eta} $} (m-2-2)
(m-1-1) edge node[left]  {$ \widetilde{f_2}_{g_2^*\eta} $} (m-2-1)
(m-1-2) edge node[right] {$ \widetilde{g_2}_{\eta} $} (m-2-2);
\end{tikzpicture}
\end{center}
is commutative.
\end{remark}
\noindent
Now we have the following result.

\begin{fact}\label{fact:identification_of_product_and_pullback_along_the_projection}
Let $X,\bd{M}$ be objects of $\cB$ and let $\xi$ be an object of $\cE$ in $p^{-1}(X)$. Assume that the cartesian product of $X$ and $\bd{M}$ exists in $\cB$ and denote by $\pi:\bd{M}\times X\ra X$ the projection. Then there exists a unique morphism (depicted by dotted arrow) such that the diagram
\begin{center}
\begin{tikzpicture}
[description/.style={fill=white,inner sep=2pt}]
\matrix (m) [matrix of math nodes, row sep=3em, column sep=4em,text height=1.5ex, text depth=0.25ex] 
{ h^{\cE}_{\pi^*\xi}         &                                          &                   \\
                             & h^{\cB}_\bd{M}\cdot p\times h^{\cE}_\xi       &  h^{\cE}_\xi      \\
  h^{\cB}_{\bd{M}\times X}\cdot p & h^{\cB}_\bd{M}\cdot p\times h^{\cB}_X\cdot p  &  h^{\cB}_X\cdot p \\} ;
\path[->,line width=1.0pt,font=\scriptsize]
(m-1-1) edge[bend left] node[above] {$ h^{\cE}_{\widetilde{\pi}_{\xi}} $} (m-2-3)
(m-1-1) edge node[left] {$ p_{\mathrm{hom}}  $} (m-3-1)
(m-3-1) edge node[below] {$ = $} (m-3-2)
(m-2-2) edge node[above]  {$ pr_{h^{\cE}_{\xi}} $} (m-2-3)
(m-3-2) edge node[below]  {$ pr_{h^{\cB}_X\cdot p} $} (m-3-3)
(m-2-2) edge node[left]  {$ 1_{h^{\cB}_\bd{M}\cdot p}\times p_{\mathrm{hom}} $} (m-3-2)
(m-2-3) edge node[right] {$ p_{\mathrm{hom}} $} (m-3-3)
(m-3-1) edge[bend right] node[below] {$ \left(h^{\cB}_{\pi}\right)_p $} (m-3-3);
\path[densely dotted,->,line width=1.0pt,font=\scriptsize]
(m-1-1) edge node[above] {$ $} (m-2-2);
\end{tikzpicture}
\end{center}
is commutative, where $pr_{h^{\cE}_{\xi}}$ and $pr_{h^{\cB}_X\cdot p}$ are projections. Moreover, this morphism is an isomorphism.
\end{fact}
\begin{proof}
This is a consequence of the fact that both squares
\begin{center}
\begin{tikzpicture}
[description/.style={fill=white,inner sep=2pt}]
\matrix (m) [matrix of math nodes, row sep=3em, column sep=3em,text height=1.5ex, text depth=0.25ex] 
{ h^{\cB}_{\bd{M}}\cdot p \times h_{\xi}^{\cE}                &  h_{\xi}^{\cE}      &   h^{\cE}_{\pi^*\xi}              &  h^{\cE}_{\xi}            \\
  h^{\cB}_\bd{M}\cdot p   \times h_{X}^{\cB}\cdot p           &  h_{X}^{\cB}\cdot p &   h^{\cB}_{\bd{M}\times X}\cdot p &  h^{\cB}_{X}\cdot p              \\} ;
\path[->,line width=1.0pt,font=\scriptsize]
(m-1-1) edge node[above] {$ pr_{h^{\cE}_{\xi}} $} (m-1-2)
(m-2-1) edge node[below] {$ pr_{h^{\cB}_{X}\cdot p} $} (m-2-2)
(m-1-1) edge node[left]  {$ 1_{h^{\cB}_\bd{M}\cdot p}\times p_{\mathrm{hom}} $} (m-2-1)
(m-1-2) edge node[right] {$ p_{\mathrm{hom}} $} (m-2-2)
(m-1-3) edge node[above] {$ h^{\cE}_{\widetilde{\pi}_{\xi}} $} (m-1-4)
(m-2-3) edge node[below] {$ \left(h^{\cB}_{\pi}\right)_p $} (m-2-4)
(m-1-3) edge node[left]  {$ p_{\mathrm{hom}} $} (m-2-3)
(m-1-4) edge node[right] {$ p_{\mathrm{hom}} $} (m-2-4);;
\end{tikzpicture}
\end{center}
are cartesian.
\end{proof}
\noindent
Fix now two objects $\bd{M}$ and $X$ on $\cB$ such that product of $\bd{M}$ and $X$ exists. Denote by $\pi: \bd{M} \times X \ra X$ the projection on $X$. Let $a:\bd{M} \times X\ra X$ be a morphism in $\cB$, let $\xi$ be an object in $p^{-1}(X)$ and let $\sigma:h^{\cB}_\bd{M} \cdot p \times h^{\cE}_{\xi}\ra h^{\cE}_{\xi}$ be a morphism of presheaves on $\cE$. Suppose that the square
\begin{center}
\begin{tikzpicture}
[description/.style={fill=white,inner sep=2pt}]
\matrix (m) [matrix of math nodes, row sep=3em, column sep=3em,text height=1.5ex, text depth=0.25ex] 
{ h^{\cB}_{{\bd{M}}}\cdot p \times h_{\xi}^{\cE}                &  h_{\xi}^{\cE}                \\
  h_{\bd{M}\times X}^{\cB}\cdot p           &  h_{X}^{\cB}\cdot p           \\} ;
\path[->,line width=1.0pt,font=\scriptsize]
(m-1-1) edge node[above] {$ \sigma $} (m-1-2)
(m-2-1) edge node[below] {$ \left(h^{\cB}_a\right)_p $} (m-2-2)
(m-1-1) edge node[left]  {$ p_{\mathrm{hom}} $} (m-2-1)
(m-1-2) edge node[right] {$ p_{\mathrm{hom}} $} (m-2-2);
\end{tikzpicture}
\end{center}
is commutative. According to Fact \ref{fact:identification_of_product_and_pullback_along_the_projection} we deduce that $\sigma$ is representable by some morphism $\alpha^{\sigma}:\pi^*\xi\ra \xi$ of $\cE$. By universal property of cartesian square
\begin{center}
\begin{tikzpicture}
[description/.style={fill=white,inner sep=2pt}]
\matrix (m) [matrix of math nodes, row sep=3em, column sep=4em,text height=1.5ex, text depth=0.25ex] 
{ a^*\xi      &  \xi      \\
  \bd{M}\times X        &  X         \\} ;
\path[->,line width=1.0pt,font=\scriptsize]
(m-1-1) edge node[above] {$ \widetilde{a}_{\xi} $} (m-1-2)
(m-2-1) edge node[below] {$ a $} (m-2-2);
\path[|->,line width=1.0pt,font=\scriptsize]
(m-1-1) edge node[left]  {$  $} (m-2-1)
(m-1-2) edge node[right] {$  $} (m-2-2);
\end{tikzpicture}
\end{center}
we deduce that there exists a unique morphism $\tau^{\sigma}:\pi^*\xi\ra a^*\xi$ in $p^{-1}\left(\bd{M}\times X\right)$ such that $\alpha^{\sigma} = \widetilde{a}_{\xi}\cdot \tau^{\sigma}$. Using this notation and Fact \ref{fact:identification_of_product_and_pullback_along_the_projection} we can now state the following result.

\begin{proposition}\label{proposition:equivalent_description_of_equivariance_in_representable_case}
Let $\bd{M}$ be a monoid object in $\cB$ and let $X$ be an object of $\cB$ equipped with an action $a:\bd{M}\times X\ra X$ of $\bd{M}$ on $X$. Denote by $\pi:\bd{M}\times X\ra X$ the projection on $X$. Consider an object $\xi$ in $p^{-1}(X)$ and let $\sigma:h^{\cB}_\bd{M} \cdot p \times h^{\cE}_{\xi}\ra h^{\cE}_{\xi}$ be a morphism of presheaves on $\cE$. Suppose that the square
\begin{center}
\begin{tikzpicture}
[description/.style={fill=white,inner sep=2pt}]
\matrix (m) [matrix of math nodes, row sep=3em, column sep=3em,text height=1.5ex, text depth=0.25ex] 
{ h^{\cB}_{{\bd{M}}}\cdot p \times h_{\xi}^{\cE}                &  h_{\xi}^{\cE}                \\
  h_{\bd{M}\times X}^{\cB}\cdot p           &  h_{X}^{\cB}\cdot p           \\} ;
\path[->,line width=1.0pt,font=\scriptsize]
(m-1-1) edge node[above] {$ \sigma $} (m-1-2)
(m-2-1) edge node[below] {$ \left(h^{\cB}_a\right)_p $} (m-2-2)
(m-1-1) edge node[left]  {$ p_{\mathrm{hom}} $} (m-2-1)
(m-1-2) edge node[right] {$ p_{\mathrm{hom}} $} (m-2-2);
\end{tikzpicture}
\end{center}
is commutative. Then the following assertions are equivalent.
\begin{enumerate}[label= \emph{\textbf{(\roman*)}}, leftmargin=3.0em]
\item $\sigma$ is an action of monoid presheaf $h^{\cB}_{\bd{M}}\cdot p$ on a presheaf $h^{\cE}_{\xi}$.
\item Morphism $\tau^{\sigma}$ satisfies (up to identifications described in Remark \ref{remark:unique_identification}) the identities
$$(\mu \times 1_X)^*\tau^{\sigma} = (1_{\bd{M}} \times  a)^*\tau^{\sigma} \cdot \pi_{2,3}^*\tau^{\sigma},\,\langle e, 1_{X} \rangle^*\tau^{\sigma} = 1_{\xi}$$
where $\mu: \bd{M} \times \bd{M} \ra \bd{M}$ is the multiplication on $\bd{M}$, $\pi_{2,3}:\bd{M}\times  \bd{M} \times  X \ra \bd{M} \times X$ is the projection on last two factors and $e:\bd{1}\ra \bd{M}$ is the unit of $\bd{M}$.
\end{enumerate}
\end{proposition}
\begin{proof}
Our first goal is to prove that
$$\sigma \cdot \left(1_{h^{\cB}_{\bd{M}}\cdot p} \times \sigma \right) = \sigma \cdot \left(1_{h^{\cB}_{\mu}\cdot p} \times 1_{h^{\cE}_{\xi}} \right)$$
if and only if
$$\left(1_{\bd{M}}\times a\right)^*\tau^{\sigma}\cdot \pi^*_{23}\tau^\sigma = \left(\mu\times 1_X\right)^*\tau^{\sigma}$$
First note that the commutative square of presheaves
\begin{center}
\begin{tikzpicture}
[description/.style={fill=white,inner sep=2pt}]
\matrix (m) [matrix of math nodes, row sep=3em, column sep=4em,text height=1.5ex, text depth=0.25ex] 
{ h^{\cB}_{\bd{M}}\cdot p \times  h_{a^*\xi}^{\cE} &  h^{\cB}_{\bd{M}}\cdot p \times  h_{\xi}^{\cE}   \\
 h^{\cB}_{\bd{M}\times \bd{M}\times X}\cdot p & h^{\cB}_{\bd{M}\times X}\cdot p  \\};
\path[->,line width=1.0pt,font=\scriptsize]
(m-1-1) edge node[above] {$ 1_{h^{\cB}_{\bd{M}}\cdot p} \times h^{\cE}_{\widetilde{a}_{\xi}} $} (m-1-2)
(m-2-1) edge node[below] {$ h^{\cB}_{1_{\bd{M}}\times a} $} (m-2-2)
(m-1-1) edge node[left]  {$ p_{\mathrm{hom}} $} (m-2-1)
(m-1-2) edge node[right] {$ p_{\mathrm{hom}} $} (m-2-2);
\end{tikzpicture}
\end{center}
on $\cE$ is cartesian. Next according to Fact \ref{fact:identification_of_product_and_pullback_along_the_projection} we infer that projections $$pr_{h^{\cE}_{a^*\xi}}:h^{\cB}_{\bd{M}}\cdot p \times  h_{a^*\xi}^{\cE} \ra h_{a^*\xi}^{\cE},\,pr_{h^{\cE}_{\xi}}:h^{\cB}_{\bd{M}}\cdot p \times  h_{\xi}^{\cE} \ra h_{\xi}^{\cE}$$
are representable by morphisms $\widetilde{\pi_{23}}_{a^*\xi}:\pi^*_{23}a^*\xi \ra a^*\xi,\,\widetilde{\pi}_{\xi}:\pi^*\xi\ra \xi$ in $\cE$, respectively. Thus $1_{h^{\cB}_{\bd{M}}\cdot p} \times h^{\cE}_{\widetilde{a}_{\xi}}$ is representable by a cartesian morphism
\begin{center}
\begin{tikzpicture}
[description/.style={fill=white,inner sep=2pt}]
\matrix (m) [matrix of math nodes, row sep=3em, column sep=4em,text height=1.5ex, text depth=0.25ex] 
{ \pi^*_{23}a^*\xi & \left(1_{\bd{M}}\times a \right)^*\pi^*\xi & \pi^*\xi   \\};
\path[->,line width=1.0pt,font=\scriptsize]
(m-1-1) edge node[above] {$ \cong $} (m-1-2)
(m-1-2) edge node[above] {$ \widetilde{\left(1_{\bd{M}}\times a\right)}_{\pi^*\xi} $} (m-1-3);
\end{tikzpicture}
\end{center}
where $\cong$ is the identification described in Remark \ref{remark:unique_identification}. Since we have equality
$$\sigma \cdot \left(1_{h^{\cB}_{\bd{M}}\cdot p} \times \sigma \right) = h^{\cE}_{\widetilde{a}_{\xi}}\cdot h^{\cE}_{\tau^{\sigma}}\cdot \left(1_{h^{\cB}_{\bd{M}}\cdot p} \times h^{\cE}_{\widetilde{a}_{\xi}} \right) \cdot \left(1_{h^{\cB}_{\bd{M}}\cdot p} \times h^{\cE}_{\tau^{\sigma}} \right)$$
we derive that $\sigma \cdot \left(1_{h^{\cB}_{\bd{M}}\cdot p} \times \sigma \right)$ is representable (again up to identifications of Remark \ref{remark:unique_identification}) by a morphism
$$\widetilde{a}_{\xi}\cdot \tau^{\sigma}\cdot \widetilde{\left(1_{\bd{M}}\times a\right)}_{\pi^*\xi}\cdot \pi^*_{23}\tau^\sigma = \widetilde{a}_{\xi} \cdot \widetilde{\left(1_{\bd{M}}\times a\right)}_{a^*\xi}\cdot \left(1_{\bd{M}}\times a\right)^*\tau^{\sigma}\cdot \pi^*_{23}\tau^\sigma$$
in $\cE$. Next note that the square of presheaves on $\cE$
\begin{center}
\begin{tikzpicture}
[description/.style={fill=white,inner sep=2pt}]
\matrix (m) [matrix of math nodes, row sep=3em, column sep=4em,text height=1.5ex, text depth=0.25ex] 
{ h^{\cB}_{\bd{M}}\cdot p \times h^{\cB}_{\bd{M}}\cdot p  \times h_{\xi}^{\cE} &  h^{\cB}_{{\bd{M}}\cdot p}\times h_{\xi}^{\cE}                \\
h^{\cB}_{\bd{M}\times \bd{M}\times X} \cdot p &  h^{\cB}_{{\bd{M}\times X}\cdot p}           \\} ;
\path[->,line width=1.0pt,font=\scriptsize]
(m-1-1) edge node[above] {$ h^{\cB}_{\mu}\cdot p \times 1_{h^{\cE}_{\xi}} $} (m-1-2)
(m-2-1) edge node[below] {$ \left(h^{\cB}_{\mu \times 1_X}\right)_p $} (m-2-2)
(m-1-1) edge node[left]  {$ \textbf{can}\times p_{\mathrm{hom}} $} (m-2-1)
(m-1-2) edge node[right] {$ p_{\mathrm{hom}} $} (m-2-2);
\end{tikzpicture}
\end{center}
is cartesian. According to Fact \ref{fact:identification_of_product_and_pullback_along_the_projection} we infer that projections $$pr_{h^{\cB}_{\bd{M}}\cdot p  \times h_{\xi}^{\cE}}:h^{\cB}_{\bd{M}}\cdot p \times h^{\cB}_{\bd{M}}\cdot p  \times h_{\xi}^{\cE} \ra h^{\cB}_{\bd{M}}\cdot p  \times h_{\xi}^{\cE},\,pr_{h^{\cE}_{\xi}}:h^{\cB}_{\bd{M}}\cdot p  \times h_{\xi}^{\cE} \ra h_{\xi}^{\cE}$$
are representable by morphisms $\widetilde{\pi_{23}}_{\pi^*\xi}:\pi^*_{23}\pi^*\xi \ra \pi^*\xi,\,\widetilde{\pi}_{\xi}:\pi^*\xi\ra \xi$ in $\cE$, respectively. Thus $h^{\cB}_{\mu}\cdot p \times 1_{h^{\cE}_{\xi}}$ is representable by a cartesian morphism
\begin{center}
\begin{tikzpicture}
[description/.style={fill=white,inner sep=2pt}]
\matrix (m) [matrix of math nodes, row sep=3em, column sep=4em,text height=1.5ex, text depth=0.25ex] 
{ \pi^*_{23}\pi^*\xi & (\mu \times 1_X)^*\pi^*\xi & \pi^*\xi   \\};
\path[->,line width=1.0pt,font=\scriptsize]
(m-1-1) edge node[above] {$ \cong $} (m-1-2)
(m-1-2) edge node[above] {$ \widetilde{\left(\mu\times 1_X\right)}_{\pi^*\xi} $} (m-1-3);
\end{tikzpicture}
\end{center}
where $\cong$ is the identification described in Remark \ref{remark:unique_identification}. Since we have equality
$$\sigma \cdot \left(1_{h^{\cB}_{\mu}\cdot p} \times 1_{h^{\cE}_{\xi}} \right) = h^{\cE}_{\widetilde{a}_{\xi}}\cdot h^{\cE}_{\tau^{\sigma}}\cdot \left(1_{h^{\cB}_{\mu}\cdot p} \times 1_{h^{\cE}_{\xi}} \right)$$
we derive that $\sigma \cdot \left(1_{h^{\cB}_{\mu}\cdot p} \times 1_{h^{\cE}_{\xi}} \right)$ is representable (again up to identifications of Remark \ref{remark:unique_identification}) by a morphism
$$\widetilde{a}_{\xi}\cdot \tau^{\sigma}\cdot \widetilde{\left(\mu\times 1_X\right)}_{\pi^*\xi} = \widetilde{a}_{\xi}\cdot \widetilde{\left(\mu\times 1_X\right)}_{a^*\xi}\cdot \left(\mu\times 1_X\right)^*\tau^{\sigma}$$
We deduce that
$$\sigma \cdot \left(1_{h^{\cB}_{\bd{M}}\cdot p} \times \sigma \right) = \sigma \cdot \left(1_{h^{\cB}_{\mu}\cdot p} \times 1_{h^{\cE}_{\xi}} \right)$$
if and only if
$$\widetilde{a}_{\xi} \cdot \widetilde{\left(1_{\bd{M}}\times a\right)}_{a^*\xi}\cdot \left(1_{\bd{M}}\times a\right)^*\tau^{\sigma}\cdot \pi^*_{23}\tau^\sigma =  \widetilde{a}_{\xi}\cdot \widetilde{\left(\mu\times 1_X\right)}_{a^*\xi}\cdot \left(\mu\times 1_X\right)^*\tau^{\sigma}$$
Since $a\cdot (1_{\bd{M}}\times a) = a\cdot (\mu \times 1_X)$ and according to Remark \ref{remark:unique_identification}, we have canonical identification $\widetilde{a}_{\xi} \cdot \widetilde{\left(1_{\bd{M}}\times a\right)}_{a^*\xi} = \widetilde{a}_{\xi}\cdot \widetilde{\left(\mu\times 1_X\right)}_{a^*\xi}$. Therefore, we deduce that the formula above holds if and only if
$$\left(1_{\bd{M}}\times a\right)^*\tau^{\sigma}\cdot \pi^*_{23}\tau^\sigma = \left(\mu\times 1_X\right)^*\tau^{\sigma}$$
This proves our first claim. Now it suffices to prove that
$$\sigma \cdot \langle h^{\cB}_{e}\cdot p, 1_{h^{\cE}_{\xi}} \rangle = 1_{h^{\cE}_{\xi}}$$
if and only if $\langle e, 1_X\rangle^*\tau^{\sigma} = 1_{\xi}$. Note that the square of presheaves on $\cE$
\begin{center}
\begin{tikzpicture}
[description/.style={fill=white,inner sep=2pt}]
\matrix (m) [matrix of math nodes, row sep=3em, column sep=4em,text height=1.5ex, text depth=0.25ex] 
{ h_{\xi}^{\cE}       &  h^{\cB}_{{\bd{M}}\cdot p}\times h_{\xi}^{\cE}                \\
  h^{\cB}_{X} \cdot p &  h^{\cB}_{{\bd{M}\times X}\cdot p}           \\} ;
\path[->,line width=1.0pt,font=\scriptsize]
(m-1-1) edge node[above] {$ \langle h^{\cB}_{e}\cdot p, 1_{h^{\cE}_{\xi}} \rangle $} (m-1-2)
(m-2-1) edge node[below] {$ \left(h^{\cB}_{\langle e, 1_X\rangle}\right)_p $} (m-2-2)
(m-1-1) edge node[left]  {$ p_{\mathrm{hom}} $} (m-2-1)
(m-1-2) edge node[right] {$ p_{\mathrm{hom}} $} (m-2-2);
\end{tikzpicture}
\end{center}
is cartesian. Thus according to Fact \ref{fact:identification_of_product_and_pullback_along_the_projection} we derive that $\langle h^{\cB}_{e}\cdot p, 1_{h^{\cE}_{\xi}} \rangle$ is representable by morphism
\begin{center}
\begin{tikzpicture}
[description/.style={fill=white,inner sep=2pt}]
\matrix (m) [matrix of math nodes, row sep=3em, column sep=4em,text height=1.5ex, text depth=0.25ex] 
{ \xi & \langle e, 1_X\rangle^*\pi^*\xi & \pi^*\xi   \\};
\path[->,line width=1.0pt,font=\scriptsize]
(m-1-1) edge node[above] {$ \cong $} (m-1-2)
(m-1-2) edge node[above] {$ \widetilde{\langle e, 1_X\rangle}_{\pi^*\xi} $} (m-1-3);
\end{tikzpicture}
\end{center}
where $\cong$ is the identification described in Remark \ref{remark:unique_identification}. Therefore, the morphism $\sigma \cdot \langle h^{\cB}_{e}\cdot p, 1_{h^{\cE}_{\xi}} \rangle$ is representable (up to identifications of Remark \ref{remark:unique_identification}) by
$$\widetilde{a}_{\xi}\cdot \tau^{\sigma}\cdot \widetilde{\langle e, 1_X\rangle}_{\pi^*\xi} = \widetilde{a}_{\xi}\cdot \widetilde{\langle e, 1_X\rangle}_{a^*\xi}\cdot \langle e, 1_X\rangle^*\tau^{\sigma} =  \langle e, 1_X\rangle^*\tau^{\sigma}$$
Thus
$$\sigma \cdot \langle h^{\cB}_{e}\cdot p, 1_{h^{\cE}_{\xi}} \rangle = 1_{h^{\cE}_{\xi}}$$
if and only if
$$\langle e, 1_X \rangle^*\tau^{\sigma} = 1_{\xi}$$
This finishes the proof.
\end{proof}

\begin{fact}\label{fact:morphisms_of_equivariant_objects}
Let $\bd{M},X$ be objects of $\cB$ such that the cartesian product of $\bd{M}$ and $X$ exist. Let $a:\bd{M}\times X\ra X$ be a morphism. Denote by $\pi:\bd{M}\times X\ra X$ the projection on $X$. Consider objects $\xi_1,\xi_2$ in $p^{-1}(X)$ and let $\sigma_1:h^{\cB}_\bd{M} \cdot p \times h^{\cE}_{\xi_1}\ra h^{\cE}_{\xi_1},\sigma_2:h^{\cB}_\bd{M} \cdot p \times h^{\cE}_{\xi_2} \ra h^{\cE}_{\xi_2}$ be morphisms of presheaves on $\cE$. Suppose that squares
\begin{center}
\begin{tikzpicture}
[description/.style={fill=white,inner sep=2pt}]
\matrix (m) [matrix of math nodes, row sep=3em, column sep=3em,text height=1.5ex, text depth=0.25ex] 
{ h^{\cB}_{{\bd{M}}}\cdot p \times h_{\xi_1}^{\cE} &  h_{\xi_1}^{\cE}      & h^{\cB}_{{\bd{M}}}\cdot p \times h_{\xi_2}^{\cE} & h_{\xi_2}^{\cE}     \\
  h_{\bd{M}\times X}^{\cB}\cdot p                &  h_{X}^{\cB}\cdot p     & h_{\bd{M}\times X}^{\cB}\cdot p                  &  h_{X}^{\cB}\cdot p \\} ;
\path[->,line width=1.0pt,font=\scriptsize]
(m-1-1) edge node[above] {$ \sigma_1 $} (m-1-2)
(m-2-1) edge node[below] {$ \left(h^{\cB}_a\right)_p $} (m-2-2)
(m-1-1) edge node[left]  {$ p_{\mathrm{hom}} $} (m-2-1)
(m-1-2) edge node[right] {$ p_{\mathrm{hom}} $} (m-2-2)

(m-1-3) edge node[above] {$ \sigma_2 $} (m-1-4)
(m-2-3) edge node[below] {$ \left(h^{\cB}_a\right)_p $} (m-2-4)
(m-1-3) edge node[left]  {$ p_{\mathrm{hom}} $} (m-2-3)
(m-1-4) edge node[right] {$ p_{\mathrm{hom}} $} (m-2-4);
\end{tikzpicture}
\end{center}
are commutative. Let $\phi:\xi_1\ra \xi_2$ be a morphism in $\cE$. Then the following assertions are equivalent.
\begin{enumerate}[label= \emph{\textbf{(\roman*)}}, leftmargin=3.0em]
\item The square
\begin{center}
\begin{tikzpicture}
[description/.style={fill=white,inner sep=2pt}]
\matrix (m) [matrix of math nodes, row sep=3em, column sep=3em,text height=1.5ex, text depth=0.25ex] 
{  h^{\cB}_{{\bd{M}}}\cdot p \times h_{\xi_1}^{\cE} &  h_{\xi_1}^{\cE}      \\
   h^{\cB}_{{\bd{M}}}\cdot p \times h_{\xi_2}^{\cE} &  h_{\xi_2}^{\cE}             \\} ;
\path[->,line width=1.0pt,font=\scriptsize]  
(m-1-1) edge node[above] {$ \sigma_1 $} (m-1-2)
(m-2-1) edge node[below] {$ \sigma_2 $} (m-2-2)
(m-1-1) edge node[left] {$ 1_{h^{\cB}_{\bd{M}}\times h^{\cE}_{\phi}} $} (m-2-1)
(m-1-2) edge node[right] {$ h^{\cE}_{\xi}  $} (m-2-2);
\end{tikzpicture}
\end{center}
is commutative.
\item The square
\begin{center}
\begin{tikzpicture}
[description/.style={fill=white,inner sep=2pt}]
\matrix (m) [matrix of math nodes, row sep=3em, column sep=3em,text height=1.5ex, text depth=0.25ex] 
{ \pi^*\xi_1 & a^*\xi_1     \\
  \pi^*\xi_2 & a^*\xi_2             \\} ;
\path[->,line width=1.0pt,font=\scriptsize]  
(m-1-1) edge node[above] {$ \tau^{\sigma_1} $} (m-1-2)
(m-2-1) edge node[below] {$ \tau^{\sigma_2} $} (m-2-2)
(m-1-1) edge node[left] {$ \pi^*\phi $} (m-2-1)
(m-1-2) edge node[right] {$ a^*\phi  $} (m-2-2);
\end{tikzpicture}
\end{center}
is commutative.
\end{enumerate}
\end{fact}
\begin{proof}
Note that up to identifications of Remark \ref{remark:unique_identification} and according to Fact \ref{fact:identification_of_product_and_pullback_along_the_projection}  morphism $h^{\cE}_{\phi}\cdot \sigma_1$ is representable by
$$\phi\cdot \alpha^{\sigma_1} = \phi\cdot \widetilde{a}_{\xi_1}\cdot \tau^{\sigma_1} = \widetilde{a}_{\xi_2}\cdot a^*\phi \cdot \tau^{\sigma_1}$$
and on the other hand morphism $\sigma_2\cdot \left(1_{h^{\cB}_{\bd{M}}\cdot p}\times h^{\cE}_{\phi}\right)$ is representable by
$$\alpha^{\sigma_2}\cdot \pi^*\phi = \widetilde{a}_{\xi_2}\cdot \tau^{\sigma_2}\cdot \pi^*\phi$$
Since $\widetilde{a}_{\xi_2}$ is cartesian with respect to $p$, we derive that
$$h^{\cE}_{\phi}\cdot \sigma_1 = \sigma_2\cdot \left(1_{h^{\cB}_{\bd{M}}\cdot p}\times h^{\cE}_{\phi}\right)$$
if and only if
$$a^*\phi \cdot \tau^{\sigma_1} = \tau^{\sigma_2}\cdot \pi^*\phi$$
This proves the assertion.
\end{proof}
\noindent
Guided by these two results we formulate a general notion of equivariant object in a fibered category.

\begin{definition}
Let $M:\cB^{\mathrm{op}}\ra \Mon$ be a presheaf of monoids on $\cB$ and assume that for some object $X$ of $\cB$ the presheaf $h^{\cB}_X$ admits an action of $M$ given by the morphism $\alpha:M\times h^{\cB}_X\ra h^{\cB}_X$. Consider an object $\xi$ in $p^{-1}(X)$. Suppose that there is an action $\sigma:M\cdot p \times h^{\cE}_\xi\ra h^{\cE}_\xi$ of a monoid presheaf $M\cdot p$ on $h^{\cE}_\xi$ such that the square
\begin{center}
\begin{tikzpicture}
[description/.style={fill=white,inner sep=2pt}]
\matrix (m) [matrix of math nodes, row sep=3em, column sep=4em,text height=1.5ex, text depth=0.25ex] 
{ M\cdot p\times h_{\xi}^{\cE} &  h_{\xi}^{\cE}    \\
  M\cdot p\times h^{\cB}_X\cdot p    &  h^{\cB}_{X}\cdot p           \\} ;
\path[->,line width=1.0pt,font=\scriptsize]
(m-1-1) edge node[above] {$ \sigma  $} (m-1-2)
(m-2-1) edge node[below] {$ \alpha_p $} (m-2-2)
(m-1-1) edge node[left]  {$1_{M\cdot p}\times p_{\mathrm{hom}} $} (m-2-1)
(m-1-2) edge node[right] {$p_{\mathrm{hom}}  $} (m-2-2);
\end{tikzpicture}
\end{center}
is commutative. Then a pair $(\xi,\sigma)$ is called \textit{an $M$-equivariant object over $\alpha$}.
\end{definition}

\begin{definition}
Let $M:\cB^{\mathrm{op}}\ra \Mon$ be a presheaf of monoids on $\cB$ and assume that for some object $X$ of $\cB$ the presheaf $h^{\cB}_X$ admits an action of $M$ given by the morphism $\alpha:M\times h^{\cB}_X\ra h^{\cB}_X$. Suppose that $(\xi_1,\sigma_1)$ and $(\xi_2,\sigma_2)$ are objects over $X$ with $M$-equivariant structures. Then a morphism $\phi:\xi_1\ra \xi_2$ in $\cE$ is \textit{$M$-equivariant} if the square
\begin{center}
\begin{tikzpicture}
[description/.style={fill=white,inner sep=2pt}]
\matrix (m) [matrix of math nodes, row sep=3em, column sep=4em,text height=1.5ex, text depth=0.25ex] 
{ M\cdot p\times h_{\xi_1}^{\cE} &  h_{\xi_1}^{\cE}    \\
  M\cdot p\times h_{\xi_2}^{\cE} &  h_{\xi_2}^{\cE}           \\} ;
\path[->,line width=1.0pt,font=\scriptsize]
(m-1-1) edge node[above] {$ \sigma_1  $} (m-1-2)
(m-2-1) edge node[below] {$ \sigma_2  $} (m-2-2)
(m-1-1) edge node[left]  {$ 1_{M\cdot p}\times h^{\cE}_\phi $} (m-2-1)
(m-1-2) edge node[right] {$ \phi  $} (m-2-2);
\end{tikzpicture}
\end{center}
is commutative.
\end{definition}
\noindent
We denote the category of $M$-equivariant objects over $\alpha$ with respect to the fibered category $p:\cE\ra \cB$ by $p^{-1}(X)_{M}$.\\
Now we can apply Proposition \ref{proposition:equivalent_description_of_equivariance_in_representable_case} and Fact \ref{fact:morphisms_of_equivariant_objects} to the fibered category $\Qcoh\ra \Sch_k$.

\begin{corollary}\label{corollary:isomorphism_between_equivariant_quasi_coherent_sheaves_and_equivariant_objects_in_fibered_category_of_quasi_coherent_sheaves}
Suppose that $\bd{M}$ is a monoid $k$-scheme that acts on a $k$-scheme $X$ through morphism $a:\bd{M}\times_k X\ra X$ of $k$-schemes. Then the category $\Qcoh(X)_{\bd{M}}$ is isomorphic to the category of $h^{\Sch_k}_{\bd{M}}$-objects over $h^{\Sch_k}_a$ with respect to the fibered category $\Qcoh\ra \Sch_k$.
\end{corollary}
\noindent
Moreover, we have the following general result.

\begin{corollary}\label{corollary:isomorphism_between_equivariant_objects_and_objects_with_action}
Let $\cB$ be a category with all finite limits. Suppose that $\bd{M}$ is a monoid object in $\cB$ that acts on an object $X$ of $\cB$ via $a:\bd{M}\times X\ra X$. Then the category of $h^{\cB}_{\bd{M}}$-objects over $h^{\cB}_a$ with respect to the fibered category $p_{\mathrm{Arr}}:\mathrm{Arr}(\cB)\ra \cB$ is isomorphic to the category of $\bd{M}$-equivariant morphisms $\pi:\widetilde{X}\ra X$ as objects and with 
\end{corollary}

\section{Equivariant sheaves of quasi-coherent algebras}
\noindent
In this section we fix a commutative ring $k$.    
Let $\bd{M}$ be a monoids scheme and let $X$ be a $k$-scheme together with an action $a:\bd{M}\times_kX\ra X$ of $\bd{M}$.


\section{Example: Principal Bundles}
\noindent
We devote this section to another important example of a fibered category. We fix a category with finite limits $\cB$ and a monoid object $\bd{M}$ of $\cB$. We denote by $\mu:\bd{M}\times \bd{M}\ra \bd{M}$ and $e:\bd{1}\ra \bd{M}$ the multiplication and unit of $\bd{M}$, respectively.

\begin{definition}
Let $\cP$ be an object of $\cB$ equipped with an action of $\bd{M}$, let $T$ be an object of $\cB$ with trivial action of $\bd{M}$ and let $\pi:\cP\ra T$ be an $\bd{M}$-equivariant morphism with respect to these $\bd{M}$-actions. We say that $\bd{M}$-equivariant morphism $\pi$ is \textit{a trivial principal $\bd{M}$-bundle on $T$} if there exists an $\bd{M}$-equivariant isomorphism $\phi:\cP\ra \bd{M}\times T$ such that $\bd{M}\times T$ is equipped with an action of $\bd{M}$ given by $\mu\times 1_T$ and the triangle
\begin{center}
\begin{tikzpicture}
[description/.style={fill=white,inner sep=2pt}]
\matrix (m) [matrix of math nodes, row sep=2em, column sep=1em,text height=1.5ex, text depth=0.25ex] 
{  \cP &        & \bd{M}\times T  \\
          &T &  \\} ;
\path[->,line width=1.0pt,font=\scriptsize]
(m-1-1) edge node[above] {$\phi $} (m-1-3)
(m-1-1) edge node[below = 6pt, left = 1pt] {$ \pi  $} (m-2-2)
(m-1-3) edge node[below = 6pt, right = 1pt] {$ \mathrm{pr}_T $} (m-2-2);
\end{tikzpicture}
\end{center}
is commutative.
\end{definition}

\begin{definition}
Let $\cP$ be an object of $\cB$ equipped with an action of $\bd{M}$, let $T$ be an object of $\cB$ with trivial action of $\bd{M}$ and let $\pi:\cP\ra T$ be a $\bd{M}$-equivariant morphism with respect to these $\bd{M}$-actions. Consider a sieve $S$ on $T$. For every arrow $g:\widetilde{T}\ra T$ in $S$ we construct a cartesian square
\begin{center}
\begin{tikzpicture}
[description/.style={fill=white,inner sep=2pt}]
\matrix (m) [matrix of math nodes, row sep=2em, column sep=2em,text height=1.5ex, text depth=0.25ex] 
{ g^*\cP &  \cP    \\
  \widetilde{T} &  T           \\} ;
\path[->,line width=1.0pt,font=\scriptsize]
(m-1-1) edge node[above] {$   $} (m-1-2)
(m-2-1) edge node[below] {$ g $} (m-2-2)
(m-1-1) edge node[left] {$ \pi_g $} (m-2-1)
(m-1-2) edge node[right] {$ \pi $} (m-2-2);
\end{tikzpicture}
\end{center}
in $\cB$. We consider $g$ as an $\bd{M}$-equivariant morphism with respect to trivial $\bd{M}$-actions on $T$ and $\widetilde{T}$. Then there exists a unique action of $\bd{M}$ on $g^*\cP$ which makes $\pi_g$ into an $\bd{M}$-equivariant morphism in such a way that the square consists of objects of $\cB$ with $\bd{M}$-actions and $\bd{M}$-equivariant morphisms. Suppose that $\bd{M}$-equivariant morphism $\pi_g$ is a trivial principal $\bd{M}$-bundle on $\widetilde{T}$ for every $g$ in $S$. Then we say that \textit{$S$ trivializes $\pi$}.
\end{definition}
\noindent
In the remaining part of this section we fix a Grothendieck topology $\cJ$ on $\cB$.

\begin{definition}
Let $\cP$ be an object of $\cB$ equipped with an action of $\bd{M}$, let $T$ be an object of $\cB$ with trivial action of $\bd{M}$ and let $\pi:\cP \ra T$ be a $\bd{M}$-equivariant morphism with respect to these $\bd{M}$-actions. Suppose that there exists a covering sieve $S$ in $\cJ(T)$ that trivializes $\pi$. Then $\pi$ is called \textit{a principal $\bd{M}$-bundle with respect to $\cJ$}.
\end{definition}
\noindent
Now we define a category $\mathbb{B}\bd{M}$ that depends on the site $(\cB,\cJ)$. Its objects are principal $\bd{M}$-bundles with respect to $\cJ$ and if $\pi:\cP\ra T$ and $\psi:Q \ra Z$ are principal $\bd{M}$-bundles with respect to $\cJ$, then a morphism $\pi\ra \psi$ is a pair $(f,\phi)$ such that $f:T\ra Z$ and $\phi:\cP\ra Q$ are morphisms in $\cB$ such that $\phi$ is $\bd{M}$-equivariant and the square
\begin{center}
\begin{tikzpicture}
[description/.style={fill=white,inner sep=2pt}]
\matrix (m) [matrix of math nodes, row sep=2em, column sep=2em,text height=1.5ex, text depth=0.25ex] 
{ \cP &  Q           \\
  T   &  Z           \\} ;
\path[->,line width=1.0pt,font=\scriptsize]
(m-1-1) edge node[above] {$ \phi  $} (m-1-2)
(m-2-1) edge node[below] {$ f $} (m-2-2)
(m-1-1) edge node[left] {$\pi $} (m-2-1)
(m-1-2) edge node[right] {$  \psi $} (m-2-2);
\end{tikzpicture}
\end{center}
is commutative. We have a functor $p_{\bd{M},\cJ}:\mathbb{B}\bd{M}\ra \cB$ given by $p_{\bd{M},\cJ}\big((f,\phi)\big) = f$. Let $\psi:Q\ra Z$ be a principal $\bd{M}$-bundle with respect to $\cJ$ and let $f:T\ra Z$ be a morphism. Consider the cartesian square
\begin{center}
\begin{tikzpicture}
[description/.style={fill=white,inner sep=2pt}]
\matrix (m) [matrix of math nodes, row sep=2em, column sep=2em,text height=1.5ex, text depth=0.25ex] 
{ f^*Q &  Q           \\
  T   &  Z           \\} ;
\path[->,line width=1.0pt,font=\scriptsize]
(m-1-1) edge node[above] {$ \phi  $} (m-1-2)
(m-2-1) edge node[below] {$ f $} (m-2-2)
(m-1-1) edge node[left] {$\pi $} (m-2-1)
(m-1-2) edge node[right] {$  \psi $} (m-2-2);
\end{tikzpicture}
\end{center}
in $\cB$. Then by the universal property there exists a unique action of $\bd{M}$ on $f^*Q$ such that the square above consists of $\bd{M}$-equivariant morphisms ($T,Z$ are equipped with trivial $\bd{M}$-actions). Moreover, with respect to this action $\psi:f^*Q\ra T$ becomes a principal $\bd{M}$-bundle with respect to $\cJ$. Indeed, if $S$ is in $\cJ(Z)$ and $S$ trivializes $\psi$, then its pullback $f^*S$ trivializes $\pi$ and is an element of $\cJ(T)$ (by definition of a Grothendieck topology). This shows that $p_{\bd{M},\cJ}:\mathbb{B}\bd{M}\ra \cB$ is a fibered category. Moreover, we have a functor $\mathbb{B}\bd{M}\ra \mathrm{Arr}(\cB)$ that forgets about $\bd{M}$-actions. Hence there exists commutative triangle
\begin{center}
\begin{tikzpicture}
[description/.style={fill=white,inner sep=2pt}]
\matrix (m) [matrix of math nodes, row sep=2em, column sep=1em,text height=1.5ex, text depth=0.25ex] 
{  \mathbb{B}\bd{M} &        & \mathrm{Arr}(\cB)  \\
          &\cB &  \\} ;
\path[->,line width=1.0pt,font=\scriptsize]
(m-1-1) edge node[above] {$ $} (m-1-3)
(m-1-1) edge node[below = 6pt, left = 1pt] {$ p_{\bd{M},\cJ}  $} (m-2-2)
(m-1-3) edge node[below = 6pt, right = 1pt] {$ p_{\mathrm{Arr}} $} (m-2-2);
\end{tikzpicture}
\end{center}
According to Example \ref{example:the_fibered_category_of_arrows} and description of cartesian morphisms of $p_{\bd{M},\cJ}$ the functor $\mathbb{B}\bd{M}\ra \mathrm{Arr}(\cB)$ described above is a morphism of fibered categories.

\begin{definition}
$p_{\bd{M},\cJ}:\mathbb{B}\bd{M}\ra \cB$ is called \textit{the fibered category of principal $\bd{M}$-bundles on $(\cB,\cJ)$}.
\end{definition}
\noindent
From now suppose that $X$ is an object of $\cB$ equipped with an action $a:\bd{M}\times X\ra X$ of $\bd{M}$. We define a category $[X/\bd{M}]$ depending on $a$ and the site $(\cB,\cJ)$ as follows. Its objects are pairs $(\pi,\alpha)$ such that $\pi$ is a principal $\bd{M}$-bundle with respect to $\cJ$ and $\alpha$ is a $\bd{M}$-equivariant morphism. We depict them by diagrams
\begin{center}
\begin{tikzpicture}
[description/.style={fill=white,inner sep=2pt}]
\matrix (m) [matrix of math nodes, row sep=2em, column sep=2em,text height=1.5ex, text depth=0.25ex] 
{ \cP &  X           \\
  T   &             \\} ;
\path[->,line width=1.0pt,font=\scriptsize]
(m-1-1) edge node[above] {$ \alpha  $} (m-1-2)
(m-1-1) edge node[left] {$\pi $} (m-2-1);
\end{tikzpicture}
\end{center}
Suppose that $(\pi:\cP\ra T,\alpha:\cP\ra X)$ and $(\psi:Q\ra Z,\beta:Q\ra X)$ are two such objects. Then a morphism $(\pi,\alpha)\ra (\psi,\beta)$ is a morphism $(f,\phi):\pi\ra \psi$ in $\mathbb{B}\bd{M}$ such that $\alpha = \beta \cdot \phi$. We have a functor $[X/\bd{M}] \ra \mathbb{B}\bd{M}$ which sends $(\pi, \alpha)$ to $\pi$. We denote by $p_{a,\cJ}:[X/B]\ra \cB$ the composition of this functor $[X/\bd{M}] \ra \mathbb{B}\bd{M}$ with $p_{\bd{M},\cJ}:\mathbb{B}\bd{M}\ra \cB$. By description of cartesian morphisms of $p_{\bd{M},\cJ}$ we deduce that $p_{a,\cJ}$ is a fibered category. We have a commutative triangle
\begin{center}
\begin{tikzpicture}
[description/.style={fill=white,inner sep=2pt}]
\matrix (m) [matrix of math nodes, row sep=2em, column sep=1em,text height=1.5ex, text depth=0.25ex] 
{ [X/\bd{M}]  &        & \mathbb{B}\bd{M}  \\
          &\cB &  \\} ;
\path[->,line width=1.0pt,font=\scriptsize]
(m-1-1) edge node[above] {$  $} (m-1-3)
(m-1-1) edge node[below = 6pt, left = 1pt] {$ p_{a,\cJ}  $} (m-2-2)
(m-1-3) edge node[below = 6pt, right = 1pt] {$ p_{\bd{M},\cJ} $} (m-2-2);
\end{tikzpicture}
\end{center}
and the functor $[X/\bd{M}]\ra \mathbb{B}\bd{M}$ described above is a morphism of fibered categories. Note that if $\bd{1}$ is a terminal object of $\cB$ equipped with trivial action of $\bd{M}$, then we have a canonical isomorphism $[\bd{1}/\bd{M}] \cong \mathbb{B}\bd{M}$ of categories over $\cB$.

\begin{definition}
$p_{\bd{M},\cJ,X}:\mathbb{B}\bd{M}\ra \cB$ is called \textit{the quotient fibered category of $\bd{M}$-object $X$ on $(\cB,\cJ)$}.
\end{definition}
\noindent
Results below show that up to some mild assumptions on Grothendieck topology $\cJ$ fibered category $p_{a,\cJ}:[X/\bd{M}]\ra \cB$ encapsulates all essential information concerning action of $\bd{M}$ on $X$. We start with the following observation.

\begin{fact}\label{fact:functor_over_principal_bundles_is_morphism_of_quotient_fibered_categories}
Let $X,Y$ be objects of $\cB$ equipped with actions $a:\bd{M}\times X\ra X$ and $b:\bd{M}\times Y\ra Y$ of $\bd{M}$. Consider a functor $F:[X/\bd{M}]\ra [Y/\bd{M}]$ such that the triangle
\begin{center}
\begin{tikzpicture}
[description/.style={fill=white,inner sep=2pt}]
\matrix (m) [matrix of math nodes, row sep=2em, column sep=1em,text height=1.5ex, text depth=0.25ex] 
{ [X/\bd{M}]   &             & {}[Y/\bd{M}] \\
          & \mathbb{B}\bd{M} &  \\} ;
\path[->,line width=1.0pt,font=\scriptsize]
(m-1-1) edge node[above] {$ F $} (m-1-3)
(m-1-1) edge node[below = 6pt, left = 1pt] {$ $} (m-2-2)
(m-1-3) edge node[below = 6pt, right = 1pt] {$ $} (m-2-2);
\end{tikzpicture}
\end{center}
is commutative, where two other sides are canonical functors. Then $F$ is a morphism of fibered categories $p_{a,\cJ}$ and $p_{b,\cJ}$.
\end{fact}
\begin{proof}
The commutativity of the triangle implies that $F\cdot p_{b,\cJ} = p_{a,\cJ}$. Since a morphism in $[X/\bd{M}]$ is cartesian with respect to $p_{a,\cJ}$ if and only if its image under the canonical functor $[X/\bd{M}]\ra \mathbb{B}\bd{M}$ is cartesian with respect to $p_{\bd{M},\cJ}$ and the same holds for $p_{b,\cJ}$, we derive that $F$ sends cartesian morphisms of $p_{a,\cJ}$ to cartesian morphisms of $p_{b,\cJ}$. 
\end{proof}
\noindent
Let $X,Y$ be objects of $\cB$ equipped with actions $a:\bd{M}\times X\ra X$ and $b:\bd{M}\times Y\ra Y$ of $\bd{M}$. We denote the class of functors in Fact \ref{fact:functor_over_principal_bundles_is_morphism_of_quotient_fibered_categories} by $\Mor_{\mathbb{B}\bd{M}}\left([X/\bd{M}],[Y/\bd{M}]\right)$. We also denote (by abuse of notation) the class of $\bd{M}$-equivariant morphism $(X,a)\ra (Y,b)$ by $\Mor_{\bd{M}}\left(X,Y\right)$.


\begin{theorem}\label{theorem:equivariant_morphisms_can_described_by_fibered_categories}
Let $(\cB,\cJ)$ be a Grothendieck site and assume that representable presheaves on $\cB$ are separated with respect to $\cJ$. Let $X,Y$ be objects of $\cB$ equipped with $\bd{M}$-actions $a:\bd{M}\times X\ra X$ and $b:\bd{M}\times Y\ra Y$, respectively. Then there exists a bijection
$$\Mor_{\bd{M}}\left(X,Y\right)\cong \Mor_{\mathbb{B}\bd{M}}\big([X/\bd{M}],[Y/\bd{M}]\big)$$
that sends an $\bd{M}$-equivariant morphism $f$ to a functor $F:[X/\bd{M}]\ra [Y/\bd{M}]$ such that
\end{theorem}
\begin{proof}
Note that $(\bd{M}\times X,\mu\times 1_X)$ is an object of $\cB$ equipped with the action of $\bd{M}$. Next the projection $\pi:\bd{M}\times X\ra X$ can be considered as a $\bd{M}$-equivariant morphism from this $\bd{M}$-object to $X$ with the trivial action of $\bd{M}$. Since the square
\begin{center}
\begin{tikzpicture}
[description/.style={fill=white,inner sep=2pt}]
\matrix (m) [matrix of math nodes, row sep=3em, column sep=4em,text height=1.5ex, text depth=0.25ex] 
{ \bd{M}\times \bd{M}\times X &  \bd{M}\times X    \\
  \bd{M}\times X              &  X           \\} ;
\path[->,line width=1.0pt,font=\scriptsize]
(m-1-1) edge node[above] {$ 1_{\bd{M}} \times a $} (m-1-2)
(m-2-1) edge node[below] {$ a  $} (m-2-2)
(m-1-1) edge node[left]  {$ \mu\times 1_X $} (m-2-1)
(m-1-2) edge node[right] {$ a $} (m-2-2);
\end{tikzpicture}
\end{center}
is commutative, we derive that $a$ is $\bd{M}$-equivariant morphism $\left(\bd{M}\times X,\mu\times 1_X\right)\ra \left(X,a\right)$. This gives $(\mathrm{pr}_X,a)$ the structure of an object of $[X/\bd{M}]$. The functor $F$ sends it to some object of $[Y/\bd{M}]$. This object is necessarily of the form $(\mathrm{pr}_X,\alpha)$ for some $\bd{M}$-equivariant morphism $\alpha:\left(\bd{M}\times X,\mu\times 1_X\right) \ra \left(Y,b\right)$. Indeed, this follows from the fact that $F$ is over $\mathbb{B}\bd{M}$. We define $f = \alpha \cdot \langle e, 1_X\rangle$. Consider now some object $T$ of $\cB$ and the projection $\mathrm{pr}_T:\bd{M}\times T\ra T$ considered as a trivial principal $\bd{M}$-bundle. Let $(\mathrm{pr}_T,c)$ be an object of $[X/\bd{M}]$. Then $c$ is an $\bd{M}$-equivariant morphism $c:\left(\bd{M}\times T,\mu\times 1_T\right) \ra (X,a)$. Functor $F$ sends $(\mathrm{pr}_T,c)$ to some object $(\mathrm{pr}_T,\gamma)$. We claim that $\gamma = f \cdot c$. Let $\mathrm{pr}_{23}:\bd{M}\times \bd{M}\times T\ra \bd{M}\times T$ be the projection on the last two factors. There are diagrams
\begin{center}
\begin{tikzpicture}
[description/.style={fill=white,inner sep=2pt}]
\matrix (m) [matrix of math nodes, row sep=3em, column sep=3em,text height=1.5ex, text depth=0.25ex] 
{ \bd{M}\times \bd{M} \times T  &  \bd{M}\times T  & X    &   \bd{M}\times \bd{M}\times T     &  \bd{M}\times X  & X          \\
  \bd{M}\times T                &  T               &      &   \bd{M}\times T                  &  X               &          \\} ;
\path[->,line width=1.0pt,font=\scriptsize]
(m-1-1) edge node[above] {$ \mu\times 1_T $} (m-1-2)
(m-2-1) edge node[below] {$ \mathrm{pr}_T $} (m-2-2)
(m-1-1) edge node[left]  {$ \mathrm{pr}_{23} $} (m-2-1)
(m-1-2) edge node[right] {$ \mathrm{pr}_T $} (m-2-2)
(m-1-2) edge node[above] {$ c $} (m-1-3)

(m-1-4) edge node[above] {$ 1_{\bd{M}}\times c $} (m-1-5)
(m-2-4) edge node[below] {$ c $} (m-2-5)
(m-1-4) edge node[left]  {$ \mathrm{pr}_{23} $} (m-2-4)
(m-1-5) edge node[right] {$ \mathrm{pr}_T $} (m-2-5)
(m-1-5) edge node[above] {$ a $} (m-1-6);
\end{tikzpicture}
\end{center}
representing morphisms
$$(\mathrm{pr}_T,\mu\times 1_T):\left(\mathrm{pr}_{23},c\cdot (\mu\times 1_T)\right)\ra \left(\mathrm{pr}_T,c\right),\,(c,1_{\bd{M}}\times c):\left(\mathrm{pr}_{23},a\cdot (1_{\bd{M}}\times c)\right) \ra \left(\mathrm{pr}_X,a\right)$$
in $[X/\bd{M}]$. Moreover, $c$ is $\bd{M}$-equivariant $\left(\bd{M}\times T,\mu\times 1_T\right) \ra (X,a)$ and we derive that $c\cdot \left(\mu\times 1_T\right) = a\cdot \left(c\times 1_{\bd{M}}\right)$. Thus the morphisms in $[X/\bd{M}]$ described above have common domain. Since $F$ is over $\mathbb{B}\bd{M}$, we derive that their images under $F$ are 
\begin{center}
\begin{tikzpicture}
[description/.style={fill=white,inner sep=2pt}]
\matrix (m) [matrix of math nodes, row sep=3em, column sep=3em,text height=1.5ex, text depth=0.25ex] 
{ \bd{M}\times \bd{M} \times T  &  \bd{M}\times T  & X    &   \bd{M}\times \bd{M}\times T     &  \bd{M}\times X  & X          \\
  \bd{M}\times T                &  T               &      &   \bd{M}\times T                  &  X               &          \\} ;
\path[->,line width=1.0pt,font=\scriptsize]
(m-1-1) edge node[above] {$ \mu\times 1_T $} (m-1-2)
(m-2-1) edge node[below] {$ \mathrm{pr}_T $} (m-2-2)
(m-1-1) edge node[left]  {$ \mathrm{pr}_{23} $} (m-2-1)
(m-1-2) edge node[right] {$ \mathrm{pr}_T $} (m-2-2)
(m-1-2) edge node[above] {$ \gamma $} (m-1-3)

(m-1-4) edge node[above] {$ 1_{\bd{M}}\times c $} (m-1-5)
(m-2-4) edge node[below] {$ c $} (m-2-5)
(m-1-4) edge node[left]  {$ \mathrm{pr}_{23} $} (m-2-4)
(m-1-5) edge node[right] {$ \mathrm{pr}_X $} (m-2-5)
(m-1-5) edge node[above] {$ \alpha $} (m-1-6);
\end{tikzpicture}
\end{center}
This implies that $\gamma \cdot (\mu\times 1_T) = \alpha \cdot (1_{\bd{M}}\times c)$. We deduce that
$$\gamma \cdot (\mu \times 1_T)\cdot \langle e, 1_{\bd{M}\times X}\rangle = \alpha \cdot (1_{\bd{M}}\times c) \cdot  \langle e, 1_{\bd{M}\times X}\rangle = \alpha \cdot \langle e,1_X\rangle \cdot c = f\cdot c$$
and the claim is proved. We apply this to $\alpha$ to derive that $\alpha = f\cdot a$. Next recall that $\alpha \cdot \left(\mu \times 1_X\right) = b\cdot \left(1_{\bd{M}} \times \alpha \right)$ because $\alpha$ is an $\bd{M}$-equivariant morphism $\left(\bd{M}\times X, \mu\times 1_X\right)\ra (Y,b)$. Thus
$$b\cdot \left(1_{\bd{M}}\times f\right) = b\cdot \left(1_{\bd{M}}\times \alpha\right)\cdot \left(1_{\bd{M}}\times \langle e, 1_X\rangle \right) = \alpha \cdot \left(\mu \times 1_X\right)\cdot \left(1_{\bd{M}}\times \langle e, 1_X\rangle \right) = \alpha$$
Hence $f\cdot a = \alpha = b\cdot \left(1_{\bd{M}}\times f\right)$. Thus $f$ is $\bd{M}$-equivariant. Now consider any principial $\bd{M}$-bundle $\pi:\cP\ra T$ with respect to $\cJ$ and let $d:\cP\ra X$ be a $\bd{M}$-equivariant morphism to $(X,a)$. We know that $F$ sends $(\pi,d)$ to some object of $[Y/\bd{M}]$ of the form $(\pi,\delta)$. It suffices to prove that $\delta = f\cdot d$. For this consider a sieve $S$ in $\cJ(T)$ such that $S$ trivializes $\pi$. Pick $g:\widetilde{T}\ra T$ in $S$ and a cartesian square
\begin{center}
\begin{tikzpicture}
[description/.style={fill=white,inner sep=2pt}]
\matrix (m) [matrix of math nodes, row sep=2em, column sep=2em,text height=1.5ex, text depth=0.25ex] 
{ g^*\cP &  \cP    \\
  \widetilde{T} &  T           \\} ;
\path[->,line width=1.0pt,font=\scriptsize]
(m-1-1) edge node[above] {$ g'  $} (m-1-2)
(m-2-1) edge node[below] {$ g $} (m-2-2)
(m-1-1) edge node[left] {$ \pi_g $} (m-2-1)
(m-1-2) edge node[right] {$ \pi $} (m-2-2);
\end{tikzpicture}
\end{center}
Then $(\pi_g,d\cdot g')$ is an object of $[X/\bd{M}]$. Since $F$ is over $\mathbb{B}\bd{M}$, we derive that $F(\pi_g,d\cdot g') = (\pi_g,\delta \cdot g')$. By definition $\pi_g$ is trivial $\bd{M}$-bundle. Thus we have
$$\delta\cdot g' = f\cdot d\cdot g'$$
This holds for pullback $g'$ of every $g$ in $S$ along $\pi$. These pullbacks $\{g'\}_{g\in S}$ generate some sieve $S'$ on $\cP$ and the formula
$$\delta\cdot h = f\cdot d\cdot h$$
holds for every $h$ in $S'$. Moreover, $S'$ is a covering sieve i.e. $S'\in \cJ(\cP)$. According to assumption on $\cJ$ we infer that $h^{\cP} = \Mor_{\cB}\left(-,\cP\right):\cB^{\mathrm{op}}\ra \Set$ is a separated presheaf with respect to $\cJ$. Thus the formula
$$\delta\cdot h = f\cdot d\cdot h$$
which holds for every $h$ in $S'$ implies that $\delta = f\cdot d$.
\end{proof}


































\end{document}