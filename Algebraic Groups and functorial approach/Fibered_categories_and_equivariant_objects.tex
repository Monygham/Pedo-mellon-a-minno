\input ../pree.tex

\begin{document}

\title{Fibered categories and equivariant objects}
\date{}
\maketitle

\section{Introduction}
\noindent
These notes are devoted to study fibered categories with special emphasis on equivariant objects. Specifically they aim at explaining the notion of equivariant qausi-coherent sheaf ({\cite[definition 1.6]{mumford1994geometric}}). With respect to this task we fill some details which are absent (although the possibility of their precise development is indicated) in Vistoli's excellent exposition {\cite[Part 1]{fantechi2005fundamental}}. In addition we give extensive treatment of fibered categories of principal bundles over any Grothendieck site culminated in Theorem \ref{theorem:equivariant_morphisms_can_described_by_fibered_categories}, which is (as we imagine) part of a folklore, but for which we did not have any reference.\\ 
In these notes we often work with several distinct categories and in particular, we consider Yoneda embeddings for all these categories at the same time. In order to make our notation clear we denote by $h^{\cC}:\cC\ra \widehat{\cC}$ the Yoneda embedding for category $\cC$. In particular, if $X$ is an object of $\cC$, then $h^{\cC}_X$ is a presheaf associated with $X$.

\section{Fibered categories}
\noindent
We fix a functor $p:\cE\ra \cB$. We introduce now some convenient notation that will help clarifying our definitions. Consider a morphism $\phi:\xi\ra \eta$ of $\cE$ such that $p(\phi) = f$ and $f:X\ra Y$. We depict this situation by the square diagram
\begin{center}
\begin{tikzpicture}
[description/.style={fill=white,inner sep=2pt}]
\matrix (m) [matrix of math nodes, row sep=3em, column sep=3em,text height=1.5ex, text depth=0.25ex] 
{ \xi      &  \eta      \\
  X        &  Y         \\} ;
\path[->,line width=1.0pt,font=\scriptsize]
(m-1-1) edge node[above] {$ \phi $} (m-1-2)
(m-2-1) edge node[below] {$ f $} (m-2-2);
\path[|->,line width=1.0pt,font=\scriptsize]
(m-1-1) edge node[left]  {$  $} (m-2-1)
(m-1-2) edge node[right] {$  $} (m-2-2);
\end{tikzpicture}
\end{center}
Note that to every such square there corresponds a commutative square 
\begin{center}
\begin{tikzpicture}
[description/.style={fill=white,inner sep=2pt}]
\matrix (m) [matrix of math nodes, row sep=3em, column sep=3em,text height=1.5ex, text depth=0.25ex] 
{ h^{\cE}_\xi      &  h^{\cE}_\eta      \\
  h^{\cB}_X\cdot p        &  h^{\cB}_Y\cdot p         \\} ;
\path[->,line width=1.0pt,font=\scriptsize]
(m-1-1) edge node[above] {$ h^{\cE}_\phi $} (m-1-2)
(m-2-1) edge node[below] {$ \left(h^{\cB}_f\right)_p $} (m-2-2)
(m-1-1) edge node[left]  {$ p_{\mathrm{hom}} $} (m-2-1)
(m-1-2) edge node[right] {$ p_{\mathrm{hom}} $} (m-2-2);
\end{tikzpicture}
\end{center}
of presheaves on $\cE$, where $p_{\mathrm{hom}}$ denotes maps induced by $p$ on sets of morphisms.

\begin{definition}
Consider a square 
\begin{center}
\begin{tikzpicture}
[description/.style={fill=white,inner sep=2pt}]
\matrix (m) [matrix of math nodes, row sep=3em, column sep=3em,text height=1.5ex, text depth=0.25ex] 
{ \xi      &  \eta      \\
  X        &  Y         \\} ;
\path[->,line width=1.0pt,font=\scriptsize]
(m-1-1) edge node[above] {$ \phi $} (m-1-2)
(m-2-1) edge node[below] {$ f $} (m-2-2);
\path[|->,line width=1.0pt,font=\scriptsize]
(m-1-1) edge node[left]  {$  $} (m-2-1)
(m-1-2) edge node[right] {$  $} (m-2-2);
\end{tikzpicture}
\end{center}
We call the square \textit{cartesian} and $\phi$ \textit{a cartesian morphism with respect to $p$} if the corresponding square of presheaves on $\cE$ is cartesian in the category of presheaves.
\end{definition}
\noindent
One can rephrase definition above in terms of presheaves as follows. Morphism $\phi:\xi\ra \eta$ is cartesian with respect to $p$ if the square 
\begin{center}
\begin{tikzpicture}
[description/.style={fill=white,inner sep=2pt}]
\matrix (m) [matrix of math nodes, row sep=3em, column sep=7em,text height=1.5ex, text depth=0.25ex] 
{ \Mor_{\cE}(\zeta,\xi) &  \Mor_{\cE}(\zeta,\eta)    \\
  \Mor_{\cB}\big(p(\zeta),p(\xi)\big) &  \Mor_{\cB}\big(p(\zeta),p(\eta)\big)           \\} ;
\path[->,line width=1.0pt,font=\scriptsize]
(m-1-1) edge node[above] {$ \Mor_{\cE}\left(1_{\zeta},\phi\right)  $} (m-1-2)
(m-2-1) edge node[below] {$ \Mor_{\cB}\left(1_{p(\zeta)},p(\phi)\right) $} (m-2-2)
(m-1-1) edge node[left]  {$ p_{\mathrm{hom}} $} (m-2-1)
(m-1-2) edge node[right] {$ p_{\mathrm{hom}} $} (m-2-2);
\end{tikzpicture}
\end{center}
of sets is cartesian for every object $\zeta$ of $\cE$.

\begin{fact}\label{fact:uniqueness_of_pullbacks}
Let $p:\cE\ra \cB$ be a functor, let $f:X\ra Y$ be a morphism of $\cB$ and let $\eta$ be an object of $\cE$. Suppose that $\phi_1:\xi_1\ra \eta,\phi_2:\xi_2\ra \eta$ are morphisms of $\cE$ that are cartesian with respect to $p$ and assume that $p(\phi_1) = p(\phi_2)$. Then there exists a unique morphism $\theta:\xi_1\ra \xi_2$ such that $\phi_1 = \phi_2\cdot \theta$. Moreover, $\theta$ is an isomorphism.
\end{fact}
\begin{proof}
We use the presheaf reformulation of a definition of cartesian morphisms of $p$. It implies that there exists a unique natural transformation $\sigma:h^{\cE}_{\xi_1}\ra h^{\cE}_{\xi_2}$ such that $h^{\cE}_{\phi_1} = h^{\cE}_{\phi_2}\cdot \sigma$. Moreover, $\sigma$ is a natural isomorphism. Since $h^{\cE}:\cE\ra \widehat{\cE}$ is full and faithful, we derive that there exists a unique morphism $\theta:\xi_1 \ra \xi_2$ such that $h^{\cE}_{\theta} = \sigma$. Then $\theta$ satisfies the assertion.
\end{proof}

\begin{definition}
Let $p:\cE\ra \cB$ be a functor, let $f:X\ra Y$ be a morphism of $\cB$ and let $\eta$ be an object of $\cE$ such that $p(\eta) = Y$. A pair $(\xi,\phi)$ such that $\xi$ is an object of $\cE$ and $\phi:\xi\ra \eta$ is a morphism of $\cE$ is called \textit{a pullback of $\eta$ along $f$} if the following conditions are satisfied.
\begin{enumerate}[label=\textbf{(\arabic*)}, leftmargin=3.0em]
\item $p(\phi) = f$
\item $\phi$ is cartesian morphism of $p$.
\end{enumerate}
\end{definition}
\noindent
Note that Fact \ref{fact:uniqueness_of_pullbacks} implies that pullbacks are unique up to a unique isomorphism.

\begin{definition}
Let $p:\cE\ra \cB$ be a functor. Then $p$ is \textit{a fibered category} if and only if for every morphism $f:X\ra Y$ of $\cB$ and every object $\eta$ of $\cE$ such that $p(\eta) = Y$ there exists a pullback of $\eta$ along $f$. If $p:\cE\ra \cB$ is a fibered category, then we say that \textit{$\cE$ is fibered over $\cB$ with respect to $p$}.
\end{definition}
\noindent
Now we give some examples of fibered categories. The first is a prototypical for the notion of a cartesian category. It shows that any category $\cB$ with fiber products gives rise in a canonical way to a fibered category over $\cB$ with cartesian arrows as cartesian squares in $\cB$.

\begin{example}[the fibered category of arrows]\label{example:the_fibered_category_of_arrows}
Let $\cB$ be a category. We define the category $\mathrm{Arr}(\cB)$ of arrows of $\cB$ as follows. Objects of $\mathrm{Arr}(\cB)$ are morphisms $\pi:\tilde{X}\ra X$ of $\cB$. Now if $\pi:\tilde{X}\ra X$ and $\psi:\tilde{Y}\ra Y$ are objects of $\mathrm{Arr}(\cB)$, then a morphism $\pi\ra \psi$ is a pair $(f,\phi)$ such that $f:X\ra Y$ and $\phi:\tilde{X}\ra \tilde{Y}$ are morphisms in $\cB$ making the square
\begin{center}
\begin{tikzpicture}
[description/.style={fill=white,inner sep=2pt}]
\matrix (m) [matrix of math nodes, row sep=2em, column sep=2em,text height=1.5ex, text depth=0.25ex] 
{ \tilde{X} &  \tilde{Y}    \\
  X &  Y           \\} ;
\path[->,line width=1.0pt,font=\scriptsize]
(m-1-1) edge node[above] {$ \phi  $} (m-1-2)
(m-2-1) edge node[below] {$ f $} (m-2-2)
(m-1-1) edge node[left] {$\pi $} (m-2-1)
(m-1-2) edge node[right] {$  \psi $} (m-2-2);
\end{tikzpicture}
\end{center}
commutative. There exists a functor $p_{\mathrm{Arr}(\cB)}:\mathrm{Arr}(\cB)\ra \cB$ given by formula $p_{\mathrm{Arr}(\cB)}\big((f,\phi)\big) = f$. Suppose now that $f:X\ra Y$ and $\psi:\tilde{Y}\ra Y$ are morphisms of $\cB$ and there exists a commutative square
\begin{center}
\begin{tikzpicture}
[description/.style={fill=white,inner sep=2pt}]
\matrix (m) [matrix of math nodes, row sep=2em, column sep=2em,text height=1.5ex, text depth=0.25ex] 
{ \tilde{X} &  \tilde{Y}    \\
  X &  Y           \\} ;
\path[->,line width=1.0pt,font=\scriptsize]
(m-1-1) edge node[above] {$ \phi  $} (m-1-2)
(m-2-1) edge node[below] {$ f $} (m-2-2)
(m-1-1) edge node[left] {$\pi $} (m-2-1)
(m-1-2) edge node[right] {$  \psi $} (m-2-2);
\end{tikzpicture}
\end{center}
It is a direct consequence of the definition that $(f,\phi)$ is a cartesian morphisms of $p_{\mathrm{Arr}(\cB)}$ if and only if the square above is cartesian. Thus $p_{\mathrm{Arr}(\cB)}$ is a fibered category provided that $\cB$ admits fiber products.
\end{example}

\begin{definition}
Suppose that $p_1:\cE_1\ra \cB$ and $p_2:\cE_2\ra \cB$ are fibered categories. Then a functor $F:\cE_1\ra \cE_2$ is \textit{a morphism of fibered categories} if the following two assertions are satisfied.
\begin{enumerate}[label=\textbf{(\arabic*)}, leftmargin=3.0em]
\item $p_1 = F\cdot p_2$ or in other words $F$ is a functor over $\cB$.
\item Image under $F$ of a cartesian morphism of $p_1$ is a cartesian morphism of $p_2$.
\end{enumerate}
\end{definition}
\noindent
Next example is closely related to the previous one, but is of more topological flavour.

\begin{example}[the fibered category vector bundles]\label{example:the_fibered_category_of_vector_bundles}
Let $\Top$ be the category of topological spaces. We define a category $\bd{VectBund}_{\RR}$ of real vector bundles as follows. Objects of $\bd{VectBund}_{\RR}$ are topological $\RR$-vector bundles $\pi:\cV \ra X$. Now if $\pi:\cV \ra X$ and $\psi:\cW \ra Y$ are topological $\RR$-vector bundles, then a morphism $\pi\ra \psi$ is a pair $(f,\phi)$ such that $f:X\ra Y$ and $\phi:\cV \ra \cW$ are continuous maps making the square
\begin{center}
\begin{tikzpicture}
[description/.style={fill=white,inner sep=2pt}]
\matrix (m) [matrix of math nodes, row sep=2em, column sep=2em,text height=1.5ex, text depth=0.25ex] 
{ \cV &  \cW    \\
  X &  Y           \\} ;
\path[->,line width=1.0pt,font=\scriptsize]
(m-1-1) edge node[above] {$ \phi  $} (m-1-2)
(m-2-1) edge node[below] {$ f $} (m-2-2)
(m-1-1) edge node[left] {$\pi $} (m-2-1)
(m-1-2) edge node[right] {$  \psi $} (m-2-2);
\end{tikzpicture}
\end{center}
commutative and moreover, $\phi$ induces an $\RR$-linear map on fibers i.e. for each point $x$ in $X$ map $\phi$ induces an $\RR$-linear map $\pi^{-1}(x)\ra \psi^{-1}\left(f(x)\right)$. We have the functor $\bd{VectBund}_{\RR}\ra \mathrm{Arr}(\Top)$ that forgets about $\RR$-vector bundle structure. Since topological vector bundles are stable under continuous change of base, we deduce (according to description of cartesian squares in Example \ref{example:the_fibered_category_of_arrows}) that the composition of this forgetful functor with $p_{\mathrm{Arr}(\Top)}:\mathrm{Arr}(\Top)\ra \Top$ is the fibered category. Thus we have a commutative triangle
\begin{center}
\begin{tikzpicture}
[description/.style={fill=white,inner sep=2pt}]
\matrix (m) [matrix of math nodes, row sep=2em, column sep=1em,text height=1.5ex, text depth=0.25ex] 
{  \bd{VectBund}_{\RR} &        & \mathrm{Arr}(\Top)  \\
          &\Top &  \\} ;
\path[->,line width=1.0pt,font=\scriptsize]
(m-1-1) edge node[above] {$  $} (m-1-3)
(m-1-1) edge node[swap] {$  $} (m-2-2)
(m-1-3) edge node[below = 6pt, right = 1pt] {$ p_{\mathrm{Arr}(\Top)} $} (m-2-2);
\end{tikzpicture}
\end{center}
and the functor $\bd{VectBund}_{\RR} \ra \mathrm{Arr}(\Top)$ is a morphism of fibered categories.
\end{example}

\section{Pseudo-functors and fibered categories of elements}
\noindent
Pseudo-functors are certain non-strict 2-functors. In this section we introduce a procedure that enables to construct a fibered category out of a pseudo-functor. We start by defining this notion.

\begin{definition}
Let $\cB$ be a category. Consider the tuple of collections
$$F = \big(\{F(X)\}_{X\in \mathrm{Ob}(\cB)},\{F(f)\}_{f\in \Mor(\cB)},\{\Theta^{f,g}\}_{f,g\in \Mor(\cB),\,\mathrm{cod}(f)=\mathrm{dom}(g)},\{\epsilon^X\}_{X\in \mathrm{Ob}(\cB)}\big)$$
of the following data.
\begin{enumerate}[label=\textbf{(\arabic*)}, leftmargin=3.0em]
\item For each object $X$ of $\cB$ a category $F(X)$.
\item For each arrow $f:X\ra Y$ a functor $F(f):F(Y)\ra F(X)$.
\item For each object $X$ of $\cB$ a natural isomorphism $\epsilon^X:1_{F(X)} \ra F(1_{X})$.
\item For any two composable morphisms $f:X\ra Y$ and $g:Y\ra Z$ of $\cB$ a natural isomorphism $\Theta^{g,f}: F(f)\cdot F(g) \ra F(g\cdot f)$
\end{enumerate}
Suppose that these data are subject to the following conditions.
\begin{enumerate}[label=\textbf{(\arabic*)}, leftmargin=3.0em]
\item For every arrow $f:X\ra Y$ in $\cB$ we have
$$1_{F(f)} = \Theta^{f,1_X} \cdot \epsilon^X_{F(f)},\,1_{F(f)} = \Theta^{1_Y,f} \cdot F(f)\left(\epsilon^Y\right)$$
\item For any three morphisms $f:X\ra Y,g:Y\ra Z,h:Z\ra W$ of $\cB$ the square of functors and natural isomorphisms
\begin{center}
\begin{tikzpicture}
[description/.style={fill=white,inner sep=2pt}]
\matrix (m) [matrix of math nodes, row sep=3em, column sep=3em,text height=1.5ex, text depth=0.25ex] 
{ F(f)\cdot F(g)\cdot F(h) & F(f)\cdot F\left(h\cdot g\right)      \\
  F\left(g\cdot f\right)\cdot F(h) & F\left(h\cdot g\cdot f\right) \\} ;
\path[->,line width=1.0pt,font=\scriptsize]
(m-1-1) edge node[above] {$  F(f)\big(\Theta^{h,g}\big)  $} (m-1-2)
(m-2-1) edge node[below] {$ \Theta^{h,g\cdot f} $} (m-2-2)
(m-1-1) edge node[left] {$  \Theta^{g,f}_{F(h)}  $} (m-2-1)
(m-1-2) edge node[right] {$ \Theta^{h\cdot g,f} $} (m-2-2);
\end{tikzpicture}
\end{center}
is commutative.
\end{enumerate}
Then $F$ is called \textit{a pseudo-functor on $\cB$}
\end{definition}
\noindent
Now we show how to construct a fibered category from a pseudo-functor. Suppose that $\cB$ is a category and
$$F = \big(\{F(X)\}_{X\in \mathrm{Ob}(\cB)},\{F(f)\}_{f\in \Mor(\cB)},\{\Theta^{f,g}\}_{f,g\in \Mor(\cB),\,\mathrm{cod}(f)=\mathrm{dom}(g)},\{\epsilon^X\}_{X\in \mathrm{Ob}(\cB)}\big)$$
is a pseudo-functor on $\cB$. We define a category $\bigint_{\cB}F$. Its objects are pairs $(X,\xi)$ such that $X$ is an object of $\cB$ and $\xi$ is an object of $F(X)$. If $(X,\xi)$ and $(Y,\eta)$ are objects of $\bigint_{\cB}F$, then a morphism between these objects is a pair $(f,\sigma)$ such that $f:X\ra Y$ is a morphism of $\cB$ and $\sigma:\xi\ra F(f)(\eta)$ is a morphism of $F(X)$. Now suppose that $(f,\sigma):(X,\xi)\ra (Y,\eta)$ and $(g,\tau):(Y,\eta)\ra (Z,\zeta)$ are morphisms of $\bigint_{\cB}F$. Then we define their composition by formula
$$(g,\tau)\cdot (f,\sigma) = \big(g\cdot f, \Theta^{g,f}_{\zeta}\cdot F(f)\left(\tau\right)\cdot \sigma\big)$$

\begin{fact}
$\bigint_{\cB}F$ is a well defined category.
\end{fact}
\begin{proof}
We first verify that the composition of morphisms in $\bigint_{\cB}F$ is associative. Suppose that $(f,\sigma):(X,\xi)\ra (Y,\eta),(g,\tau):(Y,\eta)\ra (Z,\zeta),(h,\rho):(Z,\zeta)\ra (W,\omega)$ are morphisms of $\bigint_{\cB}F$. Then
$$\big((h,\rho)\cdot (g,\tau)\big)\cdot (f,\sigma) = \big(h\cdot g,\Theta^{h,g}_{\omega}\cdot F(g)\left(\rho\right)\cdot \tau\big)\cdot (f,\sigma) = $$
$$ = \bigg(h\cdot g\cdot f,\Theta^{h\cdot g,f}_{\omega}\cdot F(f)\big(\Theta^{h,g}_{\omega}\cdot F(g)\left(\rho\right)\cdot \tau \big)\cdot \sigma\bigg) = \bigg(h\cdot g\cdot f,\Theta^{h\cdot g,f}_{\omega}\cdot F(f)\big(\Theta^{h,g}_{\omega}\big)\cdot F(f)\big(F(g)\left(\rho\right)\big)\cdot F(f)\big(\tau\big)\cdot \sigma \bigg)$$
and
$$(h,\rho)\cdot \big((g,\tau) \cdot (f,\sigma) \big) = (h, \rho)\cdot \bigg(g\cdot f, \Theta^{g,f}_{\zeta}\cdot F(f)\big(\tau\big)\cdot \sigma\bigg) = $$
$$ = \big(h\cdot g\cdot f, \Theta^{h, g\cdot f}_{\omega}\cdot F(g\cdot f)\big(\rho\big)\cdot \Theta^{g,f}_{\zeta}\cdot F(f)\big(\tau\big)\cdot \sigma \big) = \big(h\cdot g\cdot f, \Theta^{h, g\cdot f}_{\omega}\cdot  \Theta^{g,f}_{F(h)(\omega)} \cdot F(f)\big(F(g)\left(\rho\right)\big)\cdot F(f)\big(\tau\big)\cdot \sigma \big)$$
Since $\Theta^{h\cdot g,f}_{\omega}\cdot F(f)\big(\Theta^{h,g}_{\omega}\big) = \Theta^{h, g\cdot f}_{\omega}\cdot  \Theta^{g,f}_{F(h)(\omega)}$, we deduce that
$$\big((h,\rho)\cdot (g,\tau)\big)\cdot (f,\sigma) = (h,\rho)\cdot \big((g,\tau) \cdot (f,\sigma) \big)$$
and hence the composition in $\bigint_{\cB}F$ is associative. Next we prove that for each object $(X,\xi)$ of $\bigint_{\cB}F$ there exists an identity morphism. We claim that $(1_X,\epsilon^X_{\xi}):(X,\xi)\ra (X,\xi)$ is the identity. Indeed, for morphisms $(f,\sigma):(X,\xi)\ra (Y,\eta)$ and $(g,\tau):(Z,\zeta)\ra (X,\xi)$ we have
$$(f,\sigma) \cdot (1_X,\epsilon^X_{\xi}) = \big(f,\Theta^{f,1_X}_{\eta}\cdot F(1_X)\left(\sigma \right)\cdot \epsilon^X_{\xi}\big) =  \big(f,\Theta^{f,1_X}_{\eta}\cdot \epsilon^X_{F(f)(\eta)}\cdot \sigma\big) = (f,\sigma)$$
and
$$(1_X,\epsilon^X_{\xi}) \cdot (g,\tau) = \big(g,\Theta^{1_X,g}_{\xi} \cdot F(g)\left(\epsilon^X_{\xi}\right)\cdot \tau \big) = (g,\tau)$$
Therefore, $\bigint_{\cB}F$ is a category.
\end{proof}
\noindent
Next we define a functor $p_F:\bigint_{\cB}F\ra \cB$ by formula
$$p_F\bigg((f,\sigma):(X,\xi)\ra (Y,\tau)\bigg) = f:X\ra Y$$
This is clearly a well defined functor. Now we prove the following statement.

\begin{fact}
The functor $p_F:\bigint_{\cB}F\ra \cB$ is a fibered category.
\end{fact}
\begin{proof}
Let $f:X\ra Y$ be a morphism in $\cB$ and $\eta$ be an object of $F(Y)$. Thus $(Y,\eta)$ is an object of $\bigint_{\cB}F$. It suffices to show that $(Y,\eta)$ admits a pullback along $f$. We claim that
$$\big(f,1_{F(f)(\eta)}\big):\big(X,F(f)(\eta)\big)\ra (Y,\eta)$$
is a cartesian morphism of $p_F$ that yields a pullback of $\eta$ along $f$. To prove the claim consider an object $(Z,\zeta)$ of $\bigint_{\cB}F$ and suppose that $(g,\tau):(Z,\zeta)\ra (Y,\eta)$ is a morphism of $\bigint_{\cB}F$ such that $g$ factors through $f$. Then there exists $h:Z\ra X$ such that $f\cdot h = g$. Note that $\tau:\zeta\ra F(g)(\eta)$. Since $g = f\cdot h$, we have
$$\tau = \Theta^{f,h}_{\eta}\cdot \left(\Theta^{f,h}_{\eta}\right)^{-1}\cdot \tau = \Theta^{f,h}_{\eta}\cdot F(h)\big(1_{F(f)(\eta)}\big) \cdot  \left(\Theta^{f,h}_{\eta}\right)^{-1}\cdot \tau$$
and hence
$$(g,\tau) = \big(f, 1_{F(f)(\eta)} \big) \cdot \bigg(h,\left(\Theta^{f,h}_{\eta}\right)^{-1}\cdot \tau \bigg)$$
Thus $(g,\tau)$ factors through $\big(f,1_{F(f)(\eta)}\big)$ and the formula above shows that this factorization is unique. Hence $\big(f,1_{F(f)(\eta)}\big)$ is a cartesian morphism of $p_F$.
\end{proof}

\begin{definition}
Let $\cB$ be a category and let $F$ be a pseudo-functor on $\cB$. A fibered category $p_F:\bigint_{\cB}F\ra \cB$ constructed above is called \textit{the fibered category of elements of the pseudo-functor $F$}.
\end{definition}
\noindent
It is possible to construct a pseudo-functor out of a fibered category. We will give a brief outline of this construction. For this we introduce notation that will be also used in other considerations.

\begin{definition}
Let $p:\cE\ra \cB$ be a fibered category. For every object $X$ of $\cB$ we denote by $p^{-1}(X)$ a subcategory of $\cE$ consisting of all morphisms $\phi:\xi\ra \eta$ such that $p(\phi) = 1_X$. We call this category \textit{the fiber of $p$ over $X$}.
\end{definition}
\noindent
Suppose now that $p:\cE\ra \cB$ is a fibered category. Let $f:X\ra Y$ be a morphism. For every object $\eta$ in $p^{-1}(Y)$ we pick its pullback $\tilde{f}_{\eta}:f^*\eta\ra \eta$ along $f$. By universal property of cartesian morphisms we deduce that this induces a functor $f^*:p^{-1}(Y)\ra p^{-1}(X)$. Universal property of cartesian morphisms implies also the following assertions.
\begin{enumerate}[label=\textbf{(\arabic*)}, leftmargin=3.0em]
\item For each object $X$ of $\cB$ we may choose $(1_X)^* = 1_{p^{-1}(X)}$.
\item For any two composable morphisms $f:X\ra Y$ and $g:Y\ra Z$ of $\cB$ there exists a unique natural isomorphism $\Theta^{g,f}: f^*g^* \ra (g\cdot f)^*$ of functors such that for every $\zeta$ in $p^{-1}(Z)$ we have commutative diagram
\begin{center}
\begin{tikzpicture}
[description/.style={fill=white,inner sep=2pt}]
\matrix (m) [matrix of math nodes, row sep=3em, column sep=3em,text height=1.5ex, text depth=0.25ex] 
{ f^*g^*\zeta       &  g^*\zeta & \zeta    \\
  (g\cdot f)^*\zeta &           & \zeta    \\} ;
\path[->,line width=1.0pt,font=\scriptsize]
(m-1-1) edge node[above] {$ \tilde{f}_{g^*{\zeta}}  $} (m-1-2)
(m-1-2) edge node[above] {$ \tilde{g}_{\zeta}  $} (m-1-3)
(m-1-1) edge node[left] {$ \Theta^{g,f}_{\zeta}  $} (m-2-1)
(m-1-3) edge node[right] {$ 1_{\zeta}  $} (m-2-3)
(m-2-1) edge node[below] {$ \widetilde{g\cdot f}_{\zeta}  $} (m-2-3);
\end{tikzpicture}
\end{center}
\end{enumerate}
From \textbf{(1)}, \textbf{(2)} and Fact \ref{fact:uniqueness_of_pullbacks} one can deduce that the collection
$$\big(\{p^{-1}(X)\}_{X\in \mathrm{Ob}(\cB)},\{f^*\}_{f\in \Mor(\cB)},\{\Theta^{f,g}\}_{f,g\in \Mor(\cB),\,\mathrm{cod}(f)=\mathrm{dom}(g)},\{1_{p^{-1}(X)}\}_{X\in \mathrm{Ob}(\cB)}\big)$$
is a pseudo-functor.

\begin{remark}\label{remark:fibered_categories_and_pseudo_functors}
The construction of the fibered category of elements is a part of $2$-equivalence between appropriately defined category of pseudo-functors on $\cB$ and the category of fibered categories over $\cB$.
\end{remark}

\section{Example: Quasi-coherent sheaves}
\noindent
Note that all examples of fibered categories given so far were fibered subcategories of the fibered category of arrows $p_{\mathrm{Arr}(\cB)}:\mathrm{Arr}(\cB)\ra \cB$ for a given category $\cB$ with fibered-products. In this section we employ the procedure that produces a fibered category out of a pseudo-functor to obtain an important example of a category fibered over $\Sch_k$ (the category of schemes over a ring $k$), which is not of this type.\\
Let $f:X\ra Y$ be a morphism of $k$-schemes. We have an adjunction
\begin{center}   
\begin{tikzpicture}
[description/.style={fill=white,inner sep=2pt}]
\matrix (m) [matrix of math nodes, row sep=3em, column sep=1em,text height=1.5ex, text depth=0.25ex] 
{ \Qcoh(X)& \perp  &\Qcoh(Y)  \\};
\path[solid,->,line width=1.0pt,font=\scriptsize]
(m-1-1) edge [bend left=30] node[auto]  {$ f_* $} (m-1-3)
(m-1-3) edge [bend left=30] node[auto]  {$ f^* $} (m-1-1);
\end{tikzpicture}
\end{center}
It is determined by the bijection
\begin{center}
\begin{tikzpicture}
[description/.style={fill=white,inner sep=2pt}]
\matrix (m) [matrix of math nodes, row sep=3em, column sep=4em,text height=1.5ex, text depth=0.25ex] 
{ \Hom_{\cO_Y}\big(f^*\cG,\cF\big) & \Hom_{\cO_X}\big(\cG,f_*\cF\big) \\} ;
\path[->,line width=1.0pt,font=\scriptsize]
(m-1-1) edge node[above] {$ \Phi^f_{\cG,\cF}  $} (m-1-2);
\end{tikzpicture}
\end{center}
Suppose now that $f:X\ra Y$ and $g:Y\ra Z$ are morphisms of $k$-schemes. Since $(g\cdot f)_* = g_*\cdot f_*$, there exists a unique natural isomorphism $\Theta^{g,f}: f^*g^* \ra (g\cdot f)^*$ such that for every quasi-coherent sheaf $\cF$ in $\Qcoh(X)$ and every quasi-coherent sheaf $\cH$ in $\Qcoh(Z)$ we have
$$\Phi^{g\cdot f}_{\cH,\cF} = \Phi^g_{\cH,f_*\cF}\cdot \Phi^f_{g^*\cH,\cF}  \cdot \Hom_{\cO_X}\big(\Theta^{g,f}_{\cH},1_{\cF}\big)$$
Now we have the following result.

\begin{fact}\label{fact:quasi_coherent_sheaves_are_pseudo_functor_associativity}
Suppose that $f:X\ra Y$, $g:Y\ra Z$ and $h:Z\ra W$ are morphism of $k$-schemes. Then the square
\begin{center}
\begin{tikzpicture}
[description/.style={fill=white,inner sep=2pt}]
\matrix (m) [matrix of math nodes, row sep=3em, column sep=3em,text height=1.5ex, text depth=0.25ex] 
{ f^*g^*h^*                      & f^*\left(h\cdot g\right)^*       \\
  \left(g\cdot f\right)^*h^*     & \left(h\cdot g\cdot f\right)^*   \\} ;
\path[->,line width=1.0pt,font=\scriptsize]
(m-1-1) edge node[above] {$ f^*\Theta^{h,g}     $} (m-1-2)
(m-2-1) edge node[below] {$ \Theta^{h,g\cdot f} $} (m-2-2)
(m-1-1) edge node[left]  {$ \Theta^{g,f}_{h^*}  $} (m-2-1)
(m-1-2) edge node[right] {$ \Theta^{h\cdot g,f} $} (m-2-2);
\end{tikzpicture}
\end{center}
of functors and natural isomorphisms is commutative.
\end{fact}
\begin{proof}
Suppose that $\cF$ is an object of $\Qcoh(X)$ and $\cK$ is an object of $\Qcoh(W)$. Then
$$\Phi^h_{\cK,g_*f_*\cF}\cdot \Phi^g_{h^*\cK,f_*\cF}\cdot \Phi^f_{g^*h^*\cK,\cF} \cdot \Hom_{\cO_X}\big(\Theta^{g,f}_{h^*\cK},1_{\cF}\big) \cdot \Hom_{\cO_X}\big(\Theta^{h,g\cdot f}_{\cK},1_{\cF}\big)  = $$
$$= \Phi^h_{\cK,g_*f_*\cF}\cdot \Phi^{g\cdot f}_{h^*\cK,\cF} \cdot \Hom_{\cO_X}\big(\Theta^{g,f}_{h^*\cK},1_{\cF}\big) = \Phi^{h\cdot g\cdot f}_{\cK,\cF}$$
and
$$\Phi^h_{\cK,g_*f_*\cF}\cdot \Phi^g_{h^*\cK,f_*\cF}\cdot \Phi^f_{g^*h^*\cK,\cF}\cdot \Hom_{\cO_X}\big(f^*\Theta^{h,g}_{\cK},1_{\cF}\big)\cdot \Hom_{\cO_X}\big(\Theta^{h\cdot g,f}_{\cK},1_{\cF}\big) = $$
$$ = \Phi^h_{\cK,g_*f_*\cF}\cdot \Phi^g_{h^*\cK,f_*\cF}\cdot \Hom_{\cO_X}\big(\Theta^{h,g}_{\cK},1_{f_*\cF}\big) \cdot  \Phi^f_{(h\cdot g)^*\cK,\cF} \cdot \Hom_{\cO_X}\big(\Theta^{h\cdot g,f}_{\cK},1_{\cF}\big) = $$
$$ = \Phi^{h\cdot g}_{\cK,f_*\cF} \cdot  \Phi^f_{(h\cdot g)^*\cK,\cF} \cdot \Hom_{\cO_X}\big(\Theta^{h\cdot g,f}_{\cK},1_{\cF}\big) = \Phi^{h\cdot g\cdot f}_{\cK,\cF} $$
Therefore, we derive that
$$\Phi^h_{\cK,g_*f_*\cF}\cdot \Phi^g_{h^*\cK,f_*\cF}\cdot \Phi^f_{g^*h^*\cK,\cF}\cdot \Hom_{\cO_X}\big(\Theta^{g,f}_{h^*\cK},1_{\cF}\big)\cdot \Hom_{\cO_X}\big(\Theta^{h,g\cdot f}_{\cK},1_{\cF}\big) = $$
$$ = \Phi^h_{\cK,g_*f_*\cF}\cdot \Phi^g_{h^*\cK,f_*\cF}\cdot \Phi^f_{g^*h^*\cK,\cF}\cdot \Hom_{\cO_X}\big(f^*\Theta^{h,g}_{\cK},1_{\cF}\big)\cdot \Hom_{\cO_X}\big(\Theta^{h\cdot g,f}_{\cK},1_{\cF}\big)$$
and hence
$$\Hom_{\cO_X}\big(\Theta^{h,g\cdot f}_{\cK} \cdot \Theta^{g,f}_{h^*\cK},1_{\cF}\big) = \Hom_{\cO_X}\big(\Theta^{g,f}_{h^*\cK},1_{\cF}\big)\cdot \Hom_{\cO_X}\big(\Theta^{h,g\cdot f}_{\cK},1_{\cF}\big) =$$
$$= \Hom_{\cO_X}\big(f^*\Theta^{h,g}_{\cK},1_{\cF}\big) \cdot \Hom_{\cO_X}\big(\Theta^{h\cdot g,f}_{\cK},1_{\cF}\big) = \Hom_{\cO_X}\big( \Theta^{h\cdot g,f}_{\cK} \cdot f^*\Theta^{h,g}_{\cK},1_{\cF}\big)$$
Since this equality holds for every quasi-coherent sheaf $\cF$ on $X$, we deduce that
$$\Theta^{h,g\cdot f}_{\cK} \cdot \Theta^{g,f}_{h^*\cK} = \Theta^{h\cdot g,f}_{\cK} \cdot f^*\Theta^{h,g}_{\cK}$$
for every quasi-coherent sheaf $\cK$. This proves the assertion.
\end{proof}
\noindent
Note that for every $k$-scheme $X$ we may assume that $\left(1_X\right)_* = 1_{\Qcoh(X)} = \left(1_X\right)^*$ and $\Phi^{1_X}_{\cG,\cF} = \Hom_{\cO_X}\left(1_\cF,1_\cG\right)$.

\begin{fact}\label{fact:quasi_coherent_sheaves_are_pseudo_functor_unit}
Let $f:X\ra Y$ and $g:Z\ra X$ be morphisms of $k$-schemes. Then
$$\Theta^{f,1_X} = 1_{f^*},\,\Theta^{1_X,g} = 1_{g^*}$$
\end{fact}
\begin{proof}
Suppose that $\cF$ is an object of $\Qcoh(X)$ and $\cG$ is an object of $\Qcoh(Y)$. Then
$$\Phi^{f}_{\cG,\cF} = \Phi^{f\cdot 1_X}_{\cG,\cF} = \Phi^f_{\cG,\cF} \cdot \Phi^{1_X}_{f^*\cG,\cF} \cdot \Hom_{\cO_X}\left(\Theta^{f,1_X}_{\cG},1_{\cF}\right) = \Phi^f_{\cG,\cF} \cdot \Hom_{\cO_X}\left(\Theta^{f,1_X}_{\cG},1_{\cF}\right)$$
and thus $\Hom_{\cO_X}\left(\Theta^{f,1_X}_{\cG},1_{\cF}\right) = \Hom_{\cO_X}\left(1_{f^*\cG},1_{\cF}\right)$. Since this holds for every quasi-coherent sheaf $\cF$ on $X$, we derive that $\Theta^{f,1_X}_{\cG} = 1_{f^*\cG}$. Thus $\Theta^{f,1_X} = 1_{f^*}$.\\
Suppose that $\cH$ is an object of $\Qcoh(X)$ and $\cF$ is an object of $\Qcoh(Z)$. Then
$$\Phi^g_{\cH,\cF} = \Phi^{1_X\cdot g}_{\cH,\cF} = \Phi^{1_X}_{\cH,g_*\cF} \cdot \Phi^g_{\cH,\cF}\cdot \Hom_{\cO_Z}\left(\Theta^{1_X,g}_{\cH},1_{\cF}\right) = \Phi^g_{\cH,\cF}\cdot \Hom_{\cO_Z}\left(\Theta^{1_X,g}_{\cH},1_{\cF}\right)$$
and thus $\Hom_{\cO_Z}\left(\Theta^{1_X,g}_{\cH},1_{\cF}\right) = \Hom_{\cO_Z}\left(1_{g^*\cH} ,1_{\cF}\right)$. Since this holds for every quasi-coherent sheaf $\cF$ on $Z$, we derive that $\Theta^{1_X,g}_{\cH} = 1_{g^*\cH}$. Thus $\Theta^{1_X,g} = 1_{g^*}$.
\end{proof}
\noindent
Now Facts \ref{fact:quasi_coherent_sheaves_are_pseudo_functor_associativity} and \ref{fact:quasi_coherent_sheaves_are_pseudo_functor_unit} imply that the collection
$$\big(\{\Qcoh(X)\}_{X\in \Sch_k},\{f^*\}_{f\in \Mor(\Sch_k)},\{\Theta^{f,g}\}_{f,g\in \Mor(\Sch_k),\,\mathrm{cod}(f)=\mathrm{dom}(g)},\{1_{1_{\Qcoh(X)}}\}_{X\in \Sch_k}\big)$$
forms a pseudo-functor on $\Sch_k$.

\begin{definition}
\textit{The fibered category of quasi-coherent sheaves on $\Sch_k$} is the fibered category of elements of the pseudo-functor
$$\big(\{\Qcoh(X)\}_{X\in \Sch_k},\{f^*\}_{f\in \Mor(\Sch_k)},\{\Theta^{f,g}\}_{f,g\in \Mor(\Sch_k),\,\mathrm{cod}(f)=\mathrm{dom}(g)},\{1_{1_{\Qcoh(X)}}\}_{X\in \Sch_k}\big)$$
We denote it by $\Qcoh \ra \Sch_k$.
\end{definition}
\noindent
For every $k$-scheme $X$ we have a category $\Alg\left(\Qcoh(X)\right)$ of quasi-coherent $\cO_X$-algebras. If $f:X\ra Y$ is a morphism of $k$-schemes, then we have an adjuntion
\begin{center}   
\begin{tikzpicture}
[description/.style={fill=white,inner sep=2pt}]
\matrix (m) [matrix of math nodes, row sep=3em, column sep=1em,text height=1.5ex, text depth=0.25ex] 
{ \Alg\left(\Qcoh(X)\right)& \perp  & \Alg\left(\Qcoh(Y)\right)  \\};
\path[solid,->,line width=1.0pt,font=\scriptsize]
(m-1-1) edge [bend left=30] node[auto]  {$ f_* $} (m-1-3)
(m-1-3) edge [bend left=30] node[auto]  {$ f^* $} (m-1-1);
\end{tikzpicture}
\end{center}
Using similar argument as above one can show that there exists a canonical structure of a pseudo-functor on the collection
$$\big(\{\Alg\left(\Qcoh(X)\right)\}_{X\in \Sch_k},\{f^*\}_{f\in \Mor(\Sch_k)}\big)$$

\begin{definition}
\textit{The fibered category of quasi-coherent algebras on $\Sch_k$} is the fibered category of elements of the canonical pseudo-functor determined by the collection
$$\big(\{\Alg\left(\Qcoh(X)\right)\}_{X\in \Sch_k},\{f^*\}_{f\in \Mor(\Sch_k)}\big)$$
We denote it by $\Alg\left(\Qcoh\right)\ra \Sch_k$.
\end{definition}

\begin{remark}\label{remark:from_qc_algebras_to_qc_sheaves_over_schemes}
We also have a canonical functor $|-|:\Alg\left(\Qcoh\right)\ra \Qcoh$ that forgets about an algebra structure. This gives rise to a morphism of fibered categories
\begin{center}
\begin{tikzpicture}
[description/.style={fill=white,inner sep=2pt}]
\matrix (m) [matrix of math nodes, row sep=2em, column sep=1em,text height=1.5ex, text depth=0.25ex] 
{  \Alg\left(\Qcoh\right) &        & \Qcoh  \\
          &\Sch_k &  \\} ;
\path[->,line width=1.0pt,font=\scriptsize]
(m-1-1) edge node[above] {$|-|  $} (m-1-3)
(m-1-1) edge node[swap] {$  $} (m-2-2)
(m-1-3) edge node[below = 6pt, right = 1pt] {$ $} (m-2-2);
\end{tikzpicture}
\end{center}
\end{remark}

\begin{example}[The fibered category of relatively affine schemes]\label{example:fibered_category_of_affine_schemes}
Note that $p_{\mathrm{Arr}(\Sch_k)}:\mathrm{Arr}(\Sch_k)\ra \Sch_k$ admits a fibered subcategory that consists of affine morphisms $\pi:\widetilde{X}\ra X$ of $k$-schemes. We denote this fibered category by $\bd{Aff}\left(\Sch_k\right)\ra \Sch_k$.
\end{example}

\begin{remark}\label{remark:cartesian_squares_for_relative_spectra}
Let $f:X\ra Y$ be a morphism of $k$-schemes and $\cA$ is a quasi-coherent $\cO_Y$-algebra. Then there exists a morphism $\widetilde{f}:\Spec_X f^*\cA\ra \Spec_Y\cA$ such that the square
\begin{center}
\begin{tikzpicture}
[description/.style={fill=white,inner sep=2pt}]
\matrix (m) [matrix of math nodes, row sep=2em, column sep=2em,text height=1.5ex, text depth=0.25ex] 
{ \Spec_X f^*\cA & \Spec_Y\cA     \\
  X             & Y             \\} ;
\path[->,line width=1.0pt,font=\scriptsize]  
(m-1-1) edge node[above] {$ \widetilde{f} $} (m-1-2)
(m-2-1) edge node[below] {$ f $} (m-2-2)
(m-1-1) edge node[left] {$  $} (m-2-1)
(m-1-2) edge node[right] {$  $} (m-2-2);
\end{tikzpicture}
\end{center}
is cartesian in $\Sch_k$.
\end{remark}

\begin{example}\label{example:from_qc_algebras_to_the_category_of_arrows_over_schemes}
For every $k$-scheme $X$ we have the relative affine spectrum functor $\Spec_X:\Alg\left(\Qcoh(X)\right)\ra \bd{Aff}_X$, where $\bd{Aff}_X$ is the category of schemes affine over $X$. It is an equivalence of categories and note that $\bd{Aff}_X$ is the fiber of $\bd{Aff}\left(\Sch_k\right)\ra \Sch_k$ over $X$. This observation together with Remark \ref{remark:cartesian_squares_for_relative_spectra} shows that the collection of functors $\Spec_X$ for all $k$-schemes $X$ gives rise to a morphism of fibered categories
\begin{center}
\begin{tikzpicture}
[description/.style={fill=white,inner sep=2pt}]
\matrix (m) [matrix of math nodes, row sep=2em, column sep=1em,text height=1.5ex, text depth=0.25ex] 
{  \Alg\left(\Qcoh\right) &        & \bd{Aff}(\Sch_k)  \\
          &\Sch_k &  \\} ;
\path[->,line width=1.0pt,font=\scriptsize]
(m-1-1) edge node[above] {$ \Spec  $} (m-1-3)
(m-1-1) edge node[swap] {$  $} (m-2-2)
(m-1-3) edge node[below = 6pt, right = 1pt] {$ $} (m-2-2);
\end{tikzpicture}
\end{center}
Moreover, $\Spec:\Alg\left(\Qcoh\right)\ra \bd{Aff}\left(\Sch_k\right)$ is an equivalence of categories.
\end{example}

\section{Equivariant objects in fibered categories}
\noindent
Let $k$ be a commutative ring. We fix a monoid $k$-scheme $\bd{M}$. The following notion is useful for studying actions of algebraic groups and monoids.

\begin{definition}
Let $X$ be a $k$-scheme and let $\bd{M}$ be a monoid $k$-scheme with an action $a:\bd{M}\times_kX\ra X$ on $X$. We denote by $\pi:\bd{M}\times_kX\ra X$ the projection. Consider a pair $(\cF,\tau)$ consisting of a quasi-coherent sheaf $\cF$ on $X$ and a morphism $\tau:\pi^*\cF \ra a^* \cF$. We call it \textit{a quasi-coherent $\bd{M}$-sheaf on $(X,a)$} if the following equalities
$$(\mu \times 1_X)^*\tau = (1_{\bd{M}} \times  a)^*\tau \cdot \pi_{2,3}^*\tau,\,\langle e, 1_{X} \rangle^*\tau = 1_{\cF}$$
hold, where $\mu: \bd{M} \times_k \bd{M} \ra \bd{M}$ is the multiplication on $\bd{M}$, $\pi_{2,3}:\bd{M}\times_k  \bd{M} \times_k  X \ra \bd{M} \times_k X$ is the projection on the last two factors and $e:\bd{1}\ra \bd{M}$ is the unit of $\bd{M}$.
\end{definition}

\begin{definition}
Let $X$ be a $k$-scheme and let $\bd{M}$ be a monoid $k$-scheme with an action $a:\bd{M}\times_kX\ra X$ on $X$. We denote by $\pi:\bd{M}\times_kX\ra X$ the projection. Let $(\cF_1,\tau_1)$ and $(\cF_2,\tau_2)$ be quasi-coherent $\bd{M}$-sheaves on $(X,a)$. Suppose that $\phi:\cF_1\ra \cF_2$ is a morphism of quasi-coherent sheaves on $X$ such that the square
\begin{center}
\begin{tikzpicture}
[description/.style={fill=white,inner sep=2pt}]
\matrix (m) [matrix of math nodes, row sep=3em, column sep=3em,text height=1.5ex, text depth=0.25ex] 
{ \pi^*\cF_1 & a^*\cF_1     \\
  \pi^*\cF_2 & a^*\cF_2             \\} ;
\path[->,line width=1.0pt,font=\scriptsize]  
(m-1-1) edge node[above] {$ \tau_1 $} (m-1-2)
(m-2-1) edge node[below] {$ \tau_2 $} (m-2-2)
(m-1-1) edge node[left] {$ \pi^*\phi $} (m-2-1)
(m-1-2) edge node[right] {$ a^*\phi  $} (m-2-2);
\end{tikzpicture}
\end{center}
is commutative. Then $\phi$ is \textit{a morphism of quasi-coherent $\bd{M}$-sheaves on $(X,a)$}. We denote by $\Qcoh_{\bd{M}}(X)$ the category of quasi-coherent $\bd{M}$-sheaves and call it \textit{the category of quasi-coherent $\bd{M}$-sheaves on $(X,a)$}.
\end{definition}
\noindent
Let us give some example.

\begin{example}\label{example:equivariant_sheaves_on_points}
For this example recall \cite{Monoid_k_functors}. Suppose that $\bd{M}$ is an affine monoid $k$-scheme. We consider $\Spec k$ equipped with the trivial $\bd{M}$-action. Then $\Qcoh_{\bd{M}}\left(\Spec k\right)$ is isomorphic with $\bd{co}\Mod(\bd{M})$ and hence by {\cite[Theorem 15.1]{Monoid_k_functors}} with $\bd{Rep}(\bd{M})$.
\end{example}
\noindent
Our goal in this section is to explain somewhat nonintuitive notion of quasi-coherent $\bd{M}$-sheaf on a $k$-scheme $X$ equipped with action of $\bd{M}$. For this we use the machinery of fibered categories. We fix a fibered category $p:\cE \ra \cB$ . If $f:X\ra Y$ and $\eta$ is an object of $p^{-1}(Y)$, then we denote by $\tilde{f}_{\eta}:f^*\eta\ra \eta$ a pullback of $\eta$. That is the square
\begin{center}
\begin{tikzpicture}
[description/.style={fill=white,inner sep=2pt}]
\matrix (m) [matrix of math nodes, row sep=3em, column sep=3em,text height=1.5ex, text depth=0.25ex] 
{ f^*\eta      &  \eta      \\
  X        &  Y         \\} ;
\path[->,line width=1.0pt,font=\scriptsize]
(m-1-1) edge node[above] {$ \widetilde{f}_{\eta} $} (m-1-2)
(m-2-1) edge node[below] {$ f $} (m-2-2);
\path[|->,line width=1.0pt,font=\scriptsize]
(m-1-1) edge node[left]  {$  $} (m-2-1)
(m-1-2) edge node[right] {$  $} (m-2-2);
\end{tikzpicture}
\end{center}
is cartesian. Using some choice of pullback we obtain a functor $f^*:p^{-1}(Y)\ra p^{-1}(X)$. We start with the following observation.

\begin{remark}\label{remark:unique_identification}
Consider morphisms $f_1,f_2,g_1,g_2$ in $\cB$ such that $g_1\cdot f_1 = g_2\cdot f_2$ with $\mathrm{cod}(g_1) = Y = \mathrm{cod}(g_2)$. For every object $\eta$ in $p^{-1}(Y)$ we have a unique identification $f_1^*g_1^*\eta \cong f_2^*g_2^*\eta$ such that the square
\begin{center}
\begin{tikzpicture}
[description/.style={fill=white,inner sep=2pt}]
\matrix (m) [matrix of math nodes, row sep=3em, column sep=3em,text height=1.5ex, text depth=0.25ex] 
{ f_2^*g_2^*\eta \cong f_1^*g_1^*\eta &  g_1^*\eta                \\
  g_2^*\eta                       &  \eta           \\} ;
\path[->,line width=1.0pt,font=\scriptsize]
(m-1-1) edge node[above] {$ \widetilde{f_1}_{g_1^*\eta} $} (m-1-2)
(m-2-1) edge node[below] {$ \widetilde{g_2}_{\eta} $} (m-2-2)
(m-1-1) edge node[left]  {$ \widetilde{f_2}_{g_2^*\eta} $} (m-2-1)
(m-1-2) edge node[right] {$ \widetilde{g_1}_{\eta} $} (m-2-2);
\end{tikzpicture}
\end{center}
is commutative.
\end{remark}
\noindent
Now we have the following result.

\begin{fact}\label{fact:identification_of_product_and_pullback_along_the_projection}
Let $X,\bd{M}$ be objects of $\cB$ and let $\xi$ be an object of $\cE$ in $p^{-1}(X)$. Assume that the cartesian product of $X$ and $\bd{M}$ exists in $\cB$ and denote by $\pi:\bd{M}\times X\ra X$ the projection. Then there exists a unique morphism (depicted by dotted arrow) such that the diagram
\begin{center}
\begin{tikzpicture}
[description/.style={fill=white,inner sep=2pt}]
\matrix (m) [matrix of math nodes, row sep=3em, column sep=3em,text height=1.5ex, text depth=0.25ex] 
{ h^{\cE}_{\pi^*\xi}         &                                          &                   \\
                             & h^{\cB}_\bd{M}\cdot p\times h^{\cE}_\xi       &  h^{\cE}_\xi      \\
  h^{\cB}_{\bd{M}\times X}\cdot p & h^{\cB}_\bd{M}\cdot p\times h^{\cB}_X\cdot p  &  h^{\cB}_X\cdot p \\} ;
\path[->,line width=1.0pt,font=\scriptsize]
(m-1-1) edge[bend left] node[above] {$ h^{\cE}_{\widetilde{\pi}_{\xi}} $} (m-2-3)
(m-1-1) edge node[left] {$ p_{\mathrm{hom}}  $} (m-3-1)
(m-3-1) edge node[below] {$ = $} (m-3-2)
(m-2-2) edge node[above]  {$ pr_{h^{\cE}_{\xi}} $} (m-2-3)
(m-3-2) edge node[below]  {$ pr_{h^{\cB}_X\cdot p} $} (m-3-3)
(m-2-2) edge node[left]  {$ 1_{h^{\cB}_\bd{M}\cdot p}\times p_{\mathrm{hom}} $} (m-3-2)
(m-2-3) edge node[right] {$ p_{\mathrm{hom}} $} (m-3-3)
(m-3-1) edge[bend right] node[below] {$ \left(h^{\cB}_{\pi}\right)_p $} (m-3-3);
\path[densely dotted,->,line width=1.0pt,font=\scriptsize]
(m-1-1) edge node[above] {$ $} (m-2-2);
\end{tikzpicture}
\end{center}
is commutative, where $pr_{h^{\cE}_{\xi}}$ and $pr_{h^{\cB}_X\cdot p}$ are projections. Moreover, this morphism is an isomorphism.
\end{fact}
\begin{proof}
This is a consequence of the fact that both squares
\begin{center}
\begin{tikzpicture}
[description/.style={fill=white,inner sep=2pt}]
\matrix (m) [matrix of math nodes, row sep=3em, column sep=3em,text height=1.5ex, text depth=0.25ex] 
{ h^{\cB}_{\bd{M}}\cdot p \times h_{\xi}^{\cE}                &  h_{\xi}^{\cE}      &   h^{\cE}_{\pi^*\xi}              &  h^{\cE}_{\xi}            \\
  h^{\cB}_\bd{M}\cdot p   \times h_{X}^{\cB}\cdot p           &  h_{X}^{\cB}\cdot p &   h^{\cB}_{\bd{M}\times X}\cdot p &  h^{\cB}_{X}\cdot p              \\} ;
\path[->,line width=1.0pt,font=\scriptsize]
(m-1-1) edge node[above] {$ pr_{h^{\cE}_{\xi}} $} (m-1-2)
(m-2-1) edge node[below] {$ pr_{h^{\cB}_{X}\cdot p} $} (m-2-2)
(m-1-1) edge node[left]  {$ 1_{h^{\cB}_\bd{M}\cdot p}\times p_{\mathrm{hom}} $} (m-2-1)
(m-1-2) edge node[right] {$ p_{\mathrm{hom}} $} (m-2-2)
(m-1-3) edge node[above] {$ h^{\cE}_{\widetilde{\pi}_{\xi}} $} (m-1-4)
(m-2-3) edge node[below] {$ \left(h^{\cB}_{\pi}\right)_p $} (m-2-4)
(m-1-3) edge node[left]  {$ p_{\mathrm{hom}} $} (m-2-3)
(m-1-4) edge node[right] {$ p_{\mathrm{hom}} $} (m-2-4);;
\end{tikzpicture}
\end{center}
are cartesian.
\end{proof}
\noindent
Fix now two objects $\bd{M}$ and $X$ of $\cB$ such that the product of $\bd{M}$ and $X$ exists. Denote by $\pi: \bd{M} \times X \ra X$ the projection on $X$. Let $a:\bd{M} \times X\ra X$ be a morphism in $\cB$, let $\xi$ be an object in $p^{-1}(X)$ and let $\sigma:h^{\cB}_\bd{M} \cdot p \times h^{\cE}_{\xi}\ra h^{\cE}_{\xi}$ be a morphism of presheaves on $\cE$. Suppose that the square
\begin{center}
\begin{tikzpicture}
[description/.style={fill=white,inner sep=2pt}]
\matrix (m) [matrix of math nodes, row sep=3em, column sep=3em,text height=1.5ex, text depth=0.25ex] 
{ h^{\cB}_{{\bd{M}}}\cdot p \times h_{\xi}^{\cE}                &  h_{\xi}^{\cE}                \\
  h_{\bd{M}\times X}^{\cB}\cdot p           &  h_{X}^{\cB}\cdot p           \\} ;
\path[->,line width=1.0pt,font=\scriptsize]
(m-1-1) edge node[above] {$ \sigma $} (m-1-2)
(m-2-1) edge node[below] {$ \left(h^{\cB}_a\right)_p $} (m-2-2)
(m-1-1) edge node[left]  {$ p_{\mathrm{hom}} $} (m-2-1)
(m-1-2) edge node[right] {$ p_{\mathrm{hom}} $} (m-2-2);
\end{tikzpicture}
\end{center}
is commutative. According to Fact \ref{fact:identification_of_product_and_pullback_along_the_projection} we deduce that $\sigma$ is representable by some morphism $\alpha^{\sigma}:\pi^*\xi\ra \xi$ of $\cE$. By universal property of cartesian square
\begin{center}
\begin{tikzpicture}
[description/.style={fill=white,inner sep=2pt}]
\matrix (m) [matrix of math nodes, row sep=3em, column sep=3em,text height=1.5ex, text depth=0.25ex] 
{ a^*\xi      &  \xi      \\
  \bd{M}\times X        &  X         \\} ;
\path[->,line width=1.0pt,font=\scriptsize]
(m-1-1) edge node[above] {$ \widetilde{a}_{\xi} $} (m-1-2)
(m-2-1) edge node[below] {$ a $} (m-2-2);
\path[|->,line width=1.0pt,font=\scriptsize]
(m-1-1) edge node[left]  {$  $} (m-2-1)
(m-1-2) edge node[right] {$  $} (m-2-2);
\end{tikzpicture}
\end{center}
we deduce that there exists a unique morphism $\tau^{\sigma}:\pi^*\xi\ra a^*\xi$ in $p^{-1}\left(\bd{M}\times X\right)$ such that $\alpha^{\sigma} = \widetilde{a}_{\xi}\cdot \tau^{\sigma}$. Using this notation and Fact \ref{fact:identification_of_product_and_pullback_along_the_projection} we can now state the following result.

\begin{proposition}\label{proposition:equivalent_description_of_equivariance_in_representable_case}
Let $\bd{M}$ be a monoid object in $\cB$ and let $X$ be an object of $\cB$ equipped with an action $a:\bd{M}\times X\ra X$ of $\bd{M}$ on $X$. Denote by $\pi:\bd{M}\times X\ra X$ the projection on $X$. Consider an object $\xi$ in $p^{-1}(X)$ and let $\sigma:h^{\cB}_\bd{M} \cdot p \times h^{\cE}_{\xi}\ra h^{\cE}_{\xi}$ be a morphism of presheaves on $\cE$. Suppose that the square
\begin{center}
\begin{tikzpicture}
[description/.style={fill=white,inner sep=2pt}]
\matrix (m) [matrix of math nodes, row sep=3em, column sep=3em,text height=1.5ex, text depth=0.25ex] 
{ h^{\cB}_{{\bd{M}}}\cdot p \times h_{\xi}^{\cE}                &  h_{\xi}^{\cE}                \\
  h_{\bd{M}\times X}^{\cB}\cdot p           &  h_{X}^{\cB}\cdot p           \\} ;
\path[->,line width=1.0pt,font=\scriptsize]
(m-1-1) edge node[above] {$ \sigma $} (m-1-2)
(m-2-1) edge node[below] {$ \left(h^{\cB}_a\right)_p $} (m-2-2)
(m-1-1) edge node[left]  {$ p_{\mathrm{hom}} $} (m-2-1)
(m-1-2) edge node[right] {$ p_{\mathrm{hom}} $} (m-2-2);
\end{tikzpicture}
\end{center}
is commutative. Consider the following assertions.
\begin{enumerate}[label= \emph{\textbf{(\roman*)}}, leftmargin=3.0em]
\item $\sigma$ is an action of a monoid presheaf $h^{\cB}_{\bd{M}}\cdot p$ on a presheaf $h^{\cE}_{\xi}$.
\item Morphism $\tau^{\sigma}$ satisfies (up to identifications described in Remark \ref{remark:unique_identification}) the identities
$$(\mu \times 1_X)^*\tau^{\sigma} = (1_{\bd{M}} \times  a)^*\tau^{\sigma} \cdot \pi_{2,3}^*\tau^{\sigma},\,\langle e, 1_{X} \rangle^*\tau^{\sigma} = 1_{\xi}$$
where $\mu: \bd{M} \times \bd{M} \ra \bd{M}$ is the multiplication on $\bd{M}$, $\pi_{2,3}:\bd{M}\times  \bd{M} \times  X \ra \bd{M} \times X$ is the projection on the last two factors and $e:\bd{1}\ra \bd{M}$ is the unit of $\bd{M}$.
\item Morphism $\tau^{\sigma}$ is an isomorphism and satisfies (up to identifications described in Remark \ref{remark:unique_identification}) the identity
$$(\mu \times 1_X)^*\tau^{\sigma} = (1_{\bd{M}} \times  a)^*\tau^{\sigma} \cdot \pi_{2,3}^*\tau^{\sigma}$$
where $\mu: \bd{M} \times \bd{M} \ra \bd{M}$ is the multiplication on $\bd{M}$, $\pi_{2,3}:\bd{M}\times  \bd{M} \times  X \ra \bd{M} \times X$ is the projection on the last two factors.
\end{enumerate}
Then $\textbf{\emph{(i)}}\Leftrightarrow \textbf{\emph{(ii)}}$ and $\textbf{\emph{(iii)}}$ implies both these assertions. Moreover, if $\bd{M}$ is a group object, then all three assertions are equivalent.
\end{proposition}
\begin{proof}
We prove that $\textbf{(i)}\Leftrightarrow \textbf{(ii)}$. Our first goal is to prove that
$$\sigma \cdot \left(1_{h^{\cB}_{\bd{M}}\cdot p} \times \sigma \right) = \sigma \cdot \left(1_{h^{\cB}_{\mu}\cdot p} \times 1_{h^{\cE}_{\xi}} \right)$$
if and only if
$$\left(1_{\bd{M}}\times a\right)^*\tau^{\sigma}\cdot \pi^*_{23}\tau^\sigma = \left(\mu\times 1_X\right)^*\tau^{\sigma}$$
First note that the commutative square of presheaves
\begin{center}
\begin{tikzpicture}
[description/.style={fill=white,inner sep=2pt}]
\matrix (m) [matrix of math nodes, row sep=3em, column sep=3em,text height=1.5ex, text depth=0.25ex] 
{ h^{\cB}_{\bd{M}}\cdot p \times  h_{a^*\xi}^{\cE} &  h^{\cB}_{\bd{M}}\cdot p \times  h_{\xi}^{\cE}   \\
 h^{\cB}_{\bd{M}\times \bd{M}\times X}\cdot p & h^{\cB}_{\bd{M}\times X}\cdot p  \\};
\path[->,line width=1.0pt,font=\scriptsize]
(m-1-1) edge node[above] {$ 1_{h^{\cB}_{\bd{M}}\cdot p} \times h^{\cE}_{\widetilde{a}_{\xi}} $} (m-1-2)
(m-2-1) edge node[below] {$ \left(h^{\cB}_{1_{\bd{M}}\times a}\right)_p $} (m-2-2)
(m-1-1) edge node[left]  {$ p_{\mathrm{hom}} $} (m-2-1)
(m-1-2) edge node[right] {$ p_{\mathrm{hom}} $} (m-2-2);
\end{tikzpicture}
\end{center}
on $\cE$ is cartesian. Next according to Fact \ref{fact:identification_of_product_and_pullback_along_the_projection} we infer that projections $$pr_{h^{\cE}_{a^*\xi}}:h^{\cB}_{\bd{M}}\cdot p \times  h_{a^*\xi}^{\cE} \ra h_{a^*\xi}^{\cE},\,pr_{h^{\cE}_{\xi}}:h^{\cB}_{\bd{M}}\cdot p \times  h_{\xi}^{\cE} \ra h_{\xi}^{\cE}$$
are representable by morphisms $\widetilde{\pi_{23}}_{a^*\xi}:\pi^*_{23}a^*\xi \ra a^*\xi,\,\widetilde{\pi}_{\xi}:\pi^*\xi\ra \xi$ in $\cE$, respectively. Thus $1_{h^{\cB}_{\bd{M}}\cdot p} \times h^{\cE}_{\widetilde{a}_{\xi}}$ is representable by a cartesian morphism
\begin{center}
\begin{tikzpicture}
[description/.style={fill=white,inner sep=2pt}]
\matrix (m) [matrix of math nodes, row sep=3em, column sep=3em,text height=1.5ex, text depth=0.25ex] 
{ \pi^*_{23}a^*\xi & \left(1_{\bd{M}}\times a \right)^*\pi^*\xi & \pi^*\xi   \\};
\path[->,line width=1.0pt,font=\scriptsize]
(m-1-1) edge node[above] {$ \cong $} (m-1-2)
(m-1-2) edge node[above] {$ \widetilde{\left(1_{\bd{M}}\times a\right)}_{\pi^*\xi} $} (m-1-3);
\end{tikzpicture}
\end{center}
where $\cong$ is the identification described in Remark \ref{remark:unique_identification}. Since we have equality
$$\sigma \cdot \left(1_{h^{\cB}_{\bd{M}}\cdot p} \times \sigma \right) = h^{\cE}_{\widetilde{a}_{\xi}}\cdot h^{\cE}_{\tau^{\sigma}}\cdot \left(1_{h^{\cB}_{\bd{M}}\cdot p} \times h^{\cE}_{\widetilde{a}_{\xi}} \right) \cdot \left(1_{h^{\cB}_{\bd{M}}\cdot p} \times h^{\cE}_{\tau^{\sigma}} \right)$$
we derive that $\sigma \cdot \left(1_{h^{\cB}_{\bd{M}}\cdot p} \times \sigma \right)$ is representable (again up to identifications of Remark \ref{remark:unique_identification}) by a morphism
$$\widetilde{a}_{\xi}\cdot \tau^{\sigma}\cdot \widetilde{\left(1_{\bd{M}}\times a\right)}_{\pi^*\xi}\cdot \pi^*_{23}\tau^\sigma = \widetilde{a}_{\xi} \cdot \widetilde{\left(1_{\bd{M}}\times a\right)}_{a^*\xi}\cdot \left(1_{\bd{M}}\times a\right)^*\tau^{\sigma}\cdot \pi^*_{23}\tau^\sigma$$
in $\cE$. Next note that the square of presheaves on $\cE$
\begin{center}
\begin{tikzpicture}
[description/.style={fill=white,inner sep=2pt}]
\matrix (m) [matrix of math nodes, row sep=3em, column sep=3em,text height=1.5ex, text depth=0.25ex] 
{ h^{\cB}_{\bd{M}}\cdot p \times h^{\cB}_{\bd{M}}\cdot p  \times h_{\xi}^{\cE} &  h^{\cB}_{{\bd{M}}\cdot p}\times h_{\xi}^{\cE}                \\
h^{\cB}_{\bd{M}\times \bd{M}\times X} \cdot p &  h^{\cB}_{{\bd{M}\times X}\cdot p}           \\} ;
\path[->,line width=1.0pt,font=\scriptsize]
(m-1-1) edge node[above] {$ h^{\cB}_{\mu}\cdot p \times 1_{h^{\cE}_{\xi}} $} (m-1-2)
(m-2-1) edge node[below] {$ \left(h^{\cB}_{\mu \times 1_X}\right)_p $} (m-2-2)
(m-1-1) edge node[left]  {$ \textbf{can}\times p_{\mathrm{hom}} $} (m-2-1)
(m-1-2) edge node[right] {$ p_{\mathrm{hom}} $} (m-2-2);
\end{tikzpicture}
\end{center}
is cartesian. According to Fact \ref{fact:identification_of_product_and_pullback_along_the_projection} we infer that projections $$pr_{h^{\cB}_{\bd{M}}\cdot p  \times h_{\xi}^{\cE}}:h^{\cB}_{\bd{M}}\cdot p \times h^{\cB}_{\bd{M}}\cdot p  \times h_{\xi}^{\cE} \ra h^{\cB}_{\bd{M}}\cdot p  \times h_{\xi}^{\cE},\,pr_{h^{\cE}_{\xi}}:h^{\cB}_{\bd{M}}\cdot p  \times h_{\xi}^{\cE} \ra h_{\xi}^{\cE}$$
are representable by morphisms $\widetilde{\pi_{23}}_{\pi^*\xi}:\pi^*_{23}\pi^*\xi \ra \pi^*\xi,\,\widetilde{\pi}_{\xi}:\pi^*\xi\ra \xi$ in $\cE$, respectively. Thus $h^{\cB}_{\mu}\cdot p \times 1_{h^{\cE}_{\xi}}$ is representable by a cartesian morphism
\begin{center}
\begin{tikzpicture}
[description/.style={fill=white,inner sep=2pt}]
\matrix (m) [matrix of math nodes, row sep=3em, column sep=3em,text height=1.5ex, text depth=0.25ex] 
{ \pi^*_{23}\pi^*\xi & (\mu \times 1_X)^*\pi^*\xi & \pi^*\xi   \\};
\path[->,line width=1.0pt,font=\scriptsize]
(m-1-1) edge node[above] {$ \cong $} (m-1-2)
(m-1-2) edge node[above] {$ \widetilde{\left(\mu\times 1_X\right)}_{\pi^*\xi} $} (m-1-3);
\end{tikzpicture}
\end{center}
where $\cong$ is the identification described in Remark \ref{remark:unique_identification}. Since we have equality
$$\sigma \cdot \left(1_{h^{\cB}_{\mu}\cdot p} \times 1_{h^{\cE}_{\xi}} \right) = h^{\cE}_{\widetilde{a}_{\xi}}\cdot h^{\cE}_{\tau^{\sigma}}\cdot \left(1_{h^{\cB}_{\mu}\cdot p} \times 1_{h^{\cE}_{\xi}} \right)$$
we derive that $\sigma \cdot \left(1_{h^{\cB}_{\mu}\cdot p} \times 1_{h^{\cE}_{\xi}} \right)$ is representable (again up to identifications of Remark \ref{remark:unique_identification}) by a morphism
$$\widetilde{a}_{\xi}\cdot \tau^{\sigma}\cdot \widetilde{\left(\mu\times 1_X\right)}_{\pi^*\xi} = \widetilde{a}_{\xi}\cdot \widetilde{\left(\mu\times 1_X\right)}_{a^*\xi}\cdot \left(\mu\times 1_X\right)^*\tau^{\sigma}$$
We deduce that
$$\sigma \cdot \left(1_{h^{\cB}_{\bd{M}}\cdot p} \times \sigma \right) = \sigma \cdot \left(1_{h^{\cB}_{\mu}\cdot p} \times 1_{h^{\cE}_{\xi}} \right)$$
if and only if
$$\widetilde{a}_{\xi} \cdot \widetilde{\left(1_{\bd{M}}\times a\right)}_{a^*\xi}\cdot \left(1_{\bd{M}}\times a\right)^*\tau^{\sigma}\cdot \pi^*_{23}\tau^\sigma =  \widetilde{a}_{\xi}\cdot \widetilde{\left(\mu\times 1_X\right)}_{a^*\xi}\cdot \left(\mu\times 1_X\right)^*\tau^{\sigma}$$
Since $a\cdot (1_{\bd{M}}\times a) = a\cdot (\mu \times 1_X)$ and according to Remark \ref{remark:unique_identification}, we have canonical identification $\widetilde{a}_{\xi} \cdot \widetilde{\left(1_{\bd{M}}\times a\right)}_{a^*\xi} = \widetilde{a}_{\xi}\cdot \widetilde{\left(\mu\times 1_X\right)}_{a^*\xi}$ of these cartesian morphisms. Therefore, we deduce that the formula above holds if and only if
$$\left(1_{\bd{M}}\times a\right)^*\tau^{\sigma}\cdot \pi^*_{23}\tau^\sigma = \left(\mu\times 1_X\right)^*\tau^{\sigma}$$
This proves our first claim. Now it suffices to prove that
$$\sigma \cdot \langle h^{\cB}_{e}\cdot p, 1_{h^{\cE}_{\xi}} \rangle = 1_{h^{\cE}_{\xi}}$$
if and only if $\langle e, 1_X\rangle^*\tau^{\sigma} = 1_{\xi}$. Note that the square of presheaves on $\cE$
\begin{center}
\begin{tikzpicture}
[description/.style={fill=white,inner sep=2pt}]
\matrix (m) [matrix of math nodes, row sep=3em, column sep=3em,text height=1.5ex, text depth=0.25ex] 
{ h_{\xi}^{\cE}       &  h^{\cB}_{{\bd{M}}\cdot p}\times h_{\xi}^{\cE}                \\
  h^{\cB}_{X} \cdot p &  h^{\cB}_{{\bd{M}\times X}\cdot p}           \\} ;
\path[->,line width=1.0pt,font=\scriptsize]
(m-1-1) edge node[above] {$ \langle h^{\cB}_{e}\cdot p, 1_{h^{\cE}_{\xi}} \rangle $} (m-1-2)
(m-2-1) edge node[below] {$ \left(h^{\cB}_{\langle e, 1_X\rangle}\right)_p $} (m-2-2)
(m-1-1) edge node[left]  {$ p_{\mathrm{hom}} $} (m-2-1)
(m-1-2) edge node[right] {$ p_{\mathrm{hom}} $} (m-2-2);
\end{tikzpicture}
\end{center}
is cartesian. Thus according to Fact \ref{fact:identification_of_product_and_pullback_along_the_projection} we derive that $\langle h^{\cB}_{e}\cdot p, 1_{h^{\cE}_{\xi}} \rangle$ is representable by a morphism
\begin{center}
\begin{tikzpicture}
[description/.style={fill=white,inner sep=2pt}]
\matrix (m) [matrix of math nodes, row sep=3em, column sep=3em,text height=1.5ex, text depth=0.25ex] 
{ \xi & \langle e, 1_X\rangle^*\pi^*\xi & \pi^*\xi   \\};
\path[->,line width=1.0pt,font=\scriptsize]
(m-1-1) edge node[above] {$ \cong $} (m-1-2)
(m-1-2) edge node[above] {$ \widetilde{\langle e, 1_X\rangle}_{\pi^*\xi} $} (m-1-3);
\end{tikzpicture}
\end{center}
where $\cong$ is the identification described in Remark \ref{remark:unique_identification}. Therefore, the morphism $\sigma \cdot \langle h^{\cB}_{e}\cdot p, 1_{h^{\cE}_{\xi}} \rangle$ is representable (up to identifications of Remark \ref{remark:unique_identification}) by
$$\widetilde{a}_{\xi}\cdot \tau^{\sigma}\cdot \widetilde{\langle e, 1_X\rangle}_{\pi^*\xi} = \widetilde{a}_{\xi}\cdot \widetilde{\langle e, 1_X\rangle}_{a^*\xi}\cdot \langle e, 1_X\rangle^*\tau^{\sigma} =  \langle e, 1_X\rangle^*\tau^{\sigma}$$
Thus
$$\sigma \cdot \langle h^{\cB}_{e}\cdot p, 1_{h^{\cE}_{\xi}} \rangle = 1_{h^{\cE}_{\xi}}$$
if and only if
$$\langle e, 1_X \rangle^*\tau^{\sigma} = 1_{\xi}$$
This finishes the proof of $\textbf{(i)}\Leftrightarrow \textbf{(ii)}$. Now we prove that \textbf{(iii)} implies \textbf{(ii)}. For this note that up to idenitification of Remark \ref{remark:unique_identification} we have
$$\langle e, 1_X\rangle^*\tau^{\sigma} = \langle e, e, 1_X\rangle^*\left(\mu\times 1_X\right)^*\tau^{\sigma} =  \langle e, e, 1_X\rangle^*\big(\left(1_{\bd{M}}\times a\right)^*\tau^{\sigma}\cdot \pi_{23}^*\tau^{\sigma}\big) = \langle e, 1_X\rangle^*\tau^{\sigma}\cdot \langle e, 1_X\rangle^*\tau^{\sigma}$$
Since $\tau^{\sigma}$ is an isomorphism, we derive that $\langle e, 1_X\rangle^*\tau^{\sigma} = 1_{\xi}$ and hence \textbf{(ii)} holds.\\
Finally we show that if $\bd{M}$ is a group object, then \textbf{(ii)} implies \textbf{(iii)}. For this note that 
\begin{center}
\begin{tikzpicture}
[description/.style={fill=white,inner sep=2pt}]
\matrix (m) [matrix of math nodes, row sep=3em, column sep=3em,text height=1.5ex, text depth=0.25ex] 
{ h^{\cB}_{{\bd{M}}}\cdot p \times h_{\xi}^{\cE}                &  h_{\xi}^{\cE}                \\
  h_{\bd{M}\times X}^{\cB}\cdot p           &  h_{X}^{\cB}\cdot p           \\} ;
\path[->,line width=1.0pt,font=\scriptsize]
(m-1-1) edge node[above] {$ \sigma $} (m-1-2)
(m-2-1) edge node[below] {$ \left(h^{\cB}_a\right)_p $} (m-2-2)
(m-1-1) edge node[left]  {$ p_{\mathrm{hom}} $} (m-2-1)
(m-1-2) edge node[right] {$ p_{\mathrm{hom}} $} (m-2-2);
\end{tikzpicture}
\end{center}
is a cartesian square in this case. Therefore, $\alpha^{\sigma}$ is a cartesian morphism of $p$. Since $\alpha^{\sigma} = \widetilde{a}_{\xi}\cdot \tau^{\sigma}$ and according to Fact \ref{fact:uniqueness_of_pullbacks}, we derive that $\tau^{\sigma}$ is an isomorphism. Thus \textbf{(iii)} holds.
\end{proof}

\begin{fact}\label{fact:morphisms_of_equivariant_objects}
Let $\bd{M},X$ be objects of $\cB$ such that the cartesian product of $\bd{M}$ and $X$ exist. Let $a:\bd{M}\times X\ra X$ be a morphism. Denote by $\pi:\bd{M}\times X\ra X$ the projection on $X$. Consider objects $\xi_1,\xi_2$ in $p^{-1}(X)$ and let $\sigma_1:h^{\cB}_\bd{M} \cdot p \times h^{\cE}_{\xi_1}\ra h^{\cE}_{\xi_1},\sigma_2:h^{\cB}_\bd{M} \cdot p \times h^{\cE}_{\xi_2} \ra h^{\cE}_{\xi_2}$ be morphisms of presheaves on $\cE$. Suppose that squares
\begin{center}
\begin{tikzpicture}
[description/.style={fill=white,inner sep=2pt}]
\matrix (m) [matrix of math nodes, row sep=3em, column sep=3em,text height=1.5ex, text depth=0.25ex] 
{ h^{\cB}_{{\bd{M}}}\cdot p \times h_{\xi_1}^{\cE} &  h_{\xi_1}^{\cE}      & h^{\cB}_{{\bd{M}}}\cdot p \times h_{\xi_2}^{\cE} & h_{\xi_2}^{\cE}     \\
  h_{\bd{M}\times X}^{\cB}\cdot p                &  h_{X}^{\cB}\cdot p     & h_{\bd{M}\times X}^{\cB}\cdot p                  &  h_{X}^{\cB}\cdot p \\} ;
\path[->,line width=1.0pt,font=\scriptsize]
(m-1-1) edge node[above] {$ \sigma_1 $} (m-1-2)
(m-2-1) edge node[below] {$ \left(h^{\cB}_a\right)_p $} (m-2-2)
(m-1-1) edge node[left]  {$ p_{\mathrm{hom}} $} (m-2-1)
(m-1-2) edge node[right] {$ p_{\mathrm{hom}} $} (m-2-2)

(m-1-3) edge node[above] {$ \sigma_2 $} (m-1-4)
(m-2-3) edge node[below] {$ \left(h^{\cB}_a\right)_p $} (m-2-4)
(m-1-3) edge node[left]  {$ p_{\mathrm{hom}} $} (m-2-3)
(m-1-4) edge node[right] {$ p_{\mathrm{hom}} $} (m-2-4);
\end{tikzpicture}
\end{center}
are commutative. Let $\phi:\xi_1\ra \xi_2$ be a morphism in $\cE$. Then the following assertions are equivalent.
\begin{enumerate}[label= \emph{\textbf{(\roman*)}}, leftmargin=3.0em]
\item The square
\begin{center}
\begin{tikzpicture}
[description/.style={fill=white,inner sep=2pt}]
\matrix (m) [matrix of math nodes, row sep=3em, column sep=3em,text height=1.5ex, text depth=0.25ex] 
{  h^{\cB}_{{\bd{M}}}\cdot p \times h_{\xi_1}^{\cE} &  h_{\xi_1}^{\cE}      \\
   h^{\cB}_{{\bd{M}}}\cdot p \times h_{\xi_2}^{\cE} &  h_{\xi_2}^{\cE}             \\} ;
\path[->,line width=1.0pt,font=\scriptsize]  
(m-1-1) edge node[above] {$ \sigma_1 $} (m-1-2)
(m-2-1) edge node[below] {$ \sigma_2 $} (m-2-2)
(m-1-1) edge node[left] {$ 1_{h^{\cB}_{\bd{M}}\times h^{\cE}_{\phi}} $} (m-2-1)
(m-1-2) edge node[right] {$ h^{\cE}_{\xi}  $} (m-2-2);
\end{tikzpicture}
\end{center}
is commutative.
\item The square
\begin{center}
\begin{tikzpicture}
[description/.style={fill=white,inner sep=2pt}]
\matrix (m) [matrix of math nodes, row sep=3em, column sep=3em,text height=1.5ex, text depth=0.25ex] 
{ \pi^*\xi_1 & a^*\xi_1     \\
  \pi^*\xi_2 & a^*\xi_2             \\} ;
\path[->,line width=1.0pt,font=\scriptsize]  
(m-1-1) edge node[above] {$ \tau^{\sigma_1} $} (m-1-2)
(m-2-1) edge node[below] {$ \tau^{\sigma_2} $} (m-2-2)
(m-1-1) edge node[left] {$ \pi^*\phi $} (m-2-1)
(m-1-2) edge node[right] {$ a^*\phi  $} (m-2-2);
\end{tikzpicture}
\end{center}
is commutative.
\end{enumerate}
\end{fact}
\begin{proof}
Note that up to identifications of Remark \ref{remark:unique_identification} and according to Fact \ref{fact:identification_of_product_and_pullback_along_the_projection}  morphism $h^{\cE}_{\phi}\cdot \sigma_1$ is representable by
$$\phi\cdot \alpha^{\sigma_1} = \phi\cdot \widetilde{a}_{\xi_1}\cdot \tau^{\sigma_1} = \widetilde{a}_{\xi_2}\cdot a^*\phi \cdot \tau^{\sigma_1}$$
and on the other hand morphism $\sigma_2\cdot \left(1_{h^{\cB}_{\bd{M}}\cdot p}\times h^{\cE}_{\phi}\right)$ is representable by
$$\alpha^{\sigma_2}\cdot \pi^*\phi = \widetilde{a}_{\xi_2}\cdot \tau^{\sigma_2}\cdot \pi^*\phi$$
Since $\widetilde{a}_{\xi_2}$ is cartesian with respect to $p$, we derive that
$$h^{\cE}_{\phi}\cdot \sigma_1 = \sigma_2\cdot \left(1_{h^{\cB}_{\bd{M}}\cdot p}\times h^{\cE}_{\phi}\right)$$
if and only if
$$a^*\phi \cdot \tau^{\sigma_1} = \tau^{\sigma_2}\cdot \pi^*\phi$$
This proves the assertion.
\end{proof}
\noindent
Guided by these two results we formulate a general notion of an equivariant object in a fibered category.

\begin{definition}
Let $\bd{M}$ be a monoid object in $\cB$ and let $X$ be an object of $\cB$ equipped with an action $a:\bd{M}\times X\ra X$ of $\bd{M}$ on $X$. Suppose that there is an action $\sigma:h^{\cB}_{\bd{M}} \cdot p \times h^{\cE}_\xi\ra h^{\cE}_\xi$ of a monoid presheaf $h^{\cB}_{\bd{M}} \cdot p$ on $h^{\cE}_\xi$ such that the square
\begin{center}
\begin{tikzpicture}
[description/.style={fill=white,inner sep=2pt}]
\matrix (m) [matrix of math nodes, row sep=3em, column sep=3em,text height=1.5ex, text depth=0.25ex] 
{ h^{\cB}_{{\bd{M}}}\cdot p \times h_{\xi}^{\cE}                &  h_{\xi}^{\cE}                \\
  h_{\bd{M}\times X}^{\cB}\cdot p           &  h_{X}^{\cB}\cdot p           \\} ;
\path[->,line width=1.0pt,font=\scriptsize]
(m-1-1) edge node[above] {$ \sigma $} (m-1-2)
(m-2-1) edge node[below] {$ \left(h^{\cB}_a\right)_p $} (m-2-2)
(m-1-1) edge node[left]  {$ p_{\mathrm{hom}} $} (m-2-1)
(m-1-2) edge node[right] {$ p_{\mathrm{hom}} $} (m-2-2);
\end{tikzpicture}
\end{center}
is commutative. Then a pair $(\xi,\sigma)$ is called \textit{an $\bd{M}$-equivariant object over $a$}.
\end{definition}

\begin{definition}
Let $\bd{M}$ be a monoid object in $\cB$ and let $X$ be an object of $\cB$ equipped with an action $a:\bd{M}\times X\ra X$ of $\bd{M}$ on $X$. Suppose that $(\xi_1,\sigma_1)$ and $(\xi_2,\sigma_2)$ are objects over $X$ with $\bd{M}$-equivariant structures. Then a morphism $\phi:\xi_1\ra \xi_2$ in $\cE$ is \textit{$\bd{M}$-equivariant} if the square
\begin{center}
\begin{tikzpicture}
[description/.style={fill=white,inner sep=2pt}]
\matrix (m) [matrix of math nodes, row sep=3em, column sep=3em,text height=1.5ex, text depth=0.25ex] 
{ h^{\cB}_{\bd{M}}\cdot p\times h_{\xi_1}^{\cE} &  h_{\xi_1}^{\cE}    \\
  h^{\cB}_{\bd{M}}\cdot p\times h_{\xi_2}^{\cE} &  h_{\xi_2}^{\cE}           \\} ;
\path[->,line width=1.0pt,font=\scriptsize]
(m-1-1) edge node[above] {$ \sigma_1  $} (m-1-2)
(m-2-1) edge node[below] {$ \sigma_2  $} (m-2-2)
(m-1-1) edge node[left]  {$ 1_{h^{\cB}_{\bd{M}} \cdot p}\times h^{\cE}_\phi $} (m-2-1)
(m-1-2) edge node[right] {$ \phi  $} (m-2-2);
\end{tikzpicture}
\end{center}
is commutative. By abuse of notation we denote by $p^{-1}_{\bd{M}}(X)$ \textit{the category of $\bd{M}$-equivariant objects over $a$ with respect to the fibered category $p$}.
\end{definition}
\noindent
Now we apply Proposition \ref{proposition:equivalent_description_of_equivariance_in_representable_case} and Fact \ref{fact:morphisms_of_equivariant_objects} to the fibered category $\Qcoh\ra \Sch_k$. We deduce the following.

\begin{corollary}\label{corollary:isomorphism_between_equivariant_quasi_coherent_sheaves_and_equivariant_objects_in_fibered_category_of_quasi_coherent_sheaves}
Suppose that $\bd{M}$ is a monoid $k$-scheme that acts on a $k$-scheme $X$ through a morphism $a:\bd{M}\times_k X\ra X$ of $k$-schemes. Then the category $\Qcoh_{\bd{M}}(X)$ is isomorphic to the category of $\bd{M}$-equivariant objects over $a$ with respect to the fibered category $\Qcoh\ra \Sch_k$.
\end{corollary}

\section{Example: The fibered category of equivariant objects over the category of actions}\label{section:fibered_category_over_the_category_of_actions}
\noindent
Fix a category $\cB$ with finite products and a monoid object $\bd{M}$ in $\cB$. We define a category $\bd{Act}_{\bd{M}}(\cB)$ as follows. An object of $\bd{Act}_{\bd{M}}(\cB)$ is a pair $(X,a)$ such that $X$ is an object of $\cB$ and $a:\bd{M}\times X \ra X$ is an action of $\bd{M}$ on $X$. Next if $(X,a)$ and $(Y,b)$ are objects of $\bd{Act}_{M}(\cB)$, then a morphism in $\bd{Act}_{M}(\cB)$ between them consists of a morphism $f:X\ra Y$ such that the square
\begin{center}
\begin{tikzpicture}
[description/.style={fill=white,inner sep=2pt}]
\matrix (m) [matrix of math nodes, row sep=3em, column sep=3em,text height=1.5ex, text depth=0.25ex] 
{ \bd{M} \times X &  X    \\
  \bd{M} \times Y &  Y         \\} ;
\path[->,line width=1.0pt,font=\scriptsize]
(m-1-1) edge node[above] {$ a  $} (m-1-2)
(m-2-1) edge node[below] {$ b  $} (m-2-2)
(m-1-1) edge node[left]  {$ 1_{\bd{M}}\times f $} (m-2-1)
(m-1-2) edge node[right] {$ f  $} (m-2-2);
\end{tikzpicture}
\end{center}
is commutative.

\begin{definition}
Let $\cB$ be a category with finite products and let $\bd{M}$ be a monoid object in $\cB$. We call $\bd{Act}_{\bd{M}}(\cB)$ \textit{the category of actions of $\bd{M}$ in $\cB$}.
\end{definition}
\noindent
Let us note the following result.

\begin{proposition}\label{proposition:creation_of_limits_for_actions_of_monoids}
Let $\cB$ be a category with finite limits and let $\bd{M}$ be a monoid object on $\cB$. The forgetful functor $\bd{Act}_{\bd{M}}(\cB)\ra \cB$ creates limits.
\end{proposition}
\begin{proof}
We left the proof for the reader as an exercise.
\end{proof}

\begin{corollary}\label{corollary:creation_of_limits_for_actions_on_presheaf_category}
Let $\cB$ be a category and let $\bd{M}$ be a monoid object on $\cB$. Then the forgetful functor $\bd{Act}_{h^{\cB}_{\bd{M}}}\left(\widehat{\cB}\right)\ra \widehat{\cB}$ creates limits.
\end{corollary}
\noindent
We are ready to start the main discussion of this section. Let $p:\cE\ra \cB$ be a fibered category and let $\bd{M}$ be a monoid object in $\cB$. Assume that $\cB$ has finite products. We construct a fibered category $p_{\bd{M}}:\cE_\bd{M}\ra \bd{Act}_{\bd{M}}(\cB)$ out of $p:\cE\ra \cB$ and $\bd{M}$. First we construct a category $\cE_\bd{M}$. The class of objects of $\cE_\bd{M}$ is the union of all $\bd{M}$-equivariant objects over objects of $\bd{Act}_\bd{M}(\cB)$. Now suppose that $(\xi,\sigma)$ and $(\eta,\rho)$ are $\bd{M}$-equivariant object over objects $(X,a)$ and $(Y,b)$ in $\bd{Act}_{\bd{M}}(\cB)$, respectively. A morphism in $\cE_\bd{M}$ between these objects consists of a morphism $\phi:\xi\ra \eta$ in $\cE$ such that $h^{\cE}_{\phi}$ is $h^{\cB}_{\bd{M}}\cdot p$-equivariant with respect to $\sigma$ and $\rho$. We also have a functor $p_\bd{M}:\cE_\bd{M}\ra \bd{Act}_{\bd{M}}(\cB)$ that sends $\phi$ to $p(\phi)$ (clearly $p(\phi)$ is $\bd{M}$-equivariant with respect to $a$ and $b$). This means that we have a square diagram
\begin{center}
\begin{tikzpicture}
[description/.style={fill=white,inner sep=2pt}]
\matrix (m) [matrix of math nodes, row sep=3em, column sep=3em,text height=1.5ex, text depth=0.25ex] 
{ (\xi,\sigma)      &  (\eta,\rho)      \\
  (X,a)        &  (Y,b)         \\} ;
\path[->,line width=1.0pt,font=\scriptsize]
(m-1-1) edge node[above] {$ \phi $} (m-1-2)
(m-2-1) edge node[below] {$ p(\phi) $} (m-2-2);
\path[|->,line width=1.0pt,font=\scriptsize]
(m-1-1) edge node[left]  {$  $} (m-2-1)
(m-1-2) edge node[right] {$  $} (m-2-2);
\end{tikzpicture}
\end{center}
Next let us assume that $f:X\ra Y$ is a morphism of $\bd{Act}_{\bd{M}}(\cB)$ between $(X,a)$ and $(Y,b)$. Suppose that $(\eta,\rho)$ is object with $\bd{M}$-equivariant structure over $b$ and consider a cartesian square
\begin{center}
\begin{tikzpicture}
[description/.style={fill=white,inner sep=2pt}]
\matrix (m) [matrix of math nodes, row sep=3em, column sep=3em,text height=1.5ex, text depth=0.25ex] 
{ f^*\eta      &  \eta      \\
  X        &  Y         \\} ;
\path[->,line width=1.0pt,font=\scriptsize]
(m-1-1) edge node[above] {$ \widetilde{f}_{\eta} $} (m-1-2)
(m-2-1) edge node[below] {$ f $} (m-2-2);
\path[|->,line width=1.0pt,font=\scriptsize]
(m-1-1) edge node[left]  {$  $} (m-2-1)
(m-1-2) edge node[right] {$  $} (m-2-2);
\end{tikzpicture}
\end{center}
with respect to $p$. Then by Corollary \ref{corollary:creation_of_limits_for_actions_on_presheaf_category} there exists a unique $\bd{M}$-equivariant structure $(f^*\eta,\sigma)$ that gives rise to a square diagram
\begin{center}
\begin{tikzpicture}
[description/.style={fill=white,inner sep=2pt}]
\matrix (m) [matrix of math nodes, row sep=3em, column sep=3em,text height=1.5ex, text depth=0.25ex] 
{ (f^*\eta,\sigma)    &  (\eta,\rho)      \\
  (X,a)          &  (Y,b)         \\} ;
\path[->,line width=1.0pt,font=\scriptsize]
(m-1-1) edge node[above] {$ \widetilde{f}_{\eta} $} (m-1-2)
(m-2-1) edge node[below] {$ f $} (m-2-2);
\path[|->,line width=1.0pt,font=\scriptsize]
(m-1-1) edge node[left]  {$  $} (m-2-1)
(m-1-2) edge node[right] {$  $} (m-2-2);
\end{tikzpicture}
\end{center}
with respect to $p_\bd{M}$. This square is cartesian with respect to $p_\bd{M}$.

\begin{definition}
Let $p:\cE \ra \cB$ be a fibered category and let $\bd{M}$ be a presheaf of monoids on $\cB$. The fibered category $p_\bd{M}:\cE_\bd{M}\ra \bd{Act}_\bd{M}(\cB)$ is called \textit{the fibered category of $\bd{M}$-equivariant objects}.
\end{definition}

\begin{example}\label{example:the_fibered_category_of_equivariant_objects_with_respect_to_a_monoid_in_category_of_arrows}
Let $\cB$ be a category with finite limits and let $\bd{M}$ be a monoid object of $\cB$. Consider the fibered category of arrows $p_{\mathrm{Arr}(\cB)}:\mathrm{Arr}(\cB)\ra \cB$. By Proposition \ref{proposition:creation_of_limits_for_actions_of_monoids} the category $\bd{Act}_{\bd{M}}(\cB)$ admits finite limits and there exists an isomorphism
\begin{center}
\begin{tikzpicture}
[description/.style={fill=white,inner sep=2pt}]
\matrix (m) [matrix of math nodes, row sep=2em, column sep=1em,text height=1.5ex, text depth=0.25ex] 
{  \mathrm{Arr}(\cB)_{\bd{M}} &        & \mathrm{Arr}(\bd{Act}_{\bd{M}}(\cB))  \\
          &\bd{Act}_{\bd{M}}(\cB) &  \\} ;
\path[->,line width=1.0pt,font=\scriptsize]
(m-1-1) edge node[above] {$ \cong $} (m-1-3)

(m-1-1) edge node[below = 6pt, left = 1pt] {$ \left(p_{\mathrm{Arr}(\cB)}\right)_{\bd{M}} $} (m-2-2)
(m-1-3) edge node[below = 6pt, right = 1pt] {$ p_{\mathrm{Arr}(\bd{Act}_{\bd{M}}(\cB))} $} (m-2-2);
\end{tikzpicture}
\end{center}
of fibered categories.
\end{example}
\noindent
It is useful to describe $\sigma$ in terms described in Proposition \ref{proposition:equivalent_description_of_equivariance_in_representable_case}.

\begin{proposition}\label{proposition:equivariant_pullbacks_description}
Let $p:\cE \ra \cB$ be a fibered category and let $\bd{M}$ be a monoid object of $\cB$. Suppose that $(X,a)$ and $(Y,b)$ are objects of $\bd{Act}_{\bd{M}}(\cB)$ and $f:X\ra Y$ is a morphism in this category. Let $(\eta,\rho)$ be an $\bd{M}$-equivariant object over $b$ and let
\begin{center}
\begin{tikzpicture}
[description/.style={fill=white,inner sep=2pt}]
\matrix (m) [matrix of math nodes, row sep=3em, column sep=3em,text height=1.5ex, text depth=0.25ex] 
{ (f^*\eta,\sigma)    &  (\eta,\rho)      \\
  (X,a)               &  (Y,b)         \\} ;
\path[->,line width=1.0pt,font=\scriptsize]
(m-1-1) edge node[above] {$ \widetilde{f}_{\eta} $} (m-1-2)
(m-2-1) edge node[below] {$ f $} (m-2-2);
\path[|->,line width=1.0pt,font=\scriptsize]
(m-1-1) edge node[left]  {$  $} (m-2-1)
(m-1-2) edge node[right] {$  $} (m-2-2);
\end{tikzpicture}
\end{center}
be a cartesian square with respect to $p_{\bd{M}}:\cE_{\bd{M}}\ra \bd{Act}_{\bd{M}}\left(\cB\right)$. Then (under identifications of Remark \ref{remark:unique_identification}) we have
$$\tau^{\sigma} = \left(1_{\bd{M}}\times f\right)^*\tau^{\rho}$$
\end{proposition}
\begin{proof}
We use the same notational convention as in Proposition \ref{proposition:equivalent_description_of_equivariance_in_representable_case}. We denote by $\pi_X:\bd{M}\times X\ra X$ and $\pi_Y:\bd{M}\times Y\ra Y$ projections on $X$ and $Y$, respectively. Now $\alpha^{\sigma}:\pi^*_Xf^*\eta \ra a^*f^*\eta$ is the unique morphism in $\cE$ such that $p(\alpha^\sigma) = a$ and (up to identifications described in Remark \ref{remark:unique_identification})
$$\tilde{f}_{\eta}  \cdot \alpha^{\sigma} = \alpha^{\rho}\cdot \widetilde{\left(1_{\bd{M}}\times f\right)}_{\pi^*_Y\eta}$$
Under identifications described in Remark \ref{remark:unique_identification} and definitions of $\alpha^{\sigma},\alpha^{\rho}$ we have
$$\tilde{f}_{\eta} \cdot \tilde{a}_{f^*\eta} \cdot \tau^{\sigma} = \tilde{f}_{\eta}  \cdot \alpha^{\sigma} = \alpha^{\rho}\cdot \widetilde{\left(1_{\bd{M}}\times f\right)}_{\pi^*_Y\eta} = \tilde{b}_{\eta} \cdot \tau^{\rho}\cdot \widetilde{\left(1_{\bd{M}}\times f\right)}_{\pi^*_Y\eta} = \tilde{b}_{\eta}\cdot \widetilde{\left(1_{\bd{M}}\times f\right)}_{b^*\eta} \cdot \left(1_{\bd{M}}\times f\right)^*\tau^{\rho}$$
and
$$\tilde{f}_{\eta} \cdot \tilde{a}_{f^*\eta} = \tilde{b}_{\eta}\cdot \widetilde{\left(1_{\bd{M}}\times f\right)}_{b^*\eta}$$
also this last morphism is cartesian with respect to $p$. Thus we derive that $\tau^{\sigma} = \left(1_{\bd{M}}\times f\right)^*\tau^{\rho}$.
\end{proof}

\section{Applications to equivariant quasi-coherent sheaves}
\noindent
This section is devoted to applications in algebraic geometry. We fix a commutative ring $k$ and a monoid $k$-scheme $\bd{M}$. Let us first note the following consequence of Proposition \ref{proposition:equivariant_pullbacks_description}.

\begin{corollary}\label{corollary:pullbacks_of_equivariant_sheaves}
Let $X,Y$ be $k$-schemes equipped with actions of $\bd{M}$. Suppose that $f:X\ra Y$ is an $\bd{M}$-equivariant morphism. Then $f^*:\Qcoh(Y)\ra \Qcoh(X)$ induces a functor $f^*:\Qcoh_{\bd{M}}(Y)\ra \Qcoh_{\bd{M}}(X)$ that sends a quasi-coherent $\bd{M}$-sheaf $(\cG,\tau)$ on $Y$ to $\big(f^*\cG,\left(1_{\bd{M}}\times_k f\right)^*\tau\big)$.
\end{corollary}
\noindent
We are interested in a fibered category $\Alg(\Qcoh)_{\bd{M}}\ra \bd{Act}_{\bd{M}}\left(\Sch_k\right)$ constructed from the fibered category $\Alg(\Qcoh)\ra \Sch_k$ by means of procedure described in Section \ref{section:fibered_category_over_the_category_of_actions}. 

\begin{definition}
Let $X$ be a $k$-scheme equipped with $\bd{M}$-action $a:\bd{M}\times_kX\ra X$. By abuse of notation we denote the fiber of $\Alg(\Qcoh)_{\bd{M}}\ra \Sch_k$ over $(X,a)$ by $\Alg(\Qcoh)_{\bd{M}}(X)$. We call it \textit{the category of quasi-coherent of $\bd{M}$-algebras on $(X,a)$}.
\end{definition}

\begin{remark}\label{remark:description_of_equivariant_qc_algebras_in_the_usual_way}
We describe $\Alg(\Qcoh)_{\bd{M}}(X)$ for $X$ with $\bd{M}$-action $a$ in terms that are used in algebraic geometry. We denote by $\pi:\bd{M}\times_kX\ra X$ the projection. By Proposition \ref{proposition:equivalent_description_of_equivariance_in_representable_case} an object of $\Alg\left(\Qcoh\right)_{\bd{M}}(X)$ (that is a quasi-coherent $\bd{M}$-algebra) is a pair $(\cA,\tau)$ consisting of a quasi-coherent algebra $\cA$ on $X$ and a morphism $\tau:\pi^*\cA \ra a^* \cA$ such that the following equalities
$$(\mu \times 1_X)^*\tau = (1_{\bd{M}} \times  a)^*\tau \cdot \pi_{2,3}^*\tau,\,\langle e, 1_{X} \rangle^*\tau = 1_{\cA}$$
hold, where $\mu: \bd{M} \times_k \bd{M} \ra \bd{M}$ is the multiplication on $\bd{M}$, $\pi_{2,3}:\bd{M}\times_k  \bd{M} \times_k  X \ra \bd{M} \times_k X$ is the projection on the last two factors and $e:\bd{1}\ra \bd{M}$ is the unit of $\bd{M}$. Next according to Fact \ref{fact:morphisms_of_equivariant_objects} a morphism in $\Alg\left(\Qcoh\right)_{\bd{M}}(X)$ between $(\cA_1,\tau_1)$ and $(\cA_2,\tau_2)$ is a morphism $\phi:\cA_1\ra \cA_2$ of quasi-coherent sheaves on $X$ such that the square
\begin{center}
\begin{tikzpicture}
[description/.style={fill=white,inner sep=2pt}]
\matrix (m) [matrix of math nodes, row sep=3em, column sep=3em,text height=1.5ex, text depth=0.25ex] 
{ \pi^*\cA_1 & a^*\cA_1     \\
  \pi^*\cA_2 & a^*\cA_2             \\} ;
\path[->,line width=1.0pt,font=\scriptsize]  
(m-1-1) edge node[above] {$ \tau_1 $} (m-1-2)
(m-2-1) edge node[below] {$ \tau_2 $} (m-2-2)
(m-1-1) edge node[left] {$ \pi^*\phi $} (m-2-1)
(m-1-2) edge node[right] {$ a^*\phi  $} (m-2-2);
\end{tikzpicture}
\end{center}
is commutative.
\end{remark}

\begin{remark}\label{remark:equivariant_objects_in_fibered_category_of_relative_affine_schemes}
We describe the category $\bd{Aff}(\Sch_k)_{\bd{M}} \ra \bd{Act}_{\bd{M}}\left(\Sch_k\right)$ obtained from the fibered category $\bd{Aff}(\Sch_k) \ra \Sch_k$ (Example \ref{example:fibered_category_of_affine_schemes}) by means of construction described in Section \ref{section:fibered_category_over_the_category_of_actions}. $\bd{Aff}\left(\Sch_k\right)_{\bd{M}}$ is a full fibered subcategory of $\bd{Act}_{\bd{M}}\left(\mathrm{Arr}\left(\Sch_k\right)\right)$. Indeed, this follows from Examples \ref{example:fibered_category_of_affine_schemes} and \ref{example:the_fibered_category_of_equivariant_objects_with_respect_to_a_monoid_in_category_of_arrows}. More concretely, if $X$ is a $k$-scheme and $a:\bd{M}\times_kX\ra X$ is an $\bd{M}$-action, then object of $\bd{Aff}(\Sch_k)_{\bd{M}}$ over $(X,a)$ consists of an affine morphism of $k$-schemes $\psi:\widetilde{X}\ra X$ and an action $b:\bd{M}\times \widetilde{X}\ra \widetilde{X}$ such that the square
\begin{center}
\begin{tikzpicture}
[description/.style={fill=white,inner sep=2pt}]
\matrix (m) [matrix of math nodes, row sep=3em, column sep=3em,text height=1.5ex, text depth=0.25ex] 
{ \bd{M}\times_k\widetilde{X} &  \widetilde{X}      \\
  \bd{M}\times_kX             &  X         \\} ;
\path[->,line width=1.0pt,font=\scriptsize]
(m-1-1) edge node[above] {$b  $} (m-1-2)
(m-2-1) edge node[below] {$a  $} (m-2-2)
(m-1-1) edge node[left]  {$1_{\bd{M}}\times_k\psi  $} (m-2-1)
(m-1-2) edge node[right] {$\psi  $} (m-2-2);
\end{tikzpicture}
\end{center}
is commutative.
\end{remark}
\noindent
Let us note the following result.

\begin{corollary}\label{corollary:equivariant_fibered_categories_of_quasi_coherent_algebras_and_relatively_affine_schemes_are_equivalent}
The equivalence of fibered categories (Example \ref{example:from_qc_algebras_to_the_category_of_arrows_over_schemes})
\begin{center}
\begin{tikzpicture}
[description/.style={fill=white,inner sep=2pt}]
\matrix (m) [matrix of math nodes, row sep=2em, column sep=1em,text height=1.5ex, text depth=0.25ex] 
{  \Alg\left(\Qcoh\right) &        & \bd{Aff}(\Sch_k)  \\
          &\Sch_k &  \\} ;
\path[->,line width=1.0pt,font=\scriptsize]
(m-1-1) edge node[above] {$ \Spec  $} (m-1-3)
(m-1-1) edge node[swap] {$  $} (m-2-2)
(m-1-3) edge node[below = 6pt, right = 1pt] {$ $} (m-2-2);
\end{tikzpicture}
\end{center}
induces an equivalence of fibered categories
\begin{center}
\begin{tikzpicture}
[description/.style={fill=white,inner sep=2pt}]
\matrix (m) [matrix of math nodes, row sep=2em, column sep=1em,text height=1.5ex, text depth=0.25ex] 
{  \Alg\left(\Qcoh\right)_{\bd{M}} &        & \bd{Aff}(\Sch_k)_{\bd{M}}  \\
          &\bd{Act}_{\bd{M}}\left(\Sch_k\right) &  \\} ;
\path[->,line width=1.0pt,font=\scriptsize]
(m-1-1) edge node[above] {$  $} (m-1-3)
(m-1-1) edge node[swap] {$  $} (m-2-2)
(m-1-3) edge node[below = 6pt, right = 1pt] {$ $} (m-2-2);
\end{tikzpicture}
\end{center}
of $\bd{M}$-equivariant objects.
\end{corollary}

\begin{remark}\label{remark:identification_of_spec_of_pullback_along_pi_with_product}
Let $X$ be a $k$-scheme and denote by $\pi:\bd{M}\times_kX\ra X$ the projection. We fix a quasi-coherent algebra $\cA$ on $X$. By Remark \ref{remark:cartesian_squares_for_relative_spectra} we choose a cartesian square
\begin{center}
\begin{tikzpicture}
[description/.style={fill=white,inner sep=2pt}]
\matrix (m) [matrix of math nodes, row sep=3em, column sep=3em,text height=1.5ex, text depth=0.25ex] 
{ \Spec_{\bd{M}\times_kX}\,\pi^*\cA &  \Spec_X\cA      \\
  \bd{M}\times_kX             &  X         \\} ;
\path[->,line width=1.0pt,font=\scriptsize]
(m-1-1) edge node[above] {$\widetilde{\pi}  $} (m-1-2)
(m-2-1) edge node[below] {$\pi  $} (m-2-2)
(m-1-1) edge node[left]  {$  $} (m-2-1)
(m-1-2) edge node[right] {$  $} (m-2-2);
\end{tikzpicture}
\end{center}
in $\Sch_k$. Note that there is a unique isomorphism $\bd{M}\times_k\Spec_X\cA \cong \Spec_{\bd{M}\times_kX}\pi^*\cA$ such that its composition with $\widetilde{\pi}$ is the projection $\bd{M}\times_k\Spec_X\cA\ra \Spec_X\cA$ on the second factor.
\end{remark}
\noindent
Corollary \ref{corollary:equivariant_fibered_categories_of_quasi_coherent_algebras_and_relatively_affine_schemes_are_equivalent} can be stated in a more concrete form by means of language described in Remark \ref{remark:description_of_equivariant_qc_algebras_in_the_usual_way}. We realize this translation in the next two corollaries.

\begin{corollary}\label{corollary:equivariant_qc_algebras_bijective_with_actions_on_their_relative_specs}
Let $X$ be a $k$-scheme equipped with an action $a:\bd{M}\times_kX\ra X$ of $\bd{M}$ and denote by $\pi:\bd{M}\times_kX\ra X$ the projection. Suppose that $\cA$ is a quasi-coherent algebra on $X$. Consider the cartesian square
\begin{center}
\begin{tikzpicture}
[description/.style={fill=white,inner sep=2pt}]
\matrix (m) [matrix of math nodes, row sep=3em, column sep=3em,text height=1.5ex, text depth=0.25ex] 
{ \Spec_{\bd{M}\times_kX}\,a^*\cA &  \Spec_X\cA      \\
  \bd{M}\times_kX             &  X         \\} ;
\path[->,line width=1.0pt,font=\scriptsize]
(m-1-1) edge node[above] {$\widetilde{a}  $} (m-1-2)
(m-2-1) edge node[below] {$a  $} (m-2-2)
(m-1-1) edge node[left]  {$  $} (m-2-1)
(m-1-2) edge node[right] {$  $} (m-2-2);
\end{tikzpicture}
\end{center}
in $\Sch_k$. Then the map
$$\tau \mapsto \widetilde{a}\cdot \Spec_{\bd{M}\times_k X}\tau$$
together with the unique isomorphism $\bd{M}\times_k\Spec_X\cA \cong \Spec_{\bd{M}\times_kX}\pi^*\cA$ described above (Remark \ref{remark:identification_of_spec_of_pullback_along_pi_with_product}) gives rise to a bijection between the class of structures $(\cA,\tau)$ of quasi-coherent $\bd{M}$-algebras on $\cA$ and the class of actions $b:\bd{M}\times_k\Spec_X\cA\ra \Spec_X\cA$ of $\bd{M}$ such that $b$ make the square
\begin{center}
\begin{tikzpicture}
[description/.style={fill=white,inner sep=2pt}]
\matrix (m) [matrix of math nodes, row sep=3em, column sep=3em,text height=1.5ex, text depth=0.25ex] 
{ \bd{M}\times_k\Spec_X\cA &  \Spec_X\cA      \\
  \bd{M}\times_kX             &  X         \\} ;
\path[->,line width=1.0pt,font=\scriptsize]
(m-1-1) edge node[above] {$b  $} (m-1-2)
(m-2-1) edge node[below] {$a  $} (m-2-2)
(m-1-1) edge node[left]  {$  $} (m-2-1)
(m-1-2) edge node[right] {$  $} (m-2-2);
\end{tikzpicture}
\end{center}
commutative.
\end{corollary}

\begin{corollary}\label{corollary:morphisms_of_qc_equivariant_algebras_are_morphisms_of_schemes_with_actions}
Let $X$ be a $k$-scheme equipped with an action $a:\bd{M}\times_kX\ra X$ of $\bd{M}$ and denote by $\pi:\bd{M}\times_kX\ra X$ the projection. Suppose that $(\cA_1,\tau_1)$ and $(\cA_2,\tau_2)$ are quasi-coherent $\bd{M}$-algebras on $X$. Consider cartesian squares
\begin{center}
\begin{tikzpicture}
[description/.style={fill=white,inner sep=2pt}]
\matrix (m) [matrix of math nodes, row sep=3em, column sep=3em,text height=1.5ex, text depth=0.25ex] 
{ \Spec_{\bd{M}\times_kX}\,a^*\cA_1 &  \Spec_X\cA_1 & \Spec_{\bd{M}\times_kX}\,a^*\cA_2 & \Spec_X\cA_2     \\
  \bd{M}\times_kX                   &  X            & \bd{M}\times_kX                   &  X\\} ;
\path[->,line width=1.0pt,font=\scriptsize]
(m-1-1) edge node[above] {$\widetilde{a}_1  $} (m-1-2)
(m-2-1) edge node[below] {$a  $} (m-2-2)
(m-1-1) edge node[left]  {$  $} (m-2-1)
(m-1-2) edge node[right] {$  $} (m-2-2)

(m-1-3) edge node[above] {$\widetilde{a}_2  $} (m-1-4)
(m-2-3) edge node[below] {$a  $} (m-2-4)
(m-1-3) edge node[left]  {$  $} (m-2-3)
(m-1-4) edge node[right] {$  $} (m-2-4);
\end{tikzpicture}
\end{center}
in $\Sch_k$. Then the map
$$\phi \mapsto \Spec_X\phi$$
induces a bijection between the class of morphisms $\phi:\cA_1\ra \cA_2$ of quasi-coherent $\bd{M}$-algebras (with respect to $\tau_1$ and $\tau_2$) and the class of morphism of $k$-schemes with $\bd{M}$-actions $\left(\Spec_X\cA_1,\widetilde{a}_1\cdot \Spec_{\bd{M}\times_kX}\,\tau_1\right)\ra \left(\Spec_X\cA_1,\widetilde{a}_1\cdot \Spec_{\bd{M}\times_kX}\,\tau_1\right)$.
\end{corollary}
\begin{proof}[Proof of Corollaries \ref{corollary:equivariant_qc_algebras_bijective_with_actions_on_their_relative_specs} and \ref{corollary:morphisms_of_qc_equivariant_algebras_are_morphisms_of_schemes_with_actions}]
The two results above are consequences of Corollary \ref{corollary:equivariant_fibered_categories_of_quasi_coherent_algebras_and_relatively_affine_schemes_are_equivalent} and Remarks \ref{remark:description_of_equivariant_qc_algebras_in_the_usual_way}, \ref{remark:equivariant_objects_in_fibered_category_of_relative_affine_schemes}.
\end{proof}
\noindent
Corollary \ref{corollary:equivariant_qc_algebras_bijective_with_actions_on_their_relative_specs} has a consequence which makes rigorous the intuitive statement that if $X$ is a $k$-scheme equipped with an action $a:\bd{M}\times_kX\ra X$ of $\bd{M}$, then there is a canonical structure of quasi-coherent $\bd{M}$-sheaf on $\cO_X$. Indeed, observe that $\Spec_X\cO_X$ can be canonically identified with $X$ and the square
\begin{center}
\begin{tikzpicture}
[description/.style={fill=white,inner sep=2pt}]
\matrix (m) [matrix of math nodes, row sep=3em, column sep=3em,text height=1.5ex, text depth=0.25ex] 
{ \bd{M}\times_kX &  X      \\
  \bd{M}\times_kX &  X         \\} ;
\path[->,line width=1.0pt,font=\scriptsize]
(m-1-1) edge node[above] {$a  $} (m-1-2)
(m-2-1) edge node[below] {$a  $} (m-2-2)
(m-1-1) edge node[left]  {$1_{\bd{M}}\times_k1_X  $} (m-2-1)
(m-1-2) edge node[right] {$1_X  $} (m-2-2);
\end{tikzpicture}
\end{center}
is commutative. Then Corollary \ref{corollary:equivariant_qc_algebras_bijective_with_actions_on_their_relative_specs} implies that there exists a corresponding structure of quasi-coherent $\bd{M}$-sheaf on $\cO_X$.

\begin{definition}
Let $X$ be a $k$-scheme equipped with an action $a:\bd{M}\times_kX\ra X$ of $\bd{M}$. Then the structure of $\bd{M}$-sheaf on $\cO_X$ described above is called \textit{the canonical structure of $\bd{M}$-sheaf on $\cO_X$}.
\end{definition}
\noindent
We now describe this canonical structure more concretely.

\begin{fact}\label{fact:description_of_canonical_equivariant_structure_on_structure_sheaf}
Let $X$ be a $k$-scheme equipped with an action $a:\bd{M}\times_kX\ra X$ of $\bd{M}$ and denote by $\pi:\bd{M}\times_kX\ra X$ the projection. Then the canonical structure of $\bd{M}$-sheaf on $\cO_X$ is determined by the morphism $a_{\#}^{-1}\cdot \pi_{\#}$.
\end{fact}
\begin{proof}
Consider cartesian squares
\begin{center}
\begin{tikzpicture}
[description/.style={fill=white,inner sep=2pt}]
\matrix (m) [matrix of math nodes, row sep=3em, column sep=3em,text height=1.5ex, text depth=0.25ex] 
{ \Spec_{\bd{M}\times_kX}\,\pi^*\cO_X &  \Spec_X\cO_X & \Spec_{\bd{M}\times_kX}a^*\cO_X &  \Spec_X\cO_X     \\
  \bd{M}\times_kX                     &  X            & \bd{M}\times_kX                 &  X      \\} ;
\path[->,line width=1.0pt,font=\scriptsize]
(m-1-1) edge node[above] {$\widetilde{\pi}  $} (m-1-2)
(m-2-1) edge node[below] {$\pi  $} (m-2-2)
(m-1-1) edge node[left]  {$  $} (m-2-1)
(m-1-2) edge node[right] {$  $} (m-2-2)

(m-1-3) edge node[above] {$\widetilde{a}  $} (m-1-4)
(m-2-3) edge node[below] {$a  $} (m-2-4)
(m-1-3) edge node[left]  {$  $} (m-2-3)
(m-1-4) edge node[right] {$  $} (m-2-4);
\end{tikzpicture}
\end{center}
in $\Sch_k$. We have canonical identifications $\Spec_X\cO_X = X$ and $\Spec_{\bd{M}\times_kX}\cO_{\bd{M}\times_kX} = \bd{M}\times_kX$. Under these identifications squares
\begin{center}
\begin{tikzpicture}
[description/.style={fill=white,inner sep=2pt}]
\matrix (m) [matrix of math nodes, row sep=3em, column sep=5em,text height=1.5ex, text depth=0.25ex] 
{ \bd{M}\times_kX &  X & \bd{M}\times_kX &  X     \\
  \bd{M}\times_kX &  X & \bd{M}\times_kX &  X      \\} ;
\path[->,line width=1.0pt,font=\scriptsize]
(m-1-1) edge node[above] {$\widetilde{\pi}\cdot \Spec_{\bd{M}\times_kX} \pi_{\#}^{-1}  $} (m-1-2)
(m-2-1) edge node[below] {$\pi  $} (m-2-2)
(m-1-1) edge node[left]  {$1_{\bd{M}\times_kX}  $} (m-2-1)
(m-1-2) edge node[right] {$1_X  $} (m-2-2)

(m-1-3) edge node[above] {$\widetilde{a}\cdot \Spec_{\bd{M}\times_kX} a_{\#}^{-1}  $} (m-1-4)
(m-2-3) edge node[below] {$a  $} (m-2-4)
(m-1-3) edge node[left]  {$1_{\bd{M}\times_kX}  $} (m-2-3)
(m-1-4) edge node[right] {$1_X  $} (m-2-4);
\end{tikzpicture}
\end{center}
are cartesian in $\Sch_k$. Thus
$$a = \widetilde{a}\cdot \Spec_{\bd{M}\times_kX}\left(a_{\#}^{-1}\right)$$
Next let $\tau:\pi^*\cO_X\ra a^*\cO_X$ be a morphism giving the canonical structure of $\bd{M}$-sheaf on $\cO_X$. Then we have $a = \widetilde{a}\cdot \Spec_{\bd{M}\times_kX}(\tau) \cdot \Spec_{\bd{M}\times_kX}\left(\pi_{\#}^{-1}\right)$ by definition of $\tau$ and Corollary \ref{corollary:equivariant_qc_algebras_bijective_with_actions_on_their_relative_specs}. Thus
$$\widetilde{a}\cdot \Spec_{\bd{M}\times_kX}\left(a_{\#}^{-1}\right) = \widetilde{a} \cdot \Spec_{\bd{M}\times_kX}\left(\tau\cdot \pi_{\#}^{-1}\right)$$
and this implies that $a_{\#}^{-1} = \tau\cdot \pi_{\#}^{-1}$. Hence $\tau = a^{-1}_{\#}\cdot \pi_{\#}$ because $\widetilde{a}$ is cartesian with respect to $\bd{Aff}\left(\Sch_k\right)\ra \Sch_k$ and the functor $\Spec_{\bd{M}\times_kX}:\Alg\big(\Qcoh\left(\bd{M}\times_kX\right)\big)\ra \bd{Aff}_{\bd{M}\times_kX}$ is an equivalence of categories.
\end{proof}
\noindent
Our next goal is to explain that the category of quasi-coherent $\bd{M}$-sheaves carries a canonical symmetric monoidal structure. 

\begin{remark}\label{remark:monoidal_structure_on_equivariant_sheaves}
Let $X$ be a $k$-scheme equipped with an action $a:\bd{M}\times_kX\ra X$ of $\bd{M}$ and denote by $\pi:\bd{M}\times_kX\ra X$ the projection. Let $(\cF_1,\tau_1)$ and $(\cF_2,\tau_2)$ be quasi-coherent $\bd{M}$-sheaves on $X$. Then 
$$\big(\cF_1\otimes_{\cO_X}\cF_2,\tau_1\otimes_{\cO_X}\tau_2\big)$$
is a quasi-coherent $\bd{M}$-sheaf on $X$. This shows that $(-)\otimes_{\cO_X}(-):\Qcoh(X)\times \Qcoh(X)\ra \Qcoh(X)$ admits a lift to a functor $\Qcoh_{\bd{M}}(X)\times \Qcoh_{\bd{M}}(X) \ra \Qcoh_{\bd{M}}(X)$. This bifunctor together with $\left(\cO_X,a_{\#}\cdot \pi_{\#}^{-1}\right)$ and lifts of the associator isomorphisms, left and right units isomorphism and the symmetry isomorphism for $\Qcoh(X)$ give rise to a symmetric monoidal structure on $\Qcoh_{\bd{M}}(X)$. Note that the forgetful functor $\Qcoh_{\bd{M}}(X)\ra \Qcoh(X)$ is symmetric monoidal functor. Moreover, we have the category of (commutative) algebras $\Alg\left(\Qcoh_{\bd{M}}(X)\right)$ in $\Qcoh_{\bd{M}}(X)$ with respect to this symmetric monoidal structure. By definition it follows that this category is isomorphic with $\Alg\left(\Qcoh\right)_{\bd{M}}(X)$. Finally, if $f:X\ra Y$ is an $\bd{M}$-equivariant morphism of $k$-schemes equipped with $\bd{M}$-actions, then $f^*:\Qcoh_{\bd{M}}(Y)\ra \Qcoh_{\bd{M}}(X)$ is a monoidal functor.
\end{remark}
\noindent
It is interesting to note that under mild assumptions the category of quasi-coherent $\bd{M}$-sheaves on a $k$-scheme with $\bd{M}$-action is an abelian category.

\begin{theorem}\label{theorem:forgetful_functor_from_equivariant_sheaves_to_qc_sheaves_creates_colimits_and_finite_limits}
Let $X$ be a $k$-scheme equipped with an action $a:\bd{M}\times_kX\ra X$ of $\bd{M}$. Consider the forgetful functor $\Qcoh_{\bd{M}}(X)\ra \Qcoh(X)$. Then the following assertions hold.
\begin{enumerate}[label= \emph{\textbf{(\arabic*)}}, leftmargin=3.0em]
\item The forgetful functor $\Qcoh_{\bd{M}}(X)\ra \Qcoh(X)$ creates colimits.
\item If $a$ is flat morphism, then the forgetful functor $\Qcoh_{\bd{M}}(X)\ra \Qcoh(X)$ creates finite limits.
\end{enumerate}
\end{theorem}
\begin{proof}
In the proof we denote by $\pi:\bd{M}\times_kX\ra X$ the projection and by $\pi_{23}:\bd{M}\times_k\bd{M}\times_k X\ra \bd{M}\times_k X$ the projection on the last two factors.\\
We prove that \textbf{(1)} holds. Consider a category $I$ and a functor $F:I\ra \Qcoh_{\bd{M}}(X)$. Let $F(i) = (\cF_i,\tau_i)$ for every $i\in I$. Assume that $\big\{u_i:\cF_i\ra \cF\big\}_{i\in I}$ is a colimiting cocone in $\Qcoh(X)$ for the composition of $F$ with the forgetful functor $\Qcoh_{\bd{M}}(X)\ra \Qcoh(X)$. Pick a morphism $\tau:\pi^*\cF\ra a^*\cF$ of quasi-coherent sheaves on $X$. We first show that the following assertions are equivalent.
\begin{enumerate}[label= \textbf{(\roman*)}, leftmargin=3.0em]
\item $(\cF,\tau)$ is a quasi-coherent $\bd{M}$-sheaf on $X$ and $\{u_i\}_{i\in I}$ are morphisms of quasi-coherent $\bd{M}$-sheaves with codomain $(\cF,\tau)$.
\item Squares
\begin{center}
\begin{tikzpicture}
[description/.style={fill=white,inner sep=2pt}]
\matrix (m) [matrix of math nodes, row sep=3em, column sep=3em,text height=1.5ex, text depth=0.25ex] 
{ \pi^*\cF_i   & \pi^*\cF     \\
  a^*\cF_i & a^*\cF   \\} ;
\path[->,line width=1.0pt,font=\scriptsize]
(m-1-1) edge node[above] {$ \pi^*u_i  $} (m-1-2)
(m-2-1) edge node[below] {$ a^*u_i $} (m-2-2)
(m-1-1) edge node[left]  {$ \tau_i $} (m-2-1)
(m-1-2) edge node[right] {$ \tau $} (m-2-2);
\end{tikzpicture}
\end{center}
are commutative for all $i$ in $I$.
\end{enumerate}
The implication $\textbf{(i)}\Rightarrow \textbf{(ii)}$ is a consequence of the definition of quasi-coherent $\bd{M}$-sheaves. We prove the implication $\textbf{(ii)}\Rightarrow \textbf{(i)}$. Assume that \textbf{(ii)} holds. Observe that the functor $\pi_{23}^*\pi^*$ preserves colimits and hence $\big\{\pi_{23}^*\pi^*u_i:\pi^*_{23}\pi^*\cF_i\ra \pi_{23}^*\pi^*\cF \big\}_{i\in I}$ is a colimiting cone for the composition of $F$, the forgetful functor $\Qcoh_{\bd{M}}(X)\ra \Qcoh(X)$ and $\pi_{23}^*\pi^*$. Note that we have
$$\left(\mu \times_k 1_X\right)^*\tau \cdot \pi^*_{23}\pi^*u_i = \left(\mu \times_k 1_X\right)^*\tau \cdot \left(\mu\times_k 1_X\right)^*\pi^*u_i = \left(\mu \times_k 1_X\right)^*a^*u_i \cdot \left(\mu \times_k 1_X\right)^*\tau_i =$$
$$=\left(\mu \times_k 1_X\right)^*a^*u_i \cdot \left(1_{\bd{M}}\times_k a\right)^*\tau_i \cdot \pi_{23}^*\tau_i = \left(1_{\bd{M}} \times_k a \right)^*a^*u_i \cdot \left(1_{\bd{M}}\times_k a\right)^*\tau_i \cdot \pi_{23}^*\tau_i =$$
$$= \left(1_{\bd{M}}\times_k a\right)^*\tau\cdot  \left(1_{\bd{M}}\times_k a\right)^*\pi^*u_i\cdot \pi^*_{23}\tau_i = \left(1_{\bd{M}}\times_k a\right)^*\tau \cdot  \pi_{23}^*a^*u_i \cdot \pi^*_{23}\tau_i = \left(1_{\bd{M}}\times_k a\right)^*\tau \cdot  \pi_{23}^*\tau \cdot \pi^*_{23}\pi^*u_i$$
for every $i$ in $I$. This shows that $\left(\mu \times_k 1_X\right)^*\tau =  \left(1_{\bd{M}}\times_k a\right)^*\tau \cdot  \pi_{23}^*\tau$ and hence $(\cF,\tau)$ is a quasi-coherent $\bd{M}$-sheaf. Now since \textbf{(ii)} holds, we derive that $\big\{u_i:\cF_i\ra \cF\big\}_{i\in I}$ are morphisms of quasi-coherent $\bd{M}$-sheaves with codomain $(\cF,\tau)$. This proves \textbf{(i)} and hence the proposition $\textbf{(ii)}\Rightarrow \textbf{(i)}$ holds.\\
We prove now that \textbf{(ii)} holds for a unique $\tau$. Indeed, if $\tau$ exists, then it is a morphism between cocones $\big\{a^*u_i:a^*\cF_i\ra a^*\cF\big\}_{i\in I}$ and $\big\{\pi^*u_i\cdot \tau_i:a^*\cF\ra \pi^*\cF\big\}_{i\in I}$ for the composition of $F$, the forgetful functor $\Qcoh_{\bd{M}}(X)\ra \Qcoh(X)$ and $a^*$. Since $a^*$ preserves colimits, we deduce that the cocone $\big\{a^*u_i:a^*\cF_i\ra a^*\cF\big\}_{i\in I}$ is colimiting. Hence such morphism $\tau$ of cocones exists and is unique.\\
Concluding our arguments above we derive that there exists a unique $\tau:\pi^*\cF\ra a^*\cF$ such that \textbf{(i)} holds. This means that there exists precisely one lift  along the forgetful functor $\Qcoh_{\bd{M}}(X)\ra \Qcoh(X)$ of the cocone $\big\{u_i:\cF_i\ra \cF\big\}_{i\in I}$ for $F$ composed with the forgetful functor $\Qcoh_{\bd{M}}(X)\ra \Qcoh(X)$ to a cone for $F$. In order to finish the proof of \textbf{(1)} it suffices to prove that this lift is colimiting for $F$. Suppose that $(\cG,\sigma)$ is a quasi-coherent $\bd{M}$-sheaf and $\big\{v_i:\cF_i\ra \cG\big\}_{i\in I}$ is a cocone for $F$. Then there exists a unique morphism $v:\cF\ra \cG$ of quasi-coherent sheaves such that $v\cdot u_i = v_i$ for all $i$ in $I$. It suffices to prove that $v$ is a morphism between $(\cF,\tau)$ and $(\cG,\sigma)$, where $\tau$ satisfies \textbf{(ii)}. We have
$$\sigma \cdot \pi^*v \cdot \pi^* u_i = \sigma \cdot \pi^*v_i = a^*v_i \cdot \tau_i = a^*v \cdot a^*u_i\cdot \tau_i = a^*v \cdot \tau\cdot \pi^*u_i$$
for every $i$ in $I$. Since $\pi^*$ preserves colimits, we derive that $\big\{\pi^*u_i:\pi^*\cF_i\ra \pi^*\cF\big\}_{i\in I}$ is a colimiting cocone for $F$ composed with the forgetful functor $\Qcoh_{\bd{M}}(X)\ra \Qcoh(X)$ and $\pi^*$. Thus $\sigma \cdot \pi^*v = a^*v \cdot \tau$ and this finishes the proof of \textbf{(1)}.\\
The proof of \textbf{(2)} is analogical to \textbf{(1)}. As above consider a finite category $I$ and a functor $F:I\ra \Qcoh_{\bd{M}}(X)$. Let $F(i) = (\cF_i,\tau_i)$ for every $i\in I$. Assume that $\big\{p_i:\cF \ra \cF_i\big\}_{i\in I}$ is a limiting cone in $\Qcoh(X)$ for the composition of $F$ with the forgetful functor $\Qcoh_{\bd{M}}(X)\ra \Qcoh(X)$. Pick a morphism $\tau:\pi^*\cF\ra a^*\cF$ of quasi-coherent sheaves on $X$. We show that the following assertions are equivalent.
\begin{enumerate}[label= \textbf{(\roman*)}, leftmargin=3.0em]
\item $(\cF,\tau)$ is a quasi-coherent $\bd{M}$-sheaf on $X$ and $\{p_i\}_{i\in I}$ are morphisms of quasi-coherent $\bd{M}$-sheaves with domain $(\cF,\tau)$.
\item Squares
\begin{center}
\begin{tikzpicture}
[description/.style={fill=white,inner sep=2pt}]
\matrix (m) [matrix of math nodes, row sep=3em, column sep=3em,text height=1.5ex, text depth=0.25ex] 
{ \pi^*\cF   & \pi^*\cF_i     \\
  a^*\cF & a^*\cF_i   \\} ;
\path[->,line width=1.0pt,font=\scriptsize]
(m-1-1) edge node[above] {$ \pi^*p_i  $} (m-1-2)
(m-2-1) edge node[below] {$ a^*p_i $} (m-2-2)
(m-1-1) edge node[left]  {$ \tau $} (m-2-1)
(m-1-2) edge node[right] {$ \tau_i $} (m-2-2);
\end{tikzpicture}
\end{center}
are commutative for all $i$ in $I$.
\end{enumerate}
As above the implication $\textbf{(i)}\Rightarrow \textbf{(ii)}$ is a consequence of the definition of quasi-coherent $\bd{M}$-sheaf. We now prove $\textbf{(ii)}\Rightarrow \textbf{(i)}$. We have
$$(1_{\bd{M}}\times_k a)^*a^*p_i \cdot (\mu\times_k 1_X)^*\tau = (\mu \times_k 1_X)^*a^*p_i \cdot (\mu\times_k 1_X)^*\tau = \left(\mu\times_k1_X\right)^*\tau_i \cdot (\mu\times_k1_X)^*\pi^*p_i = $$
$$ = (1_{\bd{M}}\times_k a)^*\tau_i \cdot \pi_{23}^*\tau_i \cdot (\mu\times_k1_X)^*\pi^*p_i = (1_{\bd{M}}\times_k a)^*\tau_i \cdot \pi_{23}^*\tau_i\cdot \pi_{23}^*\pi^*p_i = (1_{\bd{M}}\times_k a)^*\tau_i \cdot \pi_{23}^*a^*p_i\cdot \pi_{23}^*\tau = $$
$$= (1_{\bd{M}}\times_k a)^*\tau_i \cdot (1_{\bd{M}}\times_ka)^*\pi^*p_i\cdot \pi_{23}^*\tau =(1_{\bd{M}}\times_k a)^*a^*p_i\cdot (1_{\bd{M}}\times_k a)^*\tau\cdot \pi^*_{23}\tau$$
for every $i$ in $I$ and since both $a^*, (1_{\bd{M}}\times_k a)^*$ preserve finite limits ($a$ is flat), we derive that $(\mu\times_k 1_X)^*\tau = (1_{\bd{M}}\times_k a)^*\tau\cdot \pi^*_{23}\tau$. Thus this implication holds.\\
Next due to the fact that $a^*$ preserves finite limits ($a$ is flat), we deduce that \textbf{(ii)} holds for a unique morphism $\tau:\pi^*\cF\ra a^*\cF$. Thus we conclude that there exists a unique lift along the forgetful functor $\Qcoh_{\bd{M}}(X)\ra \Qcoh(X)$ of a cone $\big\{p_i:\cF \ra \cF_i\big\}_{i\in I}$ for the composition of $F$ with $\Qcoh_{\bd{M}}(X)\ra \Qcoh(X)$ to a cone for $F$. In order to finish the proof it suffices to observe that this unique lift is a limiting cone for $F$. Suppose that $(\cG,\sigma)$ is a quasi-coherent $\bd{M}$-sheaf and $\big\{q_i:\cG\ra \cF_i\big\}_{i\in I}$ is a cone for $F$. Then there exists a unique morphism $q:\cG\ra \cF$ of quasi-coherent sheaves such that $p_i \cdot q = q_i$ for all $i$ in $I$. It suffices to prove that $q$ is a morphism between $(\cG,\sigma)$ and $(\cF,\tau)$, where $\tau$ satisfies \textbf{(ii)}. We have
$$a^*p_i\cdot a^*q \cdot \sigma = a^*q_i\cdot \sigma = \tau_i\cdot \pi^*q_i = \tau_i\cdot \pi^*p_i\cdot \pi^*q = a^*p_i\cdot \tau\cdot a^*q$$
for every $i$ in $I$. Since $a^*$ preserves finite limits ($a$ is flat), we deduce $a^*q \cdot \sigma = \tau\cdot a^*q$ and this finishes the proof of \textbf{(2)}.
\end{proof}

\begin{corollary}\label{corollary:for_flat_actions_equivariant_qc_sheaves_are_Ab5}
Let $X$ be a $k$-scheme equipped with an action $a:\bd{M}\times_kX\ra X$ of $\bd{M}$. Suppose that $a$ is flat. Then $\Qcoh_{\bd{M}}(X)$ is an $\bd{Ab}5$-category.
\end{corollary}
\begin{proof}
This follows from Theorem \ref{theorem:forgetful_functor_from_equivariant_sheaves_to_qc_sheaves_creates_colimits_and_finite_limits} and the fact that $\Qcoh(X)$ is $\bd{Ab}5$.
\end{proof}

\begin{definition}
Let $X$ be a $k$-scheme equipped with an action $a:\bd{M}\times_kX\ra X$ of $\bd{M}$. Consider a locally closed subscheme $Z$ of $X$. Then $Z$ is called $\bd{M}$-invariant if the restriction of $a$ to $\bd{M}\times_kZ$ factors scheme-theoreticaly through $Z$.
\end{definition}
\noindent
Now we discuss relation between quasi-coherent $\bd{M}$-ideals and $\bd{M}$-invariant closed subschemes.

\begin{proposition}\label{proposition:invariant_closed_subschemes}
Let $X$ be a $k$-scheme equipped with an action $a:\bd{M}\times_kX\ra X$ of $\bd{M}$ and denote by $\pi:\bd{M}\times_kX\ra X$ the projection. Suppose that $Z$ is a closed subscheme $Z$ of $X$ and recall that $\cO_X$ carries the canonical $\bd{M}$-sheaf structure. Consider the following assertions.
\begin{enumerate}[label= \emph{\textbf{(\roman*)}}, leftmargin=3.0em]
\item $Z$ is an $\bd{M}$-invariant closed subscheme of $X$.
\item There exists a unique structure of quasi-coherent $\bd{M}$-algebra on $\cO_Z$.
\item There exists a structure of quasi-coherent $\bd{M}$-algebra on $\cO_Z$ and the canonical epimorphism $\cO_X\twoheadrightarrow \cO_Z$ is a morphism of $\bd{M}$-sheaves.
\item Let $\cI$ be a quasi-coherent ideal of $Z$. There exists a unique structure of quasi-coherent $\bd{M}$-sheaf on $\cI$ such that $\cI\hookrightarrow \cO_X$ is a morphism of $\bd{M}$-sheaves.
\end{enumerate}
Then \emph{\textbf{(i)}}, \emph{\textbf{(ii)}}, \emph{\textbf{(iii)}} are equivalent and \emph{\textbf{(iv)}} implies all these assertions. Moreover, if $a$ is flat, then all assertions are equivalent.
\end{proposition}
\begin{proof}
Note that there exists at most one $\bd{M}$-action on $Z$ such that $Z\hookrightarrow X$ is $\bd{M}$-equivariant. Thus Corollary \ref{corollary:equivariant_qc_algebras_bijective_with_actions_on_their_relative_specs} implies that \textbf{(i)} is equivalent with \textbf{(ii)}.\\
Suppose that \textbf{(i)} holds. Then there exists a $\bd{M}$-action on $Z$ such that $Z\hookrightarrow X$ is $\bd{M}$-equivariant. Thus $\Spec_X \cO_Z \hookrightarrow \Spec_X \cO_X$ induced by the epimorphism $\cO_X \twoheadrightarrow \cO_Z$ is $\bd{M}$-equivariant and hence by Corollaries \ref{corollary:equivariant_qc_algebras_bijective_with_actions_on_their_relative_specs}, \ref{corollary:morphisms_of_qc_equivariant_algebras_are_morphisms_of_schemes_with_actions} we derive that there exists a structure of quasi-coherent $\bd{M}$-algebra on $\cO_Z$ such that $\cO_X \twoheadrightarrow \cO_Z$ is a morphism of $\bd{M}$-algebras. Hence $\textbf{(i)} \Rightarrow \textbf{(iii)}$.\\
Next if \textbf{(iii)} holds, then by Corollary \ref{corollary:equivariant_qc_algebras_bijective_with_actions_on_their_relative_specs} we derive that there exists $\bd{M}$-action on $Z$ such that $Z\hookrightarrow X$ is $\bd{M}$-equivariant. Therefore, the implication $\textbf{(iii)}\Rightarrow \textbf{(i)}$ holds.\\
This completes the proof that \textbf{(i)}, \textbf{(ii)}, \textbf{(iii)} are equivalent.\\
Now suppose that \textbf{(iv)} holds. Then there exists a morphism $\tau:\pi^*\cI\ra a^*\cI$ such that $\cI \hookrightarrow \cO_X$ is a morphism of quasi-coherent $\bd{M}$-sheaves with respect to $(\cI,\tau)$. Then there exists a unique morphism $\sigma:\pi^*\cO_Z\ra a^*\cO_Z$ of quasi-coherent sheaves such that the diagram
\begin{center}
\begin{tikzpicture}
[description/.style={fill=white,inner sep=2pt}]
\matrix (m) [matrix of math nodes, row sep=3em, column sep=3em,text height=1.5ex, text depth=0.25ex] 
{ \pi^*\cI & \pi^*\cO_X & \pi^*\cO_Z     \\
  a^*\cI   & a^*\cO_X  & a^*\cO_Z           \\} ;
\path[->,line width=1.0pt,font=\scriptsize]
(m-1-1) edge node[below] {$  $} (m-1-2)
(m-2-1) edge node[below] {$  $} (m-2-2)
(m-1-1) edge node[left] {$ \tau $} (m-2-1)
(m-1-2) edge node[left] {$ a_{\#}^{-1}\cdot \pi_{\#} $} (m-2-2)
(m-1-3) edge node[below] {$\sigma $} (m-2-3);
\path[->>,line width=1.0pt,font=\scriptsize]
(m-1-2) edge node[left] {$ $} (m-1-3)
(m-2-2) edge node[right] {$ $} (m-2-3);
\end{tikzpicture}
\end{center}
is commutative. Moreover, $\sigma$ is a morphism of $\cO_{\bd{M}\times_k X}$-algebras. The epimorphism $\cO_X\twoheadrightarrow \cO_Z$ is a cokernel of $\cI\hookrightarrow \cO_X$ in $\Qcoh(X)$. By Theorem \ref{theorem:forgetful_functor_from_equivariant_sheaves_to_qc_sheaves_creates_colimits_and_finite_limits} there exists a unique lift of this cokernel to $\Qcoh_{\bd{M}}(X)$. Since $\sigma$ is a unique morphism of quasi-coherent sheaves on $\bd{M}\times_k X$ that makes the diagram above commutative, we derive that $\cO_X\twoheadrightarrow \cO_Z$ is a cokernel of $\cI\hookrightarrow \cO_X$ in $\Qcoh_{\bd{M}}(X)$ with respect to the structure of $\bd{M}$-sheaf determined by $\sigma$. Thus $(\cO_Z,\sigma)$ is quasi-coherent $\bd{M}$-sheaf on $X$. Since $\sigma$ is a morphism of $\cO_{\bd{M}\times_k X}$-algebras, we derive that $(\cO_Z,\sigma)$ is a quasi-coherent $\bd{M}$-algebra on $X$. Thus $\textbf{(iv)}\Rightarrow \textbf{(iii)}$.\\
Assume that $a$ is flat and \textbf{(iii)} holds. Since $\cI\hookrightarrow \cO_X$ is a kernel in $\Qcoh(X)$ of $\cO_X\twoheadrightarrow \cO_Z$, we derive by Theorem \ref{theorem:forgetful_functor_from_equivariant_sheaves_to_qc_sheaves_creates_colimits_and_finite_limits} that there exists a structure of $\bd{M}$-sheaf on $\cI$ such that $\cI\hookrightarrow \cO_X$ is a morphism in $\Qcoh_{\bd{M}}(X)$. This shows that if $a$ is flat, then \textbf{(iii)} implies \textbf{(iv)}.
\end{proof}

\begin{definition}
Let $X$ be a locally noetherian $k$-scheme equipped with $\bd{M}$-action $a:\bd{M}\times_kX\ra X$. Then we denote by $\Coh_{\bd{M}}(X)$ the full subcategory of $\Qcoh_{\bd{M}}(X)$ consisting of quasi-coherent $\bd{M}$-sheaves $(\cF,\tau)$ such that $\cF$ is coherent sheaf on $X$. We call it \textit{the category of coherent $\bd{M}$-sheaves on $(X,a)$}.
\end{definition}

\begin{corollary}\label{corollary:coherent_equivariant_sheaves_properties}
Let $X$ be a locally noetherian $k$-scheme equipped with $\bd{M}$-action $a:\bd{M}\times_kX\ra X$. Then the following assertions hold.
\begin{enumerate}[label= \emph{\textbf{(\arabic*)}}, leftmargin=3.0em]
\item The functor $\Coh_{\bd{M}}(X)\ra \Coh(X)$ creates finite colimits and if $a$ is flat, then it also creates finite limits.
\item There exists monoidal structure on $\Coh_{\bd{M}}(X)$ such that the embedding $\Coh_{\bd{M}}(X)\hookrightarrow \Qcoh_{\bd{M}}(X)$ is a strict monoidal functor.
\end{enumerate}
If in addition $Y$ is a locally noetherian $k$-scheme equipped with action of $\bd{M}$ and $f:X\ra Y$ is an $\bd{M}$-equivariant morphism, then $f^*\Qcoh_{\bd{M}}(Y)\ra \Qcoh_{\bd{M}}(X)$ restricts to a functor $f^*:\Coh_{\bd{M}}(Y)\ra \Coh_{\bd{M}}(X)$.
\end{corollary}
\begin{proof}
The statement \textbf{(1)} is a consequence of Theorem \ref{theorem:forgetful_functor_from_equivariant_sheaves_to_qc_sheaves_creates_colimits_and_finite_limits} and the fact that the functor $\Coh(X)\hookrightarrow \Qcoh(X)$ creates both finite limits and finite colimits.\\
Next \textbf{(2)} follows from the form of the monoidal structure described in Remark \ref{remark:monoidal_structure_on_equivariant_sheaves}. Concretely, if $(\cF_1,\tau_1)$ and $(\cF_2,\tau_2)$ are coherent $\bd{M}$-sheaves, then $\left(\cF_1\otimes_{\cO_X}\cF_2,\tau_1\otimes_{\cO_X}\tau_2\right)$ is coherent $\bd{M}$-sheaf and $\left(\cO_X,a_{\#}^{-1}\cdot \pi_{\#}\right)$ is coherent $\bd{M}$-sheaf, where $\pi:\bd{M}\times_kX\ra X$ denotes the projection.\\
The last part of the statement follows from Corollary \ref{corollary:pullbacks_of_equivariant_sheaves}.
\end{proof}

\begin{proposition}\label{proposition:}
Suppose that $\bd{M}$ is affine monoid $k$-scheme. Let $X$ be a $k$-scheme equipped with $\bd{M}$-action $a:\bd{M}\times_kX\ra X$ and let $(\cF,\tau)$ be a quasi-coherent $\bd{M}$-sheaf. Assume that $X$ is quasi-compact and semi-separated. If $\tau$ is an isomorphism of sheaves, then $\Gamma\left(X,\tau^{-1}\right)\cdot a^*:\Gamma\left(X,\cF\right) \ra k[\bd{M}]\otimes_k \Gamma\left(X,\cF\right)$ is a coaction of $k[\bd{M}]$.
\end{proposition}

\section{Example: Principal Bundles}
\noindent
We devote this whole section to another important example of a fibered category. We fix a category with finite limits $\cB$ and a monoid object $\bd{M}$ of $\cB$. We denote by $\mu:\bd{M}\times \bd{M}\ra \bd{M}$ and $e:\bd{1}\ra \bd{M}$ the multiplication and unit of $\bd{M}$, respectively.

\begin{definition}
Let $\cP$ be an object of $\cB$ equipped with an action of $\bd{M}$, let $T$ be an object of $\cB$ with trivial action of $\bd{M}$ and let $\pi:\cP\ra T$ be an $\bd{M}$-equivariant morphism with respect to these $\bd{M}$-actions. We say that $\bd{M}$-equivariant morphism $\pi$ is \textit{a trivial principal $\bd{M}$-bundle on $T$} if there exists an $\bd{M}$-equivariant isomorphism $\phi:\cP\ra \bd{M}\times T$ such that $\bd{M}\times T$ is equipped with an action of $\bd{M}$ given by $\mu\times 1_T$ and the triangle
\begin{center}
\begin{tikzpicture}
[description/.style={fill=white,inner sep=2pt}]
\matrix (m) [matrix of math nodes, row sep=2em, column sep=1em,text height=1.5ex, text depth=0.25ex] 
{  \cP &        & \bd{M}\times T  \\
          &T &  \\} ;
\path[->,line width=1.0pt,font=\scriptsize]
(m-1-1) edge node[above] {$\phi $} (m-1-3)
(m-1-1) edge node[below = 6pt, left = 1pt] {$ \pi  $} (m-2-2)
(m-1-3) edge node[below = 6pt, right = 1pt] {$ \mathrm{pr}_T $} (m-2-2);
\end{tikzpicture}
\end{center}
is commutative.
\end{definition}

\begin{definition}
Let $\cP$ be an object of $\cB$ equipped with an action of $\bd{M}$, let $T$ be an object of $\cB$ with trivial action of $\bd{M}$ and let $\pi:\cP\ra T$ be a $\bd{M}$-equivariant morphism with respect to these $\bd{M}$-actions. Consider a sieve $S$ on $T$. For every arrow $g:\widetilde{T}\ra T$ in $S$ we construct a cartesian square
\begin{center}
\begin{tikzpicture}
[description/.style={fill=white,inner sep=2pt}]
\matrix (m) [matrix of math nodes, row sep=2em, column sep=2em,text height=1.5ex, text depth=0.25ex] 
{ g^*\cP &  \cP    \\
  \widetilde{T} &  T           \\} ;
\path[->,line width=1.0pt,font=\scriptsize]
(m-1-1) edge node[above] {$   $} (m-1-2)
(m-2-1) edge node[below] {$ g $} (m-2-2)
(m-1-1) edge node[left] {$ \pi_g $} (m-2-1)
(m-1-2) edge node[right] {$ \pi $} (m-2-2);
\end{tikzpicture}
\end{center}
in $\cB$. We consider $g$ as an $\bd{M}$-equivariant morphism with respect to trivial $\bd{M}$-actions on $T$ and $\widetilde{T}$. Then (Proposition \ref{proposition:creation_of_limits_for_actions_of_monoids}) there exists a unique action of $\bd{M}$ on $g^*\cP$ which makes $\pi_g$ into an $\bd{M}$-equivariant morphism in such a way that the square consists of objects of $\cB$ with $\bd{M}$-actions and $\bd{M}$-equivariant morphisms. Suppose that $\bd{M}$-equivariant morphism $\pi_g$ is a trivial principal $\bd{M}$-bundle on $\widetilde{T}$ for every $g$ in $S$. Then we say that \textit{$S$ trivializes $\pi$}.
\end{definition}
\noindent
In the remaining part of this section we fix a Grothendieck topology $\cJ$ on $\cB$.

\begin{definition}
Let $\cP$ be an object of $\cB$ equipped with an action of $\bd{M}$, let $T$ be an object of $\cB$ with trivial action of $\bd{M}$ and let $\pi:\cP \ra T$ be a $\bd{M}$-equivariant morphism with respect to these $\bd{M}$-actions. Suppose that there exists a covering sieve $S$ in $\cJ(T)$ that trivializes $\pi$. Then $\pi$ is called \textit{a principal $\bd{M}$-bundle with respect to $\cJ$}.
\end{definition}
\noindent
Now we define a category $\mathbb{B}\bd{M}$ that depends on the site $(\cB,\cJ)$. Its objects are principal $\bd{M}$-bundles with respect to $\cJ$ and if $\pi:\cP\ra T$ and $\psi:Q \ra Z$ are principal $\bd{M}$-bundles with respect to $\cJ$, then a morphism $\pi\ra \psi$ is a pair $(f,\phi)$ such that $f:T\ra Z$ and $\phi:\cP\ra Q$ are morphisms in $\cB$ such that $\phi$ is $\bd{M}$-equivariant and the square
\begin{center}
\begin{tikzpicture}
[description/.style={fill=white,inner sep=2pt}]
\matrix (m) [matrix of math nodes, row sep=2em, column sep=2em,text height=1.5ex, text depth=0.25ex] 
{ \cP &  Q           \\
  T   &  Z           \\} ;
\path[->,line width=1.0pt,font=\scriptsize]
(m-1-1) edge node[above] {$ \phi  $} (m-1-2)
(m-2-1) edge node[below] {$ f $} (m-2-2)
(m-1-1) edge node[left] {$\pi $} (m-2-1)
(m-1-2) edge node[right] {$  \psi $} (m-2-2);
\end{tikzpicture}
\end{center}
is commutative. We have a functor $p_{\bd{M},\cJ}:\mathbb{B}\bd{M}\ra \cB$ given by $p_{\bd{M},\cJ}\big((f,\phi)\big) = f$. Let $\psi:Q\ra Z$ be a principal $\bd{M}$-bundle with respect to $\cJ$ and let $f:T\ra Z$ be a morphism. Consider the cartesian square
\begin{center}
\begin{tikzpicture}
[description/.style={fill=white,inner sep=2pt}]
\matrix (m) [matrix of math nodes, row sep=2em, column sep=2em,text height=1.5ex, text depth=0.25ex] 
{ f^*Q &  Q           \\
  T   &  Z           \\} ;
\path[->,line width=1.0pt,font=\scriptsize]
(m-1-1) edge node[above] {$ \phi  $} (m-1-2)
(m-2-1) edge node[below] {$ f $} (m-2-2)
(m-1-1) edge node[left] {$\pi $} (m-2-1)
(m-1-2) edge node[right] {$  \psi $} (m-2-2);
\end{tikzpicture}
\end{center}
in $\cB$. Then by Proposition \ref{proposition:creation_of_limits_for_actions_of_monoids} there exists a unique action of $\bd{M}$ on $f^*Q$ such that the square above consists of $\bd{M}$-equivariant morphisms ($T,Z$ are equipped with trivial $\bd{M}$-actions). Moreover, with respect to this action $\psi:f^*Q\ra T$ becomes a principal $\bd{M}$-bundle with respect to $\cJ$. Indeed, if $S$ is in $\cJ(Z)$ and $S$ trivializes $\psi$, then its pullback $f^*S$ trivializes $\pi$ and is an element of $\cJ(T)$ (by definition of a Grothendieck topology). This shows that $p_{\bd{M},\cJ}:\mathbb{B}\bd{M}\ra \cB$ is a fibered category. Moreover, we have a functor $\mathbb{B}\bd{M}\ra \mathrm{Arr}(\cB)$ that forgets about $\bd{M}$-actions. Hence there exists commutative triangle
\begin{center}
\begin{tikzpicture}
[description/.style={fill=white,inner sep=2pt}]
\matrix (m) [matrix of math nodes, row sep=2em, column sep=1em,text height=1.5ex, text depth=0.25ex] 
{  \mathbb{B}\bd{M} &        & \mathrm{Arr}(\cB)  \\
          &\cB &  \\} ;
\path[->,line width=1.0pt,font=\scriptsize]
(m-1-1) edge node[above] {$ $} (m-1-3)
(m-1-1) edge node[below = 6pt, left = 1pt] {$ p_{\bd{M},\cJ}  $} (m-2-2)
(m-1-3) edge node[below = 6pt, right = 1pt] {$ p_{\mathrm{Arr}(\cB)} $} (m-2-2);
\end{tikzpicture}
\end{center}
According to Example \ref{example:the_fibered_category_of_arrows} and description of cartesian morphisms of $p_{\bd{M},\cJ}$ the functor $\mathbb{B}\bd{M}\ra \mathrm{Arr}(\cB)$ described above is a morphism of fibered categories.

\begin{definition}
$p_{\bd{M},\cJ}:\mathbb{B}\bd{M}\ra \cB$ is called \textit{the fibered category of principal $\bd{M}$-bundles on $(\cB,\cJ)$}.
\end{definition}
\noindent
Suppose that $X$ is an object of $\cB$ equipped with an action $a:\bd{M}\times X\ra X$ of $\bd{M}$. We define a category $[X/\bd{M}]$ depending on $a$ and the site $(\cB,\cJ)$ as follows. Its objects are pairs $(\pi,\alpha)$ such that $\pi$ is a principal $\bd{M}$-bundle with respect to $\cJ$ and $\alpha$ is an $\bd{M}$-equivariant morphism. We depict them by diagrams
\begin{center}
\begin{tikzpicture}
[description/.style={fill=white,inner sep=2pt}]
\matrix (m) [matrix of math nodes, row sep=2em, column sep=2em,text height=1.5ex, text depth=0.25ex] 
{ \cP &  X           \\
  T   &             \\} ;
\path[->,line width=1.0pt,font=\scriptsize]
(m-1-1) edge node[above] {$ \alpha  $} (m-1-2)
(m-1-1) edge node[left] {$\pi $} (m-2-1);
\end{tikzpicture}
\end{center}
Suppose that $(\pi:\cP\ra T,\alpha:\cP\ra X)$ and $(\psi:Q\ra Z,\beta:Q\ra X)$ are two such objects. Then a morphism $(\pi,\alpha)\ra (\psi,\beta)$ is a morphism $(f,\phi):\pi\ra \psi$ in $\mathbb{B}\bd{M}$ such that $\alpha = \beta \cdot \phi$. We have a functor $[X/\bd{M}] \ra \mathbb{B}\bd{M}$ which sends $(\pi, \alpha)$ to $\pi$. We denote by $p_{a,\cJ}:[X/\bd{M}]\ra \cB$ the composition of this functor $[X/\bd{M}] \ra \mathbb{B}\bd{M}$ with $p_{\bd{M},\cJ}:\mathbb{B}\bd{M}\ra \cB$. By description of cartesian morphisms of $p_{\bd{M},\cJ}$ we deduce that $p_{a,\cJ}$ is a fibered category. We have a commutative triangle
\begin{center}
\begin{tikzpicture}
[description/.style={fill=white,inner sep=2pt}]
\matrix (m) [matrix of math nodes, row sep=2em, column sep=1em,text height=1.5ex, text depth=0.25ex] 
{ [X/\bd{M}]  &        & \mathbb{B}\bd{M}  \\
          &\cB &  \\} ;
\path[->,line width=1.0pt,font=\scriptsize]
(m-1-1) edge node[above] {$  $} (m-1-3)
(m-1-1) edge node[below = 6pt, left = 1pt] {$ p_{a,\cJ}  $} (m-2-2)
(m-1-3) edge node[below = 6pt, right = 1pt] {$ p_{\bd{M},\cJ} $} (m-2-2);
\end{tikzpicture}
\end{center}
and the functor $[X/\bd{M}]\ra \mathbb{B}\bd{M}$ described above is a morphism of fibered categories. Note that if $\bd{1}$ is a terminal object of $\cB$ equipped with trivial action of $\bd{M}$, then we have a canonical isomorphism $[\bd{1}/\bd{M}] \cong \mathbb{B}\bd{M}$ of categories over $\cB$.

\begin{definition}
$p_{a,\cJ}:\mathbb{B}\bd{M}\ra \cB$ is called \textit{the quotient fibered category of $\bd{M}$-object $X$ on $(\cB,\cJ)$}.
\end{definition}
\noindent
Results below show that up to some mild assumptions on Grothendieck topology $\cJ$ fibered category $p_{a,\cJ}:[X/\bd{M}]\ra \cB$ encapsulates all essential information concerning action of $\bd{M}$ on $X$. We start with the following observation.

\begin{fact}\label{fact:functor_over_principal_bundles_is_morphism_of_quotient_fibered_categories}
Let $X,Y$ be objects of $\cB$ equipped with actions $a:\bd{M}\times X\ra X$ and $b:\bd{M}\times Y\ra Y$ of $\bd{M}$. Consider a functor $F:[X/\bd{M}]\ra [Y/\bd{M}]$ such that the triangle
\begin{center}
\begin{tikzpicture}
[description/.style={fill=white,inner sep=2pt}]
\matrix (m) [matrix of math nodes, row sep=2em, column sep=1em,text height=1.5ex, text depth=0.25ex] 
{ [X/\bd{M}]   &             & {}[Y/\bd{M}] \\
          & \mathbb{B}\bd{M} &  \\} ;
\path[->,line width=1.0pt,font=\scriptsize]
(m-1-1) edge node[above] {$ F $} (m-1-3)
(m-1-1) edge node[below = 6pt, left = 1pt] {$ $} (m-2-2)
(m-1-3) edge node[below = 6pt, right = 1pt] {$ $} (m-2-2);
\end{tikzpicture}
\end{center}
is commutative, where two other sides are canonical functors. Then $F$ is a morphism of fibered categories $p_{a,\cJ}$ and $p_{b,\cJ}$.
\end{fact}
\begin{proof}
The commutativity of the triangle implies that $F\cdot p_{b,\cJ} = p_{a,\cJ}$. Since a morphism in $[X/\bd{M}]$ is cartesian with respect to $p_{a,\cJ}$ if and only if its image under the canonical functor $[X/\bd{M}]\ra \mathbb{B}\bd{M}$ is cartesian with respect to $p_{\bd{M},\cJ}$ and the same holds for $p_{b,\cJ}$, we derive that $F$ sends cartesian morphisms of $p_{a,\cJ}$ to cartesian morphisms of $p_{b,\cJ}$. 
\end{proof}
\noindent
Let $X,Y$ be objects of $\cB$ equipped with actions $a:\bd{M}\times X\ra X$ and $b:\bd{M}\times Y\ra Y$ of $\bd{M}$. We denote the class of functors in Fact \ref{fact:functor_over_principal_bundles_is_morphism_of_quotient_fibered_categories} by $\Mor_{\mathbb{B}\bd{M}}\left([X/\bd{M}],[Y/\bd{M}]\right)$. We also denote (by abuse of notation) the class of $\bd{M}$-equivariant morphism $(X,a)\ra (Y,b)$ by $\Mor_{\bd{M}}\left(X,Y\right)$.

\begin{theorem}\label{theorem:equivariant_morphisms_can_described_by_fibered_categories}
Let $(\cB,\cJ)$ be a Grothendieck site and assume that representable presheaves on $\cB$ are separated with respect to $\cJ$. Let $X,Y$ be objects of $\cB$ equipped with $\bd{M}$-actions $a:\bd{M}\times X\ra X$ and $b:\bd{M}\times Y\ra Y$, respectively. Then there exists a bijection
$$\Mor_{\bd{M}}\left(X,Y\right)\cong \Mor_{\mathbb{B}\bd{M}}\big([X/\bd{M}],[Y/\bd{M}]\big)$$
that sends an $\bd{M}$-equivariant morphism $f$ to a functor $F:[X/\bd{M}]\ra [Y/\bd{M}]$ given by
\begin{center}
\begin{tikzpicture}
[description/.style={fill=white,inner sep=2pt}]
\matrix (m) [matrix of math nodes, row sep=1em, column sep=4em,text height=1.5ex, text depth=0.25ex] 
{ \cP &  X &   &     & \cP   &   Y         \\
      &    &{} & {}  &       &           \\
   T  &    &   &     &  T    &           \\} ;
\path[->,line width=1.0pt,font=\scriptsize]
(m-1-1) edge node[above] {$ d $} (m-1-2)
(m-1-1) edge node[left] {$ \pi  $} (m-3-1)

(m-1-5) edge node[above]  {$ f\cdot d $} (m-1-6)
(m-1-5) edge node[left] {$ \pi $} (m-3-5);
\path[|->,line width=1.0pt,font=\scriptsize]
(m-2-3) edge node[above]  {$ F $} (m-2-4);
\end{tikzpicture}
\end{center}
\end{theorem}
\begin{proof}
We first describe certain object of $[X/\bd{M}]$. Observe that $(\bd{M}\times X,\mu\times 1_X)$ is an object of $\cB$ equipped with the action of $\bd{M}$. Next the projection $\mathrm{pr}_X:\bd{M}\times X\ra X$ can be considered as an $\bd{M}$-equivariant morphism from this $\bd{M}$-object to $X$ with the trivial action of $\bd{M}$. Since the square
\begin{center}
\begin{tikzpicture}
[description/.style={fill=white,inner sep=2pt}]
\matrix (m) [matrix of math nodes, row sep=3em, column sep=3em,text height=1.5ex, text depth=0.25ex] 
{ \bd{M}\times \bd{M}\times X &  \bd{M}\times X    \\
  \bd{M}\times X              &  X           \\} ;
\path[->,line width=1.0pt,font=\scriptsize]
(m-1-1) edge node[above] {$ 1_{\bd{M}} \times a $} (m-1-2)
(m-2-1) edge node[below] {$ a  $} (m-2-2)
(m-1-1) edge node[left]  {$ \mu\times 1_X $} (m-2-1)
(m-1-2) edge node[right] {$ a $} (m-2-2);
\end{tikzpicture}
\end{center}
is commutative, we derive that $a$ is an $\bd{M}$-equivariant morphism $\left(\bd{M}\times X,\mu\times 1_X\right)\ra \left(X,a\right)$. This gives $(\mathrm{pr}_X,a)$ the structure of an object of $[X/\bd{M}]$. Fix a functor $F$ in $\Mor_{\mathbb{B}\bd{M}}\big([X/\bd{M}],[Y/\bd{M}]\big)$. The functor $F$ sends $\left(\mathrm{pr}_X,a\right)$ to some object of $[Y/\bd{M}]$. This object is necessarily of the form $(\mathrm{pr}_X,\alpha)$ for some $\bd{M}$-equivariant morphism $\alpha:\left(\bd{M}\times X,\mu\times 1_X\right) \ra \left(Y,b\right)$. Indeed, this follows from the fact that $F$ is over $\mathbb{B}\bd{M}$. Now if $F$ is determined by some $\bd{M}$-equivariant morphism $f$ as it is described in the statement, then $\alpha = f\cdot a$ and hence $f = \alpha \cdot \langle e,1_X\rangle$. This proves that the map $\Mor_{\bd{M}}\big(X,Y\big)\ra \Mor_{\mathbb{B}\bd{M}}\big([X/\bd{M}],[Y/\bd{M}]\big)$ described in the statement is injective. Our goal is to show that it is surjective. That is our goal is to show that for the functor $F$ in $\Mor_{\mathbb{B}\bd{M}}\big([X/\bd{M}],[Y/\bd{M}]\big)$ a morphism $f = \alpha \cdot \langle e, 1_X\rangle$ is $\bd{M}$-equivariant and determines $F$ as it is described in the statement. First we fix some object $T$ of $\cB$ and the projection $\mathrm{pr}_T:\bd{M}\times T\ra T$ considered as a trivial principal $\bd{M}$-bundle. Let $(\mathrm{pr}_T,c)$ be an object of $[X/\bd{M}]$. Then $c$ is an $\bd{M}$-equivariant morphism $c:\left(\bd{M}\times T,\mu\times 1_T\right) \ra (X,a)$. Functor $F$ sends $(\mathrm{pr}_T,c)$ to some object $(\mathrm{pr}_T,\gamma)$. We claim that $\gamma = f \cdot c$. Let $\mathrm{pr}_{23}:\bd{M}\times \bd{M}\times T\ra \bd{M}\times T$ be the projection on the last two factors. There are diagrams
\begin{center}
\begin{tikzpicture}
[description/.style={fill=white,inner sep=2pt}]
\matrix (m) [matrix of math nodes, row sep=3em, column sep=3em,text height=1.5ex, text depth=0.25ex] 
{ \bd{M}\times \bd{M} \times T  &  \bd{M}\times T  & X    &   \bd{M}\times \bd{M}\times T     &  \bd{M}\times X  & X          \\
  \bd{M}\times T                &  T               &      &   \bd{M}\times T                  &  X               &          \\} ;
\path[->,line width=1.0pt,font=\scriptsize]
(m-1-1) edge node[above] {$ \mu\times 1_T $} (m-1-2)
(m-2-1) edge node[below] {$ \mathrm{pr}_T $} (m-2-2)
(m-1-1) edge node[left]  {$ \mathrm{pr}_{23} $} (m-2-1)
(m-1-2) edge node[right] {$ \mathrm{pr}_T $} (m-2-2)
(m-1-2) edge node[above] {$ c $} (m-1-3)

(m-1-4) edge node[above] {$ 1_{\bd{M}}\times c $} (m-1-5)
(m-2-4) edge node[below] {$ c $} (m-2-5)
(m-1-4) edge node[left]  {$ \mathrm{pr}_{23} $} (m-2-4)
(m-1-5) edge node[right] {$ \mathrm{pr}_X $} (m-2-5)
(m-1-5) edge node[above] {$ a $} (m-1-6);
\end{tikzpicture}
\end{center}
representing morphisms
$$(\mathrm{pr}_T,\mu\times 1_T):\left(\mathrm{pr}_{23},c\cdot (\mu\times 1_T)\right)\ra \left(\mathrm{pr}_T,c\right),\,(c,1_{\bd{M}}\times c):\left(\mathrm{pr}_{23},a\cdot (1_{\bd{M}}\times c)\right) \ra \left(\mathrm{pr}_X,a\right)$$
in $[X/\bd{M}]$. Moreover, $c$ is $\bd{M}$-equivariant $\left(\bd{M}\times T,\mu\times 1_T\right) \ra (X,a)$ and hence we derive that $c\cdot \left(\mu\times 1_T\right) = a\cdot \left(c\times 1_{\bd{M}}\right)$. Thus the morphisms in $[X/\bd{M}]$ described above have common domain. Since $F$ is over $\mathbb{B}\bd{M}$, we derive that their images under $F$ are 
\begin{center}
\begin{tikzpicture}
[description/.style={fill=white,inner sep=2pt}]
\matrix (m) [matrix of math nodes, row sep=3em, column sep=3em,text height=1.5ex, text depth=0.25ex] 
{ \bd{M}\times \bd{M} \times T  &  \bd{M}\times T  & X    &   \bd{M}\times \bd{M}\times T     &  \bd{M}\times X  & X          \\
  \bd{M}\times T                &  T               &      &   \bd{M}\times T                  &  X               &          \\} ;
\path[->,line width=1.0pt,font=\scriptsize]
(m-1-1) edge node[above] {$ \mu\times 1_T $} (m-1-2)
(m-2-1) edge node[below] {$ \mathrm{pr}_T $} (m-2-2)
(m-1-1) edge node[left]  {$ \mathrm{pr}_{23} $} (m-2-1)
(m-1-2) edge node[right] {$ \mathrm{pr}_T $} (m-2-2)
(m-1-2) edge node[above] {$ \gamma $} (m-1-3)

(m-1-4) edge node[above] {$ 1_{\bd{M}}\times c $} (m-1-5)
(m-2-4) edge node[below] {$ c $} (m-2-5)
(m-1-4) edge node[left]  {$ \mathrm{pr}_{23} $} (m-2-4)
(m-1-5) edge node[right] {$ \mathrm{pr}_X $} (m-2-5)
(m-1-5) edge node[above] {$ \alpha $} (m-1-6);
\end{tikzpicture}
\end{center}
This implies that $\gamma \cdot (\mu\times 1_T) = \alpha \cdot (1_{\bd{M}}\times c)$. We deduce that
$$\gamma = \gamma \cdot (\mu \times 1_T)\cdot \langle e, 1_{\bd{M}\times X}\rangle = \alpha \cdot (1_{\bd{M}}\times c) \cdot  \langle e, 1_{\bd{M}\times X}\rangle = \alpha \cdot \langle e,1_X\rangle \cdot c = f\cdot c$$
and the claim is proved. We apply this to $\alpha$ to derive that $\alpha = f\cdot a$. Next recall that $\alpha \cdot \left(\mu \times 1_X\right) = b\cdot \left(1_{\bd{M}} \times \alpha \right)$ because $\alpha$ is an $\bd{M}$-equivariant morphism $\left(\bd{M}\times X, \mu\times 1_X\right)\ra (Y,b)$. Thus
$$b\cdot \left(1_{\bd{M}}\times f\right) = b\cdot \left(1_{\bd{M}}\times \alpha\right)\cdot \left(1_{\bd{M}}\times \langle e, 1_X\rangle \right) = \alpha \cdot \left(\mu \times 1_X\right)\cdot \left(1_{\bd{M}}\times \langle e, 1_X\rangle \right) = \alpha$$
Hence $f\cdot a = \alpha = b\cdot \left(1_{\bd{M}}\times f\right)$. Thus $f$ is $\bd{M}$-equivariant and $F$ is given as in the statement on the subcategory of $[X/\bd{M}]$ consisting of trivial principal $\bd{M}$-bundles. Now consider any principal $\bd{M}$-bundle $\pi:\cP\ra T$ with respect to $\cJ$ and let $d:\cP\ra X$ be a $\bd{M}$-equivariant morphism to $(X,a)$. We know that $F$ sends $(\pi,d)$ to some object of $[Y/\bd{M}]$ of the form $(\pi,\delta)$. It suffices to prove that $\delta = f\cdot d$. For this consider a sieve $S$ in $\cJ(T)$ such that $S$ trivializes $\pi$. Pick $g:\widetilde{T}\ra T$ in $S$ and a cartesian square
\begin{center}
\begin{tikzpicture}
[description/.style={fill=white,inner sep=2pt}]
\matrix (m) [matrix of math nodes, row sep=2em, column sep=2em,text height=1.5ex, text depth=0.25ex] 
{ g^*\cP &  \cP    \\
  \widetilde{T} &  T           \\} ;
\path[->,line width=1.0pt,font=\scriptsize]
(m-1-1) edge node[above] {$ g'  $} (m-1-2)
(m-2-1) edge node[below] {$ g $} (m-2-2)
(m-1-1) edge node[left] {$ \pi_g $} (m-2-1)
(m-1-2) edge node[right] {$ \pi $} (m-2-2);
\end{tikzpicture}
\end{center}
Then $(\pi_g,d\cdot g')$ is an object of $[X/\bd{M}]$. Since $F$ is over $\mathbb{B}\bd{M}$, we derive that $F(\pi_g,d\cdot g') = (\pi_g,\delta \cdot g')$. By definition $\pi_g$ is trivial $\bd{M}$-bundle. Thus (from what we proved above) we have
$$\delta\cdot g' = f\cdot d\cdot g'$$
This holds for pullback $g'$ of every $g$ in $S$ along $\pi$. These pullbacks $\{g'\}_{g\in S}$ generate some sieve $S'$ on $\cP$ and the formula
$$\delta\cdot h = f\cdot d\cdot h$$
holds for every $h$ in $S'$. Moreover, $S'$ is a covering sieve on a site $(\cB,\cJ)$ i.e. $S'\in \cJ(\cP)$. According to the assumption on $\cJ$ we infer that $h^{\cB}_{\cP} = \Mor_{\cB}\left(-,\cP\right):\cB^{\mathrm{op}}\ra \Set$ is a separated presheaf with respect to $\cJ$. Thus the formula
$$\delta\cdot h = f\cdot d\cdot h$$
which holds for every $h$ in $S'$ implies that $\delta = f\cdot d$.
\end{proof}



































\small
\bibliographystyle{alpha}
\bibliography{../zzz}

\end{document}