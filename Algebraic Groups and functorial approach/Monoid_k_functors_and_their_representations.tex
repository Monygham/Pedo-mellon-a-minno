\input ../pree.tex

\begin{document}

\title{Monoid $k$-functors and their representations}
\date{}
\maketitle

\section{Introduction and notation}
\noindent
In these notes we study algebraic structures in the category of $k$-functors with special emphasis on monoid objects.\\
Throughout these notes $k$ is a fixed commutative ring and $\Alg_k$ denote the category of commutative $k$-algebras. If $A, B$ are $k$-algebras, then we denote by $\Mor_k(A, B)$ the set of all morphisms $A\ra B$ of $k$-algebras. Similarly if $X, Y$ are $k$-schemes (i.e. schemes together with morphism to $\Spec k$), then we denote by $\Mor_k(X,Y)$ the set of all morphisms $X\ra Y$ of $k$-schemes (morphisms of schemes that preserve structure morphisms to $\Spec k$). If $M$ is a monoid, then we denote by $M^*$ the group of units of $M$. If $R$ is a ring, then we denote by $R^{\times}$ its multplicative monoid. Let $A$ be a $k$-algebra and let $V$ be an $A$-module and $v$ be an element of $V$. Then for $A$-algebra $B$ we denote by $v_B$ the element $1\otimes v$ of $B\otimes_AV$. If $V$ is an $A$-module, then we denote $\Hom_A(V,A)$ by $V^{\vee}$. Thus we have a contravariant functor
$$(-)^{\vee}:\Mod(A)^{\mathrm{op}}\ra \Mod(A)$$


\section{Algebraic structures in the category of $k$-functors}
\noindent
In the sequel we assume that the reader is familiar with notions of a monoid, group etc. in arbitrary category with finite products. For definitions and some discussion related to these notions cf. {\cite[pages 2-5]{Maclane}}.

\begin{definition}
\textit{A monoid (group, abelian group, ring) $k$-functor} is a monoid (group, abelian group, ring) object in the category of $k$-functors.
\end{definition}

\begin{example}\label{example:endomorphisms_of_k_functor}
Let $\fX$ be a $k$-functor such that $\iMor_k(\fX,\fX)$ exists. Then $\iMor_k(\fX,\fX)$ is a monoid $k$-functor with respect to composition of morphisms.
\end{example}

\begin{example}\label{example:units_k_functor_for_a_monoid}
Let $\fG$ be a monoid $k$-functor. Then we denote by $\fG^*$ the $k$-subfunctor of $\fG$ defined by
$$\fG^{*}(A) = \fG(A)^{*}$$
for every $k$-algebra $A$. We call $\fG^{*}$ \textit{the unit group $k$-functor of $\fG$}.
\end{example}

\begin{example}\label{example:constant_ring_k_functor}
Basic example of a ring $k$-functor is a $k$-functor $\ideal{K}$ given by
$$\fK(A) = k,\,\fK(f) = 1_k$$
for any $k$-algebra $A$ and morphism $f$ of $k$-algebras. It can be described as a constant $k$-functor ({\cite[page 67]{Maclane}}) corresponding to $k$.
\end{example}

\begin{definition}
Let $\fR$ be a ring $k$-functor. Then we denote by $\fR^{\times}$ the $k$-subfunctor of $\fR$ defined by
$$\fR^{\times}(A) = \fR(A)^{\times}$$
for every $k$-algebra $A$. We call $\fR^{\times}$ \textit{the multiplicative monoid $k$-functor of $\fR$}.
\end{definition}

\begin{definition}
Let $\fA$ be a commutative ring $k$-functor. \textit{An $\fA$-algebra} is an $\fA$-algebra object in the category of $k$-functors.
\end{definition}

\section{Global regular functions on a $k$-functor}
\noindent
Recall the ring $k$-functor $\fK$ from Example \ref{example:constant_ring_k_functor}. Note that a $\fK$-algebra $\fA$ can be viewed as a functor $\fA:\Alg_k\ra \Alg_k$.

\begin{definition}
The $\fK$-algebra $\fO_k$ given by the identity functor on $\Alg_k$ is called \textit{the structure $\fK$-algebra}.
\end{definition}
\noindent
Let $|-|:\Alg_k\ra \Set$ be the forgetful $k$-functor. Note that $|-|$ is the underlying $k$-functor of $\fK$-algebra $\fO_k$. Recall that the affine line $\mathbb{A}^1_k$ is an affine $k$-scheme having $k$-algebra of polynomials with one variable as a $k$-algebra of regular functions.

\begin{fact}\label{fact:affineline_as_forgetfulfunctor}
Let $|-|:\Alg_k\ra \Set$ be the forgetful $k$-functor. Then we have natural isomorphism
$$\fP_{\mathbb{A}^1_k} \cong |-|$$
\end{fact}
\begin{proof}
Let $B$ be a $k$-algebra. We have the following chain of identifications
$$\fP_{\mathbb{A}^1_k}(B) = \Mor_k\left(\Spec B, \mathbb{A}^1_k\right) = \Mor_k\left(\Spec B, \Spec k[x]\right) = \Mor_k\left(k[x], B\right) = |B|$$
natural in $B$.
\end{proof}
\noindent
In particular, since $|-|$ carries the structure $\fK$-algebra $\fO_k$, we derive that $\fP_{\mathbb{A}^1_k}$ admits a structure of $\fK$-algebra isomorphic to $\fO_k$.\\
No we introduce regular functions on $k$-functors.

\begin{definition}
Let $\fX$ be a $k$-functor and assume that $\iMor_k\left(\fX, \fO_k\right)$ is a set. Then $\Mor_k\left(\fX, \fO_k\right)$ is a $k$-algebra with respect to the structure induced by $\fO_k$. We call this $k$-algebra \textit{the $k$-algebra of global regular functions on $\fX$}. Its elements are called \textit{global regular functions on $\fX$}.
\end{definition}

\begin{definition}
Let $\fX$ be a $k$-functor. Suppose that $A$ is a $k$-algebra, $x\in \fX(A)$ and $f\in \Mor_k\left(\fX,\fO_k\right)$. The element $f(x) \in A$ is called \textit{the value of $f$ on a point $x$}.
\end{definition}
\noindent
For given $k$-functor $\fX$ we describe $k$-algebra operations on $\Mor_k\left(\fX,\fO_k\right)$ in terms of values of its elements on points of $\fX$. For this consider $\alpha \in k$ and $f$, $g_1\in \Mor_k\left(\fX, \fO_k\right)$. We have formulas
$$\left(f+g\right)(x) = f(x)+g(x),\,\left(f\cdot g\right)(x) = f(x)\cdot g(x),\,\left(\alpha \cdot f\right)(x) = \alpha \cdot f(x)$$
in which right hand side are $k$-algebra operations in $A$.

\begin{example}\label{example:regular_functions_as_an_algebra_over_structure_algebra}
Let $\fX$ be a $k$-functor and assume that $\iMor_k(\fX,\fO_k)$ exists. Fix $k$-algebra $A$. Note that $\Mor_A(\fX_A,\fO_A)$ is an $A$-algebra of global regular functions on $\fX_A$. Moreover, if $B$ is an $A$-algebra, then
$$\Mor_A(\fX_A,\fO_A) \ni f \mapsto f_B\in \Mor_B(\fX_B,\fO_B)$$
is a morphism of $A$-algebras. This implies that $\iMor_k(\fX,\fO_k)$ admits a canonical structure of an $\fO_k$-algebra $k$-functor.
\end{example}

\section{Internal hom and product of $k$-functors}
\noindent
We denote by $\bd{1}$ a $k$-functor that assigns to every $k$-algebra a set with one element. Then for every $k$-algebra $A$ the restriction $\bd{1}_A$ is a terminal object in the category of $A$-functors.

\begin{fact}\label{fact:points_and_morphisms_from_terminal_k_functor}
Let $\fX$ be a $k$-functor. Suppose $A$ is a $k$-algebra and $x\in \fX(A)$. Then $x$ determines a morphism $\bd{1}_{A}\ra \fX_A$ that for every $A$-algebra $B$ with structural morphism $f:A\ra B$ sends a unique element of $\bd{1}_{A}(B)$ to $\fX(f)(x)\in \fX_A(B)$. This gives rise to a bijection
$$\fX(A)\cong \Mor_{A}\left(\bd{1}_{A},\fX_A\right)$$
\end{fact}
\begin{proof}
Left to the reader as an exercise.
\end{proof}
\noindent
The discussion below is partially an application of the main result in {\cite[section 6]{Presheaves}}. For reader's convenience we make our presentation self-contained.

\begin{definition}
Let $\fU,\fX,\fY$ be $k$-functors and let $\sigma:\fU\times \fX\ra \fY$ be a morphism of $k$-functors. Fix $z\in \fU(A)$ for some $k$-algebra $A$. We denote by $i_z:\bd{1}_A\ra \fU_A$ the morphism of $A$-functors corresponding to $z$ by Fact \ref{fact:points_and_morphisms_from_terminal_k_functor}. Since $\bd{1}_A$ is terminal $A$-functor, a morphism $\sigma_A\cdot \left(i_z\times 1_{\fX_A}\right)$ is isomorphic to a morphism $\sigma_z:\fX_A\ra \fY_A$ of $A$-functors. We call $\sigma_z$ \textit{the slice of $\sigma$ along $z$}.
\end{definition}

\begin{definition}
Let $\fX, \fY$ be $k$-functors. Let $\fJ$ be a $k$-functor such that $\fJ(A)$ is a subset of a class $\Mor_A\left(\fX_A,\fY_A\right)$ for every $k$-algebra $A$. Assume that for every morphism $f:A\ra B$ of $k$-algebras and every $\sigma\in \fJ(A)$ we have
$$\fJ(f)(\sigma) = \sigma_B$$
where $\sigma_B\in \Mor_B\left(\fX_B,\fY_B\right)$ is the restriction of $\sigma$ along $f$. Then we call $\fJ$ \textit{a $k$-subfunctor of internal hom of $\fX$ and $\fY$}.
\end{definition}

\begin{definition}
Let $\fX, \fY,\fU$ be $k$-functors and let $\sigma:\fU\times \fX \ra \fY$ be a morphism of $k$-functors. Suppose that $\fJ$ is a $k$-subfunctor of internal hom of $\fX$ and $\fY$. Assume that $\sigma_z:\fX_A\ra \fY_A$ is contained in $\fJ(A)$ for every $k$-algebra $A$ and $z\in \fU(A)$. Then we call $\sigma$ \textit{a family of $\fJ$-morphisms parametrized by $\fU$}.
\end{definition}
\noindent
Let $\fJ$ be a $k$-subfunctor of internal hom of $\fX$ and $\fY$. Assume that $\sigma:\fU\times \fX\ra \fY$ is a $\fJ$-family of morphism parametrized by $\fU$. Then the family of maps
$$\fU(A)\ni z\mapsto \sigma_z\in \fJ(A)$$
gives rise to a morphism $\tau:\fU\ra \fJ$ of $k$-functors. Indeed, for a morphism $f:A\ra B$ of $k$-algebras and $z\in \fU(A)$ we have
$$\sigma_B\cdot \left(i_{\fU(f)(z)}\times 1_{\fX_B}\right) = \left(\sigma_A\cdot \left(i_{z}\times 1_{\fX_A}\right)\right)_B$$
and hence $\sigma_{\fU(f)(z)} = \left(\sigma_z\right)_B$. This gives rise to a map $\Phi$ of classes
$$\bigg\{\mbox{families }\fU\times \fX\ra \fY\mbox{ of }\fJ\mbox{-morphisms parametrized by }\fU\bigg\}\ni \sigma\mapsto \tau\in \Mor_k\left(\fU,\fJ\right)$$
Consider next a morphism $\tau:\fU\ra \fJ$ of $k$-functors and define $\sigma:\fU\times \fX\ra \fY$ by formula $\sigma^A(z,x) = \left(\tau^A(z)\right)^A(x)$ for every $k$-algebra $A$ and points $z\in \fU(A),\,x\in \fX(A)$. Let $f:A\ra B$ be a morphism of $k$-algebras. Then
$$\sigma^B\left(\fU(f)(z),\fX(f)(x)\right) = \left(\tau^B\left(\fU(f)(z)\right)\right)^B\left(\fX(f)(x)\right) = \left(\left(\tau^A(z)\right)_B\right)^B\left(\fX(f)(x)\right) =$$
$$ = \left(\tau^A(z)\right)^B\left(\fX(f)(x)\right) = \fY(f)\left(\left(\tau^A(z)\right)^A(x)\right) = \fY(f)\left(\sigma^A(z,x)\right)$$
Thus $\sigma:\fU\times \fX\ra \fY$ is a morphism of $k$-functors. For every $k$-algebra $A$ and $z\in\fU(A)$ we have $\sigma_z = \tau^A(z)$. Indeed, let $f:A\ra B$ be a morphism of $k$-algebras and $x$ be an element in $\fX(B)$ then we have
$$\left(\sigma_z\right)^B(x) = \sigma^B\left(\fU(f)(z),x\right) = \left(\tau^B\left(\fU(f)(z)\right)\right)^B(x) = \left(\left(\tau^A(z)\right)_B\right)^B(x) = \left(\tau^A(z)\right)^B(x)$$
Hence $\sigma$ is a family of $\fJ$-morphisms parametrized by $\fU$. This gives rise to a map $\Psi$ of classes
$$\Mor_k\left(\fU,\fJ\right) \ni \tau \mapsto \sigma\in \bigg\{\mbox{families }\fU\times \fX\ra \fY\mbox{ of }\fJ\mbox{-morphisms parametrized by }\fU\bigg\}$$
Now we have the following result, which is an instance {\cite[Theorem 6.3]{Presheaves}}. To make presentation self-contained we give a complete proof.  

\begin{theorem}\label{theorem:k_subfunctors_of_hom}
Let $\fX$, $\fY$ be $k$-functors and let $\fJ$ be a $k$-subfunctor of internal hom of $\fX$ and $\fY$. Then maps 
$$\Phi:\bigg\{\mbox{families }\fU\times \fX\ra \fY\mbox{ of }\fJ\mbox{-morphisms parametrized by }\fU\bigg\} \ra  \Mor_{k}\left(\fU,\fJ\right)$$
and
$$\Psi:\Mor_{k}\left(\fU,\fJ\right) \ra \bigg\{\mbox{families }\fU\times \fX\ra \fY\mbox{ of }\fJ\mbox{-morphisms parametrized by }\fU\bigg\}$$
are mutually inverse bijections.
\end{theorem}
\begin{proof}
Pick a morphism $\tau:\fU\ra \fJ$ of $k$-functors. Let $A$ be a $k$-algebra and $z\in \fU(A)$. In the discussion preceding the statement we showed that $\Psi(\tau)_z = \tau^A(z)$. Thus
$$\left(\Phi(\Psi(\tau))\right)^A(z) = \Psi(\tau)_z = \tau^A(z)$$
and hence $\Phi\cdot \Psi$ is the identity.\\
Pick a family of $\fJ$-morphism $\sigma:\fU\times \fX\ra \fY$ parametrized by $\fU$. Let $A$ be a $k$-algebra and $z\in \fU(A)$, $x\in \fX(A)$ be points. Then
$$\left(\Psi\left(\Phi(\sigma)\right)\right)^A(z,x) = \left(\Phi(\sigma)^A(z)\right)^A(x) = \sigma_z^A(x) = \sigma^A(z,x)$$
Thus $\Psi\cdot \Phi$ is the identity map.
\end{proof}
\noindent
Now we formulate some consequences of Theorem \ref{theorem:k_subfunctors_of_hom}.

\begin{corollary}\label{corollary:hom_k_functors}
Let $\fX$, $\fY$ be $k$-functors. Assume that for every $k$-algebra $A$ the class $\Mor_{A}\left(\fX_A,\fY_A\right)$ is a set. Then there is a bijection 
$$\Mor_{k}\left(\fU\times \fX,\fY\right)\ra  \Mor_{k}\left(\fU,\iMor_{k}\left(\fX,\fY\right)\right)$$
of classes.
\end{corollary}

\begin{definition}
Let $\fX,\fY$ be $k$-functors. If $\Iso_{A}\left(\fX_A,\fY_A\right)$ is a set for every $k$-algebra $A$, then we define a $k$-subfunctor $\iIso_k\left(\fX,\fY\right)$ of $\iMor_k\left(\fX,\fY\right)$ by
$$\iIso_k\left(\fX,\fY\right)(A) = \Iso_A\left(\fX_A,\fY_A\right)$$
for every $k$-algebra $A$. We call $\iIso_k\left(\fX,\fY\right)$ \textit{the $k$-functor of isomorphism}.
\end{definition}

\begin{definition}
Let $\fX, \fY,\fU$ be $k$-functors and let $\sigma:\fU\times \fX \ra \fY$ be a morphism of $k$-functors. Assume that $\sigma_z:\fX_A\ra \fY_A$ is an isomorphism of $A$-functors for every $k$-algebra $A$. Then we call $\sigma$ \textit{a family of isomorphisms parametrized by $\fU$}.
\end{definition}

\begin{corollary}\label{corollary:hom_isomorphism_bijection}
Let $\fU,\fX,\fY$ be $k$-functors and suppose that for every $k$-algebra $A$ the class $\Iso_A\left(\fX_A,\fY_A\right)$ is a set. The the following map
$$\bigg\{\mbox{families }\fU\times \fX\ra \fY\mbox{ of isomorphism parametrized by }\fU\bigg\}\ra \Mor_k\left(\fU,\iIso_k\left(\fX,\fY\right)\right)$$
is a bijection of classes.
\end{corollary}


\section{Actions of monoid $k$-functors}\label{section:actions_of_monoid_k_functors}
\noindent
In this section we assume that the reader is familiar with notion of an action of a monoid object in arbitrary category with finite products. For definitions and some discussion related to these notions cf. {\cite[pages 5]{Maclane}}.\\
Let $\fG$ be a monoid $k$-functor and $\fX$ be a $k$-functor together with an action $\alpha:\fG\times \fX \ra \fX$. Next assume that $k$-functor $\iMor_k(\fX,\fX)$ exists. By Example \ref{example:endomorphisms_of_k_functor} it is a monoid $k$-functor. We define a morphism $\rho:\fG\ra \iMor_k(\fX,\fX)$ of $k$-functors by formula $\rho(g) = \alpha_{g}$. Note that by discussion preceding Theorem \ref{theorem:k_subfunctors_of_hom}, we deduce that $\rho$ is a well defined morphism of $k$-functors. We show now that $\rho$ is a morphism of monoids. For this pick $k$-algebra $A$ and $g_1, g_2\in \fG(A)$. Since $\alpha$ is an action, we deduce that $\alpha_{g_1 \cdot g_2} = \alpha_{g_1} \cdot \alpha_{g_2}$ and hence also
$$\rho(g_1\cdot g_2) = \alpha_{g_1 \cdot g_2} = \alpha_{g_1}\cdot \alpha_{g_2} = \rho(g_1)\cdot \rho(g_2)$$
Therefore, $\rho$ is a morphism of monoid $k$-functors. This shows how to construct a morphism of monoid $k$-functors $\rho$ from an action $\alpha$ of $\fG$.

\begin{theorem}\label{theorem:actions_and_monoid_morphisms}
Let $\fG$ be a monoid $k$-functor and let $\fX$ be a $k$-functor such that $\iMor_{k}(\fX,\fX)$ exists. Suppose that
\begin{center}
\begin{tikzpicture}
[description/.style={fill=white,inner sep=2pt}]
\matrix (m) [matrix of math nodes, row sep=3em, column sep=4em,text height=1.5ex, text depth=0.25ex] 
{\bigg\{\mbox{actions of $\fG$ on $\fX$}\bigg\} & \bigg\{\mbox{Morphisms $\rho:\fG\ra \iMor_{k}(\fX,\fX)$ of monoid $k$-functors}\bigg\} \\};
\path[->,line width=1.0pt,font=\scriptsize]  
(m-1-1) edge node[auto] {$ $} (m-1-2);
\end{tikzpicture}
\end{center}
is a map of classes described above. Then it is bijection.
\end{theorem}
\begin{proof}
Our goal is to construct the inverse of the map. Substitute $\fJ = \iMor_k(\fX,\fX)$ in Theorem \ref{theorem:k_subfunctors_of_hom}. Consider maps
$$\Phi:\bigg\{\mbox{families }\fG\times \fX\ra \fX\mbox{ of morphisms}\bigg\} \ra  \Mor_{k}\left(\fG,\iMor_k(\fX,\fX)\right)$$
and
$$\Psi:\Mor_{k}\left(\fG, \iMor_k(\fX,\fX)\right) \ra \bigg\{\mbox{families }\fG\times \fX\ra \fX\mbox{ of morphisms}\bigg\}$$
in that Theorem. Then the map in the statement above is the restriction of $\Phi$ to $\fG$-actions on $\fX$ on the right and morphisms $\fG\ra \iMor_k(\fX,\fX)$ of monoid $k$-functors on the left. Since by Theorem \ref{theorem:k_subfunctors_of_hom} maps $\Phi$ and $\Psi$ are mutually inverse, it suffices to check that $\Psi$ sends a morphism $\rho:\fG\ra \iMor_{k}(\fX,\fX)$ of monoids to an action of $\fG$ on $\fX$. For this denote $\Psi(\rho)$ by $\alpha$. Consider $k$-algebra $A$ and $A$-points $g_1,g_2\in \fG(A),\,x\in \fX(A)$. Then
$$\alpha\left(g_1, \alpha(g_2, x)\right) = \rho(g_1)\left(\rho(g_2)(x)\right) = \left(\rho(g_1)\cdot \rho(g_2)\right)(x) = \rho\left(g_1\cdot g_2\right)(x) = \alpha\left(g_1\cdot g_2, x\right)$$
Therefore, $\alpha$ is an action of $\fG$ on $\fX$.
\end{proof}

\begin{proposition}\label{proposition:morphism_of_monoid_actions}
Let $\fG$ be a monoid $k$-functor and let $\fX_1$, $\fX_2$ be $k$-functors such that $\iMor_k(\fX_1,\fX_1),\iMor_k(\fX_2,\fX_2)$ exist. Suppose that $\alpha_1:\fG\times \fX_1 \ra \fX_1,\,\alpha_2:\fG\times \fX_2 \ra \fX_2$ are actions of $\fG$, respectively. Suppose that $\sigma:\fX_1\ra \fX_2$ is a morphism of $k$-functors. Then the following assertions are equivalent.
\begin{enumerate}[label=\emph{\textbf{(\roman*)}}, leftmargin=3.0em]
\item The square
\begin{center}
\begin{tikzpicture}
[description/.style={fill=white,inner sep=2pt}]
\matrix (m) [matrix of math nodes, row sep=3em, column sep=5em,text height=1.5ex, text depth=0.25ex] 
{  \fG\times \fX_1   & \fG\times \fX_2 \\
   \fX_1             & \fX_2 \\} ;
\path[->,line width=1.0pt,font=\scriptsize]  
(m-1-1) edge node[above] {$ 1_{\fG}\times \sigma  $} (m-1-2)
(m-2-1) edge node[below] {$ \sigma $} (m-2-2)
(m-1-1) edge node[left] {$ \alpha_1 $} (m-2-1)
(m-1-2) edge node[right] {$ \alpha_2 $} (m-2-2);
\end{tikzpicture}
\end{center}
is commutative.
\item For every $k$-algebra $A$ and $g\in \fG(A)$ we have
$$\sigma_A \cdot \rho_1(g) = \rho_2(g) \cdot \sigma_A$$
where $\rho_1:\fG\ra \iMor_k(\fX_1,\fX_1)$ and $\rho_2:\fG\ra \iMor_{k}(\fX_2,\fX_2)$ are morphism of monoid $k$-functors corresponding to $\alpha_1$ and $\alpha_2$, respectively.
\end{enumerate}
\end{proposition}
\begin{proof}
Conditions expressed in \textbf{(i)} and \textbf{(ii)} are directly translatable to each other by virtue of the bijection in Theorem \ref{theorem:actions_and_monoid_morphisms}. 
\end{proof}

\begin{definition}
Let $\fG$ be a monoid $k$-functor and let $(\fX_1,\alpha_1)$, $(\fX_2,\alpha_2)$ be $k$-functors with actions of $\fG$. Suppose that $\sigma:\fX_1\ra \fX_2$ is a morphism $k$-functors such that the square
\begin{center}
\begin{tikzpicture}
[description/.style={fill=white,inner sep=2pt}]
\matrix (m) [matrix of math nodes, row sep=3em, column sep=5em,text height=1.5ex, text depth=0.25ex] 
{  \fG\times \fX_1   & \fG\times \fX_2 \\
   \fX_1             & \fX_2 \\} ;
\path[->,line width=1.0pt,font=\scriptsize]  
(m-1-1) edge node[above] {$ 1_{\fG}\times \sigma  $} (m-1-2)
(m-2-1) edge node[below] {$ \sigma $} (m-2-2)
(m-1-1) edge node[left] {$ \alpha_1 $} (m-2-1)
(m-1-2) edge node[right] {$ \alpha_2 $} (m-2-2);
\end{tikzpicture}
\end{center}
is commutative. Then $\sigma$ is called \textit{an $\fG$-equivariant morphism}.
\end{definition}

\section{Modules over ring $k$-functors}

\begin{definition}
Let $\fR$ be a ring $k$-functor. Suppose that $\fM$ is an abelian group $k$-functor and there exists a morphism $\fR \times \fM\ra \fM$ of $k$-functors that for each $k$-algebra $A$ makes $\fM(A)$ into an $\fR(A)$-module. Then we say that $\fM$ is \textit{a module $k$-functor over $\fR$}.
\end{definition}

\begin{definition}
Let $\fR$ be an ring $k$-functor and let $\fM_1,\fM_2$ be module $k$-functors over $\fR$. Suppose that $\sigma:\fM_1\ra \fM_2$ is a morphism of abelian group $k$-functors such that the diagram
\begin{center}
\begin{tikzpicture}
[description/.style={fill=white,inner sep=2pt}]
\matrix (m) [matrix of math nodes, row sep=3em, column sep=5em,text height=1.5ex, text depth=0.25ex] 
{  \fR \times \fM_1   & \fR \times \fM_2  \\
   \fM_1              & \fM_2            \\} ;
\path[->,line width=1.0pt,font=\scriptsize]  
(m-1-1) edge node[above] {$ 1_{\fR} \times \sigma  $} (m-1-2)
(m-2-1) edge node[below] {$ \sigma $} (m-2-2)
(m-1-1) edge node[left] {$ \alpha_1 $} (m-2-1)
(m-1-2) edge node[right] {$ \alpha_2 $} (m-2-2);
\end{tikzpicture}
\end{center}
is commutative, where $\alpha_i:\fR\times \fM_i\ra \fM_i$ define $\fR$-module structure on $\fM_i$ for $i=1,2$. Then $\sigma$ is \textit{a morphism of modules over $\fR$}.
\end{definition}
\noindent
Let $\fM_1$ and $\fM_2$ be module $k$-functors over $\fR$. We denote by
$$\Hom_{\fR}\left(\fM_1,\fM_2\right)$$
the class of all morphisms of modules $\fM_1\ra \fM_2$ over $\fR$. We denote the category of $\fR$-modules by $\Mod\left(\fR\right)$.

\begin{definition}
Let $\fM_1$ and $\fM_2$ be module $k$-functors over $\fR$. Assume that $\Hom_{\fR_A}\left((\fM_1)_A,(\fM_2)_A\right)$ is a set for every $k$-algebra $A$. Then we define a $k$-subfunctor $\shHom_{\fR}(\fM_1,\fM_2)$ of internal hom of $\fM_1$ and $\fM_2$ by formula
$$\Alg_k\ni A \mapsto \Hom_{\fR_A}\left((\fM_1)_A,(\fM_2)_A\right) \in \Set$$
We call $\shHom_{\fR}(\fM_1,\fM_2)$ \textit{a $k$-functor of module morphisms of $\fM_1$ and $\fM_2$}.
\end{definition}
\noindent
If $\fM$ is a module $k$-functor over some ring $k$-functor $\fR$, then we denote (if it exists) $\shHom_{\fR}(\fM,\fM)$ by $\cE nd_{\fR}(\fM)$.

\begin{example}\label{example:endomorphisms_of_module_k_functor}
Let $\fM$ be a module over a ring $k$-functor $\fR$. Assume that $\cE nd_{\fR}(\fM)$ exists. Then $\cE nd_{\fR}(\fM)$ is a ring $k$-functor with respect to composition of morphisms of modules as the multiplication and the usual addition of module morphisms. Moreover, if $\fA$ is a commutative ring $k$-functor, then $\cE nd_{\fA}(\fM)$ (if exists) admits additional structure of a $\fA$-algebra $k$-functor induced via a unique morphism $\fA\ra \cE nd_{\fR}(\fM)$ of ring $k$-functors that sends $1\mapsto 1_{\fM}$.
\end{example}
\noindent
Let $\fA$ be a commutative ring $k$-functor and let $\fR$ be a $\fA$-algebra $k$-functor. This means that there exists a morphism $\fA\ra \fR$ of ring $k$-functors and for every $k$-algebra $A$ induced morphism $\fA(A)\ra \fR(A)$ sends $\fA(A)$ to the center of a ring $\fR(A)$. Fix a module $\fM$ over $\fA$. Next assume that $k$-functor $\cE nd_{\fA}(\fM)$ exists. By Example \ref{example:endomorphisms_of_module_k_functor} it is a ring $k$-functor.

\begin{definition}
In the setting above suppose that $\alpha:\fR\times \fM\ra \fM$ is a morphism of $k$-functors. Suppose that $\alpha$ makes $\fM$ into $\fR$-module and moreover, for every $k$-algebra $A$ and for every point $x\in \fR(A)$ morphism $\alpha_x$ is a morphism of $\fA_A$-modules. Then $\alpha$ is called \textit{a $\fA$-linear $\fR$-action on $\fM$}.
\end{definition}
\noindent
We continue the discussion. We assume that we are given an $\fA$-linear $\fR$-action $\alpha:\fR\times \fM \ra \fM$ on $\fM$. We define a morphism $\rho:\fR\ra \cE nd_{\fA}(\fM)$ of $k$-functors by formula $\rho(r) = \alpha_r$. As in Section \ref{section:actions_of_monoid_k_functors} we can prove that $\rho$ is a morphism of ring $k$-functors. Now we have the following result.

\begin{theorem}\label{theorem:linear_morphisms_and_homomorphisms_of_rings}
Let $\fR$ be an algebra $k$-functor over commutative ring $\fA$ $k$-functor and let $\fM$ be a $\fA$-module such that $\cE nd_{\fA}(\fM)$ exists. Suppose that
\begin{center}
\begin{tikzpicture}
[description/.style={fill=white,inner sep=2pt}]
\matrix (m) [matrix of math nodes, row sep=3em, column sep=4em,text height=1.5ex, text depth=0.25ex] 
{\bigg\{\mbox{$\fA$ linear actions of $\fR$ on $\fM$}\bigg\} & \bigg\{\mbox{Morphisms $\rho:\fR\ra \cE nd_{\fO_k}(\fM)$ of ring $k$-functors}\bigg\} \\};
\path[->,line width=1.0pt,font=\scriptsize]  
(m-1-1) edge node[auto] {$ $} (m-1-2);
\end{tikzpicture}
\end{center}
is a map of classes described above. Then it is bijection.
\end{theorem}
\begin{proof}
The proof is similar to the proof of Theorem \ref{theorem:actions_and_monoid_morphisms}.
\end{proof}

\section{Monoid algebra $\fO_k[\fG]$ and its modules}

\begin{definition}
Let $\fG$ be a monoid $k$-functor. Then we construct an $\fO_k$-algebra $\fO_k[\fG]$ as follows. For every $k$-algebra $A$ we define
$$\fO_k[\fG](A) = A\big[\fG(A)\big]$$
where the right hand side is monoid $A$-algebra for the abstract monoid $\fG(A)$. The structure of monoid $k$-functor on $\fG$ and $\fK$-algebra $\fO_k$ makes $\fO_k[\fG]$ into a ring $k$-functor. Moreover, we have a morphism $\fO_k\ra \fO_k[\fG]$ which for every $k$-algebra $A$ is given by the canonical inclusion
$$A \hookrightarrow A\big[\fG(A)\big]$$
Thus $\fO_k[\fG]$ is $\fO_k$-algebra. We call $\fO_k[\fG]$ \textit{a monoid $\fO_k$-algebra over $\fG$}.
\end{definition}

\begin{fact}\label{fact:universal_property_of_monoid_algebra}
Let $\fG$ be a monoid $k$-functor and let $\fR$ be an $\fO_k$-algebra $k$-functor. Then every morphism
$$\sigma:\fG\ra \fR^{\times}$$
of monoid $k$-functors admits a unique extension
$$\tilde{\sigma}:\fO_k[\fG]\ra \fR$$
to a morphism of $\fO_k$-algebras.
\end{fact}
\begin{proof}
This follows from the analogical universal property of algebras over abstract monoids.
\end{proof}

\begin{definition}
Let $\fG$ be a monoid $k$-functor and let $\fM$ be a module over $\fO_k$. Suppose that $\alpha:\fG\times \fM\ra \fM$ is an action of $\fG$ such that for any $k$-algebra $A$ and point $g\in \fG(A)$ morphism $\alpha_{g}:\fM_A\ra \fM_A$ is a morphism of $\fO_A$-modules. Then $\alpha$ is called \textit{a linear $\fG$-action on $\fM$}.
\end{definition}
\noindent
Suppose now that $\fG$ is a monoid $k$-functor and $\fM$ is a module $\fO_k$. Note that every linear $\fG$-action $\alpha:\fG\times \fM \ra \fM$ extends uniquely to a $\fO_k$-linear action $\fO_k[\fG]\times \fM\ra \fM$ of monoid $\fO_k$-algebra. This gives a bijection
\begin{center}
\begin{tikzpicture}
[description/.style={fill=white,inner sep=2pt}]
\matrix (m) [matrix of math nodes, row sep=3em, column sep=4em,text height=1.5ex, text depth=0.25ex] 
{\bigg\{\mbox{Linear actions of $\fG$ on $\fM$}\bigg\} & \bigg\{\mbox{$\fO_k$-linear actions $\fO_k[\fG]\times \fM\ra \fM$}\bigg\} \\};
\path[->,line width=1.0pt,font=\scriptsize]  
(m-1-1) edge node[auto] {$ $} (m-1-2);
\end{tikzpicture}
\end{center}
Next assume that $k$-functor $\cE nd_{\fO_k}(\fM)$ exists. By Example \ref{example:endomorphisms_of_module_k_functor} it is an $\fO_k$-algebra $k$-functor. Next by Theorem \ref{theorem:linear_morphisms_and_homomorphisms_of_rings} we have a bijection
\begin{center}
\begin{tikzpicture}
[description/.style={fill=white,inner sep=2pt}]
\matrix (m) [matrix of math nodes, row sep=3em, column sep=4em,text height=1.5ex, text depth=0.25ex] 
{\bigg\{\mbox{$\fO_k$-linear actions of $\fO_k[\fG]\times \fM\ra \fM$}\bigg\} & \bigg\{\mbox{Morphisms $\fO_k[\fG]\ra \cE nd_{\fO_k}(\fM)$ of $\fO_k$-algebras}\bigg\} \\};
\path[->,line width=1.0pt,font=\scriptsize]  
(m-1-1) edge node[auto] {$ $} (m-1-2);
\end{tikzpicture}
\end{center}
Finally Fact \ref{fact:universal_property_of_monoid_algebra} implies that we have a bijection
\begin{center}
\begin{tikzpicture}
[description/.style={fill=white,inner sep=2pt}]
\matrix (m) [matrix of math nodes, row sep=3em, column sep=4em,text height=1.5ex, text depth=0.25ex] 
{\bigg\{\mbox{Morphisms $\fO_k[\fG]\ra \cE nd_{\fO_k}(\fM)$ of $\fO_k$-algebras} \bigg\} & \bigg\{\mbox{Morphisms $\fG\ra \cE nd_{\fO_k}(\fM)$ of monoids}\bigg\} \\};
\path[->,line width=1.0pt,font=\scriptsize]  
(m-1-1) edge node[auto] {$ $} (m-1-2);
\end{tikzpicture}
\end{center}
This chain of bijections sends a linear action $\alpha:\fG\times \fM\ra \fM$ of $\fG$ to a morphism $\rho:\fG\ra \cE nd_{\fO_k}(\fM)$ of monoid $k$-functors given by $\rho(g) = \alpha_{g}$ for every $g\in \fG(A)$ and every $k$-algebra $A$. We proved the following result.

\begin{proposition}\label{proposition:decription_of_linear_monoid_actions_on_modules}
Let $\fG$ be a monoid $k$-functor and let $\fM$ be a $\fO_k$-module such that $\cE nd_{\fO_k}(\fM)$ exists. Then the following classes are in canonical bijections described above.
\begin{enumerate}[label=\emph{\textbf{(\arabic*)}}, leftmargin=3.0em]
\item Linear actions of $\fG$ on $\fM$.
\item $\fO_k$-linear actions $\fO_k[\fG]\times \fM\ra \fM$. These are precisely $\fO_k[\fG]$-modules.
\item Morphisms $\fO_k[\fG]\ra \cE nd_{\fO_k}(\fM)$ of $\fO_k$-algebras.
\item Morphisms $\fG\ra \cE nd_{\fO_k}(\fM)$ of monoids.
\end{enumerate}
Moreover, the bijection between class \emph{\textbf{(1)}} and \emph{\textbf{(2)}} does not require the existence of $\cE nd_{\fO_k}(\fM)$.
\end{proposition}
\noindent
Now in a similar manner we can describe morphisms.

\begin{proposition}\label{proposition:morphism_of_monoid_modules}
Let $\fG$ be a monoid $k$-functor and let $\fM_1$, $\fM_2$ be $k$-functors of $\fO_k$-modules such that $\cE nd_{\fO_k}(\fM_1),\cE nd_{\fO_k}(\fM_2)$ exist. Suppose that $\alpha_1:\fG\times \fM_1 \ra \fM_1,\,\alpha_2:\fG\times \fM_2 \ra \fM_2$ are linear actions of $\fG$. Suppose that $\sigma:\fM_1\ra \fM_2$ is a morphism of modules over $\fO_k$. Then the following assertions are equivalent.
\begin{enumerate}[label=\emph{\textbf{(\roman*)}}, leftmargin=1.5em]
\item The square
\begin{center}
\begin{tikzpicture}
[description/.style={fill=white,inner sep=2pt}]
\matrix (m) [matrix of math nodes, row sep=3em, column sep=5em,text height=1.5ex, text depth=0.25ex] 
{  \fG\times \fM_1   & \fG\times \fM_2 \\
   \fM_1             & \fM_2 \\} ;
\path[->,line width=1.0pt,font=\scriptsize]  
(m-1-1) edge node[above] {$ 1_{\fG}\times \sigma  $} (m-1-2)
(m-2-1) edge node[below] {$ \sigma $} (m-2-2)
(m-1-1) edge node[left] {$ \alpha_1 $} (m-2-1)
(m-1-2) edge node[right] {$ \alpha_2 $} (m-2-2);
\end{tikzpicture}
\end{center}
is commutative.
\item The square
\begin{center}
\begin{tikzpicture}
[description/.style={fill=white,inner sep=2pt}]
\matrix (m) [matrix of math nodes, row sep=3em, column sep=5em,text height=1.5ex, text depth=0.25ex] 
{  \fO_k[\fG] \times \fM_1   & \fO_k[\fG] \times \fM_2 \\
   \fM_1                     & \fM_2 \\} ;
\path[->,line width=1.0pt,font=\scriptsize]  
(m-1-1) edge node[above] {$ 1_{\fO_k[\fG]}\times \sigma  $} (m-1-2)
(m-2-1) edge node[below] {$ \sigma $} (m-2-2)
(m-1-1) edge node[left] {$ \tilde{\alpha_1} $} (m-2-1)
(m-1-2) edge node[right] {$ \tilde{\alpha_2} $} (m-2-2);
\end{tikzpicture}
\end{center}
is commutative, where $\tilde{\alpha_1}$ and $\tilde{\alpha_2}$ are $\fO_k$-linear actions of $\fO_k[\fG]$ corresponding to $\alpha_1$ and $\alpha_2$, respectively. This states that $\sigma$ is a morphism of $\fO_k[\fG]$-modules.
\item For every $k$-algebra $A$ and $g\in \fG(A)$ we have
$$\sigma_A \cdot \tilde{\rho}_1(g) = \tilde{\rho}_2(g) \cdot \sigma_A$$
where $\tilde{\rho}_1:\fO_k[\fG] \ra \cE nd_{\fO_k}(\fM_1)$ and $\tilde{\rho}_2:\fO_k[\fG] \ra \cE nd_{\fO_k}(\fM_2)$ are morphism of $\fO_k$-algebras corresponding to $\tilde{\alpha_1}$ and $\tilde{\alpha_2}$, respectively.
\item For every $k$-algebra $A$ and $g\in \fG(A)$ we have
$$\sigma_A \cdot \rho_1(g) = \rho_2(g) \cdot \sigma_A$$
where $\rho_1:\fG \ra \cE nd_{\fO_k}(\fM_1)$ and $\rho_2:\fG \ra \cE nd_{\fO_k}(\fM_2)$ are restrictions of $\tilde{\rho_1}$ and $\tilde{\rho_2}$, respectively.
\end{enumerate}
The equivalence of \emph{\textbf{(i)}} and \emph{\textbf{(ii)}} does not require the existence of $\cE nd_{\fO_k}(\fM_1)$ and $\cE nd_{\fO_k}(\fM_2)$.
\end{proposition}
\begin{proof}
Conditions expressed in \textbf{(i)}-\textbf{(iv)} are directly translatable to each other by virtue of bijections in Proposition \ref{proposition:decription_of_linear_monoid_actions_on_modules}. 
\end{proof}
\noindent
Let $\fG$ be a monoid $k$-functor. We denote by $\Mod\left(\fO_k[\fG]\right)$ the category of $\fO_k[\fG]$-modules.

\section{Linear representations of a monoid $k$-functors}
\noindent
We start the discussion with some results that relates categories $\Mod(k)$ and $\Mod\left(\fO_k\right)$.

\begin{example}\label{example:additive_k_functor}
Let $V$ be a $k$-module. We define a $k$-functor $V_{\mathrm{a}}$. We set
$$V_{\mathrm{a}}(A) = A\otimes_kV,\,V_{\mathrm{a}}(f) = f\otimes_k1_V$$
for every $k$-algebra $A$ and every morphism $f:A\ra B$ of $k$-algebras. Note that $V_{\mathrm{a}}$ is $\fO_k$-module.\\
Suppose that $\phi:V\ra W$ is a morphism of $k$-modules, then we define $\phi_{\mathrm{a}}:V_{\mathrm{a}}\ra W_{\mathrm{a}}$ by formula
$$\phi_{\mathrm{a}}^A = 1_A\otimes_k\sigma$$
for every $k$-algebra. Then $\phi_{\mathrm{a}}$ is a morphism of $\fO_k$-modules.
\end{example}

\begin{remark}\label{remark:representability_of_additive_functor}
Let $V$ be a finitely generated, projective $k$-module. Then for each $k$-algebra $A$ we have an isomorphism
$$\fP_{\Spec \Sym(V^{\vee})}(A) = \Mor_k\left(\Sym(V^{\vee}), A\right) = \Hom_k\left(V^{\vee},A\right) \cong A\otimes_kV$$
Clearly this isomorphism is natural in $A$. Thus $V_{\mathrm{a}}$ is representable by a $k$-scheme $\Spec \Sym(V^{\vee})$.
\end{remark}

\begin{proposition}\label{proposition:inclusion_of_k_modules_into_O_k_modules}
The functor $\left(-\right)_{\mathrm{a}}:\Mod\left(k\right)\ra \Mod\left(\fO_k\right)$ is full and faithful.
\end{proposition}
\begin{proof}
Fix $k$-modules $V,W$. Then
$$\Hom_{\fO_k}\left(V_{\mathrm{a}},W_{\mathrm{a}}\right)\ni \sigma \mapsto \sigma^k\in \Hom_k\left(V,W\right)$$
and
$$\Hom_k\left(V,W\right) \ni \phi \mapsto \phi_{\mathrm{a}} \in \Hom_{\fO_k}\left(V_{\mathrm{a}},W_{\mathrm{a}}\right)$$
are mutually inverse bijections. Hence the functor is full and faithful.
\end{proof}

\begin{example}\label{example:general_linear_monoid}
Let $V$ be a $k$-module. We define a $k$-functor $\cL_V$. We set
$$\cL_V(A) = \Hom_A(A\otimes_kV,A\otimes_kV)$$
for every $k$-algebra $A$. Next for every morphism $f:A\ra B$ of $k$-algebras and every morphism $\phi:A\otimes_kV\ra A\otimes_kV$ of $A$-modules we define $\cL_V(f)(\phi)$ as a unique morphism of $B$-modules such that the diagram
\begin{center}
\begin{tikzpicture}
[description/.style={fill=white,inner sep=2pt}]
\matrix (m) [matrix of math nodes, row sep=3em, column sep=3em,text height=1.5ex, text depth=0.25ex] 
{  A\otimes_kV  & A\otimes_kV           \\
   B\otimes_kV  & B\otimes_kV           \\} ;
\path[->,line width=1.0pt,font=\scriptsize]  
(m-1-1) edge node[above] {$ \phi $} (m-1-2)
(m-2-1) edge node[below] {$\cL_V(\phi)  $} (m-2-2)
(m-1-1) edge node[left] {$ f\otimes_k1_V $} (m-2-1)
(m-1-2) edge node[right] {$ f\otimes_k1_V $} (m-2-2);
\end{tikzpicture}
\end{center}
is commutative. Note also that $\cL_V(A)$ is an $A$-algebra for every $k$-algebra $A$. Hence $\cL_V$ is a monoid $\fO_k$-algebra. Note that we have natural identification
$$\cL_V(A) = \Hom_k(V,A\otimes_kV)$$
for every $k$-algebra. One can describe $\fO_k$-algebra structure on $\cL_V$ in terms of this identification as follows. Since $\Hom_k(V,A\otimes_kV)$ carries canonical structure of $A$-module it suffices to describe the multiplication. For this suppose that $d_1,d_2\in \Hom_k(V,A\otimes_kV)$. Then their product is given by
$$\left(\mu_A\otimes_k1_V\right)\cdot \left(1_A\otimes d_2\right)\cdot d_1$$
where $\mu_A:A\otimes_kA\ra A$ is the multiplication on $A$.
\end{example}

\begin{remark}\label{remark:general_linear_monoid}
Let $V$ be a $k$-module. Proposition \ref{proposition:inclusion_of_k_modules_into_O_k_modules} implies that there are bijective maps that make the square
\begin{center}
\begin{tikzpicture}
[description/.style={fill=white,inner sep=2pt}]
\matrix (m) [matrix of math nodes, row sep=3em, column sep=3em,text height=1.5ex, text depth=0.25ex] 
{ \cL_V(A)   &  \cE nd_{\fO_A}\left(\left(V_{\mathrm{a}}\right)_A,\left(V_{\mathrm{a}}\right)_A\right)           \\
  \cL_V(B)   &  \cE nd_{\fO_B}\left(\left(V_{\mathrm{a}}\right)_B,\left(V_{\mathrm{a}}\right)_B\right)           \\} ;
\path[->,line width=1.0pt,font=\scriptsize]  
(m-1-1) edge node[auto] {$\cong $} (m-1-2)
(m-2-1) edge node[below] {$\cong $} (m-2-2);
\path[->,line width=1.0pt,font=\scriptsize]
(m-1-1) edge node[left] {$ \cL_V(f) $} (m-2-1)
(m-1-2) edge node[auto] {$ \sigma \mapsto \sigma_B $} (m-2-2);
\end{tikzpicture}
\end{center}
commutative for every morphism $f:A\ra B$ of $k$-algebras. This induces an idenitification $\cL_V = \cE nd_{\fO_k}\left(V_{\mathrm{a}}\right)$ of $\fO_k$-algebras.
\end{remark}

\begin{remark}\label{remark:representability_of_linear_monoid}
Suppose that $V$ is a finitely generated, projective $k$-module. Then for each $k$-algebra $A$ we have an isomorphism
$$\cL_V(A) = \Hom_A\left(V,A\otimes_kV\right) \cong A\otimes_kV^{\vee}\otimes_kV$$
Clearly this isomorphism is natural in $A$. Hence $\cL_V$ is isomorphic with $\left(V^{\vee}\otimes_kV\right)_{\mathrm{a}}$ and thus (Remark \ref{remark:representability_of_additive_functor}) it is representable by a $k$-scheme $\Spec \Sym(V\otimes_kV^{\vee})$.
\end{remark}

\begin{definition}
Let $\fG$ be a monoid $k$-functor. A pair $\left(V,\rho\right)$ consisting of a $k$-module $V$ and a morphism $\rho:\fG\ra \cL_V$ of $k$-monoids is called \textit{a linear representation of $\fG$}.
\end{definition}
\noindent
Next result characterizes linear representations of monoid $k$-functors.

\begin{corollary}\label{corollary:linear_representations_various_characterizations}
Let $\fG$ be a monoid $k$-functor and let $V$ be a $k$-module. Then the following classes are in canonical bijections.
\begin{enumerate}[label=\emph{\textbf{(\arabic*)}}, leftmargin=1.5em]
\item Linear actions of $\fG$ on $V_{\mathrm{a}}$.
\item $\fO_k$-linear actions $\fO_k[\fG]\times V_{\mathrm{a}}\ra V_{\mathrm{a}}$. These are precisely $\fO_k[\fG]$-modules.
\item Morphisms $\fO_k[\fG]\ra \cL_V$ of $\fO_k$-algebras.
\item Morphisms $\fG\ra \cL_V$ of monoids.
\end{enumerate}
\end{corollary}
\begin{proof}
This follows from Proposition \ref{proposition:decription_of_linear_monoid_actions_on_modules}.
\end{proof}

\begin{definition}
Let $\fG$ be a monoid $k$-functor and let $(V,\rho)$, $(W,\delta)$ be its linear representations. A morphism $\phi:V\ra W$ of $k$-modules such that
$$\phi_{\mathrm{a}}^A \cdot \rho(g) = \delta(g) \cdot \phi_{\mathrm{a}}^A$$
for every $k$-algebra $A$ and $g \in \fG(A)$ is called \textit{a morphism of linear representations of $\fG$}.
\end{definition}
\noindent
Next result characterizes morphisms of linear representations of monoid $k$-functor.

\begin{corollary}\label{corollary:characterization_of_morphisms_of_linear_representations}
Let $\fG$ be a monoid $k$-functor and let $V$, $W$ be $k$-modules. Suppose that $\alpha_1:\fG\times V_{\mathrm{a}} \ra V_{\mathrm{a}},\,\alpha_2:\fG\times W_{\mathrm{a}} \ra W_{\mathrm{a}}$ are linear actions of $\fG$. Suppose that $\phi:V\ra W$ is a morphism of $k$-modules. Then the following assertions are equivalent.
\begin{enumerate}[label=\emph{\textbf{(\roman*)}}, leftmargin=1.5em]
\item The square
\begin{center}
\begin{tikzpicture}
[description/.style={fill=white,inner sep=2pt}]
\matrix (m) [matrix of math nodes, row sep=3em, column sep=5em,text height=1.5ex, text depth=0.25ex] 
{  \fG\times V_{\mathrm{a}}   & \fG\times W_{\mathrm{a}} \\
   V_{\mathrm{a}}             & W_{\mathrm{a}} \\} ;
\path[->,line width=1.0pt,font=\scriptsize]  
(m-1-1) edge node[above] {$ 1_{\fG}\times \phi_{\mathrm{a}}  $} (m-1-2)
(m-2-1) edge node[below] {$ \phi_{\mathrm{a}} $} (m-2-2)
(m-1-1) edge node[left] {$ \alpha_1 $} (m-2-1)
(m-1-2) edge node[right] {$ \alpha_2 $} (m-2-2);
\end{tikzpicture}
\end{center}
is commutative.
\item The square
\begin{center}
\begin{tikzpicture}
[description/.style={fill=white,inner sep=2pt}]
\matrix (m) [matrix of math nodes, row sep=3em, column sep=5em,text height=1.5ex, text depth=0.25ex] 
{  \fO_k[\fG] \times V_{\mathrm{a}}   & \fO_k[\fG] \times W_{\mathrm{a}} \\
   V_{\mathrm{a}}                     & W_{\mathrm{a}} \\} ;
\path[->,line width=1.0pt,font=\scriptsize]  
(m-1-1) edge node[above] {$ 1_{\fO_k[\fG]}\times \phi_{\mathrm{a}}  $} (m-1-2)
(m-2-1) edge node[below] {$ \phi_{\mathrm{a}} $} (m-2-2)
(m-1-1) edge node[left] {$ \tilde{\alpha_1} $} (m-2-1)
(m-1-2) edge node[right] {$ \tilde{\alpha_2} $} (m-2-2);
\end{tikzpicture}
\end{center}
is commutative, where $\tilde{\alpha_1}$ and $\tilde{\alpha_2}$ are $\fO_k$-linear actions of $\fO_k[\fG]$ corresponding to $\alpha_1$ and $\alpha_2$, respectively. 
\item For every $k$-algebra $A$ and $g \in \fG(A)$ we have
$$\phi_{\mathrm{a}}^A \cdot \tilde{\rho}_1(g) = \tilde{\rho}_2(g) \cdot \phi_{\mathrm{a}}^A$$
where $\tilde{\rho}_1:\fO_k[\fG] \ra \cL_V$ and $\tilde{\rho}_2:\fO_k[\fG] \ra \cL_W$ are morphism of $\fO_k$-algebras corresponding to $\tilde{\alpha_1}$ and $\tilde{\alpha_2}$, respectively.
\item For every $k$-algebra $A$ and $g \in \fG(A)$ we have
$$\phi_{\mathrm{a}}^A \cdot \rho_1(g) = \rho_2(g) \cdot \phi_{\mathrm{a}}^A$$
where $\rho_1:\fG \ra \cL_V$ and $\rho_2:\fG \ra \cL_W$ are restrictions of $\tilde{\rho_1}$ and $\tilde{\rho_2}$, respectively. This states that $\phi$ is a morphism of linear representations of $\fG$.
\end{enumerate}
\end{corollary}
\begin{proof}
This follows from Proposition \ref{proposition:morphism_of_monoid_modules}.
\end{proof}
\noindent
Let $\fG$ be a monoid $k$-functor. We denote by $\bd{Rep}(\fG)$ its category of linear representations. Note that $\bd{Rep}(\fG)$ is a full subcategory of $\Mod(\fO_k\left[\fG\right])$.

\section{Constructions of linear representations}

\begin{example}[Outer tensor product of representations]\label{example:outer_tensor_product}
Let $(V_1,\rho_1)$ and $(V_2,\rho_2)$ are linear representations of monoid $k$-functors $\fG_1$ and $\fG_2$, respectively. Then we define a linear representation of $\fG_1\times \fG_2$ with $V_1\otimes_k V_2$ as the underlying $k$-module that corresponds to a morphism $\rho:\fG_1\times \fG_2 \ra \cL_{V_1\otimes_kV_2}$ of monoid $k$-functors given by
$$\rho\left(g_1,g_2\right) = \rho_1(g_1)\otimes_A\rho_2(g_2):A\otimes_kV_1\otimes_kV_2\ra A\otimes_kV_1\otimes_kV_2$$
for $(g_1,g_2)\in \fG_1(A)\times \fG_2(A)$, where $A$ is a $k$-algebra.
\end{example}

\begin{example}[Tensor product of representations]\label{example:tensor_product}
Let $(V_1,\rho_1)$ and $(V_2,\rho_2)$ are linear representations of monoid $k$-functor $\fG$. Then we define a linear representation of $\fG$ with $V_1\otimes_k V_2$ as the underlying $k$-module given as the composition of the outer tensor product of $(V_1,\rho_1)$ and $(V_2,\rho_2)$ with the diagonal $\fG\hookrightarrow \fG\times \fG$.
\end{example}

\begin{example}[Tensor operations]\label{example:tensor_operations}
Let $\fG$ be a monoid $k$-functor, let $V$ be $k$-module and let $\rho:\fG\ra \cL_V$ be a morphism of monoid $k$-functors. Then both $\bigwedge^nV$ and $\Sym^n(V)$ for $n\in \NN$ carry canonical structure of linear representation of $\fG$.
\end{example}
\noindent
Note that if $V$ is a finitely generated, projective $k$-module, then there is a canonical isomorphism of $A$-modules $\left(V^{\vee}\right)_{\mathrm{a}}(A) \cong (A\otimes_kV)^{\vee}$ natural in $k$-algebra $A$. Under these assumptions on $V$ there exists an anti-isomorphism of $A$-algebras
$$\Hom_A\left(A\otimes_kV,A\otimes_kV\right)\ni \phi \mapsto \phi^{\vee} \in \Hom_A\left((A\otimes_kV)^{\vee},(A\otimes_kV)^{\vee}\right)$$
natural in $k$-algebra $A$. This proves the following result.

\begin{fact}\label{fact:opposite_linear_monoid}
Let $V$ be a finitely generated, projective $k$-module. Then we have an identification of $k$-functors of $\fO_k$-algebras
$$\cL_V^{\mathrm{op}} = \cL_{V^{\vee}}$$
\end{fact}

\begin{example}[Dual representation]\label{example:dual_representation}
Let $\fG$ be a monoid $k$-functor, let $V$ be $k$-module and let $\rho:\fG\ra \cL_V$ be a morphism of monoid $k$-functors. Suppose that $V$ is a projective and finitely generated $k$-module. Fact \ref{fact:opposite_linear_monoid} implies that morphism of a monoid $k$-functors $\rho^{\mathrm{op}}:\fG^{\mathrm{op}}\ra \cL_V^{\mathrm{op}}$ can be identified with $\rho^{\vee}:\fG^{\mathrm{op}}\ra \cL_{V^{\vee}}$. Hence a pair $(V^{\vee},\rho^{\vee})$ is a linear representation of $\fG^{\mathrm{op}}$.
\end{example}
\noindent

\begin{example}[Hom representation]\label{example:hom_representation}
Let $(V_1,\rho_1)$ and $(V_2,\rho_2)$ are linear representations of monoid $k$-functor $\fG$. Suppose that $V_1,V_2$ are finitely generated, projective $k$-module. Then we have an identification
$$\Hom_k\left(V_1,V_2\right)_{\mathrm{a}} = \left(V_1^{\vee}\otimes_kV_2\right)_{\mathrm{a}}$$
of $\fO_k$-modules. By Examples \ref{example:outer_tensor_product} and \ref{example:dual_representation} this isomorphism makes $\Hom_k(V_1,V_2)$ into linear representation of $\fG\times \fG^{\mathrm{op}}$.
\end{example}

\section{Example of $\fG$-action: Regular functions $k$-functor}
\noindent
First we need the following notion.

\begin{definition}
Let $\left(-\right)^{\mathrm{op}}:\Mon \ra \Mon$ be the opposite monoid functor and let $\fG$ be a monoid $k$-functor. Then the composition $\fG^{\mathrm{op}} = \left(-\right)^{\mathrm{op}}\cdot \fG$ is called \textit{the opposite monoid $k$-functor of $\fG$}.
\end{definition}
\noindent
Let $\fG$ be a monoid $k$-functor. In this section we discuss important example of a $\fO_k[\fG]$-module. Fix a $k$-functor $\fX$ for which $\iMor_k(\fX,\fO_k)$ exists. Recall that by Example \ref{example:regular_functions_as_an_algebra_over_structure_algebra} $\iMor_k\left(\fX,\fO_k\right)$ is $\fO_k$-algebra $k$-functor. Let $\alpha:\fG\times \fX\ra \fX$ be an action of $\fG$ on $\fX$. For every $k$-algebra $A$ we have a map of sets
$$\Mor_A\left(\fX_A,(\fO_k)_A\right) \ni f\mapsto f\cdot \alpha_{g}\in \Mor_A\left(\fX_A,(\fO_k)_A\right)$$
where $g\in \fG(A)$. From this description it follows that the map $f\mapsto f\cdot \alpha_{g}$ is a morphism of $A$-algebras. Moreover, note that if $g_1,g_2\in \fG(A)$, then $\left(f\cdot \alpha_{g_1}\right)\cdot \alpha_{g_2} = f\cdot \alpha_{g_1\cdot g_2}$, where $g_1\cdot g_2 \in \fG(A)$ is a product of $g_1$ and $g_2$. Thus the opposite monoid $\fG^{\mathrm{op}}(A)$ acts on the $A$-algebra $\Mor_A\left(\fX_A,\left(\fO_k\right)_A\right)$ by morphism of $A$-algebras. Next for every $A$-algebra $B$ and every point $x\in \fX(B)$ we have
$$(f\cdot \alpha_{g})(x) = f\left(\alpha_{g}(x)\right)$$
where $g\in \fG(A)$. This proves the following result.

\begin{proposition}\label{proposition:action_on_regular_k_functor}
Let $\fX$ be a $k$-functor and let $\alpha:\fG\times \fX \ra \fX$ be an action of a monoid $k$-functor $\fG$. Suppose that $\iMor_k\left(\fX,\fO_k\right)$ exists. Then $\fG^{\mathrm{op}}$ acts canonically on $\fO_k$-algebra $k$-functor $\iMor_k\left(\fX,\fO_k\right)$ by morphisms of $\fO_k$-algebras.
\end{proposition}
\noindent
Let us note one important consequence of this result.

\begin{corollary}\label{corollary:action_on_regular_k_functor}
Let $\fG$ be a monoid $k$-functor. The action of $\fG\times \fG^{\mathrm{op}}$ on $\fG$ induces the action of $\fG^{\mathrm{op}}\times \fG$ on $\fO_k$-algebra $k$-functor $\iMor_k\left(\fX,\fO_k\right)$ by morphisms of $\fO_k$-algebras.
\end{corollary}

\section{Matrix coefficients of a representation}

\begin{proposition}\label{proposition:matrix_coefficients}
Let $\fG$ be a monoid $k$-functor and let $V$ be a finitely generated, projective $k$-module. Fix a morphism $\rho:\fG \ra \cL_V$ of monoid $k$-functors. Fix $k$-algebra $A$ and elements $v\in A\otimes_kV$, $w\in A\otimes_kV^{\vee}$. For every $A$-algebra $B$ and $g\in \fG(B)$ we consider the formula
$$c_{v,w}(g) = \langle \rho_A(g) \cdot v_B, w_B \rangle$$
Then $c_{v,w}$ defines a regular function on $\fG_A$ for every $k$-algebra $A$.
\end{proposition}
\begin{proof}
Suppose that $f:B\ra C$ is a morphism of $A$-algebras and pick $g\in \fG(B)$. Since $\rho_A$ is natural and $w:A\otimes_kV\ra A$ is a morphism of $A$-modules, we derive that the diagram
\begin{center}
\begin{tikzpicture}
[description/.style={fill=white,inner sep=2pt}]
\matrix (m) [matrix of math nodes, row sep=3em, column sep=5em,text height=1.5ex, text depth=0.25ex] 
{B\otimes_kV & B\otimes_kV & B \\
 C\otimes_kV & C\otimes_kV & C\\} ;
\path[->,line width=1.0pt,font=\scriptsize]  
(m-2-1) edge node[below] {$\rho_A\big(\fG_A(f)(g)\big) $} (m-2-2)
(m-1-2) edge node[right] {$f\otimes_A1_{A\otimes_kV} $} (m-2-2) 
(m-1-1) edge node[left]  {$f\otimes_A1_{A\otimes_kV} $} (m-2-1)
(m-1-1) edge node[above] {$\rho_A(g) $} (m-1-2)
(m-1-2) edge node[above] {$w_B $} (m-1-3)
(m-2-2) edge node[below] {$w_C $} (m-2-3)
(m-1-3) edge node[right] {$f $} (m-2-3);
\end{tikzpicture}
\end{center}
is commutative. Hence 
$$c_{v,w}\big(\fG_A(f)(g)\big)=\langle \rho_A\big(\fG_A(f)(g)\big)\cdot v_C,w_C\rangle=f\big(\langle \rho_A(g)\cdot v_B, w_B\rangle \big)=f\big(c_{v,w}(g)\big)$$
and this implies that $c_{v,w}:\fG_A\ra \fO_A$ is a morphism of $A$-functors.
\end{proof}

\begin{definition}
Let $\fG$ be a monoid $k$-functor and let $(V,\rho)$ be its representation with finitely generated, projective underlying $k$-module $V$. Fix $k$-algebra $A$ and elements $v\in A\otimes_kV$, $w\in A\otimes_kV^{\vee}$. Then the regular function $c_{v,w}$ on $\fG_A$ is called \textit{the matrix coefficient of $v$ and $w$}.
\end{definition}

\begin{proposition}\label{proposition:matrix_coefficients_natural}
Let $\fG$ be a monoid $k$-functor and let $(V,\rho)$ be its representation with finitely generated projective underlying $k$-module $V$. Then the following assertions holds.
\begin{enumerate}[label=\emph{\textbf{(\arabic*)}}, leftmargin=3.0em]
\item For every $k$-algebra $A$ map
$$\left(A\otimes_kV\right)\times \left(A\otimes_kV^{\vee}\right)\ni (v,w)\mapsto c_{v,w}\in \Mor_A\left(\fG_A,\fO_A\right)$$
is $A$-bilinear.
\item Suppose that $\iMor_k\left(\fG,\fO_k\right)$ exists. Then the collection of maps
$$\bigg\{\big(A\otimes_kV\big)\times \big(A\otimes_kV^{\vee}\big)\ni (v,w)\mapsto c_{v,w}\in \Mor_A\big(\fG_A,\fO_A\big)\bigg\}_{A\in \Alg_k}$$
gives rise to a morphism of $k$-functors
\begin{center}
\begin{tikzpicture}
[description/.style={fill=white,inner sep=2pt}]
\matrix (m) [matrix of math nodes, row sep=3em, column sep=3em,text height=1.5ex, text depth=0.25ex] 
{ V_{\mathrm{a}}\times V^{\vee}_{\mathrm{a}} &  \iMor_k\left(\fG,\fO_k\right) \\} ;
\path[->,line width=1.0pt,font=\scriptsize]  
(m-1-1) edge node[above] {$ $} (m-1-2);
\end{tikzpicture}
\end{center}
\end{enumerate}
\end{proposition}
\begin{proof}
We left the proof of \textbf{(1)} to the reader.\\
We prove \textbf{(2)}. Consider $k$-algebra $A$ and an $A$-algebra $B$ with structural morphism $f:A\ra B$. Fix $v\in A\otimes_kV$, $w\in A\otimes_kV^{\vee}$. We prove that restriction of $c_{v,w}:\fG_A\ra \fO_A$ to the category $\Alg_B$ is $c_{v_B,w_B}$. For this pick a $B$-algebra $C$ and an element $g\in \fG(C)$. Note that
$$c_{v,w}(g)= \langle \rho_A(g)\cdot v_C,w_C \rangle =  \langle \rho_B(g)\cdot v_C,w_C\rangle = \langle \rho_B(g)\cdot (v_B)_C,(w_B)_C\rangle = c_{v_B,w_B}(g)$$
and hence ${c_{v,w}}_{\mid \Alg_B} = c_{v_B,w_B}$. Consider the square
\begin{center}
\begin{tikzpicture}
[description/.style={fill=white,inner sep=2pt}]
\matrix (m) [matrix of math nodes, row sep=4em, column sep=3em,text height=1.5ex, text depth=0.25ex] 
{V_{\mathrm{a}}(A)\times V^{\vee}_{\mathrm{a}}(A) & \iMor_k\left(\fG,\fO_A\right)(A)  \\
 V_{\mathrm{a}}(B)\times V^{\vee}_{\mathrm{a}}(B) & \iMor_k\left(\fG,\fO_B\right)(B)  \\} ;
\path[->,line width=1.0pt,font=\scriptsize]  
(m-2-1) edge node[below] {$ $} (m-2-2)
(m-1-2) edge node[right] {$\iMor_k(\fG,\fO_k)(f) $} (m-2-2) 
(m-1-1) edge node[left]  {$V_a(f)\times V^{\vee}_a(f)$} (m-2-1)
(m-1-1) edge node[above] {$ $} (m-1-2);
\end{tikzpicture}
\end{center}
in which both horizontal arrows are given by formula $(v,w)\mapsto c_{v,w}$. We proved that the square commutes. Since $f$ is an arbitrary morphism of $k$-algebras, we conclude the assertion.
\end{proof}

\begin{corollary}\label{corollary:matrix_coefficients_natural}
Let $\fG$ be a monoid $k$-functor and let $(V,\rho)$ be its representation with finitely generated projective underlying $k$-module $V$. Suppose that $\iMor_k\left(\fG,\fO_k\right)$ exists. Then there exists a morphism of $k$-functors
\begin{center}
\begin{tikzpicture}
[description/.style={fill=white,inner sep=2pt}]
\matrix (m) [matrix of math nodes, row sep=3em, column sep=3em,text height=1.5ex, text depth=0.25ex] 
{ \left(V \otimes_k V^{\vee}\right)_{\mathrm{a}} &  \iMor_k\left(\fG,\fO_k\right) \\} ;
\path[->,line width=1.0pt,font=\scriptsize]  
(m-1-1) edge node[above] {$ c $} (m-1-2);
\end{tikzpicture}
\end{center}
given by formula
$$\left(A\otimes_kV\right)\otimes_A\left(A\otimes_kV^{\vee}\right)\ni (v,w)\mapsto c_{v,w}\in \Mor_A\left(\fG_A,\fO_A\right)$$
Moreover, $c$ is a morphism of $k$-functors equipped with $\fG \times \fG^{\mathrm{op}}$-actions.
\end{corollary}
\begin{proof}
The first part is an immediate consequence of Proposition \ref{proposition:matrix_coefficients_natural}. We prove that $c$ is a morphism of $k$-functors equipped with $\fG\times \fG^{\mathrm{op}}$-actions. For this we fix a $k$-algebra $k$ and elements $v\in A\otimes_kV$, $w\in A\otimes_kV^{\vee}$. Pick a morphism of $k$-algebras $f:A\ra B$, $(g_1,g_2)\in \fG(A)\times \fG(A)^{\mathrm{op}}$ and $g\in \fG(B)$. Then we have 
$$c_{\rho(g_1)\cdot v,w\cdot \rho(g_2)}(g) = \big\langle \rho_A(g)\cdot \left(\rho(g_1)\cdot v\right)_B, \left(w\cdot \rho(g_2)\right)_B \big\rangle =$$
$$= \big\langle \rho_A(g)\cdot \rho_A(\left(\fG_A(f)(g_1)\right))\cdot v_B, w_B\cdot \rho_A\left(\fG_A(f)(g_2)\right) \big\rangle = w_B\big(\rho_A\left(\fG_A(f)(g_2)\right)\cdot \rho_A(g)\cdot \rho_A\left(\fG_A(f)(g_1)\right)\cdot v_B \big)=$$
$$= w_B\big(\rho_A\left(\fG_A(f)(g_2) \cdot g \cdot \fG_A(f)(g_1)\right)\cdot v_B \big) = \big\langle \rho_A\left(\fG_A(f)(g_2) \cdot g \cdot \fG_A(f)(g_1)\right)\cdot v_B, w_B \big\rangle =  $$
$$= c_{v,w}\big(\fG_A(f)(g_2) \cdot g \cdot \fG_A(f)(g_1)\big)$$
and hence $c$ is a morphism of $k$-functors equipped with actions of $\fG\times \fG^{\mathrm{op}}$.
\end{proof}

\section{Monoid $k$-schemes}

\begin{definition}
\textit{A monoid $k$-scheme $\bd{M}$} is a monoid object in the category of $k$-schemes. If $\bd{M}$ is affine, then we say that $\bd{M}$ is \textit{an affine monoid $k$-scheme}.
\end{definition}

\begin{definition}
\textit{A group $k$-scheme $\bd{G}$} is a group object in the category of $k$-schemes. If $\bd{G}$ is affine, then we say that $\bd{G}$ is \textit{an affine group $k$-scheme}.
\end{definition}

\begin{corollary}\label{corollary:monoid_k_schemes_and_their_functors_of_points}
The functor
\begin{center}
\begin{tikzpicture}
[description/.style={fill=white,inner sep=2pt}]
\matrix (m) [matrix of math nodes, row sep=3em, column sep=3em,text height=1.5ex, text depth=0.25ex] 
{ \Sch_k  & \mbox{\emph{the category of $k$-functors}} \\};
\path[->,line width=1.0pt,font=\scriptsize]  
(m-1-1) edge node[auto] {$ \fP $} (m-1-2);
\end{tikzpicture}
\end{center}
induces an equivalence of categories
$$\mbox{\emph{the category of monoid $k$-schemes}}\cong \mbox{\emph{monoid $k$-functors representable by $k$-schemes}}$$
Similarly for categories of groups.
\end{corollary}
\begin{proof}
Follows from {\cite[Fact 4.1]{kfunctors}}.
\end{proof}
\noindent
Recall that by Example \ref{example:units_k_functor_for_a_monoid} each monoid $k$-functor $\fG$ has its group $k$-functor $\fG^*$ of units.

\begin{proposition}
Let $\bd{M}$ be an affine monoid $k$-scheme. Then the $k$-functor of units $\fP_{\bd{M}}^*$ is representable (by an affine $k$-scheme).
\end{proposition}
\begin{proof}
Note that $\fP_{\bd{M}}^*$ fits into a cartesian square
\begin{center}
\begin{tikzpicture}
[description/.style={fill=white,inner sep=2pt}]
\matrix (m) [matrix of math nodes, row sep=3em, column sep=2em,text height=1.5ex, text depth=0.25ex] 
{\fP_{\bd{M}}^*                  &   \bd{1}      \\
 \fP_{\bd{M}}\times \fP_{\bd{M}} & \fP_{\bd{M}}  \\} ;
\path[->,line width=1.0pt,font=\scriptsize]  
(m-1-1) edge node[above] {$ $} (m-1-2)
(m-2-1) edge node[below] {$\fP_{m} $} (m-2-2)
(m-1-1) edge node[above] {$ $} (m-2-1)
(m-1-2) edge node[right] {$\fP_e $} (m-2-2);
\end{tikzpicture}
\end{center}
where $m:\bd{M}\times \bd{M}\ra \bd{M}$ is the multiplication and $e:\Spec k\ra \bd{M}$ is the unit. By {\cite[Fact 4.1]{kfunctors}} the functor $\fP$ preserves finite products and hence it preserves fiber-products. This implies that $\fP_{\bd{M}}^*$ is represented by a unique (up to an isomorphism) $k$-scheme $\bd{M}^*$ that fit into a cartesian square below.
\begin{center}
\begin{tikzpicture}
[description/.style={fill=white,inner sep=2pt}]
\matrix (m) [matrix of math nodes, row sep=3em, column sep=2em,text height=1.5ex, text depth=0.25ex] 
{ \bd{M}^*             &      \Spec k      \\
  \bd{M}\times \bd{M} &       \bd{M}  \\} ;
\path[->,line width=1.0pt,font=\scriptsize]  
(m-1-1) edge node[above] {$ $} (m-1-2)
(m-2-1) edge node[below] {$ m $} (m-2-2)
(m-1-1) edge node[above] {$ $} (m-2-1)
(m-1-2) edge node[right] {$ e $} (m-2-2);
\end{tikzpicture}
\end{center}
Note that if $\bd{M}$ is affine, then also $\bd{M}^*$ is affine.
\end{proof}

\begin{definition}
Let $\bd{M}$ be a monoid $k$-scheme. Then the group $k$-scheme $\bd{M}^*$ representing $\fP_{\bd{M}}^*$ is called \textit{the group of units of $\bd{M}$}.
\end{definition}

\begin{remark}\label{remark:convention_on_relation_between_notions_for_monoid_schemes_and_monoid_functors}
Under the embedding given in Corollary \ref{corollary:monoid_k_schemes_and_their_functors_of_points} notions defined for monoid $k$-functors can be translated to monoid $k$-schemes.
\end{remark}
\noindent
We give two instances of the use of Remark \ref{remark:convention_on_relation_between_notions_for_monoid_schemes_and_monoid_functors} below.

\begin{definition}
Let $\bd{M}$ be a monoid $k$-scheme. Then \textit{the category of linear representations of $\bd{M}$} is the category of linear representations of the monoid $k$-functor $\fP_{\bd{M}}$. We denote this category by $\bd{Rep}\left(\bd{M}\right)$.
\end{definition}

\begin{definition}
Let $\bd{M}$ be a monoid $k$-functor and let $\alpha:\fP_{\bd{M}}\times \fX\ra \fX$ be an action of $\fP_{\bd{M}}$ on a $k$-functor $\fX$. Then we say that $\alpha$ is \textit{an action of $\bd{M}$ on $\fX$}.
\end{definition}

\section{Bialgebras and affine monoid $k$-schemes}
\noindent
We start here with a general notion of $k$-coalgebras.

\begin{definition}
Let $(C,\Delta,\xi)$ be a triple consisting of a module $C$ over $k$ and morphisms
$$\Delta:C \ra C\otimes_{k} C,\xi:C\ra k$$
of $k$-modules such that the following diagrams are commutative.
\begin{center}
\begin{tikzpicture}
[description/.style={fill=white,inner sep=2pt}]
\matrix (m) [matrix of math nodes, row sep=3em, column sep=2em,text height=1.5ex, text depth=0.25ex] 
{C&  &    C\otimes_{k}C      &                       C& & C\otimes_{k}C &     C    & &C\otimes_{k}C \\
C\otimes_{k}C&   &   C\otimes_{k}C\otimes_{k}C &  & &C\otimes_{k}k&          && k\otimes_{k}C  \\} ;
\path[->,line width=1.0pt,font=\scriptsize]
(m-1-1) edge node[above] {$\Delta $} (m-1-3)
(m-2-1) edge node[below] {$\Delta\otimes_{k}1_C $} (m-2-3)
(m-1-3) edge node[right] {$ 1_{C}\otimes_{k}\Delta $} (m-2-3)  
(m-1-1) edge node[left] {$ \Delta $} (m-2-1)
(m-1-4) edge node[above] {$\Delta $} (m-1-6)
(m-1-6) edge node[right] {$1\otimes_{k}\xi $} (m-2-6)
(m-1-4) edge node[below = 3pt] {$\cong $} (m-2-6)
(m-1-7) edge node[above] {$\Delta $} (m-1-9)
(m-1-9) edge node[right] {$\xi\otimes_{k}1_C $} (m-2-9)
(m-1-7) edge node[below = 3pt] {$\cong $} (m-2-9);
\end{tikzpicture}
\end{center}
Then $(C,\Delta,\xi)$ is called \textit{a $k$-coalgebra}. Morphisms $\Delta$, $\xi$ are called \textit{a comultiplication} and \textit{a counit}, respectively.
\end{definition}

\begin{definition}
Let $\left(C_1,\Delta_1,\xi_1\right)$ and $\left(C_2,\Delta_2,\xi_2\right)$ are $k$-coalgebras. Then a morphism $f:C_1\ra C_2$ of $k$-modules is \textit{a morphism of $k$-coalgebras} if the following diagrams are commutative.
\begin{center}
\begin{tikzpicture}
[description/.style={fill=white,inner sep=2pt}]
\matrix (m) [matrix of math nodes, row sep=3em, column sep=2em,text height=1.5ex, text depth=0.25ex] 
{ C_1\otimes_{k}C_1 &   &   C_2\otimes_{k}C_2                            &    C_1 &   &  C_2 \\
C_1             &   &       C_2                    &      & k &    \\} ;
\path[->,line width=1.0pt,font=\scriptsize]
(m-1-1) edge node[above] {$ f\otimes_kf$} (m-1-3)
(m-2-3) edge node[right] {$\Delta_2 $} (m-1-3)
(m-2-1) edge node[left] {$\Delta_1$} (m-1-1)
(m-2-1) edge node[below] {$ f$} (m-2-3)
(m-1-4) edge node[above] {$ f$} (m-1-6)
(m-1-4) edge node[left] {$\xi_1 $} (m-2-5)
(m-1-6) edge node[right] {$\xi_2 $} (m-2-5);
\end{tikzpicture}
\end{center} 
\end{definition}
\noindent
By $k$-algebra we mean commutative and unital $k$-algebra.

\begin{definition}
Let $B$ be a $k$-module with structures of both $k$-algebra and $k$-coalgebra. Assume that the comultiplication and the counit of $B$ are morphisms of $k$-algebras with respect to $k$-algebra structure of $B$. Then we say that $B$ with these structures is \textit{a $k$-bialgebra}. 
\end{definition}

\begin{definition}
Let $B_1,B_2$ be $k$-bialgebras and let $f:B_1\ra B_2$ be a morphism of $k$-modules. We say that $f$ is \textit{a morphism of $k$-bialgebras} if it is simultaneously morphism of $k$-algebras and $k$-coalgebras. 
\end{definition}

\begin{theorem}\label{theorem:bialgebras_and_monoid_k_functors}
The functor $\Spec:\Alg_k\ra \Sch_k$ induces an equivalence of categories
$$\mbox{\emph{$k$-bialgebras}} \cong \mbox{\emph{the category of affine monoid $k$-schemes}}$$
\end{theorem}
\begin{proof}
This is an exercise in translation. For details see {\cite[II, 1.6]{demazure1970groupes}}.
\end{proof}
\noindent
Let $\bd{M}$ be an affine monoid $k$-scheme. Then we denote by $k[\bd{M}]$ its coordinate $k$-bialgebra, by $\Delta_{\bd{M}}$ its comultiplication and by $\xi_{\bd{M}}$ its counit. This is a notation that we consistently use in these notes.

\section{Comodules over $k$-coalgebras}

\begin{definition}
Let $C$ be a $k$-coalgebra with the comultiplication $\Delta$ and the counit $\xi$. A pair $(V,d)$ consisting of a $k$-module $V$ and a morphism $d:V\ra C\otimes_{k}V$ of $k$-modules such that the following diagrams are commutative
\begin{center}
\begin{tikzpicture}
[description/.style={fill=white,inner sep=2pt}]
\matrix (m) [matrix of math nodes, row sep=3em, column sep=2em,text height=1.5ex, text depth=0.25ex] 
{V&  &    C\otimes_{k}V                            &   V&  &    C\otimes_{k}V \\
C\otimes_{k}V&   &   C\otimes_{k}C\otimes_{k}V &  &   &   k\otimes_{k}V  \\} ;
\path[->,line width=1.0pt,font=\scriptsize]
(m-1-1) edge node[above] {$ d $} (m-1-3)
(m-2-1) edge node[below] {$ \Delta\otimes_{k}1_V $} (m-2-3)
(m-1-3) edge node[right] {$ 1_C\otimes_{k}d $} (m-2-3)  
(m-1-1) edge node[left] {$ d $} (m-2-1)
(m-1-4) edge node[above] {$d $} (m-1-6)
(m-1-6) edge node[right] {$\xi\otimes_{k} 1_V $} (m-2-6)
(m-1-4) edge node[below] {$ \cong$} (m-2-6);
\end{tikzpicture}
\end{center} 
is called \textit{a $C$-comodule}. Morphism $d$ is called \textit{a coaction of $C$ on $V$}.
\end{definition}

\begin{definition}
Let $C$ be a $k$-coalgebra and let $(V_1,d_1),(V_2,d_2)$ be two comodules over $C$. A morphism of $k$-modules $f:V_1\ra V_2$ is \textit{a morphism of $C$-comodules} if the diagram
\begin{center}
\begin{tikzpicture}
[description/.style={fill=white,inner sep=2pt}]
\matrix (m) [matrix of math nodes, row sep=3em, column sep=3em,text height=1.5ex, text depth=0.25ex] 
{C\otimes_{k}V_1 &    C\otimes_{k}V_2     \\
              V_1&         V_2  \\} ;
\path[->,line width=1.0pt,font=\scriptsize]
(m-1-1) edge node[above] {$ 1_{C}\otimes_{k}f$} (m-1-2)
(m-2-2) edge node[right] {$d_2 $} (m-1-2)
(m-2-1) edge node[left] {$d_1 $} (m-1-1)
(m-2-1) edge node[below] {$ f$} (m-2-2);
\end{tikzpicture}
\end{center} 
is commutative. 
\end{definition}
\noindent
We denote by $\bd{coMod}(C)$ the category of $C$-comodules for a $k$-coalgebra $C$.

\begin{theorem}\label{theorem:forgetful_coalgebras_to_modules_create_colimits}
Let $C$ be a $k$-coalgebra. Then the forgetful functor $\bd{coMod}(C)\ra \Mod(k)$ creates colimits. 
\end{theorem}
\begin{proof}
Let $\Delta,\xi$ be the comultiplication and the counit of $C$, respectively. Suppose that $I\ni i\mapsto (V_i,d_i)\in \bd{coMod}(C)$ is a diagram of $C$-comodules indexed by some category $I$. Let $V$ together with $u_i:V_i\ra V$ for $i\in I$ be a colimit of the diagram $I\ni i\mapsto V_i\in \Mod(k)$. By the universal property of colimit we deduce that there exists a unique morphism $d:V\ra C\otimes_{k}V$ such that diagrams 
\begin{center}
\begin{tikzpicture}
[description/.style={fill=white,inner sep=2pt}]
\matrix (m) [matrix of math nodes, row sep=3em, column sep=3em,text height=1.5ex, text depth=0.25ex] 
{C\otimes_{k}V_i &    C\otimes_{k}V     \\
             V_i &         V            \\} ;
\path[->,line width=1.0pt,font=\scriptsize]
(m-1-1) edge node[above] {$ 1_{C}\otimes_{k}u_i $} (m-1-2)
(m-2-2) edge node[right] {$ d $} (m-1-2)
(m-2-1) edge node[left] {$ d_i $} (m-1-1)
(m-2-1) edge node[below] {$ u_i$} (m-2-2);
\end{tikzpicture}
\end{center} 
are commutative for every $i\in I$. In order to verify that diagrams
\begin{center}
\begin{tikzpicture}
[description/.style={fill=white,inner sep=2pt}]
\matrix (m) [matrix of math nodes, row sep=3em, column sep=2em,text height=1.5ex, text depth=0.25ex] 
{V&  &    C\otimes_{k}V                            &   V&  &    C\otimes_{k}V \\
C\otimes_{k}V&   &   C\otimes_{k}C\otimes_{k}V &  &   &   k\otimes_{k}V  \\} ;
\path[->,line width=1.0pt,font=\scriptsize]
(m-1-1) edge node[above] {$ d $} (m-1-3)
(m-2-1) edge node[below] {$ \Delta \otimes_{k}1_V $} (m-2-3)
(m-1-3) edge node[right] {$ 1_C\otimes_{k}d $} (m-2-3)  
(m-1-1) edge node[left] {$ d $} (m-2-1)
(m-1-4) edge node[above] {$d $} (m-1-6)
(m-1-6) edge node[right] {$\xi \otimes_{k} 1_V $} (m-2-6)
(m-1-4) edge node[below] {$ \cong$} (m-2-6);
\end{tikzpicture}
\end{center}
are commutative it suffices to note that for every $i\in I$ we have chains of equalities
$$\left(1_C\otimes_{k}d\right)\cdot d\cdot u_i=\left(1_C\otimes_k 1_C\otimes_{k}u_i\right)\cdot \left(1_C\otimes_k 1_C \otimes_{k}d_i\right)\cdot d_i=\left(1_C\otimes 1_C \otimes_{k}u_i\right) \cdot \left(\Delta\otimes_{k}1_{V_i}\right)\cdot d_i=\left(\Delta \otimes_{k}1_V\right)\cdot d\cdot u_i$$
and
$$\left(\xi \otimes_{k}1_V\right)\cdot d\cdot u_i=\left(1_{k}\otimes_{k}u_i\right)\cdot \left(\xi \otimes_{k}1_{V_i}\right)\cdot d_i=\left(1_{k}\otimes_{k}u_i\right)\cdot j_{V_i}=j_V\cdot u_i$$
where $j_W:W\ra k\otimes_{k}W$ is the natural isomorphism for every $k$-module $W$. Hence $(V,d)$ is a $C$-comodule. Suppose now that $(W,e)$ is a $C$-comodule and $w_i:V_i\ra W$ for $i\in I$ is a family of $C$-comodule morphisms compatible with the diagram $I\ni i\mapsto (V_i,d_i)\in \bd{coMod}(C)$. Since $\{u_i:V_i\ra V\}_{i\in I}$ form a colimiting cocone for $I\ni i\mapsto V_i\in \Mod(k)$, there exists a unique morphism of $k$-modules $f:V\ra W$ such that $f\cdot u_i=w_i$. Note that
$$e\cdot f \cdot u_i= e\cdot w_i=\left(1_C\otimes_{k}w_i\right)\cdot d_i=\left(1_C\otimes_{k}f\right)\cdot \left(1_C\otimes_{k}u_i\right)\cdot d_i=\left(1_C\otimes_{k}f\right)\cdot d\cdot u_i$$
for every $i\in I$. Hence $e\cdot f=\left(1_C\otimes_{k}f\right)\cdot d$. Thus $f$ is a morphism of $C$-comodules. Thus $(V,d)$ together with family $\big\{u_i:(V_i,d_i)\ra (V,d)\big\}_{i\in I}$ is a colimit of the diagram $I\ni i\mapsto (V_i,d_i)\in \bd{coMod}(C)$ of $C$-comodules. This implies that the forgetful functor $\bd{coMod}(C)\ra \Mod(k)$ creates colimits.
\end{proof}

\begin{theorem}\label{theorem:limits_creation_forgetful_from_coalgebras_to_modules}
Let $C$ be a $k$-coalgebra such that $C$ is a flat $k$-module. Then the forgetful functor $\bd{coMod}(C)\ra \Mod(k)$ creates finite limits.
\end{theorem}
\begin{proof}
The proof is similar to the proof of Theorem \ref{theorem:forgetful_coalgebras_to_modules_create_colimits}.
\end{proof}

\begin{corollary}\label{corollary:comodules_category_properties}
Let $C$ be a coalgebra over $k$ and assume that $C$ is flat as a $k$-module. Then $\bd{coMod}(C)$ is an abelian category with small colimits.
\end{corollary}
\begin{proof}
This follows from Theorems \ref{theorem:forgetful_coalgebras_to_modules_create_colimits} and \ref{theorem:limits_creation_forgetful_from_coalgebras_to_modules}.
\end{proof}
\noindent
The next result is of fundamental importance.

\begin{theorem}\label{theorem:rationality_of_comodules}
Let $C$ be a $k$-coalgebra that is free as a $k$-module. Suppose that $V$ is a $C$-comodule over $C$. Then for every finitely generated $k$-submodule $U \subseteq V$ there exists a $C$-subcomodule $W$ of $V$ such that $U \subseteq W$ and $W$ is a finitely generated $k$-module.
\end{theorem}
\noindent
The theorem follows from the following simple lemma.

\begin{lemma}\label{lemma:for_single_element}
Let $C$ be a $k$-coalgebra over $k$ that is free as a $k$-module. Suppose that $V$ is a $C$-comodule over $C$ and fix an element $v\in V$. Then there exists a $C$-subcomodule $W$ of $V$ such that $v\in W$ and $W$ is a finitely generated $k$-module.
\end{lemma}
\begin{proof}[Proof of the lemma]
Let $\{e_j\}_{j\in J}$ be a free basis of $C$ over $k$ and let $d:V\ra C\otimes_{k}V$ be a left coaction of $C$ on $V$. Denote by $\Delta:C\ra C\otimes_{k}C$ the comultiplication of $C$. Then we have
$$d(v)=\sum_{j\in J}e_j\otimes v_{j}$$
where $v_j\in V$ are zero for almost all $j\in J$. Next according to
$$\left(\Delta\otimes_{k}1_V\right)\cdot d=\left(1_C\otimes_{k}d\right)\cdot d$$
we derive that equality
$$\sum_{j\in J}e_j\otimes d(v_{j})=(1_C\otimes_{k}d)\big(d(v)\big)=(\Delta\otimes_{k}1_V)\big(d(v)\big)=\sum_{j\in J}\Delta(e_j)\otimes v_{j}\subseteq \sum_{j\in J}C\otimes_{k}C\otimes_{k}k\cdot v_j$$
holds. This implies that $d(v_j)\subseteq C\otimes_{k}\left(\sum_{j\in J}k\cdot v_j\right)$. Hence $k$-submodule $W$ of $V$ generated by $v$ and $\{v_j\}_{j\in J}$ is $C$-subcomodule of $V$. It is finitely generated as a $k$-module and $v\in W$.
\end{proof}

\begin{proof}[Proof of the theorem]
Suppose that $U$ is generated by $\{v_1,...,v_n\}$ as a $k$-module. For each $i$ pick $C$-subcomodule $W_i$ of $V$ such that $W_i$ is finitely generated as a $k$-module and $v_i\in W_i$. This can be done by Lemma \ref{lemma:for_single_element}. Next
$$W = W_1+...+W_n$$
is a $C$-subcomodule of $V$ that is finitely generated as a $k$-module and contains $U$.
\end{proof}

\section{Linear representations and comodules}
\noindent
Let $\bd{M}$ be an affine monoid $k$-scheme and let $\rho:\fP_{\bd{M}}\ra \cL_V$ be a morphism of functors of sets, where $V$ is a $k$-module. Yoneda Lemma implies that $\rho$ is determined by some element (Example \ref{example:general_linear_monoid}) 
$$d_{\rho}\in \Hom_{k}\left(V,k[\bd{M}]\otimes_{k}V\right)$$

\begin{theorem}\label{theorem:comodules_equivalent_to_representations}
Let $\bd{M}$ be an affine monoid $k$-scheme. Then the correspondence
$$\bd{Rep}(\bd{M}) \ni \left(V,\rho\right) \mapsto \left(V,d_{\rho}\right)\in \bd{coMod}\left(k[\bd{M}]\right)$$
is an isomorphism of categories over $\Mod(k)$.
\end{theorem}
\begin{proof}
We fix notation in the proof. We denote by $\mu_A:A\otimes_kA\ra A$ the multiplication and by $\eta_A:k\ra A$ the unit for every $k$-algebra $A$. If $A$ is a $k$-algebra, then we denote by $e_A$ the composition $\eta_A \cdot \xi_{\bd{M}}:k[\bd{M}]\ra A$. Note that $e_A\in \fP_{\bd{M}}(A)$ is the neutral element.\\
We start the proof with some useful remarks. If $V$ is a $k$-module, then
$$\cL_V(A) = \Hom_k(V,A\otimes_kV)$$
for every $k$-algebra $A$ with $\fO_k$-algebra structure discussed in Example \ref{example:general_linear_monoid}. Moreover, if $\rho:\fP_{\bd{M}}\ra \cL_V$ is a morphism of $k$-functors corresponding to $d_{\rho}:V\ra k[\bd{M}]\otimes_kV$, then for every $k$-algebra $A$ and a morphism $f:k[\bd{M}]\ra A$ of $k$-algebras we have
$$\rho(f) = \left(f\otimes_k1_V\right)\cdot d_{\rho}$$
Our discussion in Example \ref{example:general_linear_monoid} and Yoneda Lemma show that the following assertions hold.
\begin{enumerate}[label=\textbf{(\arabic*)}, leftmargin=1.5em]
\item For $k$-algebra $A$ and $f_1,f_2\in \Hom_k\left(k[\bd{M}],A\right) = \fP_{\bd{M}}(A)$ we have
$$\rho(f_1)\cdot \rho(f_2) = \left(\mu_A\otimes_k1_V\right)\cdot \left(f_2\otimes_kf_1\otimes_k1_V\right) \cdot \left(1_{k[\bd{M}]}\otimes_k d_{\rho}\right) \cdot d_{\rho}$$
and
$$\rho(f_1\cdot f_2)=\left(\mu_A\otimes_k1_V\right)\cdot \left(f_2\otimes_kf_1\otimes_k1_V \right)\cdot \left(\Delta_{\bd{M}}\otimes_k 1_V\right)\cdot d_{\rho}$$
\item For $k$-algebra $A$ we have
$$\rho(e_A) = \left(\eta_A\otimes_k 1_V\right)\cdot \left(\xi_{\bd{M}}\otimes_k 1_V\right) \cdot d_{\rho}$$
\end{enumerate}
Now \textbf{(1)} imply that if $\left(\Delta_{\bd{M}}\otimes_{k}1_V\right)\cdot d_{\rho}=\left(1_{O_{\bd{M}}}\otimes_{k}d_{\rho}\right)\cdot d_{\rho}$
then $\rho(f_1\cdot f_2) = \rho(f_1)\cdot \rho(f_2)$. On the other hand suppose that $\rho(f_1\cdot f_2) = \rho(f_1)\cdot \rho(f_2)$ for any two $f_1,f_2:k[\bd{M}]\ra A$ morphism of $k$-algebras and for every $k$-algebra $A$. Pick inclusions $f_1,f_2:k[\bd{M}]\ra k[\bd{M}]\otimes_kk[\bd{M}]$ onto first and second component, respectively. Then
$$\left(\mu_{k[\bd{M}]\otimes_kk[\bd{M}]}\otimes_k1_V\right)\cdot \left(f_2\otimes_kf_1\otimes_k1_V\right) = 1_{k[\bd{M}]}\otimes_k1_{k[\bd{M}]}\otimes_k1_V$$
and hence $\left(\Delta_{\bd{M}}\otimes_{k}1_V\right)\cdot d_{\rho}=\left(1_{O_{\bd{M}}}\otimes_{k}d_{\rho}\right)\cdot d_{\rho}$ by \textbf{(1)}.\\
Now if $\left(\xi_{\bd{M}}\otimes_k 1_V\right) \cdot d_{\rho}$ is the canonical isomorphism $V\cong k\otimes_kV$. Then by \textbf{(2)} we derive that $\rho(e_A)$ is the canonical morphism $V \ra A\otimes_kV$. On the other hand if $\rho(e_A)$ is $V \ra A\otimes_kV$ for every $k$-algebra $A$, then substituting $k$ for $A$ we deduce by \textbf{(2)} that $\rho(e_k) = \left(\xi_{\bd{M}}\otimes_k 1_V\right) \cdot d_{\rho}$ is the canonical isomorphism $V\cong k\otimes_kV$.\\
These considerations prove that $\rho$ is a morphism of monoid $k$-functors if and only if $d_{\rho}$ is a coaction of $k[\bd{M}]$ on $V$.\\
Now suppose that $V_1,V_2$ are $k$-modules and $\rho_1:\fP_{\bd{M}}\ra \cL_V,\,\rho_2:\fP_{\bd{M}}\ra \cL_W$ are morphisms of $k$-functors. Suppose that $\phi:V_1\ra V_2$ is a morphism of $k$-modules. Pick a $k$-algebra $A$ and a morphism $f:k[\bd{M}]\ra A$ of $k$-algebras. Assume that the diagram
\begin{center}
\begin{tikzpicture}
[description/.style={fill=white,inner sep=2pt}]
\matrix (m) [matrix of math nodes, row sep=3em, column sep=4em,text height=1.5ex, text depth=0.25ex] 
{k[\bd{M}]\otimes_{k}V_1 &    k[\bd{M}]\otimes_{k}V_2     \\
              V_1&         V_2  \\} ;
\path[->,line width=1.0pt,font=\scriptsize]
(m-1-1) edge node[above] {$  1_{k[\bd{M}]}\otimes_{k}\phi$} (m-1-2)
(m-2-2) edge node[right] {$d_{\rho_2} $} (m-1-2)
(m-2-1) edge node[left] {$d_{\rho_1} $} (m-1-1)
(m-2-1) edge node[below] {$\phi$} (m-2-2);
\end{tikzpicture}
\end{center}
is commutative. Since the square
\begin{center}
\begin{tikzpicture}
[description/.style={fill=white,inner sep=2pt}]
\matrix (m) [matrix of math nodes, row sep=3em, column sep=4em,text height=1.5ex, text depth=0.25ex] 
{   A\otimes_k V_1        &         A\otimes_k V_2           \\
 k[\bd{M}]\otimes_{k}V_1 &    k[\bd{M}]\otimes_{k}V_2      \\} ;
\path[->,line width=1.0pt,font=\scriptsize]
(m-1-1) edge node[above] {$ 1_A\otimes_{k}\phi $} (m-1-2)
(m-2-2) edge node[right] {$ f\otimes_k1_W $} (m-1-2)
(m-2-1) edge node[left] {$ f\otimes_k1_V $} (m-1-1)
(m-2-1) edge node[below] {$ 1_{k[\bd{M}]}\otimes_k\phi $} (m-2-2);
\end{tikzpicture}
\end{center}
is commutative, we derive that
\begin{center}
\begin{tikzpicture}
[description/.style={fill=white,inner sep=2pt}]
\matrix (m) [matrix of math nodes, row sep=3em, column sep=4em,text height=1.5ex, text depth=0.25ex] 
{ A\otimes_{k}V_1 &    A\otimes_{k}V_2     \\
              V_1&         V_2  \\} ;
\path[->,line width=1.0pt,font=\scriptsize]
(m-1-1) edge node[above] {$  1_{A}\otimes_{k}\phi$} (m-1-2)
(m-2-2) edge node[right] {$\rho_2(f) $} (m-1-2)
(m-2-1) edge node[left] {$\rho_1(f) $} (m-1-1)
(m-2-1) edge node[below] {$\phi$} (m-2-2);
\end{tikzpicture}
\end{center}
Moreover, if the square above commutes for every $k$-algebra $A$, then it also commutes for $A = k[\bd{M}]$ and this recovers the commutativity of the first square. Suppose now that $(V,\rho_1)$ and $(W, \rho_2)$ are linear representations of $\bd{M}$, then the discussion above implies that $\phi$ is a morphism of linear representations if and only if $\phi$ is a morphism of $k[\bd{M}]$-comodules $(V,d_{\rho_1})$ and $(W, d_{\rho_2})$.
\end{proof}
\noindent
We obtain immediate consequence.

\begin{corollary}\label{corollary:each_linear_representation_is_rational}
Let $k$ be a field. Let $(V,\rho)$ be a linear representation of an affine monoid $k$-scheme $\bd{M}$. Then for every finitely generated $k$-subspace $U \subseteq V$ there exists a subrepresentation $W$ of $(V,\rho)$ such that $U \subseteq W$ and $W$ is a finitely generated $k$-module.
\end{corollary}
\begin{proof}
This follows from Theorems \ref{theorem:comodules_equivalent_to_representations} and \ref{theorem:rationality_of_comodules}.
\end{proof}















\small
\bibliographystyle{alpha}
\bibliography{../zzz}


\end{document}

