\input ../pree.tex

\begin{document}

\title{Monoid $k$-functors and their representations}
\date{}
\maketitle

\section{Introduction and notation}
\noindent
In these notes we study algebraic structures in the category of $k$-functors with special emphasis on monoid objects.\\
If $R$ is a ring, then we denote by $R^{\times}$ its multplicative monoid.

\section{Algebraic structures in the category of $k$-functors}
\noindent
In the sequel we assume that the reader is familiar with notions of a monoid, group etc. in arbitrary category with finite products. For definitions and some discussion related to these notions cf. {\cite[pages 2-5]{Maclane}}.

\begin{definition}
\textit{A monoid (group, abelian group, ring) $k$-functor} is a monoid (group, abelian group, ring) object in the category of $k$-functors.
\end{definition}

\begin{example}\label{example:endomorphisms_of_k_functor}
Let $\fX$ be a $k$-functor such that $\iMor_k(\fX,\fX)$ exists. Then $\iMor_k(\fX,\fX)$ is a monoid $k$-functor with respect to composition of morphisms.
\end{example}

\begin{example}\label{example:constant_ring_k_functor}
Basic example of a ring $k$-functor is a $k$-functor $\ideal{K}$ given by
$$\fK(A) = k,\,\fK(f) = 1_k$$
for any $k$-algebra $A$ and morphism $f$ of $k$-algebras. It can be described as a constant $k$-functor ({\cite[page 67]{Maclane}}) corresponding to $k$.
\end{example}

\begin{definition}
Let $\fR$ be a ring $k$-functor. Then we denote by $\fR^{\times}$ the $k$-subfunctor of $\fR$ defined by
$$\fR^{\times}(A) = \fR(A)^{\times}$$
for every $k$-algebra $A$. We call $\fR^{\times}$ \textit{the multiplicative monoid $k$-functor of $\fR$}.
\end{definition}

\begin{definition}
Let $\fA$ be a commutative ring $k$-functor. \textit{An $\fA$-algebra} is an $\fA$-algebra object in the category of $k$-functors.
\end{definition}

\section{Global regular functions on a $k$-functor}
\noindent
Recall the ring $k$-functor $\fK$ from Example \ref{example:constant_ring_k_functor}. Note that a $\fK$-algebra $\fA$ can be viewed as a functor $\fA:\Alg_k\ra \Alg_k$.

\begin{definition}
The $\fK$-algebra $\fO_k$ represented by the identity functor on $\Alg_k$ is called \textit{the structure $\fK$-algebra}.
\end{definition}
\noindent
Let $|-|:\Alg_k\ra \Set$ be the forgetful $k$-functor. Note that $|-|$ is the underlying $k$-functor of $\fK$-algebra $\fO_k$. Recall that the affine line $\mathbb{A}^1_k$ is an affine $k$-scheme having $k$-algebra of polynomials with one variable as a $k$-algebra of regular functions.

\begin{fact}\label{fact:affineline_as_forgetfulfunctor}
Let $|-|:\Alg_k\ra \Set$ be the forgetful $k$-functor. Then we have natural isomorphism
$$\fP_{\mathbb{A}^1_k} \cong |-|$$
\end{fact}
\begin{proof}
Let $B$ be a $k$-algebra. We have the following chain of identifications
$$\fP_{\mathbb{A}^1_k}(B) = \Mor_k\left(\Spec B, \mathbb{A}^1_k\right) = \Mor_k\left(\Spec B, \Spec k[x]\right) = \Mor_k\left(k[x], B\right) = |B|$$
natural in $B$.
\end{proof}
\noindent
In particular, since $|-|$ carries the structure $\fK$-algebra $\fO_k$, we derive that $\fP_{\mathbb{A}^1_k}$ admits a structure of $\fK$-algebra isomorphic to $\fO_k$.\\
No we introduce regular functions on $k$-functors.

\begin{definition}
Let $\fX$ be a $k$-functor and assume that $\iMor_k\left(\fX, \fO_k\right)$ is a set. Then $\Mor_k\left(\fX, \fO_k\right)$ is a $k$-algebra with respect to the structure induced by $\fO_k$. We call this $k$-algebra \textit{the $k$-algebra of global regular functions on $\fX$}. Its elements are called \textit{global regular functions on $\fX$}.
\end{definition}

\begin{definition}
Let $\fX$ be a $k$-functor. Suppose that $A$ is a $k$-algebra, $x\in \fX(A)$ and $f\in \Mor_k\left(\fX,\fO_k\right)$. The element $f(x) \in A$ is called \textit{the value of $f$ on a point $x$}.
\end{definition}
\noindent
For given $k$-functor $\fX$ we describe $k$-algebra operations on $\Mor_k\left(\fX,\fO_k\right)$ in terms of values of its elements on points of $\fX$. For this consider $\alpha \in k$ and $f$, $g\in \Mor_k\left(\fX, \fO_k\right)$. We have formulas
$$\left(f+g\right)(x) = f(x)+g(x),\,\left(f\cdot g\right)(x) = f(x)\cdot g(x),\,\left(\alpha \cdot f\right)(x) = \alpha \cdot f(x)$$
in which right hand side are $k$-algebra operations in $A$.

\begin{example}\label{example:regular_functions_as_an_algebra_over_structure_algebra}
Let $\fX$ be a $k$-functor and assume that $\iMor_k(\fX,\fO_k)$ exists. Fix $k$-algebra $A$. Note that $\Mor_A(\fX_A,\fO_A)$ is an $A$-algebra of global regular functions on $\fX_A$. Moreover, if $B$ is an $A$-algebra, then
$$\Mor_A(\fX_A,\fO_A) \ni f \mapsto f_B\in \Mor_B(\fX_B,\fO_B)$$
is a morphism of $A$-algebras. This implies that $\iMor_k(\fX,\fO_k)$ admits a canonical structure of an $\fO_k$-algebra $k$-functor.
\end{example}

\section{Internal hom and product of $k$-functors}
\noindent
We denote by $\bd{1}$ a $k$-functor that assigns to every $k$-algebra a set with one element. Then for every $k$-algebra $A$ the restriction $\bd{1}_A$ is a terminal object in the category of $A$-functors.

\begin{fact}\label{fact:points_and_morphisms_from_terminal_k_functor}
Let $\fX$ be a $k$-functor. Suppose $A$ is a $k$-algebra and $x\in \fX(A)$. Then $x$ determines a morphism $\bd{1}_{A}\ra \fX_A$ that for every $A$-algebra $B$ with structural morphism $f:A\ra B$ sends a unique element of $\bd{1}_{A}(B)$ to $\fX(f)(x)\in \fX_A(B)$. This gives rise to a bijection
$$\fX(A)\cong \Mor_{A}\left(\bd{1}_{A},\fX_A\right)$$
\end{fact}
\begin{proof}
Left to the reader as an exercise.
\end{proof}
\noindent
The discussion below is partially an application of the main result in {\cite[section 6]{Presheaves}}. For reader's convenience we make our presentation self-contained.

\begin{definition}
Let $\fU,\fX,\fY$ be $k$-functors and let $\sigma:\fU\times \fX\ra \fY$ be a morphism of $k$-functors. Fix $z\in \fU(A)$ for some $k$-algebra $A$. We denote by $i_z:\bd{1}_A\ra \fU_A$ the morphism of $A$-functors corresponding to $z$ by Fact \ref{fact:points_and_morphisms_from_terminal_k_functor}. Since $\bd{1}_A$ is terminal $A$-functor, a morphism $\sigma_A\cdot \left(i_z\times 1_{\fX_A}\right)$ is isomorphic to a morphism $\sigma_z:\fX_A\ra \fY_A$ of $A$-functors. We call $\sigma_z$ \textit{the slice of $\sigma$ over $z$}.
\end{definition}

\begin{definition}
Let $\fX, \fY$ be $k$-functors. Let $\fJ$ be a $k$-functor such that $\fJ(A)$ is a subset of a class $\Mor_A\left(\fX_A,\fY_A\right)$ for every $k$-algebra $A$. Assume that for every morphism $f:A\ra B$ of $k$-algebras and every $\sigma\in \fJ(A)$ we have
$$\fJ(f)(\sigma) = \sigma_B$$
where $\sigma_B\in \Mor_B\left(\fX_B,\fY_B\right)$ is the restriction of $\sigma$ along $f$. Then we call $\fJ$ \textit{a $k$-subfunctor of internal hom of $\fX$ and $\fY$}.
\end{definition}

\begin{definition}
Let $\fX, \fY,\fU$ be $k$-functors and let $\sigma:\fU\times \fX \ra \fY$ be a morphism of $k$-functors. Suppose that $\fJ$ is a $k$-subfunctor of internal hom of $\fX$ and $\fY$. Assume that $\sigma_z:\fX_A\ra \fY_A$ is contained in $\fJ(A)$ for every $k$-algebra $A$ and $z\in \fU(A)$. Then we call $\sigma$ \textit{a family of $\fJ$-morphisms parametrized by $\fU$}.
\end{definition}
\noindent
Let $\fJ$ be a $k$-subfunctor of internal hom of $\fX$ and $\fY$. Assume that $\sigma:\fU\times \fX\ra \fY$ is a $\fJ$-family of morphism parametrized by $\fU$. Then the family of maps
$$\fU(A)\ni z\mapsto \sigma_z\in \fJ(A)$$
gives rise to a morphism $\tau:\fU\ra \fJ$ of $k$-functors. Indeed, for a morphism $f:A\ra B$ of $k$-algebras and $z\in \fU(A)$ we have
$$\sigma_B\cdot \left(i_{\fU(f)(z)}\times 1_{\fX_B}\right) = \left(\sigma_A\cdot \left(i_{z}\times 1_{\fX_A}\right)\right)_B$$
and hence $\sigma_{\fU(f)(z)} = \left(\sigma_z\right)_B$. This gives rise to a map $\Phi$ of classes
$$\bigg\{\mbox{families }\fU\times \fX\ra \fY\mbox{ of }\fJ\mbox{-morphisms parametrized by }\fU\bigg\}\ni \sigma\mapsto \tau\in \Mor_k\left(\fU,\fJ\right)$$
Consider next a morphism $\tau:\fU\ra \fJ$ of $k$-functors and define $\sigma:\fU\times \fX\ra \fY$ by formula $\sigma^A(z,x) = \left(\tau^A(z)\right)^A(x)$ for every $k$-algebra $A$ and points $z\in \fU(A),\,x\in \fX(A)$. Let $f:A\ra B$ be a morphism of $k$-algebras. Then
$$\sigma^B\left(\fU(f)(z),\fX(f)(x)\right) = \left(\tau^B\left(\fU(f)(z)\right)\right)^B\left(\fX(f)(x)\right) = \left(\left(\tau^A(z)\right)_B\right)^B\left(\fX(f)(x)\right) =$$
$$ = \left(\tau^A(z)\right)^B\left(\fX(f)(x)\right) = \fY(f)\left(\left(\tau^A(z)\right)^A(x)\right) = \fY(f)\left(\sigma^A(z,x)\right)$$
Thus $\sigma:\fU\times \fX\ra \fY$ is a morphism of $k$-functors. For every $k$-algebra $A$ and $z\in\fU(A)$ we have $\sigma_z = \tau^A(z)$. Indeed, let $f:A\ra B$ be a morphism of $k$-algebras and $x$ be an element in $\fX(B)$ then we have
$$\left(\sigma_z\right)^B(x) = \sigma^B\left(\fU(f)(z),x\right) = \left(\tau^B\left(\fU(f)(z)\right)\right)^B(x) = \left(\left(\tau^A(z)\right)_B\right)^B(x) = \left(\tau^A(z)\right)^B(x)$$
Hence $\sigma$ is a family of $\fJ$-morphisms parametrized by $\fU$. This gives rise to a map $\Psi$ of classes
$$\Mor_k\left(\fU,\fJ\right) \ni \tau \mapsto \sigma\in \bigg\{\mbox{families }\fU\times \fX\ra \fY\mbox{ of }\fJ\mbox{-morphisms parametrized by }\fU\bigg\}$$
Now we have the following result, which is an instance {\cite[Theorem 6.3]{Presheaves}}. To make presentation self-contained we give a complete proof.  

\begin{theorem}\label{theorem:k_subfunctors_of_hom}
Let $\fX$, $\fY$ be $k$-functors and let $\fJ$ be a $k$-subfunctor of internal hom of $\fX$ and $\fY$. Then maps 
$$\Phi:\bigg\{\mbox{families }\fU\times \fX\ra \fY\mbox{ of }\fJ\mbox{-morphisms parametrized by }\fU\bigg\} \ra  \Mor_{k}\left(\fU,\fJ\right)$$
and
$$\Psi:\Mor_{k}\left(\fU,\fJ\right) \ra \bigg\{\mbox{families }\fU\times \fX\ra \fY\mbox{ of }\fJ\mbox{-morphisms parametrized by }\fU\bigg\}$$
are mutually inverse bijections.
\end{theorem}
\begin{proof}
Pick a morphism $\tau:\fU\ra \fJ$ of $k$-functors. Let $A$ be a $k$-algebra and $z\in \fU(A)$. In the discussion preceding the statement we showed that $\Psi(\tau)_z = \tau^A(z)$. Thus
$$\left(\Phi(\Psi(\tau))\right)^A(z) = \Psi(\tau)_z = \tau^A(z)$$
and hence $\Phi\cdot \Psi$ is the identity.\\
Pick a family of $\fJ$-morphism $\sigma:\fU\times \fX\ra \fY$ parametrized by $\fU$. Let $A$ be a $k$-algebra and $z\in \fU(A)$, $x\in \fX(A)$ be points. Then
$$\left(\Psi\left(\Phi(\sigma)\right)\right)^A(z,x) = \left(\Phi(\sigma)^A(z)\right)^A(x) = \sigma_z^A(x) = \sigma^A(z,x)$$
Thus $\Psi\cdot \Phi$ is the identity map.
\end{proof}
\noindent
Now we formulate some consequences of Theorem \ref{theorem:k_subfunctors_of_hom}.

\begin{corollary}\label{corollary:hom_k_functors}
Let $\fX$, $\fY$ be $k$-functors. Assume that for every $k$-algebra $A$ the class $\Mor_{A}\left(\fX_A,\fY_A\right)$ is a set. Then there is a bijection 
$$\Mor_{k}\left(\fU\times \fX,\fY\right)\ra  \Mor_{k}\left(\fU,\iMor_{k}\left(\fX,\fY\right)\right)$$
of classes.
\end{corollary}

\begin{definition}
Let $\fX,\fY$ be $k$-functors. If $\Iso_{A}\left(\fX_A,\fY_A\right)$ is a set for every $k$-algebra $A$, then we define a $k$-subfunctor $\iIso_k\left(\fX,\fY\right)$ of $\iMor_k\left(\fX,\fY\right)$ by
$$\iIso_k\left(\fX,\fY\right)(A) = \Iso_A\left(\fX_A,\fY_A\right)$$
for every $k$-algebra $A$. We call $\iIso_k\left(\fX,\fY\right)$ \textit{the $k$-functor of isomorphism}.
\end{definition}

\begin{definition}
Let $\fX, \fY,\fU$ be $k$-functors and let $\sigma:\fU\times \fX \ra \fY$ be a morphism of $k$-functors. Assume that $\sigma_z:\fX_A\ra \fY_A$ is an isomorphism of $A$-functors for every $k$-algebra $A$. Then we call $\sigma$ \textit{a family of isomorphisms parametrized by $\fU$}.
\end{definition}

\begin{corollary}\label{corollary:hom_isomorphism_bijection}
Let $\fU,\fX,\fY$ be $k$-functors and suppose that for every $k$-algebra $A$ the class $\Iso_A\left(\fX_A,\fY_A\right)$ is a set. The the following map
$$\bigg\{\mbox{families }\fU\times \fX\ra \fY\mbox{ of isomorphism parametrized by }\fU\bigg\}\ra \Mor_k\left(\fU,\iIso_k\left(\fX,\fY\right)\right)$$
is a bijection of classes.
\end{corollary}


\section{Actions of monoid $k$-functors}\label{section:actions_of_monoid_k_functors}
\noindent
In this section we assume that the reader is familiar with notion of an action of a monoid object in arbitrary category with finite products. For definitions and some discussion related to these notions cf. {\cite[pages 5]{Maclane}}.\\
Let $\fG$ be a monoid $k$-functor and $\fX$ be a $k$-functor together with an action $\alpha:\fG\times \fX \ra \fX$. Next assume that $k$-functor $\iMor_k(\fX,\fX)$ exists. By Example \ref{example:endomorphisms_of_k_functor} it is a monoid $k$-functor. We define a morphism $\rho:\fG\ra \iMor_k(\fX,\fX)$ of $k$-functors by formula $\rho(x) = \alpha_x$. Note that by discussion preceding Theorem \ref{theorem:k_subfunctors_of_hom}, we deduce that $\rho$ is a well defined morphism of $k$-functors. We show now that $\rho$ is a morphism of monoids. For this pick $k$-algebra $A$ and $x, y\in \fG(A)$. Since $\alpha$ is an action, we deduce that $\alpha_{x \cdot y} = \alpha_x \cdot \alpha_y$ and hence also
$$\rho(x\cdot y) = \alpha_{x \cdot y} = \alpha_x\cdot \alpha_y = \rho(x)\cdot \rho(y)$$
Therefore, $\rho$ is a morphism of monoid $k$-functors. This shows how to construct a morphism of monoid $k$-functors $\rho$ from an action $\alpha$ of $\fG$.

\begin{theorem}\label{theorem:actions_and_monoid_morphisms}
Let $\fG$ be a monoid $k$-functor and let $\fX$ be a $k$-functor such that $\iMor_{k}(\fX,\fX)$ exists. Suppose that
\begin{center}
\begin{tikzpicture}
[description/.style={fill=white,inner sep=2pt}]
\matrix (m) [matrix of math nodes, row sep=3em, column sep=4em,text height=1.5ex, text depth=0.25ex] 
{\bigg\{\mbox{actions of $\fG$ on $\fX$}\bigg\} & \bigg\{\mbox{Morphisms $\rho:\fG\ra \iMor_{k}(\fX,\fX)$ of monoid $k$-functors}\bigg\} \\};
\path[->,line width=1.0pt,font=\scriptsize]  
(m-1-1) edge node[auto] {$ $} (m-1-2);
\end{tikzpicture}
\end{center}
is a map of classes described above. Then it is bijection.
\end{theorem}
\begin{proof}
Our goal is to construct the inverse of the map. Substitute $\fJ = \iMor_k(\fX,\fX)$ in Theorem \ref{theorem:k_subfunctors_of_hom}. Consider maps
$$\Phi:\bigg\{\mbox{families }\fG\times \fX\ra \fX\mbox{ of morphisms}\bigg\} \ra  \Mor_{k}\left(\fG,\iMor_k(\fX,\fX)\right)$$
and
$$\Psi:\Mor_{k}\left(\fG, \iMor_k(\fX,\fX)\right) \ra \bigg\{\mbox{families }\fG\times \fX\ra \fX\mbox{ of morphisms}\bigg\}$$
in that Theorem. Then the map in the statement above is the restriction of $\Phi$ to $\fG$-actions on $\fX$ on the right and morphisms $\fG\ra \iMor_k(\fX,\fX)$ of monoid $k$-functors on the left. Since by Theorem \ref{theorem:k_subfunctors_of_hom} maps $\Phi$ and $\Psi$ are mutually inverse, it suffices to check that $\Psi$ sends a morphism $\rho:\fG\ra \iMor_{k}(\fX,\fX)$ of monoids to an action of $\fG$ on $\fX$. For this denote $\Psi(\rho)$ by $\alpha$. Consider $k$-algebra $A$ and $A$-points $x,y\in \fG(A),\,z\in \fX(A)$. Then
$$\alpha\left(y, \alpha(x, z)\right) = \rho(y)\left(\rho(x)(z)\right) = \left(\rho(y)\cdot \rho(x)\right)(z) = \rho\left(x\cdot y\right)(z) = \alpha\left(x\cdot y, z\right)$$
Therefore, $\alpha$ is an action of $\fG$ on $\fX$.
\end{proof}

\begin{proposition}\label{proposition:morphism_of_monoid_actions}
Let $\fG$ be a monoid $k$-functor and let $\fX_1$, $\fX_2$ be $k$-functors such that $\iMor_k(\fX_1,\fX_1),\iMor_k(\fX_2,\fX_2)$ exist. Suppose that $\alpha_1:\fG\times \fX_1 \ra \fX_1,\,\alpha_2:\fG\times \fX_2 \ra \fX_2$ are actions of $\fG$, respectively. Suppose that $\sigma:\fX_1\ra \fX_2$ is a morphism of $k$-functors. Then the following assertions are equivalent.
\begin{enumerate}[label=\emph{\textbf{(\roman*)}}, leftmargin=1.5em]
\item The square
\begin{center}
\begin{tikzpicture}
[description/.style={fill=white,inner sep=2pt}]
\matrix (m) [matrix of math nodes, row sep=3em, column sep=5em,text height=1.5ex, text depth=0.25ex] 
{  \fG\times \fX_1   & \fG\times \fX_2 \\
   \fX_1             & \fX_2 \\} ;
\path[->,line width=1.0pt,font=\scriptsize]  
(m-1-1) edge node[above] {$ 1_{\fG}\times \sigma  $} (m-1-2)
(m-2-1) edge node[below] {$ \sigma $} (m-2-2)
(m-1-1) edge node[left] {$ \alpha_1 $} (m-2-1)
(m-1-2) edge node[right] {$ \alpha_2 $} (m-2-2);
\end{tikzpicture}
\end{center}
is commutative.
\item For every $k$-algebra $A$ and $x\in \fG(A)$ we have
$$\sigma_A \cdot \rho_1(x) = \rho_2(x) \cdot \sigma_A$$
where $\rho_1:\fG\ra \iMor_k(\fX_1,\fX_1)$ and $\rho_2:\fG\ra \iMor_{k}(\fX_2,\fX_2)$ are morphism of monoid $k$-functors corresponding to $\alpha_1$ and $\alpha_2$, respectively.
\end{enumerate}
\end{proposition}
\begin{proof}
Conditions expressed in \textbf{(i)} and \textbf{(ii)} are directly translatable to each other by virtue of the bijection in Theorem \ref{theorem:actions_and_monoid_morphisms}. 
\end{proof}

\begin{definition}
Let $\fG$ be a monoid $k$-functor and let $(\fX_1,\alpha_1)$, $(\fX_2,\alpha_2)$ be $k$-functors with actions of $\fG$. Suppose that $\sigma:\fX_1\ra \fX_2$ is a morphism $k$-functors such that the square
\begin{center}
\begin{tikzpicture}
[description/.style={fill=white,inner sep=2pt}]
\matrix (m) [matrix of math nodes, row sep=3em, column sep=5em,text height=1.5ex, text depth=0.25ex] 
{  \fG\times \fX_1   & \fG\times \fX_2 \\
   \fX_1             & \fX_2 \\} ;
\path[->,line width=1.0pt,font=\scriptsize]  
(m-1-1) edge node[above] {$ 1_{\fG}\times \sigma  $} (m-1-2)
(m-2-1) edge node[below] {$ \sigma $} (m-2-2)
(m-1-1) edge node[left] {$ \alpha_1 $} (m-2-1)
(m-1-2) edge node[right] {$ \alpha_2 $} (m-2-2);
\end{tikzpicture}
\end{center}
is commutative. Then $\sigma$ is called \textit{an $\fG$-equivariant morphism}.
\end{definition}

\section{Modules over ring $k$-functors}

\begin{definition}
Let $\fR$ be a ring $k$-functor. Suppose that $\fM$ is an abelian group $k$-functor and there exists a morphism $\fR \times \fM\ra \fM$ of $k$-functors that for each $k$-algebra $A$ makes $\fM(A)$ into an $\fR(A)$-module. Then we say that $\fM$ is \textit{a module $k$-functor over $\fR$}.
\end{definition}

\begin{definition}
Let $\fR$ be an ring $k$-functor and let $\fM_1,\fM_2$ be module $k$-functors over $\fR$. Suppose that $\sigma:\fM_1\ra \fM_2$ is a morphism of abelian group $k$-functors such that the diagram
\begin{center}
\begin{tikzpicture}
[description/.style={fill=white,inner sep=2pt}]
\matrix (m) [matrix of math nodes, row sep=3em, column sep=5em,text height=1.5ex, text depth=0.25ex] 
{  \fR \times \fM_1   & \fR \times \fM_2  \\
   \fM_1              & \fM_2            \\} ;
\path[->,line width=1.0pt,font=\scriptsize]  
(m-1-1) edge node[above] {$ 1_{\fR} \times \sigma  $} (m-1-2)
(m-2-1) edge node[below] {$ \sigma $} (m-2-2)
(m-1-1) edge node[left] {$ \alpha_1 $} (m-2-1)
(m-1-2) edge node[right] {$ \alpha_2 $} (m-2-2);
\end{tikzpicture}
\end{center}
is commutative, where $\alpha_i:\fR\times \fM_i\ra \fM_i$ define $\fR$-module structure on $\fM_i$ for $i=1,2$. Then $\sigma$ is \textit{a morphism of modules over $\fR$}.
\end{definition}
\noindent
Let $\fM_1$ and $\fM_2$ be module $k$-functors over $\fR$. We denote by
$$\Hom_{\fR}\left(\fM_1,\fM_2\right)$$
the class of all morphisms of modules $\fM_1\ra \fM_2$ over $\fR$. We denote the category of $\fR$-modules by $\Mod\left(\fR\right)$.

\begin{definition}
Let $\fM_1$ and $\fM_2$ be module $k$-functors over $\fR$. Assume that $\Hom_{\fR_A}\left((\fM_1)_A,(\fM_2)_A\right)$ is a set for every $k$-algebra $A$. Then we define a $k$-subfunctor $\shHom_{\fR}(\fM_1,\fM_2)$ of internal hom of $\fM_1$ and $\fM_2$ by formula
$$\Alg_k\ni A \mapsto \Hom_{\fR_A}\left((\fM_1)_A,(\fM_2)_A\right) \in \Set$$
We call $\shHom_{\fR}(\fM_1,\fM_2)$ \textit{a $k$-functor of module morphisms of $\fM_1$ and $\fM_2$}.
\end{definition}
\noindent
If $\fM$ is a module $k$-functor over some ring $k$-functor $\fR$, then we denote (if it exists) $\shHom_{\fR}(\fM,\fM)$ by $\cE nd_{\fR}(\fM)$.

\begin{example}\label{example:endomorphisms_of_module_k_functor}
Let $\fM$ be a module over a ring $k$-functor $\fR$. Assume that $\cE nd_{\fR}(\fM)$ exists. Then $\cE nd_{\fR}(\fM)$ is a ring $k$-functor with respect to composition of morphisms of modules as the multiplication and the usual addition of module morphisms. Moreover, if $\fA$ is a commutative ring $k$-functor, then $\cE nd_{\fA}(\fM)$ (if exists) admits additional structure of a $\fA$-algebra $k$-functor induced via a unique morphism $\fA\ra \cE nd_{\fR}(\fM)$ of ring $k$-functors that sends $1\mapsto 1_{\fM}$.
\end{example}
\noindent
Let $\fA$ be a commutative ring $k$-functor and let $\fR$ be a $\fA$-algebra $k$-functor. This means that there exists a morphism $\fA\ra \fR$ of ring $k$-functors and for every $k$-algebra $A$ induced morphism $\fA(A)\ra \fR(A)$ sends $\fA(A)$ to the center of a ring $\fR(A)$. Fix a module $\fM$ over $\fA$. Next assume that $k$-functor $\cE nd_{\fA}(\fM)$ exists. By Example \ref{example:endomorphisms_of_module_k_functor} it is a ring $k$-functor.

\begin{definition}
In the setting above suppose that $\alpha:\fR\times \fM\ra \fM$ is a morphism of $k$-functors. Suppose that $\alpha$ makes $\fM$ into $\fR$-module and moreover, for every $k$-algebra $A$ and for every point $x\in \fR(A)$ morphism $\alpha_x$ is a morphism of $\fA_A$-modules. Then $\alpha$ is called \textit{a $\fA$-linear $\fR$-action on $\fM$}.
\end{definition}
\noindent
We continue the discussion. We assume that we are given an $\fA$-linear $\fR$-action $\alpha:\fR\times \fM \ra \fM$ on $\fM$. We define a morphism $\rho:\fR\ra \cE nd_{\fA}(\fM)$ of $k$-functors by formula $\rho(x) = \alpha_x$. As in Section \ref{section:actions_of_monoid_k_functors} we can prove that $\rho$ is a morphism of ring $k$-functors. Now we have the following result.

\begin{theorem}\label{theorem:linear_morphisms_and_homomorphisms_of_rings}
Let $\fR$ be an algebra $k$-functor over commutative ring $\fA$ $k$-functor and let $\fM$ be a $\fA$-module such that $\cE nd_{\fA}(\fM)$ exists. Suppose that
\begin{center}
\begin{tikzpicture}
[description/.style={fill=white,inner sep=2pt}]
\matrix (m) [matrix of math nodes, row sep=3em, column sep=4em,text height=1.5ex, text depth=0.25ex] 
{\bigg\{\mbox{$\fA$ linear actions of $\fR$ on $\fM$}\bigg\} & \bigg\{\mbox{Morphisms $\rho:\fR\ra \cE nd_{\fO_k}(\fM)$ of ring $k$-functors}\bigg\} \\};
\path[->,line width=1.0pt,font=\scriptsize]  
(m-1-1) edge node[auto] {$ $} (m-1-2);
\end{tikzpicture}
\end{center}
is a map of classes described above. Then it is bijection.
\end{theorem}
\begin{proof}
The proof is similar to the proof of Theorem \ref{theorem:actions_and_monoid_morphisms}.
\end{proof}

\section{Monoid algebra $\fO_k[\fG]$ and its modules}

\begin{definition}
Let $\fG$ be a monoid $k$-functor. Then we construct an $\fO_k$-algebra $\fO_k[\fG]$ as follows. For every $k$-algebra $A$ we define
$$\fO_k[\fG](A) = A\big[\fG(A)\big]$$
where the right hand side is monoid $A$-algebra for the abstract monoid $\fG(A)$. The structure of monoid $k$-functor on $\fG$ and $\fK$-algebra $\fO_k$ makes $\fO_k[\fG]$ into a ring $k$-functor. Moreover, we have a morphism $\fO_k\ra \fO_k[\fG]$ which for every $k$-algebra $A$ is given by the canonical inclusion
$$A \hookrightarrow A\big[\fG(A)\big]$$
Thus $\fO_k[\fG]$ is $\fO_k$-algebra. We call $\fO_k[\fG]$ \textit{a monoid $\fO_k$-algebra over $\fG$}.
\end{definition}

\begin{fact}\label{fact:universal_property_of_monoid_algebra}
Let $\fG$ be a monoid $k$-functor and let $\fR$ be an $\fO_k$-algebra $k$-functor. Then every morphism
$$\sigma:\fG\ra \fR^{\times}$$
of monoid $k$-functors admits a unique extension
$$\tilde{\sigma}:\fO_k[\fG]\ra \fR$$
to a morphism of $\fO_k$-algebras.
\end{fact}
\begin{proof}
This follows from the analogical universal property of algebras over abstract monoids.
\end{proof}

\begin{definition}
Let $\fG$ be a monoid $k$-functor and let $\fM$ be a module over $\fO_k$. Suppose that $\alpha:\fG\times \fM\ra \fM$ is an action of $\fG$ such that for any $k$-algebra $A$ and point $x\in \fG(A)$ morphism $\alpha_x:\fM_A\ra \fM_A$ is a morphism of $\fO_A$-modules. Then $\alpha$ is called \textit{a linear $\fG$-action on $\fM$}.
\end{definition}
\noindent
Suppose now that $\fG$ is a monoid $k$-functor and $\fM$ is a module $\fO_k$. Note that every linear $\fG$-action $\alpha:\fG\times \fM \ra \fM$ extends uniquely to a $\fO_k$-linear action $\fO_k[\fG]\times \fM\ra \fM$ of monoid $\fO_k$-algebra. This gives a bijection
\begin{center}
\begin{tikzpicture}
[description/.style={fill=white,inner sep=2pt}]
\matrix (m) [matrix of math nodes, row sep=3em, column sep=4em,text height=1.5ex, text depth=0.25ex] 
{\bigg\{\mbox{Linear actions of $\fG$ on $\fM$}\bigg\} & \bigg\{\mbox{$\fO_k$-linear actions $\fO_k[\fG]\times \fM\ra \fM$}\bigg\} \\};
\path[->,line width=1.0pt,font=\scriptsize]  
(m-1-1) edge node[auto] {$ $} (m-1-2);
\end{tikzpicture}
\end{center}
Next assume that $k$-functor $\cE nd_{\fO_k}(\fM)$ exists. By Example \ref{example:endomorphisms_of_module_k_functor} it is an $\fO_k$-algebra $k$-functor. Next by Theorem \ref{theorem:linear_morphisms_and_homomorphisms_of_rings} we have a bijection
\begin{center}
\begin{tikzpicture}
[description/.style={fill=white,inner sep=2pt}]
\matrix (m) [matrix of math nodes, row sep=3em, column sep=4em,text height=1.5ex, text depth=0.25ex] 
{\bigg\{\mbox{$\fO_k$-linear actions of $\fO_k[\fG]\times \fM\ra \fM$}\bigg\} & \bigg\{\mbox{Morphisms $\fO_k[\fG]\ra \cE nd_{\fO_k}(\fM)$ of $\fO_k$-algebras}\bigg\} \\};
\path[->,line width=1.0pt,font=\scriptsize]  
(m-1-1) edge node[auto] {$ $} (m-1-2);
\end{tikzpicture}
\end{center}
Finally Fact \ref{fact:universal_property_of_monoid_algebra} implies that we have a bijection
\begin{center}
\begin{tikzpicture}
[description/.style={fill=white,inner sep=2pt}]
\matrix (m) [matrix of math nodes, row sep=3em, column sep=4em,text height=1.5ex, text depth=0.25ex] 
{\bigg\{\mbox{Morphisms $\fO_k[\fG]\ra \cE nd_{\fO_k}(\fM)$ of $\fO_k$-algebras} \bigg\} & \bigg\{\mbox{Morphisms $\fG\ra \cE nd_{\fO_k}(\fM)$ of monoids}\bigg\} \\};
\path[->,line width=1.0pt,font=\scriptsize]  
(m-1-1) edge node[auto] {$ $} (m-1-2);
\end{tikzpicture}
\end{center}
This chain of bijections sends a linear action $\alpha:\fG\times \fM\ra \fM$ of $\fG$ to a morphism $\rho:\fG\ra \cE nd_{\fO_k}(\fM)$ of monoid $k$-functors given by $\rho(x) = \alpha_x$ for every $x\in \fG(A)$ and every $k$-algebra $A$. We proved the following result.

\begin{proposition}\label{proposition:decription_of_linear_monoid_actions_on_modules}
Let $\fG$ be a monoid $k$-functor and let $\fM$ be a $\fO_k$-module such that $\cE nd_{\fO_k}(\fM)$ exists. Then the following classes are in canonical bijections described above.
\begin{enumerate}[label=\emph{\textbf{(\arabic*)}}, leftmargin=1.5em]
\item Linear actions of $\fG$ on $\fM$.
\item $\fO_k$-linear actions $\fO_k[\fG]\times \fM\ra \fM$. These are precisely $\fO_k[\fG]$-modules.
\item Morphisms $\fO_k[\fG]\ra \cE nd_{\fO_k}(\fM)$ of $\fO_k$-algebras.
\item Morphisms $\fG\ra \cE nd_{\fO_k}(\fM)$ of monoids.
\end{enumerate}
Moreover, the bijection between class \emph{\textbf{(1)}} and \emph{\textbf{(2)}} does not require the existence of $\cE nd_{\fO_k}(\fM)$.
\end{proposition}
\noindent
Now in a similar manner we can describe morphisms.

\begin{proposition}\label{proposition:morphism_of_monoid_modules}
Let $\fG$ be a monoid $k$-functor and let $\fM_1$, $\fM_2$ be $k$-functors of $\fO_k$-modules such that $\cE nd_{\fO_k}(\fM_1),\cE nd_{\fO_k}(\fM_2)$ exist. Suppose that $\alpha_1:\fG\times \fM_1 \ra \fM_1,\,\alpha_2:\fG\times \fM_2 \ra \fM_2$ are linear actions of $\fG$. Suppose that $\sigma:\fM_1\ra \fM_2$ is a morphism of modules over $\fO_k$. Then the following assertions are equivalent.
\begin{enumerate}[label=\emph{\textbf{(\roman*)}}, leftmargin=1.5em]
\item The square
\begin{center}
\begin{tikzpicture}
[description/.style={fill=white,inner sep=2pt}]
\matrix (m) [matrix of math nodes, row sep=3em, column sep=5em,text height=1.5ex, text depth=0.25ex] 
{  \fG\times \fM_1   & \fG\times \fM_2 \\
   \fM_1             & \fM_2 \\} ;
\path[->,line width=1.0pt,font=\scriptsize]  
(m-1-1) edge node[above] {$ 1_{\fG}\times \sigma  $} (m-1-2)
(m-2-1) edge node[below] {$ \sigma $} (m-2-2)
(m-1-1) edge node[left] {$ \alpha_1 $} (m-2-1)
(m-1-2) edge node[right] {$ \alpha_2 $} (m-2-2);
\end{tikzpicture}
\end{center}
is commutative.
\item The square
\begin{center}
\begin{tikzpicture}
[description/.style={fill=white,inner sep=2pt}]
\matrix (m) [matrix of math nodes, row sep=3em, column sep=5em,text height=1.5ex, text depth=0.25ex] 
{  \fO_k[\fG] \times \fM_1   & \fO_k[\fG] \times \fM_2 \\
   \fM_1                     & \fM_2 \\} ;
\path[->,line width=1.0pt,font=\scriptsize]  
(m-1-1) edge node[above] {$ 1_{\fO_k[\fG]}\times \sigma  $} (m-1-2)
(m-2-1) edge node[below] {$ \sigma $} (m-2-2)
(m-1-1) edge node[left] {$ \tilde{\alpha_1} $} (m-2-1)
(m-1-2) edge node[right] {$ \tilde{\alpha_2} $} (m-2-2);
\end{tikzpicture}
\end{center}
is commutative, where $\tilde{\alpha_1}$ and $\tilde{\alpha_2}$ are $\fO_k$-linear actions of $\fO_k[\fG]$ corresponding to $\alpha_1$ and $\alpha_2$, respectively. This states that $\sigma$ is a morphism of $\fO_k[\fG]$-modules.
\item For every $k$-algebra $A$ and $x\in \fG(A)$ we have
$$\sigma_A \cdot \tilde{\rho}_1(x) = \tilde{\rho}_2(x) \cdot \sigma_A$$
where $\tilde{\rho}_1:\fO_k[\fG] \ra \cE nd_{\fO_k}(\fM_1)$ and $\tilde{\rho}_2:\fO_k[\fG] \ra \cE nd_{\fO_k}(\fM_2)$ are morphism of $\fO_k$-algebras corresponding to $\tilde{\alpha_1}$ and $\tilde{\alpha_2}$, respectively.
\item For every $k$-algebra $A$ and $x\in \fG(A)$ we have
$$\sigma_A \cdot \rho_1(x) = \rho_2(x) \cdot \sigma_A$$
where $\rho_1:\fG \ra \cE nd_{\fO_k}(\fM_1)$ and $\rho_2:\fG \ra \cE nd_{\fO_k}(\fM_2)$ are restrictions of $\tilde{\rho_1}$ and $\tilde{\rho_2}$, respectively.
\end{enumerate}
The equivalence of \emph{\textbf{(i)}} and \emph{\textbf{(ii)}} does not require the existence of $\cE nd_{\fO_k}(\fM_1)$ and $\cE nd_{\fO_k}(\fM_2)$.
\end{proposition}
\begin{proof}
Conditions expressed in \textbf{(i)}-\textbf{(iv)} are directly translatable to each other by virtue of bijections in Proposition \ref{proposition:decription_of_linear_monoid_actions_on_modules}. 
\end{proof}
\noindent
Let $\fG$ be a monoid $k$-functor. We denote by $\Mod\left(\fO_k[\fG]\right)$ the category of $\fO_k[\fG]$-modules.

\section{Example of $\fG$-action: Regular functions $k$-functor}
\noindent
First we need the following notion.

\begin{definition}
Let $\left(-\right)^{\mathrm{op}}:\Mon \ra \Mon$ be the functor of opposite monoid and let $\fG$ be a monoid $k$-functor. Then the composition $\fG^{\mathrm{op}} = \left(-\right)^{\mathrm{op}}\cdot \fG$ is called \textit{the opposite monoid $k$-functor of $\fG$}.
\end{definition}
\noindent
Let $\fG$ be a monoid $k$-functor. In this section we discuss important example of a $\fO_k[\fG]$-module. Fix a $k$-functor $\fX$ for which $\iMor_k(\fX,\fO_k)$ exists. Recall that by Example \ref{example:regular_functions_as_an_algebra_over_structure_algebra} $\iMor_k\left(\fX,\fO_k\right)$ is $\fO_k$-algebra $k$-functor. Let $\alpha:\fG\times \fX\ra \fX$ be an action of $\fG$ on $\fX$. For every $k$-algebra $A$  we have a map of sets
$$\Mor_A\left(\fX_A,(\fO_k)_A\right) \ni f\mapsto f\cdot \alpha_x\in \Mor_A\left(\fX_A,(\fO_k)_A\right)$$
where $x\in \fG(A)$. From this description it follows that the map $f\mapsto f\cdot \alpha_x$ is a morphism of $A$-algebras. Moreover, note that if $y\in \fG(A)$ is some other $A$-point, then $\left(f\cdot \alpha_x\right)\cdot \alpha_y = f\cdot \alpha_{x\cdot y}$, where $x\cdot y \in \fG(A)$ is a product of $x$ and $y$. Thus the opposite monoid $\fG^{\mathrm{op}}(A)$ acts on the $A$-algebra $\Mor_A\left(\fX_A,\left(\fO_k\right)_A\right)$ by morphism of $A$-algebras. Next for every $A$-algebra $B$ and every point $y\in \fX(B)$ we have
$$(f\cdot \alpha_x)(y) = f\left(\alpha_x(y)\right)$$
This proves the following result.

\begin{proposition}\label{proposition:action_on_regular_k_functor}
Let $\fX$ be a $k$-functor and let $\alpha:\fG\times \fX \ra \fX$ be an action of a monoid $k$-functor $\fG$. Suppose that $\iMor_k\left(\fX,\fO_k\right)$ exists. Then $\fG^{\mathrm{op}}$ acts canonically on $\fO_k$-algebra $k$-functor $\iMor_k\left(\fX,\fO_k\right)$ by morphisms of $\fO_k$-algebras.
\end{proposition}
\noindent
Let us note one important consequence of this result.

\begin{corollary}\label{corollary:action_on_regular_k_functor}
Let $\fG$ be a monoid $k$-functor. The action of $\fG\times \fG^{\mathrm{op}}$ on $\fG$ induces the action of $\fG^{\mathrm{op}}\times \fG$ on $\fO_k$-algebra $k$-functor $\iMor_k\left(\fX,\fO_k\right)$ by morphisms of $\fO_k$-algebras.
\end{corollary}

\section{Linear representations of a monoid $k$-functors}
\noindent
We start the discussion with some results that relates categories $\Mod(k)$ and $\Mod\left(\fO_k\right)$.

\begin{example}\label{example:additive_k_functor}
Let $V$ be a $k$-module. We define a $k$-functor $V_{\mathrm{a}}$. We set
$$V_{\mathrm{a}}(A) = A\otimes_kV,\,V_{\mathrm{a}}(f) = f\otimes_k1_V$$
for every $k$-algebra $A$ and every morphism $f:A\ra B$ of $k$-algebras. Note that $V_{\mathrm{a}}$ is $\fO_k$-module.\\
Suppose that $\phi:V\ra W$ is a morphism of $k$-modules, then we define $\phi_{\mathrm{a}}:V_{\mathrm{a}}\ra W_{\mathrm{a}}$ by formula
$$\phi_{\mathrm{a}}^A = 1_A\otimes_k\sigma$$
for every $k$-algebra. Then $\phi_{\mathrm{a}}$ is a morphism of $\fO_k$-modules.
\end{example}

\begin{proposition}\label{proposition:inclusion_of_k_modules_into_O_k_modules}
The functor $\left(-\right)_{\mathrm{a}}:\Mod\left(k\right)\ra \Mod\left(\fO_k\right)$ is full and faithful.
\end{proposition}
\begin{proof}
Fix $k$-modules $V,W$. Then
$$\Hom_{\fO_k}\left(V_{\mathrm{a}},W_{\mathrm{a}}\right)\ni \sigma \mapsto \sigma^k\in \Hom_k\left(V,W\right)$$
and
$$\Hom_k\left(V,W\right) \ni \phi \mapsto \phi_{\mathrm{a}} \in \Hom_{\fO_k}\left(V_{\mathrm{a}},W_{\mathrm{a}}\right)$$
are mutually inverse bijections. Hence the functor is full and faithful.
\end{proof}

\begin{example}\label{example:general_linear_monoid}
Let $V$ be a $k$-module. We define a $k$-functor $\cL_V$. We set
$$\cL_V(A) = \Hom_A(A\otimes_kV,A\otimes_kV)$$
for every $k$-algebra $A$. Next for every morphism $f:A\ra B$ of $k$-algebras and every morphism $\phi:A\otimes_kV\ra A\otimes_kV$ of $A$-modules we define $\cL_V(f)(\phi)$ as a unique morphism of $B$-modules such that the diagram
\begin{center}
\begin{tikzpicture}
[description/.style={fill=white,inner sep=2pt}]
\matrix (m) [matrix of math nodes, row sep=3em, column sep=3em,text height=1.5ex, text depth=0.25ex] 
{  A\otimes_kV  & A\otimes_kV           \\
   B\otimes_kV  & B\otimes_kV           \\} ;
\path[->,line width=1.0pt,font=\scriptsize]  
(m-1-1) edge node[above] {$ \phi $} (m-1-2)
(m-2-1) edge node[below] {$\cL_V(\phi)  $} (m-2-2)
(m-1-1) edge node[left] {$ f\otimes_k1_V $} (m-2-1)
(m-1-2) edge node[right] {$ f\otimes_k1_V $} (m-2-2);
\end{tikzpicture}
\end{center}
is commutative. Note also that $\cL_V(A)$ is an $A$-algebra. Hence $\cL_V$ is a monoid $k\fO_k$-algebra.
\end{example}

\begin{remark}\label{remark:general_linear_monoid}
Let $V$ be a $k$-module. Proposition \ref{proposition:inclusion_of_k_modules_into_O_k_modules} implies that there are bijective maps that make the square
\begin{center}
\begin{tikzpicture}
[description/.style={fill=white,inner sep=2pt}]
\matrix (m) [matrix of math nodes, row sep=3em, column sep=3em,text height=1.5ex, text depth=0.25ex] 
{ \cL_V(A)   &  \cE nd_{\fO_A}\left(\left(V_{\mathrm{a}}\right)_A,\left(V_{\mathrm{a}}\right)_A\right)           \\
  \cL_V(B)   &  \cE nd_{\fO_B}\left(\left(V_{\mathrm{a}}\right)_B,\left(V_{\mathrm{a}}\right)_B\right)           \\} ;
\path[->,line width=1.0pt,font=\scriptsize]  
(m-1-1) edge node[auto] {$\cong $} (m-1-2)
(m-2-1) edge node[below] {$\cong $} (m-2-2);
\path[->,line width=1.0pt,font=\scriptsize]
(m-1-1) edge node[left] {$ \cL_V(f) $} (m-2-1)
(m-1-2) edge node[auto] {$ \sigma \mapsto \sigma_B $} (m-2-2);
\end{tikzpicture}
\end{center}
commutative for every morphism $f:A\ra B$ of $k$-algebras. This induces an idenitification $\cL_V = \cE nd_{\fO_k}\left(V_{\mathrm{a}}\right)$ of $\fO_k$-algebras.
\end{remark}

\begin{definition}
Let $\fG$ be a monoid $k$-functor. A pair $\left(V,\rho\right)$ consisting of a $k$-module $V$ and a morphism $\rho:\fG\ra \cL_V$ of $k$-monoids is called \textit{a linear representation of $\fG$}.
\end{definition}
\noindent
Next result characterizes linear representations of monoid $k$-functors.

\begin{corollary}\label{corollary:linear_representations_various_characterizations}
Let $\fG$ be a monoid $k$-functor and let $V$ be a $k$-module. Then the following classes are in canonical bijections.
\begin{enumerate}[label=\emph{\textbf{(\arabic*)}}, leftmargin=1.5em]
\item Linear actions of $\fG$ on $V_{\mathrm{a}}$.
\item $\fO_k$-linear actions $\fO_k[\fG]\times V_{\mathrm{a}}\ra V_{\mathrm{a}}$. These are precisely $\fO_k[\fG]$-modules.
\item Morphisms $\fO_k[\fG]\ra \cL_V$ of $\fO_k$-algebras.
\item Morphisms $\fG\ra \cL_V$ of monoids.
\end{enumerate}
\end{corollary}
\begin{proof}
This follows from Proposition \ref{proposition:decription_of_linear_monoid_actions_on_modules}.
\end{proof}

\begin{definition}
Let $\fG$ be a monoid $k$-functor and let $(V,\rho)$, $(W,\delta)$ be its linear representations. A morphism $\phi:V\ra W$ of $k$-modules such that
$$\phi_{\mathrm{a}}^A \cdot \rho(x) = \delta(x) \cdot \phi_{\mathrm{a}}^A$$
for every $k$-algebra $A$ and $x\in \fG(A)$ is called \textit{a morphism of linear representations of $\fG$}.
\end{definition}
\noindent
Next result characterizes morphisms of linear representations of monoid $k$-functor.

\begin{corollary}\label{corollary:characterization_of_morphisms_of_linear_representations}
Let $\fG$ be a monoid $k$-functor and let $V$, $W$ be $k$-modules. Suppose that $\alpha_1:\fG\times V_{\mathrm{a}} \ra V_{\mathrm{a}},\,\alpha_2:\fG\times W_{\mathrm{a}} \ra W_{\mathrm{a}}$ are linear actions of $\fG$. Suppose that $\phi:V\ra W$ is a morphism of $k$-modules. Then the following assertions are equivalent.
\begin{enumerate}[label=\emph{\textbf{(\roman*)}}, leftmargin=1.5em]
\item The square
\begin{center}
\begin{tikzpicture}
[description/.style={fill=white,inner sep=2pt}]
\matrix (m) [matrix of math nodes, row sep=3em, column sep=5em,text height=1.5ex, text depth=0.25ex] 
{  \fG\times V_{\mathrm{a}}   & \fG\times W_{\mathrm{a}} \\
   V_{\mathrm{a}}             & W_{\mathrm{a}} \\} ;
\path[->,line width=1.0pt,font=\scriptsize]  
(m-1-1) edge node[above] {$ 1_{\fG}\times \phi_{\mathrm{a}}  $} (m-1-2)
(m-2-1) edge node[below] {$ \phi_{\mathrm{a}} $} (m-2-2)
(m-1-1) edge node[left] {$ \alpha_1 $} (m-2-1)
(m-1-2) edge node[right] {$ \alpha_2 $} (m-2-2);
\end{tikzpicture}
\end{center}
is commutative.
\item The square
\begin{center}
\begin{tikzpicture}
[description/.style={fill=white,inner sep=2pt}]
\matrix (m) [matrix of math nodes, row sep=3em, column sep=5em,text height=1.5ex, text depth=0.25ex] 
{  \fO_k[\fG] \times V_{\mathrm{a}}   & \fO_k[\fG] \times W_{\mathrm{a}} \\
   V_{\mathrm{a}}                     & W_{\mathrm{a}} \\} ;
\path[->,line width=1.0pt,font=\scriptsize]  
(m-1-1) edge node[above] {$ 1_{\fO_k[\fG]}\times \phi_{\mathrm{a}}  $} (m-1-2)
(m-2-1) edge node[below] {$ \phi_{\mathrm{a}} $} (m-2-2)
(m-1-1) edge node[left] {$ \tilde{\alpha_1} $} (m-2-1)
(m-1-2) edge node[right] {$ \tilde{\alpha_2} $} (m-2-2);
\end{tikzpicture}
\end{center}
is commutative, where $\tilde{\alpha_1}$ and $\tilde{\alpha_2}$ are $\fO_k$-linear actions of $\fO_k[\fG]$ corresponding to $\alpha_1$ and $\alpha_2$, respectively. 
\item For every $k$-algebra $A$ and $x\in \fG(A)$ we have
$$\phi_{\mathrm{a}}^A \cdot \tilde{\rho}_1(x) = \tilde{\rho}_2(x) \cdot \phi_{\mathrm{a}}^A$$
where $\tilde{\rho}_1:\fO_k[\fG] \ra \cL_V$ and $\tilde{\rho}_2:\fO_k[\fG] \ra \cL_W$ are morphism of $\fO_k$-algebras corresponding to $\tilde{\alpha_1}$ and $\tilde{\alpha_2}$, respectively.
\item For every $k$-algebra $A$ and $x\in \fG(A)$ we have
$$\phi_{\mathrm{a}}^A \cdot \rho_1(x) = \rho_2(x) \cdot \phi_{\mathrm{a}}^A$$
where $\rho_1:\fG \ra \cL_V$ and $\rho_2:\fG \ra \cL_W$ are restrictions of $\tilde{\rho_1}$ and $\tilde{\rho_2}$, respectively. This states that $\phi$ is a morphism of linear representations of $\fG$.
\end{enumerate}
\end{corollary}
\begin{proof}
This follows from Proposition \ref{proposition:morphism_of_monoid_modules}.
\end{proof}
\noindent
Let $\fG$ be a monoid $k$-functor. We denote by $\bd{Rep}(\fG)$ its category of linear representations. Note that $\bd{Rep}(\fG)$ is a full subcategory of $\Mod(\fO_k\left[\fG\right])$.

\small
\bibliographystyle{alpha}
\bibliography{../zzz}


\end{document}

