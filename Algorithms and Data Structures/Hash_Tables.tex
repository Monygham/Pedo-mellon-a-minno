\input ../pree.tex

\begin{document}
\title{Hash Tables}
\date{}
\maketitle

\section{Introduction}

\section{Dictionary data type}

\begin{definition}
Let $\cX$ be a set of \textit{items} and let $\cU$ be a set of \textit{keys}. Consider an abstract data type $D$ which dynamically stores a collection of pairs $(k, x)$ where $k\in \cU$ and $x\in \cX$ in such a way that $D$ does not store two pairs having the same key at the same time. Moreover, we assume that $D$ supports the following operations.
\begin{itemize}
\item[] $\textrm{INSERT}\big((k,x)\big)$\\
Adds pair $(k,x)$ into $D$ if there is no other pair stored in $D$ with $k$ as a first entry.
\item[] $\textrm{DELETE}\big(k\big)$\\
Removes a pair with $k$ as a first entry from $D$ if such pair is stored in $D$.
\item[] $\textrm{SEARCH}\big(k\big)$\\
Returns $x$ if a pair $(k,x)$ is stored in $D$. Otherwise returns \textit{nil}. 
\end{itemize}
An abstract data type with these properties and interface is called \textit{an associative array} or \textit{a dictionary}.
\end{definition}

\begin{definition}
Let $\cX$ and $\cU$ be sets. \textit{Dictionary problem for $\cX$ and $\cU$} is the task of designing a dictionary with $\cX$ as the set of items and $\cU$ as the set of keys.
\end{definition}

\section{Hash functions and hash tables with chaining}
\noindent
In this section we introduce the important notion of a hash function and we use it to solve dictionary problem.

\begin{definition}
Let $\cU$ be a set. \textit{A hash function} is a mapping $h:\cU\ra \{0,1,...,m-1\}$ where $m\in \NN_+$.
\end{definition}

\begin{definition}
Let $h:\cU \ra \{0,1,...,m-1\}$ be a hash function. \textit{A collision} is a pair of keys $k_1$, $k_2\in \cU$ such that $k(k_1) = h(k_2)$.
\end{definition}
\noindent
Using hash functions one can solve dictionary problem. We introduce the notion which describes this solution. 

\begin{definition}
Let $\cU$ and $\cX$ be sets. Let $h:\cU\ra \{0,1,...,m-1\}$ be a hash function for some $m\in \NN_+$. We consider an $m$-element array $D_h$ such that $D_h[l]$ is a linked list storing values from $\cU\times \cX$ for every $l\in \{0,1,...,m-1\}$. We describe dictionary operations.
\begin{itemize}
\item[] $\textrm{INSERT}_h\big((k,x)\big)$\\
Searches for a pair with the first entry $k$ in the list $D_h[h(k)]$. If such pair is found, then replaces its second entry with $x$. If such pair is not found, then inserts pair $(k,x)$ to the linked list $D_h[h(k)]$ as its new head.
\item[] $\textrm{DELETE}_h\big(k\big)$\\
Deletes a pair with first entry $k$ from the linked list $D_h[h(k)]$.
\item[] $\textrm{SEARCH}_h\big(k\big)$\\
Searches for the pair with the first entry $k$ in the list $D_h[h(k)]$. If such pair is found, then returns its second entry. Otherwise returns \textit{nil}. 
\end{itemize}
Then $D_h$ together with these operations is a solution of dictionary problem for $\cU$ and $\cX$. We call it \textit{the hash table with collisions resolved by chaining for $h$}.
\end{definition}

\section{Analysis of hash tables with chaining under simple uniform hashing}
\noindent
We start by introducing important stochastic property of hash functions.

\begin{definition}
Let $\cU$ be a measurable space and let $h:\cU\ra \{0,1,...,m-1\}$ be a measurable hash function for some $m\in \NN_+$. Suppose that $\mu$ is a probability distribution on $\cU$. Fix a probability space $\left(\Omega,\cF,P\right)$ and a sequence of independent random variables $K_1,...,K_n:\Omega\ra \cU$ with distribution $\mu$ for some $n\in \NN_+$. Consider the following assertions.
\begin{enumerate}[label=\textbf{(\arabic*)}, leftmargin=3.0em]
\item Event
$$\cK = \big\{\forall_{i,j\in \{1,...,n\},i\neq j}\,K_i\neq K_j\big\}$$
is of positive probability.
\item Let $K:\Omega\ra \cU$ be a random variable with distribution $\mu$ and independent of $K_1,...,K_n$. Then
$$P\big(h(K) = h(K_i)\,\big|\,\cK\big) =\frac{1}{m}$$
for every $i\in \{1,...,n\}$.
\end{enumerate}
If assertions above hold for every probability space $(\Omega,\cF,P)$, every $n\in \NN_+$ and every sequence $K_1,...,K_n:\Omega\ra \cU$ of independent random variables with distribution $\mu$, then $h$ is \textit{a simple uniform hashing with respect to $\mu$}.
\end{definition}
\noindent
Now let us give two examples of hash functions satisfying simple uniform hashing with respect to canonical probability distributions on their spaces of keys.

\begin{example}\label{example:real_interval_with_floor_hash_function_as_an_example_of_suh}
Let $\cU = [0,m]$ for some $m\in \NN_+$. Then $\cU$ is a measurable space with respect to Borel algebra $\cB\big([0,m]\big)$. We define a hash function $h:\cU\ra \{0,1,...,m-1\}$ by formula
$$h(x) = \lfloor x \rfloor$$
Then $h$ is a simple uniform hashing with respect to the normalization of Lebesgue measure on $[0,m]$.
\end{example}

\begin{example}\label{example:interval_of_integers_with_modulo_hash_function_as_an_example_of_suh}
Let $\cU = \{0,1,...,m^2-1\}$ for some $m\in \NN_+$. Then $\cU$ is a measurable space with respect to the power algebra $\cP\big(\{0,1,...,m^2-1\}\big)$. Consider $\cU$ as a probability space with respect to the uniform distribution $\mu$. We define a hash function $h:\cU \ra \{0,1,...,m-1\}$ by formula
$$h(x) = x\,\mathrm{mod}\,m$$
We verify that $h$ is a simple uniform hashing with respect to $\mu$. Fix a probability space $(\Omega,\cF,P)$ and $n\in \NN_+$. Suppose first that $K_1,...,K_n:\Omega\ra \cU$ are independent random variables with distribution $\mu$. If 
$$\cK =\big\{\forall_{i,j\in \{1,...,n\},i\neq j}\,K_i\neq K_j\big\}$$
then
$$P(\cK) = \frac{m^2 \cdot (m^2 - 1) \cdot ... \cdot (m^2 - n + 1)}{m^{2n}} > 0$$
Next suppose that $K:\Omega\ra \cU$ is a random variable with distribution $\mu$ which is independent of $K_1,...,K_n$. Then for fixed $i\in \{1,...,n\}$ we have 
$$P\big(h(K) = h(K_i)\,\big|\,\cK\big) = \frac{P\big(\big\{h(K) = h(K_i) \big\}\cap \cK\big)}{P(\cK)} =$$
$$= m\cdot \frac{m^2 \cdot (m^2 - 1)\cdot ... \cdot (m^2 - n + 1)}{ m^{2n+2}} \cdot \bigg(\frac{m^2 \cdot (m^2 - 1) \cdot ... \cdot (m^2 - n + 1)}{m^{2n}}\bigg)^{-1} =  \frac{1}{m}$$
This completes the verification that $h$ is a simple uniform hashing with respect to $\mu$.
\end{example}
\noindent
In order to analyze expected costs of dictionary operations for hash tables with chaining we introduce natural probabilistic model. 

\begin{setup}[Probabilistic model for hash tables with chaining]\label{setup:probabilistic_model_for_simple_uniform_hashing}
We fix a measurable space of keys $\cU$ and a set $\cX$ of items. We consider a measurable hash function $h:\cU\ra \{0,1,...,m-1\}$ and a probability distribution $\mu$ on $\cU$. We also fix a probability space $(\Omega,\cF,P)$ and $n\in \NN_+$. Let $K_1,...,K_n:\Omega \ra \cU$ be independent random variables with distribution $\mu$. Write
$$\cK = \big\{\forall_{i,j\in \{1,...,n\},i\neq j}\,K_i\neq K_j\big\}$$
and suppose that $K:\Omega\ra \cU$ is a random variable with distribution $\mu$ and independent from $K_1,...,K_n$. We assume that pairs with keys $K_1(\omega),...,K_n(\omega)$ for $\omega \in \cK$ were consecutively inserted into initially empty $D_h$. Under this assumption we denote by $\bd{search}_h(K)$ the function $\cK\ra \NN$ which for every $\omega \in \cK$ returns the cost (in terms of the number of basic operations) of operation 
$$\mathrm{SEARCH}_h(K(\omega))$$
Similarly for 
$$\mathrm{DELETE}_h(K(\omega))$$ 
and (for fixed $x\in \cX$)
$$\mathrm{INSERT}_h\left((K(\omega),x)\right)$$
we define functions $\bd{delete}_h(K):\cK\ra \NN$ and $\bd{insert}_h\big((K,x)\big):\cK\ra \NN$.
\end{setup}
\noindent
In the remaining part of this section we work under probabilistic model described in Setup \ref{setup:probabilistic_model_for_simple_uniform_hashing}. We have the following fundamental result.

\begin{theorem}\label{theorem:simple_uniform_hashing_expected_cost_for_search}
Let $h$ be a simple uniform hashing with respect to$\mu$. Then $\bd{search}_h(K):\cK\ra \NN$ is measurable and
$$\mathbb{E}\,\bd{search}_h(K) = \int_{\cK}\bd{search}_h(K)\,dP_{\cK} \leq 1 + \frac{n}{m}$$
\end{theorem}
\begin{proof}
First we introduce certain notation. We consider events
$$W_{i} = \big\{h\left(K_i\right) = h\left(K\right)\big\},\,Z_i = \big\{K = K_i\big\}\cap \cK,\,Z = \bigcup_{i=1}^nZ_i$$
for $i\in \{1,...,n\}$. Fix $\omega \in \cK$. For 
$$\mathrm{SEARCH}_h(K(\omega))$$
we first calculate $h(K(\omega))$. This is a single basic operation. Next if $\omega \in \cK\setminus Z$, then we run through the list $D_h[h(K(\omega))]$ with length equal to the number of keys in $\big\{K_1(\omega),...,K_n(\omega)\big\}$ mapped by $h$ to $h(K(\omega))$. This is the case of the unsuccessful search. Otherwise, if $\omega \in Z_i$ then we run through the initial segment of the list $D_h[h(K(\omega))]$ which consists of elements from the set $\big\{K_i(\omega),K_{i+1}(\omega),...,K_n(\omega)\big\}$ mapped by $h$ to $h(K(\omega))$. This is the case when the search is successful. Thus
$$\bd{search}_h\left(K\right) = \underbrace{1}_{\mathrm{computation\,of\,the\,hash}} + \underbrace{\chi_{\cK\setminus Z}\cdot\sum_{i=1}^n \chi_{W_{i}}}_{\mathrm{unsuccessful\,search}} + \underbrace{\sum_{i=1}^n\chi_{Z_i}\cdot\sum_{j = i}^n \chi_{W_{j}}}_{\mathrm{succesfull\,search}}$$
Hence $\bd{search}_h\left(K\right):\cK\ra \NN$ is measurable. Moreover, note that
$$\bd{search}_h\left(K\right) = 1 + \chi_{\cK\setminus Z}\cdot \sum_{i=1}^n \chi_{W_{i}} + \sum_{i=1}^n\chi_{Z_i}\cdot\sum_{j = i}^n \chi_{W_{j}} \leq 1 + \chi_{\cK\setminus Z}\cdot \sum_{i=1}^n \chi_{W_{i}} + \sum_{i=1}^n\chi_{Z_i}\cdot\sum_{j = 1}^n \chi_{W_{j}} = 1 +  \sum_{i=1}^n \chi_{W_{i}}$$
and hence
$$\mathbb{E}\,\bd{search}_h\left(K\right) \leq 1 +  \sum_{i=1}^n \mathbb{E}\,\chi_{W_{i}} = 1 + \sum_{i=1}^nP_{\cK}(W_i) = 1 + \sum_{i=1}^n\frac{P(W_i\cap \cK)}{P(\cK)} = 1 + \frac{n}{m}$$
The last inequality is a consequence of the fact that $h$ is a simple uniform hashing with respect to $\mu$.
\end{proof}
\noindent
Using essentially the same method (we omit the proof) one derives the following results.

\begin{theorem}\label{theorem:simple_uniform_hashing_expected_costs_of_delete_and_insert}
Let $h$ be a simple uniform hashing with respect to $\mu$. Fix $x$ in $\cX$. Then both functions $\bd{delete}_h(K),\bd{insert}_k\left((K,x)\right):\cK\ra \NN$ are measurable and
$$\mathbb{E}\,\bd{delete}_h(K), \mathbb{E}\,\bd{insert}_h((K,x))  \leq 1 + \frac{n}{m}$$
\end{theorem}
\noindent
These results have the following consequence.

\begin{corollary}\label{corollary:constant_expected_costs_of_dictionary_operations_for_simple_uniform_hashing}
Let $h$ be a simple uniform hashing with respect to $\mu$. Suppose that there exists a constant $c\in \RR_+$ such that $n\leq c\cdot m$. Then the expected costs of all dictionary operations for $D_h$ are $\cO(1)$.
\end{corollary}
\begin{proof}
The assertion follows from Theorems \ref{theorem:simple_uniform_hashing_expected_cost_for_search} and \ref{theorem:simple_uniform_hashing_expected_costs_of_delete_and_insert} and the inequality $1 + \frac{n}{m} \leq 1 + c$.
\end{proof}

\section{Universal hashing families}


\end{document}