\input ../pree.tex

\begin{document}
\title{Hash Tables}
\date{}
\maketitle

\section{Introduction}

\section{Dictionary data type}

\begin{definition}
Let $\cX$ be a set of \textit{items} and let $\cU$ be a set of \textit{keys}. Consider an abstract data type $D$ which dynamically stores a collection of pairs $(k, x)$ where $k\in \cU$ and $x\in \cX$ in such a way that $D$ does not store two pairs having the same key at the same time. Moreover, we assume that $D$ supports the following operations.
\begin{itemize}
\item[] \textrm{INSERT}$\big(D,(k,x)\big)$\\
Adds pair $(k,x)$ into $D$ if there is no other pair stored in $D$ with $k$ as a first entry.
\item[] \textrm{DELETE}$\big(D,k\big)$\\
Removes a pair with $k$ as a first entry from $D$ if such pair is stored in $D$.
\item[] \textrm{SEARCH}$\big(D,k\big)$\\
Returns $x$ if a pair $(k,x)$ is stored in $D$. Otherwise returns \textit{nil}. 
\end{itemize}
An abstract data type with these properties and interface is called \textit{an associative array} or \textit{a dictionary}.
\end{definition}

\begin{definition}
Let $\cX$ and $\cU$ be sets. \textit{Dictionary problem for $\cX$ and $\cU$} is the task of designing a dictionary with $\cX$ as the set of items and $\cU$ as the set of keys.
\end{definition}

\section{Hash functions}
\noindent
In this section we introduce the important notion of a hash function and we discuss some probabilistic properties of such functions.

\begin{definition}
Let $\cU$ be a set. \textit{A hash function} is a mapping $h:\cU\ra \{0,1,...,m-1\}$ where $m\in \NN_+$.
\end{definition}

\begin{definition}
Let $h:\cU \ra \{0,1,...,m-1\}$ be a hash function. \textit{A collision} is a pair of keys $k_1$, $k_2\in \cU$ such that $k(k_1) = h(k_2)$.
\end{definition}

\begin{definition}
Let $X$ be a set and let $n\in \NN_+$. Then a set
$$X^{\wedge n} = \big\{(x_1,...,x_n)\in X^{n}\,\big|\,\forall_{1\leq i < j \leq n}\,x_i\neq x_j\big\}$$
is called \textit{the antisymmetric cartesian power of $X$}.
\end{definition}

\begin{definition}
Let $\cU$ be a measurable space. We consider $\cU^{\wedge n}$ as the measurable subspace of the product space $\cU^{n}$. Suppose that $P$ is a probability distribution on $\cU^{\wedge n}$. Let $h:\cU\ra \{0,1,...,m-1\}$ be a measurable hash function for some $m\in \NN_+$. Assume that the following assertions hold.
\begin{enumerate}[label=\textbf{(\arabic*)}, leftmargin=*]
\item $$P\big((k_1,...,k_n)\in \cU^{\wedge n}\,\big|\,h(k_i) = l\big) = \frac{1}{m}$$
for every element $i\in \{1,...,n\}$ and every $l\in \{0,1,...,m-1\}$.
\item $$P\big((k_1,...,k_n)\in \cU^{\wedge n}\,\big|\,h(k_i) = h(k_j)\big) \leq \frac{1}{m}$$
for every pair of distinct elements $i,j\in \{1,...,n\}$.
\end{enumerate}
Then $h$ is \textit{a simple uniform hashing with respect to $P$}.
\end{definition}

\begin{example}\label{example:real_interval_with_floor_hash_function_as_an_example_of_suh}
Let $\cU = [0,m]$ for some $m\in \NN_+$. Then $\cU$ is a measurable space with respect to Borel algebra $\cB\big([0,m]\big)$. We define a hash function $h:\cU\ra \{0,1,...,m-1\}$ by formula
$$h(x) = \lfloor x \rfloor$$
Then $h$ is a simple uniform hashing with respect to the normalization of $n$-dimensional Lebesgue measure on $[0,m]^{\wedge n}$.
\end{example}

\begin{example}\label{example:interval_of_integers_with_modulo_hash_function_as_an_example_of_suh}
Let $\cU = \{0,1,...,m^2-1\}$ for some $m\in \NN_+$. Then $\cU$ is a measurable space with respect to the power algebra $\cP\big(\{0,1,...,m^2-1\}\big)$. Consider $\cU^{\wedge n}$ as a probability space with respect to the uniform distribution $P$. We define a hash function $h:\cU \ra \{0,1,...,m-1\}$ by formula
$$h(x) = x\,\mathrm{mod}\,m$$
For $i\in \{1,...,n\}$ and $l\in \{0,1,...,m-1\}$ we have
$$P\big((k_1,...,k_n)\in \cU^{\wedge n}\,\big|\,h(k_i) = l\big) = \frac{m\cdot (m^2 - 1)\cdot (m^2 - 2)\cdot ...\cdot (m^2 - n + 1)}{m^2\cdot (m^2 - 1)\cdot ...\cdot (m^2 - n + 1)} = \frac{1}{m}$$
Fix distinct $i, j \in \{1,...,n\}$ and $l\in \{0,1,...,m-1\}$. Note that
$$P\big((k_1,...,k_n)\in \cU^{\wedge n}\,\big|\,h(k_i) = l,h(k_j) = l\big) = \frac{m\cdot (m-1)\cdot (m^2 - 2) \cdot ...\cdot (m^2 - n + 1)}{m^2\cdot (m^2 - 1)\cdot ...\cdot (m^2 - n + 1)} = \frac{1}{m\cdot (m+1)}$$
Hence
$$P\big((k_1,...,k_n)\in \cU^{\wedge n}\,\big|\,h(k_i) = h(k_j)\big) = \sum_{l=0}^{m-1}P\big(h(k_i) = l,h(k_j) = l\,\big|\,(k_1,...,k_n)\in \cU^{\wedge n}\big) =$$
$$= \frac{m}{m\cdot (m + 1)} = \frac{1}{m+1}\leq \frac{1}{m}$$
Thus $h$ is a simple uniform hashing with respect to $P$.
\end{example}

\section{Hash tables with chaining as a solution to dictionary problem}
\noindent
In this section we present the solution to the dictionary problem and discuss its efficiency.

\begin{definition}
Let $\cU$ and $\cX$ be sets. Let $h:\cU\ra \{0,1,...,m-1\}$ be a hash function for some $m\in \NN_+$. We consider an $m$-element array $D_h$ such that $D_h[l]$ is a linked list storing values from $\cU\times \cX$ for every $l\in \{0,1,...,m-1\}$. We describe dictionary operations.
\begin{itemize}
\item[] \textrm{INSERT}$\big(D_h,(k,x)\big)$\\
Inserts pair $(k,x)$ to the linked list $D_h[h(k)]$ as its new head.
\item[] \textrm{DELETE}$\big(D_h,k\big)$\\
Deletes a pair with first entry $k$ from the linked list $D_h[h(k)]$.
\item[] \textrm{SEARCH}$\big(D_h,k\big)$\\
Searches for the pair with the first entry $k$ in the list $D_h[h(k)]$. If such pair is found, then returns its second entry. Otherwise returns \textit{nil}. 
\end{itemize}
Then $D_h$ together with these operations is a solution of dictionary problem for $\cU$ and $\cX$. We call it \textit{the hash table with collisions resolved by chaining for $h$}.
\end{definition}
\noindent
Suppose that $\cU$ and $\cX$ are sets. Let $h:\cU\ra \{0,1,...,m-1\}$ be a hash function. Consider the hash table $D_h$. Fix $l\in \{0,1,...,m-1\}$ and $n\in \NN_+$. Suppose that pairs $(k_1,x_1)$,...,$(k_n,x_n)$ for $(k_1,...,k_n)\in \cU^{\wedge n}$ and $x_1,...,x_n\in \cX$ are consecutively inserted to initially empty $D_h$. After these sequence of insertions is performed the length of the linked list stored in $D_h[l]$ is equal to the cardinality of the set $\big\{i\,\big|\,h(k_i) = l\big\}$. We denote the function
$$\cU^{\wedge n}\ni (k_1,...,k_n)\ra \big|\big\{i\,\big|\,h(k_i) = l\big\}\big|\in \NN$$
by $coll_l$.

\begin{theorem}\label{theorem:simple_uniform_hashing_expected_number_of_items_in_the_slot}
Let $\cU$ be a measurable space and let $\cX$ be a set. Let $h:\cU\ra \{0,1,...,m-1\}$ be a measurable hash function and fix $n\in \NN_+$. Then the following assertions hold.
\begin{enumerate}[label=\emph{\textbf{(\arabic*)}}, leftmargin=*]
\item The function $coll_l:\cU^{\wedge n}\ra \NN$ is measurable for every $l\in \{0,1,...,m-1\}$.
\item If $h$ is a simple uniform hashing with respect to some probability distribution $P$ on $\cU^{\wedge n}$, then
$$\mathbb{E}\,coll_l = \int_{\cU^{\wedge n}}coll_l\,dP = \frac{n}{m}$$
for every $l\in \{0,1,...,m-1\}$.
\end{enumerate}
\end{theorem}
\begin{proof}
Suppose that $X_i$ is the indicator function of the measurable set $\big\{(k_1,...,k_n)\in \cU^{\wedge n}\,\big|\,h(k_i) = l\big\}$. Then
$$coll_l = \sum_{i=1}^nX_i$$
This proves that $slot_l$ is measurable. If in addition $h$ is a simple uniform hashing with respect to some probability distribution $P$ on $\cU^{\wedge n}$, then
$$\mathbb{E}\,coll_l = \mathbb{E}\left(\sum_{i=1}^nX_i\right) = \sum_{i=1}^n\mathbb{E}X_i = \sum_{i=1}^nP\big((k_1,...,k_n)\in \cU^{\wedge n}\,\big|\,h(k_i) = l\big) = \frac{n}{m}$$
\end{proof}
\noindent
Suppose that $\cU$ and $\cX$ are sets. Let $h:\cU\ra \{0,1,...,m-1\}$ be a hash function. Consider the hash table $D_h$. Fix $n\in \NN_+$ and $i\in \{1,...,n\}$. Suppose that pairs $(k_1,x_1)$,...,$(k_n,x_n)$ for $(k_1,...,k_n)\in \cU^{\wedge n}$ and $x_1,...,x_n\in \cX$ are consecutively inserted to initially empty $D_h$. After these sequence of insertions is performed the number of elements in the linked list stored in $D_h[k(k_i)]$ which precede $(k_i,x_i)$ is equal to the cardinality of the set $\big\{j\in \{i+1,...,n\}\,\big|\,h(k_j) = h(k_i)\big\}$. We denote the function
$$\cU^{\wedge n}\ni (k_1,...,k_n)\ra \big|\big\{i\,\big|\,h(k_i) = l\big\}\big|\in \NN$$
by $coll_{< i}$.

\begin{theorem}\label{theorem:simple_uniform_hashing_expected_number_of_items_after_given_item}
Let $\cU$ be a measurable space and let $\cX$ be a set. Let $h:\cU\ra \{0,1,...,m-1\}$ be a measurable hash function and fix $n\in \NN_+$. Fix $i\in \{1,...,n\}$. Then the following assertions hold.
\begin{enumerate}[label=\emph{\textbf{(\arabic*)}}, leftmargin=*]
\item The function 
$$\cU^{\wedge n} \ni (k_1,...,k_n) \mapsto \#\{j\,|\,i\leq j\leq n\mbox{ and }h(k_i) = h(k_j)\}\in \NN$$
is measurable.
\item If $h$ is a simple uniform hashing with respect to some probability distribution $P$ on $\cU^{\wedge n}$, then
$$\mathbb{E}\#\{j\,|\,i\leq j\leq n\mbox{ and }h(k_i) = h(k_j)\} = \int_{\cU^{\wedge n}}\#D_h[l]\,dP = \frac{n}{m}$$
for every $l\in \{0,1,...,m-1\}$.
\end{enumerate}
\end{theorem}
\begin{proof}
Suppose that $X_i$ is the indicator function of the measurable set $\big\{(k_1,...,k_n)\in \cU^{\wedge n}\,\big|\,h(k_i) = l\big\}$. Then
$$\# D_h[l] = \sum_{i=1}^nX_i$$
This proves that $\#D_h[l]$ is measurable. If in addition $h$ is a simple uniform hashing with respect to some probability distribution $P$ on $\cU^{\wedge n}$, then
$$\mathbb{E}\# D_h[l] = \mathbb{E}\left(\sum_{i=1}^nX_i\right) = \sum_{i=1}^n\mathbb{E}X_i = \sum_{i=1}^nP\big((k_1,...,k_n)\in \cU^{\wedge n}\,\big|\,h(k_i) = l\big) = \frac{n}{m}$$
\end{proof}





\end{document}