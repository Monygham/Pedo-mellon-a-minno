\input ../pree.tex

\begin{document}

\title{Linearly Reductive Groups}
\date{}
\maketitle

\section{Motivation -- linear representations of compact topological groups}
\noindent
In this section we fix a compact topological group $\bd{G}$.
Assume that $\rho:\bd{G}\ra \mathrm{GL}_n(\CC)$ is a continuous homomorphism i.e. a complex, $n$-dimensional linear representation of $\bd{G}$. For every $g\in \bd{G}$ we get a matrix
$$\rho(g)=\left[c_{ij}(g)\right]_{1\leq i, j \leq n}$$
For $i$, $j$ function $c_{ij}:\bd{G}\ra \CC$ is a continuous complex valued function. Alternatively suppose that $\{e_1,e_2,...,e_n\}$ is the standard basis of $\CC^n$ on which $\mathrm{GL}_n(\CC)$ act. Then $c_{ij}$ is equal to a function 
$$\bd{G}\ni g \mapsto \langle g\cdot e_i, e_j\rangle \in \CC$$
Fix now $g_1$, $g_2\in \bd{G}$ and note that
$$\left[c_{ij}(g_2\cdot g_1)\right]_{1\leq i,j\leq n}  = \rho(g_2\cdot g_1) = \rho(g_2)\cdot \rho(g_1) = \left[\sum_{k=1}^nc_{ik}(g_2)\cdot c_{kj}(g_1)\right]_{1\leq i,j\leq n}$$
Hence 
$$c_{ij}(g_2\cdot g_1) = \sum_{k=1}^nc_{ik}(g_2)\cdot c_{kj}(g_1)$$
for every $1\leq i,j\leq n$. This implies that $\sum_{1\leq i,j\leq n}\CC\cdot c_{ij}\subseteq \cL^2(\bd{G},\CC)$ is a linear $\bd{G}\times \bd{G}^{\mathrm{op}}$-subrepresentation of the regular representation $\cL^2(\bd{G},\CC)$. We call it \textit{the matrix coefficients of $\rho$}. 

\section{Matrix coefficients of a representation}

\begin{proposition}\label{proposition:matrixcoefficients}
Let $\fX$ be a monoid $k$-functor and let $V$ be a finitely generated, projective $k$-module. Fix a morphism of monoids $\rho:\fX \ra \cL_V$. Fix $k$-algebra $A$ and elements $v\in A\otimes_kV$, $w\in A\otimes_kV^{\vee}$. For every $A$-algebra $B$ and $x\in \fX_A(B)$ we consider the formula
$$c_{v,w}(x) = \langle \rho_A(x) \cdot v_B, w_B \rangle$$
Then $c_{v,w}$ defines a regular function on $\fX_A$ for every $k$-algebra $A$.
\end{proposition}
\begin{proof}
Suppose that $f:B\ra C$ is a morphism of $A$-algebras and pick $x\in \fX_A(B)$. Since $\rho_A$ is natural and $w:A\otimes_kV\ra A$ is a morphism of $A$-modules, we derive that the diagram
\begin{center}
\begin{tikzpicture}
[description/.style={fill=white,inner sep=2pt}]
\matrix (m) [matrix of math nodes, row sep=3em, column sep=5em,text height=1.5ex, text depth=0.25ex] 
{V_B & V_B & B \\
 V_C & V_C & C\\} ;
\path[->,line width=1.0pt,font=\scriptsize]  
(m-2-1) edge node[below] {$\rho_A\big(\fX_A(f)(x)\big) $} (m-2-2)
(m-1-2) edge node[right] {$1_{V_A}\otimes_Af $} (m-2-2) 
(m-1-1) edge node[left]  {$1_{V_A}\otimes_Af $} (m-2-1)
(m-1-1) edge node[above] {$\rho_A(x) $} (m-1-2)
(m-1-2) edge node[above] {$w_B $} (m-1-3)
(m-2-2) edge node[below] {$w_C $} (m-2-3)
(m-1-3) edge node[right] {$f $} (m-2-3);
\end{tikzpicture}
\end{center}
is commutative. Hence 
$$c_{v,w}\big(\fX_A(f)(x)\big)=\langle \rho_A\big(\fX_A(f)(x)\big)\cdot v_C,w_C\rangle=f\big(\langle \rho_A(x)\cdot v_B, w_B\rangle \big)=f\big(c_{v,w}(x)\big)$$
and this implies that $c_{v,w}:\fX_A\ra \mathbb{A}^1_A$ is natural.
\end{proof}

\begin{definition}
Let $\fX$ be a monoid $k$-functor and let $(V,\rho)$ be its representation with finitely generated, projective underlying $k$-module $V$. Fix $k$-algebra $A$ and elements $v\in A\otimes_kV$, $w\in A\otimes_kV^{\vee}$. Then the regular function $c_{v,w}$ on $\fX_A$ is called \textit{the matrix coefficient of $v$ and $w$}.
\end{definition}

\begin{proposition}\label{proposition:matrixcoefficientsnatural}
Let $\fX$ be a monoid $k$-functor and let $(V,\rho)$ be its representation with finitely generated projective underlying $k$-module $V$. Then the following assertions holds.
\begin{enumerate}[label=\emph{\textbf{(\arabic*)}}, leftmargin=1.5em]
\item For every $k$-algebra $A$ map
$$\left(A\otimes_kV\right)\times \left(A\otimes_kV^{\vee}\right)\ni (v,w)\mapsto c_{v,w}\in \Mor_A\left(\fX_A,\mathbb{A}^1_A\right)$$
is $A$-bilinear.
\item The collection of maps
$$\big\{\left(A\otimes_kV\right)\times \left(A\otimes_kV^{\vee}\right)\ni (v,w)\mapsto c_{v,w}\in \Mor_A\left(\fX_A,\mathbb{A}^1_A\right)\big\}_{A\in \Alg_k}$$
gives rise to a morphism of $k$-functors
\begin{center}
\begin{tikzpicture}
[description/.style={fill=white,inner sep=2pt}]
\matrix (m) [matrix of math nodes, row sep=3em, column sep=3em,text height=1.5ex, text depth=0.25ex] 
{ V_{\mathrm{a}}\times V^{\vee}_{\mathrm{a}} &  \iMor_k\left(\fX,\mathbb{A}^1_k\right) \\} ;
\path[->,line width=1.0pt,font=\scriptsize]  
(m-1-1) edge node[above] {$ $} (m-1-2);
\end{tikzpicture}
\end{center}
\end{enumerate}
\end{proposition}
\begin{proof}
We left the proof of \textbf{(1)} to the reader.\\
We prove \textbf{(2)}. Consider $k$-algebra $A$ and an $A$-algebra $B$ with structural morphism $f:A\ra B$. Fix $v\in A\otimes_kV$, $w\in A\otimes_kV^{\vee}$. We prove that restriction of $c_{v,w}:\fX_A\ra \mathbb{A}^1_A$ to the category $\Alg_B$ is $c_{v_B,w_B}$. For this pick a $B$-algebra $C$ and an element $x\in \fX_A(C)=\fX_B(C)$. Note that
$$c_{v,w}(x)= \langle \rho_A(x)\cdot v_C,w_C \rangle =  \langle \rho_B(x)\cdot v_C,w_C\rangle = \langle \rho_B(x)\cdot (v_B)_C,(w_B)_C\rangle = c_{v_B,w_B}(x)$$
and hence ${c_{v,w}}_{\mid \Alg_B}=c_{v_B,w_B}$. Consider the square
\begin{center}
\begin{tikzpicture}
[description/.style={fill=white,inner sep=2pt}]
\matrix (m) [matrix of math nodes, row sep=4em, column sep=3em,text height=1.5ex, text depth=0.25ex] 
{V_{\mathrm{a}}(A)\times V^{\vee}_{\mathrm{a}}(A) & \iMor_k\left(\fX,\mathbb{A}^1\right)(A)  \\
 V_{\mathrm{a}}(B)\times V^{\vee}_{\mathrm{a}}(B) & \iMor_k\left(\fX,\mathbb{A}^1\right)(B)  \\} ;
\path[->,line width=1.0pt,font=\scriptsize]  
(m-2-1) edge node[below] {$ $} (m-2-2)
(m-1-2) edge node[right] {$\iMor_k(\fX,\mathbb{A}^1)(f) $} (m-2-2) 
(m-1-1) edge node[left]  {$V_a(f)\times V^{\vee}_a(f)$} (m-2-1)
(m-1-1) edge node[above] {$ $} (m-1-2);
\end{tikzpicture}
\end{center}
in which both horizontal arrows are given by formula $(v,w)\mapsto c_{v,w}$. We proved that the square commutes. Since $f$ is an arbitrary morphism of $k$-algebras, we conclude the assertion.
\end{proof}

\begin{corollary}\label{corollary:matrixcoefficientsnatural}
Let $\fX$ be a monoid $k$-functor and let $(V,\rho)$ be its representation with finitely generated projective underlying $k$-module $V$. Then there exists a morphism of $k$-functors
\begin{center}
\begin{tikzpicture}
[description/.style={fill=white,inner sep=2pt}]
\matrix (m) [matrix of math nodes, row sep=3em, column sep=3em,text height=1.5ex, text depth=0.25ex] 
{ \left(V \otimes_k V^{\vee}\right)_{\mathrm{a}} &  \iMor_k\left(\fX,\mathbb{A}^1_k\right) \\} ;
\path[->,line width=1.0pt,font=\scriptsize]  
(m-1-1) edge node[above] {$c $} (m-1-2);
\end{tikzpicture}
\end{center}
given by formula
$$\left(A\otimes_kV\right)\otimes_A\left(A\otimes_kV^{\vee}\right)\ni (v,w)\mapsto c_{v,w}\in \Mor_A\left(\fX_A,\mathbb{A}^1_A\right)$$
Moreover, $c$ is a morphism of $k$-functors equipped with $\fX \times \fX^{\mathrm{op}}$-actions.
\end{corollary}
\begin{proof}
The first part is an immediate consequence of Proposition \ref{proposition:matrixcoefficientsnatural}. We prove that $c$ is a morphism of $k$-functors equipped with $\fX\times \fX^{\mathrm{op}}$-actions. For this we fix a $k$-algebra $k$ and elements $v\in A\otimes_kV$, $w\in A\otimes_kV^{\vee}$. Pick a morphism of $k$-algebras $f:A\ra B$, $(y,z)\in \fX(A)\times \fX(A)^{\mathrm{op}}$ and $x\in \fX_A(B)$. Then we have 
$$c_{\rho(y)\cdot v,w\cdot \rho(z)}(x) = \big\langle \rho_A(x)\cdot \left(\rho(y)\cdot v\right)_B, \left(w\cdot \rho(z)\right)_B \big\rangle =$$
$$= \big\langle \rho_A(x)\cdot \rho_A(\left(\fX_A(f)(y)\right))\cdot v_B, w_B\cdot \rho_A\left(\fX_A(f)(z)\right) \big\rangle = w_B\big(\rho_A\left(\fX_A(f)(z)\right)\cdot \rho_A(x)\cdot \rho_A\left(\fX_A(f)(y)\right)\cdot v_B \big)=$$
$$= w_B\big(\rho_A\left(\fX_A(f)(z) \cdot x \cdot \fX_A(f)(y)\right)\cdot v_B \big) = \big\langle \rho_A\left(\fX_A(f)(z) \cdot x \cdot \fX_A(f)(y)\right)\cdot v_B, w_B \big\rangle =  $$
$$= c_{v,w}\big(\fX_A(f)(z) \cdot x \cdot \fX_A(f)(y)\big)$$
and hence $c$ is a morphism of $k$-functors equipped with actions of $\fX\times \fX^{\mathrm{op}}$.
\end{proof}






\section{Algebra of regular functions of a \textit{k}-functor}

\begin{example}
For every $k$-algebra $A$ we denote by $|A|$ its underlying set. We denote by $\mathbb{A}^1_k$ a $k$-functor given by assignment $\mathbb{A}^1_k(A)=|A|$ for every $A$. We call $\mathbb{A}^1_k$ \textit{the affine line over $k$}. Let $k[x]$ be a polynomial $k$-algebra with variable $x$. For every $k$-algebra $A$ map of sets 
$$\Mor_{k}\left(k[x],A\right)\ni f \mapsto f(x)\in |A|$$
is a bijection. The family of such maps gives rise to an isomorphism of $k$-functors 
$$\Mor_k\left(\Spec(-),\Spec k[x]\right)\cong \Mor_k\left(k[x],-\right)\cong \mathbb{A}^1_k$$
and hence $\mathbb{A}^1_k$ is representable by an affine $k$-scheme $\Spec k[x]$.
\end{example}

\begin{definition}
Let $\fX$ be a $k$-functor. Consider $\alpha \in k$ and $f$, $g\in \Mor_k\left(\fX, \mathbb{A}^1_k\right)$. Then for every $k$-algebra $A$ and $x\in \fX(A)$ formulas
$$\left(f+g\right)(x) = f(x)+g(x),\,\left(f\cdot g\right)(x) = f(x)\cdot g(x),\,\left(\alpha \cdot f\right)(x) = \alpha \cdot f(x)$$
define $k$-algebra operations on the class $\Mor_k\left(\fX,\mathbb{A}^1_k\right)$. We call them \textit{pointwise $k$-algebra operations}. In particular, if $\Mor_k\left(\fX,\mathbb{A}^1_k\right)$ is a set, then pointwise $k$-algebras operations on this set give rise to \textit{the $k$-algebra of regular functions on $\fX$}.
\end{definition}


\section{$k$-functors of monoids and their linear representations}
\noindent
In the sequel we assume that the reader is familiar with notions of a monoid, group etc. in arbitrary category with finite products. For definitions and some discussion related to these notions cf. {\cite[pages 2-5]{Maclane}}.

\begin{definition}
\textit{A monoid (group) $k$-functor} is a monoid (group) object in the category of $k$-functors.
\end{definition}
\noindent
Next we introduce an important notion of a linear representation of a monoid $k$-functor. For this we define $k$-functors associated with modules over $k$ and discuss their properties.

\begin{example}\label{example:additivekfunctor}
Let $V$ be a $k$-module. We define a $k$-functor $V_{\mathrm{a}}$. We set
$$V_{\mathrm{a}}(A) = A\otimes_kV,\,V_{\mathrm{a}}(f) = f\otimes_k1_V$$
for every $k$-algebra $A$ and every morphism $f:A\ra B$ of $k$-algebras. Moreover, $V_{\mathrm{a}}$ admits a structure of a commutative group $k$-functor. Indeed, $V_{\mathrm{a}}(A)$ is a commutative group with respect to addition induced by its structure of $A$-module and $V_{\mathrm{a}}(f):V_{\mathrm{a}}(A)\ra V_{\mathrm{a}}(B)$ preserves the addition.
\end{example}
\noindent
Suppose now that $V$, $W$ are $k$-modules and $\sigma:\left(V_{\mathrm{a}}\right)_A\ra \left(W_{\mathrm{a}}\right)_A$ is a morphism of $A$-functors for some $k$-algebra $A$. Then for every $A$-algebra $B$ we denote by $\sigma^B:B\otimes_kV\ra B\otimes_kW$ the component of $\sigma$ for $B$.

\begin{definition}
Let $V,W$ be $k$-modules and let $A$ be a $k$-algebra. A morphism $\sigma:\left(V_{\mathrm{a}}\right)_A\ra \left(W_{\mathrm{a}}\right)_A$ of $A$-functors is \textit{linear} if for every $A$-algebra $B$ the component $\sigma^B:B\otimes_kV\ra B\otimes_kW$ is a morphism of $B$-modules.
\end{definition}
\noindent
Next result characterizes linear morphism.

\begin{fact}\label{fact:frommorphismsofmodulestolinear}
Let $V$, $W$ be $k$-modules and let $A$ be a $k$-algebra. Suppose that  $\phi:A\otimes_kV\ra A\otimes_kW$ is a morphism of $A$-modules. Then there exists a unique linear morphism $\sigma:\left(V_{\mathrm{a}}\right)_A\ra \left(W_{\mathrm{a}}\right)_A$ of $A$-functors such that $\sigma^A = \phi$.
\end{fact}
\begin{proof}
Note that if such $\sigma$ exists, then by requirement $\sigma^A = \phi$ for every morphism $f:A\ra B$ of $k$-algebras the following diagram
\begin{center}
\begin{tikzpicture}
[description/.style={fill=white,inner sep=2pt}]
\matrix (m) [matrix of math nodes, row sep=3em, column sep=3em,text height=1.5ex, text depth=0.25ex] 
{  A\otimes_kV  & A\otimes_kW           \\
   B\otimes_kV  & B\otimes_kW           \\} ;
\path[->,line width=1.0pt,font=\scriptsize]  
(m-1-1) edge node[above] {$ \phi $} (m-1-2)
(m-2-1) edge node[below] {$\sigma^B $} (m-2-2)
(m-1-1) edge node[left] {$ f\otimes_k1_V $} (m-2-1)
(m-1-2) edge node[right] {$ f\otimes_k1_W $} (m-2-2);
\end{tikzpicture}
\end{center}
must commute. We make this into a definition of a morphism $\sigma^B$ of $B$-modules. It is a matter of linear algebra that this diagram uniquely determines $\sigma^B$ and also that $\sigma^A = \phi$. It remains to verify that $\sigma = \{\sigma^B\}_{B\in \Alg_A}$ defined in such a way is a morphism of $A$-functors. For this suppose that $f:A\ra B$ and $g:B\ra C$ are morphisms of $k$-algebras. Then we have
$$\sigma^C \cdot (g \otimes_k 1_V) \cdot (f\otimes_k 1_V) = \sigma^C \cdot (\left (g\cdot f) \otimes_k 1_V\right) = \left((g\cdot f) \otimes_k 1_W \right) \cdot \phi = $$
$$ = (g \otimes_k 1_W) \cdot (f \otimes_k 1_V)\cdot \phi= (g \otimes_k 1_W)\cdot \sigma^B \cdot (f \otimes_k 1_V)$$
and hence $\sigma^C \cdot (g \otimes_k 1_V) = (g \otimes_k 1_W) \cdot \sigma^B$. Thus $\sigma$ is a linear morphism of $A$-functors.
\end{proof}
\noindent
We restate Fact \ref{fact:frommorphismsofmodulestolinear} in the form of the following result.

\begin{corollary}\label{corollary:idenitificationoflinearmorphismsoffunctors}
Let $V, W$ be $k$-modules and $A$ be a $k$-algebra. Consider the map
\begin{center}
\begin{tikzpicture}
[description/.style={fill=white,inner sep=2pt}]
\matrix (m) [matrix of math nodes, row sep=3em, column sep=3em,text height=1.5ex, text depth=0.25ex] 
{ \Hom_A(A\otimes_kV,A\otimes_kW) & \Mor_{A}\left(\left(V_{\mathrm{a}}\right)_A,\left(W_{\mathrm{a}}\right)_A\right) \\};
\path[->,line width=1.0pt,font=\scriptsize]  
(m-1-1) edge node[auto] {$ $} (m-1-2);
\end{tikzpicture}
\end{center}
that sends morphism $\phi$ to a unique linear morphism $\sigma:\left(V_{\mathrm{a}}\right)_A\ra \left(W_{\mathrm{a}}\right)_A$ of $A$-functors such that $\sigma^A = \phi$. Then this map is injective and its image consists of all linear morphisms of $A$-functors.
\end{corollary}

\begin{example}
Let $V$ be a $k$-module. We define a $k$-functor $\cL_V$. We set
$$\cL_V(A) = \Hom_A(A\otimes_kV,A\otimes_kV)$$
for every $k$-algebra $A$. Next for every morphism $f:A\ra B$ of $k$-algebras and every morphism $\phi:A\otimes_kV\ra A\otimes_kV$ of $A$-modules we define $\cL_V(f)(\phi)$ as a unique morphism of $B$-modules such that the diagram
\begin{center}
\begin{tikzpicture}
[description/.style={fill=white,inner sep=2pt}]
\matrix (m) [matrix of math nodes, row sep=3em, column sep=3em,text height=1.5ex, text depth=0.25ex] 
{  A\otimes_kV  & A\otimes_kW           \\
   B\otimes_kV  & B\otimes_kW           \\} ;
\path[->,line width=1.0pt,font=\scriptsize]  
(m-1-1) edge node[above] {$ \phi $} (m-1-2)
(m-2-1) edge node[below] {$\cL_V(\phi)  $} (m-2-2)
(m-1-1) edge node[left] {$ f\otimes_k1_V $} (m-2-1)
(m-1-2) edge node[right] {$ f\otimes_k1_W $} (m-2-2);
\end{tikzpicture}
\end{center}
is commutative. Note also that $\cL_V(A)$ is a monoid with respect to the usual composition of morphism of $A$-modules and $\cL_V(f):\cL_V(A)\ra \cL_V(B)$ preserves this composition. Hence $\cL_V$ is a monoid $k$-functor.
\end{example}

\begin{remark}\label{remark:generallinearmonoid}
Corollary \ref{corollary:idenitificationoflinearmorphismsoffunctors} implies that there are injective maps that make the square
\begin{center}
\begin{tikzpicture}
[description/.style={fill=white,inner sep=2pt}]
\matrix (m) [matrix of math nodes, row sep=3em, column sep=3em,text height=1.5ex, text depth=0.25ex] 
{ \cL_V(A)   &  \Mor_{A}\left(\left(V_{\mathrm{a}}\right)_A,\left(V_{\mathrm{a}}\right)_A\right)           \\
  \cL_V(B)   &  \Mor_{B}\left(\left(V_{\mathrm{a}}\right)_B,\left(V_{\mathrm{a}}\right)_B\right)           \\} ;
\path[right hook->,line width=1.0pt,font=\scriptsize]  
(m-1-1) edge node[auto] {$ $} (m-1-2)
(m-2-1) edge node[below] {$ $} (m-2-2);
\path[->,line width=1.0pt,font=\scriptsize]
(m-1-1) edge node[left] {$ \cL_V(f) $} (m-2-1)
(m-1-2) edge node[auto] {$ \sigma \mapsto \sigma_B $} (m-2-2);
\end{tikzpicture}
\end{center}
commutative for every morphism $f:A\ra B$ of $k$-algebras. It also shows that for every $k$-algebra $A$ this identifies $\cL_V(A)$ with a subset of the class $\Mor_A\left(\left(V_{\mathrm{a}}\right)_A,\left(V_{\mathrm{a}}\right)_A\right)$ consisting of all linear morphism of the $A$-functor $\left(V_{\mathrm{a}}\right)_A$.
\end{remark}
\noindent
The discussion below is partially an application of the main result in {\cite[section 6]{Presheaves}} (Remark \ref{remark:generallinearmonoid} shows that $\cL_V$ is a subcopresheaf of internal endomorphisms of $V_{\mathrm{a}}$ and hence the machinery developed in the citation above can be applied), but for the reader's convenience we decide to include all essential details even if this requires repetition.\\
Let $\fX$ be a monoid $k$-functor and let be $V$ be a $k$-module. Suppose that $\alpha:\fX\times V_{\mathrm{a}}\ra V_{\mathrm{a}}$ is an action of $\fX$ on $V_{\mathrm{a}}$. Assume that $A$ is a $k$-algebra and $x\in \fX(A)$. We denote by $i_x:\bd{1}_A\ra \fX_A$ the morphism of $A$-functors corresponding to $x$ by means of {\cite[Fact 2.4]{kfunctors}}. Since $\bd{1}_A$ is terminal $A$-functor, a morphism $\alpha_A\cdot \left(i_x \times 1_{\left(V_{\mathrm{a}}\right)_A}\right)$ is isomorphic to a morphism $\alpha_x:\left(V_{\mathrm{a}}\right)_A\ra \left(V_{\mathrm{a}}\right)_A$ of $A$-functors. Suppose now that for any $k$-algebra $A$ and point $x\in \fX(A)$ morphism $\alpha_x$ is linear. Then we define a morphism $\rho:\fX\ra \cL_V$ of $k$-functors by formula $\rho(x) = \alpha_x^A$. We first check that $\rho$ really is a morphism of $k$-functors. For this fix morphism $f:A\ra B$ of $k$-algebras and $x\in \fX(A)$. Then $\alpha_{\fX(f)(x)}$ is a morphism of $B$-functors isomorphic with $\alpha_B\cdot \left(i_{\fX(f)(x)}\times 1_{\left(V_{\mathrm{a}}\right)_B}\right)$ and since 
$$\alpha_B\cdot \left(i_{\fX(f)(x)}\times 1_{\left(V_{\mathrm{a}}\right)_B}\right) = \alpha_B\cdot \left(i_{x}\times 1_{\left(V_{\mathrm{a}}\right)_A}\right)_B = \left(\alpha_A\cdot \left(i_{x}\times 1_{\left(V_{\mathrm{a}}\right)_A}\right)\right)_B$$
we derive that $\alpha_{\fX(f)(x)} = \left(\alpha_x\right)_B$. This implies that
$$\rho\left(\fX(f)(x)\right) = \alpha_{\fX(f)(x)}^B = \left(\left(\alpha_x\right)_B\right)^B= \alpha_x^B = \cL_V(f)\left(\alpha_x^A\right) = \cL_V(f)(\rho(x))$$
and thus $\rho$ is a morphism of $k$-functors. Now we show that $\rho$ is a morphism of monoids. For this pick $k$-algebra $A$ and $x, y\in \fX(A)$. Since $\alpha$ is an action, we deduce that $\alpha_{x \cdot y} = \alpha_x\cdot \alpha_y$ and hence also
$$\rho(x\cdot y) = \alpha^A_{x \cdot y} = \alpha^A_x\cdot \alpha^A_y = \rho(x)\cdot \rho(y)$$
Therefore, $\rho$ is a morphism of monoid $k$-functors.

\begin{theorem}\label{theorem:characterizationsoflinearrepresentations}
Let $\fX$ be a monoid $k$-functor and let $V$ be a $k$-module. Consider the following classes. 
\begin{enumerate}[label=\emph{\textbf{(\arabic*)}}, leftmargin=1.5em]
\item The class of actions $\alpha:\fX\times V_{\mathrm{a}}\ra V_{\mathrm{a}}$ of $\fX$ such that for any $k$-algebra $A$ and point $x\in \fX(A)$ morphism $\alpha_x$ is linear.
\item The class of morphisms $\rho:\fX\ra \cL_V$ of monoid $k$-functors.
\end{enumerate}
Let $\alpha$ be an element of \emph{\textbf{(1)}} and $\rho:\fX\ra \cL_V$ be the element of \emph{\textbf{(2)}} such that $\rho(x) = \alpha_x^A$ for any $k$-algebra $A$ and $x\in \fX(A)$. Then the correspondence $\alpha \mapsto \rho$ is a bijection between these classes.
\end{theorem}
\begin{proof}
We may refer to {\cite[Theorem 6.3]{Presheaves}}, but for self-containment of the presentation let us give a direct proof of this important result.
\end{proof}




\small
\bibliographystyle{alpha}
\bibliography{../zzz}


\end{document}