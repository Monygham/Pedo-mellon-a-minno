\input ../pree.tex

\begin{document}
\title{Geometry of $k$-functors}
\date{}
\maketitle

\section{Introduction}
\noindent
In this notes we present functorial approach to algebraic geometry. Our aim is to show that functorial and geometrical techniques are interrelated in a very efficient way.

\section{\textit{k}-functors}

\begin{definition}
The category $\Fun(\Alg_k,\Set)$ of copresheaves on $\Alg_k$ is called \textit{the category of $k$-functors}.
\end{definition}
\noindent
If $\fX$ and $\fY$ are $k$-functors, then we denote by $\Mor_k(\fX,\fY)$ the class of morphisms $\fX\ra \fY$ of $k$-functors.\\
Since the category of $k$-functors is a category of copresheaves, under assumptions specified in {\cite[section 5]{Presheaves}} for given $k$-functors $\fX$, $\fY$ there exists an internal hom $\iMor_k(\fX,\fY)$. Let us discuss this important notion and also related ones. For details and proofs for general case we refer to {\cite[section 5]{Presheaves}}.\\
Let $\fX$ and $\fY$ be $A$-functors for some $k$-algebra $A$. Then we denote by $\Mor_A\left(\fX,\fY\right)$ the class of morphisms of $A$-functors $\fX\ra \fY$. For every $A$-algebra $B$ and a morphism $\sigma:\fX\ra \fY$ of $A$-functors we denote by $\fX_{B}$, $\fY_{B}$, $\sigma_{B}$ the restrictions $\fX_{\mid \Alg_B}$, $\fY_{\mid \Alg_B}$, $\sigma_{\mid \Alg_B}$ of these entities to the category of $B$-algebras. 

\begin{fact}\label{fact:restrictionsworkasexpected}
Let $\fX$ and $\fY$ be $k$-functors. Assume that $A$ is a $k$-algebra, $B$ is an $A$-algebra, $C$ is an $B$-algebra. Then the composition of maps of classes
\begin{center}
\begin{tikzpicture}
[description/.style={fill=white,inner sep=2pt}]
\matrix (m) [matrix of math nodes, row sep=3em, column sep=3em,text height=1.5ex, text depth=0.25ex] 
{ \Mor_A\left(\fX_A,\fY_A\right) &  \Mor_B\left(\fX_B,\fY_B\right) & \Mor_C\left(\fX_C,\fY_C\right)\\} ;
\path[->,line width=1.0pt,font=\scriptsize]  
(m-1-1) edge node[above] {$\sigma\mapsto \sigma_{B} $} (m-1-2)
(m-1-2) edge node[above] {$\sigma\mapsto \sigma_{B} $} (m-1-3);
\end{tikzpicture}
\end{center}
equals
\begin{center}
\begin{tikzpicture}
v[description/.style={fill=white,inner sep=2pt}]
\matrix (m) [matrix of math nodes, row sep=3em, column sep=3em,text height=1.5ex, text depth=0.25ex] 
{ \Mor_A\left(\fX_A,\fY_A\right) &  \Mor_C\left(\fX_C,\fY_C\right)\\} ;
\path[->,line width=1.0pt,font=\scriptsize]  
(m-1-1) edge node[above] {$\sigma\mapsto \sigma_C $} (m-1-2);
\end{tikzpicture}
\end{center}
\end{fact}
\begin{proof}
Left to the reader.
\end{proof}

\begin{definition}
Let $\fX$ and $\fY$ be $k$-functors and suppose that for every $k$-algebra $A$ the class $\Mor_A\left(\fX_A,\fY_A\right)$ is a set. We define
$$\iMor_k(\fX,\fY)(A)=\Mor_A\left(\fX_A,\fY_A\right)$$
for every $k$-algebra $A$. This is a $k$-functor, since for every $k$-algebra $A$ and $A$-algebra $B$, we can compose a morphism $\sigma:\fX_A\ra \fY_A$ of $k$-functors with the forgetful functor $\Alg_B \ra \Alg_A$ i.e. we have a map 
$$\iMor_{k}(\fX,\fY)(A)\ni \sigma \mapsto \sigma_{B}\in \iMor_{k}(\fX,\fY)(B)$$
and these according to Fact \ref{fact:restrictionsworkasexpected} make $\iMor_{k}(\fX,\fY)$ a $k$-functor. The $k$-functor $\iMor_{\cC}(\fX,\fY)$ is called \textit{a hom $k$-functor of $\fX$ and $\fY$}.
\end{definition}
\noindent
We define a $k$-functor $\bd{1}$ that assigns to every $k$-algebra a set with one element. For every $k$-algebra $A$ the restriction $\bd{1}_A$ is a terminal object in the category of $A$-functors.

\begin{fact}\label{fact:points}
Let $\fX$ be a $k$-functor. Suppose $A$ is a $k$-algebra and $x\in \fX(A)$. Then $x$ determines a morphism $\bd{1}_{A}\ra \fX_A$ that for every $A$-algebra $B$ with structural morphism $f:A\ra B$ sends a unique element of $\bd{1}_{A}(B)$ to $\fX(f)(x)\in \fX_A(B)$. This gives rise to a bijection
$$\fX(A)\cong \Mor_{A}\left(\bd{1}_{A},\fX_A\right)$$
\end{fact}
\begin{proof}
We left to the reader as an exercise.
\end{proof}

\begin{definition}
Let $\fX$ be a $k$-functor and $A$ be a $k$-algebra. The set $\fX(A)$ is called \textit{the set of $A$-points of $\fX$}.
\end{definition}
\noindent
Now let $\fX$, $\fY$ be $k$-functors such that for every $k$-algebra $A$ the class $\Mor_A\left(\fX_A,\fY_A\right)$ is a set. Suppose next that $\fU$ is a $k$-functor and $\sigma:\fU\times \fX\ra \fY$ is a morphism of $k$-functors. Fix $x\in \fU(A)$. We denote by $i_x:\bd{1}_A\ra \fU_A$ the morphism of $A$-functors corresponding to $x$ by means of Fact \ref{fact:points}. Since $\bd{1}_A$ is terminal $A$-functor, a morphism $\sigma_A\cdot \left(1_{\fX_A}\times i_x\right)$ is isomorphic to a morphism $\tau_x:\fX_A\ra \fY_A$ of $A$-functors. Next $x\mapsto \tau_x$ gives rise to a morphism $\tau:\fU\ra \iMor_k\left(\fX,\fY\right)$ of $k$-functors and hence we have a map of classes
$$\Mor_k(\fU\times \fX,\fY)\ni \sigma\mapsto \tau\in \Mor_k\left(\fU,\iMor_k(\fX,\fY)\right)$$
Now we have the following result {\cite[Theorem 5.3]{Presheaves}}.

\begin{theorem}\label{theorem:homforkfunctors}
Let $\fX$, $\fY$ be $k$-functors. Assume that for every $k$-algebra $A$ the class $\Mor_{A}\left(\fX_A,\fY_A\right)$ is a set. Then the map 
$$\Mor_{k}\left(\fU\times \fX,\fY\right)\ra  \Mor_{k}\left(\fU,\iMor_{k}\left(\fX,\fY\right)\right)$$
described above is a bijection natural in $\fU$. 
\end{theorem}

\section{Zariski local $k$-functors and Zariski sheaves}
\noindent
In this part we use notion of a Grothendieck topology on a category. For this notion we refer the reader to \cite{Sheaves}.

\begin{definition}
Let $\big\{f_i:X_i\ra X\big\}_{i\in I}$ be a family of morphisms of $k$-schemes. We say that $\{f_i\}_{i\in I}$ is \textit{a Zariski covering of $X$} if the following conditions are satisfied.
\begin{enumerate}[label=\textbf{(\arabic*)}, leftmargin=1.5em]
\item For every $i\in I$ morphism $f_i$ is an open immersion of schemes.
\item Morphism $\coprod_{i\in I}X_i\ra X$ induced by $\big\{f_i\big\}_{i\in I}$ is surjective.
\end{enumerate}
\end{definition}
\noindent
The collection of all Zariski coverings on $\Sch_k$ is a Grothendieck pretopology.

\begin{definition}
We call the Grothendieck topology generated by pretopology of Zariski coverings \textit{the Zariski topology on $\Sch_k$}. A presheaf on $\Sch_k$ that is a sheaf with respect to Zariski topology on $\Sch_k$ is called \textit{a Zariski sheaf}.
\end{definition}
\noindent
Let $\fX$ be a presheaf on the category of $k$-schemes. Recall that by {\cite[Theorem 3.5]{Sheaves}} $\fX$ is a Zariski sheaf if and only if for every $k$-scheme $X$ and for every Zariski covering $\big\{f_i:X_i\ra X\big\}$ of $X$ the diagram
\begin{center}
\begin{tikzpicture}
[description/.style={fill=white,inner sep=2pt}]
\matrix (m) [matrix of math nodes, row sep=3em, column sep=6em,text height=1.5ex, text depth=0.25ex] 
{\fX(X) &   \prod_{i\in I}\fX(X_i)&  \prod_{(i,j)\in I\times I} \fX(X_i\times_XX_j)  \\} ;
\path[->,line width=0.8pt,font=\scriptsize]
(m-1-1) edge node[above] {$ \langle \fX(f_i) \rangle_{i\in I} $} (m-1-2)
(m-1-2) edge[transform canvas={yshift=0.5ex}] node[above] {$ \langle \fX(f'_{ij}) \cdot pr_i\rangle_{(i,j)}$} (m-1-3)
(m-1-2) edge[transform canvas={yshift=-0.5ex}] node[below] {$ \langle \fX(f''_{ij}) \cdot pr_j\rangle_{(i,j)}$} (m-1-3);
\end{tikzpicture}
\end{center}
is a kernel of a pair of arrows, where for every $(i,j)\in I\times I$ morphisms $f'_{ij}$ and $f'_{ji}$ form a cartesian square
\begin{center}
\begin{tikzpicture}
[description/.style={fill=white,inner sep=2pt}]
\matrix (m) [matrix of math nodes, row sep=3em, column sep=3em,text height=1.5ex, text depth=0.25ex] 
{X_i\times_XX_j   &   X_j   \\
 X_i  & X   \\} ;
\path[->,line width=0.8pt,font=\scriptsize]
(m-1-1) edge node[above] {$ f''_{ij}$} (m-1-2)
(m-2-1) edge node[below] {$ f_i $} (m-2-2)
(m-1-1) edge node[left] {$ f'_{ij} $} (m-2-1)
(m-1-2) edge node[right] {$ f_j  $} (m-2-2);
\end{tikzpicture}
\end{center}
\noindent
Now we repeat this definitions for $k$-algebras and $k$-functors.

\begin{definition}
Let $\big\{f_i:A\ra A_i\big\}_{i\in I}$ be a family of morphisms of $k$-algebras. We say that $\{f_i\}_{i\in I}$ is \textit{a Zariski covering of $A$} if the following conditions are satisfied.
\begin{enumerate}[label=\textbf{(\arabic*)}, leftmargin=1.5em]
\item For every $i\in I$ morphism $\Spec f_i$ is an open immersion of schemes.
\item Morphism $\coprod_{i\in I}\Spec A_i\ra \Spec A$ induced by $\big\{\Spec f_i\big\}_{i\in I}$ is surjective.
\end{enumerate}
\end{definition}
\noindent
The collection of all Zariski coverings on $\Alg_k$ induces on its opposite category $\Aff_k$ of affine $k$-schemes a Grothendieck pretopology.

\begin{definition}
We call the Grothendieck topology generated by pretopology of Zariski coverings \textit{the Zariski topology on $\Aff_k$}. A $k$-functor that is a sheaf with respect to Zariski topology on $\Aff_k$ is called \textit{a Zariski local $k$-functor}.
\end{definition}
\noindent
Let $\fX$ be a $k$-functor. Again by {\cite[Theorem 3.5]{Sheaves}} $\fX$ is a Zariski local $k$-functor if and only if for every $k$-algebra $A$ and for every Zariski covering $\big\{f_i:A\ra A_i\big\}$ of $A$ the diagram
\begin{center}
\begin{tikzpicture}
[description/.style={fill=white,inner sep=2pt}]
\matrix (m) [matrix of math nodes, row sep=3em, column sep=6em,text height=1.5ex, text depth=0.25ex] 
{\fX(A) &   \prod_{i\in I}\fX(A_i)&  \prod_{(i,j)\in I\times I} \fX(A_i\otimes_AA_j)  \\} ;
\path[->,line width=0.8pt,font=\scriptsize]
(m-1-1) edge node[above] {$ \langle \fX(f_i) \rangle_{i\in I} $} (m-1-2)
(m-1-2) edge[transform canvas={yshift=0.5ex}] node[above] {$ \langle \fX(f'_{ij}) \cdot pr_i\rangle_{(i,j)}$} (m-1-3)
(m-1-2) edge[transform canvas={yshift=-0.5ex}] node[below] {$ \langle \fX(f''_{ij}) \cdot pr_j\rangle_{(i,j)}$} (m-1-3);
\end{tikzpicture}
\end{center}
is a kernel of a pair of arrows, where for every $(i,j)\in I\times I$ morphisms $f'_{ij}$ and $f'_{ji}$ form a cocartesian square
\begin{center}
\begin{tikzpicture}
[description/.style={fill=white,inner sep=2pt}]
\matrix (m) [matrix of math nodes, row sep=3em, column sep=3em,text height=1.5ex, text depth=0.25ex] 
{A &  A_j   \\
 A_i&  A_i\otimes_AA_j   \\} ;
\path[->,line width=0.8pt,font=\scriptsize]
(m-1-1) edge node[above] {$ f_j $} (m-1-2)
(m-2-1) edge node[below] {$ f'_{ij} $} (m-2-2)
(m-1-1) edge node[left] {$ f_i $} (m-2-1)
(m-1-2) edge node[right] {$ f'_{ji}  $} (m-2-2);
\end{tikzpicture}
\end{center}
\noindent
Now we state the main result of this section.

\begin{theorem}\label{theorem:sheavesonschemesarelocalkfunctors}
Let
\begin{center}
\begin{tikzpicture}
[description/.style={fill=white,inner sep=2pt}]
\matrix (m) [matrix of math nodes, row sep=3em, column sep=3em,text height=1.5ex, text depth=0.25ex] 
{ \widehat{\Sch_k}  & \mbox{the category of $k$-functors} \\};
\path[->,line width=1.0pt,font=\scriptsize]  
(m-1-1) edge node[auto] {$ $} (m-1-2);
\end{tikzpicture}
\end{center}
be the restriction of presheaves on $\Sch_k$ to copresheaves on $\Alg_k$ ($k$-functors) induced by the contravariant functor $\Spec:\Alg_k\ra \Sch_k$. Then it induces an equivalence of categories between Zariski sheaves on $\Sch_k$ and Zariski local $k$-functors.
\end{theorem}
\begin{proof}
Note that $\Aff_k$ with Zariski topology is a dense subsite ({\cite[definition 4.4]{Sheaves}}) of $\Sch_k$ with Zariski topology. Hence the result is a special case of a more general theorem {\cite[Theorem 4.6]{Sheaves}}. 
\end{proof}

\section{Schemes and their functors of points}
\noindent
Let $X$ be a $k$-scheme. We define a $k$-functor $\fP_X$ by formula
$$\fP_X(A) = \Mor_k\left(\Spec A,X\right)$$
That is $\fP_X$ is the restriction of the presheaf on $\Sch_k$ represented by $X$ to the category $Alg_k$ along the functor $\Spec:\Alg_k\ra \Sch_k$. Next if $f:X\ra Y$ is a morphism of $k$-schemes, then $\fP_f$ is the restriction of a morphism of presheaves on $\Sch_k$ represented by $f$ to the category of $k$-algebras along $\Spec:\Alg_k\ra \Sch_k$. Thus we have a functor
\begin{center}
\begin{tikzpicture}
[description/.style={fill=white,inner sep=2pt}]
\matrix (m) [matrix of math nodes, row sep=3em, column sep=3em,text height=1.5ex, text depth=0.25ex] 
{ \Sch_k  & \mbox{the category of $k$-functors} \\};
\path[->,line width=1.0pt,font=\scriptsize]  
(m-1-1) edge node[auto] {$ \fP $} (m-1-2);
\end{tikzpicture}
\end{center}

\begin{fact}\label{fact:functorsofpoints}
Functor
\begin{center}
\begin{tikzpicture}
[description/.style={fill=white,inner sep=2pt}]
\matrix (m) [matrix of math nodes, row sep=3em, column sep=3em,text height=1.5ex, text depth=0.25ex] 
{ \Sch_k  & \mbox{the category of $k$-functors} \\};
\path[->,line width=1.0pt,font=\scriptsize]  
(m-1-1) edge node[auto] {$ \fP $} (m-1-2);
\end{tikzpicture}
\end{center}
is full, faithful and its image consists of Zariski local $k$-functors. Moreover, $\fB$ preserves limits.
\end{fact}
\begin{proof}
Note that the presheaf $h_X$ on $\Sch_k$ represented by $X$ is a Zariski sheaf. Indeed, this just rephrase standard fact that morphism of schemes can be glued in Zariski topology. Next according to Theorem \ref{theorem:sheavesonschemesarelocalkfunctors} the functor $\Spec:\Alg_k\ra \Sch_k$ induces an equivalence between the category of Zariski sheaves and the category of local Zariski $k$-functors. Thus $\fP_X$ is a local Zariski $k$-functor and $\fB$ it is full and faithful. Note that Yoneda embedding $h:\Sch_k\ra \widehat{\Sch_k}$ and the functor
\begin{center}
\begin{tikzpicture}
[description/.style={fill=white,inner sep=2pt}]
\matrix (m) [matrix of math nodes, row sep=3em, column sep=7em,text height=1.5ex, text depth=0.25ex] 
{ \widehat{\Sch_k}  & \mbox{the category of $k$-functors} \\};
\path[->,line width=1.0pt,font=\scriptsize]  
(m-1-1) edge node[auto] {$ \textbf{induced by $\Spec$} $} (m-1-2);
\end{tikzpicture}
\end{center}
preserve limits. Thus their composition $\fB$ also preserves limits.
\end{proof}

\begin{definition}
Let $X$ be a $k$-scheme. Then $\fP_X$ is called \textit{the $k$-functor of points of $X$}.
\end{definition}
\noindent
Finally note that for every $k$-algebra $A$ we have an identification $\fP_{\Spec A} = \Hom_k\left(A,-\right)$ and this identification is natural with respect to $A$. In other words $\fB\cdot \Spec$ is the (co)Yoneda embedding of $\Alg_k$ into the category of $k$-functors.\\
Suppose now that $A$ is a $k$-algebra and $\ideal{a}\subseteq A$ is an ideal. Then we define $V(\ideal{a}) = \Spec A/\ideal{a}$ as a closed subscheme $\Spec A$ induced by the quotient morphism $A\ra A/\ideal{a}$. We define an open subscheme $D(\ideal{a}) = \Spec A\setminus V(\ideal{a})$ of $\Spec A$.

\begin{definition}
Let $\sigma:\fX\ra \fY$ be a morphism of $k$-functors. Assume that for every $k$-algebra $A$ and every morphism $\tau:\fB_{\Spec A}\ra \fY$ of $k$-functors there exist an ideal $\ideal{a}$ in $A$ and a morphism $\tau':\fB_{D(\ideal{a})}\ra \fX$ of $k$-functors such that the square
\begin{center}
\begin{tikzpicture}
[description/.style={fill=white,inner sep=2pt}]
\matrix (m) [matrix of math nodes, row sep=3em, column sep=3em,text height=1.5ex, text depth=0.25ex] 
{  \fB_{D(\ideal{a})}        & \fX           \\
   \fB_{\Spec A}             & \fY           \\} ;
\path[->,line width=1.0pt,font=\scriptsize]
(m-1-1) edge node[above] {$ \tau' $} (m-1-2)
(m-2-1) edge node[below] {$ \tau $} (m-2-2)
(m-1-2) edge node[right] {$ \sigma $} (m-2-2);
\path[right hook->,line width=1.0pt,font=\scriptsize]
(m-1-1) edge node[left] {$   $} (m-2-1);
\end{tikzpicture}
\end{center}
is cartesian. Then $\sigma$ is \textit{an open immersion of $k$-functors}.
\end{definition}

\begin{fact}\label{fact:openimmersionsclosedunderbasechangeandcomposition}
The class of open immersions of $k$-functors is closed under base change and composition.
\end{fact}
\begin{proof}
Left to the reader.
\end{proof}

\begin{definition}
Let $\fX$ be a $k$-functor and $\big\{\sigma_i:\fX_i\ra \fX\big\}_{i\in I}$ be a family of open immersions. Then for every $k$-algebra $A$ and $x\in \fX(A)$ we have a family of ideals $\{\ideal{a}_i\}_{i\in I}$ defined by cartesian squares
\begin{center}
\begin{tikzpicture}
[description/.style={fill=white,inner sep=2pt}]
\matrix (m) [matrix of math nodes, row sep=3em, column sep=3em,text height=1.5ex, text depth=0.25ex] 
{\fB_{D(\ideal{a}_i)}   &    \fX_i   \\
 \fB_{\Spec A}          &    \fX  \\} ;
\path[->,line width=1.0pt,font=\scriptsize]  
(m-1-1) edge node[above] {$ \tau'  $} (m-1-2)
(m-2-1) edge node[below] {$ \tau $} (m-2-2)
(m-1-2) edge node[right] {$\sigma_i$} (m-2-2);
\path[right hook->,line width=1.0pt,font=\scriptsize]
(m-1-1) edge node[left] {$ $} (m-2-1);
\end{tikzpicture}
\end{center}
in which bottom vertical morphism $\tau:\fB_{\Spec A}\ra \fX$ corresponds to $x$. We say that $\{\sigma_i\}_{i\in I}$ is \textit{an open cover of $\fX$} if for every $k$-scheme $X$ and $x\in \fX(X)$ we have
$$\Spec A=\bigcup_{i\in I}D(\ideal{a}_i)$$
or in other words $A = \sum_{i\in I}\ideal{a}_i$.
\end{definition}

\begin{theorem}\label{theorem:representabilitybasicresult}
Let $\fX$ be a $k$-functor. Then the following are equivalent.
\begin{enumerate}[label=\emph{\textbf{(\roman*)}}, leftmargin=1.5em]
\item $\fX$ is isomorphic with functor of points of some $k$-scheme.
\item $\fX$ is a Zariski local $k$-functor and there exists an open cover $\big\{\sigma_i:\fB_{X_i}\ra \fX\big\}_{i\in I}$ such that $\{X_i\}_{i\in I}$ are $k$-schemes.
\item $\fX$ is a Zariski local $k$-functor and there exists an open cover $\big\{\sigma_i:\fB_{\Spec A_i}\ra \fX\big\}_{i\in I}$ such that $\{A_i\}_{i\in I}$ are $k$-algebras.
\end{enumerate}
\end{theorem}
\noindent
The proof depends on few lemmas.

\begin{lemma}\label{lemma:openimmersionslocallyondomainarezariskilocalkfunctors}
Let $f:X\ra Y$ be a morphism of $k$-schemes. Suppose that $f$ is surjective morphism and an open immersion locally on $X$. Then $\fB_f$ is a locally surjective morphism of Zariski local $k$-functors. 
\end{lemma}
\begin{proof}[Proof of the lemma]
Let $A$ be a $k$-algebra and $g:\Spec A\ra Y$ be an open immersion of schemes. Since $f$ is surjective and an open immersion locally on $X$, there exist a Zariski cover $\big\{f_i:A\ra A_i\big\}_{i\in I}$ and a family $\big\{g_i:\Spec A_i\ra X\big\}_{i\in I}$ of open immersions such that $f\cdot g_i = g\cdot \Spec f_i$ for every $i\in I$. This implies that $\fB_f(g_i) = \fB_Y(f_i)(g)$ for every $i\in I$. Thus $\fB_f$ is a locally surjective morphism of Zariski local $k$-functors.
\end{proof}
\noindent
The next lemma is a categorical formulation of the \textit{recollement} technique {\cite[Chapitre 0, 4.1.7]{EGA1new}} and as such it is more convienient in our settings.

\begin{lemma}\label{lemma:recollement}
Let $X = \coprod_{i\in I}X_i, R = \coprod_{i,j\in I}R_{ij}$ be disjoint sums of $k$-schemes and let $p,q:R\ra X$ be morphisms of $k$-schemes such that
\begin{enumerate}[label=\emph{\textbf{(\arabic*)}}, leftmargin=1.5em]
\item For any $i,j\in I$ morphism $p_{\mid R_{ij}}$ induces an open immersion $R_{ij}\hookrightarrow X_i$ and morphism $q_{\mid R_{ij}}$ induces an open immersion $R_{ij}\hookrightarrow X_j$.
\item For every $i\in I$ morphisms $p_{\mid R_{ii}}$ and $q_{\mid R_{ii}}$ are equal and induce an isomorphisms $R_{ii}\ra X_i$.  
\item Triple $\left(R,p,q\right)$ is an equivalence relation on $X$ in the category of $k$-schemes.
\end{enumerate}
Then there exist a $k$-scheme $Y$ and a morphism $f:X\ra Y$ of $k$-schemes such that
\begin{center}
\begin{tikzpicture}
[description/.style={fill=white,inner sep=2pt}]
\matrix (m) [matrix of math nodes, row sep=3em, column sep=3em,text height=1.5ex, text depth=0.25ex] 
{\fB_{R} &  \fB_X & \fB_Y  \\} ;
\path[->,line width=0.8pt,font=\scriptsize]
(m-1-1) edge[transform canvas={yshift=0.5ex}] node[above] {$ \fB_{p}  $} (m-1-2)
(m-1-1) edge[transform canvas={yshift=-0.5ex}] node[below] {$ \fB_{q} $} (m-1-2)
(m-1-2) edge node[above] {$ \fB_f  $} (m-1-3);
\end{tikzpicture}
\end{center}
is a cokernel of a pair $(\fB_{p},\fB_{q})$ in the category of Zariski local $k$-functors.
\end{lemma}
\begin{proof}[Proof of the lemma]
We define first $Y$ and $f$ in the category of topological spaces and continuous maps. For this let $Y$ be the quotient space of $X$ by $R$ and let $f$ be the quotient topological map. Since topologically we have $f\cdot p = f\cdot q$, we deduce that $f_*p_*\cO_R = f_*q_*\cO_R$ and in addition we have a pair
\begin{center}
\begin{tikzpicture}
[description/.style={fill=white,inner sep=2pt}]
\matrix (m) [matrix of math nodes, row sep=3em, column sep=3em,text height=1.5ex, text depth=0.25ex] 
{f_*\cO_X &  f_*\cdot {p}_*\cO_R = f_*\cdot {q}_*\cO_R  \\} ;
\path[->,line width=0.8pt,font=\scriptsize]
(m-1-1) edge[transform canvas={yshift=0.5ex}] node[above] {$ f_*p^{\#}  $} (m-1-2)
(m-1-1) edge[transform canvas={yshift=-0.5ex}] node[below] {$ f_*q^{\#} $} (m-1-2);
\end{tikzpicture}
\end{center}
of morphisms of sheaves of $k$-algebras on $Y$. We define a sheaf $\cO_Y$ of $k$-algebras to be a kernel of a pair $(f_*p^{\#},f_*q^{\#})$. This sheaf comes with the canonical morphism $f^{\#}:\cO_Y\ra f_*\cO_X$ of sheaves of $k$-algebras. This makes $Y$ into a ringed topological space and $f$ into a morphism of ringed spaces such that $$f \cdot p = f\cdot q$$
Now we show that $Y$ is a $k$-scheme and $f$ is a morphism of $k$-schemes. For every pair $i,j\in I$ we denote $p(R_{ij}) = X_{ij}$. This is an open subscheme of $X_i$ and moreover, $X_{ii} = X_i$ for every $i\in I$. Next for every $i,j\in I$ we have a commutative triangle
\begin{center}
\begin{tikzpicture}
[description/.style={fill=white,inner sep=2pt}]
\matrix (m) [matrix of math nodes, row sep=2em, column sep=1em,text height=1.5ex, text depth=0.25ex] 
{R_{ij} &  & R_{ji}   \\
&  X_{ij} &  \\} ;
\path[->,line width=1.0pt, font=\scriptsize]  
(m-1-1) edge node[above] {$ s_{ij} $} (m-1-3)
(m-1-3) edge node[auto] {$q_{ij} $} (m-2-2)
(m-1-1) edge node[left = 5pt, below = 1pt] {$p_{ij} $} (m-2-2);
\end{tikzpicture}
\end{center}
where $s_{ij}$ is induced by symmetry on $X\times X$ and isomorphisms $p_{ij}$ and $q_{ij}$ are induced by $p$ and $q$, respectively. In particular, $q_{ii} = p_{ii}$ for every $i\in I$. Since both $p, q$ are open maps, we derive that $f$ is an open map. Hence for every $i\in I$ a bijection $X_i\ra f(X_i)$ induced by $f$ is a topological isomorphism. Fix now $i\in I$ and denote $f(X_i)$ by $Y_i$. Note that this is an open subset of $Y$. Morphism $\left(f_*p^{\#}\right)_{Y_i}:\cO_{X_i}\times \prod_{j\neq i}\cO_{X_{ji}} \ra \prod_{j\in I}\cO_{R_{ij}} \times  \prod_{j\neq i}\cO_{R_{ji}}$ is given by
$$\left(a, (b_{ji})_{j\neq i}\right)  \mapsto \left(\left(p_{ij}^{\#}(a_{\mid X_{ij}})\right)_{j\in I}, \left(p_{ji}^{\#}(b_{ji})\right)_{j\neq i}\right) $$
and morphism $\left(f_*q^{\#}\right)_{Y_i}:\cO_{X_i}\times \prod_{j\neq i}\cO_{X_{ji}} \ra \prod_{j\in I}\cO_{R_{ji}} \times  \prod_{j\neq i}\cO_{R_{ij}}$ is given by
$$\left(a, (b_{ji})_{j\neq i}\right)  \mapsto \left(\left(q_{ij}^{\#}(a_{\mid X_{ij}})\right)_{j\in I}, \left(q_{ji}^{\#}(b_{ji})\right)_{j\neq i}\right)$$
Thus $\left(a, (b_{ji})_{j\neq i}\right)$ lies in the kernel of $\left(\left(f_*p^{\#}\right)_{Y_i}, \left(f_*q^{\#}\right)_{Y_i}\right)$ if and only if
$$p_{ij}^{\#}(a_{\mid X_{ij}}) = q_{ji}^{\#}(b_{ji}),\,q_{ij}^{\#}(a_{\mid X_{ij}}) = p_{ji}^{\#}(b_{ji})$$
for every $j\neq i$. This is equivalent with the system of equations
$$b_{ji} = \left(\left(q_{ji}^{\#}\right)^{-1}\cdot p_{ij}^{\#}\right)(a_{\mid X_{ij}}),\,b_{ji} = \left(\left(p_{ji}^{\#}\right)^{-1}\cdot q_{ij}^{\#}\right)(a_{\mid X_{ij}})$$
for every $j\neq i$. Since
$$p_{ji}^{\#}\cdot \left(q^{\#}_{ji}\right)^{-1} = s^{\#}_{ji} = s^{\#}_{ij} = p^{\#}_{ij}\cdot \left(q_{ij}^{\#}\right)^{-1}$$
for every $i,j\in I$, we derive that the system of equations has a unique solution for every section of $\cO_{X_i}$. This implies that $f^{\#}_{Y_i}:\cO_{Y_i}\ra \cO_{f^{-1}(Y_i)}$ composed with the restriction $\cO_{f^{-1}(Y_i)}\ra \cO_{X_i}$ is an isomorphism. Hence $f$ induces an isomorphism of ringed spaces $X_i\cong Y_i$. Thus for every $i\in I$ ringed space $Y_i$ is a $k$-scheme and $f$ induces isomorphism of $k$-schemes $X_i\cong Y_i$. Note that
$$Y = f(X) = \bigcup_{i\in I}f(X_i) = \bigcup_{i\in I}Y_i$$
Therefore, $Y$ is a $k$-scheme and $f:X\ra Y$ is a morphism of $k$-schemes.\\
Now we verify that $\fB_f$ is the quotient in the category of Zariski local $k$-functors. For this note that we proved above that $f$ is open immersion of $k$-schemes locally on $X$ and it is surjective. Thus by Lemma \ref{lemma:openimmersionslocallyondomainarezariskilocalkfunctors} we derive that $\fB_f$ is a locally surjective morphism of Zariski local $k$-functors. Therefore ({\cite[Theorem 7.3]{Sheaves}}), it suffices to show that the square
\begin{center}
\begin{tikzpicture}
[description/.style={fill=white,inner sep=2pt}]
\matrix (m) [matrix of math nodes, row sep=3em, column sep=3em,text height=1.5ex, text depth=0.25ex] 
{ \fB_R  &   \fB_X   \\
  \fB_X             &   \fB_Y   \\};
\path[->,line width=1.0pt,font=\scriptsize]  
(m-1-1) edge node[above] {$ \fB_q $} (m-1-2)
(m-2-1) edge node[below] {$ \fB_f $} (m-2-2)
(m-1-1) edge node[left] {$ \fB_p $} (m-2-1)
(m-1-2) edge node[right] {$ \fB_f $} (m-2-2);
\end{tikzpicture}
\end{center}
is cartesian. Since $\fB$ preserves limits (Fact \ref{fact:functorsofpoints}), we derive that it suffices to check that
\begin{center}
\begin{tikzpicture}
[description/.style={fill=white,inner sep=2pt}]
\matrix (m) [matrix of math nodes, row sep=3em, column sep=3em,text height=1.5ex, text depth=0.25ex] 
{ R  &   X   \\
  X  &   Y   \\};
\path[->,line width=1.0pt,font=\scriptsize]  
(m-1-1) edge node[above] {$ q $} (m-1-2)
(m-2-1) edge node[below] {$ f $} (m-2-2)
(m-1-1) edge node[left] {$ p $} (m-2-1)
(m-1-2) edge node[right] {$ f $} (m-2-2);
\end{tikzpicture}
\end{center}
is cartesian square of $k$-schemes. This is clear since $R_{ij} = X_i\times_YX_j$ for every $i, j\in I$ and hence
$$X\times_YX = \left(\coprod_{i\in I}X_i\right)\times_Y \left(\coprod_{i\in I}X_i\right) = \coprod_{i,j\in I}\left(X_i\times_YX_j\right) = \coprod_{i,j\in I}R_{ij} = R$$
Thus the result follows.
\end{proof}

\begin{proof}[Proof of the theorem]
If \textbf{(i)} holds, then we may assume that $\fX=\fB_Y$ for some $k$-scheme $Y$. Fact \ref{fact:functorsofpoints} states that $\fB_Y$ is a Zariski local $k$-functor and clearly $1_{\fB_Y}:\fB_Y\ra \fB_Y$ is an open cover. Thus $\textbf{(i)}\Rightarrow \textbf{(ii)}$.\\
Every functor of points of a $k$-scheme admits open cover by functors of points of affine $k$-schemes. Indeed, it suffices to take open affine subschemes that cover given $k$-scheme and apply $\fB$. This implies that every open cover of a $k$-functor $\fX$ by functors of points of $k$-schemes admits refinement by open cover of functors of points of affine $k$-schemes. Therefore, implication $\textbf{(ii)}\Rightarrow \textbf{(iii)}$ holds.\\
Suppose that a $k$-functor $\fX$ is Zariski local and $\big\{\sigma_i:\fB_{\Spec A_i}\ra \fX\big\}_{i\in I}$ is an open cover of $\fX$. Note that for every $i,j\in I$ there exist a $k$-scheme $R_{ij}$ and open immersions $p_{ij}:R_{ij}\hookrightarrow \Spec A_i$, $q_{ij}:R_{ij}\hookrightarrow \Spec A_j$ such that the square
\begin{center}
\begin{tikzpicture}
[description/.style={fill=white,inner sep=2pt}]
\matrix (m) [matrix of math nodes, row sep=3em, column sep=3em,text height=1.5ex, text depth=0.25ex] 
{  \fB_{R_{ij}}                &   \fB_{\Spec A_j}   \\
  \fB_{\Spec A_i}  &   \fX   \\};
\path[->,line width=1.0pt,font=\scriptsize]  
(m-1-1) edge node[above] {$\fB_{q_{ij}} $} (m-1-2)
(m-2-1) edge node[below] {$ \sigma_i $} (m-2-2)
(m-1-1) edge node[left] {$ \fB_{p_{ij}} $} (m-2-1)
(m-1-2) edge node[right] {$ \sigma_j $} (m-2-2);
\end{tikzpicture}
\end{center}
is cartesian. Consider $k$-scheme $X = \coprod_{i\in I}\Spec A_i$ and morphism $\sigma:\fB_X\ra \fX$ induced by $\{\sigma_i\}_{i\in I}$. Moreover, consider $k$-scheme $R = \coprod_{i,j\in I}R_{ij}$ and morphisms $p,q:R\ra X$ induced by $\{p_{ij}\}_{i,j\in I}$ and $\{q_{ij}\}_{i,j\in I}$, respectively. Note that the square
\begin{center}
\begin{tikzpicture}
[description/.style={fill=white,inner sep=2pt}]
\matrix (m) [matrix of math nodes, row sep=3em, column sep=3em,text height=1.5ex, text depth=0.25ex] 
{ \fB_R  &   \fB_X   \\
  \fB_X             &   \fX   \\};
\path[->,line width=1.0pt,font=\scriptsize]  
(m-1-1) edge node[above] {$ \fB_q $} (m-1-2)
(m-2-1) edge node[below] {$ \sigma $} (m-2-2)
(m-1-1) edge node[left] {$ \fB_p $} (m-2-1)
(m-1-2) edge node[right] {$ \sigma $} (m-2-2);
\end{tikzpicture}
\end{center}
is cartesian and hence $\left(\fB_R,\fB_p,\fB_q\right)$ is an equivalence relation. By Lemma \ref{lemma:recollement} there exist a $k$-scheme $Y$ and a morphism $f:X\ra Y$ such that
\begin{center}
\begin{tikzpicture}
[description/.style={fill=white,inner sep=2pt}]
\matrix (m) [matrix of math nodes, row sep=3em, column sep=3em,text height=1.5ex, text depth=0.25ex] 
{\fB_{R} &  \fB_X & \fB_Y  \\} ;
\path[->,line width=0.8pt,font=\scriptsize]
(m-1-1) edge[transform canvas={yshift=0.5ex}] node[above] {$ \fB_{p}  $} (m-1-2)
(m-1-1) edge[transform canvas={yshift=-0.5ex}] node[below] {$ \fB_{q} $} (m-1-2)
(m-1-2) edge node[above] {$ \fB_f  $} (m-1-3);
\end{tikzpicture}
\end{center}
is a cokernel of $\left(\fB_p,\fB_q\right)$. Moreover, $\sigma$ is locally surjective morphism of Zariski local $k$-functors and hence also
\begin{center}
\begin{tikzpicture}
[description/.style={fill=white,inner sep=2pt}]
\matrix (m) [matrix of math nodes, row sep=3em, column sep=3em,text height=1.5ex, text depth=0.25ex] 
{\fB_{R} &  \fB_X & \fX  \\} ;
\path[->,line width=0.8pt,font=\scriptsize]
(m-1-1) edge[transform canvas={yshift=0.5ex}] node[above] {$ \fB_{p}  $} (m-1-2)
(m-1-1) edge[transform canvas={yshift=-0.5ex}] node[below] {$ \fB_{q} $} (m-1-2)
(m-1-2) edge node[above] {$ \sigma  $} (m-1-3);
\end{tikzpicture}
\end{center}
is a cokernel of $\left(\fB_p,\fB_q\right)$. Thus $\fB_Y$ is isomorphic with $\fX$. This proves $\textbf{(iii)}\Rightarrow \textbf{(i)}$.
\end{proof}

\section{Representable morphisms of $k$-functors}

\begin{definition}
Let $\sigma:\fX\ra \fY$ be a morphism of $k$-functors. Assume that for every $k$-algebra $A$ and every morphism $\tau:\fB_{\Spec A}\ra \fY$ of $k$-functors there exist a $k$-scheme $X$, a morphism $f:X\ra \Spec A$ and a morphism $\tau':\fB_{X}\ra \fX$ of $k$-functors such that the square
\begin{center}
\begin{tikzpicture}
[description/.style={fill=white,inner sep=2pt}]
\matrix (m) [matrix of math nodes, row sep=3em, column sep=3em,text height=1.5ex, text depth=0.25ex] 
{  \fB_{X}        & \fX           \\
   \fB_{\Spec A}             & \fY           \\} ;
\path[->,line width=1.0pt,font=\scriptsize]
(m-1-1) edge node[above] {$ \tau' $} (m-1-2)
(m-2-1) edge node[below] {$ \tau $} (m-2-2)
(m-1-2) edge node[right] {$ \sigma $} (m-2-2)
(m-1-1) edge node[left] {$ \fB_f  $} (m-2-1);
\end{tikzpicture}
\end{center}
is cartesian. Then $\sigma$ is \textit{a representable morphism of $k$-functors}.
\end{definition}

\begin{fact}\label{fact:representablemorphismsunderbasechangeandcomposition}
The class of representable morphisms of $k$-functors is closed under base change and composition.
\end{fact}
\begin{proof}
Left to the reader.
\end{proof}

\begin{proposition}\label{proposition:representablemonomorphismsaresheaves}
Let $\sigma:\fX \ra \fY$ be a representable monomorphism of $k$-functors. Suppose that $\fY$ is a Zariski local $k$-functor. Then $\fX$ is a Zariski local $k$-functor.
\end{proposition}
\begin{proof}
Let $\big\{f_i:A\ra A_i\big\}_{i\in I}$ be a Zariski cover and let $x_i\in \fX(A_i)$ for $i\in I$ be a family of elements such that $\sigma(x_i) = \fY(f_i)(y)$ for some $y\in \fY(A)$. It suffices to prove that there exists $x\in \fX(A)$ such that $\sigma(x)=y$. Define $\tau:\fB_{\Spec A}\ra \fY$ as a morphism of $k$-functors corresponding to $y$. Since $\sigma$ is representable, there exists a $k$-scheme $X$, a morphism $g:X\ra \Spec A$ and the cartesian square
\begin{center}
\begin{tikzpicture}
[description/.style={fill=white,inner sep=2pt}]
\matrix (m) [matrix of math nodes, row sep=3em, column sep=3em,text height=1.5ex, text depth=0.25ex] 
{  \fB_X               & \fX           \\
   \fB_{\Spec A}    & \fY           \\} ;
\path[->,line width=1.0pt,font=\scriptsize]
(m-1-1) edge node[above] {$\tau'  $} (m-1-2)
(m-2-1) edge node[below] {$\tau  $} (m-2-2)
(m-1-2) edge node[right] {$ \sigma $} (m-2-2);
\path[right hook->,line width=1.0pt,font=\scriptsize]
(m-1-1) edge node[left] {$ \fB_g  $} (m-2-1);
\end{tikzpicture}
\end{center}
of $k$-functors. Now $\fB_g$ is a monomorphism of $k$-functors and hence by Fact \ref{fact:functorsofpoints} we deduce that $g$ is a monomorphism of $k$-functors. According to Corollary {\cite[Corollary 3.5]{Presheaves}} fiber products in categories of presheaves are taken pointwise. Thus we derive that for every $i\in I$ there exists $h_i\in \fB_X(A_i)$ such that $\tau'(h_i) = x_i$ and $\fB_g(h_i) = \Spec f_i$. Then $h_i:\Spec A_i\ra X$ is a morphism of $k$-schemes and $\Spec f_i = g\cdot h_i$ for every $i\in I$. Since $\Spec f_i$ is an open immersion and by cancellation for monomorphisms, we deduce that $h_i$ is an open immersion for every $i\in I$. Moreover, the morphism
$$h:\coprod_{i\in I}\Spec A_i\ra X$$
induced by $\{h_i\}_{i\in I}$ is surjective. Indeed, this follows by two observations: $g\cdot h:\coprod_{i\in I}\Spec A_i\ra \Spec A$ is surjective and $g$ is a monomorphism of schemes. Therefore, $g$ is surjective monomorphism and is open immersion locally on its domain. Thus $g$ is an isomorphism. Now for $x = \tau'(g^{-1})$ we have
$$\sigma(x) = \sigma\left(\tau'(g^{-1})\right) = \tau\left(g\cdot g^{-1}\right) = \tau(1_{\Spec A}) = y$$
This finishes the proof.
\end{proof}

\begin{proposition}\label{proposition:representablearerepresentableafterarbitrarybasechange}
Let $\sigma:\fX\ra \fY$ be a representable morphism of Zariski local $k$-functors. Fix a $k$-scheme $Y$ and a morphism $\tau:\fB_Y\ra \fY$. Then there exist a $k$-scheme $X$, a morphism $f:X\ra Y$ and a morphism $\tau':\fB_X\ra \fX$ such that the square
\begin{center}
\begin{tikzpicture}
[description/.style={fill=white,inner sep=2pt}]
\matrix (m) [matrix of math nodes, row sep=3em, column sep=3em,text height=1.5ex, text depth=0.25ex] 
{  \fB_{X}        & \fX           \\
   \fB_{Y}             & \fY           \\} ;
\path[->,line width=1.0pt,font=\scriptsize]
(m-1-1) edge node[above] {$ \tau' $} (m-1-2)
(m-2-1) edge node[below] {$ \tau $} (m-2-2)
(m-1-2) edge node[right] {$ \sigma $} (m-2-2)
(m-1-1) edge node[left] {$ \fB_f  $} (m-2-1);
\end{tikzpicture}
\end{center}
is cartesian.
\end{proposition}
\begin{proof}
Let
\begin{center}
\begin{tikzpicture}
[description/.style={fill=white,inner sep=2pt}]
\matrix (m) [matrix of math nodes, row sep=3em, column sep=3em,text height=1.5ex, text depth=0.25ex] 
{  \fZ        & \fX           \\
   \fB_{Y}    & \fY           \\} ;
\path[->,line width=1.0pt,font=\scriptsize]
(m-1-1) edge node[above] {$ \tau' $} (m-1-2)
(m-2-1) edge node[below] {$ \tau $} (m-2-2)
(m-1-2) edge node[right] {$ \sigma $} (m-2-2)
(m-1-1) edge node[left] {$ \sigma'  $} (m-2-1);
\end{tikzpicture}
\end{center}
be a cartesian square. According to {\cite[Theorem 2.12]{Sheaves}} $k$-functor $\fZ$ is Zariski local. Suppose that $\big\{f_i:\Spec A_i\ra Y\big\}_{i\in I}$ is an open cover of $Y$. Then $\big\{\fB_{f_i}:\fB_{\Spec A_i}\ra \fB_Y\big\}_{i\in I}$ is an open cover of $\fB_Y$ and hence its base change $\big\{\tau_i:\fZ_i\ra \fZ\big\}_{i\in I}$ is an open cover of $\fZ$. Since $\sigma$ is representable, we deduce that $\cZ_i$ is a functor of points of some $k$-scheme for $i\in I$. Now by Theorem \ref{theorem:representabilitybasicresult} we derive that there exists a $k$-scheme $X$ such that $\fZ$ is isomorphic with $\fB_X$. This proves the result.
\end{proof}
\noindent
The next result is frequently used in the theory of \textit{algebraic spaces}.

\begin{proposition}\label{proposition:universalrepresentabilityofmorphisms}
Let $\fY$ be a $k$-functor such that the diagonal $\fY\ra \fY\times \fY$ is representable. Then every morphism $\sigma:\fX\ra \fY$ of $k$-functors is representable.
\end{proposition}
\begin{proof}
Fix a morphism of $k$-functors $\sigma:\fX\ra \fY$. Let $Y$ be a $k$-scheme and let $\tau:\fB_Y\ra \fY$ be a morphism of $k$-functors. Consider the cartesian square
\begin{center}
\begin{tikzpicture}
[description/.style={fill=white,inner sep=2pt}]
\matrix (m) [matrix of math nodes, row sep=3em, column sep=3em,text height=1.5ex, text depth=0.25ex] 
{  \fZ        & \fX           \\
   \fB_{Y}    & \fY           \\} ;
\path[->,line width=1.0pt,font=\scriptsize]
(m-1-1) edge node[above] {$ \tau' $} (m-1-2)
(m-2-1) edge node[below] {$ \tau $} (m-2-2)
(m-1-2) edge node[right] {$ \sigma $} (m-2-2)
(m-1-1) edge node[left] {$ \sigma'  $} (m-2-1);
\end{tikzpicture}
\end{center}
Then there exists a cartesian square
\begin{center}
\begin{tikzpicture}
[description/.style={fill=white,inner sep=2pt}]
\matrix (m) [matrix of math nodes, row sep=3em, column sep=3em,text height=1.5ex, text depth=0.25ex] 
{  \fZ                  & \fY           \\
   \fB_{Y}\times \fY    & \fY\times \fY           \\} ;
\path[->,line width=1.0pt,font=\scriptsize]
(m-1-1) edge node[above] {$  $} (m-1-2)
(m-2-1) edge node[below] {$ \tau\times \sigma $} (m-2-2)
(m-1-2) edge node[right] {$ \textbf{diagonal} $} (m-2-2)
(m-1-1) edge node[left] {$   $} (m-2-1);
\end{tikzpicture}
\end{center}
Since the diagonal of $\fY$ is representable, we derive that $\fZ$ is isomorphic with functor of points of some $k$-scheme. This finishes the proof.
\end{proof}

\section{Transporters}

\begin{definition}
Let $X$ be a $k$-scheme. Suppose that there exists an open affine cover $X = \bigcup_{i\in I}X_i$ such that $k$-algebra $\Gamma(X_i,\cO_{X_i})$ is free as a $k$-module. Then we say that $X$ is \textit{a locally free $k$-scheme}.
\end{definition}
\noindent
Next theorem is the main result of this section.

\begin{theorem}\label{theorem:closedimmersionsandinternalhom}
Let $j:\fY'\ra \fY$ be a closed immersion of $k$-functors and $X$ be a locally free $k$-scheme. Suppose that classes $\Mor_A\left(X_A,\fY_A\right)$ are sets for every $k$-algebra $A$. Then classes $\Mor_A\left(X_A,\fY'_A\right)$ are sets for every $k$-algebra $A$ and the morphism
$$\iMor_k\left(1_X,j\right):\iMor_k\left(X,\fY'\right)\ra \iMor_k\left(X,\fY\right)$$
is a closed immersion of $k$-functors.
\end{theorem}
\noindent
It is useful to isolate crucial steps in the argument. For this we proceed by proving some lemmas.

\begin{lemma}\label{lemma:foraffinelocalfactorization}
Suppose that $A$ is a commutative ring. Let $j:\fY'\ra \fY$ be a closed immersion of $A$-functors and $X$ be an affine $A$-scheme such that $\Gamma(X,\cO_X)$ is a free $A$-module. Assume that $\tau:X\ra \fY$ is a morphism of $A$-functors. Then there exists an ideal $\ideal{a}\subseteq A$ such that for every $A$-algebra $B$ the restriction $\tau_B$ factors through $j_B$ if and only if the structure morphism $f:A\ra B$ of $B$ satisfies $\ideal{a}\subseteq \ker(f)$.
\end{lemma}
\begin{proof}[Proof of the lemma]
Consider a cartesian square
\begin{center}
\begin{tikzpicture}
[description/.style={fill=white,inner sep=2pt}]
\matrix (m) [matrix of math nodes, row sep=3em, column sep=3em,text height=1.5ex, text depth=0.25ex] 
{  X' & \fY' \\
   X  & \fY           \\} ;
\path[->,line width=1.0pt,font=\scriptsize]  
(m-1-1) edge node[above] {$ $} (m-1-2)
(m-2-1) edge node[below] {$\tau  $} (m-2-2)
(m-1-1) edge node[left] {$ j' $} (m-2-1)
(m-1-2) edge node[right] {$ j $} (m-2-2);
\end{tikzpicture}
\end{center}
Since $j$ is a closed immersion of $A$-functors, we derive by Fact \ref{fact:openclosedimmersionsclosedunderbasechange} that $j'$ is a closed immersion. By assumption $X$ is affine. Hence $X'$ is a functor of points of some $A$-scheme and $j':X'\ra X$ is (induced by) a closed immersion of $A$-schemes. Next let $B$ be an $A$-algebra with the structure morphism $f:A\ra B$. Then $\tau_B$ factors through $j_B$ if and only if the projection $\Spec B\times_{\Spec A}X\ra X$ induced by $f$ factors through $X'$. Let $A[X]$ be the $A$-algebra of global regular functions on $X$ and let $\ideal{J}$ be an ideal in $A[X]$ such that $A[X]/\ideal{J} = A[X']$ is the $A$-algebra of global regular functions of $X'$. With this notation we derive that the projection $\Spec B\times_{\Spec A}X\ra X$ induced by $f$ factors through $X'$ if and only if the morphism $A[X]\ra B\otimes_AA[X]$ induced by $f$ sends every element of $\ideal{J}$ to zero. Since $A[X]$ is a free $A$-module, we write $A[X] = A^{\oplus I}$ for some index set $I$. Then the morphism $A[X]\ra B\otimes_AA[X]$ induced by $f$ is just $f^{\oplus I}:A^{\oplus I}\ra B^{\oplus I}$. We have $f^{\oplus I}\left(\ideal{J}\right)=0$ if and only if $\left(pr^B_i\cdot f^{\oplus I}\right)\left(\ideal{J}\right)=$ for every $i\in I$, where $pr^B_i:B^{\oplus I}\ra B$ is the projection on $i$-th component. Pick $i\in I$ and consider the commutative diagram
\begin{center}
\begin{tikzpicture}
[description/.style={fill=white,inner sep=2pt}]
\matrix (m) [matrix of math nodes, row sep=3em, column sep=3em,text height=1.5ex, text depth=0.25ex] 
{  A^{\oplus I} & B^{\oplus I}  \\
   A  & B           \\} ;
\path[->,line width=1.0pt,font=\scriptsize]  
(m-1-1) edge node[above] {$ f^{\oplus I} $} (m-1-2)
(m-2-1) edge node[below] {$ f  $} (m-2-2)
(m-1-1) edge node[left] {$ pr^A_i $} (m-2-1)
(m-1-2) edge node[right] {$ pr^B_i $} (m-2-2);
\end{tikzpicture}
\end{center}
In the diagram $pr^A_i$ is the projection on $i$-th component. Diagram implies that $\left(pr^B_i\cdot f^{\oplus I}\right)\left(\ideal{J}\right)=$ for every $i\in I$ if and only if $\left(f\cdot pr_i^A\right)(\ideal{J}) = 0$ for every $i\in I$. This is equivalent with the condition that $f(\ideal{a})=0$ for ideal $\ideal{a}$ in $A$ generated by $\sum_{i\in I}pr_i^A(\ideal{J})$. Thus the lemma is proved.
\end{proof}

\begin{lemma}\label{lemma:coveringsandfactorizations}
Suppose that $A$ is a commutative ring. Let $j:\fY'\ra \fY$ be a closed immersion of $A$-functors and $X$ be an $A$-scheme with open cover
$$X=\bigcup_{i\in I}X_i$$
Assume that $\tau:X\ra \fY$ is a morphism of $A$-functors. Fix an $A$-algebra $B$. Then $\tau_B$ factors through $j_B$ if and only if $\left(\tau_{\mid X_i}\right)_B$ factors through $j_B$ for every $i\in I$.
\end{lemma}
\begin{proof}[Proof of the lemma]
If $\tau_B$ factors through $j_B$, then also $\left(\tau_{\mid X_i}\right)_B$ factors through $j_B$ for every $i\in I$. It suffices to prove the converse. So suppose that $\left(\tau_{\mid X_i}\right)_B$ factors through $j_B$ for every $i\in I$. Since $j$ is a closed immersion of $A$-functors and $X$ is an $A$-scheme, there exists a cartesian square
\begin{center}
\begin{tikzpicture}
[description/.style={fill=white,inner sep=2pt}]
\matrix (m) [matrix of math nodes, row sep=3em, column sep=3em,text height=1.5ex, text depth=0.25ex] 
{  X' & \fY' \\
   X  & \fY           \\} ;
\path[->,line width=1.0pt,font=\scriptsize]  
(m-1-1) edge node[above] {$ $} (m-1-2)
(m-2-1) edge node[below] {$\tau  $} (m-2-2)
(m-1-1) edge node[left] {$ j' $} (m-2-1)
(m-1-2) edge node[right] {$ j $} (m-2-2);
\end{tikzpicture}
\end{center}
where $j':X'\ra X$ is (induced by) a closed immersion of $A$-schemes (this follows from Fact \ref{fact:openclosedimmersionsclosedunderbasechange}  and Fact \ref{fact:kschemesandkfunctors}). For each $i\in I$ let $j'_i:j'^{-1}(X_i)\ra X_i$ be the restriction of $j'$. We have the induced cartesian square
\begin{center}
\begin{tikzpicture}
[description/.style={fill=white,inner sep=2pt}]
\matrix (m) [matrix of math nodes, row sep=3em, column sep=3em,text height=1.5ex, text depth=0.25ex] 
{  j'^{-1}(X_i) & \fY' \\
   X_i  & \fY           \\} ;
\path[->,line width=1.0pt,font=\scriptsize]  
(m-1-1) edge node[above] {$ $} (m-1-2)
(m-2-1) edge node[below] {$\tau_{\mid X_i}  $} (m-2-2)
(m-1-1) edge node[left] {$j'_i  $} (m-2-1)
(m-1-2) edge node[right] {$ j $} (m-2-2);
\end{tikzpicture}
\end{center}
Now $\left(\tau_{\mid X_i}\right)_B$ factors through $j_B$. Together with Fact \ref{fact:kschemesandkfunctors} this shows that $\left(j'_i\right)_B$ is an isomorphism of $B$-schemes. This holds for every $i\in I$. Hence $j'_B$ is an isomorphism of $B$-schemes (again by application of Fact \ref{fact:kschemesandkfunctors}). Therefore, $\tau_B$ factors through $j_B$.
\end{proof}

\begin{proof}[Proof of the theorem]
Let $A$ be a $k$-algebra. The restriction functor $(-)_{\mid \Alg_A} = (-)_A$ preserves all closed immersions. Thus $j_A$ is a closed immersion of $A$-functors and hence we derive that $j_A:\fY'_A\ra \fY_A$ is a monomorphism of $A$-functors. Thus we have an injective  map of classes
$$\Mor_A\left(1_{X_A},j_A\right):\Mor_A\left(X_A,\fY'_A\right)\hookrightarrow \Mor_A\left(X_A,\fY_A\right)$$
Hence if $\Mor_A\left(X_A,\fY_A\right)$ is a set, then $\Mor_A\left(X_A,\fY'_A\right)$ is a set. All these facts imply that both internal homs
$$\iMor_k\left(X,\fY'\right),\,\iMor_k\left(X,\fY\right)$$
exist and morphism $\iMor_k(1_X,j)$ of $k$-functors is a monomorphism. Our task is to prove that it is a closed immersion. For this consider a $k$-algebra $A$ and a morphism $\sigma:k_A\ra \iMor_k\left(X,\fY\right)$ of $k$-functors that sends $1_A$ to some morphism $\tau:X_A\ra \fY_A$ of $A$-functors. Consider a cartesian square
\begin{center}
\begin{tikzpicture}
[description/.style={fill=white,inner sep=2pt}]
\matrix (m) [matrix of math nodes, row sep=3em, column sep=3em,text height=1.5ex, text depth=0.25ex] 
{  \fU  & \iMor_k\left(X,\fY'\right) \\
   k_A  & \iMor_k\left(X,\fY\right)           \\} ;
\path[->,line width=1.0pt,font=\scriptsize]  
(m-1-1) edge node[above] {$ $} (m-1-2)
(m-2-1) edge node[below] {$ \sigma $} (m-2-2)
(m-1-1) edge node[left] {$  $} (m-2-1)
(m-1-2) edge node[right] {$ \iMor_k\left(1_X,j\right) $} (m-2-2);
\end{tikzpicture}
\end{center}
Since $\iMor_k\left(1_X,j\right)$ is a monomorphism, we may consider $\fU$ as a $k$-subfunctor of $k_A$. For every $k$-algebra $B$ subset $\fU(B)\subseteq \Mor_k(A,B)= k_A(B)$ consists of $A$-algebras $B$ with structure morphisms $f:A\ra B$ such that $\tau_B$ factors through $j_B:\fY'_B\ra \fY_B$. Since $X$ is a locally free $k$-scheme, we deduce that $X_A$ is (a functor of points of) a locally free $A$-scheme. Pick an open affine cover $X_A = \bigcup_{i\in I}X_i$ such that $\Gamma(X_i,\cO_X)$ is a free $A$-module. Now Lemma \ref{lemma:coveringsandfactorizations} implies that $\tau_B$ factors through $j_B$ if and only if $\left(\tau_{\mid X_i}\right)_B$ factors through $j_B$ for every $i\in I$. Next by Lemma \ref{lemma:foraffinelocalfactorization} we deduce that $\left(\tau_{\mid X_i}\right)_B$  factors through $j_B$ for given $i\in I$ if and only if $f(\ideal{a}_i)=0$ for some ideal $\ideal{a}_i\subseteq A$ independent of $f$. Thus $\fU$ consists of all morphisms $f:A\ra B$ of $k$-algebras such that $f(\ideal{a})=0$ where $\ideal{a} = \sum_{i\in I}\ideal{a}_i$. Therefore, $\fU\hookrightarrow k_A$ is isomorphic with $k_{A/\ideal{a}}\hookrightarrow k_A$ and hence $\iMor_k(1_X,j)$ is a closed immersion of $k$-functors.
\end{proof}
\noindent
The Theorem \ref{theorem:closedimmersionsandinternalhom} is a simple yet powerful result. Before giving any interesting applications we state its immediate consequence.

\section{Equivalence relations and quotients}
\noindent
In this section we introduce internal binary relations in categories. The first part of the section are categorical reformulations and generalizations of standard notions. We encourage the reader to interpret everything that we introduce in the category of sets.

\begin{definition}
Let $\cC$ be a category with products and let $X$ be an object of $\cC$. Suppose that $p_1,p_2:R\ra X$ are morphisms such that $\langle p_1, p_2\rangle:R\ra X\times X$ is a monomorphism. Then a triple $(R,p_1,p_2)$ is called \textit{a binary relation on $X$}.
\end{definition}
\noindent
If $\cC$ is a category with products and $(R,p_1,p_2)$ is a binary relation on some object $X$ of $\cC$, then for every object $Y$ of $\cC$ we have the inclusion
\begin{center}
\begin{tikzpicture}
[description/.style={fill=white,inner sep=2pt}]
\matrix (m) [matrix of math nodes, row sep=3em, column sep=8em,text height=1.5ex, text depth=0.25ex] 
{ \Mor_{\cC}\left(Y,R\right)  &  \Mor_{\cC}\left(Y,X\times X\right) = \Mor_{\cC}\left(Y,X\right)\times \Mor_{\cC}\left(Y,X\right)   \\} ;
\path[right hook->,line width=1.0pt,font=\scriptsize]  
(m-1-1) edge node[above] {$ \Mor_{\cC}\left(1_Y,\langle p_1,p_2 \rangle\right) $} (m-1-2);
\end{tikzpicture}
\end{center}
and thus we can interpret it as a binary relation on the class $\Mor_{\cC}\left(Y,X\right)$.\\
Suppose now that $\cC$ is a category with fiber products and $(R,p_1,p_2)$ is a binary relation on some object $X$ of $\cC$. Consider a cartesian square
\begin{center}
\begin{tikzpicture}
[description/.style={fill=white,inner sep=2pt}]
\matrix (m) [matrix of math nodes, row sep=3em, column sep=3em,text height=1.5ex, text depth=0.25ex] 
{ T  &    R   \\
  R  &    X   \\} ;
\path[->,line width=1.0pt,font=\scriptsize]  
(m-1-1) edge node[above] {$ q_2 $} (m-1-2)
(m-2-1) edge node[below] {$ p_2 $} (m-2-2)
(m-1-1) edge node[left] {$ q_1 $} (m-2-1)
(m-1-2) edge node[right] {$ p_1 $} (m-2-2);
\end{tikzpicture}
\end{center}
Denote by $r$ morphism $p_2\cdot q_1 = p_1\cdot q_2$. Let $\pi_1,\pi_3:X\times X\times X\ra $ be projections on first and third factor, respectively. The morphism $\langle p_1\cdot q_1, r, p_2\cdot q_2\rangle:T\ra X\times X \times X$ composed with the morphism $\langle \pi_1,\pi_3\rangle:X\times X\times X\ra X\times X$ factors through $\langle p_1,p_2\rangle:R\hookrightarrow X\times X$ if there exists $t:T\ra R$ such that the square
\begin{center}
\begin{tikzpicture}
[description/.style={fill=white,inner sep=2pt}]
\matrix (m) [matrix of math nodes, row sep=4em, column sep=6em,text height=1.5ex, text depth=0.25ex] 
{ T  &   X\times X\times X   \\
  R  &    X\times X                   \\} ;
\path[->,line width=1.0pt,font=\scriptsize]
(m-1-1) edge node[above] {$\langle p_1\cdot q_1, r, p_2\cdot q_2\rangle  $} (m-1-2)
(m-1-2) edge node[right] {$ \langle \pi_1,\pi_3\rangle $} (m-2-2);
\path[densely dotted,->,line width=1.0pt,font=\scriptsize]
(m-1-1) edge node[left] {$ t $} (m-2-1);
\path[right hook->,line width=1.0pt,font=\scriptsize]
(m-2-1) edge node[below] {$ \langle p_1, p_2\rangle $} (m-2-2);
\end{tikzpicture}
\end{center}
is commutative. Note that $t$ is unique. Also from the diagram it follows that $t:T\ra R$ can be described as the unique morphism such that $p_1\cdot t =  p_1\cdot q_1$ and $p_2\cdot t = p_2\cdot q_2$.

\begin{definition}
Let $\cC$ be a category with fiber products and let $(R,p_1,p_2)$ be a binary relation on some object $X$ of $\cC$.
\begin{enumerate}[label=\textbf{(\arabic*)}, leftmargin=1.5em]
\item $(R,p_1,p_2)$ is \textit{reflexive} if the diagonal $\delta:X \ra X\times X$ factors through $\langle p_1,p_2\rangle$.
\item $(R,p_1,p_2)$ is \textit{symmetric} if $\langle p_1,p_2\rangle = s\cdot \langle p_1\cdot p_2\rangle$, there $s:X\times X\ra X\times X$ is the cartesian symmetry. 
\item Consider the cartesian square
\begin{center}
\begin{tikzpicture}
[description/.style={fill=white,inner sep=2pt}]
\matrix (m) [matrix of math nodes, row sep=3em, column sep=3em,text height=1.5ex, text depth=0.25ex] 
{ T  &    R   \\
  R  &    X   \\} ;
\path[->,line width=1.0pt,font=\scriptsize]  
(m-1-1) edge node[above] {$ q_2 $} (m-1-2)
(m-2-1) edge node[below] {$ p_2 $} (m-2-2)
(m-1-1) edge node[left] {$ q_1 $} (m-2-1)
(m-1-2) edge node[right] {$ p_1 $} (m-2-2);
\end{tikzpicture}
\end{center}
$(R,p_1,p_2)$ is \textit{transitive} if there exists $t:T\ra R$ such that $p_1\cdot t = p_1\cdot q_1$ and $p_2\cdot t = p_2\cdot q_2$.
\item Finally, we say that $(R,p_1,p_2)$ is \textit{an equivalence relation} if it is reflexive, symmetric and transitive. 
\end{enumerate}
\end{definition}

\begin{fact}\label{fact:relationsintermsofpoints}
Let $\cC$ be a category with fiber products and let $(R,p_1,p_2)$ be a binary relation on some object $X$ of $\cC$. Then the following assertions are equivalent.
\begin{enumerate}[label=\emph{\textbf{(\roman*)}}, leftmargin=1.5em]
\item For every object $Y$ of $\cC$ the binary relation
\begin{center}
\begin{tikzpicture}
[description/.style={fill=white,inner sep=2pt}]
\matrix (m) [matrix of math nodes, row sep=3em, column sep=8em,text height=1.5ex, text depth=0.25ex] 
{ \Mor_{\cC}\left(Y,R\right)  &  \Mor_{\cC}\left(Y,X\times X\right) = \Mor_{\cC}\left(Y,X\right)\times \Mor_{\cC}\left(Y,X\right)   \\} ;
\path[right hook->,line width=1.0pt,font=\scriptsize]  
(m-1-1) edge node[above] {$ \Mor_{\cC}\left(1_Y,\langle p_1,p_2 \rangle\right) $} (m-1-2);
\end{tikzpicture}
\end{center}
is reflexive (symmetric, transitive).
\item $(R,p_1,p_2)$ is reflexive (symmetric, transitive).
\end{enumerate}
\end{fact}
\begin{proof}
We may pass to larger Grothendieck universe $V$ such that $\cC$ is locally $V$-small. Then one can use the standard Yoneda lemma in order to derive translation from \textbf{(i)} and \textbf{(ii)}.
\end{proof}

\begin{proposition}\label{proposition:eachmorphismgivesequivalencerelation}
Let $\cC$ be a category with fiber products and let $f:X\ra Y$ be a morphism $\cC$. Consider a cartesian square
\begin{center}
\begin{tikzpicture}
[description/.style={fill=white,inner sep=2pt}]
\matrix (m) [matrix of math nodes, row sep=3em, column sep=3em,text height=1.5ex, text depth=0.25ex] 
{ X \times_Y X  &    X   \\
  X             &    Y   \\};
\path[->,line width=1.0pt,font=\scriptsize]  
(m-1-1) edge node[above] {$ p_2 $} (m-1-2)
(m-2-1) edge node[below] {$ f $} (m-2-2)
(m-1-1) edge node[left] {$ p_1 $} (m-2-1)
(m-1-2) edge node[right] {$ f $} (m-2-2);
\end{tikzpicture}
\end{center}
Then $\left(X \times_Y X,p_1,p_2\right)$ is an equivalence relation on $X$.
\end{proposition}
\begin{proof}
For the proof pick an object of $Z$ and by Fact \ref{fact:relationsintermsofpoints} it suffices to prove that
\begin{center}
\begin{tikzpicture}
[description/.style={fill=white,inner sep=2pt}]
\matrix (m) [matrix of math nodes, row sep=3em, column sep=8em,text height=1.5ex, text depth=0.25ex] 
{ \Mor_{\cC}\left(Z,X\times_Y X\right)  &  \Mor_{\cC}\left(Z,X\times X\right) = \Mor_{\cC}\left(Z,X\right)\times \Mor_{\cC}\left(Z,X\right)   \\} ;
\path[right hook->,line width=1.0pt,font=\scriptsize]  
(m-1-1) edge node[above] {$ \Mor_{\cC}\left(1_Z,\langle p_1,p_2 \rangle\right) $} (m-1-2);
\end{tikzpicture}
\end{center}
is an equivalence relation. Note that we have fiber product of classes
\begin{center}
\begin{tikzpicture}
[description/.style={fill=white,inner sep=2pt}]
\matrix (m) [matrix of math nodes, row sep=5em, column sep=5em,text height=1.5ex, text depth=0.25ex] 
{ \Mor_{\cC}\left(Z,X \times_Y X\right)  &   \Mor_{\cC}\left(Z,X\right)   \\
 \Mor_{\cC}\left(Z, X\right)             &   \Mor_{\cC}\left(Z, Y\right)   \\};
\path[->,line width=1.0pt,font=\scriptsize]  
(m-1-1) edge node[above] {$ \Mor_{\cC}\left(1_Z,p_2\right) $} (m-1-2)
(m-2-1) edge node[below] {$ \Mor_{\cC}\left(1_Z,f\right) $} (m-2-2)
(m-1-1) edge node[left] {$  \Mor_{\cC}\left(1_Z, p_1\right) $} (m-2-1)
(m-1-2) edge node[right] {$ \Mor_{\cC}\left(1_Z,f\right) $} (m-2-2);
\end{tikzpicture}
\end{center}
and hence $\Mor_{\cC}\left(Z, X\times_Y X\right)$ contains these pairs $(g,h)$ of morphisms in $\Mor_{\cC}(Z,X)$ such that $f\cdot g = f\cdot h$. This is clearly an equivalence relation.
\end{proof}


\begin{definition}
Let $\cC$ be a category with fiber products and let $f:X\ra Y$ be a morphism in $\cC$. Consider a cartesian square
\begin{center}
\begin{tikzpicture}
[description/.style={fill=white,inner sep=2pt}]
\matrix (m) [matrix of math nodes, row sep=3em, column sep=3em,text height=1.5ex, text depth=0.25ex] 
{ X \times_Y X  &    X   \\
  X             &    Y   \\};
\path[->,line width=1.0pt,font=\scriptsize]  
(m-1-1) edge node[above] {$ p_2 $} (m-1-2)
(m-2-1) edge node[below] {$ f $} (m-2-2)
(m-1-1) edge node[left] {$ p_1 $} (m-2-1)
(m-1-2) edge node[right] {$ f $} (m-2-2);
\end{tikzpicture}
\end{center}
Then the equivalence relation $\left(X\times_YX,p_1,p_2\right)$ is a called \textit{a kernel pair of $f$}.
\end{definition}

\begin{definition}
Let $\cC$ be a category with fiber products and let $(R,p_1,p_2)$ be a equivalence relation on some object $X$ of $\cC$. The morphism $f:X\ra Y$ is \textit{a quotient of $(R,p_1,p_2)$} if the fork
\begin{center}
\begin{tikzpicture}
[description/.style={fill=white,inner sep=2pt}]
\matrix (m) [matrix of math nodes, row sep=3em, column sep=3em,text height=1.5ex, text depth=0.25ex] 
{R &  X & Y  \\} ;
\path[->,line width=0.8pt,font=\scriptsize]
(m-1-1) edge[transform canvas={yshift=0.5ex}] node[above] {$ p_1  $} (m-1-2)
(m-1-1) edge[transform canvas={yshift=-0.5ex}] node[below] {$ p_2 $} (m-1-2)
(m-1-2) edge node[above] {$ f  $} (m-1-3);
\end{tikzpicture}
\end{center}
is a cokernel of $(p_1,p_2)$. An equivalence relation that admits a quotient is called \textit{effective}.
\end{definition}

\begin{proposition}\label{proposition:coequalizersarequotientsoftheirkernelpairs}
Let $\cC$ be a category and let $g_1,g_2:Z\ra X$ be morphisms in $\cC$. Suppose that
\begin{center}
\begin{tikzpicture}
[description/.style={fill=white,inner sep=2pt}]
\matrix (m) [matrix of math nodes, row sep=3em, column sep=3em,text height=1.5ex, text depth=0.25ex] 
{Z &  X & Y  \\} ;
\path[->,line width=0.8pt,font=\scriptsize]
(m-1-1) edge[transform canvas={yshift=0.5ex}] node[above] {$ g_1  $} (m-1-2)
(m-1-1) edge[transform canvas={yshift=-0.5ex}] node[below] {$ g_2 $} (m-1-2)
(m-1-2) edge node[above] {$ f  $} (m-1-3);
\end{tikzpicture}
\end{center}
is a cokernel of $(g_1,g_2)$. Then $f$ is also a cokernel of its kernel pair.
\end{proposition}
\begin{proof}
Let $\left(X\times_YX, p_1,p_2\right)$ be a kernel pair of $f$. Since $f\cdot g_1 = f\cdot g_2$, we derive that there exists a unique morphism $u:Z\ra X\times_Y X$ such that $g_1 = p_1\cdot u$ and $g_2 = p_2\cdot u$. Hence we have a commutative diagram
\begin{center}
\begin{tikzpicture}
[description/.style={fill=white,inner sep=2pt}]
\matrix (m) [matrix of math nodes, row sep=3em, column sep=3em,text height=1.5ex, text depth=0.25ex] 
{         Z  &  X  &      Y  \\
X\times_Y X  &  X  &        \\};
\path[->,line width=0.8pt,font=\scriptsize]
(m-1-1) edge[transform canvas={yshift=0.5ex}] node[above] {$ g_1  $} (m-1-2)
(m-1-1) edge[transform canvas={yshift=-0.5ex}] node[below] {$ g_2 $} (m-1-2)
(m-1-2) edge node[above] {$ f  $} (m-1-3)
(m-2-1) edge[transform canvas={yshift=0.5ex}] node[above] {$ p_1  $} (m-2-2)
(m-2-1) edge[transform canvas={yshift=-0.5ex}] node[below] {$ p_2 $} (m-2-2)
(m-1-1) edge node[left] {$ u  $} (m-2-1)
(m-1-2) edge node[left] {$ =  $} (m-2-2);
\end{tikzpicture}
\end{center}
Suppose now that $h:X\ra T$ is a morphism such that $h \cdot p_1 = h\cdot p_2$. Then
$$h\cdot g_1 = h\cdot p_1\cdot u = h\cdot p_1\cdot u = h\cdot g_2$$
and hence there exists a unique $k:Y\ra T$ such that $k \cdot f = h$. This finishes the proof.
\end{proof}

\begin{proposition}
Let $\cC$ be a category with fiber products and let $(R,p_1,p_2)$ be an equivalence relation on some object $X$ of $\cC$. If $(R,p_1,p_2)$ is effective, then it is a kernel pair of its quotient.
\end{proposition}
\begin{proof}
Let $f:X\ra Y$ be a quotient of $(R,p_1,p_2)$.
\end{proof}

\begin{proposition}
Let $\sigma:\fX\ra \fY$ be a representable morphism of $k$-functors. Then for every $k$-scheme $Y$ and every morphism $\tau:\fB_Y\ra \fY$ of $K$-functors there exist morphism of $k$-schemes $f:X\ra Y$ and a morphism $\tau':\fB_X\ra \fX$ of $k$-functors such that the square 
\begin{center}
\begin{tikzpicture}
[description/.style={fill=white,inner sep=2pt}]
\matrix (m) [matrix of math nodes, row sep=3em, column sep=3em,text height=1.5ex, text depth=0.25ex] 
{ \fB_X    & \fX           \\
  \fB_Y         & \fY           \\} ;
\path[->,line width=1.0pt,font=\scriptsize]
(m-1-1) edge node[above] {$ \tau' $} (m-1-2)
(m-2-1) edge node[below] {$ \tau $} (m-2-2)
(m-1-2) edge node[right] {$ \sigma $} (m-2-2)
(m-1-1) edge node[left] {$ \fB_f  $} (m-2-1);
\end{tikzpicture}
\end{center}
is cartesian.
\end{proposition}
\begin{proof}
Note that representable morphisms of $k$-functors are closed under base change.
\end{proof}

\begin{definition}
Let $\sigma:\fX\ra \fY$ be a morphism of $k$-functors. Assume that for every $k$-algebra $A$ and every morphism $\tau:\fB_{\Spec_A}\ra \fY$ of $k$-functors there exist an ideal $\ideal{a}$ in $A$ and morphism $\tau':\fB_{\Spec A/\ideal{a}}\ra \fX$ such that the square
\begin{center}
\begin{tikzpicture}
[description/.style={fill=white,inner sep=2pt}]
\matrix (m) [matrix of math nodes, row sep=3em, column sep=3em,text height=1.5ex, text depth=0.25ex] 
{  \fB_{V(\ideal{a})}  & \fX           \\
   \fB_{\Spec A}  & \fY           \\} ;
\path[->,line width=1.0pt,font=\scriptsize]  
(m-1-1) edge node[above] {$\tau' $} (m-1-2)
(m-2-1) edge node[below] {$ \tau $} (m-2-2)
(m-1-1) edge node[left] {$\fB_{\Spec q}  $} (m-2-1)
(m-1-2) edge node[right] {$ \sigma $} (m-2-2);
\end{tikzpicture}
\end{center}
is cartesian, where $q:A\ra A/\ideal{a}$ is the quotient morphism of $k$-algebras. Then $\sigma$ is \textit{a closed immersion of $k$-functors}.
\end{definition}

\begin{proposition}\label{proposition:openimmersionsbasechange}
Let $\sigma:\fX\ra \fY$ be an open immersion of $k$-functors. Suppose that $Y$ is a $k$-scheme and assume that $\tau:\fB_Y\ra \fY$ is a morphism of $k$-functors. Then there exists a $k$-scheme $X$, an open immersion $f:X\ra Y$ of schemes and a morphism $\tau':\fB_X\ra \fX$ such that the square
\begin{center}
\begin{tikzpicture}
[description/.style={fill=white,inner sep=2pt}]
\matrix (m) [matrix of math nodes, row sep=3em, column sep=3em,text height=1.5ex, text depth=0.25ex] 
{  \fB_X               & \fX           \\
   \fB_{Y}             & \fY           \\} ;
\path[->,line width=1.0pt,font=\scriptsize]
(m-1-1) edge node[above] {$ \tau' $} (m-1-2)
(m-2-1) edge node[below] {$ \tau $} (m-2-2)
(m-1-2) edge node[right] {$ \sigma $} (m-2-2)
(m-1-1) edge node[left] {$ \fB_f  $} (m-2-1);
\end{tikzpicture}
\end{center}
is cartesian.
\end{proposition}



\small
\bibliographystyle{alpha}
\bibliography{../zzz}


\end{document}