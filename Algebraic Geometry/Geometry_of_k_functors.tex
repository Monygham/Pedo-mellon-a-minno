\input pree.tex

\begin{document}
\title{Geometry of $k$-functors}
\date{}
\maketitle

\section{Introduction}
\noindent
In this notes we present functorial approach to algebraic geometry. Our aim is to show that functorial and geometrical techniques are interrelated in a very efficient way.

\section{\textit{k}-functors}

\begin{definition}
The category $\Fun(\Alg_k,\Set)$ of copresheaves on $\Alg_k$ is called \textit{the category of $k$-functors}.
\end{definition}
\noindent
If $\fX$ and $\fY$ are $k$-functors, then we denote by $\Mor_k(\fX,\fY)$ the class of morphisms $\fX\ra \fY$ of $k$-functors.\\
Since the category of $k$-functors is a category of copresheaves, under assumptions specified in {\cite[section 5]{Presheaves}} for given $k$-functors $\fX$, $\fY$ there exists an internal hom $\iMor_k(\fX,\fY)$. Let us discuss this important notion and also related ones. For details and proofs for general case we refer to {\cite[section 5]{Presheaves}}.\\
Let $\fX$ and $\fY$ be $A$-functors for some $k$-algebra $A$. Then we denote by $\Mor_A\left(\fX,\fY\right)$ the class of morphisms of $A$-functors $\fX\ra \fY$. For every $A$-algebra $B$ and a morphism $\sigma:\fX\ra \fY$ of $A$-functors we denote by $\fX_{B}$, $\fY_{B}$, $\sigma_{B}$ the restrictions $\fX_{\mid \Alg_B}$, $\fY_{\mid \Alg_B}$, $\sigma_{\mid \Alg_B}$ of these entities to the category of $B$-algebras. 

\begin{fact}\label{fact:restrictionsworkasexpected}
Let $\fX$ and $\fY$ be $k$-functors. Assume that $A$ is a $k$-algebra, $B$ is an $A$-algebra, $C$ is an $B$-algebra. Then the composition of maps of classes
\begin{center}
\begin{tikzpicture}
[description/.style={fill=white,inner sep=2pt}]
\matrix (m) [matrix of math nodes, row sep=3em, column sep=3em,text height=1.5ex, text depth=0.25ex] 
{ \Mor_A\left(\fX_A,\fY_A\right) &  \Mor_B\left(\fX_B,\fY_B\right) & \Mor_C\left(\fX_C,\fY_C\right)\\} ;
\path[->,line width=1.0pt,font=\scriptsize]  
(m-1-1) edge node[above] {$\sigma\mapsto \sigma_{B} $} (m-1-2)
(m-1-2) edge node[above] {$\sigma\mapsto \sigma_{B} $} (m-1-3);
\end{tikzpicture}
\end{center}
equals
\begin{center}
\begin{tikzpicture}
v[description/.style={fill=white,inner sep=2pt}]
\matrix (m) [matrix of math nodes, row sep=3em, column sep=3em,text height=1.5ex, text depth=0.25ex] 
{ \Mor_A\left(\fX_A,\fY_A\right) &  \Mor_C\left(\fX_C,\fY_C\right)\\} ;
\path[->,line width=1.0pt,font=\scriptsize]  
(m-1-1) edge node[above] {$\sigma\mapsto \sigma_C $} (m-1-2);
\end{tikzpicture}
\end{center}
\end{fact}
\begin{proof}
Left to the reader.
\end{proof}

\begin{definition}
Let $\fX$ and $\fY$ be $k$-functors and suppose that for every $k$-algebra $A$ the class $\Mor_A\left(\fX_A,\fY_A\right)$ is a set. We define
$$\iMor_k(\fX,\fY)(A)=\Mor_A\left(\fX_A,\fY_A\right)$$
for every $k$-algebra $A$. This is a $k$-functor, since for every $k$-algebra $A$ and $A$-algebra $B$, we can compose a morphism $\sigma:\fX_A\ra \fY_A$ of $k$-functors with the forgetful functor $\Alg_B \ra \Alg_A$ i.e. we have a map 
$$\iMor_{k}(\fX,\fY)(A)\ni \sigma \mapsto \sigma_{B}\in \iMor_{k}(\fX,\fY)(B)$$
and these according to Fact \ref{fact:restrictionsworkasexpected} make $\iMor_{k}(\fX,\fY)$ a $k$-functor. The $k$-functor $\iMor_{\cC}(\fX,\fY)$ is called \textit{a hom $k$-functor of $\fX$ and $\fY$}.
\end{definition}
\noindent
We define a $k$-functor $\bd{1}$ that assigns to every $k$-algebra a set with one element. For every $k$-algebra $A$ the restriction $\bd{1}_A$ is a terminal object in the category of $A$-functors.

\begin{fact}\label{fact:points}
Let $\fX$ be a $k$-functor. Suppose $A$ is a $k$-algebra and $x\in \fX(A)$. Then $x$ determines a morphism $\bd{1}_{A}\ra \fX_A$ that for every $A$-algebra $B$ with structural morphism $f:A\ra B$ sends a unique element of $\bd{1}_{A}(B)$ to $\fX(f)(x)\in \fX_A(B)$. This gives rise to a bijection
$$\fX(A)\cong \Mor_{A}\left(\bd{1}_{A},\fX_A\right)$$
\end{fact}
\begin{proof}
We left to the reader as an exercise.
\end{proof}

\begin{definition}
Let $\fX$ be a $k$-functor and $A$ be a $k$-algebra. The set $\fX(A)$ is called \textit{the set of $A$-points of $\fX$}.
\end{definition}
\noindent
Now let $\fX$, $\fY$ be $k$-functors such that for every $k$-algebra $A$ the class $\Mor_A\left(\fX_A,\fY_A\right)$ is a set. Suppose next that $\fU$ is a $k$-functor and $\sigma:\fU\times \fX\ra \fY$ is a morphism of $k$-functors. Fix $x\in \fU(A)$. We denote by $i_x:\bd{1}_A\ra \fU_A$ the morphism of $A$-functors corresponding to $x$ by means of Fact \ref{fact:points}. Since $\bd{1}_A$ is terminal $A$-functor, a morphism $\sigma_A\cdot \left(1_{\fX_A}\times i_x\right)$ is isomorphic to a morphism $\tau_x:\fX_A\ra \fY_A$ of $A$-functors. Next $x\mapsto \tau_x$ gives rise to a morphism $\tau:\fU\ra \iMor_k\left(\fX,\fY\right)$ of $k$-functors and hence we have a map of classes
$$\Mor_k(\fU\times \fX,\fY)\ni \sigma\mapsto \tau\in \Mor_k\left(\fU,\iMor_k(\fX,\fY)\right)$$
Now we have the following result {\cite[Theorem 5.3]{Presheaves}}.

\begin{theorem}\label{theorem:homforkfunctors}
Let $\fX$, $\fY$ be $k$-functors. Assume that for every $k$-algebra $A$ the class $\Mor_{A}\left(\fX_A,\fY_A\right)$ is a set. Then the map 
$$\Mor_{k}\left(\fU\times \fX,\fY\right)\ra  \Mor_{k}\left(\fU,\iMor_{k}\left(\fX,\fY\right)\right)$$
described above is a bijection natural in $\fU$. 
\end{theorem}
\noindent
Now we start a discussion of a more geometric flavour. For every $k$-algebra $A$ we denote by $k_A$ the $k$-functor given by
$$k_A(B) = \Hom_k(A,B),\,k_A(g) = \Hom_k(1_A,f)$$
for every $k$-algebra $B$ and for every morphism $g:B\ra C$ of $k$-algebras. Note that if $f:A\ra B$ is a morphism of $k$-algebras, then there exists a morphism of $k$-functors $k_f:k_B\ra k_A$ given by formula
$$k_f(C) = \Hom_k(f,1_C)$$
where $C$ is a $k$-algebra. These are general definitions that make sense in any category of copresheaves c.f. {\cite[section 7]{Presheaves}}.

\begin{definition}
Let $\fX$ be a $k$-functor. We say that $\fX$ is \textit{corepresentable} if $\fX$ is isomorphic to $k_A$ for some $k$-algebra $A$. 
\end{definition}

\begin{definition}
Let $\sigma:\fX\ra \fY$ be a morphism of $k$-functors. Assume that for every $k$-algebra $A$ and every morphism $\tau:k_A\ra \fY$ of $k$-functors there exist an ideal $\ideal{a}$ in $A$ and morphism $\tau':k_{A/\ideal{a}}\ra \fX$ such that the square
\begin{center}
\begin{tikzpicture}
[description/.style={fill=white,inner sep=2pt}]
\matrix (m) [matrix of math nodes, row sep=3em, column sep=3em,text height=1.5ex, text depth=0.25ex] 
{  k_{A/\ideal{a}}  & \fX           \\
   k_A  & \fY           \\} ;
\path[->,line width=1.0pt,font=\scriptsize]  
(m-1-1) edge node[above] {$\tau' $} (m-1-2)
(m-2-1) edge node[below] {$ \tau $} (m-2-2)
(m-1-1) edge node[left] {$k_q  $} (m-2-1)
(m-1-2) edge node[right] {$ \sigma $} (m-2-2);
\end{tikzpicture}
\end{center}
is cartesian, where $q:A\ra A/\ideal{a}$ is the quotient morphism of $k$-algebras. Then $\sigma$ is \textit{a closed immersion of $k$-functors}.
\end{definition}
\noindent
Consider a $k$-algebra $A$ and an ideal $\ideal{a}$ in $A$. We define a $k$-subfunctor $D(\ideal{a})$ of $k_A$ by 
$$D(\ideal{a})(B) = \{f:A\ra B\,|\,1 \in f(\ideal{a})B\}$$
In other words $f:A\ra B$ is a $B$-point of $D(\ideal{a})$ if and only if we have cartesian diagram in the category of $k$-schemes
\begin{center}
\begin{tikzpicture}
[description/.style={fill=white,inner sep=2pt}]
\matrix (m) [matrix of math nodes, row sep=3em, column sep=3em,text height=1.5ex, text depth=0.25ex] 
{  \emptyset  & \Spec A/\ideal{a}       \\
   \Spec B  & \Spec A           \\} ;
\path[->,line width=1.0pt,font=\scriptsize]  
(m-1-1) edge node[above] {$  $} (m-1-2)
(m-2-1) edge node[below] {$ \Spec f $} (m-2-2)
(m-1-1) edge node[left] {$   $} (m-2-1)
(m-1-2) edge node[right] {$ \Spec q  $} (m-2-2);
\end{tikzpicture}
\end{center}
where $q:A\ra A/\ideal{a}$ is the quotient morphism.

\begin{definition}
Let $\sigma:\fX\ra \fY$ be a morphism of $k$-functors. Assume that for every $k$-algebra $A$ and every morphism $\tau:k_A\ra \fY$ of $k$-functors there exist an ideal $\ideal{a}$ in $A$ and a morphism $\tau':D(\ideal{a})\ra \fX$ of $k$-functors such that the square
\begin{center}
\begin{tikzpicture}
[description/.style={fill=white,inner sep=2pt}]
\matrix (m) [matrix of math nodes, row sep=3em, column sep=3em,text height=1.5ex, text depth=0.25ex] 
{  D(\ideal{a})    & \fX           \\
   k_A             & \fY           \\} ;
\path[->,line width=1.0pt,font=\scriptsize]
(m-1-1) edge node[above] {$ \tau' $} (m-1-2)
(m-2-1) edge node[below] {$ \tau $} (m-2-2)
(m-1-2) edge node[right] {$ \sigma $} (m-2-2);
\path[right hook->,line width=1.0pt,font=\scriptsize]
(m-1-1) edge node[left] {$   $} (m-2-1);
\end{tikzpicture}
\end{center}
is cartesian. Then $\sigma$ is \textit{an open immersion of $k$-functors}.
\end{definition}

\begin{fact}\label{fact:openclosedimmersionsclosedunderbasechange}
The class of open (closed) immersions of $k$-functors is closed under base change.
\end{fact}
\begin{proof}
Left to the reader.
\end{proof}

\section{Zariski local $k$-functors and Zariski sheaves}
\noindent
In this part we use notion of a Grothendieck topology on a category. For this notion we refer the reader to \cite{Sheaves}.

\begin{definition}
Let $\big\{f_i:X_i\ra X\big\}_{i\in I}$ be a family of morphisms of $k$-schemes. We say that $\{f_i\}_{i\in I}$ is \textit{a Zariski covering of $X$} if the following conditions are satisfied.
\begin{enumerate}[label=\textbf{(\arabic*)}, leftmargin=1.5em]
\item For every $i\in I$ morphism $f_i$ is an open immersion of schemes.
\item Morphism $\coprod_{i\in I}X_i\ra X$ induced by $\big\{f_i\big\}_{i\in I}$ is surjective.
\end{enumerate}
\end{definition}
\noindent
The collection of all Zariski coverings on $\Sch_k$ is a Grothendieck pretopology.

\begin{definition}
We call the Grothendieck topology generated by pretopology of Zariski coverings \textit{the Zariski topology on $\Sch_k$}. A presheaf on $\Sch_k$ that is a sheaf with respect to Zariski topology on $\Sch_k$ is called \textit{a Zariski sheaf}.
\end{definition}
\noindent
Let $\fX$ be a presheaf on the category of $k$-schemes. Recall that by {\cite[Theorem 3.5]{Sheaves}} $\fX$ is a Zariski sheaf if and only if for every $k$-scheme $X$ and for every Zariski covering $\big\{f_i:X_i\ra X\big\}$ of $X$ the diagram
\begin{center}
\begin{tikzpicture}
[description/.style={fill=white,inner sep=2pt}]
\matrix (m) [matrix of math nodes, row sep=3em, column sep=6em,text height=1.5ex, text depth=0.25ex] 
{\fX(X) &   \prod_{i\in I}\fX(X_i)&  \prod_{(i,j)\in I\times I} \fX(X_i\times_XX_j)  \\} ;
\path[->,line width=0.8pt,font=\scriptsize]
(m-1-1) edge node[above] {$ \langle \fX(f_i) \rangle_{i\in I} $} (m-1-2)
(m-1-2) edge[transform canvas={yshift=0.5ex}] node[above] {$ \langle \fX(f'_{ij}) \cdot pr_i\rangle_{(i,j)}$} (m-1-3)
(m-1-2) edge[transform canvas={yshift=-0.5ex}] node[below] {$ \langle \fX(f''_{ij}) \cdot pr_j\rangle_{(i,j)}$} (m-1-3);
\end{tikzpicture}
\end{center}
is a kernel of a pair of arrows, where for every $(i,j)\in I\times I$ morphisms $f'_{ij}$ and $f'_{ji}$ form a cartesian square
\begin{center}
\begin{tikzpicture}
[description/.style={fill=white,inner sep=2pt}]
\matrix (m) [matrix of math nodes, row sep=3em, column sep=3em,text height=1.5ex, text depth=0.25ex] 
{X_i\times_XX_j   &   X_j   \\
 X_i  & X   \\} ;
\path[->,line width=0.8pt,font=\scriptsize]
(m-1-1) edge node[above] {$ f''_{ij}$} (m-1-2)
(m-2-1) edge node[below] {$ f_i $} (m-2-2)
(m-1-1) edge node[left] {$ f'_{ij} $} (m-2-1)
(m-1-2) edge node[right] {$ f_j  $} (m-2-2);
\end{tikzpicture}
\end{center}
\noindent
Now we repeat this definitions for $k$-algebras and $k$-functors.

\begin{definition}
Let $\big\{f_i:A\ra A_i\big\}_{i\in I}$ be a family of morphisms of $k$-algebras. We say that $\{f_i\}_{i\in I}$ is \textit{a Zariski covering of $A$} if the following conditions are satisfied.
\begin{enumerate}[label=\textbf{(\arabic*)}, leftmargin=1.5em]
\item For every $i\in I$ morphism $\Spec f_i$ is an open immersion of schemes.
\item Morphism $\coprod_{i\in I}\Spec A_i\ra \Spec A$ induced by $\big\{\Spec f_i\big\}_{i\in I}$ is surjective.
\end{enumerate}
\end{definition}
\noindent
The collection of all Zariski coverings on $\Alg_k$ induces on its opposite category $\Aff_k$ of affine $k$-schemes a Grothendieck pretopology.

\begin{definition}
We call the Grothendieck topology generated by pretopology of Zariski coverings \textit{the Zariski topology on $\Aff_k$}. A $k$-functor that is a sheaf with respect to Zariski topology on $\Aff_k$ is called \textit{a Zariski local $k$-functor}.
\end{definition}
\noindent
Let $\fX$ be a $k$-functor. Again by {\cite[Theorem 3.5]{Sheaves}} $\fX$ is a Zariski local $k$-functor if and only if for every $k$-algebra $A$ and for every Zariski covering $\big\{f_i:A\ra A_i\big\}$ of $A$ the diagram
\begin{center}
\begin{tikzpicture}
[description/.style={fill=white,inner sep=2pt}]
\matrix (m) [matrix of math nodes, row sep=3em, column sep=6em,text height=1.5ex, text depth=0.25ex] 
{\fX(A) &   \prod_{i\in I}\fX(A_i)&  \prod_{(i,j)\in I\times I} \fX(A_i\otimes_AA_j)  \\} ;
\path[->,line width=0.8pt,font=\scriptsize]
(m-1-1) edge node[above] {$ \langle \fX(f_i) \rangle_{i\in I} $} (m-1-2)
(m-1-2) edge[transform canvas={yshift=0.5ex}] node[above] {$ \langle \fX(f'_{ij}) \cdot pr_i\rangle_{(i,j)}$} (m-1-3)
(m-1-2) edge[transform canvas={yshift=-0.5ex}] node[below] {$ \langle \fX(f''_{ij}) \cdot pr_j\rangle_{(i,j)}$} (m-1-3);
\end{tikzpicture}
\end{center}
is a kernel of a pair of arrows, where for every $(i,j)\in I\times I$ morphisms $f'_{ij}$ and $f'_{ji}$ form a cocartesian square
\begin{center}
\begin{tikzpicture}
[description/.style={fill=white,inner sep=2pt}]
\matrix (m) [matrix of math nodes, row sep=3em, column sep=3em,text height=1.5ex, text depth=0.25ex] 
{A &  A_j   \\
 A_i&  A_i\otimes_AA_j   \\} ;
\path[->,line width=0.8pt,font=\scriptsize]
(m-1-1) edge node[above] {$ f_j $} (m-1-2)
(m-2-1) edge node[below] {$ f'_{ij} $} (m-2-2)
(m-1-1) edge node[left] {$ f_i $} (m-2-1)
(m-1-2) edge node[right] {$ f'_{ji}  $} (m-2-2);
\end{tikzpicture}
\end{center}
\noindent
Now we state the main result of this section.

\begin{theorem}\label{theorem:sheavesonschemesarelocalkfunctors}
Let
\begin{center}
\begin{tikzpicture}
[description/.style={fill=white,inner sep=2pt}]
\matrix (m) [matrix of math nodes, row sep=3em, column sep=3em,text height=1.5ex, text depth=0.25ex] 
{ \widehat{\Sch_k}  & \mbox{the category of $k$-functors} \\};
\path[->,line width=1.0pt,font=\scriptsize]  
(m-1-1) edge node[auto] {$ $} (m-1-2);
\end{tikzpicture}
\end{center}
be the restriction of presheaves on $\Sch_k$ to copresheaves on $\Alg_k$ ($k$-functors) induced by the contravariant functor $\Spec:\Alg_k\ra \Sch_k$. Then it induces an equivalence of categories between Zariski sheaves on $\Sch_k$ and Zariski local $k$-functors.
\end{theorem}
\begin{proof}
Note that $\Aff_k$ with Zariski topology is a dense subsite ({\cite[definition 4.4]{Sheaves}}) of $\Sch_k$ with Zariski topology. Hence the result is a special case of a more general theorem {\cite[Theorem 4.6]{Sheaves}}. 
\end{proof}

\section{Transporters}

\begin{definition}
Let $X$ be a $k$-scheme. Suppose that there exists an open affine cover $X = \bigcup_{i\in I}X_i$ such that $k$-algebra $\Gamma(X_i,\cO_{X_i})$ is free as a $k$-module. Then we say that $X$ is \textit{a locally free $k$-scheme}.
\end{definition}
\noindent
Next theorem is the main result of this section.

\begin{theorem}\label{theorem:closedimmersionsandinternalhom}
Let $j:\fY'\ra \fY$ be a closed immersion of $k$-functors and $X$ be a locally free $k$-scheme. Suppose that classes $\Mor_A\left(X_A,\fY_A\right)$ are sets for every $k$-algebra $A$. Then classes $\Mor_A\left(X_A,\fY'_A\right)$ are sets for every $k$-algebra $A$ and the morphism
$$\iMor_k\left(1_X,j\right):\iMor_k\left(X,\fY'\right)\ra \iMor_k\left(X,\fY\right)$$
is a closed immersion of $k$-functors.
\end{theorem}
\noindent
It is useful to isolate crucial steps in the argument. For this we proceed by proving some lemmas.

\begin{lemma}\label{lemma:foraffinelocalfactorization}
Suppose that $A$ is a commutative ring. Let $j:\fY'\ra \fY$ be a closed immersion of $A$-functors and $X$ be an affine $A$-scheme such that $\Gamma(X,\cO_X)$ is a free $A$-module. Assume that $\tau:X\ra \fY$ is a morphism of $A$-functors. Then there exists an ideal $\ideal{a}\subseteq A$ such that for every $A$-algebra $B$ the restriction $\tau_B$ factors through $j_B$ if and only if the structure morphism $f:A\ra B$ of $B$ satisfies $\ideal{a}\subseteq \ker(f)$.
\end{lemma}
\begin{proof}[Proof of the lemma]
Consider a cartesian square
\begin{center}
\begin{tikzpicture}
[description/.style={fill=white,inner sep=2pt}]
\matrix (m) [matrix of math nodes, row sep=3em, column sep=3em,text height=1.5ex, text depth=0.25ex] 
{  X' & \fY' \\
   X  & \fY           \\} ;
\path[->,line width=1.0pt,font=\scriptsize]  
(m-1-1) edge node[above] {$ $} (m-1-2)
(m-2-1) edge node[below] {$\tau  $} (m-2-2)
(m-1-1) edge node[left] {$ j' $} (m-2-1)
(m-1-2) edge node[right] {$ j $} (m-2-2);
\end{tikzpicture}
\end{center}
Since $j$ is a closed immersion of $A$-functors, we derive by Fact \ref{fact:openclosedimmersionsclosedunderbasechange} that $j'$ is a closed immersion. By assumption $X$ is affine. Hence $X'$ is a functor of points of some $A$-scheme and $j':X'\ra X$ is (induced by) a closed immersion of $A$-schemes. Next let $B$ be an $A$-algebra with the structure morphism $f:A\ra B$. Then $\tau_B$ factors through $j_B$ if and only if the projection $\Spec B\times_{\Spec A}X\ra X$ induced by $f$ factors through $X'$. Let $A[X]$ be the $A$-algebra of global regular functions on $X$ and let $\ideal{J}$ be an ideal in $A[X]$ such that $A[X]/\ideal{J} = A[X']$ is the $A$-algebra of global regular functions of $X'$. With this notation we derive that the projection $\Spec B\times_{\Spec A}X\ra X$ induced by $f$ factors through $X'$ if and only if the morphism $A[X]\ra B\otimes_AA[X]$ induced by $f$ sends every element of $\ideal{J}$ to zero. Since $A[X]$ is a free $A$-module, we write $A[X] = A^{\oplus I}$ for some index set $I$. Then the morphism $A[X]\ra B\otimes_AA[X]$ induced by $f$ is just $f^{\oplus I}:A^{\oplus I}\ra B^{\oplus I}$. We have $f^{\oplus I}\left(\ideal{J}\right)=0$ if and only if $\left(pr^B_i\cdot f^{\oplus I}\right)\left(\ideal{J}\right)=$ for every $i\in I$, where $pr^B_i:B^{\oplus I}\ra B$ is the projection on $i$-th component. Pick $i\in I$ and consider the commutative diagram
\begin{center}
\begin{tikzpicture}
[description/.style={fill=white,inner sep=2pt}]
\matrix (m) [matrix of math nodes, row sep=3em, column sep=3em,text height=1.5ex, text depth=0.25ex] 
{  A^{\oplus I} & B^{\oplus I}  \\
   A  & B           \\} ;
\path[->,line width=1.0pt,font=\scriptsize]  
(m-1-1) edge node[above] {$ f^{\oplus I} $} (m-1-2)
(m-2-1) edge node[below] {$ f  $} (m-2-2)
(m-1-1) edge node[left] {$ pr^A_i $} (m-2-1)
(m-1-2) edge node[right] {$ pr^B_i $} (m-2-2);
\end{tikzpicture}
\end{center}
In the diagram $pr^A_i$ is the projection on $i$-th component. Diagram implies that $\left(pr^B_i\cdot f^{\oplus I}\right)\left(\ideal{J}\right)=$ for every $i\in I$ if and only if $\left(f\cdot pr_i^A\right)(\ideal{J}) = 0$ for every $i\in I$. This is equivalent with the condition that $f(\ideal{a})=0$ for ideal $\ideal{a}$ in $A$ generated by $\sum_{i\in I}pr_i^A(\ideal{J})$. Thus the lemma is proved.
\end{proof}

\begin{lemma}\label{lemma:coveringsandfactorizations}
Suppose that $A$ is a commutative ring. Let $j:\fY'\ra \fY$ be a closed immersion of $A$-functors and $X$ be an $A$-scheme with open cover
$$X=\bigcup_{i\in I}X_i$$
Assume that $\tau:X\ra \fY$ is a morphism of $A$-functors. Fix an $A$-algebra $B$. Then $\tau_B$ factors through $j_B$ if and only if $\left(\tau_{\mid X_i}\right)_B$ factors through $j_B$ for every $i\in I$.
\end{lemma}
\begin{proof}[Proof of the lemma]
If $\tau_B$ factors through $j_B$, then also $\left(\tau_{\mid X_i}\right)_B$ factors through $j_B$ for every $i\in I$. It suffices to prove the converse. So suppose that $\left(\tau_{\mid X_i}\right)_B$ factors through $j_B$ for every $i\in I$. Since $j$ is a closed immersion of $A$-functors and $X$ is an $A$-scheme, there exists a cartesian square
\begin{center}
\begin{tikzpicture}
[description/.style={fill=white,inner sep=2pt}]
\matrix (m) [matrix of math nodes, row sep=3em, column sep=3em,text height=1.5ex, text depth=0.25ex] 
{  X' & \fY' \\
   X  & \fY           \\} ;
\path[->,line width=1.0pt,font=\scriptsize]  
(m-1-1) edge node[above] {$ $} (m-1-2)
(m-2-1) edge node[below] {$\tau  $} (m-2-2)
(m-1-1) edge node[left] {$ j' $} (m-2-1)
(m-1-2) edge node[right] {$ j $} (m-2-2);
\end{tikzpicture}
\end{center}
where $j':X'\ra X$ is (induced by) a closed immersion of $A$-schemes (this follows from Fact \ref{fact:openclosedimmersionsclosedunderbasechange}  and Fact \ref{fact:kschemesandkfunctors}). For each $i\in I$ let $j'_i:j'^{-1}(X_i)\ra X_i$ be the restriction of $j'$. We have the induced cartesian square
\begin{center}
\begin{tikzpicture}
[description/.style={fill=white,inner sep=2pt}]
\matrix (m) [matrix of math nodes, row sep=3em, column sep=3em,text height=1.5ex, text depth=0.25ex] 
{  j'^{-1}(X_i) & \fY' \\
   X_i  & \fY           \\} ;
\path[->,line width=1.0pt,font=\scriptsize]  
(m-1-1) edge node[above] {$ $} (m-1-2)
(m-2-1) edge node[below] {$\tau_{\mid X_i}  $} (m-2-2)
(m-1-1) edge node[left] {$j'_i  $} (m-2-1)
(m-1-2) edge node[right] {$ j $} (m-2-2);
\end{tikzpicture}
\end{center}
Now $\left(\tau_{\mid X_i}\right)_B$ factors through $j_B$. Together with Fact \ref{fact:kschemesandkfunctors} this shows that $\left(j'_i\right)_B$ is an isomorphism of $B$-schemes. This holds for every $i\in I$. Hence $j'_B$ is an isomorphism of $B$-schemes (again by application of Fact \ref{fact:kschemesandkfunctors}). Therefore, $\tau_B$ factors through $j_B$.
\end{proof}

\begin{proof}[Proof of the theorem]
Let $A$ be a $k$-algebra. The restriction functor $(-)_{\mid \Alg_A} = (-)_A$ preserves all closed immersions. Thus $j_A$ is a closed immersion of $A$-functors and hence we derive that $j_A:\fY'_A\ra \fY_A$ is a monomorphism of $A$-functors. Thus we have an injective  map of classes
$$\Mor_A\left(1_{X_A},j_A\right):\Mor_A\left(X_A,\fY'_A\right)\hookrightarrow \Mor_A\left(X_A,\fY_A\right)$$
Hence if $\Mor_A\left(X_A,\fY_A\right)$ is a set, then $\Mor_A\left(X_A,\fY'_A\right)$ is a set. All these facts imply that both internal homs
$$\iMor_k\left(X,\fY'\right),\,\iMor_k\left(X,\fY\right)$$
exist and morphism $\iMor_k(1_X,j)$ of $k$-functors is a monomorphism. Our task is to prove that it is a closed immersion. For this consider a $k$-algebra $A$ and a morphism $\sigma:k_A\ra \iMor_k\left(X,\fY\right)$ of $k$-functors that sends $1_A$ to some morphism $\tau:X_A\ra \fY_A$ of $A$-functors. Consider a cartesian square
\begin{center}
\begin{tikzpicture}
[description/.style={fill=white,inner sep=2pt}]
\matrix (m) [matrix of math nodes, row sep=3em, column sep=3em,text height=1.5ex, text depth=0.25ex] 
{  \fU  & \iMor_k\left(X,\fY'\right) \\
   k_A  & \iMor_k\left(X,\fY\right)           \\} ;
\path[->,line width=1.0pt,font=\scriptsize]  
(m-1-1) edge node[above] {$ $} (m-1-2)
(m-2-1) edge node[below] {$ \sigma $} (m-2-2)
(m-1-1) edge node[left] {$  $} (m-2-1)
(m-1-2) edge node[right] {$ \iMor_k\left(1_X,j\right) $} (m-2-2);
\end{tikzpicture}
\end{center}
Since $\iMor_k\left(1_X,j\right)$ is a monomorphism, we may consider $\fU$ as a $k$-subfunctor of $k_A$. For every $k$-algebra $B$ subset $\fU(B)\subseteq \Mor_k(A,B)= k_A(B)$ consists of $A$-algebras $B$ with structure morphisms $f:A\ra B$ such that $\tau_B$ factors through $j_B:\fY'_B\ra \fY_B$. Since $X$ is a locally free $k$-scheme, we deduce that $X_A$ is (a functor of points of) a locally free $A$-scheme. Pick an open affine cover $X_A = \bigcup_{i\in I}X_i$ such that $\Gamma(X_i,\cO_X)$ is a free $A$-module. Now Lemma \ref{lemma:coveringsandfactorizations} implies that $\tau_B$ factors through $j_B$ if and only if $\left(\tau_{\mid X_i}\right)_B$ factors through $j_B$ for every $i\in I$. Next by Lemma \ref{lemma:foraffinelocalfactorization} we deduce that $\left(\tau_{\mid X_i}\right)_B$  factors through $j_B$ for given $i\in I$ if and only if $f(\ideal{a}_i)=0$ for some ideal $\ideal{a}_i\subseteq A$ independent of $f$. Thus $\fU$ consists of all morphisms $f:A\ra B$ of $k$-algebras such that $f(\ideal{a})=0$ where $\ideal{a} = \sum_{i\in I}\ideal{a}_i$. Therefore, $\fU\hookrightarrow k_A$ is isomorphic with $k_{A/\ideal{a}}\hookrightarrow k_A$ and hence $\iMor_k(1_X,j)$ is a closed immersion of $k$-functors.
\end{proof}
\noindent
The Theorem \ref{theorem:closedimmersionsandinternalhom} is a simple yet powerful result. Before giving any interesting applications we state its immediate consequence.

\section{Schemes as $k$-functors}
\noindent
Let $X$ be a $k$-scheme. We define a $k$-functor $\fP_X$ by formula
$$\fP_X(A) = \Mor_k\left(\Spec A,X\right)$$
That is $\fP_X$ is the restriction of the presheaf on $\Sch_k$ represented by $X$ to the category $Alg_k$ along the functor $\Spec:\Alg_k\ra \Sch_k$. Next if $f:X\ra Y$ is a morphism of $k$-schemes, then $\fP_f$ is the restriction of a morphism of presheaves on $\Sch_k$ represented by $f$ to the category of $k$-algebras along $\Spec:\Alg_k\ra \Sch_k$. Thus we have a functor
\begin{center}
\begin{tikzpicture}
[description/.style={fill=white,inner sep=2pt}]
\matrix (m) [matrix of math nodes, row sep=3em, column sep=3em,text height=1.5ex, text depth=0.25ex] 
{ \Sch_k  & \mbox{the category of $k$-functors} \\};
\path[->,line width=1.0pt,font=\scriptsize]  
(m-1-1) edge node[auto] {$ \fP $} (m-1-2);
\end{tikzpicture}
\end{center}

\begin{definition}
Let $X$ be a $k$-scheme. Then $\fP_X$ is called \textit{the $k$-functor of points of $X$}.
\end{definition}

\begin{fact}\label{fact:functorsofpoints}
Functor
\begin{center}
\begin{tikzpicture}
[description/.style={fill=white,inner sep=2pt}]
\matrix (m) [matrix of math nodes, row sep=3em, column sep=3em,text height=1.5ex, text depth=0.25ex] 
{ \Sch_k  & \mbox{the category of $k$-functors} \\};
\path[->,line width=1.0pt,font=\scriptsize]  
(m-1-1) edge node[auto] {$ \fP $} (m-1-2);
\end{tikzpicture}
\end{center}
is full, faithful and its image consists of Zariski local $k$-functors.
\end{fact}
\begin{proof}
Note that the presheaf $h_X$ on $\Sch_k$ represented by $X$ is a Zariski sheaf. Indeed, this just rephrase standard fact that morphism of schemes can be glued in Zariski topology. Next according to Theorem \ref{theorem:sheavesonschemesarelocalkfunctors} the functor $\Spec:\Alg_k\ra \Sch_k$ induces an equivalence of the category of Zariski sheaves and local Zariski $k$-functors. Thus $\fP_X$ is a local Zariski $k$-functor and it is full and faithful.
\end{proof}

\begin{definition}
Let $F$, $G$ be presheaves on $\Sch_k$ and let $f:F\ra G$ be their morphism. Suppose that $x\in G(X)$ for some $k$-scheme $X$. To every $x$ of this type one can associate the cartesian square of presheaves
\begin{center}
\begin{tikzpicture}
[description/.style={fill=white,inner sep=2pt}]
\matrix (m) [matrix of math nodes, row sep=3em, column sep=3em,text height=1.5ex, text depth=0.25ex] 
{h_X\times_GF&  F   \\
h_X&    G  \\} ;
\path[->,line width=1.0pt,font=\scriptsize]  
(m-1-1) edge node[above] {$ $} (m-1-2)
(m-2-1) edge node[below] {$ $} (m-2-2)
(m-1-1) edge node[left] {$ \pi_x$} (m-2-1)
(m-1-2) edge node[right] {$f$} (m-2-2);
\end{tikzpicture}
\end{center}
in which bottom vertical morphism $h_X\ra G$ is canonically identified with $x$. We say that $f$ is:
\begin{enumerate}[label=\textbf{(\arabic*)}, leftmargin=1.5em]
\item \textit{an open immersion} if for every $k$-scheme $X$ and $x\in G(X)$ morphism $\pi_x$ is isomorphic to the image under Yoneda embedding of some open immersion of $k$-schemes. 
\item \textit{a closed immersion} if for every $k$-scheme $X$ and $x\in G(X)$ morphism $\pi_x$ is isomorphic to the image under Yoneda embedding of some closed immersion of $k$-schemes.
\end{enumerate}
\end{definition}

\begin{proposition}\label{proposition:openclosedimmersionsaremonomorphisms}
Let $f:F\ra G$ be a morphism of presheaves on $\Sch_k$. Suppose that $f$ is either open or closed immersion. Then $f$ is a monomorphism of presheaves.
\end{proposition}
\begin{proof}
Fix an element $y\in G(X)$. Consider a cartesian square
\begin{center}
\begin{tikzpicture}
[description/.style={fill=white,inner sep=2pt}]
\matrix (m) [matrix of math nodes, row sep=3em, column sep=3em,text height=1.5ex, text depth=0.25ex] 
{h_X\times_GF&  F   \\
h_X&    G  \\} ;
\path[->,line width=1.0pt,font=\scriptsize]  
(m-1-1) edge node[above] {$ $} (m-1-2)
(m-2-1) edge node[below] {$ $} (m-2-2)
(m-1-1) edge node[left] {$ \pi_y$} (m-2-1)
(m-1-2) edge node[right] {$f$} (m-2-2);
\end{tikzpicture}
\end{center}
in which $y$ determines a morphism $h_X\ra G$. Morphism $f$ is either open or closed immersion. Hence there exists a monomorphism $j:Y\ra X$ of $k$-schemes such that $\pi_y$ is isomorphic with $h_j$. Yoneda embedding preserves monomorphisms. Thus $h_j$ is a monomorphism of presheaves. This implies that $\pi_y$ is a monomorphism of presheaves for every $k$-scheme $X$ and $y\in G(X)$. In particular, there exists at most one element $x\in F(X)$ such that $f(x)=y$. Since $y\in G(X)$ is arbitrary, we deduce that $f$ is a monomorphism of presheaves.
\end{proof}

\begin{definition}
Let $F$ be a presheaf on $\Sch_k$ and $\big\{f_i:F_i\ra F\big\}_{i\in I}$ be a family of open immersions. Then for every $k$-scheme $X$ and $x\in F(X)$ we have a family of open immersions $\big\{f_{i,x}:U_{i,x}\ra X\big\}_{i\in I}$ defined by cartesian squares
\begin{center}
\begin{tikzpicture}
[description/.style={fill=white,inner sep=2pt}]
\matrix (m) [matrix of math nodes, row sep=3em, column sep=3em,text height=1.5ex, text depth=0.25ex] 
{h_{U_{i,x}}&   F_i   \\
h_X&    F  \\} ;
\path[->,line width=1.0pt,font=\scriptsize]  
(m-1-1) edge node[above] {$ $} (m-1-2)
(m-2-1) edge node[below] {$ $} (m-2-2)
(m-1-1) edge node[left] {$ h_{f_{i,x}}$} (m-2-1)
(m-1-2) edge node[right] {$f_i$} (m-2-2);
\end{tikzpicture}
\end{center}
in which bottom vertical morphism $h_X\ra G$ is canonically identified with $x$. We say that $\{f_i\}_{i\in I}$ is \textit{an open cover of $F$} if for every $k$-scheme $X$ and $x\in F(X)$ we have
$$X=\bigcup_{i\in I}f_{i,x}\left(U_{i,x}\right)$$
\end{definition}

\begin{theorem}
Let $F$ be a presheaf on $\Sch_k$. Then the following are equivalent. 
\begin{enumerate}[label=\emph{\textbf{(\roman*)}}, leftmargin=1.5em]
\item $F\cong h_X$ for some $k$-scheme $X$.
\item $F$ is a Zariski sheaf and there exists an open cover $\big\{v_i:h_{V_i}\ra F\big\}_{i\in I}$ such that $\{V_i\}_{i\in I}$ are affine $k$-schemes.
\item $F$ is a Zariski sheaf and there exists an open cover $\big\{v_i:h_{V_i}\ra F\big\}_{i\in I}$ such that $\{V_i\}_{i\in I}$ are $k$-schemes.
\end{enumerate}
\end{theorem}
\begin{proof}
We prove $\textbf{(i)}\Rightarrow \textbf{(ii)}$. Since $F\cong h_X$ and properties in \textbf{(ii)} are stable under isomorphism, we deduce that we can replace $F$ by $h_X$. So it suffices to show that $h_X$ satisfies \textbf{(ii)}. By definition every $k$-scheme $X$ admits an open cover $\big\{v_i:V_i\ra X\big\}_{i\in I}$ by affine $k$-schemes. Since Yoneda embedding $h:\Sch_k\ra \widehat{\Sch_k}$ preserves fiber-products, we derive that $\big\{h_{v_i}\big\}_{i\in I}$ is an open cover in the category of presheaves. Thus $h_X$ admits an open cover by presheaves representable by affine $k$-schemes. Next suppose that $\big\{f_i:U_i\ra U\big\}_{i\in I}$ is a Zariski covering of a $k$-scheme $U$ and $\big\{g_i:U_i\ra X\big\}_{i\in I}$ is a family of morphisms of $k$-schemes such that ${g_i}_{\mid U_i\times_UU_j}={g_j}_{\mid U_i\times_UU_j}$ for every pair $(i,j)\in I\times I$. Then we can glue $\{g_i\}_{i\in I}$ into a unique morphism of $k$-schemes $g:U\ra X$ such that $g\cdot f_i=g_i$ for every $i\in I$. This shows that $h_X$ is a Zariski sheaf.\\
The implication $\textbf{(ii)}\Rightarrow \textbf{(iii)}$ is a consequence of the fact that every affine $k$-scheme is a $k$-scheme.\\
Assume now that \textbf{(iii)} holds. Fix elements $i$, $j\in I$ and consider a cartesian square
\begin{center}
\begin{tikzpicture}
[description/.style={fill=white,inner sep=2pt}]
\matrix (m) [matrix of math nodes, row sep=3em, column sep=3em,text height=1.5ex, text depth=0.25ex] 
{h_{V_i}\times_Fh_{V_j}  & h_{V_j}   \\
h_{V_i}&    F  \\} ;
\path[->,line width=1.0pt,font=\scriptsize]  
(m-1-1) edge node[above] {$v'_j $} (m-1-2)
(m-2-1) edge node[below] {$v_i $} (m-2-2)
(m-1-1) edge node[left] {$v'_i $} (m-2-1)
(m-1-2) edge node[right] {$v_j$} (m-2-2);
\end{tikzpicture}
\end{center}
in the category $\widehat{\Sch_k}$. Since $v_i$ is an open immersion, we derive that there exists an open subscheme $V_{ij}\subseteq V_i$ and an isomorphism $p_{ij}:h_{V_{ij}}\ra h_{V_i}\times_Fh_{V_j}$ such that the triangle
\begin{center}
\begin{tikzpicture}
[description/.style={fill=white,inner sep=2pt}]
\matrix (m) [matrix of math nodes, row sep=2em, column sep=1em,text height=1.5ex, text depth=0.25ex] 
{h_{V_{ij}}&  & h_{V_i}\times_Fh_{V_j}   \\
&  h_{V_i} &  \\} ;
\path[->,line width=1.0pt, font=\scriptsize]  
(m-1-1) edge node[above] {$p_{ij} $} (m-1-3)
(m-1-3) edge node[auto] {$v'_i $} (m-2-2);
\path[right hook ->,line width=1.0pt, font=\scriptsize]  
(m-1-1) edge node[auto] {$ $} (m-2-2);
\end{tikzpicture}
\end{center}
is commutative. Similarly since $v_j$ is an open immersion, we derive that there exists an open subscheme $V_{ji}\subseteq V_j$ and an isomorphism $p_{ji}:h_{V_{ij}}\ra h_{V_i}\times_Fh_{V_j}$ such that the triangle
\begin{center}
\begin{tikzpicture}
[description/.style={fill=white,inner sep=2pt}]
\matrix (m) [matrix of math nodes, row sep=2em, column sep=1em,text height=1.5ex, text depth=0.25ex] 
{h_{V_{ji}}&  & h_{V_i}\times_Fh_{V_j}   \\
&  h_{V_j} &  \\} ;
\path[->,line width=1.0pt, font=\scriptsize]  
(m-1-1) edge node[above] {$p_{ji} $} (m-1-3)
(m-1-3) edge node[auto] {$v'_j $} (m-2-2);
\path[right hook ->,line width=1.0pt, font=\scriptsize]  
(m-1-1) edge node[auto] {$ $} (m-2-2);
\end{tikzpicture}
\end{center}
is commutative. Now we define an isomorphism of $k$-schemes $\phi_{ij}:V_{ij}\ra V_{ji}$ by requirement $h_{\phi_{ij}}=p^{-1}_{ji}\cdot p_{ij}$. Then the data consisting of families $\big\{V_i\big\}_{i\in I}$ , $\big\{V_{ij}\big\}_{(i,j)\in I\times I}$ and $\big\{ \phi_{ij}\big\}_{(i,j)\in I\times I}$ satisfy the following assertions.
\begin{enumerate}[label=\textbf{(\arabic*)}, leftmargin=1.5em]
\item $V_{ij}\subseteq V_i$ is an open subscheme for every $i\in I$ and $j\in J$.
\item $V_{ii}=V_i$ and $\phi_{ii}=1_{V_i}$ for every $i\in I$.
\item $\phi_{ij}:V_{ij}\ra V_{ji}$ is an isomorphism of $k$-schemes for every $(i,j)\in I\times I$.
\item For every pair $(i,j)\in I\times I$ and $k\in I$ isomorphism $\phi_{ij}$ restricts to an isomorphism 
$$\phi'_{ij,k}:V_{ij}\cap V_{ik}\ra V_{ji}\cap V_{jk}$$
of $k$-schemes.
\item For every triple $(i,j,k)\in I\times I\times I$ we have
$$\phi'_{ik,j}=\phi'_{jk,i}\cdot \phi'_{ij,k}$$
\end{enumerate}
Thus by {\cite[Chapitre 0, 4.1.7]{EGA1new}} family $\big\{V_i\big\}_{i\in I}$ can be considered as an open cover of a ringed $k$-space $X$ in such a way that for any elements $i$, $j\in I$ the square
\begin{center}
\begin{tikzpicture}
[description/.style={fill=white,inner sep=2pt}]
\matrix (m) [matrix of math nodes, row sep=3em, column sep=3em,text height=1.5ex, text depth=0.25ex] 
{h_{V_i\cap V_j}  & h_{V_j}   \\
h_{V_i}&    F \\} ;
\path[->,line width=1.0pt,font=\scriptsize]  
(m-2-1) edge node[below] {$v_i $} (m-2-2)
(m-1-2) edge node[right] {$v_j$} (m-2-2);
\path[right hook->,line width=1.0pt,font=\scriptsize]  
(m-1-1) edge node[left] {$ $} (m-2-1)
(m-1-1) edge node[above] {$ $} (m-1-2);
\end{tikzpicture}
\end{center}
is cartesian (the intersection $V_i\cap V_j$ in the diagram is taken inside $X$). Since $X$ admits an open cover by a $k$-schemes, it is itself a $k$-scheme. Next we construct a morphism $f:h_X\ra F$. For this note that for each $i\in I$ morphism $v_i$ gives rise to an element $x_i\in F(V_i)$. Since ${v_i}_{\mid h_{V_i\cap V_j}}={v_j}_{\mid h_{V_i\cap V_j}}$ for any two $i$, $j\in I$, we deduce that ${x_i}_{\mid V_i\cap V_j}={x_j}_{\mid V_i\cap V_j}$. Next we apply the fact that $F$ is a Zariski sheaf to construct an element $x\in F(X)$ such that $x_{\mid V_i}=x_i$ for every $i\in I$. Now $x$ determines a morphism $f:h_X\ra F$ such that the following square
\begin{center}
\begin{tikzpicture}
[description/.style={fill=white,inner sep=2pt}]
\matrix (m) [matrix of math nodes, row sep=3em, column sep=3em,text height=1.5ex, text depth=0.25ex] 
{ &  h_{V_i} &  \\
h_X&  & F \\} ;
\path[->,line width=1.0pt, font=\scriptsize]  
(m-2-1) edge node[below] {$g $} (m-2-3)
(m-1-2) edge node[auto] {$v_i $} (m-2-3);
\path[right hook ->,line width=1.0pt, font=\scriptsize]  
(m-1-2) edge node[auto] {$ $} (m-2-1);
\end{tikzpicture}
\end{center}
Now let $Y$ be a $k$-scheme and pick $y\in F(Y)$. Suppose that $g:h_Y\ra F$ is a morphism corresponding to $y$. Pick $i\in I$. Since $v_i:h_{V_i}\ra F$ is an open immersion, there exists open subscheme $W_i\subseteq Y$ that fits in a cartesian square
\begin{center}
\begin{tikzpicture}
[description/.style={fill=white,inner sep=2pt}]
\matrix (m) [matrix of math nodes, row sep=3em, column sep=3em,text height=1.5ex, text depth=0.25ex] 
{h_{W_i}  & h_{V_i}   \\
h_Y&    F \\} ;
\path[->,line width=1.0pt,font=\scriptsize]
(m-1-1) edge node[above] {$g_i $} (m-1-2)  
(m-2-1) edge node[below] {$g $} (m-2-2)
(m-1-2) edge node[right] {$v_i$} (m-2-2);
\path[right hook->,line width=1.0pt,font=\scriptsize]  
(m-1-1) edge node[left] {$ $} (m-2-1);
\end{tikzpicture}
\end{center}
By Yoneda lemma $g_i$ corresponds to $k_i \in h_{V_i}(W_i)$. By definition $k_i:W_i\ra V_i$ is a morphism of $k$-schemes. Next for $i\in I$ and $j\in I$ we have 
\end{proof}

\section{Equivalence relations and quotients}
\noindent
In this section we introduce internal binary relations in categories. The first part of the section are categorical reformulations and generalizations of standard notions. We encourage the reader to interpret everything that we introduce in the category of sets.

\begin{definition}
Let $\cC$ be a category with products and let $X$ be an object of $\cC$. Suppose that $p_1,p_2:R\ra X$ are morphisms such that $\langle p_1, p_2\rangle:R\ra X\times X$ is a monomorphism. Then a triple $(R,p_1,p_2)$ is called \textit{a binary relation on $X$}.
\end{definition}
\noindent
If $\cC$ is a category with products and $(R,p_1,p_2)$ is a binary relation on some object $X$ of $\cC$, then for every object $Y$ of $\cC$ we have the inclusion
\begin{center}
\begin{tikzpicture}
[description/.style={fill=white,inner sep=2pt}]
\matrix (m) [matrix of math nodes, row sep=3em, column sep=8em,text height=1.5ex, text depth=0.25ex] 
{ \Mor_{\cC}\left(Y,R\right)  &  \Mor_{\cC}\left(Y,X\times X\right) = \Mor_{\cC}\left(Y,X\right)\times \Mor_{\cC}\left(Y,X\right)   \\} ;
\path[right hook->,line width=1.0pt,font=\scriptsize]  
(m-1-1) edge node[above] {$ \Mor_{\cC}\left(1_Y,\langle p_1,p_2 \rangle\right) $} (m-1-2);
\end{tikzpicture}
\end{center}
and thus we can interpret it as a binary relation on the class $\Mor_{\cC}\left(Y,X\right)$.\\
Suppose that $\cC$ is a category with fiber products and $(R,p_1,p_2)$ is a binary relation on some object $X$ of $\cC$. Consider a cartesian square
\begin{center}
\begin{tikzpicture}
[description/.style={fill=white,inner sep=2pt}]
\matrix (m) [matrix of math nodes, row sep=3em, column sep=3em,text height=1.5ex, text depth=0.25ex] 
{ T  &    R   \\
  R  &    X   \\} ;
\path[->,line width=1.0pt,font=\scriptsize]  
(m-1-1) edge node[above] {$ q_2 $} (m-1-2)
(m-2-1) edge node[below] {$ p_2 $} (m-2-2)
(m-1-1) edge node[left] {$ q_1 $} (m-2-1)
(m-1-2) edge node[right] {$ p_1 $} (m-2-2);
\end{tikzpicture}
\end{center}
Denote by $r$ morphism $p_2\cdot q_1 = p_1\cdot q_2$. Let $\pi_1,\pi_3:X\times X\times X\ra $ be projections on first and third factor, respectively. The morphism $\langle p_1\cdot q_1, r, p_2\cdot q_2\rangle:T\ra X\times X \times X$ composed with the morphism $\langle \pi_1,\pi_3\rangle:X\times X\times X\ra X\times X$ factors through $\langle p_1,p_2\rangle:R\hookrightarrow X\times X$ if there exists $t:T\ra R$ such that the square
\begin{center}
\begin{tikzpicture}
[description/.style={fill=white,inner sep=2pt}]
\matrix (m) [matrix of math nodes, row sep=4em, column sep=6em,text height=1.5ex, text depth=0.25ex] 
{ T  &   X\times X\times X   \\
  R  &    X\times X                   \\} ;
\path[->,line width=1.0pt,font=\scriptsize]
(m-1-1) edge node[above] {$\langle p_1\cdot q_1, r, p_2\cdot q_2\rangle  $} (m-1-2)
(m-1-2) edge node[right] {$ \langle \pi_1,\pi_3\rangle $} (m-2-2);
\path[densely dotted,->,line width=1.0pt,font=\scriptsize]
(m-1-1) edge node[left] {$ t $} (m-2-1);
\path[right hook->,line width=1.0pt,font=\scriptsize]
(m-2-1) edge node[below] {$ \langle p_1, p_2\rangle $} (m-2-2);
\end{tikzpicture}
\end{center}
is commutative. Note that $t$ is unique. Also from the diagram it follows that $t:T\ra R$ can be described as the unique morphism such that $p_1\cdot t =  p_1\cdot q_1$ and $p_2\cdot t = p_2\cdot q_2$.

\begin{definition}
Let $\cC$ be a category with fiber products and let $(R,p_1,p_2)$ be a binary relation on some object $X$ of $\cC$.
\begin{enumerate}[label=\textbf{(\arabic*)}, leftmargin=1.5em]
\item $(R,p_1,p_2)$ is \textit{reflexive} if the diagonal $\delta:X \ra X\times X$ factors through $\langle p_1,p_2\rangle$.
\item $(R,p_1,p_2)$ is \textit{symmetric} if $\langle p_1,p_2\rangle = s\cdot \langle p_1\cdot p_2\rangle$, there $s:X\times X\ra X\times X$ is the cartesian symmetry. 
\item Consider the cartesian square
\begin{center}
\begin{tikzpicture}
[description/.style={fill=white,inner sep=2pt}]
\matrix (m) [matrix of math nodes, row sep=3em, column sep=3em,text height=1.5ex, text depth=0.25ex] 
{ T  &    R   \\
  R  &    X   \\} ;
\path[->,line width=1.0pt,font=\scriptsize]  
(m-1-1) edge node[above] {$ q_2 $} (m-1-2)
(m-2-1) edge node[below] {$ p_2 $} (m-2-2)
(m-1-1) edge node[left] {$ q_1 $} (m-2-1)
(m-1-2) edge node[right] {$ p_1 $} (m-2-2);
\end{tikzpicture}
\end{center}
$(R,p_1,p_2)$ is \textit{transitive} if there exists $t:T\ra R$ such that $p_1\cdot t = p_1\cdot q_1$ and $p_2\cdot t = p_2\cdot q_2$.
\item Finally, we say that $(R,p_1,p_2)$ is \textit{an equivalence relation} if it is reflexive, symmetric and transitive. 
\end{enumerate}
\end{definition}

\begin{fact}\label{fact:relationsintermsofpoints}
Let $\cC$ be a category with fiber products and let $(R,p_1,p_2)$ be a binary relation on some object $X$ of $\cC$. Then the following assertions are equivalent.
\begin{enumerate}[label=\emph{\textbf{(\roman*)}}, leftmargin=1.5em]
\item For every object $Y$ of $\cC$ the binary relation
\begin{center}
\begin{tikzpicture}
[description/.style={fill=white,inner sep=2pt}]
\matrix (m) [matrix of math nodes, row sep=3em, column sep=8em,text height=1.5ex, text depth=0.25ex] 
{ \Mor_{\cC}\left(Y,R\right)  &  \Mor_{\cC}\left(Y,X\times X\right) = \Mor_{\cC}\left(Y,X\right)\times \Mor_{\cC}\left(Y,X\right)   \\} ;
\path[right hook->,line width=1.0pt,font=\scriptsize]  
(m-1-1) edge node[above] {$ \Mor_{\cC}\left(1_Y,\langle p_1,p_2 \rangle\right) $} (m-1-2);
\end{tikzpicture}
\end{center}
is reflexive (symmetric, transitive).
\item $(R,p_1,p_2)$ is reflexive (symmetric, transitive).
\end{enumerate}
\end{fact}
\begin{proof}
We may pass to larger $V$ Grothendieck universe such that $\cC$ is locally $V$-small. Then one can use the standard Yoneda lemma in order to derive translation from \textbf{(i)} and \textbf{(ii)}.
\end{proof}

\begin{proposition}\label{proposition:eachmorphismgivesequivalencerelation}
Let $\cC$ be a category with fiber products and let $f:X\ra Y$ be a morphism $\cC$. Consider a cartesian square
\begin{center}
\begin{tikzpicture}
[description/.style={fill=white,inner sep=2pt}]
\matrix (m) [matrix of math nodes, row sep=3em, column sep=3em,text height=1.5ex, text depth=0.25ex] 
{ X \times_Y X  &    X   \\
  X             &    Y   \\};
\path[->,line width=1.0pt,font=\scriptsize]  
(m-1-1) edge node[above] {$ p_2 $} (m-1-2)
(m-2-1) edge node[below] {$ f $} (m-2-2)
(m-1-1) edge node[left] {$ p_1 $} (m-2-1)
(m-1-2) edge node[right] {$ f $} (m-2-2);
\end{tikzpicture}
\end{center}
Then $\left(X \times_Y X,p_1,p_2\right)$ is an equivalence relation on $X$.
\end{proposition}
\begin{proof}
For the proof pick an object of $Z$ and by Fact \ref{fact:relationsintermsofpoints} it suffices to prove that
\begin{center}
\begin{tikzpicture}
[description/.style={fill=white,inner sep=2pt}]
\matrix (m) [matrix of math nodes, row sep=3em, column sep=8em,text height=1.5ex, text depth=0.25ex] 
{ \Mor_{\cC}\left(Z,X\times_Y X\right)  &  \Mor_{\cC}\left(Z,X\times X\right) = \Mor_{\cC}\left(Z,X\right)\times \Mor_{\cC}\left(Z,X\right)   \\} ;
\path[right hook->,line width=1.0pt,font=\scriptsize]  
(m-1-1) edge node[above] {$ \Mor_{\cC}\left(1_Z,\langle p_1,p_2 \rangle\right) $} (m-1-2);
\end{tikzpicture}
\end{center}
is an equivalence relation. Note that we have fiber product of classes
\begin{center}
\begin{tikzpicture}
[description/.style={fill=white,inner sep=2pt}]
\matrix (m) [matrix of math nodes, row sep=5em, column sep=5em,text height=1.5ex, text depth=0.25ex] 
{ \Mor_{\cC}\left(Z,X \times_Y X\right)  &   \Mor_{\cC}\left(Z,X\right)   \\
 \Mor_{\cC}\left(Z, X\right)             &   \Mor_{\cC}\left(Z, Y\right)   \\};
\path[->,line width=1.0pt,font=\scriptsize]  
(m-1-1) edge node[above] {$ \Mor_{\cC}\left(1_Z,p_2\right) $} (m-1-2)
(m-2-1) edge node[below] {$ \Mor_{\cC}\left(1_Z,f\right) $} (m-2-2)
(m-1-1) edge node[left] {$  \Mor_{\cC}\left(1_Z, p_1\right) $} (m-2-1)
(m-1-2) edge node[right] {$ \Mor_{\cC}\left(1_Z,f\right) $} (m-2-2);
\end{tikzpicture}
\end{center}
and hence $\Mor_{\cC}\left(Z, X\times_Y X\right)$ contains these pairs $(g,h)$ of morphisms in $\Mor_{\cC}(Z,X)$ such that $f\cdot g = f\cdot h$. This is clearly an equivalence relation.
\end{proof}


\begin{definition}
Let $\cC$ be a category with fiber products and let $f:X\ra Y$ be a morphism in $\cC$. Consider a cartesian square
\begin{center}
\begin{tikzpicture}
[description/.style={fill=white,inner sep=2pt}]
\matrix (m) [matrix of math nodes, row sep=3em, column sep=3em,text height=1.5ex, text depth=0.25ex] 
{ X \times_Y X  &    X   \\
  X             &    Y   \\};
\path[->,line width=1.0pt,font=\scriptsize]  
(m-1-1) edge node[above] {$ p_2 $} (m-1-2)
(m-2-1) edge node[below] {$ f $} (m-2-2)
(m-1-1) edge node[left] {$ p_1 $} (m-2-1)
(m-1-2) edge node[right] {$ f $} (m-2-2);
\end{tikzpicture}
\end{center}
Then the equivalence relation $\left(X\times_YX,p_1,p_2\right)$ is a called \textit{a kernel pair of $f$}.
\end{definition}

\begin{definition}
Let $\cC$ be a category with fiber products and let $(R,p_1,p_2)$ be a equivalence relation on some object $X$ of $\cC$. The morphism $f:X\ra Y$ is \textit{a quotient of $(R,p_1,p_2)$} if the fork
\begin{center}
\begin{tikzpicture}
[description/.style={fill=white,inner sep=2pt}]
\matrix (m) [matrix of math nodes, row sep=3em, column sep=3em,text height=1.5ex, text depth=0.25ex] 
{R &  X & Y  \\} ;
\path[->,line width=0.8pt,font=\scriptsize]
(m-1-1) edge[transform canvas={yshift=0.5ex}] node[above] {$ p_1  $} (m-1-2)
(m-1-1) edge[transform canvas={yshift=-0.5ex}] node[below] {$ p_2 $} (m-1-2)
(m-1-2) edge node[above] {$ f  $} (m-1-3);
\end{tikzpicture}
\end{center}
is a cokernel of $(p_1,p_2)$. An equivalence relation that admits a quotient is called \textit{effective}.
\end{definition}

\begin{proposition}
Let $\cC$ be a category with fiber products and let $(R,p_1,p_2)$ be an equivalence relation on some object $X$ of $\cC$. If $(R,p_1,p_2)$ is effective, then it is a kernel pair of its quotient.
\end{proposition}
\begin{proof}
Let $f:X\ra Y$ be a quotient of $(R,p_1,p_2)$.
\end{proof}

\begin{proposition}\label{proposition:coequalizersarequotientsoftheirkernelpairs}
Let $\cC$ be a category and let $g_1,g_2:Z\ra X$ be morphisms in $\cC$. Suppose that
\begin{center}
\begin{tikzpicture}
[description/.style={fill=white,inner sep=2pt}]
\matrix (m) [matrix of math nodes, row sep=3em, column sep=3em,text height=1.5ex, text depth=0.25ex] 
{Z &  X & Y  \\} ;
\path[->,line width=0.8pt,font=\scriptsize]
(m-1-1) edge[transform canvas={yshift=0.5ex}] node[above] {$ g_1  $} (m-1-2)
(m-1-1) edge[transform canvas={yshift=-0.5ex}] node[below] {$ g_2 $} (m-1-2)
(m-1-2) edge node[above] {$ f  $} (m-1-3);
\end{tikzpicture}
\end{center}
is a cokernel of $(g_1,g_2)$. Then $f$ is also a cokernel of its kernel pair.
\end{proposition}
\begin{proof}
Let $\left(X\times_YX, p_1,p_2\right)$ be a kernel pair of $f$. Then $\langle g_1,g_2\rangle:Z\ra X\times X$ factors through $\langle p_1,p_2  \rangle$. Hence we have a commutative diagram
\begin{center}
\begin{tikzpicture}
[description/.style={fill=white,inner sep=2pt}]
\matrix (m) [matrix of math nodes, row sep=3em, column sep=3em,text height=1.5ex, text depth=0.25ex] 
{         Z  &  X  &      Y  \\
X\times_Y X  &  X  &        \\};
\path[->,line width=0.8pt,font=\scriptsize]
(m-1-1) edge[transform canvas={yshift=0.5ex}] node[above] {$ g_1  $} (m-1-2)
(m-1-1) edge[transform canvas={yshift=-0.5ex}] node[below] {$ g_2 $} (m-1-2)
(m-1-2) edge node[above] {$ f  $} (m-1-3)
(m-2-1) edge[transform canvas={yshift=0.5ex}] node[above] {$ p_1  $} (m-2-2)
(m-2-1) edge[transform canvas={yshift=-0.5ex}] node[below] {$ p_2 $} (m-2-2)
(m-1-1) edge node[left] {$ u  $} (m-2-1)
(m-1-2) edge node[left] {$ =  $} (m-2-2);
\end{tikzpicture}
\end{center}
in which $u$ is uniquely determined.
\end{proof}

\small
\bibliographystyle{alpha}
\bibliography{zzz}


\end{document}