\input ../pree.tex

\begin{document}

\title{Topics in theory of fields}
\date{}
\maketitle

\section{Introduction}
\noindent
In this notes we discuss more advanced topics in theory of fields. Our main result is famous Mac Lane's characterization (Theorem \ref{theorem:macLanes_criterion}) of infinite separable extensions. From this result we derive interesting characterization of perfect fields in Corollary \ref{corollary:characterization_of_perfect_fields} and geometrically reduced algebras in Proposition \ref{proposition:characterization_of_geometrically_reduced_algebras}.

\section{Separable degree}

\begin{definition}
Suppose that $k\subseteq K$ is a finite extension of fields and $k\subseteq \ol{k}$ is an algebraic closure of $k$. The number of elements of the set
$$\big\{\sigma:K\ra \ol{k}\mid \text{$\sigma$ is a $k$-morphism}\big\}$$
is denoted by $[K:k]_s$ and is called \textit{the separable degree of the extension $k\subseteq K$}. 
\end{definition}

\begin{fact}\label{fact:separable_degree_is_multiplicative_for_tower}
Suppose that $k\subseteq K$ and $K\subseteq L$ are finite extensions of fields. Then $[L:k]_s=[L:K]_s\cdot [K:k]_s$.
\end{fact}

\begin{proposition}\label{proposition:general_form_of_irreducible_polynomial_in_one_variable}
Suppose that $k$ is a field and $f\in k[x]$ is an irreducible polynomial. Then
\begin{center}
$f(x)=\begin{cases} \prod^n_{i=1}(x-a_i) &\mbox{if } \mathrm{char}(k)=0 \\
 \prod^n_{i=1}(x-a_i)^{p^m}& \mbox{if } \mathrm{char}(k)=p>0  \end{cases} $
\end{center}
for pairwise disjoint elements $a_1,...,a_n$ of algebraic closure of $k$.
\end{proposition}
\begin{proof}
Fix an algebraic closure $K$ of $k$. Observe that $f$ has multiple roots if and only if $f$ and $\frac{\partial f}{\partial x}$ have noninvertible common divisors in $K[x]$.\\
Suppose that $\mathrm{char}(k)=0$. Then $\frac{\partial f}{\partial x}\neq 0$ and $f$ cannot have multiple roots, since by irreducibility of $f$ this would imply that $f\mid \frac{\partial f}{\partial x}$ in $k[x]$.\\
Assume now that $\mathrm{char}(k)=p>0$. If $f$ has multiple roots, then $f\mid \frac{\partial f}{\partial x}$ and hence  $\frac{\partial f}{\partial x}=0$. Thus $f(x)=h(x^p)$. Clearly $h$ is irreducible over $k$. We can apply this argument again to $h$. Eventually we derive that $f=g(x^{p^m})$ for some positive $m\in \NN$ and $g$ is irreducible without multiple roots. Hence the result follows.
\end{proof}

\begin{corollary}\label{corollary:separable_degree_divides_degree}
Suppose that $k\subseteq K$ is a finite extension. Then $[K:k]_s\mid [K:k]$.
\end{corollary}
\begin{proof}
This is a combination of Fact \ref{fact:separable_degree_is_multiplicative_for_tower} and Proposition \ref{proposition:general_form_of_irreducible_polynomial_in_one_variable}.
\end{proof}

\begin{definition}
Suppose that $k$ is a field and $f$ is an irreducible polynomial in $k[x]$. Then $f$ is \textit{separable} if and only if it has no multiple roots.
\end{definition}

\begin{definition}
Suppose that $k\subseteq K$ is an extension of fields and $a$ is an element of $K$ algebraic over $k$. Then $a$ is \textit{separable} if and only if its minimal polynomial over $k$ is separable.
\end{definition}


\begin{corollary}\label{corollary:equivalent_conditions_on_finite_separable_extensions}
Suppose that $k\subseteq K$ is a finite extension. Then the following assertions are equivalent.
\begin{enumerate}[label=\emph{\textbf{(\roman*)}}, leftmargin=3.0em]
\item Every $a$ in $K$ is separable over $k$.
\item $K$ is generated by elements separable over $k$.
\item $[K:k]_s=[K:k]$
\end{enumerate}
\end{corollary}

\begin{theorem}[Abel's theorem]\label{theorem:abels_primitive_element_theorem}
Suppose that $k\subseteq K$ is a finite extension of fields generated by separable elements. Then there exists an element $a$ in $K$ such that $K=k(a)$. 
\end{theorem}
\begin{proof}
This is clear for $k$ finite.\\
Suppose that $k$ is infinite. By easy induction we may assume that $K=k(b,c)$. Let $\sigma_1,...\sigma_n$ be all distinct $k$-morphisms of $K$ into some algebraic closure $\ol{k}$ of $k$. Then $n=[K:k]$ by Corollary \ref{corollary:equivalent_conditions_on_finite_separable_extensions}. Consider the polynomial
$$f(x)=\prod_{1\leq i< j\leq n}\bigg(\big(\sigma_i(b)-\sigma_j(b)\big)x+\big(\sigma_i(c)-\sigma_j(c)\big)\bigg)$$
Since $k$ is infinite, we may assume that there exists $l\in k$ such that $f(l)\neq 0$. Then we have $\sigma_i(bl+c)\neq \sigma_j(bl+c)$ for all $i\neq j$. Hence
$$[k(bl+c):k] \leq [K:k]=n= [k(bl+c):k]_s\leq [k(bl+c):k]$$
Thus $k(bl+c)=K$.
\end{proof}

\begin{definition}
Suppose that $k\subseteq K$ is a finite extension of fields. Then $[K:k]_i=\frac{[K:k]}{[K:k]_s}$ is called \textit{inseparable degree}.
\end{definition}

\begin{corollary}
Suppose that $k\subseteq K$ is a finite extension of fields. Then $[K:k]_i$ is some power of $\mathrm{char}(k)$.
\end{corollary}

\begin{definition}
A field $k$ is \textit{perfect} if for every finite algebraic extension $k\subseteq K$ every element of $K$ is separable over $k$.
\end{definition}

\section{Purely inseparable extensions}
\begin{definition}
Let $k\subseteq K$ be a fields extension. We say that it is \textit{purely inseparable} if for every extension of fields $k\subseteq L$ ring $K\otimes_kL$ has the unique prime ideal.
\end{definition}

\begin{proposition}\label{proposition:characterization_of_purely_inseparable_extensions}
Suppose that $k\subseteq K$ is an extension of fields. Then the following assertions are equivalent.
\begin{enumerate}[label=\emph{\textbf{(\roman*)}}, leftmargin=3.0em]
\item $k\subseteq K$ is purely inseparable.
\item For every fields extension $k\subseteq L$ there exists at most one morphism $K\ra L$ over $k$.
\item $\mathrm{char}(k)=p>0$ and for every $a$ in $K$ element $a^{p^n}$ is in $k$ for some $n\in \NN$. 
\end{enumerate}
\end{proposition}
\begin{proof}
Suppose that $k\subseteq K$ is purely inseparable and assume that $f,g:K\ra L$ are $k$-morphisms. Let $\ol{f}:K\otimes_kL\ra L$ and $\ol{g}:K\otimes_kL\ra L$ be morphisms given by $\ol{f}(x\otimes y)=f(x)\cdot y$ and $\ol{g}(x\otimes y)=g(x)\cdot y$. Then $\ol{f}$ and $\ol{g}$ are surjective morphisms of $L$-algebras. Since $K\otimes_kL$ has the unique prime ideal, we have that $\ol{f}=\ol{g}$. Therefore, $f(x)=\ol{f}(x\otimes 1)=\ol{g}(x\otimes 1)=g(x)$.\\
Suppose that the second assertion holds. Then clearly $k\subseteq K$ is algebraic. Suppose that there exists $a\in K\setminus k$ that is separable over $k$. Then $k\ra k(a)$ have at least two distinct $k$-morphisms to $\ol{k}$. Thus $K$ has at least two distinct $k$-morphisms to $\ol{k}$. Therefore, there is no $a\in K\setminus k$ that is separable. This shows that $\mathrm{char}(k)=p>0$ and for any element $a\in K$ its minimal polynomial over $k$ is of the form $x^{p^n}-a^{p^n}$.\\
Suppose now that the third assertion holds. Fix an extension $k\ra L$ of fields. Then we have an integral extension of rings $L\ra K\otimes_kL$ such that, for every $x$ in $K\otimes_kL$ we have $x^{p^{n}}\in L$. Hence $K\otimes_kL$ has the unique prime ideal.
\end{proof}


\section{Separable extensions}

\begin{definition}
Suppose that $k\subseteq K$ is an extension of fields. If for every extension $k\subseteq L$ of fields ring $K\otimes_kL$ is reduced, then $k\subseteq K$ \textit{is a separable extension}.
\end{definition}

\begin{definition}
Suppose that $k\subseteq K$ is an extension of fields. A set of elements $\{x_i\}_{i\in I}$ of $K$ is \textit{a separating transcendence base of $K$ over $k$} if it is a transcendence base of $K$ over $k$ and $k(\{x_i\}_{ i\in I})\subseteq K$ is an algebraic extension generated by separable elements.
\end{definition}

\begin{theorem}[Mac Lane's characterization]\label{theorem:macLanes_criterion}
Let $k$ be a field of characteristic $p > 0$ and $k\subseteq K$ be an extension of fields. Then the following assertions are equivalent.
\begin{enumerate}[label=\emph{\textbf{(\roman*)}}, leftmargin=3.0em]
\item $k\subseteq K$ is separable.
\item $K\otimes_kk^{\frac{1}{p^{\infty}}}$ is reduced.
\item $K\otimes_kk^{\frac{1}{p}}$ is reduced.
\item Every finite set $x_1,...x_n\in K$ contains a separating transcendence base of $k(x_1,...,x_n)$ over $k$. 
\end{enumerate}
\end{theorem}
\begin{proof}
The only nontrivial implications are $\textbf{(iii)}\Rightarrow \textbf{(iv)}$ and $\textbf{(iv)}\Rightarrow \textbf{(i)}$.\\
Suppose that the third assertion hold. Since $k\subseteq k^{\frac{1}{p}}$ is purely inseparable and $K\otimes_kk^{\frac{1}{p}}$ is reduced, we derive by Proposition \ref{proposition:characterization_of_purely_inseparable_extensions} that $K\otimes_kk^{\frac{1}{p}}$ is a field. Fix algebraically closed field $L$ containing $K$ and consider a monomorphism $K\otimes_kk^{\frac{1}{p}}\ra L$. Its image is a subfield of $L$ isomorphic with $K\otimes_kk^{\frac{1}{p}}$. We denote it by $K k^{\frac{1}{p}}$. Assume that $k(x_1,...,x_{s})\subseteq K$ and $\mathrm{tr}_kk(x_1,...,x_{s}) = s-1$. Suppose that $f(t_1,...,t_{s})\in k[t_1,...,t_{s}]$ is a nonzero polynomial such that
$$f(x_1,...,x_{s})=0$$
Moreover, we may assume that $f$ is of the smallest total degree among such polynomials in $k[t_1,...,t_s]$. Write
$$f(t)=\sum_{\alpha \in \NN^{s}}c_{\alpha}M_{\alpha}(t)$$
where $M_{\alpha}(t)=t^{\alpha_1}_1...t^{\alpha_{s}}_{s}$ and $\alpha=(\alpha_1,...,\alpha_{s})$. Let $M_{\alpha}(x)$ be an evaluation of $M_{\alpha}(t)$ in $(x_1,...,x_{s})$. If $p\mid \alpha$ for every $\alpha\in \NN^{s}$ such that $c_{\alpha}\neq 0$, then $f=g^p$ for some polynomial $g\in  k^{\frac{1}{p}}[t_1,...,t_{s}]$. In this case $$0=f(x_1,...,x_{s})=g(x_1,...,x_{s})^p$$
and $g(x_1,...,x_{s})\in Kk^{\frac{1}{p}}$. Moreover, elements $M_{\frac{\alpha}{p}}(x)$ for $c_{\alpha}\neq 0$ are linearly indepedent over $k$, since their linear dependence will lead to contradiction with the fact that $f$ is of the smallest total degree. Since $k(x_1,...,x_{s})\otimes_kk^{\frac{1}{p}}\ra Kk^{\frac{1}{p}}$ is a monomorphism, we derive that elements $M_{\frac{\alpha}{p}}(x)$ for $c_{\alpha}\neq 0$ are linearly indepedent over $k^{\frac{1}{p}}$. Thus $g(x_1,...,x_{s})\neq 0$, which leads to a contradiction with the fact that $g(x_1,...,x_s)^p = 0$. Therefore, there exists $\gamma \in \NN^{s}$ such that, $c_{\gamma}\neq 0$ and $p \nmid \gamma$. There exists $r\in \{1,...,s\}$ such that $p\nmid \gamma_r$. Thus by minimality assumption on $f$ and Proposition \ref{proposition:general_form_of_irreducible_polynomial_in_one_variable} we can interpreted $f(x_1,...,x_{s})=0$ as an irreducible algebraic relation of $x_r$ over $k(x_1,...,x_{r-1},x_{r+1},...,x_{s})$ in such a way that $x_r$ becomes separable over $k(x_1,...,x_{r-1},x_{r+1},...,x_s)$. Now the general case of \textbf{(iv)} follows by easy induction.\\
Suppose now that \textbf{(iv)} holds. Fix a field extension $k\subseteq L$. Pick any finite set of elements $x_1,...x_n\in K$. We may assume that $x_1,...,x_m$ for some $m\leq n$ form a separating transcendence base of $k(x_1,...,x_n)$ over $k$. Then
$$k(x_1,...,x_n)\otimes_kL\cong k(x_1,...,x_n)\otimes_{k(x_1,...,x_m)}(k(x_1,...,x_m)\otimes_kL)\subseteq k(x_1,...,x_n)\otimes_{k(x_1,...,x_m)}L(x_1,...,x_m)$$
is reduced by Theorem \ref{theorem:abels_primitive_element_theorem}. Thus $K\otimes_kL$ is reduced, since $K$ is a colimit of finitely generated extensions.
\end{proof}

\begin{corollary}\label{corollary:characterization_of_perfect_fields}
Suppose that $k$ is a field. Then the following assertions are equivalent.
\begin{enumerate}[label=\emph{\textbf{(\roman*)}}, leftmargin=3.0em]
\item Every extension $k\subseteq K$ is separable.
\item $k=k^{\frac{1}{p^{\infty}}}$
\item $k$ is perfect.
\end{enumerate}
\end{corollary}

\begin{definition}
Let $A$ be an algebra over a field $k$. Suppose that for every reduced $k$-algebra $B$ tensor product $A\otimes_kB$ is reduced. Then $A$ is \textit{geometrically reduced}.
\end{definition}

\begin{proposition}\label{proposition:characterization_of_geometrically_reduced_algebras}
Let $A$ be an algebra over a field $k$. Then the following assertions are equivalent.
\begin{enumerate}[label=\emph{\textbf{(\roman*)}}, leftmargin=3.0em]
\item $A$ is a geometrically reduced $k$-algebra.
\item $A\otimes_k\ol{k}$ is reduced.
\item $A\otimes_kk^{\frac{1}{p^{\infty}}}$ is reduced.
\end{enumerate}
\end{proposition}
\begin{proof}
We show that $\textbf{(iii)}\Rightarrow \textbf{(i)}$. First let us make two general remarks.\\
Suppose that $A$ and $B$ are some $k$-algebras, $B$ is reduced and we want to show that $A\otimes_kB$ is reduced. Since every $k$-algebra is a colimit of its finitely generated subalgebras, we may assume that $B$ is a finitely generated and reduced $k$-algebra. Next, using the fact that noetherian ring has finitely many minimal primes, we may assume that $B$ is a product of finitely many domains. Finally each domain can be embedded into its field of fractions. Thus we may assume that $B$ is a field over $k$.\\
Secondly observe that if $k\subseteq L$ is a separable extension, then $L$ is geometrically reduced over $k$. This follows easily from what we have said above.\\
Now we can prove the implication. By the first remark it suffices to show that $A\otimes_kL$ is reduced for every extension $k\subseteq L$ of fields. Obviously we may assume that $L$ contains $k^{\frac{1}{p^{\infty}}}$. Since the field $k^{\frac{1}{p^{\infty}}}$ is perfect, the extension $k^{\frac{1}{p^{\infty}}}\subseteq L$ is separable. By the second remark, $L$ is geometrically reduced over $k^{\frac{1}{p^{\infty}}}$. Now observe that
$$A\otimes_kL\cong (A\otimes_kk^{\frac{1}{p^{\infty}}})\otimes_{k^{\frac{1}{p^{\infty}}}}L$$
is reduced, since $A\otimes_kk^{\frac{1}{p^{\infty}}}$ is reduced and $k^{\frac{1}{p^{\infty}}}\subseteq L$ is geometrically reduced. This completes the proof of the implication.\\
The implications $\textbf{(i)}\Rightarrow \textbf{(ii)}$ and $\textbf{(ii)}\Rightarrow \textbf{(iii)}$ are immediate.
\end{proof}




































\end{document}