\input ../pree.tex

\begin{document}

\title{Uniform spaces}
\date{}
\maketitle

\section{Introduction}
\noindent
This notes are devoted to uniform spaces. In the first section we prove important result on existence of pseudometrics originally due to Weil. This result is crucial for further developments.

\section{Existence of pseudometrics}
\noindent
Let $X$ be a set. We start by introducing some set-theoretic notions concerning subsets of $X\times X$.

\begin{definition}
Let $X$ be a set. A subset $V$ of $X\times X$ such that
$$\forall_{x\in X}\,(x,x)\in V,\,\forall_{x,y\in X}\,(x,y)\in V\,\Leftrightarrow\,(y,x)\in V$$
is \textit{a surrounding of the diagonal in $X\times X$}.    
\end{definition}

\begin{definition}
Let $X$ be a set and let $V,W$ be subsets of $X\times X$. We define a subset $W\cdot V$ of $X\times X$ called \textit{a composition of $V$ with $W$} such that 
$$(x,z) \in W\cdot V\,\Leftrightarrow\,\exists_{y \in X}\,(x,y)\in V\mbox{ and }(y,z) \in W$$
for each $x,z\in X$.
\end{definition}
\noindent
Finally we recall the notion of pseudometric.

\begin{definition}
Let $X$ be a set. Suppose that a function $\rho:X\times X\ra [0,+\infty)$ satisfies the following assertions.
\begin{enumerate}[label=\textbf{(\arabic*)}, leftmargin=*]
\item $\rho(x,x) = 0$ for all $x\in X$.
\item $\rho(x,y) = \rho(y,x)$ for all $x,y\in X$.
\item $\rho(x,z)\leq \rho(x,y) + \rho(y,z)$ for all $x,y,z\in X$.
\end{enumerate}
Then $\rho$ is \textit{a pseudometric on $X$}.
\end{definition}
\noindent
Now we state and prove the fundamental result on the existence of pseudometrics.

\begin{theorem}\label{theorem:Weils_theorem_on_pseudometrics}
Let $X$ be a set and let $\{V_n\}_{n\in \NN}$ be a sequence of surroundings of the diagonal in $X\times X$ such that
$$V_{n+1}\cdot V_{n+1}\cdot V_{n+1} \subseteq V_n$$
for every $n\in \NN$. Then there exists a pseudometric $\rho$ on $X$ bounded by $1$ such that
$$\bigg\{(x,y)\in X\times X\,\bigg|\,\rho(x,y)<\frac{1}{2^n}\bigg\} \subseteq V_n \subseteq \bigg\{(x,y)\in X\times X\,\bigg|\,\rho(x,y) \leq \frac{1}{2^n}\bigg\}$$
for every $n\in \NN$.
\end{theorem}
\noindent
For the proof consider a function $f$ defined on $X\times X$ given by formula 
$$\begin{cases}
0 & \mbox{ if }(x,y)\in V_n\mbox{ for each }n\in \NN\\
\frac{1}{2^n} & \mbox{ if }(x,y)\in V_n\setminus V_{n+1}\\
1 & \mbox{ if }(x,y)\not \in V_0
\end{cases}$$
The proof relies on the following result. 

\begin{lemma}\label{lemma:inclusion_of_surroundings_for_pseudometric}
For each $n \in \NN$ and every finite sequence $x_0,...,x_m$ the inequality
$$\sum_{i=1}^mf(x_{i-1},x_i) < \frac{1}{2^n}$$
implies that $(x_0,x_m) \in V_n$.
\end{lemma}
\begin{proof}[Proof of the lemma]
The proof goes by induction on $m$. For $m = 0$ and $m = 1$ the claim is trivial. Assume that $m$ is greater than one and suppose that the assertion holds for all numbers smaller than $m$. Suppose that
$$\sum_{i=1}^mf(x_{i-1},x_i) < \frac{1}{2^n}$$
for some sequence $x_0,...,x_m$ of elements in $X$. We have
$$\mbox{ either }f(x_0,x_1) < \frac{1}{2^{n+1}}\mbox{ or }f(x_{m-1},x_m) < \frac{1}{2^{n+1}}$$
Without loss of generality we may assume that the first inequality holds. Let $k$ be the greatest number in $\{1,...,m-1\}$ such that
$$\sum_{i=1}^kf(x_{i-1},x_i) < \frac{1}{2^{n+1}}$$
Next we consider two cases.
\begin{itemize}
\item If $k < m-1$, then we have
$$\sum_{i=1}^{k}f(x_{i-1},x_i) < \frac{1}{2^{n+1}},\,f(x_k,x_{k+1}) \leq \frac{1}{2^{n+1}},\,\sum_{i=k+1}^mf(x_{i-1},x_i) < \frac{1}{2^{n+1}}$$
By induction hypothesis we have $(x_0,x_k)\in V_{n+1},(x_{k+1},x_m)\in V_{n+1}$ and by definition of $f$ we have $(x_k,x_{k+1})\in V_{n+1}$. Hence
$$(x_0,x_m) \in V_{n+1}\cdot V_{n+1}\cdot V_{n+1}\subseteq V_n$$
and the assertion holds.
\item If $k = m-1$. Then 
$$\sum_{i=1}^{m-1}f(x_{i-1},x_i) < \frac{1}{2^{n+1}},\,f(x_{m-1},x_m)\leq \frac{1}{2^{n+1}}$$
By induction hypothesis we have $(x_0,x_{m-1})\in V_{n+1}$ and by definition of $f$ we have $(x_{m-1},x_{m})\in V_{n+1}$. Hence
$$(x_0,x_{m}) \in V_{n+1}\cdot V_{n+1} \subseteq V_{n+1}\cdot V_{n+1}\cdot V_{n+1} \subseteq V_n$$
and the assertion holds.
\end{itemize}
Thus the result follows from induction.
\end{proof}

\begin{proof}[Proof of the theorem]
For $x,y\in X$ we define
$$\rho(x,y) = \inf \bigg\{\sum_{i=1}^mf(x_{i-1},x_i)\,\bigg|\,\mbox{ for every }m\in \NN\mbox{ an every finite sequence }x_0,...,x_m\mbox{ such that }x_0 = x,\,x_m = y\bigg\}$$
It is easy to verify that the function $\rho$ is a pseudometric on $X$. It remains to prove that
$$\bigg\{(x,y)\in X\times X\,\bigg|\,\rho(x,y)<\frac{1}{2^n}\bigg\} \subseteq V_n \subseteq \bigg\{(x,y)\in X\times X\,\bigg|\,\rho(x,y) \leq \frac{1}{2^n}\bigg\}$$
The first inclusion follows from Lemma \ref{lemma:inclusion_of_surroundings_for_pseudometric} and the second follows from the fact that $\rho(x,y) \leq f(x,y)$ for every $x,y\in X$.
\end{proof}

\section{Uniform structures and uniform spaces}
\noindent
In this section we introduce main object of our study.

\begin{definition}
Let $X$ be a set. Suppose that $\fU$ is a collection of surroundings of the diagonal in $X\times X$ which satisfies the following two assertions.
\begin{enumerate}[label=\textbf{(\arabic*)}, leftmargin=*]
\item If $U \in \fU$ and $W$ is a surrounding of the diagonal in $X\times X$ such that $V\subseteq W$, then $W\in \fU$.
\item If $U,W\in \fU$, then $U\cap W \in \fU$. 
\item If $U \in \fU$, then there exists $W\in \fU$ such that $W\cdot W \subseteq U$.
\end{enumerate}
Then $\fU$ is \textit{a uniform structure on $X$}.
\end{definition}

\begin{example}\label{example:discrete_uniform_structure}
Let $X$ be a set. Then the family $\fD_X$ of all surrounding of the diagonal in $X\times X$ is a uniform structure on $X$. It is called \textit{the discrete uniform structure on $X$}.
\end{example}

\begin{fact}\label{fact:uniform_structures_are_closed_under_intersections}
Let $X$ be a set and let $\{\fU_i\}_{i\in I}$ be a family of uniform structures on $X$. Then 
$$\bigcap_{i\in I}\fU_i$$
is a uniform structure on $X$.
\end{fact}
\begin{proof}
Left for the reader.
\end{proof}

\begin{corollary}\label{corollary:smallest_uniform_structure_containing_given_family_of_surroundings}
Let $X$ be a set and let $\cF$ be a family of surrounding of the diagonal in $X\times X$. Then there exists the smallest (with respect to inclusion) uniform structure $\fU$ on $X$ which contain $\cF$.
\end{corollary}
\begin{proof}
Let $\{\fU_i\}_{i\in I}$ be a family of all uniform structures on $X$ which contain $\cF$. The family is nonempty, since it contains the discrete uniform structure on $X$. The intersection $$\fU = \bigcap_{i\in I}\fU_i$$
is a uniform structure on $X$ by Fact \ref{fact:uniform_structures_are_closed_under_intersections}. Hence it is the smallest uniform structure on $X$ which contain $\cF$.
\end{proof}

% \begin{example}[Uniform structure induced by the family of pseudometrics]\label{example:uniform_structure_introduced_by_the_set_of_pseudometrics}
% Let $X$ be a set and let $\fP$ be a family of pseudometrics on $X$. Then a collection
% $$\bigg\{V\,\bigg|\,\exists_{\epsilon>0}\,\exists_{\rho \in \fP}\,\big\{(x,y)\in X\times X\,\big|\,\rho(x,y)\leq \epsilon\big\}\subseteq V\bigg\}$$
% is a uniform structure on $X$. 
% \end{example}

\begin{definition}
A pair $(X,\fU)$ consisting of a set $X$ and a uniform structure $\fU$ on $X$ is \textit{a uniform space}.
\end{definition}

\begin{definition}
Let $(X,\fU)$ be a uniform space. A surrounding $V$ in $\fU$ is called \textit{an entourage of the diagonal in $(X,\fU)$}. 
\end{definition}

\begin{definition}
Let $(X,\fU),(Y,\fV)$ be uniform spaces and let $f:X\ra Y$ be a map. Suppose that $\left(f\times f\right)^{-1}(V) \in \fU$ for every $V\in \fV$. Then $f$ is \textit{a morphism of uniform spaces}. 
\end{definition}

\begin{remark}\label{remark:category_of_uniform_spaces}
Uniform spaces and their morphisms form a category. We denote this category by $\Unif$.
\end{remark}
\noindent
Now we study limits in $\Unif$. For this we use the following result.

\begin{theorem}\label{theorem:description_of_uniform_structure_introduced_by_a_family_of_maps}
Let $X$ be a set and let $\{(X_i,\fU_i)\}_{i\in I}$ be a family of uniform spaces. Consider a family $\big\{f_i:X\ra X_i\big\}_{i\in I}$ of maps. Suppose that $\fU$ is the smallest uniform structure on $X$ which makes $\{f_i\}_{i\in I}$ into a family of uniform morphisms. Then 
$$U \in \fU$$
if and only if there exist $n\in \NN_+$, $i_1,...,i_n\in I$ and $U_1 \in \fU_{i_1},...,U_n\in \fU_{i_n}$ such that
$$\bigcap_{k=1}^n\left(f_{i_k}\times f_{i_k}\right)^{-1}(U_k) \subseteq U$$
\end{theorem}
\begin{proof}
Consider the family $\cU$ of all surrounding $U$ of the diagonal in $X\times X$ such that there exist $n\in \NN_+$, $i_1,...,i_n\in I$ and $U_1 \in \fU_{i_1},...,U_n\in \fU_{i_n}$ satisfying
$$\bigcap_{k=1}^n\left(f_{i_k}\times f_{i_k}\right)^{-1}(U_k) \subseteq U$$
It is easy to verify (we left for the reader) that $\cU$ is a uniform structure on $X$. Moreover, for every $n\in \NN_+$, $i_1,...,i_n\in I$ and $U_1 \in \fU_{i_1},...,U_n\in \fU_{i_n}$ we have
$$\bigcap_{k=1}^n\left(f_{i_k}\times f_{i_k}\right)^{-1}(U_k) \in \fU$$
Hence $\cU \subseteq \fU$. Note also that $f_i$ is a uniform morphism $(X,\cU)\ra (X_i,\fU_i)$ for each $i\in I$. Thus $\fU \subseteq \cU$. Therefore, $\cU = \fU$ and this proves the theorem. 
\end{proof}

\begin{definition}
Let $(X,\fU)$ be a uniform space and let $Z$ be a subset of $X$. Then $Z$ together with the smallest uniform structure which makes the inclusion $Z\hookrightarrow X$ into a uniform morphism is \textit{a uniform subspace of $(X,\fU)$ with $Z$ as the underlying set}.
\end{definition}

\section{Topology induced by uniform structure}
\noindent
We start by introducing the notion of a ball with respect to surrounding of the diagonal.

\begin{definition}
Let $X$ be a set. For every $x$ in $X$ and $U$ in $\fD_X$ the set
$$B(x,U) = \big\{y\in X\,\big|\,(x,y)\in U\big\}$$
is \textit{a ball with center $x$ and radius $U$}.
\end{definition}

\begin{fact}\label{fact:topology_induced_by_uniform_structure}
Let $X$ be a set and let $\fU$ be a uniform structure on $X$. The family
$$\tau_{\fU} = \big\{\cO\subseteq X\,\big|\,\mbox{ for each }x\in \cO\mbox{ there exists }U\in \fU\mbox{ such that }B(x,U)\subseteq \cO\big\}$$
is a topology on $X$.
\end{fact}
\begin{proof}
We left the proof for the reader as an exercise.
\end{proof}

\begin{definition}
Let $X$ be a set and let $\fU$ be a uniform structure on $X$. Then the topology $\tau_{\fU}$ is \textit{the topology on $X$ induced by $\fU$}.
\end{definition}
\noindent
The following result is a useful property of a topology induced by a uniform structure. 

\begin{proposition}\label{proposition:each_ball_contains_an_open_subset_with_respect_to_topology_induced_by_uniform_structure}
Let $(X,\fU)$ be a uniform space and let $U \in \fU$. For every $x$ in $X$ there exists an open neighborhood $\cO$ of $x$ with respect to $\tau_{\fU}$ such that $\cO\subseteq B(x,U)$.
\end{proposition}
\begin{proof}[Proof of the lemma]
We pick $U_1 \in \fU$ such that $U_1\cdot U_1 \subseteq U$. Next suppose that $U_n$ is defined for some $n\in \NN_+$. Then there exists $U_{n+1}\in \fU$ such that $U_{n+1}\cdot U_{n+1} \subseteq U_n$. Thus by recursive method we construct a sequence $\{U_n\}_{n\in \NN}$ of elements of $\fU$. Easy induction shows that
$$U_1\cdot U_2\cdot ...\cdot U_n \subseteq U$$
for each $n\in \NN_+$. Then
$$\cO = \bigcup_{n\in \NN_+}B\big(x, U_{1}\cdot U_{2}\cdot ...\cdot U_{n}\big)$$
is in $\tau_{\fU}$ and is a subset of $B(x,U)$.
\end{proof}

\begin{example}\label{example:uniform_structure_on_interval}
The interval $[0,1]$ admits a uniform structure given by
$$\big\{U\subseteq \fD_{[0,1]}\,\big|\,\mbox{ there exists }\epsilon > 0\mbox{ such that }|x - y| < \epsilon\mbox{ for some }x,y\in [0,1]\mbox{ implies }(x,y)\in U\big\}$$
Now the topology induced by this uniform structure coincides with the natural topology on $[0,1]$.
\end{example}

\begin{fact}\label{fact:uniform_morphism_is_a_continuous_map}
Let $(X,\fU)$ and $(Y,\fV)$ be uniform spaces and let $f:X\ra Y$ be a morphism of uniform spaces. Then $f$ is a continuous map $\left(X,\tau_{\fU}\right)\ra \left(Y,\tau_{\fY}\right)$.
\end{fact}
\begin{proof}
Pick open subset $\cO$ with respect to the topology induced by $\fY$ on $Y$. Suppose that $f(x) \in \cO$ for some $x$ in $X$. Then there exists $V_x \in \fY$ such that $B(f(x),V_x)\subseteq \cO$. Note that the image of $B\big(x,\left(f\times f\right)^{-1}(V_x)\big)$ under $f$ is contained in $\cO$. Therefore,
$$f^{-1}(\cO) = \bigcup_{x\in f^{-1}(\cO)}B\big(x,\left(f\times f\right)^{-1}(V_x)\big)$$
is open in the topology induced by $\fU$.
\end{proof}
\noindent
Fact \ref{fact:topology_induced_by_uniform_structure} and Fact \ref{fact:uniform_morphism_is_a_continuous_map} imply the existence of the functor 
$$\Unif \ni (X,\fU) \mapsto (X,\tau_{\fU})\in   \Top$$
In the remaining part of this section we shall investigate the properties of this functor. We start by describing the image of the functor.

\begin{definition}
Let $X$ be a topological space. Suppose that for every closed subset $F$ of $X$ and for every point $x$ in $X\setminus  F$ there exists a continuous function $f:X\ra [0,1]$ such that $f(F) \subseteq \{1\}$ and $f(x) = 0$. Then $X$ is \textit{a completely regular space}. 
\end{definition}

\begin{theorem}\label{theorem:image_of_the_canonical_functor_is_completely_regular_space}
The image of the object part of the functor
$$\Unif \ni (X,\fU) \mapsto (X,\tau_{\fU}) \in \Top$$
consists of the class of completely regular spaces.
\end{theorem}
\begin{proof}
Let $(X,\fU)$ be a uniform space. Consider a closed set $F$ with respect to $\tau_{\fU}$. Let $x$ be a point in $X\setminus F$. Since $x \not \in F$ and $F$ is closed in $\tau_{\fU}$, we derive that there exists $U \in \fU$ such that $B(x,U)\cap F = \emptyset$. Next we define a sequence $\{V_n\}_{n\in \NN}$ of elements in $\fU$ by recursion. We set $V_0 = U$ and if $V_0,...,V_n$ are defined for some $n\in \NN$, then we pick an element $V_{n+1}$ of $\fU$ such that 
$$V_{n+1}\cdot V_{n+1}\cdot V_{n+1} \subseteq V_n$$
According to Theorem \ref{theorem:Weils_theorem_on_pseudometrics} there exists a pseudometric $\rho$ on $X$ bounded by $1$ such that
$$\bigg\{(x,y)\in X\times X\,\bigg|\,\rho(x,y)<\frac{1}{2^n}\bigg\} \subseteq V_n \subseteq \bigg\{(x,y)\in X\times X\,\bigg|\,\rho(x,y) \leq \frac{1}{2^n}\bigg\}$$
for every $n\in \NN$. Note that
$$|\rho(x,y_1) - \rho(x,y_2)| \leq \rho(y_1,y_2)$$
for any pair $y_1,y_2\in X$. Indeed, this is the triangle inequality for $\rho$. Thus if $(y_1,y_2) \in V_n$ for some $n\in \NN$, then
$$|\rho(x,y_1) - \rho(x,y_2)| \leq \rho(y_1,y_2) \leq \frac{1}{2^n}$$
Hence the map $f:X \ra [0,1]$ given by formula $f(y) = \rho(x,y)\in [0,1]$ is a morphism of uniform spaces, where $X$ is a uniform space with respect to $\fU$ and $[0,1]$ is considered with uniform structure described in Example \ref{example:uniform_structure_on_interval}. This implies (by Fact \ref{fact:uniform_morphism_is_a_continuous_map}) that $f$ is a continuous map, where $X$ carries topology $\tau_{\fU}$ and $[0,1]$ is considered with natural topology. Pick $y\in F$. Then $y \not \in B(x,U)$ and hence $(x,y) \not \in U$. Since $V_0 = U$ and $\rho$ is bounded by $1$, we derive that $f(y) = \rho(x,y) = 1$. On the other hand $f(x) = \rho(x,x) = 0$. Therefore, $f(F) \subseteq \{1\}$ and $f(x) = 0$. Thus $(X,\tau_{\fU})$ is a completely regular space.\\
Suppose now that $(X,\tau)$ is a completely regular space. Consider the set $C(\tau,\RR)$ of all continuous real valued functions on $(X,\tau)$. For $m\in \NN_+$ and set of $m$ functions $f_1,...,f_m\in C(\tau,\RR)$ define
$$\rho_{f_1,...,f_m}(x,y) = \max \big\{|f_1(x) - f_1(y)|,...,|f_m(x) - f_m(y)|\big\}$$
where $x,y\in X$. Clearly $\rho_{f_1,...,f_m}$ is a pseudometric on $X$. Next consider a family $\fU$ of all $U\in \fD_X$ such that there exist a finite subset $\{f_1,...,f_m\}\subseteq C(\tau,\RR)$ for some $m \in \NN_+$ and $\epsilon > 0$ such that
$$\big\{(x,y)\in X\times X\,\big|\,\rho_{f_1,...,f_m}(x,y) < \epsilon\big\}\subseteq U$$
Clearly $\fU$ is a uniform structure on $X$. Suppose that $\cO \in \tau_{\fU}$. Then for each point $z$ in $\cO$ there exists $U$ in $\fU$ such that $B(z,U)\subseteq \cO$. By definition there exist a finite subset $\{f_1,...,f_m\}\subseteq C(\tau,\RR)$ for some $m \in \NN_+$ and $\epsilon > 0$ such that
$$\big\{(x,y)\in X\times X\,\big|\,\rho_{f_1,...,f_m}(x,y) < \epsilon\big\}\subseteq U$$
Thus
$$\bigcap_{i=1}^mf_i^{-1}\bigg(\big(f_i(z) - \epsilon, f_i(z) + \epsilon\big)\bigg) = \big\{y\in X \,\big|\, \rho_{f_1,...,f_m}(z,y) < \epsilon\big\} \subseteq B(z,U)\subseteq \cO$$
Since $f_1,...,f_m$ are continuous on $X$ with respect to $\tau$, we derive from the inclusion above that there exists an open neighborhood of $z$ with respect to $\tau$ contained in $\cO$. According to the fact that $z$ is an arbitrary point in $\cO$ it follows that $\cO \in \tau$. This proves that $\tau_{\fU}\subseteq \tau$. Now we prove the converse. For this assume that $\cO \in \tau$. We claim that $\cO$ is also open in the topology induced by $\fU$. For this pick $z \in \cO$. Since $(X,\tau)$ is completely regular, there exists a function $f_z:X\ra \RR$ continuous with respect to $\tau$ such that $f_z(X\setminus \cO)\subseteq \{1\}$ and $f_z(z) = 0$. Let $U_z$ be a set consisting of all pairs in $X\times X$ for which $\rho_{f_z}$ is smaller than $1$. Then $U_z\in \fU$ and obviously $B(z,U_z)\subseteq \cO$. Thus
$$\cO = \bigcup_{z\in \cO}B(z,U_z)$$
and this proves the claim that $\cO$ is in $\tau_{\fU}$. Hence $\tau \subseteq \tau_{\fU}$. This completes the proof.
\end{proof}
\noindent
Next we prove the following important fact.

\begin{proposition}\label{proposition:the_induced_topology_preserves_uniform_subspaces}
Let $(X,\fU)$ be a uniform space and let $Z$ be a subset of $X$. Let $\fU_Z$ be the subspace uniform structure on $Z$. Then $\tau_{\fU_Z}$ coincide with the subspace topology on $Z$ induced by $\tau_{\fU}$.
\end{proposition}
\begin{proof}
Let $\cO$ be a set in $\tau_{\fU}$. Then for each $x$ in $\cO$ there exists $U_x\in \fU$ such that $B(x,U_x)\subseteq \cO$. Thus
$$\cO \cap Z = \bigcup_{z\in \cO\cap Z}B\big(z,U_z\cap \left(Z\times Z\right)\big)$$
Since $U_z\cap \left(Z\times Z\right) \in \fU_Z$ for every $z \in \cO\cap Z$, it follows that $\cO\cap Z$ is open with respect to $\tau_{\fU_Z}$. hence
$$\big\{Z\cap \cO\,\big|\,\cO \in \tau_{\fU}\big\} \subseteq \tau_{\fU_Z}$$
Suppose now that $\cO_Z \in \tau_{\fU_Z}$. Then for each $z\in \cO_Z$ there exists $U_z \in \fU$ such that $B\big(z,U_z\cap \left(Z\times Z\right)\big)\subseteq \cO_Z$. By Proposition \ref{proposition:each_ball_contains_an_open_subset_with_respect_to_topology_induced_by_uniform_structure} there exists $\cO_z \in \tau_{\fU}$ such that $\cO_z\subseteq B(z,U_z)$. Thus 
$$\cO = \bigcup_{z \in \cO_Z}\cO_z$$
is an element of $\tau_{\fU}$ and
$$\cO \subseteq \bigcap_{z\in \cO_Z}B(z,U_z)$$
and hence $Z\cap \cO = \cO_Z$. Therefore, $\cO_Z$ is an open subset in the subspace topology induced on $Z$ by $\tau_{\fU}$. This completes the proof.
\end{proof}

\begin{theorem}\label{theorem:limits_of_uniform_spaces_description}
Let $\cI$ be a small category and let $F:\cI\ra \Unif$ be a functor given by
$$F(i) = (X_i,\fU_i)$$
for $i\in \cI$. Let $\big\{f_i:X \ra X_i \big\}_{i\in \cI}$ be a limiting cone of the composition of $F$ with the functor $\Unif \ra \Set$ which sends each uniform space to its underlying set. Consider the smallest uniform structure $\fU$ on $X$ which makes $\{f_i\}_{i\in \cI}$ into a family of uniform morphisms. Then $(X,\fU)$ together with $\{f_i\}_{i\in \cI}$ is a limiting cone of $F$.
\end{theorem}
\begin{proof}
We may equivalently describe $\fU$ as the smallest uniform structure on $X$ such that 
$$\left(f_i\times f_i\right)^{-1}(U) \in \fU$$
for every $i\in \cI$ and every $U \in \fU_i$. Suppose that $\big\{g_i:(Y,\fV) \ra (X_i,\fU_i)\big\}_{i\in \cI}$ is some cone over $F$. Then there exists a unique map $h:Y \ra X$ such that $h\cdot f_i = g_i$ for every $i\in \cI$. It is easy to verify that
$$\big\{U\in \fU\,\big|\,\left(h\times h\right)^{-1}(U)\mbox{ is an entourage of the diagonal in }\fV\big\}$$
is a uniform structure on $X$. Moreover, it contains $\left(f_i\times f_i\right)^{-1}(U)$ for every $i\in \cI$ and every $U \in \fU_i$. Since $\fU$ is the smallest such uniform structure, we derive that $\fU$ and
$$\big\{U\in \fU\,\big|\,\left(h\times h\right)^{-1}(U)\mbox{ is an entourage of the diagonal in }\fB\big\}$$
coincide and hence $h$ is a morphism of uniform spaces $\left(Y,\fB\right) \ra \left(X,\fU\right)$. This shows that $(X,\fU)$ together with $\big\{f_i:X \ra X_i\big\}_{i\in \cI}$ is a limiting cone of $F$.
\end{proof}

\begin{theorem}\label{theorem:limits_of_uniform_spaces_description}
The functor 
$$\Unif \ni (X,\fU) \mapsto (X,\tau_{\fU}) \in \Top$$
preserves small limits.
\end{theorem}
\begin{proof}
Let $\cI$ be a set and let $\left\{\left(X_i,\fU_i\right)\right\}_{i\in \cI}$ be a family of uniform spaces parametrized by $\cI$. Consider the cartesian product $X = \prod_{i\in \cI}X_i$ and let $pr_i:X\ra X_i$ be the projection for $i\in \cI$. Let $\fU$ be the smallest uniform structure on $X$ which makes $\big\{pr_i:\left(X,\fU\right) \ra \left(X_i,\fU_i\right)\big\}_{i\in \cI}$ into a family of morphisms of uniform spaces. By Theorem \ref{theorem:description_of_uniform_structure_introduced_by_a_family_of_maps} family $\fU$ consists of all surrounding $U$ of the diagonal in $X\times X$ such that there exist $n\in \NN_+$, $i_1,...,i_n\in \cI$ and $U_1 \in \fU_{i_1},...,U_n\in \fU_{i_n}$ satisfying
$$\bigcap_{k=1}^n\left(pr_{i_k}\times pr_{i_k}\right)^{-1}(U_k) \subseteq U$$
Note that we have
$$B\bigg(x, \bigcap_{k=1}^n\left(pr_{i_k}\times pr_{i_k}\right)^{-1}(U_k)\bigg) = \prod_{k=1}^nB\big(pr_{i_k}(x), U_k\big)\times \prod_{i\in \cI\setminus \{i_1,...,i_k\}}X_i$$
for every $x \in X$. By Proposition \ref{proposition:each_ball_contains_an_open_subset_with_respect_to_topology_induced_by_uniform_structure} there exist $\cO_1 \in \tau_{\fU_{i_1}},...,\cO_n \in \tau_{\fU_{i_n}}$ such that 
$$pr_{i_k}(x) \in \cO_k \subseteq \big(pr_{i_k}(x), U_k\big)$$
for each $k$. Thus
$$\prod_{k=1}^n\cO_k \times \prod_{i\in \cI\setminus \{i_1,...,i_k\}}X_i \subseteq B\bigg(x, \bigcap_{k=1}^n\left(pr_{i_k}\times pr_{i_k}\right)^{-1}(U_k)\bigg) \subseteq B(x,U)$$
Therefore, each ball centered in some point $x$ of $X$ and with radius $U$ in $\fU$ contains open neighborhood of $x$ with respect to the product of topologies $\{\tau_{\fU_i}\}_{i\in  \cI}$. This implies that $\tau_{\fU}$ is contained in the product of topologies $\{\tau_{\fU_i}\}_{i\in  \cI}$. On the other hand the fact that $pr_i:(X,\tau_{\fU}) \ra (X_i,\tau_{\fU_i})$ is continuous for every $i\in \cI$ implies that the product of topologies $\{\tau_{\fU_i}\}_{i\in \cI}$ is contained in $\tau_{\fU}$. Thus $\tau_{\fU}$ is the product topology determined by $\{\tau_{\fU_i}\}_{i\in \cI}$. Hence $(X,\tau_{\fU})$ together with $\{pr_i\}_{i\in \cI}$ is a product of topological spaces $\{(X_i,\tau_{\fU_i})\}_{i\in \cI}$. By Theorem \ref{theorem:limits_of_uniform_spaces_description} it follows that
$$\Unif \ni (X,\fU) \mapsto (X,\tau_{\fU}) \in \Top$$
preserves small products. Since every small limit is a combination of small product and kernel pair, it remains to show that the functor above preserves kernel pairs. Suppose that 
\begin{center}
\begin{tikzpicture}
[description/.style={fill=white,inner sep=2pt}]
\matrix (m) [matrix of math nodes, row sep=3em, column sep=3em,text height=1.5ex, text depth=0.25ex] 
{(X,\fW) & \left(Y_1,\fV_1\right)&  \left(Y_1,\fV_2\right)  \\} ;
\path[right hook->,line width=0.8pt,font=\scriptsize]
(m-1-1) edge node[above] {$  $} (m-1-2);
\path[->,line width=0.8pt,font=\scriptsize]
(m-1-2) edge[transform canvas={yshift=0.5ex}] node[above] {$ f_1 $} (m-1-3)
(m-1-2) edge[transform canvas={yshift=-0.5ex}] node[below] {$ f_2 $} (m-1-3);
\end{tikzpicture}
\end{center}
is a kernel pair of $f_1$ and $f_2$ in $\Unif$. Then Theorem \ref{theorem:limits_of_uniform_spaces_description} shows that 
$$Z = \big\{y \in Y_1\,\big|\,f_1(y) = f_2(y)\big\}\subseteq Y_1$$
and $\fW$ is the subspace uniform structure on $Z$ induced by $\fV_1$. Now Proposition \ref{proposition:the_induced_topology_preserves_uniform_subspaces} implies that
\begin{center}
\begin{tikzpicture}
[description/.style={fill=white,inner sep=2pt}]
\matrix (m) [matrix of math nodes, row sep=3em, column sep=3em,text height=1.5ex, text depth=0.25ex] 
{(z,\tau_{\fW}) & \left(Y_1,\tau_{\fV_1}\right)&  \left(Y_1,\tau_{\fV_2}\right)  \\} ;
\path[right hook->,line width=0.8pt,font=\scriptsize]
(m-1-1) edge node[above] {$  $} (m-1-2);
\path[->,line width=0.8pt,font=\scriptsize]
(m-1-2) edge[transform canvas={yshift=0.5ex}] node[above] {$ f_1 $} (m-1-3)
(m-1-2) edge[transform canvas={yshift=-0.5ex}] node[below] {$ f_2 $} (m-1-3);
\end{tikzpicture}
\end{center}
is a kernel pair in the category $\Top$. Therefore, the functor
$$\Unif \ni (X,\fU) \mapsto (X,\tau_{\fU}) \in \Top$$
preserves kernel pairs. The proof is complete.
\end{proof}

\section{Pseudometrizable uniform spaces and their products}
\noindent
In this section we use Theorem \ref{theorem:Weils_theorem_on_pseudometrics} to prove certain structure theorems concerning uniform spaces.

\begin{definition}
A uniform space $(X,\fU)$ is \textit{pseudometrizable} if there exists a pseudometric $\rho$ on $X$ such that the uniform structure
$$\bigg\{U \in \fD_{X}\,\bigg|\,\mbox{ there exists }\epsilon>0\mbox{ such that for all }x,y\in X\mbox{ if }\rho(x,y)\leq \epsilon\mbox{ then }(x,y)\in U\bigg\}$$
coincides with $\fU$.
\end{definition}

\begin{theorem}\label{theorem:characterization_of_pseudometrizable_uniform_spaces}
Let $(X,\fU)$ be a uniform space. The following assertions are equivalent.
\begin{enumerate}[label=\emph{\textbf{(\roman*)}}, leftmargin=*]
\item $(X,\fU)$ is a pseudometrizable uniform space.
\item There exists a sequence $\{U_n\}_{n\in \NN}$ of elements in $\fU$ such that the family
$$\bigg\{U\in \fD_{X}\,\bigg|\exists_{n\in \NN}\,U_n \subseteq U\bigg\}$$
coincides with $\fU$.
\end{enumerate}
\end{theorem}
\begin{proof}
For $\textbf{(i)}\Rightarrow \textbf{(ii)}$ observe that if $\rho$ is a pseudometric on $X$ such that 
$$\bigg\{U \in \fD_{X}\,\bigg|\,\mbox{ there exists }\epsilon>0\mbox{ such that for all }x,y\in X\mbox{ if }\rho(x,y)\leq \epsilon\mbox{ then }(x,y)\in U\bigg\}$$
coincides with $\fU$, then the sequence $\{U_n\}_{n\in \NN}$ given by formula
$$U_n = \bigg\{(x,y)\in X\times X\,\bigg|\,\rho(x,y) \leq \frac{1}{2^n}\bigg\}$$
satisfies \textbf{(ii)}.\\
Suppose now that \textbf{(ii)} holds. We define a sequence $\{V_n\}_{n\in \NN}$ of elements in $\fU$ by recursion. We set $V_0 = U_0$ and if $V_0,...,V_n$ are defined for some $n\in \NN$, then we pick an element $W$ of $\fU$ such that 
$$W\cdot W\cdot W \subseteq V_n$$
and define $V_{n+1} = W\cap U_{n+1}$. Note that $\{V_n\}_{n\in \NN}$ satisfies
$$V_{n+1}\cdot V_{n+1}\cdot V_{n+1} \subseteq V_n$$
for each $n\in \NN$. Moreover, we have
$$\fU = \bigg\{U\in \fD_{X}\,\bigg|\exists_{n\in \NN}\,V_n \subseteq U\bigg\}$$
By Theorem \ref{theorem:Weils_theorem_on_pseudometrics} there exists a pseudometric $\rho$ on $X$ such that
$$\bigg\{(x,y)\in X\times X\,\bigg|\,\rho(x,y)<\frac{1}{2^n}\bigg\} \subseteq V_n \subseteq \bigg\{(x,y)\in X\times X\,\bigg|\,\rho(x,y) \leq \frac{1}{2^n}\bigg\}$$
for every $n\in \NN$. This implies that
$$\bigg\{U \in \fD_{X}\,\bigg|\,\mbox{ there exists }\epsilon>0\mbox{ such that for all }x,y\in X\mbox{ if }\rho(x,y)\leq \epsilon\mbox{ then }(x,y)\in U\bigg\}$$
coincides with $\fU$. Hence $\textbf{(ii)}\Rightarrow \textbf{(i)}$.
\end{proof}

\begin{corollary}\label{corollary:uniform_space_is_subspace_of_product_of_pseudometrizable_spaces}
Every uniform space is a uniform subspace of a product of pseudometrizable uniform spaces.
\end{corollary}

































\end{document}