\input ../pree.tex

\begin{document}

\title{Uniform spaces}
\date{}
\maketitle

\section{Introduction}
\noindent
These notes are devoted to general theory of uniform spaces, which are mathematical objects rigorously encapsulating the intuitive notion of uniformity. In the first section we introduce uniform spaces and their category $\Unif$. We describe uniform structures induced by the family of maps, small limits in $\Unif$ and introduce embeddings of uniform spaces. Next we construct the canonical functor $\Unif\ra \Top$ and show that it preserves small limits and embeddings. In the forth section we define complete uniform spaces and prove results concerning their properties with respect to products, embeddings and extensions of uniform morphisms. We also devote a separate section to study completion of a uniform space by means of minimal Cauchy filters. 
The sixth section is devoted to uniform Urysohn's lemma, which implies that the object-image of the canonical functor $\Unif \ra \Top$ consists of completely regular spaces.

\section{Uniform structures and Uniform spaces}
\noindent
We start by recalling some elementary notions.

\begin{definition}
Let $X$ be a set. The set
$$\Delta_X = \big\{(x,x) \in X\times X\,\big|\,x\in X\big\}$$
is \textit{the diagonal of $X\times X$}.
\end{definition}

\begin{definition}
Let $X,Y,Z$ be sets and let $U \subseteq X\times Y$ and $W \subseteq Y\times Z$ be relations. Consider
$$W\cdot U = \big\{(x,z) \in X\times Z\,\big|\,(x,y) \in U\mbox{ and }(y,z)\in W\mbox{ for some }y\in Y\big\}$$
Then the relation $W \cdot U$ is \textit{the composition of $U$ and $W$}.
\end{definition}
\noindent
The following notion is the main object of our study in these notes.

\begin{definition}
Let $X$ be a set. Suppose that $\fU$ is a collection of reflexive and symmetric relations on $X$ which satisfies the following two assertions.
\begin{enumerate}[label=\textbf{(\arabic*)}, leftmargin=*]
\item If $U \in \fU$ and $W$ is a reflexive and symmetric relation on $X$ such that $U \subseteq W$, then $W\in \fU$.
\item If $U,W \in \fU$, then $U\cap W \in \fU$. 
\item If $U \in \fU$, then there exists $W \in \fU$ such that $W\cdot W \subseteq U$.
\end{enumerate}
Then $\fU$ is \textit{a uniform structure on $X$}.
\end{definition}

\begin{example}\label{example:discrete_uniform_structure}
Let $X$ be a set. Then the family $\fD_X$ of all reflexive and symmetric relations on $X$ is a uniform structure on $X$. It is called \textit{the discrete uniform structure on $X$}.
\end{example}

\begin{definition}
A pair $(X,\fU)$ consisting of a set $X$ and a uniform structure $\fU$ on $X$ is \textit{a uniform space}.
\end{definition}

\begin{definition}
Let $(X,\fU)$ be a uniform space. Suppose that
$$\Delta_X = \bigcap_{U\in \fU}U$$
Then $(X,\fU)$ is \textit{a Hausdorff uniform space}.
\end{definition}

\begin{definition}
Let $(X,\fU),(Y,\fY)$ be uniform spaces and let $f:X\ra Y$ be a map. Suppose that $\left(f\times f\right)^{-1}(V) \in \fU$ for every $V\in \fY$. Then $f$ is \textit{a morphism of uniform spaces}. 
\end{definition}

\begin{remark}\label{remark:category_of_uniform_spaces}
We denote by $\Unif$ the category which consists of uniform spaces and uniform morphisms with respect to the usual composition of maps. 
\end{remark}

\begin{proposition}\label{proposition:description_of_uniform_structure_introduced_by_a_family_of_maps}
Let $X$ be a set and let $\{(X_i,\fU_i)\}_{i\in I}$ be a family of uniform spaces. Consider a family $\big\{f_i:X\ra X_i\big\}_{i\in I}$ of maps. Then the following assertions hold.
\begin{enumerate}[label=\emph{\textbf{(\arabic*)}}, leftmargin=*]
\item There exists the smallest (with respect to inclusion) uniform structure $\fU$ on $X$ such that $f_i:\left(X,\fU\right) \ra \left(X_i,\fU_i\right)$ is a uniform morphism for every $i\in I$. 
\item Let $U$ be a reflexive and symmetric relation on $X$. Then $U \in \fU$ if and only if there exist $n\in \NN_+$, $i_1,...,i_n\in I$ and $W_1 \in \fU_{i_1},...,W_n\in \fU_{i_n}$ such that
$$\bigcap_{k=1}^n\left(f_{i_k}\times f_{i_k}\right)^{-1}(W_k) \subseteq U$$
\end{enumerate}
\end{proposition}
\begin{proof}
Consider the family $\fU$ of all $U\in \fD_X$ such that there exist $n\in \NN_+$, $i_1,...,i_n\in I$ and $W_1 \in \fU_{i_1},...,W_n\in \fU_{i_n}$ satisfying
$$\bigcap_{k=1}^n\left(f_{i_k}\times f_{i_k}\right)^{-1}(W_k) \subseteq U$$
It is easy to verify (we left it for the reader) that $\fU$ is a uniform structure on $X$. By definition of $\fU$ map $f_i$ is a uniform morphism $\left(X,\fU\right) \ra \left(X_i,\fU_i\right)$ for every $i \in I$. Moreover, if $\fF$ is a uniform structure on $X$ which makes $f_i$ into a uniform morphism $\left(X,\fF\right) \ra \left(X_i,\fU_i\right)$ for every $i \in I$, then clearly $\fU \subseteq \fF$.
\end{proof}

\begin{definition}
Let $X$ be a set, let $\{(X_i,\fU_i)\}_{i\in I}$ be a family of uniform spaces and let $\big\{f_i:X\ra X_i\big\}_{i\in I}$ be a family of maps. Then the smallest (with respect to inclusion) uniform structure $\fU$ on $X$ such that $f_i:\left(X,\fU\right) \ra \left(X_i,\fU_i\right)$ is a uniform morphism for every $i\in I$ is called \textit{the uniform structure induced by families $\{f_i\}_{i\in I}$ and $\{(X_i,\fU_i)\}_{i\in I}$}.
\end{definition}
\noindent
Now we describe small limits in $\Unif$.

\begin{theorem}\label{theorem:limits_of_uniform_spaces_description}
Let $\cI$ be a small category and let $F:\cI\ra \Unif$ be a functor given by $F(i) = (X_i,\fU_i)$ for $i\in \cI$. Let $\big\{f_i:X \ra X_i \big\}_{i\in \cI}$ be a limiting cone of the composition of $F$ with the functor $\Unif \ra \Set$ which sends each uniform space to its underlying set. Consider the uniform structure $\fU$ induced by $\{f_i\}_{i \in \cI}$ on $X$. Then $(X,\fU)$ together with family $\{f_i\}_{i\in \cI}$ is a limiting cone of $F$.
\end{theorem}
\begin{proof}
We may equivalently describe $\fU$ as the smallest uniform structure on $X$ such that 
$$\left(f_i\times f_i\right)^{-1}(W) \in \fU$$
for every $i\in \cI$ and every $W \in \fU_i$. Suppose that $\big\{g_i:(Z,\fO) \ra (X_i,\fU_i)\big\}_{i\in \cI}$ is some cone over $F$. Then there exists a unique map $g:Z \ra X$ such that $f_i \cdot g = g_i$ for every $i\in \cI$. It is easy to verify that
$$\big\{U\in \fU\,\big|\,\left(g\times g\right)^{-1}(U)\in \fO\big\}$$
is a uniform structure on $X$. Moreover, it contains $\left(f_i\times f_i\right)^{-1}(W)$ for every $i\in \cI$ and every $W \in \fU_i$. Since $\fU$ is the smallest such uniform structure, we derive that $\fU$ and
$$\big\{U \in \fU\,\big|\,\left(g\times g\right)^{-1}(U) \in \fO\big\}$$
coincide and hence $g$ is a morphism of uniform spaces $\left(Z,\fO\right) \ra \left(X,\fU\right)$. This shows that $(X,\fU)$ together with $\big\{f_i:(X,\fU) \ra (X_i,\fU_i)\big\}_{i\in \cI}$ is a limiting cone of $F$.
\end{proof}

\begin{definition}
Let $j:\left(Z,\fO\right)\ra \left(X,\fU\right)$ be a morphism of uniform spaces. If $j$ is injective and $\fO$ is the uniform structure induced by $j$, then $j$ is \textit{an embedding of uniform spaces}.
\end{definition}

\section{Topology induced by uniform structure}

\begin{definition}
Let $X$ be a set and let $U$ be a symmetric and reflexive relation on $X$. Let $Z$ be a subset of $X$. A set
$$U(Z) = \big\{x\in X\,\big|\,(z,x) \in U\mbox{ for some }z\in Z\big\}$$
is \textit{the $U$-neighborhood of $Z$}. 
\end{definition}

\begin{remark}\label{remark:notation_for_ball}
Let $X$ be a set. For every $x$ in $X$ and every symmetric and reflexive relation $U$ on $X$ we denote by $U(x)$ the set $U(\{x\})$.
\end{remark}

\begin{fact}\label{fact:topology_induced_by_uniform_structure}
Let $X$ be a set and let $\fU$ be a uniform structure on $X$. The family
$$\tau_{\fU} = \big\{\cO\subseteq X\,\big|\,\mbox{ for each }x\in \cO\mbox{ there exists }U\in \fU\mbox{ such that }U(x) \subseteq \cO\big\}$$
is a topology on $X$.
\end{fact}
\begin{proof}
We left the proof for the reader as an exercise.
\end{proof}

\begin{definition}
Let $X$ be a set and let $\fU$ be a uniform structure on $X$. Then the topology $\tau_{\fU}$ is \textit{the topology on $X$ induced by $\fU$}.
\end{definition}

\begin{example}\label{example:natural_uniform_structure_on_real_line}
We define
$$\fE = \big\{U\subseteq \fD_{\RR}\,\big|\,\mbox{ there exists }\epsilon > 0\mbox{ such that if }|\alpha_1 - \alpha_2| < \epsilon\mbox{ then }(\alpha_1,\alpha_2)\in U\big\}$$
Then $\fE$ is a uniform structure on $\RR$ called \textit{the natural uniform structure on $\RR$}. Note that $\tau_{\fE}$ coincides with the natural topology of $\RR$. 
\end{example}

\begin{proposition}\label{proposition:interiors_in_topology_induced_by_uniform_structure}
Let $(X,\fU)$ be a uniform space and let $Z$ be a subset of $X$. Then $x\in Z$ is an interior point of $Z$ with respect to $\tau_{\fU}$ if and only if there exists $U \in \fU$ such that $U(x) \subseteq Z$.
\end{proposition}
\begin{proof}
The "only if" part is clear. For the proof of "if" part consider the set
$$\tilde{Z} = \big\{x\in Z\,\big|\,U(x)\mbox{ is a subset of }Z\mbox{ for some }U\in \fU\big\}$$ 
It suffices to prove that $\tilde{Z}$ is open with respect to $\tau_{\fU}$. Fix $x$ in $\tilde{Z}$. Then there exists $U \in \fU$ such that $U(x)$ is a subset of $Z$. Since $\fU$ is a uniform structure, there exists $W$ in $\fU$ such that $W\cdot W \subseteq U$. Then for every $z \in W(x)$ we have $W(z)\subseteq U(x)\subseteq Z$. Hence if $z \in W(x)$, then $z \in \tilde{Z}$ and thus $\tilde{Z}$ is open with respect to $\tau_{\fU}$.
\end{proof}
\noindent
The following consequence of the previous result is very useful.

\begin{corollary}\label{corollary:each_ball_is_an_open_neighborhood_with_respect_to_topology_induced_by_uniformity}
Let $(X,\fU)$ be a uniform space. For every $x \in X$ and $U \in \fU$ set $U(x)$ contains an open neighborhood of $x$ with respect to $\tau_{\fU}$.
\end{corollary}
\begin{proof}
By Proposition \ref{proposition:interiors_in_topology_induced_by_uniform_structure} point $x$ is an interior point of $U(x)$ with respect to $\tau_{\fU}$.
\end{proof}

\begin{fact}\label{fact:uniform_morphism_is_a_continuous_map}
Let $(X,\fU)$ and $(Y,\fY)$ be uniform spaces and let $f:X\ra Y$ be a morphism of uniform spaces. Then $f$ is a continuous map $\left(X,\tau_{\fU}\right)\ra \left(Y,\tau_{\fY}\right)$.
\end{fact}
\begin{proof}
Pick an open subset $\cO$ with respect to the topology induced by $\fY$ on $Y$. Suppose that $f(x) \in \cO$ for some $x$ in $X$. Then there exists $V_x \in \fY$ such that $V_x(f(x))\subseteq \cO$. Note that the image of $\big(\left(f\times f\right)^{-1}(V_x)\big)(x)$ under $f$ is contained in $\cO$. Therefore,
$$f^{-1}(\cO) = \bigcup_{x\in f^{-1}(\cO)}\big(\left(f\times f\right)^{-1}(V_x)\big)(x)$$
is open in the topology induced by $\fU$.
\end{proof}

\begin{remark}\label{remark:functor_of_topology_induced_by_uniform_structure}
According to Fact \ref{fact:uniform_morphism_is_a_continuous_map} topology induced by uniformity determines a functor $\Unif \ra \Top$.
\end{remark}

\begin{proposition}\label{proposition:topology_induced_by_the_uniform_structure_introduced_by_a_family_of_maps}
Let $X$ be a set and let $\{(X_i,\fU_i)\}_{i\in I}$ be a family of uniform spaces. Consider a family $\big\{f_i:X\ra X_i\big\}_{i\in I}$ of maps. If $\fU$ is the uniform structure induced on $X$ by $\{f_i\}_{i\in I}$, then $\tau_{\fU}$ is the topology induced by the family of maps $\{f_i\}_{i\in I}$ with codomains $\{(X_i,\tau_{\fU_i})\}_{i\in I}$.
\end{proposition}
\begin{proof}
Let $\tau$ denote the topology on $X$ induced by the family of maps $\{f_i\}_{i\in I}$ with codomains $\{(X_i,\tau_{\fU_i})\}_{i\in I}$. Our goal is to prove that $\tau_{\fU} = \tau$. Pick an open subset $\cO$ of $\tau_{\fU}$. Fix $x$ in $\cO$. According to definition of $\tau_{\fU}$ and Proposition \ref{proposition:description_of_uniform_structure_introduced_by_a_family_of_maps} there exist $n\in \NN$, $i_1,...,i_n \in I$ and $V_{i_1}\in \fU_{i_1},...,V_{i_n}\in \fU_{i_n}$ such that 
$$\bigg(\bigcap_{k=1}^n\left(f_{i_k}\times f_{i_k}\right)^{-1}(V_{i_k})\bigg)(x) \subseteq \cO$$ 
We have
$$\bigcap_{k=1}^nf_{i_k}^{-1}\bigg(V_{i_k}\big(f_{i_k}(x)\big)\bigg) = \bigcap_{k=1}^n\bigg(\left(f_{i_k}\times f_{i_k}\right)^{-1}(V_{i_k})\bigg)(x) = \bigg(\bigcap_{k=1}^n\left(f_{i_k}\times f_{i_k}\right)^{-1}(V_{i_k})\bigg)(x)$$
By Corollary \ref{corollary:each_ball_is_an_open_neighborhood_with_respect_to_topology_induced_by_uniformity} for each $k$ there exists $\cO_{k}\in \tau_{\fU_{i_k}}$ such that 
$$f_{i_k}(x) \in \cO_{k} \subseteq V_{i_k}\left(f_{i_k}(x)\right)$$
Therefore, we deduce that
$$x \in \bigcap_{k=1}^nf_{i_k}^{-1}(\cO_k) \subseteq \cO$$
and hence $\cO$ is open with respect to $\tau$. This means that $\tau_{\fU}$ is coarser than $\tau$. On the other hand $f_i:\left(X,\tau_{\fU}\right) \ra \left(X_i,\tau_{\fU_i}\right)$ is continuous for every $i\in I$ and thus $\tau_{\fU}$ is stronger than $\tau$. Therefore, we have $\tau_{\fU} = \tau$.
\end{proof}

\begin{corollary}\label{corollary:induced_topology_functor_preserves_limits_of_uniform_spaces}
The functor 
$$\Unif \ni (X,\fU) \mapsto (X,\tau_{\fU}) \in \Top$$
preserves small limits.
\end{corollary}
\begin{proof}
This follows from the combination of Theorem \ref{theorem:limits_of_uniform_spaces_description} and Proposition \ref{proposition:topology_induced_by_the_uniform_structure_introduced_by_a_family_of_maps}
\end{proof}

\begin{corollary}\label{corollary:the_induced_topology_preserves_subspaces}
If $j:(Z,\fO)\hookrightarrow \left(X,\fU\right)$ is an embedding of uniform spaces, then $j:\left(Z,\tau_{\fO}\right)\ra \left(X,\tau_{\fU}\right)$ is an embedding of topological spaces.
\end{corollary}
\begin{proof}
This follows from definition of embeddings of uniform spaces and Proposition \ref{proposition:topology_induced_by_the_uniform_structure_introduced_by_a_family_of_maps}.
\end{proof}

\begin{fact}\label{fact:Hausdorff_uniform_spaces_and_Hausdorff_topological_spaces}
Let $\left(X,\fU\right)$ be a uniform space. Then $\left(X,\fU\right)$ is a Hausdorff uniform space if and only if $\left(X,\tau_{\fU}\right)$ is a Hausdorff topological space.
\end{fact}
\begin{proof}
Left for the reader as an exercise.
\end{proof}

\section{Cauchy filters and complete uniform spaces}
\noindent
In this section we study very important notion of completeness of uniform spaces. For this we use the notion of filter of subsets defined in \cite{Filters_in_topology}.

\begin{definition}
Let $(X,\fU)$ be a uniform space. Suppose that $\cF$ is a proper filter of subsets of $X$. Assume that for every $U \in \fU$ there exists $F \in \cF$ such that $F\times F\subseteq U$. Then $\cF$ is \textit{a Cauchy filter in $\left(X,\fU\right)$}.
\end{definition}
\noindent
First we prove that the image of a Cauchy filter under a morphism of uniform spaces is a Cauchy filter.

\begin{fact}\label{fact:images_of_Cauchy_filters_under_uniform_morphisms_are_Cauchy}
Let $f:\left(X,\fU\right)\ra \left(Y,\fY\right)$ be a morphism of uniform spaces. If $\cF$ is a Cauchy filter in $\left(X,\fU\right)$, then $f(\cF)$ is a Cauchy filter in $\left(Y,\fY\right)$.
\end{fact}
\begin{proof}
Pick $V \in \fY$. Since $\cF$ is a Cauchy filter in $\left(X,\fU\right)$ and $f$ is a morphism of uniform spaces, there exists $F \in \cF$ such that $F\times F \subseteq (f\times f)^{-1}(V)$. Then $f(F)\times f(F)\subseteq V$. Since $f(F) \in f(\cF)$ and $V$ is an arbitrary element of $\fY$, this proves the assertion.
\end{proof}

\begin{definition}
Let $(X,\fU)$ be a uniform space. Suppose that every Cauchy filter in in $\left(X,\fU\right)$ is convergent with respect to $\tau_{\fU}$. Then $\left(X,\fU\right)$ is \textit{a complete uniform space}. 
\end{definition}
\noindent
The following theorem is analogical to famous Tychonoff's theorem for compact topological spaces.

\begin{theorem}\label{theorem:completeness_of_factors_imply_product_completeness}
Let $\left(X_i,\fU_i\right)$ be complete uniform spaces for $i \in I$. Then the product
$$\prod_{i\in I}\left(X_i,\fU_i\right)$$
is a complete uniform space.
\end{theorem}
\begin{proof}
Let $X = \prod_{i\in I}X_i$ and let $\fU$ be the product uniform structure of $\fU_i$ for $i\in I$. For each $i\in I$ we denote by $pr_i:X\ra X_i$ the canonical projection. Suppose that $\left(X_i,\fU_i\right)$ is a complete uniform space for every $i\in I$. Fix a Cauchy filter $\cF$ in $\left(X,\fU\right)$. Then $\cF_i = pr_i(\cF)$ is a Cauchy filter on $\left(X_i,\fU_i\right)$ for every $i\in I$ according to Fact \ref{fact:images_of_Cauchy_filters_under_uniform_morphisms_are_Cauchy}. Since $\left(X_i,\fU_i\right)$ is a complete uniform space and $\cF_i$ is a Cauchy filter, we derive that $\cF_i$ is convergent to some point $x_i \in X_i$ with respect to $\tau_{\fU_i}$ for each $i\in I$. Let $x$ be a point in $X$ such that $pr_i(x) = x_i$ for every $i \in I$. Then $\cF$ is convergent to $x$ with respect to the product of topologies $\tau_{\fU_i}$ for $i\in I$. By Corollary \ref{corollary:induced_topology_functor_preserves_limits_of_uniform_spaces} we infer that $\tau_{\fU}$ is the product of $\tau_{\fU_i}$ for $i\in I$. Hence $\cF$ is convergent to $x$ with respect to $\tau_{\fU}$. Thus $(X,\fU)$ is a complete uniform space.
\end{proof}

\begin{theorem}\label{theorem:completeness_of_product_and_nonemptiness_of_factors_imply_their_completeness}
Let $\left(X_i,\fU_i\right)$ be nonempty uniform spaces for $i \in I$. If
$$\prod_{i\in I}\left(X_i,\fU_i\right)$$
is a complete uniform space, then $\left(X_i,\fU_i\right)_{i\in I}$ is complete for every $i \in I$.
\end{theorem}
\begin{proof}
Denote $X = \prod_{i\in I}X_i$ and let $\fU$ be the product uniform structure induced by $\fU_i$ for all $i\in I$. For each $i\in I$ we denote by $pr_i:X\ra X_i$ the canonical projection. Assume that $(X,\fU)$ is a complete uniform space. Fix $i_0 \in I$. Suppose that $\cF$ is a Cauchy filter in $(X_{i_0},\fU_{i_0})$. Since $X_i\neq \emptyset$ for every $i\in I$, we may pick $z \in \prod_{i\neq i_0}X_i$. Consider a filter 
$$\tilde{\cF} = \big\{\tilde{F}\subseteq X\,\big|\,F\times \{z\} \subseteq \tilde{F}\mbox{ for some }F\in \cF\big\}$$
Then $\tilde{\cF}$ is a Cauchy filter in $(X,\fU)$ and $\cF = pr_{i_0}\left(\tilde{\cF}\right)$. Since $\tilde{\cF}$ is a Cauchy filter in a complete uniform space $(X,\fU)$, it is convergent with respect to $\tau_{\fU}$ to some $x \in X$. Since $\tau_{\fU}$ is a product of topologies $\tau_{\fU_i}$ for $i \in I$ according to Corollary \ref{corollary:induced_topology_functor_preserves_limits_of_uniform_spaces}, we derive that $pr_{i_0}\left(\tilde{\cF}\right)$ is convergent to $pr_{i_0}(x)$ with respect to $\tau_{\fU_{i_0}}$. Finally according to $\cF = pr_{i_0}\left(\tilde{\cF}\right)$ we derive that $\cF$ is convergent to $pr_{i_0}(x)$. This shows that $(X_{i_0},\fU_{i_0})$ is a complete uniform space. Since $i_0 \in I$ is arbitrary, we derive that $\left(X_i,\fU_i\right)$ is a complete uniform space for every $i \in I$.
\end{proof}
\noindent
Now we study completeness under embeddings of uniform spaces.

\begin{theorem}\label{theorem:closed_embeddings_into_complete_space_are_complete}
Let $j:\left(Z,\fO \right) \hookrightarrow \left(X,\fU\right)$ be an embedding of uniform spaces. If $\left(X,\fU\right)$ is a complete uniform space and $j(Z)$ is closed with respect to $\tau_{\fU}$, then $\left(Z,\fO\right)$ is a complete uniform space.
\end{theorem}
\begin{proof}
Pick a Cauchy filter $\cF$ in $\left(Z,\fO\right)$. Then according to Fact \ref{fact:images_of_Cauchy_filters_under_uniform_morphisms_are_Cauchy} the filter $j(\cF)$ is a Cauchy filter in $\left(X,\fU\right)$. Thus $j(\cF)$ converges to some point $x\in X$ with respect to $\tau_{\fU}$. Since $j(Z)$ is closed in $\tau_{\fU}$, we infer that $x \in j(Z)$. Thus $x = j(z)$ for some $z\in Z$. According to Corollary \ref{corollary:the_induced_topology_preserves_subspaces} the map $j:\left(Z,\tau_{\fO}\right) \hookrightarrow \left(X,\tau_{\fU}\right)$ is a topological embedding. Hence every open neighborhood of $z$ in $\left(Z,\tau_{\fO}\right)$ is of the form $j^{-1}(\cO)$ for an open neighborhood $\cO$ of $j(z)$ in $\left(X,\tau_{\fU}\right)$. Since $j(\cF)$ is convergent to $j(z)$ in $\tau_{\fU}$, we have $\cO \in j(\cF)$ and hence $j^{-1}(\cO) \in \cF$. This proves that $\cF$ is convergent to $z$ in $\tau_{\fO}$.
\end{proof}

\begin{theorem}\label{theorem:embeddings_of_complete_space_into_Hausdorff_space_are_closed}
Let $j:\left(Z,\fO \right) \hookrightarrow \left(X,\fU\right)$ be an embedding of uniform spaces. If $\left(X,\fU\right)$ is a Hausdorff uniform space and $\left(Z,\fO\right)$ is a complete uniform space, then $j(Z)$ is closed with respect to $\tau_{\fU}$.
\end{theorem}
\begin{proof}
Fix a point $x$ of $X$ in the closure of $j(Z)$ with respect to $\tau_{\fU}$. Consider the filter
$$\cF = \big\{F\subseteq Z\,\big|\,j^{-1}\left(U(x)\right)\subseteq F\mbox{ for some }U \in \fU\big\}$$
According to Corollary \ref{corollary:each_ball_is_an_open_neighborhood_with_respect_to_topology_induced_by_uniformity} and the fact that $x$ is in the closure of $j(Z)$ with respect to $\tau_{\fU}$, we derive that $\cF$ is a proper filter of subsets of $Z$. Since $j$ is an embedding of uniform spaces, every element of $\fO$ is of the form $\left(j\times j\right)^{-1}(U)$ for some $U \in \fU$. Fix some $U \in \fU$ and consider $W\in \fU$ such that $W\cdot W\subseteq U$. Note that
$$j^{-1}\left(U(x)\right)\times j^{-1}\left(U(x)\right) \subseteq \left(j\times j\right)^{-1}\big(W\cdot W\big) \subseteq \left(j\times j\right)^{-1}\big(U\big)$$
This shows that $\cF$ is a Cauchy filter in $\left(Z,\fO\right)$. Since $(Z,\fO)$ is complete, filter $\cF$ is convergent to some $z$ in $Z$ with respect to $\tau_{\fO}$. Hence $j(\cF)$ is convergent to $j(z)$ with respect to $\tau_{\fU}$. Clearly $U(x) \in j(\cF)$ for every $U \in \fU$. This implies that $j(\cF)$ is convergent to $x$ with respect to $\tau_{\fU}$. Since $\left(X,\fU\right)$ is Hausdorff, Fact \ref{fact:Hausdorff_uniform_spaces_and_Hausdorff_topological_spaces} implies that $(X,\tau_{\fU})$ is Hausdorff and hence we deduce that $x = j(z)$. This completes the proof that $j(Z)$ is closed with respect to $\tau_{\fU}$.
\end{proof}
\noindent
Finally we discuss extensions of uniform morphisms defined on dense uniform subspaces and with values in complete uniform spaces.

\begin{theorem}\label{theorem:extensions_of_uniform_morphisms_to_complete_spaces}
Let $j:\left(Z,\fO \right) \hookrightarrow \left(X,\fU\right)$ be an embedding of uniform spaces and let $f:\left(Z,\fO\right)\ra \left(Y,\fY\right)$ be a uniform morphism. If $\left(Y,\fY\right)$ is a complete uniform space and $j(Z)$ is dense in $X$ with respect to $\tau_{\fU}$, then there exists a uniform morphism $\tilde{f}:\left(X,\fU\right)\ra \left(Y,\fY\right)$ such that $\tilde{f}\cdot j = f$. Moreover, if $\left(Y,\fY\right)$ is Hausdorff, then $\tilde{f}$ is unique.
\end{theorem}
\begin{proof}
For each point $x$ in $X$ define a filter of subsets of $Z$ by formula
$$\cF_x = \big\{F\subseteq Z\,\big|\,j^{-1}\left(U(x)\right)\subseteq F\mbox{ for some }U \in \fU\big\}$$
Since $j(Z)$ is dense in $X$ with respect to $\tau_{\fU}$ and by Corollary \ref{corollary:each_ball_is_an_open_neighborhood_with_respect_to_topology_induced_by_uniformity}, the filter $\cF_x$ is proper. Since $j$ is an embedding of uniform spaces, every element of $\fO$ is of the form $\left(j\times j\right)^{-1}(U)$ for some $U \in \fU$. Fix some $U \in \fU$ and consider $W\in \fU$ such that $W\cdot W\subseteq U$. Note that
$$j^{-1}\left(U(x)\right)\times j^{-1}\left(U(x)\right) \subseteq \left(j\times j\right)^{-1}\big(W\cdot W\big) \subseteq \left(j\times j\right)^{-1}\big(U\big)$$
This shows that $\cF_x$ is a Cauchy filter in $\left(Z,\fO\right)$. By Fact \ref{fact:images_of_Cauchy_filters_under_uniform_morphisms_are_Cauchy} the filter $f(\cF_x)$ is Cauchy in $\left(Y,\fY\right)$. Since $\left(Y,\fY\right)$ is a complete uniform space, we derive that $f(\cF_x)$ is convergent with respect to $\tau_{\fY}$. If $z\in Z$, then according to Corollary \ref{corollary:each_ball_is_an_open_neighborhood_with_respect_to_topology_induced_by_uniformity} and Corollary \ref{corollary:the_induced_topology_preserves_subspaces} filter $\cF_{j(z)}$ contains all open neighborhoods of $z$ with respect to $\tau_{\fO}$ and hence $\cF_{j(z)}$ is convergent to $z$ with respect to $\tau_{\fO}$. Hence $f(\cF_{j(z)})$ is convergent to $f(z)$ with respect to $\tau_{\fY}$. We define $\tilde{f}:X\ra Y$ in such a way that $f(\cF_x)$ converges to $\tilde{f}(x)$ with respect to $\tau_{\fY}$ for every $x\in X$ and $\tilde{f} \cdot j = f$.\\
We claim that $\tilde{f}$ is a uniform morphism $\left(X,\fU\right)\ra \left(Y,\fY\right)$. For this fix $V \in \fY$ and consider $E \in \fY$ such that $E\cdot E\cdot E\subseteq V$. Then $\left(f\times f\right)^{-1}(E) \in \fO$. Since $j$ is an embedding of uniform spaces, there exists $U \in \fU$ such that 
$$\left(j\times j\right)^{-1}\left(U\cdot U \cdot U\right) \subseteq \left(f\times f\right)^{-1}(E)$$
Pick now $x_1,x_2 \in X$ such that $(x_1,x_2) \in U$. Denote $j^{-1}\big(U(x_1)\cup U(x_2)\big)$ by $F$. First note that $F$ is an element of both $\cF_{x_1}$ and $\cF_{x_2}$. Filter $f(\cF_i)$ is convergent to $\tilde{f}(x_i)$ in $\tau_{\fY}$ for $i=1,2$. By Corollary \ref{corollary:each_ball_is_an_open_neighborhood_with_respect_to_topology_induced_by_uniformity} we deduce that $f(F)$ has nonempty intersection with $E(\tilde{f}(x_i))$ for $i=1,2$. On the other hand
$$F\times F \subseteq \left(j\times j\right)^{-1}\left(U\cdot U \cdot U\right)\subseteq \left(f\times f\right)^{-1}(E)$$
and hence $f(F)\times f(F) \subseteq E$. Thus
$$\big(\tilde{f}(x_1),\tilde{f}(x_2)\big) \in E\cdot E\cdot E\subseteq V$$
and this implies that for every $x_1,x_2\in X$ if $(x_1,x_2) \in U$, then $\big(\tilde{f}(x_1),\tilde{f}(x_2)\big) \in V$. Thus $\tilde{f}$ is a morphism of uniform spaces.\\
Suppose that $(Y,\fY)$ is Hausdorff. Fact \ref{fact:Hausdorff_uniform_spaces_and_Hausdorff_topological_spaces} implies that $(Y,\tau_{\fY})$ is a Hausdorff topological space. Moreover, $\tilde{f}$ is a continuous map $\left(X,\tau_{\fU}\right) \ra \left(Y,\tau_{\fY}\right)$ and by Corollary \ref{corollary:the_induced_topology_preserves_subspaces} we have dense embedding of topological spaces $j:\left(Z,\tau_{\fO}\right)\hookrightarrow \left(X,\tau_{\fU}\right)$. Since each continuous map with codomain in Hausdorff topological space is uniquely determined by its restriction to the dense subspace of its domain, we derive that the map $\tilde{f}$ such that $\tilde{f} \cdot j = f$ is unique.
\end{proof}

\section{Completion of uniform spaces}

\begin{definition}
Let $(X,\fU)$ be a uniform space and let $\cF_1,\cF_2$ be Cauchy filters in $(X,\fU)$. If for every $U\in \fU$ there exist $F_1 \in \cF_1$ and $F_2 \in \cF_2$ such that $F_1\times F_2 \subseteq U$, then \textit{$\cF_1$ is equivalent to $\cF_2$ in $(X,\fU)$}.
\end{definition}

\begin{fact}\label{fact:equivalence_of_Cauchy_filters_is_equivalence_relation}
Let $(X,\fU)$ be a uniform space. Then equivalence of Cauchy filters in $(X,\fU)$ is reflexive, symmetric and transitive relation.
\end{fact}
\begin{proof}
Left for the reader as an exercise.
\end{proof}

\begin{fact}\label{fact:inclusion_of_Cauchy_filters}
Let $(X,\fU)$ be a uniform space. If $\cF_1$ and $\cF_2$ are proper filters of subsets in $X$ such that $\cF_1$ is a Cauchy filter in $(X,\fU)$ and $\cF_1 \subseteq \cF_2$, then $\cF_2$ is a Cauchy filter in $(X,\fU)$ equivalent to $\cF_2$.
\end{fact}
\begin{proof}
Left for the reader as an exercise.
\end{proof}

\begin{definition}
Let $(X,\fU)$ be a uniform space. \textit{A minimal Cauchy filter in $(X,\fU)$} is a minimal element of the set of all Cauchy filters in $(X,\fU)$ ordered by inclusion.  
\end{definition}
\noindent
The following result describes main properties of minimal Cauchy filters.

\begin{theorem}\label{theorem:minimal_Cauchy_filters}
Let $(X,\fU)$ be a uniform space and let $\bd{K}$ be an equivalence class of Cauchy filters in $(X,\fU)$. Then the following assertions hold.
\begin{enumerate}[label=\emph{\textbf{(\arabic*)}}, leftmargin=*]
\item $$\bigcap \bd{K} = \bigg\{Z\subseteq X\,\bigg|\,U(F)\subseteq Z\mbox{ for some }\cF \in \bd{K},\,F \in \cF\mbox{ and }U \in \fU\bigg\}$$
\item $\bigcap \bd{K}$ is an element of $\bd{K}$.
\item $\bigcap \bd{K}$ is a minimal Cauchy filter in $(X,\fU)$.
\end{enumerate}
\end{theorem}
\begin{proof}
For fixed filter $\cF \in \bd{K}$ we denote by $\fU(\cF)$ the family
$$\bigg\{Z\subseteq X\,\bigg|\,U(F)\subseteq Z\mbox{ for some }F \in \cF\mbox{ and }U \in \fU\bigg\}$$
Note that $\fU(\cF)$ is a Cauchy filter in $(X,\fU)$ such that $\fU(\cF)\subseteq \cF$. Thus by Fact \ref{fact:inclusion_of_Cauchy_filters} it follows that $\fU(\cF) \in \bd{K}$ for every $\cF \in \bd{K}$. Thus we have
$$\bigcap \bd{K} = \bigcap_{\cF \in \bd{K}}\fU(\cF)$$
Fix now $\cF_1,\cF_2\in \bd{K}$, $F\in \cF_1$ and $U \in \fU$. Pick $W \in \fU$ such that $W\cdot W\subseteq U$. Then there exist $F_1 \in \cF_1$ and $F_2\in \cF_2$ such that $F_1\times F_2 \subseteq W$. In particular, we have $\left(F\cap F_1\right)\times F_2 \subseteq W$. Hence 
$$W(F_2)\subseteq U\left(F\cap F_1\right) \subseteq U(F)$$
It follows that $U(F) \in \fU(\cF_2)$. This proves that $\fU(\cF_1)\subseteq \fU(\cF_2)$. By symmetry we deduce that $\fU(\cF_1) = \fU(\cF_2)$ and this holds for each pair $\cF_1,\cF_2\in \bd{K}$. Combining this with the fact that $\bigcap \bd{K}$ is the intersection of all $\fU(\cF)$ for $\cF \in \bd{K}$, we deduce that
$$\bigcap \bd{K} = \fU(\cF)$$
for every $\cF \in \bd{K}$. This completes the proof of \textbf{(1)} and \textbf{(2)}. The assertion \textbf{(3)} follows from Fact \ref{fact:inclusion_of_Cauchy_filters}.
\end{proof}

\begin{theorem}\label{theorem:existence_of_completion}
Let $\left(X,\fU\right)$ be a uniform space. Then there exists a complete Hausdorff uniform space $\left(\hat{X},\hat{\fU}\right)$ and a morphism $j:\left(X,\fU\right)\ra \left(\hat{X},\hat{\fU}\right)$ of uniform spaces such that the following assertions hold.
\begin{enumerate}[label=\emph{\textbf{(\arabic*)}}, leftmargin=*]
\item Equivalence classes of relation
$$\bigcap_{U\in \fU}U$$
coincide with fibers of $j$.
\item $j(X)$ is dense in $\hat{X}$ with respect to topology induced by $\hat{\fU}$.
\item $\fU$ is induced by $j$ and $\left(\hat{X},\hat{\fU}\right)$.
\end{enumerate}
\end{theorem}
\begin{proof}
Let $\hat{X}$ be the set of all minimal Cauchy filters in $\left(X,\fU\right)$. For $U \in \fU$ we set
$$\hat{U} = \bigg\{\left(\bd{x}_1,\bd{x}_2\right)\in \hat{X}\times \hat{X}\,\bigg|\,\mbox{ there exist }F_1\in \bd{x}_1,F_2\in \bd{x}_2\mbox{ such that }F_1\times F_2\subseteq U\bigg\}$$
Clearly $\hat{U} \in \fD_{\hat{X}}$. We define
$$\hat{\fU} = \big\{\bd{U} \in \fD_{\hat{X}}\,\big|\,\hat{U}\subseteq \bd{U}\mbox{ for some }U\in \fU\big\}$$
For each $x \in X$ consider the minimal Cauchy filter $j(x)$ in $\left(X,\fU\right)$ given by formula
$$\bigg\{F\subseteq X\,\big|\,U(x)\subseteq F\mbox{ for some }U\in \fU\bigg\}$$
This gives rise to a map $j:X\ra \hat{X}$. We are going to prove the theorem in a series of claims.\\
We claim that $\left(\hat{X},\hat{\fU}\right)$ is a uniform space. Note that 
$$\hat{U}_1\cdot \hat{U}_2 \subseteq \widehat{U_1\cdot U}_2,\,\hat{U}_1\cap \hat{U}_2 = \widehat{U_1\cap U}_2$$
for every $U_1,U_2\in \fU$ and if in addition $U_1 \subseteq U_2$ for $U_1,U_2 \in \fU$, then also $\hat{U}_1\subseteq \hat{U}_2$. These assertions imply that $\hat{\fU}$ is a uniform structure on $\hat{X}$ and the claim is proved.\\
Next we claim that $j$ is a morphism of uniform spaces $\left(X,\fU\right) \ra \left(\hat{X},\hat{\fU}\right)$ such that $\fU$ is induced by $j$ and $\left(\hat{X},\hat{\fU}\right)$. This follows from the fact that we have inclusions
$$W \subseteq \left(j\times j\right)^{-1}(\hat{U}) \subseteq U$$  
which hold for every $W,U \in \fU$ such that $W\cdot W \subseteq U$.\\
Now we show that if $\bd{x}$ is a minimal Cauchy filter in $\left(X,\fU\right)$, then $\hat{U}(\bd{x})\in j(\bd{x})$ for every $U\in \fU$. For this we again fix $U,W \in \fU$ such that $W\cdot W \subseteq U$. Since there exists $F \in \bd{x}$ such that $F\times F\subseteq W$, we derive that for every $x \in F$ we have $W(x) \times F\subseteq U$. Hence for every $x\in F$ we have $j(x) \in \hat{U}$. Thus $F\subseteq  j^{-1}\left(\hat{U}(\bd{x})\right)$. Therefore, $j^{-1}\left(\hat{U}(\bd{x})\right) \in \bd{x}$ for every $U \in \fU$. Hence $\hat{U}(\bd{x}) \subseteq j\left(\bd{x}\right)$ for each $U\in \fU$ and each $\bd{x} \in \hat{X}$. This proves the claim.\\
The claim above combined with Corollary \ref{corollary:each_ball_is_an_open_neighborhood_with_respect_to_topology_induced_by_uniformity} this implies that $j(X)$ is dense in $\hat{X}$ with respect to the topology induced by $\hat{\fU}$.\\
Next we claim that $\left(\hat{X},\hat{\fU}\right)$ is complete. Suppose that $\fF$ is a minimal Cauchy filter in $\left(\hat{X},\hat{\fU}\right)$. By Theorem \ref{theorem:minimal_Cauchy_filters} and Proposition \ref{proposition:interiors_in_topology_induced_by_uniform_structure} sets in $\fF$ have nonempty interiors with respect to the topology induced by $\hat{\fU}$. Hence, according to the fact that $j(X)$ is dense in $\hat{X}$ with respect to $\tau_{\hat{\fU}}$, family
$$j^{-1}(\fF) = \big\{F \subseteq X\,\big|\,j^{-1}(\hat{F})\subseteq F\mbox{ for some }\hat{F} \in \fF\big\}$$
is a proper filter of subsets of $X$. Since $\fU$ is induced by $j$ and $\left(\hat{X},\hat{\fU}\right)$, we derive that $j^{-1}(\fF)$ is a Cauchy filter in $\left(X,\fU\right)$. Theorem \ref{theorem:minimal_Cauchy_filters} shows that there exists a minimal Cauchy filter $\bd{x}$ such that $\bd{x} \subseteq j^{-1}(\fF)$. Then $j(\bd{x})\subseteq j\left(j^{-1}(\fF)\right)$ and hence $\hat{U}(\bd{x}) \in j\left(j^{-1}(\fF)\right)$ for each $U \in \fU$. This together with Corollary \ref{corollary:each_ball_is_an_open_neighborhood_with_respect_to_topology_induced_by_uniformity} shows that $j\left(j^{-1}(\fF)\right)$ converges to $\bd{x}$ with respect to the topology induced by $\hat{\fU}$. Clearly we have $\fF \subseteq j\left(j^{-1}(\fF)\right)$ and Fact \ref{fact:inclusion_of_Cauchy_filters} implies that $\fF$ and $j\left(j^{-1}(\fF)\right)$ are equivalent Cauchy filters in $\left(\hat{X},\hat{U}\right)$. Hence $\fF$ is also convergent to $\bd{x}$ with respect to $\tau_{\hat{\fU}}$. This proves that every minimal Cauchy filter in $(\hat{X},\hat{\fU})$ is convergent with respect to $\tau_{\hat{\fU}}$. According to Theorem \ref{theorem:minimal_Cauchy_filters} it follows that $(\hat{X},\hat{\fU})$ is complete.\\
Now we prove that $\left(\hat{X},\hat{U}\right)$ is Hausdorff. If $\bd{x}_1$ and $\bd{x}_2$ are distinct minimal Cauchy filters in $\left(X,\fU\right)$, then they are not equivalent and hence there exists $U \in \fU$ such that for all $F_1 \in \bd{x}_1$ and $F_2 \in \bd{x}_2$ we have $\left(F_1\times F_2\right)\setminus U \neq \emptyset$. This shows that $(\bd{x}_1,\bd{x}_2) \not \in \hat{U}$ and the claim on Hausdorffness of $\left(\hat{X},\hat{\fU}\right)$ follows.\\
Finally we prove that fibers of $j$ coincide with equivalence classes of the relation given by the intersection of all elements of $\fU$. For this fix $x_1,x_2\in X$. Then $j(x_1) = j(x_2)$ if and only if for every $U\in \fU$ there exists $W_1,W_2\in \fU$ such that 
$$W_1(x_1)\subseteq U(x_2),\,W_2(x_2)\subseteq U(x_1)$$
This last assertion is equivalent to
$$(x_1,x_2) \in \bigcap_{U\in \fU}U$$
The proof of the claim is completed and thus the theorem is proved.
\end{proof}

\begin{corollary}\label{corollary:completion_of_a_Hausdorff_uniform_space}
Let $\left(X,\fU\right)$ be a Hausdorff uniform space. Then there exists a complete Hausdorff uniform space $\left(\hat{X},\hat{\fU}\right)$ and an embedding $j:\left(X,\fU\right)\ra \left(\hat{X},\hat{\fU}\right)$ of uniform spaces such that $j(X)$ is dense in $\hat{X}$ with respect to topology induced by $\hat{\fU}$.
\end{corollary}
\begin{proof}
This is a direct consequence of Theorem \ref{theorem:existence_of_completion}.
\end{proof}

\begin{definition}
Let $(X,\fU)$ be a uniform space, let $\left(\hat{X},\hat{\fU}\right)$ be a complete Hausdorff uniform space and let $j:\left(X,\fU\right) \ra \left(\hat{X},\hat{\fU}\right)$ be a morphism of uniform spaces. Suppose that for every complete Hausdorff uniform space $\left(Y,\fY\right)$ and every morphism $f:\left(X,\fU\right) \ra \left(Y,\fY\right)$ there exists a unique morphism $\hat{f}:\left(\hat{X},\hat{\fU}\right) \ra \left(Y,\fY\right)$ of uniform spaces such that triangle 
\begin{center}
\begin{tikzpicture}
[description/.style={fill=white,inner sep=2pt}]
\matrix (m) [matrix of math nodes, row sep=4em, column sep=5em,text height=1.5ex, text depth=0.25ex] 
{ \left(X,\fU\right) &  \left(Y,\fY\right)  \\
    \left(\hat{X},\hat{\fU}\right) & \\ } ;
\path[->,line width=0.8pt,font=\scriptsize]
(m-1-1) edge node[above] {$ f $} (m-1-2)
(m-1-1) edge node[left] {$ j $} (m-2-1);
\path[densely dotted,->,line width=0.8pt,font=\scriptsize]
(m-2-1) edge node[right = 2pt, below = 2pt] {$ \hat{f} $} (m-1-2);
\end{tikzpicture}
\end{center}
is commutative. Then $\left(\hat{X},\hat{\fU}\right)$ together with $j$ is \textit{the completion of $\left(X,\fU\right)$}.
\end{definition}

\begin{corollary}\label{corollary:universal_property_of_completion}
Let $\left(X,\fU\right)$ be a uniform space. Then its completion exists and if $j:\left(X,\fU\right)\ra \left(\hat{X},\hat{\fU}\right)$ is the completion of $\left(X,\fU\right)$, then $\fU$ is induced by $j$ and $\left(\hat{X},\hat{\fU}\right)$. Moreover, $j(X)$ is dense in $\hat{X}$ with respect to topology induced by $\hat{\fU}$.
\end{corollary}
\begin{proof}
This is simply the consequence of Theorem \ref{theorem:extensions_of_uniform_morphisms_to_complete_spaces} and Theorem \ref{theorem:existence_of_completion}.
\end{proof}

\begin{corollary}\label{corollary:universal_property_of_completion_for_Hausdorff_spaces}
Let $\left(X,\fU\right)$ be a Hausdorff uniform space. If $j:\left(X,\fU\right)\ra \left(\hat{X},\hat{\fU}\right)$ is the completion of $\left(X,\fU\right)$, then $j$ is an embedding of uniform spaces.
\end{corollary}
\begin{proof}
This is simply the consequence of Theorem \ref{theorem:extensions_of_uniform_morphisms_to_complete_spaces} and Corollary \ref{corollary:completion_of_a_Hausdorff_uniform_space}.
\end{proof}

\section{Pseudometric and metric spaces}
\noindent
In this section we introduce uniform category of pseudometric spaces.

\begin{definition}
Let $X$ be a set. Consider a function $\rho:X\times X\ra \RR$ and suppose that the following assertions hold.
\begin{enumerate}[label=\textbf{(\arabic*)}, leftmargin=*]
\item $\rho(x_1,x_2) \geq 0$ for every $x_1,x_2\in X$.
\item $\rho(x,x) = 0$ for every $x \in X$.
\item $\rho(x_1,x_2) = \rho(x_2,x_1)$ for every $x_1,x_2 \in X$.
\item $\rho(x_1,x_3) \leq \rho(x_1,x_2) + \rho(x_2, x_3)$ for every $x_1,x_2,x_3 \in X$.
\end{enumerate}
Then $\rho$ is \textit{a pseudometric on $X$}.
\end{definition}

\begin{definition}
A pair $(X,\rho)$ consisting of a set $X$ and a pseudometric $\rho$ on $X$ is \textit{a pseudometric space}.
\end{definition}

\begin{definition}
Let $(X,\rho)$ be a pseudometric space. We define
$$\fU_{\rho} = \bigg\{U \in \fD_X\,\bigg|\,\exists_{\epsilon>0}\,\forall_{x_1,x_2\in X}\,\bigg(\rho(x_1,x_2) < \epsilon\,\Rightarrow\,(x_1,x_2)\in U\bigg)\bigg\}$$
Then $\fU_{\rho}$ is a uniform structure called \textit{the uniform structure induced by $\rho$ on $X$}.
\end{definition}
\noindent
Next we characterize these maps of pseudometric spaces, which are morphism of corresponding uniform spaces.

\begin{fact}\label{fact:uniformly_continuous_maps_are_precisely_morphisms_of_uniform_spaces}
Let $(X,\rho)$ and $(Y,\upsilon)$ be pseudometric spaces and let $f:X\ra Y$ be a map. Then the following assertions are equivalent.
\begin{enumerate}[label=\emph{\textbf{(\roman*)}}, leftmargin=*]
\item For every $\epsilon > 0$ there exists $\delta > 0$ such that
$$\rho(x_1,x_2) < \delta\,\Rightarrow\,\upsilon\left(f(x_1),f(x_2)\right) < \epsilon$$
for every $x_1,x_2 \in X$.
\item $f$ is a morphism of uniform spaces $\left(X,\fU_{\rho}\right) \ra \left(Y,\fU_{\upsilon}\right)$.
\end{enumerate}
\end{fact}
\begin{proof}
This follows from the definition of uniform structure induced by pseudometric. We left it for the reader.
\end{proof}

\begin{definition}
Let $(X,\rho)$ and $(Y,\upsilon)$ be pseudometric spaces and let $f:X\ra Y$ be a map. Suppose that for every $\epsilon > 0$ there exists $\delta > 0$ such that
$$\rho(x_1,x_2) < \delta\,\Rightarrow\,\upsilon\left(f(x_1),f(x_2)\right) < \epsilon$$
for every $x_1,x_2 \in X$. Then $f$ is \textit{uniformly continuous}.
\end{definition}

\begin{remark}\label{remark:functor_of_uniform_structure_induced_by_pseudometric}
Let $\PsMet$ be the category of pseudometric spaces and uniformly continuous maps. According to Fact \ref{fact:uniformly_continuous_maps_are_precisely_morphisms_of_uniform_spaces} assignment 
$$\PsMet \ni \left(X,\rho\right) \mapsto \left(X,\fU_{\rho}\right) \in \Unif$$
is a full and faithful functor.
\end{remark}

\begin{definition}
Let $X$ be a set and let $\rho$ be a pseudometric on $X$. If $\rho(x_1,x_2) = 0$ imply that $x_1 = x_2$ for all $x_1,x_2\in X$, then $\rho$ is \textit{a metric}. In that case $\left(X,\rho\right)$ is \textit{a metric space}.
\end{definition}

\begin{remark}\label{remark:the_category_of_metric_spaces}
We denote by $\Met$ the category of metric spaces and uniformly continuous maps. Clearly $\Met$ is a full subcategory of $\PsMet$.
\end{remark}

\begin{fact}\label{fact:metric_spaces_induce_Hausdorff_uniform_structures}
Let $(X,\rho)$ be a pseudometric space. Then the following assertions are equivalent.
\begin{enumerate}[label=\emph{\textbf{(\roman*)}}, leftmargin=*]
\item $(X,\rho)$ is a metric space.
\item $\left(X,\fU_{\rho}\right)$ is a Hausdorff uniform space.
\end{enumerate}
\end{fact}
\begin{proof}
Left for the reader as an exercise.
\end{proof}

\begin{definition}
Let $(X,\rho)$ be a pseudometric space. Then $\tau_{\fU_{\rho}}$ is also called \textit{the topology induced by $\rho$}.
\end{definition}

\begin{definition}
Let $(X,\rho)$ be a pseudometric space and let $\{x_n\}_{n\in \NN}$ be a sequence of elements of $X$. If for every $\epsilon > 0$ there exists $N\in \NN$ such that for all $n,m\geq N$ we have
$$\rho(x_n,x_m) \leq \epsilon$$
then $\{x_n\}_{n\in \NN}$ is \textit{a Cauchy sequence in $(X,\rho)$}.
\end{definition}

\begin{definition}
Let $(X,\rho)$ be a pseudometric space. Suppose that every Cauchy sequence in $(X,\rho)$ is convergent with respect to the topology induced by $\rho$. Then $(X,\rho)$ is \textit{a complete pseudometric space}. 
\end{definition}

\begin{theorem}\label{theorem:complete_pseudometric_spaces}
Let $(X,\rho)$ be a pseudometric space. Then the following conditions are equivalent.
\begin{enumerate}[label=\emph{\textbf{(\roman*)}}, leftmargin=*]
\item $(X,\rho)$ is a complete pseudometric space. 
\item $(X,\fU_{\rho})$ is a complete uniform space.
\end{enumerate}
\end{theorem}
\begin{proof}
We define 
$$U_{n} = \bigg\{(x,y)\in X\times X\,\bigg|\,\rho(x,y) < \frac{1}{n + 1}\bigg\}$$
for every $n\in \NN$. Then $\fU_{\rho}$ is the uniform structure consisting of all $U \in \fD_X$ such that $U_n\subseteq U$ for some $n\in \NN$.\\
Now assume that $(X,\rho)$ is a complete pseudometric space. Let $\cF$ be a Cauchy filter in $\left(X,\fU_{\rho}\right)$. We construct a sequence $\{F_n\}_{n\in \NN}$ of elements of $\cF$ by recursion. We set $F_0 = X$. Suppose that $F_0,...,F_n$ are constructed for some $n\in \NN$. We pick $F_{n+1}\in \cF$ such that 
$$F_{n+1}\times F_{n+1} \subseteq U_{n+1}$$
and $F_{n+1} \subseteq F_n$. Note that sequence $\{F_n\}_{n\in \NN}$ is nonincreasing and
$$F_{n}\times F_{n} \subseteq U_n$$
for every $n\in \NN$. Now pick $x_n\in F_n$ for every $n\in \NN$. Then according to properties of $\{F_n\}_{n\in \NN}$ sequence $\{x_n\}_{n\in \NN}$ is a Cauchy sequence in $(X,\rho)$. Since $(X,\rho)$ is complete, $\{x_n\}_{n\in \NN}$ is convergent to some point $x$ with respect to the topology induced by $\rho$. We claim that $\cF$ is convergent to $x$ with respect to the same topology. Note that $F_{n+1} \subseteq U_n(x)$ for every $n\in \NN$. This implies that all open neighborhoods of $x$ with respect to $\tau_{\fU_{\rho}}$ are contained in $\cF$. Thus $\cF$ is convergent to $x$ with respect to $\tau_{\fU_{\rho}}$. This proves that $\textbf{(i)}\Rightarrow \textbf{(ii)}$.\\
Next assume that $(X,\fU_{\rho})$ is a complete uniform space. Pick a Cauchy sequence $\{x_n\}_{n\in \NN}$ in $(X,\rho)$. Define
$$F_n = \big\{x_k\,\big|\,k\geq n\big\}$$
for every $n\in \NN$ and
$$\cF = \big\{F\subseteq X\,\big|\,F_n\subseteq F\mbox{ for some }n\in \NN\big\}$$
Since $(X,\fU_{\rho})$ is complete, the filter $\cF$ is convergent to some point $x$ in $X$ with respect to $\tau_{\fU_{\rho}}$. Therefore, for every open neighborhood $\cO$ of $x$ with respect to $\tau_{\fU_{\rho}}$ there exists $n\in \NN$ such that $F_n \subseteq \cO$. Hence the sequence $\{x_n\}_{n\in \NN}$ is convergent to $x$ with respect to $\tau_{\fU_{\rho}}$ and the implication $\textbf{(ii)}\Rightarrow \textbf{(i)}$ holds.
\end{proof}

\begin{example}\label{example:real_line_is_complete}
Recall uniform space $\left(\RR,\fE\right)$ described in Example \ref{example:natural_uniform_structure_on_real_line}. Note that $\fE$ is a uniform structure induced by the metric
$$\RR\times \RR\ni (\alpha_1,\alpha_2) \mapsto |\alpha_1 - \alpha_2|\in \RR$$
It is well known fact that each Cauchy sequence with respect to this metric is convergent. It follows from Theorem \ref{theorem:complete_pseudometric_spaces} and Fact \ref{fact:metric_spaces_induce_Hausdorff_uniform_structures} that $\left(\RR,\fE\right)$ is a complete Hausdorff uniform space.
\end{example}

\begin{theorem}\label{theorem:completion_of_a_pseudometric_space}
Let $(X,\rho)$ be a pseudometric space. Then there exists a complete pseudometric space $(\hat{X},\hat{\rho})$ and a map $j:X\ra \hat{X}$ such that $j(X)$ is dense in $\hat{X}$ with respect to topology induced by $\hat{\rho}$ and
$$\rho(x_1,x_2) = \hat{\rho}(j(x_1),j(x_2))$$
for every $x_1,x_2\in X$. Moreover, if $(X,\rho)$ is a metric space, then we may assume that $(\hat{X},\hat{\rho})$ is a metric space and $j$ is injective.
\end{theorem}
\begin{proof}
According to Theorem \ref{theorem:existence_of_completion} there exists a complete uniform space $\left(\hat{X},\hat{\fU}\right)$ and a morphism $j:\left(X,\fU_{\rho}\right) \ra \left(\hat{X},\hat{\fU}\right)$ of uniform spaces such that $j$ and $\hat{\fU}$ induce $\fU_{\rho}$ and $j(X)$ is dense in $\hat{X}$ with respect to $\tau_{\hat{\fU}}$. Note that $\rho$ gives rise to a uniform morphism 
$$\left(X,\fU_{\rho}\right)\times \left(X,\fU_{\rho}\right)\ra \left(\RR,\fE\right)$$
Indeed, this is a consequence of the inequality
$$|\rho(x_1,y_1) - \rho(x_2,y_2)|\leq \rho(x_1,x_2) + \rho(y_1,y_2)$$
which holds for every $x_1,x_2,y_1,y_2\in X$. Thus Theorems \ref{theorem:completeness_of_factors_imply_product_completeness} and \ref{theorem:extensions_of_uniform_morphisms_to_complete_spaces} imply that there exists a unique uniform morphism $\hat{\rho}$ making the triangle
\begin{center}
\begin{tikzpicture}
[description/.style={fill=white,inner sep=2pt}]
\matrix (m) [matrix of math nodes, row sep=4em, column sep=5em,text height=1.5ex, text depth=0.25ex] 
{ \left(X,\fU_{\rho}\right) \times \left(X,\fU_{\rho}\right) &  \left(\RR,\fE\right)  \\
    \left(\hat{X},\hat{\fU}\right)\times \left(\hat{X},\hat{\fU}\right) & \\ } ;
\path[->,line width=0.8pt,font=\scriptsize]
(m-1-1) edge node[above] {$ \rho $} (m-1-2)
(m-1-1) edge node[left] {$ j\times j $} (m-2-1);
\path[densely dotted,->,line width=0.8pt,font=\scriptsize]
(m-2-1) edge node[right = 2pt, below = 2pt] {$ \hat{\rho} $} (m-1-2);
\end{tikzpicture}
\end{center}
commutative. We claim that $\hat{\rho}$ is a pseudometric. Corollary \ref{corollary:induced_topology_functor_preserves_limits_of_uniform_spaces} shows that we have commutative triangle of topological spaces
\begin{center}
\begin{tikzpicture}
[description/.style={fill=white,inner sep=2pt}]
\matrix (m) [matrix of math nodes, row sep=4em, column sep=5em,text height=1.5ex, text depth=0.25ex] 
{ \left(X,\tau_{\fU_{\rho}}\right) \times \left(X,\tau_{\fU_{\rho}}\right) &  \RR  \\
    \left(\hat{X},\tau_{\hat{\fU}}\right)\times \left(\hat{X},\tau_{\hat{\fU}}\right) & \\ } ;
\path[->,line width=0.8pt,font=\scriptsize]
(m-1-1) edge node[above] {$ \rho $} (m-1-2)
(m-1-1) edge node[left] {$ j\times j $} (m-2-1);
\path[densely dotted,->,line width=0.8pt,font=\scriptsize]
(m-2-1) edge node[right = 2pt, below = 2pt] {$ \hat{\rho} $} (m-1-2);
\end{tikzpicture}
\end{center}
where $\RR$ is equipped with natural topology. Thus sets
$$\big\{(\bd{x}_1,\bd{x}_2)\in \hat{X}\times \hat{X}\,\big|\,\hat{\rho}\left(\bd{x}_1,\bd{x}_2\right) = \hat{\rho}\left(\bd{x}_2,\bd{x}_1\right)\big\}$$
and
$$\big\{(\bd{x}_1,\bd{x}_2,\bd{x}_3)\in \hat{X}\times \hat{X}\times \hat{X}\,\big|\,\hat{\rho}\left(\bd{x_1},\bd{x}_3\right) \leq  \hat{\rho}\left(\bd{x}_1,\bd{x}_2\right) + \hat{\rho}\left(\bd{x}_2,\bd{x}_3\right)\big\}$$
are closed in the corresponding products of copies of $\left(\hat{X},\tau_{\hat{\fU}}\right)$ and they contain $j(X)\times j(X)$ and $j(X)\times j(X)\times j(X)$, respectively. Note that $j(X)\times j(X)$ and $j(X)\times j(X)\times j(X)$ are dense in the corresponding products of copies of $\left(\hat{X},\tau_{\hat{\fU}}\right)$. Therefore, we have
$$\hat{X}\times \hat{X} = \big\{(\bd{x}_1,\bd{x}_2)\in \hat{X}\times \hat{X}\,\big|\,\hat{\rho}\left(\bd{x}_1,\bd{x}_2\right) = \hat{\rho}\left(\bd{x}_2,\bd{x}_1\right)\big\}$$
and
$$\hat{X}\times \hat{X}\times \hat{X} = \big\{(\bd{x}_1,\bd{x}_2,\bd{x}_3)\in \hat{X}\times \hat{X}\times \hat{X}\,\big|\,\hat{\rho}\left(\bd{x}_1,\bd{x}_3\right) \leq  \hat{\rho}\left(\bd{x}_1,\bd{x}_2\right) + \hat{\rho}\left(\bd{x}_2,\bd{x}_3\right)\big\}$$
This proves that $\hat{\rho}$ is a pseudometric.
\end{proof}

% \section{Uniform Urysohn's lemma}
% \noindent
% The following result is a uniform version of Urysohn lemma.

% \begin{theorem}\label{theorem:uniform_Urysohn_lemma}
% Let $X$ be a set and let $\{U_n\}_{n\in \NN}$ be a sequence of reflexive and symmetric relations on $X$ such that
% $$U_{n+1} \cdot U_{n+1} \subseteq U_n$$
% for every $n \in \NN$. Let $Z$ be a subset of $X$. Then there exists a map $f:X\ra \RR$ such that the following assertions hold.
% \begin{enumerate}[label=\emph{\textbf{(\arabic*)}}, leftmargin=*]
% \item The inequality
% $$0 \leq f(x) \leq 1$$
% holds for each $x$ in $X$.
% \item $f_{\mid Z}$ is the constant zero function on $Z$.
% \item Fix $n \in \NN$. If 
% $$f(x) < \frac{1}{2^n}$$
% for some $x$ in $X$, then $x$ in $U_n(Z)$.
% \item If $x_1,x_2 \in X$ and $(x_1,x_2) \in U_{n+1}$ for some $n \in \NN$, then the inequality
% $$|f(x_1) - f(x_2)| < \frac{1}{2^n}$$
% holds.
% \end{enumerate}
% \end{theorem}
% \begin{proof}
% For each $n\in \NN$ and each integer $k \in \{0,1,...,2^n\}$ we construct a set $Z_{\frac{k}{2^n}}$. The construction goes by recursion on $n\in \NN$. We set $Z_0 = Z$ and $Z_1 = U_0(Z)$. Next suppose that for $n\in \NN$ and $k\in \{0,1,...,2^n\}$ sets $Z_{\frac{k}{2^n}}$ are defined. Then we define
% $$Z_{\frac{2k + 1}{2^{n+1}}} = U_{n+1}\left(Z_{\frac{k}{2^n}}\right)$$
% for each $k \in \{0,1,...,2^{n}\}$. By construction
% $$Z_{\frac{k}{2^n}} \subseteq U_n\left(Z_{\frac{k}{2^n}}\right) \subseteq Z_{\frac{k+1}{2^n}}$$
% for every $n\in \NN$ and every $k\in \{0,1,...,2^{n}\}$. If $x \in U_0(Z)$, then we define 
% $$f(x) = \inf \bigg\{\frac{k}{2^n}\,\bigg|\,x\in Z_{\frac{k}{2^n}}\mbox{ for some }n\in \NN\mbox{ and } k\in \{0,1,...,2^n\}\bigg\}
% $$
% For $x\not \in U_0(Z)$ we set $f(x) = 1$. Then $f$ is a real valued function on $X$ such that $0\leq f(x) \leq 1$ for every $x$ in $X$. Moreover, $f_{\mid Z} \equiv 0$ and for $n\in \NN$ and $x \in X$ the inequality
% $$f(x) < \frac{1}{2^n}$$
% implies that $x \in U_n(Z)$. Fix now $x_1,x_2\in X$ such that $(x_1,x_2)\in U_{n+1}$. If $f(x_1) = 1$, then 
% $$f(x_2) \leq 1 < f(x_1) + \frac{1}{2^n}$$
% If $f(x_1) < 1$, then there exists the smallest $k\in \{0,1,...,2^{n+1}\}$ such that $x_1 \in Z_{\frac{k}{2^{n+1}}}$. In that case we have $x_2 \in U_{n+1}\left(Z_{\frac{k}{2^{n+1}}}\right) \subseteq Z_{\frac{k+1}{2^{n+1}}}$. Thus 
% $$f(x_2) \leq \frac{k+1}{2^{n+1}} = \frac{k-1}{2^{n+1}} + \frac{2}{2^{n+1}} < f(x_1) + \frac{1}{2^n}$$
% We proved that
% $$f(x_2) < f(x_1) + \frac{1}{2^n}$$
% By symmetry we have 
% $$f(x_2) < f(x_1) + \frac{1}{2^n}$$
% and hence the inequality
% $$|f(x_1) - f(x_2)| < \frac{1}{2^n}$$
% holds.
% \end{proof}
% \noindent
% Now we give interesting application of uniform Urysohn's lemma. First we need the following notion.

% \begin{definition}
% Let $(X,\tau)$ be a topological space. Suppose that for every closed subset $F$ of $(X,\tau)$ and for every point $x$ in $X\setminus  F$ there exists a function $f:X\ra \RR$ continuous with respect to $\tau$ and natural topology in $\RR$ such that $f(F) = \{1\}$ and $f(x) = 0$. Then $(X,\tau)$ is \textit{a completely regular space}. 
% \end{definition}

% \begin{theorem}\label{theorem:image_of_the_canonical_functor_is_completely_regular_space}
% The image of the object part of the functor
% $$\Unif \ni (X,\fU) \mapsto (X,\tau_{\fU}) \in \Top$$
% consists of the class of completely regular spaces. 
% \end{theorem}
% \begin{proof}
% Let $(X,\fU)$ be a uniform space. Consider a closed set $F$ with respect to $\tau_{\fU}$. Let $x$ be a point in $X\setminus F$. Since $x \not \in F$ and $F$ is closed in $\tau_{\fU}$, we derive that there exists $U \in \fU$ such that $U(x) \cap F = \emptyset$. Suppose that $\{U_n\}_{n\in \NN}$ is a sequence of elements of $\fU$ such that $U_0 = U$ and $U_{n+1}\cdot U_{n+1} \subseteq U_n$ for each $n \in \NN$. According to Theorem \ref{theorem:uniform_Urysohn_lemma} there exists a map $f:X\ra \RR$ such that $f(F) = \{1\}, f(x) = 0$ and if $x_1,x_2 \in X$ and $(x_1,x_2) \in U_{n+1}$ for some $n \in \NN$, then the inequality
% $$|f(x_1) - f(x_2)| < \frac{1}{2^n}$$
% holds. By virtue of Example \ref{example:natural_uniform_structure_on_real_line} and the fact above $f$ gives rise to a uniform morphism $(X,\fU)\ra (\RR,\fE)$. According to Fact \ref{fact:uniform_morphism_is_a_continuous_map} the map $f:X\ra \RR$ is continuous with respect to $\tau_{\fU}$ and the natural topology on $\RR$. This shows that $(X,\tau_{\fU})$ is completely regular.\\ 
% Suppose now that $(X,\tau)$ is a completely regular space. Consider the set $C(\tau,\RR)$ of all continuous real valued functions on $(X,\tau)$. For $m\in \NN_+$ and of $m$-functions $f_1,...,f_m\in C(\tau,\RR)$ define
% $$\rho_{f_1,...,f_m}(x,y) = \max \big\{|f_1(x) - f_1(y)|,...,|f_m(x) - f_m(y)|\big\}$$
% where $x,y\in X$. Clearly $\rho_{f_1,...,f_m}$ is a pseudometric on $X$. Next consider a family $\fU$ of all $U\in \fD_X$ such that there exist a finite subset $\{f_1,...,f_m\}\subseteq C(\tau,\RR)$ for some $m \in \NN_+$ and $\epsilon > 0$ such that
% $$\big\{(x,y)\in X\times X\,\big|\,\rho_{f_1,...,f_m}(x,y) < \epsilon\big\}\subseteq U$$
% Clearly $\fU$ is a uniform structure on $X$. Suppose that $\cO \in \tau_{\fU}$. Then for each point $z$ in $\cO$ there exists $U$ in $\fU$ such that $B(z,U)\subseteq \cO$. By definition there exists a finite subset $\{f_1,...,f_m\}\subseteq C(\tau,\RR)$ for some $m \in \NN_+$ and $\epsilon > 0$ such that
% $$\big\{(x,y)\in X\times X\,\big|\,\rho_{f_1,...,f_m}(x,y) < \epsilon\big\}\subseteq U$$
% Thus
% $$\bigcap_{i=1}^mf_i^{-1}\bigg(\big(f_i(z) - \epsilon, f_i(z) + \epsilon\big)\bigg) = \big\{y\in X \,\big|\, \rho_{f_1,...,f_m}(z,y) < \epsilon\big\} \subseteq B(z,U)\subseteq \cO$$
% Since $f_1,...,f_m$ are continuous on $X$ with respect to $\tau$, we derive from the inclusion above that there exists an open neighborhood of $z$ with respect to $\tau$ contained in $\cO$. According to the fact that $z$ is an arbitrary point in $\cO$ it follows that $\cO \in \tau$. This proves that $\tau_{\fU}\subseteq \tau$. Now we prove the converse. For this assume that $\cO \in \tau$. We claim that $\cO$ is also open in the topology induced by $\fU$. For this pick $z \in \cO$. Since $(X,\tau)$ is completely regular, there exists a function $f_z:X\ra \RR$ continuous with respect to $\tau$ such that $f_z(X\setminus \cO)\subseteq \{1\}$ and $f_z(z) = 0$. Let $U_z$ be a set consisting of all pairs in $X\times X$ for which $\rho_{f_z}$ is smaller than $1$. Then $U_z\in \fU$ and obviously $B(z,U_z)\subseteq \cO$. Thus
% $$\cO = \bigcup_{z\in \cO}B(z,U_z)$$
% and this proves the claim that $\cO$ is in $\tau_{\fU}$. Hence $\tau \subseteq \tau_{\fU}$. This completes the proof of the first assertion.\\
% The proof of the second assertion is left for the reader as an exercise. Note that in the proof one can use complete regularity of topologies induced by uniform structures.   
% \end{proof}



% \section{Completeness for pseudometric spaces}
% \noindent
% In this section we discuss the notion of completeness in the important special case of pseudometrizable uniform spaces.






% \noindent
% The following theorem and its proof is due to Felix Hausdorff.



% \begin{corollary}\label{corollary:completion_of_metric_spaces}
% Let $(X,\rho)$ be a metric space. Then there exists a complete metric space $(\ol{X},\ol{\rho})$ and a map $j:X\ra \ol{X}$ such that $j(X)$ is dense in $\ol{X}$ with respect to topology induced by $\ol{\rho}$ and
% $$\rho(x_1,x_2) = \ol{\rho}(j(x_1),j(x_2))$$
% for every $x_1,x_2\in X$.
% \end{corollary}
% \begin{proof}
% According to Theorem \ref{theorem:completion_of_a_pseudometric_space} there exists a complete pseudometric space $\left(\tilde{X},\tilde{\rho}\right)$ and a map $\tilde{j}:X\ra \tilde{X}$ such that $j(X)$ is dense in $\tilde{X}$ with respect to topology induced by $\tilde{\rho}$ and
% $$\rho(x_1,x_2) = \tilde{\rho}(j(x_1),j(x_2))$$
% for every $x_1,x_2\in X$. Let $\ol{X}$ be the quotient set of $X$ with respect to equivalence relation $\simeq$ given by 
% $$\bd{x}_1\simeq \bd{x}_2\,\Leftrightarrow\,\tilde{\rho}(\bd{x}_1,\bd{x}_2) = 0$$
% and let $q:\tilde{X}\ra \ol{X}$ be the quotient map. Define a pseudometric $\ol{\rho}$ on $\ol{X}$ by formula
% $$\ol{\rho}\left(q(\bd{x}_1),q(\bd{x}_2)\right) = \tilde{\rho}(\bd{x}_1,\bd{x}_2)$$
% for all $\bd{x}_1,\bd{x}_2\in \tilde{X}$. Then $\ol{\rho}$ is a metric on $\ol{X}$ and $\left(\ol{X},\ol{\rho}\right)$ is complete. We define $j = q\cdot \tilde{j}$. Then $j:X\ra \ol{X}$ satisfies$$\rho(x_1,x_2) = \ol{\rho}(j(x_1),j(x_2))$$
% for every $x_1,x_2\in X$ and $j(X)$ is dense in $\ol{X}$ with respect to the topology induced by $\ol{\rho}$.
% \end{proof}






% \section{Pseudometrization theorem of Alexandroff and Urysohn}
% \noindent
% Now we prove classical result due to Alexandroff and Urysohn. First we introduce the important notion.

% \begin{definition}
% Let $X$ be a set and let $\rho$ be a pseudometric on $X$. Then we denote by $\fU_{\rho}$ a family of all $U \in \fD_X$ such that there exists $\epsilon > 0$ with the property that
% $$\big\{(x_1,x_2) \in X\times X\,\big|\,\rho(x_1,x_2) < \epsilon \big\}\subseteq U$$
% Then $\fU_{\rho}$ is a uniform structure called \textit{the uniform structure induced by $\rho$ on $X$}.
% \end{definition}

% \begin{theorem}[Alexandroff-Urysohn]\label{theorem:Alexandrov_Urysohn}
% Let $(X,\fU)$ be a uniform space. Then the following conditions are equivalent.
% \begin{enumerate}[label=\emph{\textbf{(\roman*)}}, leftmargin=*]
% \item There exists a sequence $\{U_n\}_{n\in \NN}$ of elements in $\fU$ such that for every $U$ in $\fU$ there exists $n\in \NN$ with the property that $U_n \subseteq U$.
% \item There exists a pseudometric $\rho$ on $X$ such that $\fU$ coincides with $\fU_{\rho}$.
% \end{enumerate}
% \end{theorem}
% \begin{proof}
% Suppose that for every $U \in \fU$ there exists $n\in \NN$ such that $U_n\subseteq U$. Without loss of generality we may assume that for the sequence $\{U_n\}_{n\in \NN}$ the condition
% $$U_{n+1} \cdot U_{n+1} \subseteq U_n$$
% holds for every $n \in \NN$. For each $y \in X$ let $f_y:X\ra \RR$ be a map such that the following assertions hold.
% \begin{enumerate}[label=\textbf{(\arabic*)}, leftmargin=*]
% \item The inequality
% $$0 \leq f_y(x) \leq 1$$
% holds for each $x$ in $X$.
% \item $y$ is in the zero set of $f_y$.
% \item Fix $n \in \NN$. If 
% $$f_y(x) < \frac{1}{2^n}$$
% for some $x$ in $X$, then $(x,y) \in U_n$.
% \item If $x_1,x_2 \in X$ and $(x_1,x_2) \in U_{n+1}$ for some $n \in \NN$, then the inequality
% $$|f_y(x_1) - f_y(x_2)| < \frac{1}{2^n}$$
% holds.
% \end{enumerate}
% Function $f_y$ exists according to Theorem \ref{theorem:uniform_Urysohn_lemma}. Next for every $x_1,x_2\in X$ define
% $$\rho_U(x_1,x_2) = \sup_{y\in X}|f_y(x_1) - f_y(x_2)|$$
% It is easy to verify that $\rho_U$ is a pseudometric on $X$. We have
% $$U_{n+1} \subseteq \bigg\{(x_1,x_2)\in X\times X\,\bigg|\,\rho_U(x_1,x_2) < \frac{1}{2^n}\bigg\} \subseteq \bigg\{(x_1,x_2)\in X\times X\,\bigg|\,f_{x_1}(x_2) < \frac{1}{2^n}\bigg\} \subseteq U_n$$
% for every $n\in \NN$. Hence $\fU = \fU_{\rho}$. This completes the proof of $\textbf{(i)}\Rightarrow \textbf{(ii)}$.\\
% Suppose that $\fU = \fU_{\rho}$ for some pseudometric $\rho$ on $X$. Define
% $$U_n = \big\{(x_1,x_2)\in X\times X\,\big|\,\rho(x_1,x_2) < \frac{1}{2^n}\big\}$$
% for $n\in \NN$. Then for every $U$ in $\fU$ there exists $n\in \NN$ such that $U_n \subseteq U$. This proves $\textbf{(ii)}\Rightarrow \textbf{(i)}$
% \end{proof}
% \noindent
% Next we use Alexandroff's and Urysohn's result in order to characterize uniform structures by means of pseudometrics.

% \begin{definition}\label{example:uniformity_induced_by_family_of_pseudometrics}
% Let $X$ be a set, let $\fU$ be a uniform structure and let $\Phi$ be a family of pseudometrics on $X$. If
% $$\fU = \bigcup_{\rho \in \Phi}\fU_{\rho}$$
% then $\fU$ is \textit{the uniform structure on $X$ induced by $\Phi$}. 
% \end{definition}

% \begin{theorem}\label{theorem:each_uniform_space_can_be_described_by_family_of_pseudometrics}
% Every uniform structure on a set $X$ is induced by some family of pseudometrics.
% \end{theorem}
% \begin{proof}
% Suppose that $\fU$ is a uniform structure on $X$. Pick $U$ in $\fU$ and consider the sequence $\{U_n\}_{n\in \NN}$ of elements of $\fU$ such that $U_0 = U$ and
% $$U_{n+1} \cdot U_{n+1} \subseteq U_n$$
% for every $n \in \NN$. Define 
% $$\fU_{U} = \big\{W\in \fD_X\,\big|\,U_n\subseteq W\mbox{ for some }n\in \NN\big\}$$
% Then $\fU_{U}$ is a uniform structure on $X$ such that $U\in \fU_{U}\subseteq \fU$. By Theorem \ref{theorem:Alexandrov_Urysohn} there exists a pseudometric $\rho_U$ on $X$ such that $\fU_{\rho_U} = \fU_U$. Define 
% $$\Phi = \big\{\rho_U\,\big|\,U\in \fU\big\}$$
% Then
% $$\bigcup_{\rho \in \Phi}\fU_{\rho} = \bigcup_{U\in \fU}\fU = \fU$$
% and this completes the proof.
% \end{proof}

























































\small
\bibliographystyle{apalike}
\bibliography{../zzz}

\end{document}