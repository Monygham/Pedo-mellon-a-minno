\input ../pree.tex

\begin{document}

\title{Uniform spaces}
\date{}
\maketitle

\section{Introduction}
\noindent
These notes are devoted to general theory of uniform spaces, which are mathematical objects rigorously encapsulating the intuitive notion of uniformity. In the first section we prove  important pseudometrizability theorem due to Andr{\'e} Weil. This result is crucial for further developments. Then we introduce uniform spaces and their category $\Unif$. We describe small limits in $\Unif$. Next we construct a small limits preserving functor $\Unif\ra \Top$. We show that the image of the functor consists of completely regular spaces. We also discuss pseudometrizable unform spaces and show that each uniform space can be embedded into a product of pseudometrizable uniform spaces. Next we define complete uniform spaces and prove results concerning their properties with respect to products, embeddings and extensions of uniform morphisms. Then we identify the general notion of completeness for uniform spaces  to completeness in terms of Cauchy sequences in the context of pseudometric spaces. In the last section we prove that each Hausdorff uniform space admits universal completion.

\section{Existence of pseudometrics}
\noindent
This section is devoted to very important result on existence of pseudometrics due to Andr{\'e} Weil. Let $X$ be a set. We denote 
$$\Delta_X = \big\{(x,x)\in X\,\big|\,x\in X\big\}$$
For subsets $V,W$ of $X\times X$ we denote
$$W\cdot V = \big\{(x,z)\in X\times X\,\big|\,(x,y)\in V\mbox{ and }(y,z)\in W\mbox{ for some }y\in X\big\}$$
Now we are ready to introduce basic notions.

\begin{definition}
Let $X$ be a set. Suppose that $V$ is a subset of $X\times X$ satisfying the following assertions.
\begin{enumerate}[label=\textbf{(\arabic*)}, leftmargin=*]
\item If $(x,y)\in V$ for some $x,y\in X$, then $(y,x)\in V$.
\item $V$ contains $\Delta_X$.
\end{enumerate}
Then $V$ \textit{is a surrounding of $\Delta_X$}.
\end{definition}
\noindent
Finally we recall the notion of pseudometric.

\begin{definition}
Let $X$ be a set. Suppose that a function $\rho:X\times X\ra \RR$ satisfies the following assertions.
\begin{enumerate}[label=\textbf{(\arabic*)}, leftmargin=*]
\item $\rho(x,y) \geq 0$ for all $x,y\in X$.
\item $\rho(x,x) = 0$ for all $x\in X$.
\item $\rho(x,y) = \rho(y,x)$ for all $x,y\in X$.
\item $\rho(x,z)\leq \rho(x,y) + \rho(y,z)$ for all $x,y,z\in X$.
\end{enumerate}
Then $\rho$ is \textit{a pseudometric on $X$}. The last condition is called \textit{the triangle inequality}.
\end{definition}

\begin{definition}
Let $\rho$ be a pseudometric on $X$. Suppose that $\rho(x,y) = 0$ implies $x = y$ for all $x,y\in X$. Then $\rho$ is \textit{a metric on $X$}.
\end{definition}
\noindent
Now we state and prove the fundamental result on the existence of pseudometrics.

\begin{theorem}\label{theorem:Weils_theorem_on_pseudometrics}
Let $X$ be a set and let $\{V_n\}_{n\in \NN}$ be a sequence of surroundings of $\Delta_X$ such that
$$V_{n+1}\cdot V_{n+1}\cdot V_{n+1} \subseteq V_n$$
for every $n\in \NN$. Then there exists a pseudometric $\rho$ on $X$ bounded by $1$ such that
$$\bigg\{(x,y)\in X\times X\,\bigg|\,\rho(x,y)<\frac{1}{2^n}\bigg\} \subseteq V_n \subseteq \bigg\{(x,y)\in X\times X\,\bigg|\,\rho(x,y) \leq \frac{1}{2^n}\bigg\}$$
for every $n\in \NN$.
\end{theorem}
\noindent
For the proof consider a function $f$ defined on $X\times X$ given by formula 
$$\begin{cases}
0 & \mbox{ if }(x,y)\in V_n\mbox{ for each }n\in \NN\\
\frac{1}{2^n} & \mbox{ if }(x,y)\in V_n\setminus V_{n+1}\\
1 & \mbox{ if }(x,y)\not \in V_0
\end{cases}$$
The proof relies on the following result. 

\begin{lemma}\label{lemma:inclusion_of_surroundings_for_pseudometric}
For each $n \in \NN$ and every finite sequence $x_0,...,x_m$ the inequality
$$\sum_{i=1}^mf(x_{i-1},x_i) < \frac{1}{2^n}$$
implies that $(x_0,x_m) \in V_n$.
\end{lemma}
\begin{proof}[Proof of the lemma]
The proof goes by induction on $m$. For $m = 0$ and $m = 1$ the claim is trivial. Assume that $m$ is greater than one and suppose that the assertion holds for all numbers smaller than $m$. Suppose that
$$\sum_{i=1}^mf(x_{i-1},x_i) < \frac{1}{2^n}$$
for some sequence $x_0,...,x_m$ of elements in $X$. We have
$$\mbox{ either }f(x_0,x_1) < \frac{1}{2^{n+1}}\mbox{ or }f(x_{m-1},x_m) < \frac{1}{2^{n+1}}$$
Without loss of generality we may assume that the first inequality holds. Let $k$ be the greatest number in $\{1,...,m-1\}$ such that
$$\sum_{i=1}^kf(x_{i-1},x_i) < \frac{1}{2^{n+1}}$$
Next we consider two cases.
\begin{itemize}
\item If $k < m-1$, then we have
$$\sum_{i=1}^{k}f(x_{i-1},x_i) < \frac{1}{2^{n+1}},\,f(x_k,x_{k+1}) \leq \frac{1}{2^{n+1}},\,\sum_{i=k+1}^mf(x_{i-1},x_i) < \frac{1}{2^{n+1}}$$
By induction hypothesis we have $(x_0,x_k)\in V_{n+1},(x_{k+1},x_m)\in V_{n+1}$ and by definition of $f$ we have $(x_k,x_{k+1})\in V_{n+1}$. Hence
$$(x_0,x_m) \in V_{n+1}\cdot V_{n+1}\cdot V_{n+1}\subseteq V_n$$
and the assertion holds.
\item If $k = m-1$. Then 
$$\sum_{i=1}^{m-1}f(x_{i-1},x_i) < \frac{1}{2^{n+1}},\,f(x_{m-1},x_m)\leq \frac{1}{2^{n+1}}$$
By induction hypothesis we have $(x_0,x_{m-1})\in V_{n+1}$ and by definition of $f$ we have $(x_{m-1},x_{m})\in V_{n+1}$. Hence
$$(x_0,x_{m}) \in V_{n+1}\cdot V_{n+1} \subseteq V_{n+1}\cdot V_{n+1}\cdot V_{n+1} \subseteq V_n$$
and the assertion holds.
\end{itemize}
Thus the result follows from induction.
\end{proof}

\begin{proof}[Proof of the theorem]
For $x,y\in X$ we define
$$\rho(x,y) = \inf \bigg\{\sum_{i=1}^mf(x_{i-1},x_i)\,\bigg|\,\mbox{ for every }m\in \NN\mbox{ an every finite sequence }x_0,...,x_m\mbox{ such that }x_0 = x,\,x_m = y\bigg\}$$
It is easy to verify that the function $\rho$ is a pseudometric on $X$. It remains to prove that
$$\bigg\{(x,y)\in X\times X\,\bigg|\,\rho(x,y)<\frac{1}{2^n}\bigg\} \subseteq V_n \subseteq \bigg\{(x,y)\in X\times X\,\bigg|\,\rho(x,y) \leq \frac{1}{2^n}\bigg\}$$
The first inclusion follows from Lemma \ref{lemma:inclusion_of_surroundings_for_pseudometric} and the second follows from the fact that $\rho(x,y) \leq f(x,y)$ for every $x,y\in X$.
\end{proof}

\section{Uniform structures and uniform spaces}
\noindent
In this section we introduce main object of our study.

\begin{definition}
Let $X$ be a set. Suppose that $\fU$ is a collection of surroundings of $\Delta_X$ which satisfies the following two assertions.
\begin{enumerate}[label=\textbf{(\arabic*)}, leftmargin=*]
\item If $U \in \fU$ and $W$ is a surrounding of $\Delta_X$ such that $V\subseteq W$, then $W\in \fU$.
\item If $U,W\in \fU$, then $U\cap W \in \fU$. 
\item If $U \in \fU$, then there exists $W\in \fU$ such that $W\cdot W \subseteq U$.
\end{enumerate}
Then $\fU$ is \textit{a uniform structure on $X$}.
\end{definition}
\noindent
No we give two basic examples.

\begin{example}\label{example:discrete_uniform_structure}
Let $X$ be a set. Then the family $\fD_X$ of all surroundings of $\Delta_X$ is a uniform structure on $X$. It is called \textit{the discrete uniform structure on $X$}.
\end{example}

\begin{example}\label{example:uniform_structure_on_reals}
The family
$$\big\{U\subseteq \fD_{\RR}\,\big|\,\mbox{ there exists }\epsilon > 0\mbox{ such that }|x - y| < \epsilon\mbox{ implies }(x,y)\in U\mbox{ for every }x,y\in X\big\}$$
gives a uniform structure on $\RR$. It is called \textit{the natural uniform structure on $\RR$}.
\end{example}
\noindent
The example above can be generalized.

\begin{example}\label{example:uniform_structure_induced_by_pseudometric}
Let $X$ be a set and let $\rho$ be a pseudometric on $X$. The family
$$\fU_{\rho} = \big\{U\subseteq \fD_{X}\,\big|\,\mbox{ there exists }\epsilon > 0\mbox{ such that }\rho(x,y) < \epsilon\mbox{ implies }(x,y)\in U\mbox{ for every }x,y\in X\big\}$$
is a uniform structure on $X$. It is called \textit{the uniform structure induced by $\rho$}.
\end{example}
    
\begin{fact}\label{fact:uniform_structures_are_closed_under_intersections}
Let $X$ be a set and let $\{\fU_i\}_{i\in I}$ be a family of uniform structures on $X$. Then 
$$\bigcap_{i\in I}\fU_i$$
is a uniform structure on $X$.
\end{fact}
\begin{proof}
Left for the reader.
\end{proof}

\begin{corollary}\label{corollary:smallest_uniform_structure_containing_given_family_of_surroundings}
Let $X$ be a set and let $\cF$ be a family of surrounding of $\Delta_X$. Then there exists the smallest (with respect to inclusion) uniform structure $\fU$ on $X$ which contains $\cF$.
\end{corollary}
\begin{proof}
Let $\{\fU_i\}_{i\in I}$ be a family of all uniform structures on $X$ which contain $\cF$. The family is nonempty, since it contains the discrete uniform structure on $X$. The intersection $$\fU = \bigcap_{i\in I}\fU_i$$
is a uniform structure on $X$ by Fact \ref{fact:uniform_structures_are_closed_under_intersections}. Hence it is the smallest uniform structure on $X$ which contains $\cF$.
\end{proof}

\begin{definition}
A pair $(X,\fU)$ consisting of a set $X$ and a uniform structure $\fU$ on $X$ is \textit{a uniform space}.
\end{definition}

\begin{definition}
Let $(X,\fU)$ be a uniform space. A surrounding $V$ in $\fU$ is called \textit{an entourage of the diagonal in $(X,\fU)$}. 
\end{definition}

\begin{definition}
Let $(X,\fU),(Y,\fV)$ be uniform spaces and let $f:X\ra Y$ be a map. Suppose that $\left(f\times f\right)^{-1}(V) \in \fU$ for every $V\in \fV$. Then $f$ is \textit{a morphism of uniform spaces}. 
\end{definition}

\begin{remark}\label{remark:category_of_uniform_spaces}
Uniform spaces and their morphisms form a category. We denote this category by $\Unif$.
\end{remark}
\noindent
In order to study categorical properties of $\Unif$ we use the following result.

\begin{theorem}\label{theorem:description_of_uniform_structure_introduced_by_a_family_of_maps}
Let $X$ be a set and let $\{(X_i,\fU_i)\}_{i\in I}$ be a family of uniform spaces. Consider a family $\big\{f_i:X\ra X_i\big\}_{i\in I}$ of maps. Suppose that $\fU$ is the smallest uniform structure on $X$ which makes $\{f_i\}_{i\in I}$ into a family of uniform morphisms. Then $U \in \fU$ if and only if there exist $n\in \NN_+$, $i_1,...,i_n\in I$ and $U_1 \in \fU_{i_1},...,U_n\in \fU_{i_n}$ such that
$$\bigcap_{k=1}^n\left(f_{i_k}\times f_{i_k}\right)^{-1}(U_k) \subseteq U$$
\end{theorem}
\begin{proof}
Consider the family $\cU$ of all surrounding $U$ of $\Delta_X$ such that there exist $n\in \NN_+$, $i_1,...,i_n\in I$ and $U_1 \in \fU_{i_1},...,U_n\in \fU_{i_n}$ satisfying
$$\bigcap_{k=1}^n\left(f_{i_k}\times f_{i_k}\right)^{-1}(U_k) \subseteq U$$
It is easy to verify (we left it for the reader) that $\cU$ is a uniform structure on $X$. Moreover, for every $n\in \NN_+$, $i_1,...,i_n\in I$ and $U_1 \in \fU_{i_1},...,U_n\in \fU_{i_n}$ we have
$$\bigcap_{k=1}^n\left(f_{i_k}\times f_{i_k}\right)^{-1}(U_k) \in \fU$$
Hence $\cU \subseteq \fU$. Note also that $f_i$ is a uniform morphism $(X,\cU)\ra (X_i,\fU_i)$ for each $i\in I$. Thus $\fU \subseteq \cU$. Therefore, $\cU = \fU$ and this proves the theorem. 
\end{proof}
\noindent
Now we describe small limits in $\Unif$.

\begin{theorem}\label{theorem:limits_of_uniform_spaces_description}
Let $\cI$ be a small category and let $F:\cI\ra \Unif$ be a functor given by $F(i) = (X_i,\fU_i)$ for $i\in \cI$. Let $\big\{f_i:X \ra X_i \big\}_{i\in \cI}$ be a limiting cone of the composition of $F$ with the functor $\Unif \ra \Set$ which sends each uniform space to its underlying set. Consider the smallest uniform structure $\fU$ on $X$ which makes $\{f_i\}_{i\in \cI}$ into a family of uniform morphisms. Then $(X,\fU)$ together with family $\{f_i\}_{i\in \cI}$ is a limiting cone of $F$.
\end{theorem}
\begin{proof}
We may equivalently describe $\fU$ as the smallest uniform structure on $X$ such that 
$$\left(f_i\times f_i\right)^{-1}(U) \in \fU$$
for every $i\in \cI$ and every $U \in \fU_i$. Suppose that $\big\{g_i:(Y,\fV) \ra (X_i,\fU_i)\big\}_{i\in \cI}$ is some cone over $F$. Then there exists a unique map $h:Y \ra X$ such that $h\cdot f_i = g_i$ for every $i\in \cI$. It is easy to verify that
$$\big\{U\in \fU\,\big|\,\left(h\times h\right)^{-1}(U)\mbox{ is an entourage of the diagonal in }\fV\big\}$$
is a uniform structure on $X$. Moreover, it contains $\left(f_i\times f_i\right)^{-1}(U)$ for every $i\in \cI$ and every $U \in \fU_i$. Since $\fU$ is the smallest such uniform structure, we derive that $\fU$ and
$$\big\{U\in \fU\,\big|\,\left(h\times h\right)^{-1}(U)\mbox{ is an entourage of the diagonal in }\fB\big\}$$
coincide and hence $h$ is a morphism of uniform spaces $\left(Y,\fB\right) \ra \left(X,\fU\right)$. This shows that $(X,\fU)$ together with $\big\{f_i:X \ra X_i\big\}_{i\in \cI}$ is a limiting cone of $F$.
\end{proof}

\begin{definition}
Let $j:\left(Z,\fO\right)\ra \left(X,\fU\right)$ be a morphism of uniform spaces. If $j$ is injective and $\fO$ is the smallest uniform structure which makes $j$ into a uniform morphism to $\left(X,\fU\right)$, then $j$ is \textit{an embedding of uniform spaces}.
\end{definition}

\begin{proposition}\label{proposition:embeddings_are_closed_under_small_limits}
Let $\cI$ be a small category, let $F_1,F_2:\cI\ra \Unif$ be functors and let $\tau:F_1\ra F_2$ be a natural transformation such that $\tau_i$ is an embedding of uniform spaces for each $i \in \cI$. Then
$$\lim_{i\in \cI}\tau_i:\lim_{i\in \cI}F_1\ra \lim_{i\in \cI}F_2$$
is an embedding of uniform spaces.
\end{proposition}
\begin{proof}
Write $F_1(i) = \left(X_{1i},\fU_{1i}\right),\,F_2(i) = \left(X_{2i},\fU_{2i}\right)$ for $i\in \cI$. Suppose that 
$$\big\{f_{1i}:\left(X_1,\fU_1\right) \ra \left(X_{1i},\fU_{1i}\right)\big\}_{i\in \cI},\,\big\{f_{2i}:\left(X_2,\fU_2\right) \ra \left(X_{2i},\fU_{2i}\right)\big\}_{i\in \cI}$$
are limiting cones of $F_1$ and $F_2$, respectively. Finally let $\tilde{\tau}:\left(X_1,\fU_1\right)\ra \left(X_2,\fU_2\right)$ be a morphism induced by $\tau$. Injectivity of $\tilde{\tau}$ follows from Theorem \ref{theorem:limits_of_uniform_spaces_description} and the assumption that all components of $\tau$ are injective. Since each component of $\tau$ is an embedding, we derive by Theorem \ref{theorem:limits_of_uniform_spaces_description} that $\fU_{1}$ is the smallest uniform structure which contains
$$\big\{W \in \fD_{X_1}\,\big|\,W = \left(f_{1i}\times f_{1i}\right)^{-1}\left(\left(\tau_{i}\times \tau_{i}\right)^{-1}(U_i)\right)\mbox{ for some }U_i\in \fU_{2i}\mbox{ and }i\in \cI\big\}$$
$$= \big\{W \in \fD_{X_1}\,\big|\,W = \left(\tilde{\tau}\times \tilde{\tau}\right)^{-1}\left(\left(f_{2i}\times f_{2i}\right)^{-1}(U_i)\right)\mbox{ for some }U_i\in \fU_{2i}\mbox{ and }i\in \cI\big\}$$
According to Theorem \ref{theorem:limits_of_uniform_spaces_description} family $\fU_{2}$ is the smallest uniform structure which contains 
$$\big\{U\in \fD_{X_2}\,\big|\,U=\left(f_{2i}\times f_{2i}\right)^{-1}(U_i)\mbox{ for some }U_i\in \fU_{2i}\mbox{ and }i\in \cI\big\}$$
Thus $\fU_{2}$ is the smallest uniform structure which contains
$$\big\{W \in \fD_{X_1}\,\big|\,W = \left(\tilde{\tau}\times \tilde{\tau}\right)^{-1}(U)\mbox{ for some }U\in \fU_{2}\big\}$$
This means that $\fU_1$ is the smallest uniform structure on $X_1$ which makes $\tilde{\tau}$ into a uniform morphism to $\left(X_2,\fU_2\right)$. Hence $\tilde{\tau}$ is an embedding of uniform spaces.
\end{proof}
\noindent
Next result is a consequence of Theorem \ref{theorem:Weils_theorem_on_pseudometrics} which is useful in developing theory of uniform spaces.

\begin{theorem}\label{theorem:pseudometric_associated_to_entourage_is_uniform}
Let $(X,\fU)$ be a uniform space and let $U$ be an entourage of the diagonal in $(X,\fU)$. Then there exists a pseudometric $\rho$ on $X$ bounded by $1$ such that the following assertions hold.
\begin{enumerate}[label=\emph{\textbf{(\arabic*)}}, leftmargin=*]
\item If $\rho(x,y) < 1$ for some $x,y\in X$, then $(x,y)\in U$. 
\item We have
$$\big\{(x,y)\in X\times X\,\big|\,\rho(x,y) < \epsilon \big\}\in \fU$$
for every positive element $\epsilon$ in $\RR$.
\item For each $x$ in $X$ the map $X\ni y \mapsto \rho(x,y)\in \RR$ is a morphism of uniform spaces $\left(X,\fU\right) \ra \RR$ where $\RR$ is considered with the natural uniform structure.
\end{enumerate}
\end{theorem}
\begin{proof}
Construct a sequence $\{V_n\}_{n\in \NN}$ of entourages of the diagonal in $\left(X,\fU\right)$ by recursion. We set $V_0 = U$. Suppose that $V_n$ is defined for some $n\in \NN$. Then pick $V_{n+1}\in \fU$ such that 
$$V_{n+1}\cdot V_{n+1}\cdot V_{n+1} \subseteq V_n$$
By Theorem \ref{theorem:Weils_theorem_on_pseudometrics} there exists a pseudometric $\rho$ on $X$ which is bounded by $1$ and
$$\bigg\{(x,y)\in X\times X\,\bigg|\,\rho(x,y)<\frac{1}{2^n}\bigg\} \subseteq V_n \subseteq \bigg\{(x,y)\in X\times X\,\bigg|\,\rho(x,y) \leq \frac{1}{2^n}\bigg\}$$
for every $n\in \NN$. From this description of $\rho$ it follows that \textbf{(1)} and \textbf{(2)} hold. It remains to verify that for every $x$ in $X$ the map $f:X\ra \RR$ given by $f(y) = \rho(x,y)$ for $y\in X$ is a morphism of uniform spaces $\left(X,\fU\right) \ra \RR$ where $\RR$ is considered with the natural uniform structure. For this pick $\epsilon > 0$ and consider $n\in \NN$ such that 
$$\frac{1}{2^n} < \epsilon$$
If $(y_1,y_2) \in V_n$, then
$$\big|f(y_1) - f(y_2)\big| = \big|\rho(x,y_1) - \rho(x,y_2)\big|\leq \rho(y_1,y_2) \leq \frac{1}{2^n}  < \epsilon$$
and hence 
$$V_n \subseteq \left(f\times f\right)^{-1}\left(\big\{(\alpha, \beta)\in \RR\times \RR\big|\,|\alpha - \beta| < \epsilon\big\}\right)$$
This inclusion together with Example \ref{example:uniform_structure_on_reals} show that $f$ is a morphism of uniform spaces $\left(X,\fU\right) \ra \RR$.
\end{proof}

\section{Topology induced by uniform structure}
\noindent
We start by introducing the notion of a ball with radius given by surrounding of the diagonal.

\begin{definition}
Let $X$ be a set. For every $x$ in $X$ and $U$ in $\fD_X$ the set
$$B(x,U) = \big\{y\in X\,\big|\,(x,y)\in U\big\}$$
is \textit{the ball with center $x$ and radius $U$}.
\end{definition}

\begin{fact}\label{fact:topology_induced_by_uniform_structure}
Let $X$ be a set and let $\fU$ be a uniform structure on $X$. The family
$$\tau_{\fU} = \big\{\cO\subseteq X\,\big|\,\mbox{ for each }x\in \cO\mbox{ there exists }U\in \fU\mbox{ such that }B(x,U)\subseteq \cO\big\}$$
is a topology on $X$.
\end{fact}
\begin{proof}
We left the proof for the reader as an exercise.
\end{proof}

\begin{definition}
Let $X$ be a set and let $\fU$ be a uniform structure on $X$. Then the topology $\tau_{\fU}$ is \textit{the topology on $X$ induced by $\fU$}.
\end{definition}

\begin{fact}\label{fact:uniform_morphism_is_a_continuous_map}
Let $(X,\fU)$ and $(Y,\fV)$ be uniform spaces and let $f:X\ra Y$ be a morphism of uniform spaces. Then $f$ is a continuous map $\left(X,\tau_{\fU}\right)\ra \left(Y,\tau_{\fY}\right)$.
\end{fact}
\begin{proof}
Pick open subset $\cO$ with respect to the topology induced by $\fY$ on $Y$. Suppose that $f(x) \in \cO$ for some $x$ in $X$. Then there exists $V_x \in \fY$ such that $B(f(x),V_x)\subseteq \cO$. Note that the image of $B\big(x,\left(f\times f\right)^{-1}(V_x)\big)$ under $f$ is contained in $\cO$. Therefore,
$$f^{-1}(\cO) = \bigcup_{x\in f^{-1}(\cO)}B\big(x,\left(f\times f\right)^{-1}(V_x)\big)$$
is open in the topology induced by $\fU$.
\end{proof}

\begin{definition}
Let $X$ be a set and let $\rho$ be a pseudometric. Then $\tau_{\fU_{\rho}}$ is \textit{the topology induced by $\rho$ on $X$}.
\end{definition}

\begin{definition}
Let $X$ be a set and let $\rho$ be a pseudometric. For every point $x$ in $X$ and $\epsilon$ in $\RR_+$ the set
$$B_{\rho}(x,\epsilon) = \big\{y\in X\,\big|\,\rho(x,y) < \epsilon\big\}$$
is \textit{the open ball with respect to $\rho$ centered in $x$ and with radius $\epsilon$}.
\end{definition}

\begin{remark}\label{remark:topology_induced_by_pseudometric_open_balls_basis}
Let $X$ be a set and let $\rho$ be a pseudometric. Then $\tau_{\fU_{\rho}}$ is the topology with open basis consisting of balls $B_{\rho}(x,\epsilon)$ for $x\in X$ and $\epsilon \in \RR_+$.
\end{remark}
\noindent
The following result is a useful property of topology induced by a uniform structure. 

\begin{theorem}\label{theorem:each_entourage_contains_open_entourage}
Let $(X,\fU)$ be a uniform space and let $U \in \fU$. Fix $n$ in $\NN_+$. Then there exists $W \in \fU$ such that the following two assertions hold.
\begin{enumerate}[label=\emph{\textbf{(\arabic*)}}, leftmargin=*]
\item $\underbrace{W\cdot ...\cdot W}_{n\,\mathrm{times}}$ is a subset of $U$.
\item For every $x$ in $X$ the ball $B(x,W)$ is open with respect to the topology induced by $\fU$ on $X$. 
\item $W$ is open in $\left(X,\tau_{\fU}\right)\times \left(X,\tau_{\fU}\right)$.
\end{enumerate}
\end{theorem}
\begin{proof}
Let $\rho$ be a pseudometric on $X$ having the properties described in Theorem \ref{theorem:pseudometric_associated_to_entourage_is_uniform}. For each positive $\epsilon$ in $\RR$ define 
$$W_{\epsilon} = \big\{(x,y)\in X\times X\,\big|\,\rho(x,y) < \epsilon\big\}$$
Then $W_{\epsilon} \in \fU$. Denote $W_{\frac{1}{n}}$ by $W$. We prove that $W$ satisfies assertions in the statement. By construction $\underbrace{W\cdot ...\cdot W}_{n\,\mathrm{times}} \subseteq U$ and hence \textbf{(1)} holds. Fix now $x \in X$ and consider a map $f:X\ra \RR$ given by $f_x(y) = \rho(x,y)$ for $y\in X$. Then Theorem \ref{theorem:pseudometric_associated_to_entourage_is_uniform} asserts that $f$ is a morphism of uniform spaces $\left(X,\fU\right)\ra \RR$ where $\RR$ admits the natural uniform structure (Example \ref{example:uniform_structure_on_reals}). By Fact \ref{fact:uniform_morphism_is_a_continuous_map} and Remark \ref{remark:topology_induced_by_pseudometric_open_balls_basis} the map $f$ is continuous with respect to $\tau_{\fU}$ and the usual topology on $\RR$. It follows that
$$B(x,W_{\epsilon}) = \big\{y\in X\,\big|\,\rho(x,y) < \epsilon\big\} = \big\{y\in X\,\big|\,f(y) < \epsilon\big\}\in \tau_{\fU}$$
for every $x$ in $X$ and positive $\epsilon$. In particular, this implies \textbf{(2)}. Now we prove $\textbf{(3)}$. Fix $(x,y) \in W$. Since $\rho(x,y) < \frac{1}{n}$, there exist positive numbers $\epsilon_1,\epsilon_2$ such that $\epsilon_1 + \rho(x,y) + \epsilon_2 < \frac{1}{n}$. Hence 
$$B(x,W_{\epsilon_1})\times B(y,W_{\epsilon_2}) \subseteq W$$
Thus $(x,y)$ is an interior point of $W$ in $\left(X,\tau_{\fU}\right)\times \left(X,\tau_{\fU}\right)$. Since $(x,y)$ is an arbitrary point of $W$, we derive that $W$ is an open subset of $\left(X,\tau_{\fU}\right)\times \left(X,\tau_{\fU}\right)$. This completes the proof of \textbf{(3)}.
\end{proof}
\noindent
Fact \ref{fact:topology_induced_by_uniform_structure} and Fact \ref{fact:uniform_morphism_is_a_continuous_map} imply the existence of the functor 
$$\Unif \ni (X,\fU) \mapsto (X,\tau_{\fU})\in   \Top$$
In the remaining part of this section we shall investigate the properties of this functor. We start by describing the image of the functor.

\begin{definition}
Let $(X,\tau)$ be a topological space. Suppose that for every closed subset $F$ of $X$ and for every point $x$ in $X\setminus  F$ there exists a continuous function $f:X\ra \RR$ such that $f(F) \subseteq \{1\}$ and $f(x) = 0$. Then $X$ is \textit{a completely regular space}. 
\end{definition}

\begin{definition}
Let $(X,\fU)$ be a uniform space. Suppose that
$$\Delta_X = \bigcap_{U\in \fU}U$$
Then $(X,\fU)$ is \textit{a Hausdorff uniform space}.
\end{definition}

\begin{theorem}\label{theorem:image_of_the_canonical_functor_is_completely_regular_space}
The image of the object part of the functor
$$\Unif \ni (X,\fU) \mapsto (X,\tau_{\fU}) \in \Top$$
consists of the class of completely regular spaces. Moreover, $\left(X,\fU\right)$ is a Hausdorff uniform space if and only if $\left(X,\tau_{\fU}\right)$ is a Hausdorff topological space. 
\end{theorem}
\begin{proof}
Let $(X,\fU)$ be a uniform space. Consider a closed set $F$ with respect to $\tau_{\fU}$. Let $x$ be a point in $X\setminus F$. Since $x \not \in F$ and $F$ is closed in $\tau_{\fU}$, we derive that there exists $U \in \fU$ such that $B(x,U)\cap F = \emptyset$. Fix a pseudometric $\rho$ on $X$ corresponding to $U$ as in Theorem \ref{theorem:pseudometric_associated_to_entourage_is_uniform}. Consider a map $f:X\ra \RR$ given by $f(y) = \rho(x,y)$ for $y\in X$. Then Theorem \ref{theorem:pseudometric_associated_to_entourage_is_uniform} asserts that $f$ is a morphism of uniform spaces $\left(X,\fU\right)\ra \RR$ where $\RR$ admits the natural uniform structure (Example \ref{example:uniform_structure_on_reals}). By Fact \ref{fact:uniform_morphism_is_a_continuous_map} and Remark \ref{remark:topology_induced_by_pseudometric_open_balls_basis} the map $f$ is continuous with respect to $\tau_{\fU}$ and the usual topology on $\RR$. Note also that the image of $f$ is contained in $[0,1]$. Pick $y\in F$. Then $y \not \in B(x,U)$ and hence $(x,y) \not \in U$. We deduce that $f(y) = \rho(x,y) = 1$. On the other hand $f(x) = \rho(x,x) = 0$. Therefore, $f(F) \subseteq \{1\}$ and $f(x) = 0$. Thus $(X,\tau_{\fU})$ is a completely regular space.\\
Suppose now that $(X,\tau)$ is a completely regular space. Consider the set $C(\tau,\RR)$ of all continuous real valued functions on $(X,\tau)$. For $m\in \NN_+$ and set of $m$ functions $f_1,...,f_m\in C(\tau,\RR)$ define
$$\rho_{f_1,...,f_m}(x,y) = \max \big\{|f_1(x) - f_1(y)|,...,|f_m(x) - f_m(y)|\big\}$$
where $x,y\in X$. Clearly $\rho_{f_1,...,f_m}$ is a pseudometric on $X$. Next consider a family $\fU$ of all $U\in \fD_X$ such that there exist a finite subset $\{f_1,...,f_m\}\subseteq C(\tau,\RR)$ for some $m \in \NN_+$ and $\epsilon > 0$ such that
$$\big\{(x,y)\in X\times X\,\big|\,\rho_{f_1,...,f_m}(x,y) < \epsilon\big\}\subseteq U$$
Clearly $\fU$ is a uniform structure on $X$. Suppose that $\cO \in \tau_{\fU}$. Then for each point $z$ in $\cO$ there exists $U$ in $\fU$ such that $B(z,U)\subseteq \cO$. By definition there exists a finite subset $\{f_1,...,f_m\}\subseteq C(\tau,\RR)$ for some $m \in \NN_+$ and $\epsilon > 0$ such that
$$\big\{(x,y)\in X\times X\,\big|\,\rho_{f_1,...,f_m}(x,y) < \epsilon\big\}\subseteq U$$
Thus
$$\bigcap_{i=1}^mf_i^{-1}\bigg(\big(f_i(z) - \epsilon, f_i(z) + \epsilon\big)\bigg) = \big\{y\in X \,\big|\, \rho_{f_1,...,f_m}(z,y) < \epsilon\big\} \subseteq B(z,U)\subseteq \cO$$
Since $f_1,...,f_m$ are continuous on $X$ with respect to $\tau$, we derive from the inclusion above that there exists an open neighborhood of $z$ with respect to $\tau$ contained in $\cO$. According to the fact that $z$ is an arbitrary point in $\cO$ it follows that $\cO \in \tau$. This proves that $\tau_{\fU}\subseteq \tau$. Now we prove the converse. For this assume that $\cO \in \tau$. We claim that $\cO$ is also open in the topology induced by $\fU$. For this pick $z \in \cO$. Since $(X,\tau)$ is completely regular, there exists a function $f_z:X\ra \RR$ continuous with respect to $\tau$ such that $f_z(X\setminus \cO)\subseteq \{1\}$ and $f_z(z) = 0$. Let $U_z$ be a set consisting of all pairs in $X\times X$ for which $\rho_{f_z}$ is smaller than $1$. Then $U_z\in \fU$ and obviously $B(z,U_z)\subseteq \cO$. Thus
$$\cO = \bigcup_{z\in \cO}B(z,U_z)$$
and this proves the claim that $\cO$ is in $\tau_{\fU}$. Hence $\tau \subseteq \tau_{\fU}$. This completes the proof of the first assertion.\\
The proof of the second assertion is left for the reader as an exercise. Note that in the proof one can use complete regularity of topologies induced by uniform structures.   
\end{proof}
\noindent
Next we prove the following important result.

\begin{proposition}\label{proposition:the_induced_topology_preserves_uniform_subspaces}
Let $j:(Z,\fO)\hookrightarrow \left(X,\fU\right)$ be an embedding of uniform spaces. Then $j:\left(Z,\tau_{\fO}\right)\ra \left(X,\tau_{\fU}\right)$ is an embedding of topological spaces.
\end{proposition}
\begin{proof}
Suppose that $\cO \in \tau_{\fO}$. Since $j$ is an embedding, for each $z\in \cO$ there exists $U_z \in \fU$ such that we have inclusion $B\big(z,\left(j\times j\right)^{-1}(U_z)\big)\subseteq \cO$. By Theorem \ref{theorem:each_entourage_contains_open_entourage} there exists $W_z\in \fU$ such that $W_z\subseteq U_z$ and $B\big(j(z),W_z\big) \in \tau_{\fU}$. Thus 
$$\widetilde{\cO} = \bigcup_{z \in \cO}B(j(z),W_z)$$
is an element of $\tau_{\fU}$. We have
$$j^{-1}\left(\widetilde{\cO}\right) = \bigcup_{z \in \cO}j^{-1}\left(B(j(z),W_z)\right) = \bigcup_{z\in \cO}B\big(z,\left(j\times j\right)^{-1}(W_z)\big)$$
From this equality we deduce that $\cO \subseteq j^{-1}\left(\widetilde{\cO}\right)$. Moreover, it follows that
$$j^{-1}\left(\widetilde{\cO}\right) = \bigcup_{z\in \cO}B\big(z,\left(j\times j\right)^{-1}(W_z)\big) \subseteq \bigcup_{z\in \cO}B\big(z,\left(j\times j\right)^{-1}(U_z)\big)\subseteq \cO$$
and hence $j^{-1}(\widetilde{\cO}) = \cO$. Therefore, each open subset $\cO \in \tau_{\fO}$ is the preimage under $j$ of some open subset $\cO \in \tau_{\fU}$. This completes the proof.
\end{proof}

\begin{theorem}\label{theorem:induced_topology_functor_preserves_limits_of_uniform_spaces}
The functor 
$$\Unif \ni (X,\fU) \mapsto (X,\tau_{\fU}) \in \Top$$
preserves small limits.
\end{theorem}
\begin{proof}
Let $\cI$ be a set and let $\left\{\left(X_i,\fU_i\right)\right\}_{i\in \cI}$ be a family of uniform spaces parametrized by $\cI$. Consider the cartesian product $X = \prod_{i\in \cI}X_i$ and let $pr_i:X\ra X_i$ be the projection for $i\in \cI$. Let $\fU$ be the smallest uniform structure on $X$ which makes $\big\{pr_i:\left(X,\fU\right) \ra \left(X_i,\fU_i\right)\big\}_{i\in \cI}$ into a family of morphisms of uniform spaces. By Theorem \ref{theorem:description_of_uniform_structure_introduced_by_a_family_of_maps} family $\fU$ consists of all surrounding $U$ of $\Delta_X$ such that there exist $n\in \NN_+$, $i_1,...,i_n\in \cI$ and $U_1 \in \fU_{i_1},...,U_n\in \fU_{i_n}$ satisfying
$$\bigcap_{k=1}^n\left(pr_{i_k}\times pr_{i_k}\right)^{-1}(U_k) \subseteq U$$
Note that we have
$$B\bigg(x, \bigcap_{k=1}^n\left(pr_{i_k}\times pr_{i_k}\right)^{-1}(U_k)\bigg) = \prod_{k=1}^nB\big(pr_{i_k}(x), U_k\big)\times \prod_{i\in \cI\setminus \{i_1,...,i_k\}}X_i$$
for every $x \in X$. By Theorem \ref{theorem:each_entourage_contains_open_entourage} there exist $W_1 \in \fU_{i_1},...,W_n \in \fU_{i_n}$ such that 
$$W_k\subseteq U_k,\,B\big(pr_{i_k}(x), W_k\big) \in \tau_{\fU_{i_k}}$$
for each $k$. Thus
$$\prod_{k=1}^nB\big(pr_{i_k}(x), W_k\big) \times \prod_{i\in \cI\setminus \{i_1,...,i_k\}}X_i \subseteq B\bigg(x, \bigcap_{k=1}^n\left(pr_{i_k}\times pr_{i_k}\right)^{-1}(U_k)\bigg) \subseteq B(x,U)$$
Therefore, each ball centered in some point $x$ of $X$ and with radius $U$ in $\fU$ contains open neighborhood of $x$ with respect to the product of topologies $\{\tau_{\fU_i}\}_{i\in  \cI}$. This implies that $\tau_{\fU}$ is contained in the product of topologies $\{\tau_{\fU_i}\}_{i\in  \cI}$. On the other hand the fact that $pr_i:(X,\tau_{\fU}) \ra (X_i,\tau_{\fU_i})$ is continuous for every $i\in \cI$ implies that the product of topologies $\{\tau_{\fU_i}\}_{i\in \cI}$ is contained in $\tau_{\fU}$. Thus $\tau_{\fU}$ is the product topology determined by $\{\tau_{\fU_i}\}_{i\in \cI}$. Hence $(X,\tau_{\fU})$ together with $\{pr_i\}_{i\in \cI}$ is a product of topological spaces $\{(X_i,\tau_{\fU_i})\}_{i\in \cI}$. By Theorem \ref{theorem:limits_of_uniform_spaces_description} it follows that
$$\Unif \ni (X,\fU) \mapsto (X,\tau_{\fU}) \in \Top$$
preserves small products. Since every small limit is a combination of small products and a kernel pair, it remains to show that the functor above preserves kernel pairs. Suppose that 
\begin{center}
\begin{tikzpicture}
[description/.style={fill=white,inner sep=2pt}]
\matrix (m) [matrix of math nodes, row sep=1em, column sep=2.5em,text height=1.5ex, text depth=0.25ex] 
{(Z,\fO) & \left(X_1,\fU_1\right)&  \left(X_1,\fU_2\right)  \\} ;
\path[right hook->,line width=0.8pt,font=\scriptsize]
(m-1-1) edge node[above] {$ j $} (m-1-2);
\path[->,line width=0.8pt,font=\scriptsize]
(m-1-2) edge[transform canvas={yshift=0.5ex}] node[above] {$ f_1 $} (m-1-3)
(m-1-2) edge[transform canvas={yshift=-0.5ex}] node[below] {$ f_2 $} (m-1-3);
\end{tikzpicture}
\end{center}
is a kernel pair of $f_1$ and $f_2$ in $\Unif$. Then Theorem \ref{theorem:limits_of_uniform_spaces_description} shows that $Z$ consists of all $x \in X_1$ such that $f_1(x) = f_2(x)$ and $j$ is an embedding of uniform spaces induced by the inclusion $Z \hookrightarrow X_1$. Now Proposition \ref{proposition:the_induced_topology_preserves_uniform_subspaces} implies that
\begin{center}
\begin{tikzpicture}
[description/.style={fill=white,inner sep=2pt}]
\matrix (m) [matrix of math nodes, row sep=3em, column sep=3em,text height=1.5ex, text depth=0.25ex] 
{(Z,\tau_{\fO}) & \left(X_1,\tau_{\fU_1}\right)&  \left(X_1,\tau_{\fU_2}\right)  \\} ;
\path[right hook->,line width=0.8pt,font=\scriptsize]
(m-1-1) edge node[above] {$  $} (m-1-2);
\path[->,line width=0.8pt,font=\scriptsize]
(m-1-2) edge[transform canvas={yshift=0.5ex}] node[above] {$ f_1 $} (m-1-3)
(m-1-2) edge[transform canvas={yshift=-0.5ex}] node[below] {$ f_2 $} (m-1-3);
\end{tikzpicture}
\end{center}
is a kernel pair in the category $\Top$. Therefore, the functor
$$\Unif \ni (X,\fU) \mapsto (X,\tau_{\fU}) \in \Top$$
preserves kernel pairs. The proof is complete.
\end{proof}

\section{Pseudometrizable uniform spaces}
 
\begin{definition}
A uniform space $(X,\fU)$ is \textit{pseudometrizable} (\textit{metrizable}) if there exists a pseudometric (metric) $\rho$ on $X$ such that $\fU$ is the uniform structure induced by $\rho$.
\end{definition}
\noindent
The result below shows that pseudometrizable Hausdorff uniform spaces are metric spaces. 

\begin{fact}\label{fact:Hausdorff_and_pseudometrizable_is_metrizable}
Let $(X,\fU)$ be a pseudometrizable uniform space. Then the following are equivalent.
\begin{enumerate}[label=\emph{\textbf{(\roman*)}}, leftmargin=*]
\item $(X,\fU)$ is a Hausdorff uniform space.
\item Every pseudometric $\rho$ on $X$ which induces $\fU$ is a metric.
\end{enumerate} 
\end{fact}
\begin{proof}
The result follows from equality
$$\bigcap_{U\in \fU}U =  \bigcap_{\epsilon \in \RR_+}\big\{(x,y)\in X\times X\,\big|\,\rho\left(x,y\right) < \epsilon\big\} = \big\{(x,y)\in X\times X\,\big|\,\rho\left(x,y\right) = 0\big\}$$
Details are left for the reader.
\end{proof}
\noindent
The next theorem characterizes pseudometrizable uniform spaces.

\begin{theorem}\label{theorem:characterization_of_pseudometrizable_uniform_spaces}
Let $(X,\fU)$ be a uniform space. The following assertions are equivalent.
\begin{enumerate}[label=\emph{\textbf{(\roman*)}}, leftmargin=*]
\item $(X,\fU)$ is a pseudometrizable uniform space.
\item There exists a sequence $\{U_n\}_{n\in \NN}$ of elements in $\fU$ such that the family
$$\bigg\{U\in \fD_{X}\,\bigg|\exists_{n\in \NN}\,U_n \subseteq U\bigg\}$$
coincides with $\fU$.
\end{enumerate}
\end{theorem}
\begin{proof}
For $\textbf{(i)}\Rightarrow \textbf{(ii)}$ observe that if $\rho$ is a pseudometric on $X$ such that 
$$\bigg\{U \in \fD_{X}\,\bigg|\,\mbox{ there exists }\epsilon>0\mbox{ such that for all }x,y\in X\mbox{ if }\rho(x,y) < \epsilon\mbox{ then }(x,y)\in U\bigg\}$$
coincides with $\fU$, then the sequence $\{U_n\}_{n\in \NN}$ given by formula
$$U_n = \bigg\{(x,y)\in X\times X\,\bigg|\,\rho(x,y) < \frac{1}{2^n}\bigg\}$$
satisfies \textbf{(ii)}.\\
Suppose now that \textbf{(ii)} holds. We define a sequence $\{V_n\}_{n\in \NN}$ of elements in $\fU$ by recursion. We set $V_0 = U_0$ and if $V_0,...,V_n$ are defined for some $n\in \NN$, then we pick an element $W$ of $\fU$ such that 
$$W\cdot W\cdot W \subseteq V_n$$
and define $V_{n+1} = W\cap U_{n+1}$. Note that $\{V_n\}_{n\in \NN}$ satisfies
$$V_{n+1}\cdot V_{n+1}\cdot V_{n+1} \subseteq V_n$$
for each $n\in \NN$. Moreover, we have
$$\fU = \bigg\{U\in \fD_{X}\,\bigg|\exists_{n\in \NN}\,V_n \subseteq U\bigg\}$$
By Theorem \ref{theorem:Weils_theorem_on_pseudometrics} there exists a pseudometric $\rho$ on $X$ such that
$$\bigg\{(x,y)\in X\times X\,\bigg|\,\rho(x,y)<\frac{1}{2^n}\bigg\} \subseteq V_n \subseteq \bigg\{(x,y)\in X\times X\,\bigg|\,\rho(x,y) \leq \frac{1}{2^n}\bigg\}$$
for every $n\in \NN$. This implies that
$$\bigg\{U \in \fD_{X}\,\bigg|\,\mbox{ there exists }\epsilon>0\mbox{ such that for all }x,y\in X\mbox{ if }\rho(x,y)\leq \epsilon\mbox{ then }(x,y)\in U\bigg\}$$
coincides with $\fU$. Hence $\textbf{(ii)}\Rightarrow \textbf{(i)}$.
\end{proof}

\begin{theorem}\label{theorem:uniform_space_is_subspace_of_product_of_pseudometrizable_spaces}
Let $(X,\fU)$ be a uniform space. Then there exists a family $\left(X_i,\fU_i\right)_{i\in I}$ of pseudometrizable uniform spaces and an embedding
$$\left(X,\fU\right)\hookrightarrow \prod_{i\in I}\left(X_i,\fU_i\right)$$
of uniform spaces. Moreover, if $\left(X,\fU\right)$ is Hausdorff, then $\left(X_i,\fU_i\right)$ can be chosen metrizable for every $i\in I$.
\end{theorem}
\begin{proof}
For each $U \in \fU$ pick a pseudometric $\rho_{U}$ with properties described as in Theorem \ref{theorem:pseudometric_associated_to_entourage_is_uniform}. Recall from Example \ref{example:uniform_structure_induced_by_pseudometric} that the uniform structure on $X$ induced by $\rho_U$ is dentoted by $\fU_{\rho_{U}}$. It follows from Theorem \ref{theorem:pseudometric_associated_to_entourage_is_uniform} that $\fU_{\rho_{U}}\subseteq \fU$ and hence the identity $1_X$ is a morphism of uniform spaces
$$\left(X,\fU\right)\ra \left(X,\fU_{\rho_{U}}\right)$$
We denote it by $j_U$. Now we prove that the morphism 
$$j = \langle j_U\rangle_{U\in \fU}:\left(X,\fU\right)\ra \prod_{U\in \fU}\left(X,\fU_{\rho_U}\right)$$
is an embedding. Clearly $j$ is injective (each $j_U$ is the identity on the underlying sets). Moreover, for each $U \in \fU$ we have
$$U = \left(j\times j\right)^{-1}\left(U\times \prod_{W\in \fU\setminus \{U\}}(X\times X)\right)$$
and hence $U$ is contained in the smallest uniform structure on $X$ which makes $j$ into a uniform morphism to $\prod_{U\in \fU}\left(X,\fU_{\rho_U}\right)$. Since $U$ is an arbitrary element of $\fU$, we derive that $j$ is an embedding. This proves the first part of the statement. Suppose in addition that $\left(X,\fU\right)$ is Hausdorff. For each $U$ in $\fU$ consider the quotient map $q_U:X\ra X_U$ of $X$ with respect to equivalence relation 
$$x\sim_U y\,\Leftrightarrow\,\rho_U(x,y) = 0$$
Then $\rho_U$ induces a metric $\tilde{\rho}_U$ on $X_U$. Let $\fU_{\tilde{\rho}_U}$ be a uniform structure induced by $\tilde{\rho}_U$ on $X_U$. Clearly $q_U$ is a morphism of uniform structures $\left(X,\fU_{\rho_U}\right)\ra \left(X_U,\fU_{\tilde{\rho}_U}\right)$ and for $\tilde{U} = \left(q_U\times q_U\right)(U) \in \fU_{\tilde{\rho}_U}$ we have $U = \left(q_U\times q_U\right)^{-1}\left(\tilde{U}\right)$. Now we define $\tilde{j}_U = q_U\cdot j_U$ for every $U \in \fU$ and 
$$\tilde{j} = \langle \tilde{j}_U\rangle_{U\in \fU}:\left(X,\fU\right)\ra \prod_{U\in \fU}\left(X_U,\fU_{\tilde{\rho}_U}\right)$$
Then $\tilde{j}$ is a morphism of uniform spaces. Suppose that $\tilde{j}(x) = \tilde{j}(y)$ for some $x,y\in X$. Then $\rho_U(x,y) = 0$ for each $U \in \fU$ and hence $(x,y) \in U$ for every $U\in \fU$. Since $(X,\fU)$ is Hausdorff, this implies that $x = y$. Hence $\tilde{j}$ is injective. Moreover, for each $U \in \fU$ we have
$$U = \left(j\times j\right)^{-1}\left(\tilde{U}\times \prod_{W\in \fU\setminus \{U\}}(X_W\times X_W)\right)$$
and hence $U$ is contained in the smallest uniform structure on $X$ which makes $\tilde{j}$ into a uniform morphism to $\prod_{U\in \fU}\left(X_U,\fU_{\tilde{\rho}_U}\right)$. Since $U$ is an arbitrary element of $\fU$, we derive that $\tilde{j}$ is an embedding.
\end{proof}

\section{Complete uniform spaces}
\noindent
In this section we study very important notion of completeness of uniform spaces. For this we use the notion of filter of subsets defined in \cite{Filters_in_topology}.

\begin{definition}
Let $(X,\fU)$ be a uniform space. Suppose that $\cF$ is a proper filter of subsets of $X$. Assume that for every $U$ in $\fU$ there exists $F \in \cF$ such that $F\times F\subseteq U$. Then $\cF$ is \textit{a Cauchy filter in $\left(X,\fU\right)$}.
\end{definition}
\noindent
First we prove that the image of a Cauchy filter under a morphism of uniform spaces is a Cauchy filter.

\begin{fact}\label{fact:images_of_Cauchy_filters_under_uniform_morphisms_are_Cauchy}
Let $f:\left(X,\fU\right)\ra \left(Y,\fV\right)$ be a morphism of uniform spaces. If $\cF$ is a Cauchy filter in $\left(X,\fU\right)$, then $f(\cF)$ is a Cauchy filter in $\left(Y,\fV\right)$.
\end{fact}
\begin{proof}
Pick $V \in \fV$. Since $\cF$ is a Cauchy filter in $\left(X,\fU\right)$ and $f$ is a morphism of uniform spaces, there exists $F \in \cF$ such that $$F\times F \subseteq (f\times f)^{-1}(V)$$
Then
$$f(F)\times f(F)\subseteq V$$
Since $f(F) \in f(\cF)$ and $V$ is an arbitrary element of $\fV$, this proves the assertion.
\end{proof}

\begin{definition}
Let $(X,\fU)$ be a uniform space. Suppose that every Cauchy filter in in $\left(X,\fU\right)$ is convergent with respect to $\tau_{\fU}$. Then $\left(X,\fU\right)$ is \textit{a complete uniform space}. 
\end{definition}
\noindent
The following theorem is analogical to famous Tychonoff's theorem for compact topological spaces.

\begin{theorem}\label{theorem:completeness_of_factors_imply_product_completeness}
Let $\left(X_i,\fU_i\right)$ be complete uniform spaces for $i \in I$. Then the product
$$\prod_{i\in I}\left(X_i,\fU_i\right)$$
is a complete uniform space.
\end{theorem}
\begin{proof}
Let $X = \prod_{i\in I}X_i$ and let $\fU$ be the product uniform structure of $\fU_i$ for $i\in I$. For each $i\in I$ we denote by $pr_i:X\ra X_i$ the canonical projection. Suppose that $\left(X_i,\fU_i\right)$ is a complete uniform space for every $i\in I$. Fix a Cauchy filter $\cF$ in $\left(X,\fU\right)$. Then $\cF_i = pr_i(\cF)$ is a Cauchy filter on $\left(X_i,\fU_i\right)$ for every $i\in I$ according to Fact \ref{fact:images_of_Cauchy_filters_under_uniform_morphisms_are_Cauchy}. Since $\left(X_i,\fU_i\right)$ is a complete uniform space and $\cF_i$ is a Cauchy filter, we derive that $\cF_i$ is convergent to some point $x_i \in X_i$ with respect to $\tau_{\fU_i}$ for each $i\in I$. Let $x$ be a point in $X$ such that $pr_i(x) = x_i$ for every $i \in I$. Then $\cF$ is convergent to $x$ with respect to the product of topologies $\tau_{\fU_i}$ for $i\in I$. By Theorem \ref{theorem:induced_topology_functor_preserves_limits_of_uniform_spaces} we infer that $\tau_{\fU}$ is the product of $\tau_{\fU_i}$ for $i\in I$. Hence $\cF$ is convergent to $x$ with respect to $\tau_{\fU}$. Thus $(X,\fU) = \prod_{i\in I}\left(X_i,\fU_i\right)$ is a complete uniform space.
\end{proof}

\begin{theorem}\label{theorem:completeness_of_product_and_nonemptiness_of_factors_imply_their_completeness}
Let $\left(X_i,\fU_i\right)$ be nonempty uniform spaces for $i \in I$. If
$$\prod_{i\in I}\left(X_i,\fU_i\right)$$
is a complete uniform space, then $\left(X_i,\fU_i\right)_{i\in I}$ is complete for every $i \in I$.
\end{theorem}
\begin{proof}
Denote $X = \prod_{i\in I}X_i$ and let $\fU$ be the product uniform structure induced by $\fU_i$ for all $i\in I$. For each $i\in I$ we denote by $pr_i:X\ra X_i$ the canonical projection. Assume that $(X,\fU)$ is a complete uniform space. Fix $i_0 \in I$. Suppose that $\cF$ is a Cauchy filter in $(X_{i_0},\fU_{i_0})$. Since $X_i\neq \emptyset$ for every $i\in I$, we may pick $z \in \prod_{i\neq i_0}X_i$. Consider a filter 
$$\tilde{\cF} = \big\{\tilde{F}\subseteq X\,\big|\,F\times \{z\} \subseteq \tilde{F}\mbox{ for some }F\in \cF\big\}$$
Then $\tilde{\cF}$ is a Cauchy filter in $(X,\fU)$ and $\cF = pr_{i_0}\left(\tilde{\cF}\right)$. Since $\tilde{\cF}$ is a Cauchy filter in a complete uniform space $(X,\fU)$, it is convergent with respect to $\tau_{\fU}$ to some $x \in X$. Since $\tau_{\fU}$ is a product of topologies $\tau_{\fU_i}$ for $i \in I$ according to Theorem \ref{theorem:induced_topology_functor_preserves_limits_of_uniform_spaces}, we derive that $pr_{i_0}\left(\tilde{\cF}\right)$ is convergent to $pr_{i_0}(x)$ with respect to $\tau_{\fU_{i_0}}$. Finally according to $\cF = pr_{i_0}\left(\tilde{\cF}\right)$ we derive that $\cF$ is convergent to $pr_{i_0}(x)$. This shows that $(X_{i_0},\fU_{i_0})$ is a complete uniform space. Since $i_0 \in I$ is arbitrary, we derive that $\left(X_i,\fU_i\right)$ is a complete uniform space for every $i \in I$.
\end{proof}
\noindent
Now we study completeness under embeddings of uniform spaces.

\begin{theorem}\label{theorem:closed_embeddings_into_complete_space_are_complete}
Let $j:\left(Z,\fO \right) \hookrightarrow \left(X,\fU\right)$ be an embedding of uniform spaces. If $\left(X,\fU\right)$ is a complete uniform space and $j(Z)$ is closed with respect to $\tau_{\fU}$, then $\left(Z,\fO\right)$ is a complete uniform space.
\end{theorem}
\begin{proof}
Pick a Cauchy filter $\cF$ in $\left(Z,\fO\right)$. Then according to Fact \ref{fact:images_of_Cauchy_filters_under_uniform_morphisms_are_Cauchy} the filter $j(\cF)$ is Cauchy in $\left(X,\fU\right)$. Thus $j(\cF)$ converges to some point $x\in X$ with respect to $\tau_{\fU}$. Since $j(Z)$ is closed in $\tau_{\fU}$, we infer that $x \in j(Z)$. Thus $x = j(z)$ for some $z\in Z$. According to Proposition \ref{proposition:the_induced_topology_preserves_uniform_subspaces} the map $j:\left(Z,\tau_{\fO}\right) \hookrightarrow \left(X,\tau_{\fU}\right)$ is a topological embedding and $j(\cF)$ is convergent to $j(z)$ with respect to $\tau_{\fU}$. By {\cite[Proposition 3.3]{Filters_in_topology}} we infer that $\cF$ is convergent to $z$ with respect to $\tau_{\fO}$. Hence $\left(Z,\fO\right)$ is a complete uniform space.
\end{proof}

\begin{theorem}\label{theorem:embeddings_of_complete_space_into_Hausdorff_space_are_closed}
Let $j:\left(Z,\fO \right) \hookrightarrow \left(X,\fU\right)$ be an embedding of uniform spaces. If $\left(X,\fU\right)$ is a Hausdorff uniform space and $\left(Z,\fO\right)$ is a complete uniform space, then $j(Z)$ is closed with respect to $\tau_{\fU}$.
\end{theorem}
\begin{proof}
Fix a point $x$ of $X$ in the closure of $j(Z)$ with respect to $\tau_{\fU}$. Consider the filter
$$\cF = \big\{F\subseteq Z\,\big|\,j^{-1}(\cO)\subseteq F\mbox{ for some open neighborhood }\cO\mbox{ of }x\mbox{ with respect to }\tau_{\fU}\big\}$$
Clearly $\cF$ is a proper filter of subsets of $Z$. Since $j$ is an embedding of uniform spaces, every entourage of the diagonal in $(Z,\fO)$ is of the form $\left(j\times j\right)^{-1}(U)$ for some $U$ in $\fU$. Fix some $U$ in $\fU$ and let $W$ be an entourage of the diagonal in $(X,\fU)$ such that $W\cdot W\subseteq U$ and $B(x,W) \in \tau_{\fU}$. Such entourage of the diagonal in $(X,\fU)$ exists according to Theorem \ref{theorem:each_entourage_contains_open_entourage}. Note that
$$B(x,W) \times B(x,W) \subseteq U$$
Now we have 
$$j^{-1}\big(B(x,W)\big)\times j^{-1}\big(B(x,W)\big) \subseteq \left(j\times j\right)^{-1}(U)$$
and hence $\cF$ is a Cauchy filter in $\left(Z,\fO\right)$. Since $(Z,\fO)$ is complete, filter $\cF$ is convergent to some $z$ in $Z$ with respect to $\tau_{\fO}$. Hence $j(\cF)$ is convergent to $j(z)$ with respect to $\tau_{\fU}$. Note that
$$\big\{\cO \in \tau_{\fU}\,\big|\,x\in \cO \big\} \subseteq j(\cF)$$
This implies that filter $j(\cF)$ is convergent to $x$ with respect to $\tau_{\fU}$. Thus $j(\cF)$ is convergent to both $j(z)$ and $x$ with respect to $\tau_{\fU}$. Uniform space $(X,\fU)$ is Hausdorff and hence $(X,\tau_{\fU})$ is Hausdorff topological space (Theorem \ref{theorem:image_of_the_canonical_functor_is_completely_regular_space}). This implies that each filter of subsets of $X$ has at most one limit with respect to $\tau_{\fU}$. Thus $x = j(z)$ and $j(Z)$ is closed with respect to $\tau_{\fU}$.
\end{proof}
\noindent
Finally we discuss extensions of uniform morphisms defined on dense uniform subspaces and with values in complete uniform spaces.

\begin{theorem}\label{theorem:extensions_of_uniform_morphisms_to_complete_spaces}
Let $j:\left(Z,\fO \right) \hookrightarrow \left(X,\fU\right)$ be an embedding of uniform spaces and let $f:\left(Z,\fO\right)\ra \left(Y,\fV\right)$ be a uniform morphism. If $\left(Y,\fV\right)$ is a complete uniform space and $j(Z)$ is dense in $X$ with respect to $\tau_{\fU}$, then there exists a uniform morphism $\tilde{f}:\left(X,\fU\right)\ra \left(Y,\fV\right)$ such that $\tilde{f}\cdot j = f$. Moreover, if $\left(Y,\fV\right)$ is Hausdorff, then $\tilde{f}$ is unique.
\end{theorem}
\begin{proof}
For each point $x$ in $X$ define a filter of subsets of $Z$ by formula
$$\cF_x = \big\{F\subseteq Z\,\big|\,j^{-1}(\cO)\subseteq F\mbox{ for some open neighborhood }\cO\mbox{ of }x\mbox{ with respect to }\tau_{\fU}\big\}$$
Since $j(Z)$ is dense in $X$ with respect to $\tau_{\fU}$, the filter $\cF_x$ is proper. Fix some $U$ in $\fU$ and let $W$ be an entourage of the diagonal in $(X,\fU)$ such that $W\cdot W\subseteq U$ and $B(x,W) \in \tau_{\fU}$. Such entourage of the diagonal in $(X,\fU)$ exists according to Theorem \ref{theorem:each_entourage_contains_open_entourage}. Note that
$$B(x,W) \times B(x,W) \subseteq U$$
Now we have 
$$j^{-1}\big(B(x,W)\big)\times j^{-1}\big(B(x,W)\big) \subseteq \left(j\times j\right)^{-1}(U)$$
and by definition $j^{-1}\big(B(x,W)\big) \in \cF_x$. Since $j$ is an embedding of uniform spaces, every entourage of the diagonal in $(Z,\fO)$ is of the form $\left(j\times j\right)^{-1}(U)$ for some $U$ in $\fU$. Hence $\cF_x$ is a Cauchy filter in $\left(Z,\fO\right)$. By Fact \ref{fact:images_of_Cauchy_filters_under_uniform_morphisms_are_Cauchy} the filter $f(\cF_x)$ is Cauchy in $\left(Y,\fV\right)$. Since $\left(Y,\fV\right)$ is a complete uniform space, we derive that $f(\cF_x)$ is convergent to some point in $Y$. If $z\in Z$, then according to Proposition \ref{proposition:the_induced_topology_preserves_uniform_subspaces} filter $\cF_{j(z)}$ contains all open neighborhoods of $z$ with respect to $\tau_{\fU}$ and hence $\cF_{j(z)}$ is convergent to $z$ with respect to $\tau_{\fO}$. Hence $f(\cF_{j(z)})$ is convergent to $f(z)$ with respect to $\tau_{\fV}$. We define $\tilde{f}:X\ra Y$ in such a way that $f(\cF_x)$ converges to $\tilde{f}(x)$ with respect to $\tau_{\fV}$ for every $x\in X$ and if $x = j(z)$ for some $z\in Z$, then $\tilde{f}(x) = f(z)$. These two conditions can be simultaneously satisfied and give rise to a map $\tilde{f}:X\ra Y$ such that $\tilde{f}\cdot j = f$.\\
We claim that $\tilde{f}$ is a uniform morphism $\left(X,\fU\right)\ra \left(Y,\fV\right)$. Fix an entourage of the diagonal $V$ in $(Y,\fV)$. Let $E$ be an entourage of the diagonal in $(Y,\fV)$ such that $E\cdot E\cdot E \subseteq V$ and $B(y,E) \in \tau_{\fV}$ for every $y$ in $Y$. Such entourage of the diagonal in $(Y,\fV)$ exists according to Theorem \ref{theorem:each_entourage_contains_open_entourage}. Since $f:\left(Z,\fO\right)\ra \left(Y,\fV\right)$ is a uniform morphism, $\left(f\times f\right)^{-1}(E)$ is an element of $\fO$. Since $j$ is an embedding of uniform spaces, there exists $U \in \fU$ such that $\left(j\times j\right)^{-1}(U) = \left(f\times f\right)^{-1}(E)$. Let $W$ be an entourage of the diagonal in $(X,\fU)$ such that $W\cdot W\cdot W \subseteq U$ and $B(x,W) \in \tau_{\fU}$ for every $x$ in $X$. Again such entourage of the diagonal in $(X,\fU)$ exists according to Theorem \ref{theorem:each_entourage_contains_open_entourage}. Now pick $(x_1,x_2) \in W$. Then 
$$\cO = B(x_1,W)\cap B(x_2,W)$$
is an open neighborhood of both $x_1$ and $x_2$ such that $\cO\times \cO \subseteq W\cdot W \cdot W \subseteq U$. Thus 
$$f\big(j^{-1}(\cO)\big) \times f\big(j^{-1}(\cO)\big) \subseteq \left(f\times f\right)\left(\left(j\times j\right)^{-1}(U)\right) \subseteq E$$
and hence $(y_1,y_2)\in E$ for every pair $y_1,y_2$ of elements of $f\big(j^{-1}(\cO)\big)$. Moreover, $j^{-1}(\cO)\in \cF_{x_1}\cap \cF_{x_2}$ and hence the closure of $f\big(j^{-1}(\cO)\big)$ with respect to $\tau_{\fV}$ contains both $\tilde{f}(x_1)$ and $\tilde{f}(x_2)$. Thus $B\left(\tilde{f}(x_1),E\right)\cap f\big(j^{-1}(\cO)\big) \neq \emptyset$ and
$B\left(\tilde{f}(x_2),E\right)\cap f\big(j^{-1}(\cO)\big) \neq \emptyset$. Hence there exist $y_1,y_2\in f\big(j^{-1}(\cO)\big)$ such that $\left(\tilde{f}(x_1),y_1\right)\in E$ and $\left(\tilde{f}(x_2),y_2\right)\in E$. Since $(y_1,y_2)\in E$, we derive that
$$\left(\tilde{f}(x_1),\tilde{f}(x_1)\right)\in E\cdot E\cdot E\subseteq V$$
This implies that $W\subseteq \left(\tilde{f}\times \tilde{f}\right)^{-1}(V)$ and hence $\left(\tilde{f}\times \tilde{f}\right)^{-1}(V) \in \fU$. This completes the proof of the claim.\\
Finally it remains to prove that if $\left(Y,\fV\right)$ is Hausdorff, then $\tilde{f}$ is unique. For this note that $\tilde{f}:\left(X,\tau_{\fU}\right)\ra \left(Y,\tau_{\fV}\right)$ is a continuous map into a Hausdorff space such that $\tilde{f}\cdot j = f$. Since $j:\left(Z,\tau_{\fO}\right)\hookrightarrow \left(X,\tau_{\fU}\right)$ is an embedding of a dense subspace and $f:\left(Z,\tau_{\fO}\right)\ra \left(Y,\tau_{\fV}\right)$ is continuous, we derive that $\tilde{f}$ is a unique continuous extension of $f$.
\end{proof}

\section{Completeness for pseudometric spaces}
\noindent
In this section we discuss the notion of completeness in the important special case of pseudometrizable uniform spaces.

\begin{definition}
Let $(X,\rho)$ be a pseudometric space and let $\{x_n\}_{n\in \NN}$ be a sequence of elements of $X$. If for every $\epsilon > 0$ there exists $N\in \NN$ such that for all $n,m\geq N$ we have
$$\rho(x_n,x_m) \leq \epsilon$$
then $\{x_n\}_{n\in \NN}$ is \textit{a Cauchy sequence in $(X,\rho)$}.
\end{definition}

\begin{definition}
Let $(X,\rho)$ be a pseudometric space. Suppose that every Cauchy sequence in $(X,\rho)$ is convergent with respect to the topology induced by $\rho$. Then $(X,\rho)$ is \textit{a complete pseudometric space}. 
\end{definition}

\begin{theorem}\label{theorem:complete_pseudometric_spaces}
Let $(X,\rho)$ be a pseudometric space. Then the following conditions are equivalent.
\begin{enumerate}[label=\emph{\textbf{(\roman*)}}, leftmargin=*]
\item $(X,\rho)$ is a complete pseudometric space. 
\item $(X,\fU_{\rho})$ is a complete uniform space.
\end{enumerate}
\end{theorem}
\begin{proof}
We define 
$$U_{n} = \bigg\{(x,y)\in X\times X\,\bigg|\,\rho(x,y) < \frac{1}{n + 1}\bigg\}$$
for every $n\in \NN_+$. Moreover, $\fU_{\rho}$ is the uniform structure consisting of all $U \in \fD_X$ such that $U_n\subseteq U$ for some $n\in \NN_+$. Indeed, this follows from Example \ref{example:uniform_structure_induced_by_pseudometric}.\\
Now assume that $(X,\rho)$ is a complete pseudometric space. Let $\cF$ be a Cauchy filter in $\left(X,\fU_{\rho}\right)$. We construct a sequence $\{F_n\}_{n\in \NN}$ of elements of $\cF$ by recursion. We set $F_0 = X$. Suppose that $F_0,...,F_n$ are constructed for some $n\in \NN$. We pick $F_{n+1}\in \cF$ such that 
$$F_{n+1}\times F_{n+1} \subseteq U_{n+1}$$
and $F_{n+1} \subseteq F_n$. Note that sequence $\{F_n\}_{n\in \NN}$ is nonincreasing and
$$F_{n}\times F_{n} \subseteq U_n$$
for every $n\in \NN_+$. Now pick $x_n\in F_n$ for every $n\in \NN$. Then according to properties of $\{F_n\}_{n\in \NN}$ sequence $\{x_n\}_{n\in \NN}$ is a Cauchy sequence in $(X,\rho)$. Since $(X,\rho)$ is complete, $\{x_n\}_{n\in \NN}$ is convergent to some point $x$ with respect to the topology induced by $\rho$. We claim that $\cF$ is convergent to $x$ with respect to $\tau_{\fU_{\rho}}$. By Remark \ref{remark:topology_induced_by_pseudometric_open_balls_basis} sets $B_{\rho}(x,\epsilon)$ for $\epsilon \in \RR_+$ form a local basis of open neighborhoods of $x$ with respect to $\tau_{\fU_{\rho}}$. Pick positive real $\epsilon$. Suppose now that $n\in \NN_+$ satisfies two conditions
$$\frac{1}{n+1} < \frac{\epsilon}{2},\,\rho(x_n,x) < \frac{\epsilon}{2}$$
Then $F_{n} \subseteq B_{\rho}(x,\epsilon)$. This implies that all open neighborhoods of $x$ with respect to $\tau_{\fU_{\rho}}$ are contained in $\cF$. Thus $\cF$ is convergent to $x$ with respect to $\tau_{\fU_{\rho}}$. This proves that $\textbf{(i)}\Rightarrow \textbf{(ii)}$.\\
Next assume that $(X,\fU_{\rho})$ is a complete uniform space. Pick a Cauchy sequence $\{x_n\}_{n\in \NN}$ in $(X,\rho)$. Define
$$F_n = \big\{x_k\,\big|\,k\geq n\big\}$$
for every $n\in \NN$ and
$$\cF = \big\{F\subseteq X\,\big|\,F_n\subseteq F\mbox{ for some }n\in \NN\big\}$$
Since $(X,\fU_{\rho})$ is complete, the filter $\cF$ is convergent to some point $x$ in $X$ with respect to $\tau_{\fU_{\rho}}$. Therefore, for every open neighborhood $\cO$ of $x$ with respect to $\tau_{\fU_{\rho}}$ there exists $n\in \NN$ such that $F_n \subseteq \cO$. Hence the sequence $\{x_n\}_{n\in \NN}$ is convergent to $x$ with respect to $\tau_{\fU_{\rho}}$. Hence the implication $\textbf{(ii)}\Rightarrow \textbf{(i)}$ holds.
\end{proof}
\noindent
The following theorem and its proof is due to Felix Hausdorff.

\begin{theorem}\label{theorem:completion_of_a_pseudometric_space}
Let $(X,\rho)$ be a pseudometric space. Then there exists a complete pseudometric space $(\ol{X},\ol{\rho})$ and an injective map $j:X\ra \ol{X}$ such that $j(X)$ is dense in $\ol{X}$ with respect to topology induced by $\ol{\rho}$ and
$$\rho(x_1,x_2) = \ol{\rho}(j(x_1),j(x_2))$$
for every $x_1,x_2\in X$.
\end{theorem}
\begin{proof}
Let $\ol{X}$ be the set of all Cauchy sequences in $(X,\rho)$. For each $\bd{x}$ in $\ol{X}$ and $n\in \NN$ we denote by $\bd{x}(n)$ the $n$-th element of $\bd{x}$. Pick two sequences $\bd{x}$ and $\bd{y}$ in $\ol{X}$. Then
$$|\rho(\bd{x}(n),\bd{y}(n)) - \rho(\bd{x}(m),\bd{y}(m))|\leq \rho(\bd{x}(n),\bd{x}(m)) + \rho(\bd{y}(n),\bd{y}(m))$$
by the triangle inequality. Thus the sequence of real numbers $\big\{\rho(\bd{x}(n),\bd{y}(n))\big\}_{n\in \NN}$ is Cauchy sequence in $\RR$ with respect to the metric induced by the absolute value. Hence the sequence $\big\{\rho(\bd{x}(n),\bd{y}(n))\big\}_{n\in \NN}$ is convergent. We define $\ol{\rho}:\ol{X}\times \ol{X}\ra \RR$ by formula 
$$\ol{\rho}\left(\bd{x}, \bd{y}\right) = \lim_{n\ra +\infty}\rho(\bd{x}(n),\bd{y}(n))$$
for each $\bd{x}$ and $\bd{y}$ in $\ol{X}$. It is clear that $\left(\ol{X},\ol{\rho}\right)$ is a pseudometric space. For each $x$ in $X$ let $j(x) \in \ol{X}$ be the constant sequence $j(x)(n) = x$ for every $n\in \NN$. This gives rise to an injective map $j:X\ra \ol{X}$. Note that $\ol{\rho}\left(j(x_1),j(x_2)\right) = \rho(x_1, x_2)$ for every $x_1,x_2\in X$. Moreover, if $\bd{x}$ is an element of $\ol{X}$, then 
$$\lim_{n\ra +\infty}\ol{\rho}\big(j(\bd{x}(n)),\bd{x}\big) = 0$$
Thus $j(X)$ is dense in $\ol{X}$ with respect to topology induced by $\ol{\rho}$. It remains to prove that $\left(\ol{X},\ol{\rho}\right)$ is complete. Let $\{\bd{x}_n\}_{n\in \NN}$ be a Cauchy sequence in $\left(\ol{X},\ol{\rho}\right)$. Since $j(X)$ is dense in $\ol{X}$ with respect to topology induced by $\ol{\rho}$, we derive that for each $n\in \NN$ there exists $x_n \in X$ such that 
$$\ol{\rho}\left(j(x_n), \bd{x}_n\right) \leq \frac{1}{2^n}$$
Then $\{j(x_n)\}_{n\in \NN}$ is a Cauchy sequence in $\left(\ol{X},\ol{\rho}\right)$ and hence $\{x_n\}_{n\in \NN}$ is a Cauchy sequence in $\left(X,\rho\right)$. We denote $\{x_n\}_{n\in \NN}$ by $\bd{x}$. We have
$$\lim_{n\ra +\infty}\ol{\rho}\left(j(x_n),\bd{x}\right) = 0$$
and hence
$$\lim_{n\ra +\infty}\ol{\rho}\left(\bd{x}_n,\bd{x}\right) = 0$$
This implies that $\{\bd{x}_n\}_{n\in \NN}$ is convergent to $\bd{x}$ with respect to topology induced by $\ol{\rho}$. Thus $\left(\ol{X},\ol{\rho}\right)$ is complete.   
\end{proof}

\begin{corollary}\label{corollary:completion_of_metric_spaces}
Let $(X,\rho)$ be a metric space. Then there exists a complete metric space $(\ol{X},\ol{\rho})$ and a map $j:X\ra \ol{X}$ such that $j(X)$ is dense in $\ol{X}$ with respect to topology induced by $\ol{\rho}$ and
$$\rho(x_1,x_2) = \ol{\rho}(j(x_1),j(x_2))$$
for every $x_1,x_2\in X$.
\end{corollary}
\begin{proof}
According to Theorem \ref{theorem:completion_of_a_pseudometric_space} there exists a complete pseudometric space $\left(\tilde{X},\tilde{\rho}\right)$ and a map $\tilde{j}:X\ra \tilde{X}$ such that $j(X)$ is dense in $\tilde{X}$ with respect to topology induced by $\tilde{\rho}$ and
$$\rho(x_1,x_2) = \tilde{\rho}(j(x_1),j(x_2))$$
for every $x_1,x_2\in X$. Let $\ol{X}$ be the quotient set of $X$ with respect to equivalence relation $\simeq$ given by 
$$\bd{x}_1\simeq \bd{x}_2\,\Leftrightarrow\,\tilde{\rho}(\bd{x}_1,\bd{x}_2) = 0$$
and let $q:\tilde{X}\ra \ol{X}$ be the quotient map. Define a pseudometric $\ol{\rho}$ on $\ol{X}$ by formula
$$\ol{\rho}\left(q(\bd{x}_1),q(\bd{x}_2)\right) = \tilde{\rho}(\bd{x}_1,\bd{x}_2)$$
for all $\bd{x}_1,\bd{x}_2\in \tilde{X}$. Then $\ol{\rho}$ is a metric on $\ol{X}$ and $\left(\ol{X},\ol{\rho}\right)$ is complete. We define $j = q\cdot \tilde{j}$. Then $j:X\ra \ol{X}$ satisfies$$\rho(x_1,x_2) = \ol{\rho}(j(x_1),j(x_2))$$
for every $x_1,x_2\in X$ and $j(X)$ is dense in $\ol{X}$ with respect to the topology induced by $\ol{\rho}$.
\end{proof}






































\small
\bibliographystyle{apalike}
\bibliography{../zzz}

\end{document}