\input ../pree.tex

\begin{document}

\title{Uniform spaces}
\date{}
\maketitle

\section{Introduction}
\noindent
These notes are devoted to uniform spaces. In the first section we prove important result on existence of pseudometrics originally due to Weil. This result is crucial for further developments.

\section{Existence of pseudometrics}
\noindent
Let $X$ be a set. We denote 
$$\Delta_X = \big\{(x,x)\in X\,\big|\,x\in X\big\}$$
For subsets $V,W$ of $X\times X$ we denote
$$W\cdot V = \big\{(x,z)\in X\times X\,\big|\,(x,y)\in V\mbox{ and }(y,z)\in W\mbox{ for some }y\in X\big\}$$
Now we are ready to introduce basic notions.

\begin{definition}
Let $X$ be a set. Suppose that $V$ is a subset of $X\times X$ satisfying the following assertions.
\begin{enumerate}[label=\textbf{(\arabic*)}, leftmargin=*]
\item If $(x,y)\in V$ for some $x,y\in X$, then $(y,x)\in V$.
\item $V$ contains $\Delta_X$.
\end{enumerate}
Then $V$ \textit{is a surrounding of $\Delta_X$}.
\end{definition}
\noindent
Finally we recall the notion of pseudometric.

\begin{definition}
Let $X$ be a set. Suppose that a function $\rho:X\times X\ra \RR$ satisfies the following assertions.
\begin{enumerate}[label=\textbf{(\arabic*)}, leftmargin=*]
\item $\rho(x,y) \geq 0$ for all $x,y\in X$.
\item $\rho(x,x) = 0$ for all $x\in X$.
\item $\rho(x,y) = \rho(y,x)$ for all $x,y\in X$.
\item $\rho(x,z)\leq \rho(x,y) + \rho(y,z)$ for all $x,y,z\in X$.
\end{enumerate}
Then $\rho$ is \textit{a pseudometric on $X$}.
\end{definition}

\begin{definition}
Let $\rho$ be a pseudometric on $X$. Suppose that $\rho(x,y) = 0$ implies $x = y$ for all $x,y\in X$. Then $\rho$ is \textit{a metric on $X$}.
\end{definition}
\noindent
Now we state and prove the fundamental result on the existence of pseudometrics.

\begin{theorem}\label{theorem:Weils_theorem_on_pseudometrics}
Let $X$ be a set and let $\{V_n\}_{n\in \NN}$ be a sequence of surroundings of $\Delta_X$ such that
$$V_{n+1}\cdot V_{n+1}\cdot V_{n+1} \subseteq V_n$$
for every $n\in \NN$. Then there exists a pseudometric $\rho$ on $X$ bounded by $1$ such that
$$\bigg\{(x,y)\in X\times X\,\bigg|\,\rho(x,y)<\frac{1}{2^n}\bigg\} \subseteq V_n \subseteq \bigg\{(x,y)\in X\times X\,\bigg|\,\rho(x,y) \leq \frac{1}{2^n}\bigg\}$$
for every $n\in \NN$.
\end{theorem}
\noindent
For the proof consider a function $f$ defined on $X\times X$ given by formula 
$$\begin{cases}
0 & \mbox{ if }(x,y)\in V_n\mbox{ for each }n\in \NN\\
\frac{1}{2^n} & \mbox{ if }(x,y)\in V_n\setminus V_{n+1}\\
1 & \mbox{ if }(x,y)\not \in V_0
\end{cases}$$
The proof relies on the following result. 

\begin{lemma}\label{lemma:inclusion_of_surroundings_for_pseudometric}
For each $n \in \NN$ and every finite sequence $x_0,...,x_m$ the inequality
$$\sum_{i=1}^mf(x_{i-1},x_i) < \frac{1}{2^n}$$
implies that $(x_0,x_m) \in V_n$.
\end{lemma}
\begin{proof}[Proof of the lemma]
The proof goes by induction on $m$. For $m = 0$ and $m = 1$ the claim is trivial. Assume that $m$ is greater than one and suppose that the assertion holds for all numbers smaller than $m$. Suppose that
$$\sum_{i=1}^mf(x_{i-1},x_i) < \frac{1}{2^n}$$
for some sequence $x_0,...,x_m$ of elements in $X$. We have
$$\mbox{ either }f(x_0,x_1) < \frac{1}{2^{n+1}}\mbox{ or }f(x_{m-1},x_m) < \frac{1}{2^{n+1}}$$
Without loss of generality we may assume that the first inequality holds. Let $k$ be the greatest number in $\{1,...,m-1\}$ such that
$$\sum_{i=1}^kf(x_{i-1},x_i) < \frac{1}{2^{n+1}}$$
Next we consider two cases.
\begin{itemize}
\item If $k < m-1$, then we have
$$\sum_{i=1}^{k}f(x_{i-1},x_i) < \frac{1}{2^{n+1}},\,f(x_k,x_{k+1}) \leq \frac{1}{2^{n+1}},\,\sum_{i=k+1}^mf(x_{i-1},x_i) < \frac{1}{2^{n+1}}$$
By induction hypothesis we have $(x_0,x_k)\in V_{n+1},(x_{k+1},x_m)\in V_{n+1}$ and by definition of $f$ we have $(x_k,x_{k+1})\in V_{n+1}$. Hence
$$(x_0,x_m) \in V_{n+1}\cdot V_{n+1}\cdot V_{n+1}\subseteq V_n$$
and the assertion holds.
\item If $k = m-1$. Then 
$$\sum_{i=1}^{m-1}f(x_{i-1},x_i) < \frac{1}{2^{n+1}},\,f(x_{m-1},x_m)\leq \frac{1}{2^{n+1}}$$
By induction hypothesis we have $(x_0,x_{m-1})\in V_{n+1}$ and by definition of $f$ we have $(x_{m-1},x_{m})\in V_{n+1}$. Hence
$$(x_0,x_{m}) \in V_{n+1}\cdot V_{n+1} \subseteq V_{n+1}\cdot V_{n+1}\cdot V_{n+1} \subseteq V_n$$
and the assertion holds.
\end{itemize}
Thus the result follows from induction.
\end{proof}

\begin{proof}[Proof of the theorem]
For $x,y\in X$ we define
$$\rho(x,y) = \inf \bigg\{\sum_{i=1}^mf(x_{i-1},x_i)\,\bigg|\,\mbox{ for every }m\in \NN\mbox{ an every finite sequence }x_0,...,x_m\mbox{ such that }x_0 = x,\,x_m = y\bigg\}$$
It is easy to verify that the function $\rho$ is a pseudometric on $X$. It remains to prove that
$$\bigg\{(x,y)\in X\times X\,\bigg|\,\rho(x,y)<\frac{1}{2^n}\bigg\} \subseteq V_n \subseteq \bigg\{(x,y)\in X\times X\,\bigg|\,\rho(x,y) \leq \frac{1}{2^n}\bigg\}$$
The first inclusion follows from Lemma \ref{lemma:inclusion_of_surroundings_for_pseudometric} and the second follows from the fact that $\rho(x,y) \leq f(x,y)$ for every $x,y\in X$.
\end{proof}

\section{Uniform structures and uniform spaces}
\noindent
In this section we introduce main object of our study.

\begin{definition}
Let $X$ be a set. Suppose that $\fU$ is a collection of surroundings of $\Delta_X$ which satisfies the following two assertions.
\begin{enumerate}[label=\textbf{(\arabic*)}, leftmargin=*]
\item If $U \in \fU$ and $W$ is a surrounding of $\Delta_X$ such that $V\subseteq W$, then $W\in \fU$.
\item If $U,W\in \fU$, then $U\cap W \in \fU$. 
\item If $U \in \fU$, then there exists $W\in \fU$ such that $W\cdot W \subseteq U$.
\end{enumerate}
Then $\fU$ is \textit{a uniform structure on $X$}.
\end{definition}

\begin{example}\label{example:discrete_uniform_structure}
Let $X$ be a set. Then the family $\fD_X$ of all surroundings of $\Delta_X$ is a uniform structure on $X$. It is called \textit{the discrete uniform structure on $X$}.
\end{example}

\begin{fact}\label{fact:uniform_structures_are_closed_under_intersections}
Let $X$ be a set and let $\{\fU_i\}_{i\in I}$ be a family of uniform structures on $X$. Then 
$$\bigcap_{i\in I}\fU_i$$
is a uniform structure on $X$.
\end{fact}
\begin{proof}
Left for the reader.
\end{proof}

\begin{corollary}\label{corollary:smallest_uniform_structure_containing_given_family_of_surroundings}
Let $X$ be a set and let $\cF$ be a family of surrounding of $\Delta_X$. Then there exists the smallest (with respect to inclusion) uniform structure $\fU$ on $X$ which contains $\cF$.
\end{corollary}
\begin{proof}
Let $\{\fU_i\}_{i\in I}$ be a family of all uniform structures on $X$ which contain $\cF$. The family is nonempty, since it contains the discrete uniform structure on $X$. The intersection $$\fU = \bigcap_{i\in I}\fU_i$$
is a uniform structure on $X$ by Fact \ref{fact:uniform_structures_are_closed_under_intersections}. Hence it is the smallest uniform structure on $X$ which contains $\cF$.
\end{proof}

\begin{definition}
A pair $(X,\fU)$ consisting of a set $X$ and a uniform structure $\fU$ on $X$ is \textit{a uniform space}.
\end{definition}

\begin{definition}
Let $(X,\fU)$ be a uniform space. A surrounding $V$ in $\fU$ is called \textit{an entourage of the diagonal in $(X,\fU)$}. 
\end{definition}

\begin{definition}
Let $(X,\fU),(Y,\fV)$ be uniform spaces and let $f:X\ra Y$ be a map. Suppose that $\left(f\times f\right)^{-1}(V) \in \fU$ for every $V\in \fV$. Then $f$ is \textit{a morphism of uniform spaces}. 
\end{definition}

\begin{remark}\label{remark:category_of_uniform_spaces}
Uniform spaces and their morphisms form a category. We denote this category by $\Unif$.
\end{remark}
\noindent
In order to study categorical properties of $\Unif$ we use the following result.

\begin{theorem}\label{theorem:description_of_uniform_structure_introduced_by_a_family_of_maps}
Let $X$ be a set and let $\{(X_i,\fU_i)\}_{i\in I}$ be a family of uniform spaces. Consider a family $\big\{f_i:X\ra X_i\big\}_{i\in I}$ of maps. Suppose that $\fU$ is the smallest uniform structure on $X$ which makes $\{f_i\}_{i\in I}$ into a family of uniform morphisms. Then $U \in \fU$ if and only if there exist $n\in \NN_+$, $i_1,...,i_n\in I$ and $U_1 \in \fU_{i_1},...,U_n\in \fU_{i_n}$ such that
$$\bigcap_{k=1}^n\left(f_{i_k}\times f_{i_k}\right)^{-1}(U_k) \subseteq U$$
\end{theorem}
\begin{proof}
Consider the family $\cU$ of all surrounding $U$ of $\Delta_X$ such that there exist $n\in \NN_+$, $i_1,...,i_n\in I$ and $U_1 \in \fU_{i_1},...,U_n\in \fU_{i_n}$ satisfying
$$\bigcap_{k=1}^n\left(f_{i_k}\times f_{i_k}\right)^{-1}(U_k) \subseteq U$$
It is easy to verify (we left it for the reader) that $\cU$ is a uniform structure on $X$. Moreover, for every $n\in \NN_+$, $i_1,...,i_n\in I$ and $U_1 \in \fU_{i_1},...,U_n\in \fU_{i_n}$ we have
$$\bigcap_{k=1}^n\left(f_{i_k}\times f_{i_k}\right)^{-1}(U_k) \in \fU$$
Hence $\cU \subseteq \fU$. Note also that $f_i$ is a uniform morphism $(X,\cU)\ra (X_i,\fU_i)$ for each $i\in I$. Thus $\fU \subseteq \cU$. Therefore, $\cU = \fU$ and this proves the theorem. 
\end{proof}

\begin{definition}
Let $(X,\fU)$ be a uniform space and let $Z$ be a subset of $X$. Then $Z$ together with the smallest uniform structure which makes the inclusion $Z\hookrightarrow X$ into a uniform morphism is \textit{a uniform subspace of $(X,\fU)$ with $Z$ as the underlying set}.
\end{definition}

\section{Topology induced by uniform structure}
\noindent
We start by introducing the notion of a ball with respect to surrounding of the diagonal.

\begin{definition}
Let $X$ be a set. For every $x$ in $X$ and $U$ in $\fD_X$ the set
$$B(x,U) = \big\{y\in X\,\big|\,(x,y)\in U\big\}$$
is \textit{the ball with center $x$ and radius $U$}.
\end{definition}

\begin{fact}\label{fact:topology_induced_by_uniform_structure}
Let $X$ be a set and let $\fU$ be a uniform structure on $X$. The family
$$\tau_{\fU} = \big\{\cO\subseteq X\,\big|\,\mbox{ for each }x\in \cO\mbox{ there exists }U\in \fU\mbox{ such that }B(x,U)\subseteq \cO\big\}$$
is a topology on $X$.
\end{fact}
\begin{proof}
We left the proof for the reader as an exercise.
\end{proof}

\begin{definition}
Let $X$ be a set and let $\fU$ be a uniform structure on $X$. Then the topology $\tau_{\fU}$ is \textit{the topology on $X$ induced by $\fU$}.
\end{definition}
\noindent
The following result is a useful property of a topology induced by a uniform structure. 

\begin{proposition}\label{proposition:each_ball_contains_open_ball}
Let $(X,\fU)$ be a uniform space and let $U \in \fU$. Then there exists $W \in \fU$ contained in $U$ such that for every $x$ in $X$ the ball $B(x,W)$ is open with respect to the topology induced by $\fU$ on $X$.
\end{proposition}
\begin{proof}
We pick $U_1 \in \fU$ such that $U_1\cdot U_1 \subseteq U$. Next suppose that $U_n$ is defined for some $n\in \NN_+$. Then there exists $U_{n+1}\in \fU$ such that $U_{n+1}\cdot U_{n+1} \subseteq U_n$. Thus by recursive method we construct a sequence $\{U_n\}_{n\in \NN}$ of elements of $\fU$. Easy induction shows that
$$U_1\cdot U_2\cdot ...\cdot U_n \subseteq U$$
for each $n\in \NN_+$. Then
$$W = \bigcup_{n\in \NN_+}U_{1}\cdot U_{2}\cdot ...\cdot U_{n}$$
is in $\fU$ and is contained in $U$. Moreover, for every $x$ in $X$ the ball $B(x,W)$ is open with respect to $\tau_{\fU}$.
\end{proof}

\begin{corollary}\label{corollary:each_entourage_contains_open_subset_in_square}
Let $(X,\fU)$ be a uniform space and let $U \in \fU$. Then there exists $W \in \fU$ such that $W\subseteq U$ and $W$ is open in $\left(X,\tau_{\fU}\right) \times \left(X,\tau_{\fU}\right)$.
\end{corollary}
\begin{proof}
By Proposition \ref{proposition:each_ball_contains_open_ball} and the fact that $\fU$ is a uniform structure for $U \in \fU$ there exists $V \in \fU$ such that $V\cdot V\subseteq U$ and $B(x,V) \in \tau_{\fU}$ for every $x \in X$. We claim that $W = V\cdot V$ has the required properties. Indeed, we have
$$W = \bigcup_{x\in X}B(x,V)\times B(x,V)$$
and hence $W$ is open in the product $\left(X,\tau_{\fU}\right) \times \left(X,\tau_{\fU}\right)$. Since $V \in \fU$, we derive $W \in \fU$. Clearly $W\subseteq U$. 
\end{proof}

\begin{example}\label{example:uniform_structure_on_interval}
The interval $[0,1]$ admits a uniform structure given by
$$\big\{U\subseteq \fD_{[0,1]}\,\big|\,\mbox{ there exists }\epsilon > 0\mbox{ such that }|x - y| < \epsilon\mbox{ for some }x,y\in [0,1]\mbox{ implies }(x,y)\in U\big\}$$
Now the topology induced by this uniform structure coincides with the natural topology on $[0,1]$.
\end{example}

\begin{fact}\label{fact:uniform_morphism_is_a_continuous_map}
Let $(X,\fU)$ and $(Y,\fV)$ be uniform spaces and let $f:X\ra Y$ be a morphism of uniform spaces. Then $f$ is a continuous map $\left(X,\tau_{\fU}\right)\ra \left(Y,\tau_{\fY}\right)$.
\end{fact}
\begin{proof}
Pick open subset $\cO$ with respect to the topology induced by $\fY$ on $Y$. Suppose that $f(x) \in \cO$ for some $x$ in $X$. Then there exists $V_x \in \fY$ such that $B(f(x),V_x)\subseteq \cO$. Note that the image of $B\big(x,\left(f\times f\right)^{-1}(V_x)\big)$ under $f$ is contained in $\cO$. Therefore,
$$f^{-1}(\cO) = \bigcup_{x\in f^{-1}(\cO)}B\big(x,\left(f\times f\right)^{-1}(V_x)\big)$$
is open in the topology induced by $\fU$.
\end{proof}
\noindent
Fact \ref{fact:topology_induced_by_uniform_structure} and Fact \ref{fact:uniform_morphism_is_a_continuous_map} imply the existence of the functor 
$$\Unif \ni (X,\fU) \mapsto (X,\tau_{\fU})\in   \Top$$
In the remaining part of this section we shall investigate the properties of this functor. We start by describing the image of the functor.

\begin{definition}
Let $(X,\tau)$ be a topological space. Suppose that for every closed subset $F$ of $X$ and for every point $x$ in $X\setminus  F$ there exists a continuous function $f:X\ra [0,1]$ such that $f(F) \subseteq \{1\}$ and $f(x) = 0$. Then $X$ is \textit{a completely regular space}. 
\end{definition}

\begin{theorem}\label{theorem:image_of_the_canonical_functor_is_completely_regular_space}
The image of the object part of the functor
$$\Unif \ni (X,\fU) \mapsto (X,\tau_{\fU}) \in \Top$$
consists of the class of completely regular spaces.
\end{theorem}
\begin{proof}
Let $(X,\fU)$ be a uniform space. Consider a closed set $F$ with respect to $\tau_{\fU}$. Let $x$ be a point in $X\setminus F$. Since $x \not \in F$ and $F$ is closed in $\tau_{\fU}$, we derive that there exists $U \in \fU$ such that $B(x,U)\cap F = \emptyset$. Next we define a sequence $\{V_n\}_{n\in \NN}$ of elements in $\fU$ by recursion. We set $V_0 = U$ and if $V_0,...,V_n$ are defined for some $n\in \NN$, then we pick an element $V_{n+1}$ of $\fU$ such that 
$$V_{n+1}\cdot V_{n+1}\cdot V_{n+1} \subseteq V_n$$
According to Theorem \ref{theorem:Weils_theorem_on_pseudometrics} there exists a pseudometric $\rho$ on $X$ bounded by $1$ such that
$$\bigg\{(x,y)\in X\times X\,\bigg|\,\rho(x,y)<\frac{1}{2^n}\bigg\} \subseteq V_n \subseteq \bigg\{(x,y)\in X\times X\,\bigg|\,\rho(x,y) \leq \frac{1}{2^n}\bigg\}$$
for every $n\in \NN$. Note that
$$|\rho(x,y_1) - \rho(x,y_2)| \leq \rho(y_1,y_2)$$
for any pair $y_1,y_2\in X$. Indeed, this is the triangle inequality for $\rho$. Thus if $(y_1,y_2) \in V_n$ for some $n\in \NN$, then
$$|\rho(x,y_1) - \rho(x,y_2)| \leq \rho(y_1,y_2) \leq \frac{1}{2^n}$$
Hence the map $f:X \ra [0,1]$ given by formula $f(y) = \rho(x,y)\in [0,1]$ is a morphism of uniform spaces, where $X$ is a uniform space with respect to $\fU$ and $[0,1]$ is considered with uniform structure described in Example \ref{example:uniform_structure_on_interval}. This implies (by Fact \ref{fact:uniform_morphism_is_a_continuous_map}) that $f$ is a continuous map, where $X$ carries topology $\tau_{\fU}$ and $[0,1]$ is considered with natural topology. Pick $y\in F$. Then $y \not \in B(x,U)$ and hence $(x,y) \not \in U$. Since $V_0 = U$ and $\rho$ is bounded by $1$, we derive that $f(y) = \rho(x,y) = 1$. On the other hand $f(x) = \rho(x,x) = 0$. Therefore, $f(F) \subseteq \{1\}$ and $f(x) = 0$. Thus $(X,\tau_{\fU})$ is a completely regular space.\\
Suppose now that $(X,\tau)$ is a completely regular space. Consider the set $C(\tau,\RR)$ of all continuous real valued functions on $(X,\tau)$. For $m\in \NN_+$ and set of $m$ functions $f_1,...,f_m\in C(\tau,\RR)$ define
$$\rho_{f_1,...,f_m}(x,y) = \max \big\{|f_1(x) - f_1(y)|,...,|f_m(x) - f_m(y)|\big\}$$
where $x,y\in X$. Clearly $\rho_{f_1,...,f_m}$ is a pseudometric on $X$. Next consider a family $\fU$ of all $U\in \fD_X$ such that there exist a finite subset $\{f_1,...,f_m\}\subseteq C(\tau,\RR)$ for some $m \in \NN_+$ and $\epsilon > 0$ such that
$$\big\{(x,y)\in X\times X\,\big|\,\rho_{f_1,...,f_m}(x,y) < \epsilon\big\}\subseteq U$$
Clearly $\fU$ is a uniform structure on $X$. Suppose that $\cO \in \tau_{\fU}$. Then for each point $z$ in $\cO$ there exists $U$ in $\fU$ such that $B(z,U)\subseteq \cO$. By definition there exist a finite subset $\{f_1,...,f_m\}\subseteq C(\tau,\RR)$ for some $m \in \NN_+$ and $\epsilon > 0$ such that
$$\big\{(x,y)\in X\times X\,\big|\,\rho_{f_1,...,f_m}(x,y) < \epsilon\big\}\subseteq U$$
Thus
$$\bigcap_{i=1}^mf_i^{-1}\bigg(\big(f_i(z) - \epsilon, f_i(z) + \epsilon\big)\bigg) = \big\{y\in X \,\big|\, \rho_{f_1,...,f_m}(z,y) < \epsilon\big\} \subseteq B(z,U)\subseteq \cO$$
Since $f_1,...,f_m$ are continuous on $X$ with respect to $\tau$, we derive from the inclusion above that there exists an open neighborhood of $z$ with respect to $\tau$ contained in $\cO$. According to the fact that $z$ is an arbitrary point in $\cO$ it follows that $\cO \in \tau$. This proves that $\tau_{\fU}\subseteq \tau$. Now we prove the converse. For this assume that $\cO \in \tau$. We claim that $\cO$ is also open in the topology induced by $\fU$. For this pick $z \in \cO$. Since $(X,\tau)$ is completely regular, there exists a function $f_z:X\ra \RR$ continuous with respect to $\tau$ such that $f_z(X\setminus \cO)\subseteq \{1\}$ and $f_z(z) = 0$. Let $U_z$ be a set consisting of all pairs in $X\times X$ for which $\rho_{f_z}$ is smaller than $1$. Then $U_z\in \fU$ and obviously $B(z,U_z)\subseteq \cO$. Thus
$$\cO = \bigcup_{z\in \cO}B(z,U_z)$$
and this proves the claim that $\cO$ is in $\tau_{\fU}$. Hence $\tau \subseteq \tau_{\fU}$. This completes the proof.
\end{proof}
\noindent
Next we prove the following important fact.

\begin{proposition}\label{proposition:the_induced_topology_preserves_uniform_subspaces}
Let $(X,\fU)$ be a uniform space and let $Z$ be a subset of $X$. Let $\fU_Z$ be the subspace uniform structure on $Z$. Then $\tau_{\fU_Z}$ coincide with the subspace topology on $Z$ induced by $\tau_{\fU}$.
\end{proposition}
\begin{proof}
Let $\cO$ be a set in $\tau_{\fU}$. Then for each $x$ in $\cO$ there exists $U_x\in \fU$ such that $B(x,U_x)\subseteq \cO$. Thus
$$\cO \cap Z = \bigcup_{z\in \cO\cap Z}B\big(z,U_z\cap \left(Z\times Z\right)\big)$$
Since $U_z\cap \left(Z\times Z\right) \in \fU_Z$ for every $z \in \cO\cap Z$, it follows that $\cO\cap Z$ is open with respect to $\tau_{\fU_Z}$. Hence
$$\big\{Z\cap \cO\,\big|\,\cO \in \tau_{\fU}\big\} \subseteq \tau_{\fU_Z}$$
Suppose now that $\cO_Z \in \tau_{\fU_Z}$. Then for each $z\in \cO_Z$ there exists $U_z \in \fU$ such that $B\big(z,U_z\cap \left(Z\times Z\right)\big)\subseteq \cO_Z$. By Proposition \ref{proposition:each_ball_contains_open_ball} there exists $\cO_z \in \tau_{\fU}$ such that $z \in \cO_z\subseteq B(z,U_z)$. Thus 
$$\cO = \bigcup_{z \in \cO_Z}\cO_z$$
is an element of $\tau_{\fU}$ and
$$\cO \subseteq \bigcup_{z\in \cO_Z}B(z,U_z)$$
and hence $Z\cap \cO = \cO_Z$. Therefore, $\cO_Z$ is an open subset in the subspace topology induced on $Z$ by $\tau_{\fU}$. This completes the proof.
\end{proof}

\begin{theorem}\label{theorem:limits_of_uniform_spaces_description}
Let $\cI$ be a small category and let $F:\cI\ra \Unif$ be a functor given by
$$F(i) = (X_i,\fU_i)$$
for $i\in \cI$. Let $\big\{f_i:X \ra X_i \big\}_{i\in \cI}$ be a limiting cone of the composition of $F$ with the functor $\Unif \ra \Set$ which sends each uniform space to its underlying set. Consider the smallest uniform structure $\fU$ on $X$ which makes $\{f_i\}_{i\in \cI}$ into a family of uniform morphisms. Then $(X,\fU)$ together with $\{f_i\}_{i\in \cI}$ is a limiting cone of $F$.
\end{theorem}
\begin{proof}
We may equivalently describe $\fU$ as the smallest uniform structure on $X$ such that 
$$\left(f_i\times f_i\right)^{-1}(U) \in \fU$$
for every $i\in \cI$ and every $U \in \fU_i$. Suppose that $\big\{g_i:(Y,\fV) \ra (X_i,\fU_i)\big\}_{i\in \cI}$ is some cone over $F$. Then there exists a unique map $h:Y \ra X$ such that $h\cdot f_i = g_i$ for every $i\in \cI$. It is easy to verify that
$$\big\{U\in \fU\,\big|\,\left(h\times h\right)^{-1}(U)\mbox{ is an entourage of the diagonal in }\fV\big\}$$
is a uniform structure on $X$. Moreover, it contains $\left(f_i\times f_i\right)^{-1}(U)$ for every $i\in \cI$ and every $U \in \fU_i$. Since $\fU$ is the smallest such uniform structure, we derive that $\fU$ and
$$\big\{U\in \fU\,\big|\,\left(h\times h\right)^{-1}(U)\mbox{ is an entourage of the diagonal in }\fB\big\}$$
coincide and hence $h$ is a morphism of uniform spaces $\left(Y,\fB\right) \ra \left(X,\fU\right)$. This shows that $(X,\fU)$ together with $\big\{f_i:X \ra X_i\big\}_{i\in \cI}$ is a limiting cone of $F$.
\end{proof}

\begin{theorem}\label{theorem:limits_of_uniform_spaces_description}
The functor 
$$\Unif \ni (X,\fU) \mapsto (X,\tau_{\fU}) \in \Top$$
preserves small limits.
\end{theorem}
\begin{proof}
Let $\cI$ be a set and let $\left\{\left(X_i,\fU_i\right)\right\}_{i\in \cI}$ be a family of uniform spaces parametrized by $\cI$. Consider the cartesian product $X = \prod_{i\in \cI}X_i$ and let $pr_i:X\ra X_i$ be the projection for $i\in \cI$. Let $\fU$ be the smallest uniform structure on $X$ which makes $\big\{pr_i:\left(X,\fU\right) \ra \left(X_i,\fU_i\right)\big\}_{i\in \cI}$ into a family of morphisms of uniform spaces. By Theorem \ref{theorem:description_of_uniform_structure_introduced_by_a_family_of_maps} family $\fU$ consists of all surrounding $U$ of $\Delta_X$ such that there exist $n\in \NN_+$, $i_1,...,i_n\in \cI$ and $U_1 \in \fU_{i_1},...,U_n\in \fU_{i_n}$ satisfying
$$\bigcap_{k=1}^n\left(pr_{i_k}\times pr_{i_k}\right)^{-1}(U_k) \subseteq U$$
Note that we have
$$B\bigg(x, \bigcap_{k=1}^n\left(pr_{i_k}\times pr_{i_k}\right)^{-1}(U_k)\bigg) = \prod_{k=1}^nB\big(pr_{i_k}(x), U_k\big)\times \prod_{i\in \cI\setminus \{i_1,...,i_k\}}X_i$$
for every $x \in X$. By Proposition \ref{proposition:each_ball_contains_open_ball} there exist $\cO_1 \in \tau_{\fU_{i_1}},...,\cO_n \in \tau_{\fU_{i_n}}$ such that 
$$pr_{i_k}(x) \in \cO_k \subseteq \big(pr_{i_k}(x), U_k\big)$$
for each $k$. Thus
$$\prod_{k=1}^n\cO_k \times \prod_{i\in \cI\setminus \{i_1,...,i_k\}}X_i \subseteq B\bigg(x, \bigcap_{k=1}^n\left(pr_{i_k}\times pr_{i_k}\right)^{-1}(U_k)\bigg) \subseteq B(x,U)$$
Therefore, each ball centered in some point $x$ of $X$ and with radius $U$ in $\fU$ contains open neighborhood of $x$ with respect to the product of topologies $\{\tau_{\fU_i}\}_{i\in  \cI}$. This implies that $\tau_{\fU}$ is contained in the product of topologies $\{\tau_{\fU_i}\}_{i\in  \cI}$. On the other hand the fact that $pr_i:(X,\tau_{\fU}) \ra (X_i,\tau_{\fU_i})$ is continuous for every $i\in \cI$ implies that the product of topologies $\{\tau_{\fU_i}\}_{i\in \cI}$ is contained in $\tau_{\fU}$. Thus $\tau_{\fU}$ is the product topology determined by $\{\tau_{\fU_i}\}_{i\in \cI}$. Hence $(X,\tau_{\fU})$ together with $\{pr_i\}_{i\in \cI}$ is a product of topological spaces $\{(X_i,\tau_{\fU_i})\}_{i\in \cI}$. By Theorem \ref{theorem:limits_of_uniform_spaces_description} it follows that
$$\Unif \ni (X,\fU) \mapsto (X,\tau_{\fU}) \in \Top$$
preserves small products. Since every small limit is a combination of small product and kernel pair, it remains to show that the functor above preserves kernel pairs. Suppose that 
\begin{center}
\begin{tikzpicture}
[description/.style={fill=white,inner sep=2pt}]
\matrix (m) [matrix of math nodes, row sep=3em, column sep=3em,text height=1.5ex, text depth=0.25ex] 
{(X,\fW) & \left(Y_1,\fV_1\right)&  \left(Y_1,\fV_2\right)  \\} ;
\path[right hook->,line width=0.8pt,font=\scriptsize]
(m-1-1) edge node[above] {$  $} (m-1-2);
\path[->,line width=0.8pt,font=\scriptsize]
(m-1-2) edge[transform canvas={yshift=0.5ex}] node[above] {$ f_1 $} (m-1-3)
(m-1-2) edge[transform canvas={yshift=-0.5ex}] node[below] {$ f_2 $} (m-1-3);
\end{tikzpicture}
\end{center}
is a kernel pair of $f_1$ and $f_2$ in $\Unif$. Then Theorem \ref{theorem:limits_of_uniform_spaces_description} shows that $X$ consists of all $y \in Y_1$ such that $f_1(y) = f_2(y)$ and $\fW$ is the subspace uniform structure on $X$ induced by $\fV_1$. Now Proposition \ref{proposition:the_induced_topology_preserves_uniform_subspaces} implies that
\begin{center}
\begin{tikzpicture}
[description/.style={fill=white,inner sep=2pt}]
\matrix (m) [matrix of math nodes, row sep=3em, column sep=3em,text height=1.5ex, text depth=0.25ex] 
{(X,\tau_{\fW}) & \left(Y_1,\tau_{\fV_1}\right)&  \left(Y_1,\tau_{\fV_2}\right)  \\} ;
\path[right hook->,line width=0.8pt,font=\scriptsize]
(m-1-1) edge node[above] {$  $} (m-1-2);
\path[->,line width=0.8pt,font=\scriptsize]
(m-1-2) edge[transform canvas={yshift=0.5ex}] node[above] {$ f_1 $} (m-1-3)
(m-1-2) edge[transform canvas={yshift=-0.5ex}] node[below] {$ f_2 $} (m-1-3);
\end{tikzpicture}
\end{center}
is a kernel pair in the category $\Top$. Therefore, the functor
$$\Unif \ni (X,\fU) \mapsto (X,\tau_{\fU}) \in \Top$$
preserves kernel pairs. The proof is complete.
\end{proof}

\section{Hausdorff uniform spaces and Tychonoff spaces}
\noindent
In this section we introduce important classes of topological spaces.

\begin{definition}
Let $(X,\tau)$ be a topological space. Suppose that for any pair of points $x,y$ in $X$ there exists set $\cO$ in $\tau$ such that either $x \in \cO$ and $y \not \in \cO$ or $y \in \cO$ and $x \not \in \cO$. Then $(X,\tau)$ is \textit{a Kolmogorov space}.
\end{definition}

\begin{definition}
Let $(X,\tau)$ be a topological space and let $\fX$ be a Kolmogorov space. Let $q:\left(X,\tau\right)\ra \fX$ be a morphism of topological spaces. Suppose that for every Kolmogorov space $\left(Y,\theta\right)$ and every morphism $f:\left(X,\tau\right)\ra \left(Y,\theta\right)$ of topological spaces there exists a unique morphism $p:\fX \ra \left(Y,\theta\right)$ of topological spaces which makes the triangle
\begin{center}
\begin{tikzpicture}
[description/.style={fill=white,inner sep=2pt}]
\matrix (m) [matrix of math nodes, row sep=4em, column sep=5em,text height=1.5ex, text depth=0.25ex] 
{ X &  Y  \\
    {|}\fX{|} & \\ } ;
\path[->,line width=0.8pt,font=\scriptsize]
(m-1-1) edge node[above] {$ f $} (m-1-2)
(m-1-1) edge node[left] {$ q $} (m-2-1);
\path[densely dotted,->,line width=0.8pt,font=\scriptsize]
(m-2-1) edge node[right = 2pt, below = 2pt] {$ p $} (m-1-2);
\end{tikzpicture}
\end{center}
commutative where $|\fX|$ is the underlying set of $\fX$. Then $q:\left(X,\tau\right)\ra \fX$ is \textit{a universal Kolmogorov quotient of $\left(X,\tau\right)$}.
\end{definition}
\noindent
The following result shows that every topological space admits universal Kolmogorov quotient.

\begin{theorem}\label{theorem:universal_Kolomogorov_quotient}
Every topological space admits a universal Kolmogorov quotient. Moreover, if $(X,\tau)$ is a topological space and $q:\left(X,\tau\right) \ra \fX$ is its universal Kolmogorov quotient. Then the following assertions hold.
\begin{enumerate}[label=\emph{\textbf{(\arabic*)}}, leftmargin=*]
\item We have equality of sets
$$\big\{(x,y)\in X\times X\,\big|\,q(x) = q(y)\big\} = \big\{(x,y) \in X\times X\,\big|\,\forall_{\cO\in \tau}\,x\in \cO\,\Leftrightarrow\,y\in \cO\big\}$$
\item Morphism $q$ is surjective, open and closed.
\end{enumerate}
\end{theorem}
\begin{proof}
For each $x \in X$ denote by $\cN_x$ the family of all open neighborhoods of $x$. Note that $\cN_x = \cN_y$ if and only if $\forall_{\cO\in \tau}\,x\in \cO\,\Leftrightarrow\,y\in \cO$. Define
$$x \sim_{Kol} y\,\Leftrightarrow\,\cN_x = \cN_y$$
for every $x,y\in X$. From definition it follows that $\sim_{Kol}$ is an equivalence relation. We define $\fX$ as the quotient space of $\left(X,\tau\right)$ with respect to $\sim_{Kol}$. Let $q:\left(X,\tau\right)\ra \fX$ be the quotient map. In particular it is surjective. Observe that every set $\cO$ in $\tau$ is a union of equivalence classes of $\sim_{Kol}$. Thus $q:\left(X,\tau\right)\twoheadrightarrow \fX$ is both open and closed continuous map. We claim that $\fX$ is a Kolmogorov space. Suppose that $q(x) \neq q(y)$ for some $x,y\in X$. Then $x \sim_{Kol} y$ and hence there exists $\cO$ in $\tau$ such that (without loss of generality) $x \in \cO$ and $y \not \in \cO$. Then $q(x) \in q(\cO)$ and $q(y) \not \in q(\cO)$. Since $q$ is open, we derive that $q(\cO)$ is an open subset of $\fX$. Since $q(x),q(y)$ are arbitrary distinct points of $\fX$, we derive that $\fX$ is a Kolmogorov space. It remains to prove the universal property of $q$. Assume that $\left(Y,\theta\right)$ is a Kolmogorov space and consider a morphism $f:\left(X,\tau\right)\ra \left(Y,\theta\right)$ of topological spaces. We derive that $x \sim_{Kol} y$ implies $f(x) = f(y)$ for every $x,y\in X$. Since $q$ is the quotient map with respect to $\sim_{Kol}$, we deduce that there exists a unique continuous map $p:\fX \ra \left(Y,\theta\right)$ such that $p\cdot q = f$.
\end{proof}

\begin{definition}
A topological space $(X,\tau)$ which is both completely regular and Kolmogorov is \textit{a Tychonoff space}. 
\end{definition}

\begin{proposition}\label{proposition:Kolmogorov_quotient_of_completely_regular_space_is_Tychonoff}
Let $(X,\tau)$ be a completely regular space. Then its Kolmogorov quotient is a Tychonoff space.
\end{proposition}
\begin{proof}
Suppose that $q:\left(X,\tau\right) \ra \fX$ is a universal Kolmogorov quotient of $\left(X,\tau\right)$. Consider a point $\tilde{x}$ in $\fX$ and let $F$ be a closed subset in $\fX$ such that $\tilde{x} \not \in F$. It follows that $q^{-1}(\tilde{x}) \cap q^{-1}(F) = \emptyset$. Fix $x \in q^{-1}(\tilde{x})$. Since $\left(X,\tau\right)$ is a completely regular space, there exists a continuous function $f:X\ra [0,1]$ such that $f\left(q^{-1}(F)\right) \subseteq \{1\}$ and $f\left(x\right) = 0$. By universal property of $q$ there exists continuous map $\tilde{f}:\fX\ra [0,1]$ such that $\tilde{f} \cdot q = f$. Then 
$$\tilde{f}(F) = \left(\tilde{f}\cdot q\right)\left(q^{-1}(F)\right) = f\left(q^{-1}\left(F\right)\right) \subseteq \{1\}$$
and 
$$\tilde{f}\left(\tilde{x}\right) = \left(\tilde{f}\cdot q\right)(x) = f(x) = 0$$
This proves that $\fX$ is completely regular. By definition $\fX$ is a Kolmogorov space. This completes the proof.
\end{proof}
\noindent
Now we study uniform spaces which satisfy the following separation axiom. 

\begin{definition}
Let $(X,\fU)$ be a uniform space. Suppose that
$$\Delta_X = \bigcap_{U\in \fU}U$$
Then $(X,\fU)$ is \textit{a Hausdorff uniform space}.
\end{definition}

\begin{definition}
Let $(X,\fU)$ be a uniform space and let $\fX$ be a Hausdorff uniform space. Let $q:\left(X,\fU\right)\ra \fX$ be a morphism of uniform spaces. Suppose that for every Hausdorff uniform space $\left(Y,\fV\right)$ and every morphism $f:\left(X,\fU\right)\ra \left(Y,\fV\right)$ of uniform spaces there exists a unique morphism $p:\fX \ra \left(Y,\fV\right)$ of uniform spaces which makes the triangle
\begin{center}
\begin{tikzpicture}
[description/.style={fill=white,inner sep=2pt}]
\matrix (m) [matrix of math nodes, row sep=4em, column sep=5em,text height=1.5ex, text depth=0.25ex] 
{ X &  Y  \\
    {|}\fX{|} & \\ } ;
\path[->,line width=0.8pt,font=\scriptsize]
(m-1-1) edge node[above] {$ f $} (m-1-2)
(m-1-1) edge node[left] {$ q $} (m-2-1);
\path[densely dotted,->,line width=0.8pt,font=\scriptsize]
(m-2-1) edge node[right = 2pt, below = 2pt] {$ p $} (m-1-2);
\end{tikzpicture}
\end{center}
commutative where $|\fX|$ is the underlying set of $\fX$. Then $q:\left(X,\fU\right)\ra \fX$ is \textit{a universal Kolmogorov quotient of $\left(X,\tau\right)$}.
Let $(X,\fU)$ be a uniform space. Then a morphism $q:\left(X,\fU\right)\ra \fX$ is \textit{a universal Hausdorff quotient of $\left(X,\fU\right)$}.
\end{definition}
\noindent
The following result shows that every uniform space admits universal Hausdorff quotient.

\begin{theorem}\label{theorem:universal_Hausdorff_quotient}
Every uniform space admits a universal Haudorff quotient. Moreover, if $(X,\fU)$ is a uniform space and $q:\left(X,\tau\right) \ra \fX$ is its universal Haudorff quotient. Then the following assertions hold.
\begin{enumerate}[label=\emph{\textbf{(\arabic*)}}, leftmargin=*]
\item We have equality of sets
$$\big\{(x,y)\in X\times X\,\big|\,q(x) = q(y)\big\} = \bigcap_{U\in \fU}U$$
\item $q$ is surjective.
\item The uniform structure on $\fX$ is the family
$$\big\{U \in \fD_{|\fX|}\,\big|\,(q\times q)^{-1}(U) \in \fU\big\}$$
\end{enumerate}
\end{theorem}
\noindent
Denote the intersection of all $U \in \fU$ by $\Delta_{\fU}$. For the proof we need lemma.

\begin{lemma}\label{lemma:entourage_closed_under_pseudodiagonal}
For every $U$ in $\fU$ and there exists $V$ in $\fU$ such that $V\cdot V\subseteq U$ and $\Delta_{\fU}\cdot V\cdot \Delta_{\fU}\subseteq V$.
\end{lemma}
\begin{proof}[Proof of the lemma]
Left for the reader as an exercise.
\end{proof}

\begin{proof}[Proof of the theorem]
Clearly $\Delta_{\fU}$ is reflexive and symmetric. Fix $U$ in $\fU$. Then there exists $W \in \fU$ such that $W\cdot W\subseteq U$. Hence 
$$\Delta_{\fU}\cdot \Delta_{\fU} \subseteq W\cdot W \subseteq U$$
Since $U$ is an arbitrary element of $\fU$, we derive that $\Delta_{\fU}\cdot \Delta_{\fU}\subseteq \Delta_{\fU}$. Hence $\Delta_{\fU}$ is transitive. This shows that $\Delta_{\fU}$ is an equivalence relation on $X$. Let $q:X\ra \tilde{X}$ be the quotient map of $X$ with respect to $\Delta_{\fU}$. Define a family
$$\tilde{\fU} = \big\{U \in \fD_{\tilde{X}}\,\big|\,(q\times q)^{-1}(U) \in \fU\big\}$$
We claim that $\tilde{\fU}$ is a uniform structure. The nontrivial part is verification that for $U$ in $\tilde{\fU}$ there exists $W \in \tilde{\fU}$ such that $W\cdot W \subseteq U$. By Lemma \ref{lemma:entourage_closed_under_pseudodiagonal} there exists $V\in \fU$ such that 
$$V\cdot V\subseteq \left(q\times q\right)^{-1}(U)$$ 
and 
$$\Delta_{\fU}\cdot V\cdot \Delta_{\fU}\subseteq V$$
Since $\Delta_{\fU} \cdot V\cdot \Delta_{\fU}\subseteq V$, we have $V = \left(q\times q\right)^{-1}(W)$ where $W = \left(q\times q\right)\left(V\right)$. This implies that $W \in \tilde{\fU}$ and $W\cdot W\subseteq U$. Therefore, $\tilde{\fU}$ is a uniform structure. Then $q$ is a morphism of uniform spaces $\left(X,\fU\right)\ra \left(\tilde{X},\tilde{\fU}\right)$. We claim now that $\left(\tilde{X},\tilde{\fU}\right)$ is a Hausdorff uniform space. Suppose that $x,y \in X$ satisfy $q(x)\neq q(y)$ or in other words $\left(q(x),q(y)\right)\not \in \Delta_{\tilde{X}}$. Then $(x,y) \not \in \Delta_{\fU}$ and there exists $U$ in $\fU$ such that $(x,y) \not \in U$. Invoking Lemma \ref{lemma:entourage_closed_under_pseudodiagonal} there exists $V\in \fU$ such that $V\subseteq V\cdot V \subseteq U$ and $\Delta_{\fU}\cdot V\cdot \Delta_{\fU}\subseteq V$. Then $W = \left(q\times q\right)(V)$ is an element of $\tilde{\fU}$ and $\left(q(x),q(y)\right)\not \in W$. Hence $\left(\tilde{X},\tilde{\fU}\right)$ is Hausdorff. Finally it suffices to check that $q:\left(X,\fU\right) \ra \left(\tilde{X},\tilde{\fU}\right)$ satisfies the universal property. Assume that $\left(Y,\fV\right)$ is a Hausdorff uniform space and consider a morphism $f:\left(X,\fU\right)\ra \left(Y,\fV\right)$ of uniform spaces. Observe that $\left(f\times f\right)^{-1}\left(\Delta_Y\right)$ is an equivalence relation containing $\Delta_{\fU}$. Indeed, it is an equivalence relation due to the fact that $f$ is a map and it contains $\Delta_{\fU}$ due to the fact that $f$ is a uniform map and $(Y,\fV)$ is Hausdorff. It follows that $f = p\cdot q$ for a unique map $p:\tilde{X}\ra Y$. Fix now an entourage $V \in \fV$. Then 
$$\left(q\times q\right)^{-1}\big(\left(p\times p\right)^{-1}(V)\big) = \left(f\times f\right)^{-1}(V) \in \fU$$
Hence $\left(p\times p\right)^{-1}(V) \in \tilde{\fU}$ by definition. Thus $p$ is a unique uniform morphism such that $f = p\cdot q$.
\end{proof}
\noindent
We show that there exists intimate universal Hausdorff uniform quotients are universal Kolmogorov quotients in topological category.

\begin{proposition}\label{proposition:universal_Hausdordff_quotients_are_universal_Kolmogorov_quotients}
The functor
$$\Unif \ni (X,\fU) \mapsto (X,\tau_{\fU}) \in \Top$$
sends universal Hausdorff quotients to universal Kolmogorov quotients.
\end{proposition}
\begin{proof}
Let $(X,\fU)$ be a uniform space and let $\tau_{\fU}$ be the topology on $X$ induced by $\fU$.
\end{proof}



\section{Pseudometrizable and Hausdorff uniform spaces}
\noindent
In this section we use Theorem \ref{theorem:Weils_theorem_on_pseudometrics} to prove certain structure theorems concerning uniform spaces. 

\begin{definition}
A uniform space $(X,\fU)$ is \textit{pseudometrizable} if there exists a pseudometric $\rho$ on $X$ such that the uniform structure
$$\bigg\{U \in \fD_{X}\,\bigg|\,\mbox{ there exists }\epsilon>0\mbox{ such that for all }x,y\in X\mbox{ if }\rho(x,y)\leq \epsilon\mbox{ then }(x,y)\in U\bigg\}$$
coincides with $\fU$.
\end{definition}

\begin{theorem}\label{theorem:characterization_of_pseudometrizable_uniform_spaces}
Let $(X,\fU)$ be a uniform space. The following assertions are equivalent.
\begin{enumerate}[label=\emph{\textbf{(\roman*)}}, leftmargin=*]
\item $(X,\fU)$ is a pseudometrizable uniform space.
\item There exists a sequence $\{U_n\}_{n\in \NN}$ of elements in $\fU$ such that the family
$$\bigg\{U\in \fD_{X}\,\bigg|\exists_{n\in \NN}\,U_n \subseteq U\bigg\}$$
coincides with $\fU$.
\end{enumerate}
\end{theorem}
\begin{proof}
For $\textbf{(i)}\Rightarrow \textbf{(ii)}$ observe that if $\rho$ is a pseudometric on $X$ such that 
$$\bigg\{U \in \fD_{X}\,\bigg|\,\mbox{ there exists }\epsilon>0\mbox{ such that for all }x,y\in X\mbox{ if }\rho(x,y)\leq \epsilon\mbox{ then }(x,y)\in U\bigg\}$$
coincides with $\fU$, then the sequence $\{U_n\}_{n\in \NN}$ given by formula
$$U_n = \bigg\{(x,y)\in X\times X\,\bigg|\,\rho(x,y) \leq \frac{1}{2^n}\bigg\}$$
satisfies \textbf{(ii)}.\\
Suppose now that \textbf{(ii)} holds. We define a sequence $\{V_n\}_{n\in \NN}$ of elements in $\fU$ by recursion. We set $V_0 = U_0$ and if $V_0,...,V_n$ are defined for some $n\in \NN$, then we pick an element $W$ of $\fU$ such that 
$$W\cdot W\cdot W \subseteq V_n$$
and define $V_{n+1} = W\cap U_{n+1}$. Note that $\{V_n\}_{n\in \NN}$ satisfies
$$V_{n+1}\cdot V_{n+1}\cdot V_{n+1} \subseteq V_n$$
for each $n\in \NN$. Moreover, we have
$$\fU = \bigg\{U\in \fD_{X}\,\bigg|\exists_{n\in \NN}\,V_n \subseteq U\bigg\}$$
By Theorem \ref{theorem:Weils_theorem_on_pseudometrics} there exists a pseudometric $\rho$ on $X$ such that
$$\bigg\{(x,y)\in X\times X\,\bigg|\,\rho(x,y)<\frac{1}{2^n}\bigg\} \subseteq V_n \subseteq \bigg\{(x,y)\in X\times X\,\bigg|\,\rho(x,y) \leq \frac{1}{2^n}\bigg\}$$
for every $n\in \NN$. This implies that
$$\bigg\{U \in \fD_{X}\,\bigg|\,\mbox{ there exists }\epsilon>0\mbox{ such that for all }x,y\in X\mbox{ if }\rho(x,y)\leq \epsilon\mbox{ then }(x,y)\in U\bigg\}$$
coincides with $\fU$. Hence $\textbf{(ii)}\Rightarrow \textbf{(i)}$.
\end{proof}

\begin{corollary}\label{corollary:uniform_space_is_subspace_of_product_of_pseudometrizable_spaces}
Every uniform space is a uniform subspace of a product of pseudometrizable uniform spaces.
\end{corollary}
\begin{proof}
Let $(X,\fU)$ be a uniform space. For each $U$ in $\fU$ we construct a nonincreasing sequence $\{U_n\}_{n\in \NN}$ of elements of $\fU$ such that $U_0 = U$ and $U_{n+1}\cdot U_{n+1} \cdot U_{n+1} \subseteq U_n$ for each $n\in \NN$. Next define $\fV_U$ as a family 
$$\big\{W \in \fD_X\,\big|\,\exists_{n\in \NN}\,U_n\subseteq W\big\}$$
Then $\fV_U$ is a uniform structure on $X$ and $U\in \fV_U \subseteq \fU$. Moreover, by Theorem \ref{theorem:characterization_of_pseudometrizable_uniform_spaces}

\end{proof}


































\end{document}