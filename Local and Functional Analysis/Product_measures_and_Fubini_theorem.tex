\input ../pree.tex

\begin{document}

\title{Product measures and Fubini theorem}
\date{}
\maketitle

\section{Introduction}
\noindent
In these notes we develop theory of iterated integrals and product measures. In the first two sections we introduce important results concerning induction on set and function classes. In particular, we prove famous Dynkin's $\pi-\lambda$ lemma and Sierpiński lemma on monotone classes. Next we use this results to prove the existence and uniqueness of product measures in $\sigma$-finite case. In the final section we prove celebrated theorems due to Tonelli and Fubini.

We assume familiarity with \cite{Integration}.

\section{Induction for families of sets}
\noindent
In these section we study families of sets closed under certain set-theoretic operations. We first introduce all neccessary notions. The notion of $\sigma$-algebra was defined in \cite{Integration}. Here we recall it.

\begin{definition}\label{definition:families_of_sets}
    Let $X$ be a set and let $\cF$ be a family of subsets of $X$.
    \begin{enumerate}[label=\textbf{(\arabic*)}, leftmargin=*]
        \item $\cF$ is \textit{an algebra} if it contains $X$ and is closed under finite unions, intersections and complements.
        \item $\cF$ is \textit{a $\sigma$-algebra} if it is an algebra and is closed under countable unions.
        \item $\cF$ is \textit{a monotone family} if it is closed under unions of countable non-decreasing sequences of subsets and under intersections of countable non-increasing sequences of subsets.
        \item $\cF$ is \textit{a $\pi$-system} if it is closed under finite intersections.
        \item $\cF$ is \textit{a $\lambda$-system} if it contains $X$ and is closed under complements and countable disjoint unions.
    \end{enumerate}
\end{definition}

\begin{remark}\label{remark:intersections}
    Let $X$ be a set and let $\{\cF_i\}_{i\in I}$ be a class of families of subsets of $X$. Suppose that $\cF_i$ is an algebra for each $i \in I$, then $\bigcap_{i\in I}\cF_i$ is an algebra. Similarly for other types of families described in Definition \ref{definition:families_of_sets}.
\end{remark}

\begin{definition}
    Let $X$ be a set and let $\cF$ be a family of subsets of $X$.
    \begin{enumerate}[label=\textbf{(\arabic*)}, leftmargin=*]
        \item Then \textit{the $\sigma$-algebra generated by $\cF$ in $X$} is the intersection of all $\sigma$-algebras of subsets of $X$ which contain $\cF$. It is also denoted by $\sigma(\cF)$.
        \item Then \textit{the $\lambda$-system generated by $\cF$ in $X$} is the intersection of all $\lambda$-systems of subsets of $X$ which contain $\cF$. It is also denoted by $\lambda(\cF)$.
        \item Then \textit{the monotone family generated by $\cF$ in $X$} is the intersection of all monotone families of subsets of $X$ which contain $\cF$. It is also denoted by $\cM(\cF)$.
    \end{enumerate}
\end{definition}
\noindent
Now we shall prove the important observation.

\begin{theorem}[Dynkin's $\pi$-$\lambda$ lemma]\label{theorem:dynkins_lemma}
    Let $X$ be a set and let $\cP$ be a $\pi$-system of its subsets. Then $\lambda(\cP)=\sigma(\cP)$.
\end{theorem}
\noindent
For the proof we need the following result.

\begin{lemma}\label{lemma:dynkins_lemma}
    Let $\cL$ be a $\lambda$-system and let $A \in \cL$. Then
    $$\cL_A=\big\{B\subseteq X\,\big|\,A\cap B\in \cL \big\}$$
    is a $\lambda$-system.
\end{lemma}
\begin{proof}[Proof of the lemma]
    We verify that $\cL_A$ is closed under complements. Fix $B \in \cL_A$. We have $X\setminus A \in \cL$ and $A\cap B\in \cL$. Since $\cL$ is $\lambda$-system, we obtain
    $$\left(X\setminus A\right)\cup \left(A\cap B\right)\in \cL$$
    and thus
    $$A\setminus B = X\setminus \bigg(\left(X\setminus A\right)\cup \left(A\cap B\right)\bigg) \in \cL$$
    This proves that $X\setminus B \in \cL_A$ and this proves that $\cL_A$ is closed under complements. Since $A\in cL$, we derive that $X\in \cL_A$ and the fact that $\cL_A$ is closed under disjoint countable unions is straightforward.
\end{proof}

\begin{proof}[Proof of the theorem]
    Fix $A\in \cP$. Define $\cL_A$ as in Lemma \ref{lemma:dynkins_lemma} with $\cL = \lambda(\cP)$. Then $\cL_A$ is a $\lambda$-system. Moreover, $\cL_A$ contains $\cP$. Hence $\lambda(\cP)\subseteq \cL_A$. This shows that $\lambda(\cP)$ is closed under intersections with members of $\cP$. Now fix $A\in \lambda(\cP)$ and define $\cL_A$ as in Lemma \ref{lemma:dynkins_lemma} with $\cL = \lambda(\cP)$. Then $\cP \subseteq \cL_A$ and $\cL_A$ is a $\lambda$-system. Thus $\lambda(\cP)\subseteq \cL_A$. This proves that $\lambda(\cP)$ is a $\pi$-system. A $\pi$-system that is simultaneously a $\lambda$-system is a $\sigma$-algebra. Thus $\sigma(\cP)\subseteq \lambda(\cP)$. Since it is clear that $\lambda(\cP) \subseteq \sigma(\cP)$, we derive that $\lambda(\cP)=\sigma(\cP)$.
\end{proof}
\noindent
The next result is similar to Dynkin's theorem, but concerns monotone classes.

\begin{theorem}[Sierpi{\'n}ski's lemma on monotone classes]\label{theorem:monotone_classes}
    Let $X$ be a set and let $\cA$ be an algebra of its subsets. Then $\cM(\cA)=\sigma(\cA)$.
\end{theorem}
\noindent
For the proof we need the following easy results. Their proofs are left to the reader.

\begin{lemma}\label{lemma:first_monotone_classes}
    Let $\cM$ be a monotone family. Then for every $A \in \cM$ family
    $$\cM_A = \big\{B\subseteq X\,\big|\,A\cap B\in \cM\big\}$$
    is monotone.
\end{lemma}

\begin{lemma}\label{lemma:second_monotone_classes}
    Let $\cM$ be a monotone family. Then a family
    $$\cM^c= \big\{A \subseteq X \,\big|\,A\in \cM\mbox{ and }X\setminus A\in \cM\big\}$$
    is monotone.
\end{lemma}

\begin{proof}[Proof of the theorem]
    Fix $A\in \cA$. Define $\cM_A$ as in Lemma \ref{lemma:first_monotone_classes} with $\cM = \cM(\cA)$. Then $\cM_A$ is a monotone family. Moreover, $\cM_A$ contains $\cA$. Hence $\cM(\cA)\subseteq \cM_A$. This shows that $\cM(\cA)$ is closed under intersections with members of $\cA$. Now fix $A\in \cM(\cA)$ and define $\cM_A$ as in Lemma \ref{lemma:first_monotone_classes} with $\cM = \cM(\cA)$. Then $\cA \subseteq \cM_A$ and $\cM_A$ is a monotone family. Thus $\cM(\cA)\subseteq \cM_A$. This proves that $\cM(\cA)$ is closed under finite intersections. According to Lemma \ref{lemma:second_monotone_classes} we derive that $\cM(\cA)^c$ is a monotone family and contains $\cA$. Hence $\cM(\cA)\subseteq \cM(\cA)^c$ and thus $\cM(\cA)$ is closed under complements. Therefore, $\cM(\cA)$ is a $\sigma$-algebra. Thus $\sigma(\cA)\subseteq \cM(\cA)$. Since it is clear that $\cM(\cA) \subseteq \sigma(\cA)$, we derive that $\cM(\cA) = \sigma(\cA)$.
\end{proof}

\begin{definition}
    Let $(X,\Sigma,\mu)$ be a space with measure.
    \begin{enumerate}[label=\textbf{(\arabic*)}, leftmargin=*]
        \item $\mu$ is \textit{finite} if $\mu(X) \in \RR$.
        \item $\mu$ is \textit{$\sigma$-finite} if there exists a sequence $\{X_n\}_{n\in \NN}$ in $\Sigma$ such that
              $$X = \bigcup_{n\in \NN}X_n$$
              and $\mu(X_n)\in \RR$ for each $n\in \NN$.
    \end{enumerate}
\end{definition}

\begin{theorem}\label{theorem:uniqueness_of_sigma_finite_measure_on_pi_system}
    Let $(X,\Sigma)$ be a measurable space and let $\mu_1,\mu_2$ be measures on $(X,\Sigma)$. Suppose that $\{X_n\}_{n\in \NN}$ is a partition of $X$ onto sets in $\Sigma$ such that $\mu_1(X_n),\mu_2(X_n)$ are finite and equal for every $n \in \NN$. Let $\cP$ be a $\pi$-system of subsets of $X$ such that 
    $$\mu_1(P\cap X_n) = \mu_2(P\cap X_n)$$
    for every $P \in \cP$ and $n \in \NN$. Then $\mu_1(A) = \mu_2(A)$ for every $A \in \sigma(\cP)$.
\end{theorem}
\begin{proof}
    We set
    $$\cF_n = \big\{A\in \Sigma\,\big|\,A\in \Sigma \mbox{ and }\mu_1(A\cap X_n)=\mu_2(A\cap X_n) \big\}$$
    Then $\cF_n$ is a $\lambda$-system of subsets of $X$ and $\cP \subseteq \cF_n$ for each $n \in \NN$. By Theorem \ref{theorem:dynkins_lemma} we deduce that $\sigma(\cP) \subseteq \cF_n$ for every $n \in \NN$. Pick $A \in \sigma(\cP)$. Then
    $$\mu_1(A) = \sum_{n\in \NN}\mu_1(A\cap X_n) = \sum_{n\in \NN}\mu_2(A\cap X_n) = \mu_2(A)$$
    This completes the proof.
\end{proof}


\section{Induction for families of functions}
\noindent
The following sequence of results is a useful tool for studying classes of functions in integration theory.

\begin{proposition}\label{proposition:measurable_induction_for_nonnegative}
    Let $(X,\Sigma)$ be a measurable space and let $\cF$ be a family of functions defined on $X$ and with values in $\ol{\RR}$. Suppose that the following assertions hold.
    \begin{enumerate}[label=\emph{\textbf{(\arabic*)}}, leftmargin=*]
        \item $\mathbb{1}_A\in \cF$ for every $A\in \Sigma$.
        \item $\cF$ is closed under $\RR$-linear combinations of nonnegative functions with nonnegative coefficients.
        \item $\cF$ is closed under pointwise limits of nondecreasing sequences of nonnegative functions.
    \end{enumerate}
    Then $\cF$ contains all nonnegative, measurable functions on $X$ with values in $\ol{\RR}$.
\end{proposition}
\begin{proof}
    By \textbf{(1)} and \textbf{(2)} family $\cF$ contains all nonnegative and real valued, measurable functions with finite target. Since \cite{Integration} proves that every measurable nonnegative function is pointwise limit of a nondecreasing sequence of nonnegative and real valued, measurable functions with finite target, \textbf{(3)} implies that $\cF$ contains all nonnegative, measurable functions on $X$ with values in $\ol{\RR}$.
\end{proof}

\begin{proposition}\label{proposition:measurable_induction_for_complex}
    Let $(X,\Sigma,\mu)$ be a space with measure and let $\cF$ be a family of complex valued, $\mu$-integrable functions defined on $X$. Suppose that the following assertions hold.
    \begin{enumerate}[label=\emph{\textbf{(\arabic*)}}, leftmargin=*]
        \item $\mathbb{1}_A\in \cF$ for every $A\in \Sigma$ with $\mu(A)\in \RR$.
        \item If $f, g\in \cF$ and $\alpha, \beta\in \CC$, then
              $$\alpha f + \beta g\in \cF$$
        \item If $\{f_n:X\ra \CC\}_{n\in \NN}$ is a nondecreasing sequence of nonnegative functions in $\cF$ which converges to $\mu$-integrable function $f$, then $f\in \cF$.
    \end{enumerate}
    Then $\cF$ is $L^1(\mu, \CC)$.
\end{proposition}
\begin{proof}
    By \textbf{(1)} and \textbf{(2)} family $\cF$ contains all $\mu$-simple functions. In particular, it contains all nonnegative, $\mu$-simple functions. According to \textbf{(3)} and the fact from \cite{Integration} that every nonnegative, $\mu$-integrable and real valued function is pointwise limit of a nondecreasing sequence of $\mu$-simple functions, we derive that $\cF$ contains all nonnegative, $\mu$-integrable functions. Suppose now that $f:X\ra \CC$ is real valued and $\mu$-integrable. Then $f_+ = \sup \{f, 0\}$ and $f_- = \sup\{-f, 0\}$ are $\mu$-integrable and nonnegative. Hence they are elements of $\cF$. By \textbf{(2)} we deduce that $f = f_+ - f_-$ is in $\cF$. Finally, if $f:X\ra \CC$ is an arbitrary function in $L^1(\mu, \CC)$, then we write $f = f_r + i\cdot f_i$, where $f_r, f_i$ are real valued and $\mu$-integrable. Then by previous considerations $f_r,f_i\in \cF$ and hence $f\in \cF$ as their $\CC$-linear combination.
\end{proof}

\begin{proposition}\label{proposition:measurable_induction_for_banach_valued}
    Fix a positive real number $p$. Let $(X,\Sigma,\mu)$ be a space with measure and let $Y$ be a Banach space over a field $\mathbb{K}$ with absolute value. Suppose that $\cF$ is a family of $\mu$-integrable, $Y$-valued functions defined on $X$. Suppose that the following assertions hold.
    \begin{enumerate}[label=\emph{\textbf{(\arabic*)}}, leftmargin=*]
        \item $y\cdot \mathbb{1}_A\in \cF$ for every $y\in Y$ and $A\in \Sigma$ with $\mu(A)\in \RR$.
        \item If $f, g\in \cF$ and $\alpha, \beta\in \mathbb{K}$, then
              $$\alpha f + \beta g\in \cF$$
        \item Suppose that $\{f_n:X\ra Y\}_{n\in \NN}$ is a sequence of functions in $\cF$ and $g:X\ra \ol{\RR}$ is a nonnegative, $\mu$-integrable function such that $\lVert f_n\rVert^p \leq g$ for every $n\in \NN$. Let $f$ be a pointwise limit of $\{f_n\}_{n\in \NN}$. Then $f\in \cF$.
    \end{enumerate}
    Then $\cF$ is $L^p(\mu,Y)$.
\end{proposition}
\begin{proof}
    By \textbf{(1)} and \textbf{(2)} family $\cF$ contains all $\mu$-simple functions. According to \cite{Integration} every $f \in L^p(\mu,Y)$ is a pointwise limit of a sequence of $\mu$-simple functions $\{s_n\}_{n\in \NN}$ such that
    $$\lVert s_n\rVert^p \leq g$$
    for some $\mu$-integrable and nonnegative function $g:X\ra \ol{\RR}$. Hence \textbf{(3)} implies that $\cF$ contains every element of $L^p(\mu,Y)$.
\end{proof}

\section{Product measures}
\noindent
In this section we discuss integration on the product of spaces with measures.

\begin{fact}\label{fact:product_algebra}
    Let $(X_1,\Sigma_1)$ and $(X_2,\Sigma_2)$ be measurable spaces. Consider a class of sets which are finite disjoint unions of sets from $\big\{A_1\times A_2\big\}_{A_1\in \Sigma_1,A_2\in \Sigma_2}$. Then this class of sets is an algebra of subsets of $X_1\times X_2$.
\end{fact}
\begin{proof}
    Left to the reader as an exercise.
\end{proof}

\begin{definition}
    Let $(X_1,\Sigma_1)$ and $(X_2,\Sigma_2)$ be measurable spaces. Let $\Sigma_1\times \Sigma_2$ be the algebra of subsets of $X_1\times X_2$ which are finite disjoint unions of sets of the form $A_1\times A_2$ for some $A_1 \in \Sigma_1,A_2\in \Sigma_2$. Then $\Sigma_1\times \Sigma_2$ is \textit{the product algebra of $\Sigma_1$ and $\Sigma_2$}. 
\end{definition}

\begin{definition}
    Let $(X_1,\Sigma_1)$ and $(X_2,\Sigma_2)$ be measurable spaces. A $\sigma$-algebra $\Sigma_1\otimes \Sigma_2$ generated by $\Sigma_1\times \Sigma_2$ is \textit{the product $\sigma$-algebra of $\Sigma_1$ and $\Sigma_2$}.
\end{definition}
\noindent
Suppose that $Y$ is a set and $f:X_1\times X_2\ra Y$ is a function. For every $x_1\in X_1$ we define a function $f_{x_1}:X_2\ra Y$ by formula
$$f_{x_1}(x) = f(x_1,x)$$
for every $x \in X_2$. Similarly for every $x_2\in X_2$ we define a function $f_{x_2}:X_1\ra Y$ by formula
$$f_{x_2}(x) = f(x,x_2)$$
for every $x \in X_1$. There is also a version of this notation for sets. Let $E\subseteq X_1\times X_2$ be a subset. Then we define
$$E_{x_1} = \{x\in X_2\,|\,(x_1,x)\in E\},\,E_{x_2} = \{x\in X_1\,|\,(x,x_2)\in E\}$$
for each $x_1\in X_1$ and $x_2\in X_2$. Note that
$$\mathbb{1}_{E_{x_1}} = \left(\mathbb{1}_E\right)_{x_1},\,\mathbb{1}_{E_{x_2}} = \left(\mathbb{1}_E\right)_{x_2}$$

\begin{proposition}\label{proposition:measurable_functions_have_measurable_sections}
    Let $(X_1,\Sigma_1), (X_2,\Sigma_2)$ be measurable spaces. Then the following assertions hold.
    \begin{enumerate}[label=\emph{\textbf{(\arabic*)}}, leftmargin=*]
        \item For every function $f:X_1\times X_2\ra \ol{\RR}$ measurable with respect to $\Sigma_1\otimes \Sigma_2$ and any $x_1\in X_1,x_2\in X_2$ function $f_{x_1}$ is measurable with respect to $\Sigma_2$ and function $f_{x_2}$ is measurable with respect to $\Sigma_1$.
        \item Let $Y$ be a Banach space over a field $\mathbb{K}$ with absolute value. For every function $f:X_1\times X_2\ra Y$ strongly measurable with respect to $\Sigma_1\otimes \Sigma_2$ and any $x_1\in X_1,x_2\in X_2$ function $f_{x_1}$ is strongly measurable with respect to $\Sigma_2$ and $f_{x_2}$ is strongly measurable with respect to $\Sigma_1$.
    \end{enumerate}
\end{proposition}
\begin{proof}
    First let $\cS$ be a family of all subsets $E$ in $\Sigma_1\otimes \Sigma_2$ such that $E_{x_1}\in \Sigma_2$ and $E_{x_2}\in \Sigma_1$ for every $x_1\in X_1$ and $x_2\in X_2$. Then $\Sigma_1\times \Sigma_2\subseteq \cS$ and $\cS$ is a monotone family. Thus by Sierpiński's theorem on monotone classes we have $\Sigma_1\otimes \Sigma_2\subseteq \cS$.

    Now we prove the first assertion. Let $\cF$ be a family of all functions $f:X_1\times X_2\ra \ol{\RR}$ such that $f_{x_1}$ is measurable with respect to $\Sigma_2$ and $f_{x_2}$ is measurable with respect to $\Sigma_1$ for every $x_1\in X_1,x_2\in X_2$. Since $\Sigma_1\otimes \Sigma_2\subseteq \cS$, we derive that $\cF$ contains $\mathbb{1}_E$ for every $E\in \Sigma_1\otimes \Sigma_2$. Thus the intersection of $\cF$ with nonnegative, $\ol{\RR}$-valued functions on $X_1\times X_2$ satisfy all conditions of Proposition \ref{proposition:measurable_induction_for_nonnegative} and hence $\cF$ contains all nonnegative, $\Sigma_1\otimes \Sigma_2$-measurable functions with values in $\ol{\RR}$. Now suppose that $f:X_1\times X_2\ra \ol{\RR}$ is $\Sigma_1\otimes \Sigma_2$-measurable. Write $f_+ = \sup\{f, 0\}$ and $f_- = \sup\{-f, 0\}$. Then $f = f_+ - f_-$ and both functions $f_+,f_-:X_1\times X_2\ra \ol{\RR}$ are measurable with respect to $\Sigma_1\otimes \Sigma_2$ and nonnegative. Thus $f_+,f_-\in \cF$. Hence also $f\in \cF$. This proves \textbf{(1)}.

    Now we prove \textbf{(2)}. Let $\cF$ be a family of all functions $f:X_1\times X_2\ra Y$ such that $f_{x_1}$ is measurable with respect to $\Sigma_2$ and $f_{x_2}$ is measurable with respect to $\Sigma_1$ for every $x_1\in X_1,x_2\in X_2$. As above we can derive that for every $y\in Y$ and for every $E\in \Sigma_1\otimes \Sigma_2$ we have $y\cdot \mathbb{1}_E\in \cF$. Moreover, $\cF$ is a $\mathbb{K}$-vector space with respect to pointwise operations. Hence $\cF$ contains every $\Sigma_1\otimes \Sigma_2$-measurable function $s:X_1\times X_2\ra Y$ such that $s(X_1\times X_2)$ is finite. Next by \cite{Integration} for every strongly $\Sigma_1\otimes \Sigma_2$-measurable function $f:X_1\times X_2\ra Y$ there exists a sequence $\{s_n:X_1\times X_2\ra Y\}_{n\in \NN}$ of strongly $\Sigma_1\otimes \Sigma_2$-measurable functions such that $s_n(X_1\times X_2)$ is finite for every $n\in \NN$ and
    $$f = \lim_{n\ra +\infty}s_n$$
    Since $\cF$ is closed under pointwise limits, we derive that $f$ is in $\cF$.
\end{proof}

\begin{theorem}\label{theorem:existence_of_product_measure}
    Let $(X,\Sigma_1,\mu_1)$ and $(X_2,\Sigma_2,\mu_2)$ be spaces with measures. Suppose that both $\mu_1$ and $\mu_2$ are $\sigma$-finite. Then the following assertions hold.
    \begin{enumerate}[label=\emph{\textbf{(\arabic*)}}, leftmargin=*]
        \item For every $E\in \Sigma_1\otimes \Sigma_2$ function
              $$X_1\ni x_1\mapsto \mu_2(E_{x_1})\in \ol{\RR}$$
              is measurable with respect to $\Sigma_1$.
        \item For every $E\in \Sigma_1\otimes \Sigma_2$ function
              $$X_2\ni x_2\mapsto \mu_1(E_{x_2})\in \ol{\RR}$$
              is measurable with respect to $\Sigma_2$.
        \item There exists a unique measure $\mu_1\otimes \mu_2$ defined on $\Sigma_1\otimes \Sigma_2$ such that
              $$\left(\mu_1\otimes \mu_2\right)\left(A_1\times A_2\right) = \mu_1(A_1)\mu_2(A_2)$$
              for $A_1\in \Sigma_1, A_2\in \Sigma_2$.
        \item $\mu_1\otimes \mu_2$ is $\sigma$-finite.
        \item For every $E\in \Sigma_1\otimes \Sigma_2$ we have
              $$\int_{X_1}\mu_2(E_{x_1})\,d\mu_1 = \left(\mu_1\otimes \mu_2\right)(E) = \int_{X_2}\mu_1(E_{x_2})\,d\mu_2$$
    \end{enumerate}
\end{theorem}
\begin{proof}
    Since $\mu_1$ and $\mu_2$ are $\sigma$-finite, we can fix partitions
    $$X_1 = \bigcup_{n\in \NN}X_{1,n},\,X_2 = \bigcup_{n\in \NN}X_{2,n}$$
    such that $X_{1,n}\in \Sigma_1,X_{2,n}\in \Sigma_2$ and $\mu_1(X_{1,n})\in \RR,\mu_2(X_{2,n})\in \RR$ for every $n\in \NN$. For every $E \in \Sigma_1\otimes \Sigma_2$ and all pairs $n,m\in \NN$ denote $E\cap (X_{1,n}\times X_{2,m})$ by $E_{n,m}$.

    For every $E$ in $\Sigma_1\otimes \Sigma_2$ we denote by $f_E$ the function
    $$X_1\ni x_1 \mapsto \mu_2\left(E_{x_1}\right)\in \ol{\RR}$$
    This function is well defined according to Proposition \ref{proposition:measurable_functions_have_measurable_sections}. For all $n,m\in \NN$ define 
    $$\cF_{n,m} = \big\{E \in \Sigma_1\otimes \Sigma_2 \big|\,f_{E_{n,m}}\mbox{ is measurable with respect to }\Sigma_1\big\}$$
    Note that $\cF_{n,m}$ is a monotone class and by straightforward calculation $\Sigma_1\times \Sigma_2\subseteq \cF_{n,m}$ for all $n,m\in \NN$. Theorem \ref{theorem:monotone_classes} implies that $\cF_{n,m}$ coincides with $\Sigma_1\otimes \Sigma_2$ for every $n,m \in \NN$. Since
    $$f_{E} = \sum_{n,m\in \NN}f_{E_{n,m}}$$
    for every $E \in \Sigma_1\otimes \Sigma_2$, we derive that $f_E$ is measurable with respect to $\Sigma_1$ for each $E \in \Sigma_1\otimes \Sigma_2$. This proves \textbf{(1)} and by symmetry also \textbf{(2)}.
    
    Now by \textbf{(1)} it makes sense to define
    $$(\mu_1\otimes \mu_2)(E) = \int_{X_1}\mu_2(E_{x_1})\,d\mu_1$$
    for every $E\in \Sigma_1\otimes \Sigma_2$. Clearly $(\mu_1\otimes \mu_2)(\emptyset) = 0$ and if $\{E_n\}_{n\in \NN}$ is a family of disjoint subsets in $\Sigma_1\otimes \Sigma_2$, then by monotone convergence theorem we have
    $$(\mu_1\otimes \mu_2)\left(\bigcup_{n\in \NN}E_n\right) = \sum_{n\in \NN}(\mu_1\otimes \mu_2)(E_n)$$
    Hence $\mu_1\otimes \mu_2$ is a measure on $\Sigma_1\otimes \Sigma_2$. We also have
    $$(\mu_1\otimes \mu_2)\left(A_1\times A_2\right) = \int_{X_1}\mu_2(A_2)\mathbb{1}_{A_1}\,d\mu_1 = \mu_1(A_1)\mu_2(A_2)$$
    for every $A_1\in \Sigma_1, A_2\in \Sigma_2$. This proves the existence. The uniqueness is a direct consequence of Theorem \ref{theorem:uniqueness_of_sigma_finite_measure_on_pi_system}. It suffices to notice that
    $$X_1\times X_2 = \bigcup_{n\in \NN}\bigcup_{m\in \NN}X_{1,n}\times X_{2,m}$$
    and
    $$(\mu_1\otimes \mu_2)\left((A_1\times A_2)_{n,m}\right) = \mu_1(A_1\cap X_n)\mu_2(A_2\cap X_m) \in \RR$$
    for all $A_1 \in \Sigma_1,A_2\in \Sigma_2$ and $n,m\in \NN$. This completes the proof of \textbf{(3)} and also shows \textbf{(4)}.
    
    Finally by symmetry we derive that
    $$\Sigma_1\otimes \Sigma_2\ni E \mapsto \int_{X_2}\mu_1(E_{x_2})\,d\mu_2\in [0,+\infty]$$
    is a measure on $\Sigma_1\otimes \Sigma_2$ which takes exactly the same values on sets $\big\{A_1\times A_2\big\}_{A_1\in \Sigma_1,A_2\in \Sigma_2}$ as $\mu_1\otimes \mu_2$. By uniqueness of $\mu_1\otimes \mu_2$ we have
    $$(\mu_1\otimes \mu_2)(E) = \int_{X_2}\mu_1(E_{x_2})\,d\mu_2$$
    This proves \textbf{(5)}.
\end{proof}

\begin{definition}
    Let $(X,\Sigma_1,\mu_1)$ and $(X_2,\Sigma_2,\mu_2)$ be spaces with $\sigma$-finite measures. The unique measure $\mu_1\otimes \mu_2$ on $\Sigma_1\otimes \Sigma_2$ such that
    $$\left(\mu_1\otimes \mu_2\right)\left(A_1\times A_2\right) = \mu_1(A_1)\mu_2(A_2)$$
    for every $A_1\in \Sigma_1, A_2\in \Sigma_2$ is \textit{the product measure of $\mu_1$ and $\mu_2$}.
\end{definition}

\section{Tonelli and Fubini theorems}
\noindent
Next results relate integration with respect to product measures to iterated integration with respect to their factors.

\begin{theorem}[Tonelli]\label{theorem:Tonelli}
    Let $(X,\Sigma_1,\mu_1)$ and $(X_2,\Sigma_2,\mu_2)$ be spaces with measures. Suppose that both $\mu_1$ and $\mu_2$ are $\sigma$-finite. Let $f:X_1\times X_2\ra \ol{\RR}$ be a nonnegative function measurable with respect to $\Sigma_1\otimes \Sigma_2$. Then functions
    $$X_1\ni x_1 \mapsto \int_{X_2}f_{x_1}\,d\mu_2\in \ol{\RR}$$
    and
    $$X_2\ni x_2 \mapsto \int_{X_1}f_{x_2}\,d\mu_1\in \ol{\RR}$$
    are measurable with respect to $\Sigma_1$ and $\Sigma_2$, respectively. Moreover, we have equality
    $$\int_{X_1}\int_{X_2}f_{x_1}\,d\mu_2d\mu_1 = \int_{X_1\times X_2}f\,d(\mu_1\otimes \mu_2) = \int_{X_2}\int_{X_1}f_{x_2}\,d\mu_1d\mu_2$$
\end{theorem}
\begin{proof}
    Let $\cF$ be a family of all nonnegative functions $f:X_1\times X_2\ra \ol{\RR}$ that are measurable with respect to $\Sigma_1\otimes \Sigma_2$ such that functions
    $$X_1\ni x_1\mapsto \int_{X_2}f_{x_1}\,d\mu_2\in \ol{\RR},\,X_2\ni x_2\mapsto \int_{X_1}f_{x_2}\,d\mu_1\in \ol{\RR}$$
    are measurable with respect to $\Sigma_1, \Sigma_2$, respectively, and the formula
    $$\int_{X_1}\int_{X_2}f_{x_1}\,d\mu_2d\mu_1 = \int_{X_1\times X_2}f\,d(\mu_1\otimes \mu_2) = \int_{X_2}\int_{X_1}f_{x_2}\,d\mu_1d\mu_2$$
    holds. Then $\cF$ is closed under linear combinations of its elements with nonnegative coefficients. Next if $\{f_n:X_1\times X_2\ra \ol{\RR}\}_{n\in \NN}$ is a nondecreasing sequence of elements of $\cF$, then
    $$\lim_{n\ra +\infty}f_n\in \cF$$
    by monotone convergence theorem. Finally $\mathbb{1}_E\in \cF$ for every $E\in \Sigma_1\otimes \Sigma_2$ by Theorem \ref{theorem:existence_of_product_measure}. According to Proposition \ref{proposition:measurable_induction_for_nonnegative} we derive that $\cF$ consists of all nonnegative functions measurable with respect to $\Sigma_1\otimes \Sigma_2$.
\end{proof}

\begin{theorem}[Fubini]\label{theorem:fubini}
    Let $(X,\Sigma_1,\mu_1)$ and $(X_2,\Sigma_2,\mu_2)$ be spaces with measures and let $Y$ be a Banach space over $\RR$ or $\CC$. Suppose that both $\mu_1$ and $\mu_2$ are $\sigma$-finite. Let $f:X_1\times X_2\ra Y$ be a function integrable with respect to $\mu_1\otimes \mu_2$. Then there are sets $N_i$ in $\Sigma_i$ for $i=1,2$ such that
    $$\mu_1(N_1) = \mu_2(N_2) = 0$$
    and functions
    $$X_1 \ni x_1\mapsto \int_{X_2}\left(\mathbb{1}_{\left(X_1\setminus N_1\right)\times X_2}\right)_{x_1}\cdot f_{x_1}\,d\mu_2\in Y,\,X_2 \ni x_2\mapsto \int_{X_1}\left(\mathbb{1}_{X_1\times \left(X_2\setminus N_2\right)}\right)_{x_2}\cdot f_{x_2}\,d\mu_1\in Y$$
    are well defined and integrable with respect to $\mu_1, \mu_2$, respectively. Moreover, we have equality
    $$\int_{X_1}\int_{X_2} \left(\mathbb{1}_{\left(X_1\setminus N_1\right)\times X_2}\right)_{x_1}\cdot f_{x_1}\,d\mu_2d\mu_1 = \int_{X_1\times X_2}f\,d(\mu_1\otimes \mu_2) = \int_{X_2}\int_{X_1} \left(\mathbb{1}_{X_1\times \left(X_2\setminus N_2\right)}\right)_{x_2}\cdot f_{x_2}\,d\mu_1d\mu_2$$
\end{theorem}
\begin{proof}
    Let $\cF$ be a family of all $\left(\mu_1\otimes \mu_2\right)$-integrable functions $f:X_1\times X_2\ra Y$ such that the statement holds for $f$. Then according to Theorem \ref{theorem:existence_of_product_measure} for every $y\in Y$ and $E\in \Sigma_1\otimes \Sigma_2$ such that $\left(\mu_1\otimes \mu_2\right)(E) \in \RR$ we have $y\cdot \mathbb{1}_E \in \cF$. Moreover, if $f, g\in \cF$, then for scalars $\alpha,\beta$ we have $\alpha f + \beta g\in \cF$. Suppose that $\{f_n:X_1\times X_2\ra Y\}_{n\in \NN}$ is a sequence of functions in $\cF$ which is pointwise convergent and $g:X_1\times X_2\ra \ol{\RR}$ is a nonnegative measurable function such that
    $$\int_{X_1\times X_2}g\,d(\mu_1\otimes \mu_2) \in \RR$$
    and $\lVert f_n\rVert \leq g$ holds for every $n\in \NN$. Let $f$ be pointwise limit of $\{f_n\}_{n\in \NN}$. Then by dominated convergence theorem and Theorem \ref{theorem:Tonelli} we have $f\in \cF$. From Proposition \ref{proposition:measurable_induction_for_banach_valued} we derive that $\cF$ contains all $\left(\mu_1\otimes \mu_2\right)$-integrable functions.
\end{proof}


\small
\bibliographystyle{apalike}
\bibliography{../zzz}


\end{document}