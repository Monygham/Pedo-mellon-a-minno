\input ../pree.tex

\begin{document}

\title{Lebesgue spaces and their duals}
\date{}
\maketitle


\section{Introduction}
\noindent
In these notes 



\section{Dual spaces to $L^p$ for $p \in (1,+\infty)$}
\noindent
Let $(X,\Sigma,\mu)$ be a space with measure and let $p$ be a real in $(1,+\infty)$. Define $q \in (1,+\infty)$ to be the unique number which satisfies
$$\frac{1}{p} + \frac{1}{q} = 1$$
Assume that $\mathbb{K}$ is either $\RR$ or $\CC$ with usual absolute value. We start by proving the following result.

\begin{proposition}\label{proposition:norm_of_standard_functional_for_positive_p}
Let $g$ be a function in $L^q(\mu,\mathbb{K})$. Then
$$\lVert g\rVert_q = \sup \bigg\{\left|\int_X g\cdot f\,d\mu\right|\in \RR_+\cup \{0\}\,\bigg|\,f\in L^p(\mu,\mathbb{K})\mbox{ such that }\lVert f\rVert_p = 1\bigg\}$$
\end{proposition}
\begin{proof}
According to H{\"o}lder inequality
$$\left|\int_X g\cdot f\,d\mu\right| \leq \int_X |g|\cdot |f|\,d\mu \leq \lVert g\rVert_q \cdot \lVert f\rVert_p$$
Thus for $f \in L^p(\mu,\mathbb{K})$ such that $\lVert f\rVert_p = 1$ we have
$$\left|\int_X g\cdot f\,d\mu\right| \leq \lVert g\rVert_q$$
Therefore, it suffices to prove that 
$$\lVert g\rVert_q \leq \sup \bigg\{\left|\int_X g\cdot f\,d\mu\right|\in \RR_+\cup \{0\}\,\bigg|\,f\in L^p(\mu,\mathbb{K})\mbox{ such that }\lVert f\rVert_p = 1\bigg\}$$
under the assumption that $\lVert g \rVert_q \neq 0$. Define
$$f(x) = \begin{cases}
\lVert g \rVert_q^{1 - q}\cdot \frac{|g(x)|^q}{g(x)}&\mbox{ if }g(x) \neq 0\\
0  
\end{cases}
$$
Then $f \in L^p(\mu,\mathbb{K})$ and even more precisely we have
$$\lVert f \rVert_p = \left(\int_X\lVert g \rVert_q^{(1 - q)\cdot p}\cdot |g|^{(q-1)\cdot p}\,d\mu\right)^{\frac{1}{p}} = \left(\int_X\lVert g \rVert_q^{-q}\cdot |g|^{q}\,d\mu\right)^{\frac{1}{p}} = \left(\lVert g \rVert_q^{-q}\cdot \int_X\cdot |g|^{q}\,d\mu\right)^{\frac{1}{p}} = 1
$$
Note that
$$\left|\int_X g\cdot f\,d\mu\right| = \int_X \lVert g \rVert_q^{(1 - q)}\cdot |g|^{q}\,d\mu = \lVert g \rVert_q^{(1 - q)}\cdot \int_X |g|^{q}\,d\mu = \lVert g \rVert_q^{(1 - q)}\cdot \lVert g \rVert_q^{(q)} = \lVert g \rVert_q $$
and this completes the proof.
\end{proof}
\noindent
The following theorem is the main result of this section.

\begin{theorem}\label{theorem:dual_to_L_p_for_p_in_(1_00)}
Let $\Lambda:L^p(\mu,\mathbb{K})\ra \mathbb{K}$ be a continuous $\mathbb{K}$-linear map. Then there exists $g \in L^q(\mu,\mathbb{K})$ such that 
$$\Lambda(f) = \int_Xg\cdot f\,d\mu$$
for every $f \in L^p(\mu,\mathbb{K})$. Moreover, $g$ is uniquely defined up to a set of measure $\mu$ equal to zero.
\end{theorem}
\noindent
We start by the following observation.

\begin{lemma}\label{lemma:restriction_of_functionals_lemma}
Let $\Lambda:L^p(\mu,\RR)\ra \RR$ be a continuous $\RR$-linear map. For each set $S \in \Sigma$ we define
$$\Lambda_S(f) = \Lambda\left(\mathbb{1}_S\cdot f\right)$$
for every $f \in L^p(\mu,\RR)$. Then the following assertions hold.
\begin{enumerate}[label=\emph{\textbf{(\arabic*)}}, leftmargin=*]
\item $\Lambda_S:L^p(\mu,\RR)\ra \RR$ is a continuous $\RR$-linear map.
\item Now if $S\subseteq T$ are two sets in $\Sigma$, then
$$\lVert \Lambda_S\rVert \leq \lVert \Lambda_T\rVert \leq \lVert \Lambda \rVert$$
\item There exists a $\sigma$-finite subset $S$ in $\Sigma$ such that $\lVert \Lambda_S\rVert = \lVert \Lambda \rVert$.
\end{enumerate}  
\end{lemma}
\begin{proof}[Proof of the lemma]
Assertions \textbf{(1)} and \textbf{(2)} are left for the reader as an exercises.\\
We prove \textbf{(3)}. Suppose that $f \in L^p(\mu, \RR)$ satisfies $\lVert f \rVert_p \leq 1$. Then there exists a nondecreasing sequence $\{S_n\}_{n\in \NN}$ of sets in $\Sigma$ such that $\mu(S_n)$ is finite for every $n\in \NN$ and $\{\mathbb{1}_{S_n}\cdot f\}_{n\in \NN}$ converges to $f$ in $L^p(\mu, \RR)$. Hence 
$$\Lambda(f) = \lim_{n\ra +\infty}\Lambda_{S_n}\left(f\right)$$
It follows that
$$\lVert \Lambda \rVert = \sup \big\{\lVert \Lambda_S \rVert\,\big|\,S\in \Sigma\mbox{ such that }\mu(S)\mbox{ is finite }\}$$
Hence there exists a nondecreasing sequence $\{S_n\}_{n\in \NN}$ of sets in $\Sigma$ such that $\mu(S_n)$ is finite for every $n\in \NN$ and 
$$\lVert \Lambda \rVert = \lim_{n\ra +\infty}\lVert \Lambda_{S_n} \rVert$$
Then the union
$$S = \bigcup_{n \in \NN}S_n$$
is in $\Sigma$ is $\sigma$-finite and satisfies $\lVert \Lambda \rVert = \lVert \Lambda_S\rVert$. 
\end{proof}
\noindent
We prove the theorem by gradually considering more general cases.

\begin{proof}[Proof for finite $\mu$ and $\mathbb{K}=\RR$]
Assume that $\mu$ is finite measure and $\mathbb{K}$ is equal to $\RR$. We have finite signed measure
$$\Sigma \ni A \mapsto  \Lambda\left(\mathbb{1}_A\right) \in \RR$$
According to Radon-Nikodym there exists $g \in L^1(\mu,\RR)$ such that
$$\Lambda\left(\mathbb{1}_A\right) = \int_Xg\cdot \mathbb{1}_A\,d\mu$$
for every $A$ in $\Sigma$. It follows that
$$\Lambda(f) = \int_Xg\cdot f\,d\mu$$
for every $f \in L^{\infty}(\mu,\RR)$. For each $n \in \NN_+$ define $A_n = \big\{x\in X\,\big|\,|g(x)|\leq n\big\}$ and consider a measurable and bounded function $f_n:X\ra \RR$ given by formula
$$f_n(x) = \begin{cases}
\mathbb{1}_{A_n}(x)\cdot \frac{|g(x)|^q}{g(x)}&\mbox{ if }g(x) \neq 0\\
0&\mbox{ otherwise }  
\end{cases}
$$
Then
$$\int_X \mathbb{1}_{A_n} \cdot |g|^{q}\,d\mu = \int_X g \cdot f_n\,d\mu = \Lambda\left(f_n\right) \leq \lVert \Lambda \rVert\cdot \Vert f_n \rVert_p = $$
$$= \lVert \Lambda \rVert\cdot \Vert \mathbb{1}_{A_n}\cdot |g|^{q-1}\rVert_p = \lVert \Lambda \rVert\cdot \left(\int_X \mathbb{1}_{A_n}\cdot\left(|g|^{q-1}\right)^p\,d\mu\right)^{\frac{1}{p}} = \lVert \Lambda \rVert\cdot \left(\int_X \mathbb{1}_{A_n}\cdot |g|^{q}\,d\mu\right)^{\frac{1}{p}}$$
and thus
$$\left(\int_X \mathbb{1}_{A_n} \cdot |g|^{q}\,d\mu\right)^{\frac{1}{q}} \leq \lVert \Lambda \rVert$$
By monotone convergence we have
$$\lVert g \rVert_q = \lim_{n\ra +\infty}\left(\int_X \mathbb{1}_{A_n} \cdot |g|^{q}\,d\mu\right)^{\frac{1}{q}} \leq \lVert \Lambda \rVert$$
Hence $g \in L^q(\mu,\RR)$. It follows that 
$$L^p(\mu,\RR) \ni f \mapsto \int_{X}g\cdot f\,d\mu \in \RR$$
is continuous $\RR$-linear map, which coincides with $\Lambda$ on the space of $\mu$-simple functions. Since $\mu$-simple functions are dense in $L^p(\mu,\RR)$, we derive that
$$\Lambda(f) = \int_{X}g\cdot f\,d\mu$$
for every $f \in L^p(\mu,\RR)$. This completes the proof.
\end{proof}

\begin{proof}[Proof for $\sigma$-finite $\mu$ and $\mathbb{K}=\RR$]
Assume that $\mu$ is $\sigma$-finite measure and $\mathbb{K}$ is equal to $\RR$. Since $\mu$ is $\sigma$-finite, there exist a nondecreasing sequence $\{X_n\}_{n\in \NN}$ of sets in $\Sigma$ such that 
$$X = \bigcup_{n\in \NN}X_n$$
and $\mu(X_n)$ is finite for $n\in \NN$. According to the case considered above and Lemma \ref{lemma:restriction_of_functionals_lemma} for each $n \in \NN$ there exists $g_n\in L^q(\mu,\RR)$ such that
$$\Lambda_{X_n}\left(f\right) = \int_Xg_n\cdot f\,d\mu$$
for every $f \in L^p(\mu,\RR)$. We may also assume that ${g_n}_{\mid X\setminus X_n} = 0$ and ${g_{n+1}}_{\mid X_n} = {g_n}_{\mid X_n}$ for every $n \in \NN$. Let $g$ be a pointwise limit of a sequence $\{g_n\}_{n\in \NN}$. Then $g$ is a measurable real valued function on $X$. Moreover, we have $g_n = \mathbb{1}_{X_n}\cdot g$ for each $n \in \NN$. By Proposition \ref{proposition:norm_of_standard_functional_for_positive_p}, Lemma \ref{lemma:restriction_of_functionals_lemma} and monotone convergence we have
$$\lVert g \rVert_q = \lim_{n\ra +\infty}\lVert g_n\rVert_q = \lim_{n\ra +\infty}\lVert \Lambda_{X_n} \rVert \leq \lVert \Lambda \rVert$$
This implies that $g \in L^q(\mu,\RR)$. Fix $f \in L^p(\mu, \RR)$. Then sequence $\{\mathbb{1}_{X_n}\cdot f\}_{n \in \NN}$ converges to $f$ in $L^p(\mu,\RR)$ and hence
$$\Lambda(f) = \lim_{n\ra +\infty}\Lambda(\mathbb{1}_{X_n}\cdot f) = \lim_{n\ra +\infty}\Lambda_{X_n}(f)$$
On the other hand by dominated convergence theorem
$$\int_Xg\cdot f\,d\mu = \lim_{n\ra +\infty}\int_Xg_n\cdot f\,d\mu = \lim_{n\ra +\infty}\Lambda_{X_n}(f)$$
This completes the proof.
\end{proof}


\begin{proof}[Proof for $\mathbb{K} = \RR$]
According to Lemma \ref{lemma:restriction_of_functionals_lemma} there exists a $\sigma$-finite set $S$ in $\Sigma$ such that $\lVert \Lambda_S \rVert = \lVert \Lambda \rVert$. According to previous case there exists $g \in L^q(\mu,\RR)$ such that 
$$\Lambda(f) = \int_Xg\cdot f\,d\mu$$
for every $f \in L^p(\mu,\RR)$. We may also assume that $g_{\mid X\setminus S} = 0$. Suppose now that $T$ is a $\sigma$-finite set in $\Sigma$ such that $S\subseteq T$. Then there exists $g_T \in L^q(\mu,\RR)$ such that
$$\Lambda_{T}(f) = \int_Xg_T\cdot f\,d\mu$$
for every $f \in L^p(\mu,\RR)$. We may assume that $\mathbb{1}_S \cdot g_T = g$. Proposition \ref{proposition:norm_of_standard_functional_for_positive_p} implies that
$$\lVert \Lambda \rVert = \lVert \Lambda_S \rVert = \lVert g \rVert_q \leq \lVert h \rVert_q \leq \lVert \Lambda_{S\cup T}\rVert \leq \lVert \Lambda \rVert$$
Thus $\lVert g \rVert_q = \lVert g_T \rVert_q$ and this proves that we may assume that $g_T = g$. Fix now $f \in L^p(\mu,\RR)$ and consider
$$T = \big\{x\in X\,\big|\,f(x) \neq 0\big\} \cup S$$
Then $T$ is $\sigma$-finite set in $\Sigma$ and $S\subseteq T$. Hence
$$\Lambda(f) = \Lambda_T(f) = \int_Xg_T\cdot f\,d\mu = \int_Xg\cdot f\,d\mu$$
Since $f \in L^p(\mu,\RR)$ is arbitrary, the proof is completed.
\end{proof}

\begin{proof}[Proof for $\mathbb{K} = \CC$]
According to already proved case there exist $g_r,g_i\in L^q(\mu,\RR)$ such that
$$\mathrm{Re}\,\Lambda(f) = \int_X g_r\cdot f\,d\mu,\,\mathrm{Im}\,\Lambda(f) = \int_Xg_i\cdot f\,d\mu$$
for every $f \in L^p(\mu,\RR)$. Then $g = g_r + i\cdot g_r$ is a function in $L^q(\mu,\CC)$ and 
$$\Lambda(f) = \int_Xg\cdot f\,d\mu$$
for every $f \in L^p(\mu,\CC)$.
\end{proof}

\section{Dual to $L^1$}
\noindent
In this section we fix a space with measure $(X,\Sigma,\mu)$. Assume that $\mathbb{K}$ is either $\RR$ or $\CC$ with usual absolute value. We start by proving the following result.

\begin{proposition}\label{proposition:norm_of_standard_functional_for_infinity}
Let $g$ be a function in $L^1(\mu,\mathbb{K})$ and let $\Lambda:L^1\left(\mu,\mathbb{K}\right)\ra \mathbb{K}$ be a continuous map. Assume that
$$\Lambda(f) = \int_Xg\cdot f d\mu$$
for every $f \in L^1(\mu,\mathbb{K})$ and
$$L^1(\mu,\RR)\ni f \mapsto \int_Xg\cdot f\,d\mu\in \mathbb{K}$$
is continuous. Then 
$$\lVert g\rVert_{\infty} = \sup \bigg\{\left|\int_X g\cdot f\,d\mu\right|\in \RR_+\cup \{0\}\,\bigg|\,f\in L^1(\mu,\RR)\mbox{ such that }\lVert f\rVert_1 = 1\bigg\}$$
\end{proposition}
\begin{proof}
Suppose that
$$h(x) = \begin{cases}
\frac{|g(x)|}{g(x)}&\mbox{ if }g(x) \neq 0\\
0&\mbox{ otherwise}\\
\end{cases}
$$
Then $h$ is a bounded and measurable function. For $r \in \RR_+$ we define
$$A_{r} = \big\{x\in X\,\big|\,|g(x)| \geq r \big\}$$
If $\mu(A_r) > 0$ for some $r\in \RR_+$, then we also define
$$m_r = \int_X\mathbb{1}_{A_r}\cdot |g|\,d\mu,\,f_r = h\cdot \frac{1}{m_r}\cdot \mathbb{1}_{A_{r}}\cdot |g|$$
Then $f_r:X\ra \mathbb{K}$ is $\mu$-integrable and $\lVert f_r\rVert_1 = 1$. Let $L$ be a norm of a continuous $\mathbb{K}$-linear map
$$L^1(\mu,\mathbb{K})\ni f \mapsto \int_Xg\cdot f\,d\mu\in \mathbb{K}$$
Thus if $r \in \RR_+$ satisfies $\mu(A_r) > 0$, then
$$\left| \int_Xg\cdot f_r\,d\mu\right| \leq L$$
On the other hand
$$\left| \int_Xg\cdot f_r\,d\mu\right| = \left|\int_X \left(g\cdot h\right) \cdot \frac{1}{m_r}\cdot \mathbb{1}_{A_r}\cdot |g|\,d\mu\right| = $$
$$= \int_X \left(\mathbb{1}_{A_r}\cdot |g|\right) \cdot \frac{1}{m_r}\cdot \mathbb{1}_{A_r}\cdot |g|\,d\mu \geq r\cdot \int_X\frac{1}{m_r}\cdot \mathbb{1}_{A_r}\cdot |g|\,d\mu = r$$
This implies that $r \leq L$. We derive that $g$ is essentially bounded and $\lVert g\rVert_{\infty} \leq L$.
\end{proof}
\noindent
The following theorem is the main result of this section.

\begin{theorem}\label{theorem:dual_to_L_1}
Let $\Lambda:L^{1}(\mu,\mathbb{K})\ra \mathbb{K}$ be a continuous $\mathbb{K}$-linear map. Then there exists $g \in L^{\infty}(\mu,\mathbb{K})$ such that 
$$\Lambda(f) = \int_Xg\cdot f\,d\mu$$
for every $f \in L^1(\mu,\mathbb{K})$ and $\lVert g \rVert_{\infty} = \lVert \Lambda \rVert$. Moreover, $g$ is uniquely defined up to a set of measure $\mu$ equal to zero.
\end{theorem}
\noindent
We prove the theorem by gradually considering more general cases.

\begin{proof}[Proof for finite $\mu$ and $\mathbb{K}=\RR$]
Assume that $\mu$ is finite measure and $\mathbb{K}$ is equal to $\RR$. We have finite signed measure
$$\Sigma \ni A \mapsto  \Lambda\left(\mathbb{1}_A\right) \in \RR$$
According to Radon-Nikodym there exists $g \in L^1(\mu,\RR)$ such that
$$\Lambda\left(\mathbb{1}_A\right) = \int_Xg\cdot \mathbb{1}_A\,d\mu$$
for every $A$ in $\Sigma$. It follows that
$$\Lambda(f) = \int_Xg\cdot f\,d\mu$$
for every $f \in L^{\infty}(\mu,\RR)$. For every $r \in \RR_+$ define
$$A_r = \big\{x\in X\,\big|\,|g(x)|\geq r\big\},\,B_r = \big\{x\in X\,\big|\,|g(x)|\leq r\big\}$$
Fix $r\in \RR_+$ such that $\mu(A_r) > 0$. We define
$$f(x) = \begin{cases}
  \frac{1}{\mu(A_r)}\cdot \frac{|g(x)|}{g(x)}&\mbox{ if }x\in A_r\\
  0&\mbox{ otherwise }
\end{cases}
$$
Then $\lVert f\rVert_1 = 1$ and $f \in L^{\infty}(\mu,\RR)$. Note that 
$$r = \int_X r\cdot |f|\,d\mu \leq \int_X |g|\cdot |f|\,d\mu =\int_X g \cdot  f\,d\mu = \Lambda(f) \leq \lVert \Lambda\rVert$$
It follows that $\lVert g \rVert_{\infty} \leq \Lambda$. Therefore, $g \in L^{\infty}(\mu,\RR)$. It follows that 
$$L^1(\mu,\RR) \ni f \mapsto \int_{X}g\cdot f\,d\mu \in \RR$$
is continuous $\RR$-linear map, which coincides with $\Lambda$ on the space of $\mu$-simple functions. Since $\mu$-simple functions are dense in $L^1(\mu,\RR)$, we derive that
$$\Lambda(f) = \int_{X}g\cdot f\,d\mu$$
for every $f \in L^1(\mu,\RR)$. It remains to prove that $\lVert \Lambda \rVert \leq \lVert g\rVert_{\infty}$. For this pick $f \in L^1(\mu,\RR)$ such that $\lVert f \rVert_1 = 1$. Then
$$\left|\Lambda(f)\right| = \left|\int_X g\cdot f \,d\mu \right| \leq \int_X|g|\cdot |f|\,d\mu \leq \lVert g \rVert_{\infty}\cdot \int_X|f|\,d\mu = \lVert g \rVert_{\infty}$$
and hence $\lVert \Lambda \rVert \leq \lVert g\rVert_{\infty}$. This completes the proof.
\end{proof}

\begin{proof}[Proof for $\mathbb{K}=\RR$]
For each set $S$ in $\Sigma$ such that $\mu(S) \in \RR_+\cup \{0\}$ we define
$$\Lambda_S(f) = \Lambda\left(\mathbb{1}_S\cdot f\right)$$
for every $f \in L^1(\mu,\RR)$. Then $\Lambda_S:L^1(\mu,\RR)\ra \RR$ is a continuous $\RR$-linear map. Now if $S\subseteq T$ are two sets in $\Sigma$ such that $\mu(S),\mu(T) \in \RR_+\cup \{0\}$, then
$$\lVert \Lambda_S\rVert \leq \lVert \Lambda_T\rVert \leq \lVert \Lambda \rVert$$
Suppose now that $f$ is a function in $L^1(\mu, \RR)$ such that $\lVert f \rVert_1 \leq 1$. Then there exists a nondecreasing sequence $\{S_n\}_{n\in \NN}$ of sets in $\Sigma$ such that $\mu(S_n)$ is finite for every $n\in \NN$ and $\{\mathbb{1}_{S_n}\cdot f\}_{n\in \NN}$ converges to $f$ in $L^1(\mu, \RR)$. Hence 
$$\Lambda(f) = \lim_{n\ra +\infty}\Lambda_{S_n}\left(f\right)$$
It follows that
$$\lVert \Lambda \rVert = \sup \big\{\lVert \Lambda_S \rVert\,\big|\,S\in \Sigma\mbox{ such that }\mu(S)\mbox{ is finite }\}$$
It follows that there exists a nondecreasing sequence $\{S_n\}_{n\in \NN}$ of sets in $\Sigma$ such that $\mu(S_n)$ is finite for every $n\in \NN$ and 
$$\lVert \Lambda \rVert = \lim_{n\ra +\infty}\lVert \Lambda_{S_n} \rVert$$
According to already proved case there exists $g_n \in L^{\infty}(\mu,\RR)$ such that $g_n = 0$ outside $S_n$ and
$$\Lambda_{S_n}(f) = \int_X g_n\cdot f\,d\mu$$
for every $f\in L^1(\mu,\RR)$. Moreover, $\lVert g_n \rVert_{\infty} = \lVert \Lambda_{S_n}\rVert$ for each $n \in \NN$. We may assume that $g_{n+1}(x) = g_n(x)$ for $x\in S_n$. Define $g:X\ra \RR$ as a pointwise limit of $\{g_n\}_{n\in \NN}$. According to Proposition \ref{proposition:norm_of_standard_functional_for_positive_p} we have
$$\lVert g_n \rVert_{q} = \lVert \Lambda_{S_n} \rVert$$
for every $n \in \NN$. Taking limits of both sides and using monotone convergence we obtain $\lVert g \rVert_{q} = \lVert \Lambda \rVert$. In particular, we have $g\in L^q(\mu, \RR)$. Fix $f \in L^p(\mu,\RR)$. Then $g\cdot f \in L^1(\mu,\RR)$ which follows from H{\"o}lder inequality. By dominated convergence theorem
$$\Lambda(f) = \lim_{n\ra +\infty}\Lambda_{S_n}\left(f\right) = \lim_{n\ra +\infty}\int_X g_n\cdot f\,d\mu = \lim_{n\ra +\infty} \int_X g\cdot \mathbb{1}_{S_n}\cdot f\,d\mu = \int_X g\cdot f\,d\mu$$
Thus
$$\Lambda(f) = \int_X g\cdot f\,d\mu$$
for every $f \in L^p(\mu, \RR)$.
\end{proof}
    

















\end{document}