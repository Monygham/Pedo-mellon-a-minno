\input ../pree.tex

\begin{document}

\title{Hahn-Banach theorem}
\date{}
\maketitle

\section{Introduction}
\noindent
In these notes we study geometric and analytic versions of Hahn-Banach theorem. For this we introduce topological vector spaces and study their properties over arbitrary fields with absolute value. Next we prove that all one-dimensional Hausdorff topological spaces are isomorphic. This result is used in the characterization of finite dimensional Hausdorff topological vector spaces over a complete field and it is one of the crucial ingredients of Mazur's separation theorem (also called geometric version of Hahn-Banach). Next we intoduce locally convex topological vector spaces and prove separation of convex sets for this spaces. Finally we use Mazur's theorem to deduce analytic version of Hahn-Banach theorem.\\
Throughout the notes $\mathbb{K}$ is a field with absolute value $|-|$. The closed disc in $\mathbb{K}$ centered in the origin and with unit radius is denoted by $\mathbb{D}$.

\begin{definition}
Suppose that every Cauchy sequence in $\mathbb{K}$ with respect to $|-|$ is convergent, then $\mathbb{K}$ is \textit{a complete field}.   
\end{definition}

\section{Preliminaries on topological vector spaces}
\noindent
In this section we introduce topological vector spaces and study their basic properties.

\begin{definition}
Let $\fX$ be a vector space over $\mathbb{K}$ together with a topology such that the multiplication by scalars $\cdot_{\fX}:\mathbb{K}\times \fX \ra \fX$ and the addition $+_{\fX}:\fX\times \fX\ra \fX$ are continuous. Then $\fX$ is \textit{a topological vector space over $\mathbb{K}$}.
\end{definition}

\begin{fact}\label{fact:topological_vector_subspaces}
Let $\fX$ be a topological vector space over $\mathbb{K}$ and let $\fZ$ be its $\mathbb{K}$-subspace. Then $\fZ$ with subspace topology is a topological vector space over $\mathbb{K}$.
\end{fact}
\begin{proof}
Left for the reader as an exercise.
\end{proof}
\noindent
Recall that $\mathbb{D}$ is the unit disc in $\mathbb{K}$ centered in the origin.

\begin{fact}\label{fact:supercircled_open_basis_at_zero}
Let $\fX$ be a topological vector space over $\mathbb{K}$ and let $U$ be an open neighborhood of zero in $\fX$. Then there exists an open neighborhood $W$ of zero in $\fX$ such that $W \subseteq U$ and $W = \mathbb{D}\cdot W$.
\end{fact}
\begin{proof}
Since the multiplication by scalars $\mathbb{K}\times \fX \ra \fX$ is continuous, there exists an open neighborhood $V$ of zero in $\fX$ and a positive real number $r$ such that
$$W = \bigcup_{\alpha\in \mathbb{K},\,|\alpha| \leq r}\alpha \cdot V \subseteq U$$
Then $W$ is an open neighborhood of zero in $\fX$, $W\subseteq U$ and $W = \mathbb{D}\cdot W$.
\end{proof}

\begin{definition}
Let $\fX,\fY$ are topological vector spaces over $\mathbb{K}$. A map $f:\fX\ra \fY$ which is both continuous and $\mathbb{K}$-linear is \textit{a morphism of topological vector spaces over $\mathbb{K}$}.
\end{definition}

\begin{theorem}\label{theorem:quotients_of_topological_vector_spaces}
Let $\fX$ be a topological vector space over $\mathbb{K}$ and let $\fU$ be its $\mathbb{K}$-subspace. Consider the quotient map $q:\fX\twoheadrightarrow \fX/\fU$ in the category of vector spaces over $\mathbb{K}$ and equip $\fX/\fU$ with the quotient topology of $\fX$. Then the following assertions holds.
\begin{enumerate}[label=\emph{\textbf{(\arabic*)}}, leftmargin=*]
\item $q$ is an open map.
\item $\fX/\fU$ is a topological vector space over $\mathbb{K}$ and $q$ is a morphism of topological vector spaces.
\item For every morphism $f:\fX\ra \fY$ of topological vector spaces over $\mathbb{K}$ such that $f\left(\fU\right) = 0$ there exists a unique morphism $p:\fX/\fU\ra \fY$ of topological vector spaces over $\mathbb{K}$ which makes the triangle
\begin{center}
\begin{tikzpicture}
[description/.style={fill=white,inner sep=2pt}]
\matrix (m) [matrix of math nodes, row sep=4em, column sep=5em,text height=1.5ex, text depth=0.25ex] 
{ \fX &  \fY  \\
   \fX/\fU & \\ } ;
\path[->,line width=0.8pt,font=\scriptsize]
(m-1-1) edge node[above] {$ f $} (m-1-2)
(m-1-1) edge node[left] {$ q $} (m-2-1);
\path[densely dotted,->,line width=0.8pt,font=\scriptsize]
(m-2-1) edge node[right = 2pt, below = 2pt] {$ p $} (m-1-2);
\end{tikzpicture}
\end{center}
commutative.
\item $\fU$ is a closed in $\fX$ if and ony if $\fX/\fU$ is a Hausdorff topological space.
\end{enumerate}
\end{theorem}
\noindent
For the proof we need the following result.

\begin{lemma}\label{lemma:Hausdorff_topological_vector_spaces}
Let $\fX$ be a topological vector space over $\mathbb{K}$. Then $\fX$ is Hausdorff if and only if zero subspace of $\fX$ is closed.
\end{lemma}
\begin{proof}[Proof of the lemma]
If $\fX$ is Hausdorff, then each singleton subset of $\fX$ is closed. Hence zero subspace of $\fX$ is closed.\\
Conversely, assume that the singleton of zero in $\fX$ is closed. Pick two distinct points $x_1,x_2 \in \fX$. There exists an open neighborhood $U$ of zero in $\fX$ such that $x_1 - x_2\not \in U$. Since the addition $\fX\times \fX\ra \fX$ is continuous, there exists an open neighborhood $W$ of zero in $\fX$ such that $W + W\subseteq U$. Define $V$ to be $W\cap (-W)$. Then $V$ is an open neighborhood of zero such that $V + V\subseteq U$ and $V = - V$. If
$$z \in (x_1 + V)\cap (x_2 + V)$$
then $z = x_1 + z_1$ and $z = x_2 + z_2$ for some $z_1,z_2 \in V$. Hence
$$x_1 - x_2 = (z_2 - z_1) \in V + (-V) = V + V\subseteq U$$
This is a contradiction with $x_1 - x_2 \not \in U$. Thus
$$\emptyset = (x_1 + V)\cap (x_2 + V)$$
and $\fX$ is Hausdorff.  
\end{proof}

\begin{proof}[Proof of the theorem]
Fix an open subset $U$ of $\fX$, then the set 
$$q^{-1}\left(q\left(U\right)\right) = \bigcup_{u \in \fU}\left(u + U\right)$$
is open. According to the fact that $q:\fX\twoheadrightarrow \fX/\fU$ is a quotient topological map, we infer that $q(U)$ is open in $\fX/\fU$. Hence $q$ is an open map and the proof of \textbf{(1)} is completed.\\
Since $q$ is open, we derive that $1_{\mathbb{K}}\times q$ and $q\times q$ are open. Since squares
\begin{center}
\begin{tikzpicture}
[description/.style={fill=white,inner sep=2pt}]
\matrix (m) [matrix of math nodes, row sep=4em, column sep=5em,text height=1.5ex, text depth=0.25ex] 
{ \fX \times \fX         &  \fX     &  \mathbb{K} \times \fX     &  \fX    \\
\fX/\fU \times \fX/\fU   &  \fX/\fU &  \mathbb{K} \times \fX/\fU &  \fX/\fU  \\ } ;
\path[->,line width=0.8pt,font=\scriptsize]
(m-1-1) edge node[above] {$ +_{\fX} $} (m-1-2)
(m-1-1) edge node[left] {$ q\times q $} (m-2-1)
(m-1-2) edge node[right] {$ q $} (m-2-2)
(m-2-1) edge node[below] {$ +_{\fX/\fU} $} (m-2-2)
(m-1-3) edge node[above] {$ \cdot_{\fX} $} (m-1-4)
(m-1-3) edge node[left] {$ 1_{\mathbb{K}}\times q $} (m-2-3)
(m-1-4) edge node[right] {$ q $} (m-2-4)
(m-2-3) edge node[below] {$ \cdot_{\fX/\fU} $} (m-2-4);
\end{tikzpicture}
\end{center}
are commutative, we deduce that the addition $+_{\fX/\fU}:\fX/\fU \times \fX/\fU \ra \fX/\fU$ and the multiplication of scalars $\cdot_{\fX/\fU}:\mathbb{K}\times \fX/\fU\ra \fX/\fU$ are continuous. Therefore, $\fX/\fU$ is a topological vector space over $\mathbb{K}$. It follows that $q$ is a morphism of topological vector spaces over $\mathbb{K}$ and hence \textbf{(2)} holds.\\
The assertion \textbf{(3)} describes the universal property which follows easily from definition and \textbf{(2)}.\\
For \textbf{(4)} observe that
$$\fU\mbox{ is closed subset of }\fX\,\Leftrightarrow\,\mbox{zero subspace of }\fX/\fU\mbox{ is closed }$$
Thus it suffices to prove that
$$\mbox{ zero subspace of }\fX/\fU\mbox{ is closed }\,\Leftrightarrow\,\fX/\fU\mbox{ is a Hausdorff topological space}$$
but this is a consequence of Lemma \ref{lemma:Hausdorff_topological_vector_spaces}.    
\end{proof}

\section{Complete topological vector spaces}
\noindent
We need some basic results on complete topological vector spaces. We start by defining this important notion.

\begin{definition}
Let $\fX$ be a topological vector space over $\mathbb{K}$. Suppose that $\cF$ is a proper filter of subsets of $\fX$ such that for every open neighborhood $U$ of zero in $\fX$ there exists $F \in \cF$ such that
$$F - F \subseteq U$$
Then $\cF$ is \textit{a Cauchy filter in $\fX$}.
\end{definition}

\begin{definition}
A topological vector space $\fX$ over $\mathbb{K}$ is \textit{complete} if every Cauchy filter in $\fX$ is convergent.
\end{definition}

\begin{theorem}\label{theorem:complete_subspaces_of_topological_vector_spaces}
Let $\fX$ be a topological vector space over $\mathbb{K}$ and let $\fZ$ be its $\mathbb{K}$-subspace. Consider $\fZ$ as a topological vector space over $\mathbb{K}$ with subspace topology. Then the following assertions hold.
\begin{enumerate}[label=\emph{\textbf{(\arabic*)}}, leftmargin=*]
\item If $\fX$ is complete and $\fZ$ is a closed in $\fX$, then $\fZ$ is complete.
\item If $\fZ$ is complete and $\fX$ is Hausdorff, then $\fZ$ is closed in $\fX$.
\end{enumerate}
\end{theorem}
\begin{proof}
Consider a Cauchy filter $\cF$ in $\fZ$. We define
$$\tilde{\cF} = \big\{\tilde{F}\subseteq \fX\,\big|\,\mbox{ there exists }F\in \cF\mbox{ such that }F\subseteq \tilde{F}\big\}$$
Clearly $\tilde{\cF}$ is a Cauchy filter in $\fX$. Since $\fX$ is complete, we derive that $\tilde{\cF}$ is convergent to some $x$ in $\fX$. This together with definition of $\tilde{\cF}$ show that for every open neighborhood $U$ of zero in $\fX$ there exists $F \in \cF$ such that $F \subseteq x + U$. In particular, for every open neighborhood $U$ of zero in $\fX$ intersection $\left(x + U\right)\cap \fZ$ is nonempty. Since $\fZ$ is closed in $\fX$, it follows that $x \in \fZ$ and $\cF$ is convergent to $x$. Thus $\fZ$ is complete.\\
Suppose now that $\fZ$ is complete. Assume that for some point $x$ in $\fX$ and for every open neighborhood of zero $U$ in $\fX$ intersection $\left(x + U\right) \cap \fZ$ is nonempty. Define
$$\cF = \big\{F \subseteq \fZ\,\big|\,\mbox{ there exists open neighborhood }U\mbox{ of zero in }\fX\mbox{ such that }\left(x + U\right) \cap \fZ \subseteq F\big\}$$
Then $\cF$ is a Cauchy filter in $\fZ$. Since $\fZ$ is complete, $\cF$ is convergent to some point $z$ in $\fZ$. By definition of $\cF$ we have $z \in x + U$ for every open neighborhood $U$ of zero $x$. Since $\fX$ is Hausdorff, it follows that $z$ is identical to $x$. This proves that $\fZ$ is closed in $\fX$.  
\end{proof}

\begin{theorem}\label{theorem:pseudometrizability_completeness_is_determined_by_Cauchy_sequences}
Let $\fX$ be a topological vector space over $\mathbb{K}$. Suppose that there exists a pseudometric $\rho:\fX\times \fX \ra \RR_+\cup \{0\}$ which induces topology on $\fX$. Then the following assertions hold.
\begin{enumerate}[label=\emph{\textbf{(\roman*)}}, leftmargin=*]
\item $\fX$ is complete.
\item Every Cauchy sequence with respect to $\rho$ is convergent.
\end{enumerate}
\end{theorem}
\begin{proof}
Assume that $\fX$ is complete and $\{x_n\}_{n\in \NN}$ is a Cauchy sequence with respect to $\rho$. Define
$$F_n = \big\{x_k\,\big|\,k\geq n\big\}$$
for every $n\in \NN$ and let 
$$\cF = \big\{F\subseteq \fX\,\big|\,F_n \subseteq F\mbox{ for some }n\in \NN\big\}$$
Since $\{x_n\}_{n\in \NN}$ is a Cauchy sequence with respect to $\rho$ and this pseudometric induces topology on $\fX$, we derive that $\cF$ is a Cauchy filter in $\fX$. Hence $\cF$ is convergent to some point of $\fX$. This proves that $\{x_n\}_{n\in \NN}$ is convergent to some point of $\fX$. Hence $\{x_n\}_{n\in \NN}$ is convergent with respect to $\rho$. This completes the proof of $\textbf{(i)}\Rightarrow \textbf{(ii)}$.\\
Suppose that every Cauchy sequence with respect to $\rho$ is convergent in $\fX$. Consider a Cauchy filter $\cF$ in $\fX$. Since topology of $\fX$ is pseudometrizable, we derive that there exists a countable basis $\{U_n\}_{n\in \NN}$ of open neighborhoods of zero in $\fX$. There exists a decreasing sequence $\{F_n\}$ of elements of $\cF$ such that
$$F_n - F_n\subseteq U_n$$
for each $n\in \NN$. For each $n\in \NN$ let $x_n \in F_n$. Then $\{x_n\}_{n\in \NN}$ is a Cauchy sequence with respect to $\rho$. Hence it is convergent to some point $x$ in $\fX$. Pick an open neighborhood $U$ of zero in $\fX$. Consider open neighborhood $W$ of zero in $\fX$ such that $W + W \subseteq U$. For sufficiently large $n\in \NN$ we have 
$$F_n - F_n \subseteq W,\,x_n - x \in W$$
If $z \in F_n$, then
$$x - z = (x - x_n) + (x_n - z) \in W + (F_n - F_n) \subseteq W + W \subseteq U$$
Hence $F_n \subseteq x + U$. This proves that $\cF$ is convergent to $x$. The implication $\textbf{(ii)}\Rightarrow \textbf{(i)}$ holds.  
\end{proof}

\begin{theorem}\label{theorem:completeness_of_product}
Let $\{\fX_i\}_{i \in I}$ be a family of topological vector space over $\mathbb{K}$. Then the following assertions are equivalent.
\begin{enumerate}[label=\emph{\textbf{(\roman*)}}, leftmargin=*]
\item $\fX_i$ is complete for every $i \in I$.
\item $\prod_{i \in I}\fX_i$ is complete topological vector space over $\mathbb{K}$.
\end{enumerate}
\end{theorem}
\begin{proof}
We denote $\prod_{i\in I}\fX_i$ by $\fX$ and let $pr_i:\fX \ra \fX_i$ be canonical projection on $i$-th axis.\\
Assume that $\fX_i$ is complete for every $i \in I$. Suppose that $\cF$ is a Cauchy filter in $\fX$. Then $pr_i(\cF)$ is a Cauchy filter in $\fX_i$ for each $i$. Since $\fX_i$ is complete, we derive that $pr_i(\cF)$ is convergent to some point $x_i$ in $\fX_i$. Define $x \in \fX$ by condition $pr_i(x) = x_i$ for each $i \in I$. Then $\cF$ is convergent to $x$. Thus $\fX$ is a complete topological vector space over $\mathbb{K}$.\\
Suppose now that $\fX$ is complete. Fix $i_0$ in $I$ and consider a Cauchy filter $\cF$ in $\fX_{i_0}$. Define
$$\tilde{F} = \big\{\underbrace{F}_{i_0} \times \underbrace{\{0\}}_{i\neq i_0} \subseteq \fX\,\big|\,F\in \cF\big\}$$
Then $\tilde{\cF}$ is a Cauchy filter in $\fX$. Hence $\tilde{\cF}$ is convergent to some point $x$ in $\fX$. Then $\cF = pr_{i_0}(\tilde{\cF})$ is convergent to $pr_{i_0}(x)$. Thus $\fX_{i_0}$ is complete. Since $i_0$ is arbitrary, we derive that $\fX_i$ is complete for every $i\in I$.
\end{proof}

\begin{corollary}\label{corollary:finite_products_of_copies_of_field_are_complete}
Let $\mathbb{K}$ be a complete field. Topological vector spaces $\mathbb{K}^n$ over $\mathbb{K}$ are complete for each $n\in \NN$.
\end{corollary}
\begin{proof}
This is a direct consequence of Theorems \ref{theorem:pseudometrizability_completeness_is_determined_by_Cauchy_sequences} and \ref{theorem:completeness_of_product}.
\end{proof}

\section{Finite dimensional topological vector spaces}

\begin{fact}\label{fact:linear_morphisms_from_standard_finite_spaces_are_always_continuous}
Let $\fX$ be a topological vector space over $\mathbb{K}$. Suppose that $f:\mathbb{K}^n \ra \fX$ is a $\mathbb{K}$-linear map for some $n\in \NN$. Then $f$ is continuous.
\end{fact}
\begin{proof}
Let $\{e_1,...,e_n\}$ be the canonical basis of $\mathbb{K}^n$. For every $i$ let $pr_i:\mathbb{K}^n\ra \mathbb{K}$ be the projection onto $i$-th axis and let $m_{i}:\mathbb{K}\ra \fX$ be the composition of the multiplication of scalars $\mathbb{K}\times \fX\ra \fX$ with the continuous embedding $\mathbb{K} \ni \alpha \mapsto \left(\alpha, f(e_i)\right) \in \mathbb{K}\times \fX$. Since $\mathrm{pr}_i$ and $m_{i}$ are continuous for each $i$, we derive that their compositions $m_{i}\cdot pr_i$ are also continuous. According to the fact that the addition $\fX\times \fX\ra \fX$ is continuous, we infer that the sum
$$\sum_{i=1}^n m_{i}\cdot pr_{i}$$
is continuous. This sum is equal to $f$. Thus $f$ is continuous. 
\end{proof}
\begin{theorem}\label{theorem:line_topological_spaces}
Let $\fX$ be a one-dimensional topological vector space over $\mathbb{K}$. Then the following assertions hold.
\begin{enumerate}[label=\emph{\textbf{(\arabic*)}}, leftmargin=*]
\item If $\fX$ is Hausdorff, then every $\mathbb{K}$-linear isomorphisn $\fX\ra \mathbb{K}$ is a homeomorphism.
\item If $\fX$ is not Hausdorff, then the topology on $\fX$ is indiscrete.
\end{enumerate}
\end{theorem}
\begin{proof}
Assume that $\fX$ is Hausdorff. Let $f:\fX\ra \mathbb{K}$ be $\mathbb{K}$-linear isomorphism. If the topology of $\mathbb{K}$ is discrete, then $f$ is a homeomorphism. Hence without loss of generality we may assume that the topology on $\mathbb{K}$ is not discrete. In particular, for each positive real number $r$ there exists nonzero $\gamma \in \mathbb{K}$ such that $|\gamma| < r$. Consider $x_{\gamma}$ in $\fX$ such that $f(x_{\gamma}) = \gamma$. It is unique element of $\fX$. Since $\fX$ is Hausdorff, by Fact \ref{fact:supercircled_open_basis_at_zero} there exists open neighborhood $W$ of zero in $\fX$ such that $\mathbb{D}\cdot W = W$ and $x_{\gamma} \not \in W$. Then $\mathbb{D}\cdot f(W) = f(W)$ and $\gamma \not \in f(W)$.
This proves that $f(W)$ is a subset of
$$\big\{\alpha \in \mathbb{K}\,\big|\,|\alpha| < r\big\}$$
Therefore, $f$ is continuous at zero and hence $f$ is continuous. On the other hand map $f^{-1}:\mathbb{K}\ra \fX$ is continuous by Fact \ref{fact:linear_morphisms_from_standard_finite_spaces_are_always_continuous}. This means that $f$ is a homeomorphism.\\
Suppose now that $\fX$ is not Hausdorff. Theorem \ref{theorem:quotients_of_topological_vector_spaces} implies that zero subspace is not closed in $\fX$. Since in every topological vector space closure of a subspace is a subspace, we derive that $\fX$ is the closure of its zero subspace. This shows that $\fX$ is indiscrete. 
\end{proof}

\begin{corollary}\label{corollary:criteria_for_continuity_of_linear_functionals}
Let $f:\fX \ra \mathbb{K}$ be a $\mathbb{K}$-linear map between topological vector spaces over $\mathbb{K}$. Then the following are equivalent.
\begin{enumerate}[label=\emph{\textbf{(\roman*)}}, leftmargin=*]
\item $f$ is continuous.
\item $\Ker(f)$ is a closed subspace of $\fX$.
\end{enumerate}
\end{corollary}
\begin{proof}
Follows immediately from Theorems \ref{theorem:quotients_of_topological_vector_spaces} and \ref{theorem:line_topological_spaces}.
\end{proof}

\begin{theorem}\label{theorem:uniqueness_of_finite_dimensional_Hausdorff_top_vec_spaces}
Let $\mathbb{K}$ be a complete field and let $\fX$ be a topological vector space over $\mathbb{K}$. If $\fX$ is Hausdorff and of dimension $n$ over $\mathbb{K}$ for some $n\in \NN$, then $\fX$ is isomorphic with $\mathbb{K}^n$.
\end{theorem}
\begin{proof}
The proof goes on induction by $n\in \NN$. For $n = 0$ it is clear. Suppose that the result holds for $n \in \NN$. Assume that $\fX$ is a Hausdorff topological vector space over $\mathbb{K}$ of dimension $n + 1$. By induction each $n$-dimensional subspace of $\fX$ is isomorphic to $\mathbb{K}^n$ and hence by Corollary \ref{corollary:finite_products_of_copies_of_field_are_complete} it is complete. Thus Theorem \ref{theorem:complete_subspaces_of_topological_vector_spaces} asserts that all $n$-dimensional subspaces are closed in $\fX$. Corollary \ref{corollary:criteria_for_continuity_of_linear_functionals} implies that each $\mathbb{K}$-linear map $f:\fX\ra \mathbb{K}$ is continuous. Therefore, every $\mathbb{K}$-linear map $\Phi:\fX \ra \mathbb{K}^{n+1}$ is continuous. Next $\Phi^{-1}$ is continuous according to Fact \ref{fact:linear_morphisms_from_standard_finite_spaces_are_always_continuous}. Therefore, $\fX$ is isomorphic to $\mathbb{K}^{n+1}$ as a topological vector space over $\mathbb{K}$. The proof is completed.
\end{proof}

\section{Mazur's theorem}
\noindent
In this section assume that $\mathbb{K}$ is either real numbers field $\RR$ of complex numbers field $\CC$.

\begin{theorem}[Mazur]\label{theorem:Mazurs_hyperplane_separation}
Let $\fX$ be a topological vector space over $\mathbb{K}$ and let $U$ be an open and convex subset of $\fX$. Suppose that $\fU$ is a $\mathbb{K}$-subspace of $\fX$ such that $\fU$ does not intersect with $U$. Then there exists a $\mathbb{K}$-linear continuous map $f:\fX\ra \mathbb{K}$ such that $\fU \subseteq \Ker(f)$ and $0 \not \in f(U)$.
\end{theorem}
\noindent
For the proof we need the following result.

\begin{lemma}\label{lemma:two_dimensional_hyperplane_separation}
Let $\fX$ be a two-dimensional Hausdorff topological vector space over $\RR$ and let $U$ be an open and convex subset which does not contain zero of $\fX$. Then there exists one-dimensional subspace $L$ of $\fX$ which does not intersect $U$. 
\end{lemma}
\begin{proof}[Proof of the lemma]
Theorem \ref{theorem:uniqueness_of_finite_dimensional_Hausdorff_top_vec_spaces} implies that we may assume that $\fX$ is $\RR^2$. Consider
$$S^1 = \big\{(x, y)\in \RR^2\,\big|\,x^2 + y^2 = 1\big\}$$
and a retraction $r:\RR^2\setminus \{0\} \ra S^1$ given by formula 
$$r(x, y) = \bigg(\frac{x}{\sqrt{x^2 + y^2}},\frac{y}{\sqrt{x^2 + y^2}}\bigg)$$
Note that $r$ is a continuous open map. Thus $\tilde{U} = r(U)$ is an open subset of $S^1$. Let $i:S^1\ra S^1$ be a homeomorphism given by formula $i(x, y) = (-x, -y)$. Since $U$ is convex and does not contain zero, sets $i(\tilde{U})$ and $\tilde{U}$ have empty intersection. According to the fact that $S^1$ is connected, we deduce that $i(\tilde{U}) \cup \tilde{U}$ is a proper subset of $S^1$. This is the case if and only if there exists $(x, y) \in S^1$ such that $(x, y) \not \in \tilde{U}$ and $(-x, -y) \not \in \tilde{U}$. Then one-dimensional subspace $\RR \cdot (x, y)$ of $\fX$ does not intersect $U$.
\end{proof}

\begin{proof}[Proof of the theorem]
Assume first that $\mathbb{K}$ is $\RR$. By Zorn's lemma there exists maximal $\RR$-subspace $\fZ$ such that $\fU\subseteq \fZ$ and $\fZ$ does not intersect $U$. Since $U$ is open, we derive that $\bd{cl}(\fZ)$ does not intersect $U$. This shows that $\fZ$ is a closed subspace of $\fX$. Now consider the quotient map $q:\fX\twoheadrightarrow \fX/\fZ$. By Theorem \ref{theorem:quotients_of_topological_vector_spaces} space $\fX/\fZ$ is Hausdorff and $q(U)$ is an open set. Moreover, $q(U)$ does not intersect zero and is convex. Suppose that there exists two-dimensional $\RR$-subspace $\fY$ of $\fX/\fZ$. Applying Lemma \ref{lemma:two_dimensional_hyperplane_separation} to $\fY$ and $\fY\cap q(U)$ we deduce that there exists one-dimensional $\RR$-subspace $L$ of $\fX/\fZ$ such that $L$ does not intersect $q(U)$. Then $q^{-1}(L)$ is $\RR$-subspace of $\fX$ strictly containing $\fZ$ which does not intersect $U$. This is contradiction with maximality of $\fZ$. Thus $\fX/\fZ$ contains no two-dimensional subspaces and hence it is one-dimensional. According to Theorem \ref{theorem:uniqueness_of_finite_dimensional_Hausdorff_top_vec_spaces} we have isomorphism $\phi:\fX/\fZ \ra \RR$ of topological vector spaces over $\RR$. The composition $f = \phi \cdot q$ satisfies the assertion of the theorem and this completes the proof for $\RR$.\\
Next assume that $\mathbb{K}$ is $\CC$. Since $\fX$ is a topological vector space over $\CC$, it is also topological vector space over $\RR$. Hence there exists an $\RR$-linear continuous map $\tilde{f}:\fX\ra \RR$ such that $\fU \subseteq \Ker(\tilde{f})$ and $0 \not \in \tilde{f}(U)$. Consider $f:\fX\ra \CC$ given by formula
$$f(x) = \tilde{f}(x) - \sqrt{-1}\cdot \tilde{f}\left(\sqrt{-1}\cdot x\right)$$
for $x$ in $\fX$. Then $f$ is a $\CC$-linear continuous map such that $\fU\subseteq \Ker(f)$ and $0\not \in f(U)$.
\end{proof}
\noindent
The result above is often called geometric Hahn-Banach theorem.

\section{Analytic Hahn-Banach theorem}

\begin{definition}
Let $\fX$ be a vector space over $\RR$ and let $p:\fX\ra \RR$ be a map. Suppose that
$$p(x_1 + x_2)\leq p(x_1) + p(x_2)$$
for all $x_1,x_2\in \fX$ and 
$$p(r\cdot x) = r\cdot p(x)$$
for each $x\in \fX$ and each $r\in \RR_+$. Then $p$ is \textit{a sublinear map}.
\end{definition}

\begin{theorem}[Hahn-Banach]\label{theorem:Hahn_Banach_real_case}
Let $\fX$ be a vector space over $\RR$ and let $p:\fX\ra \RR$ be a sublinear map. Suppose that $\fU$ is an $\RR$-subspace of $\fX$ and $g:\fU\ra \RR$ is an $\RR$-linear map such that $f(x) \leq p(x)$ for every $x$ in $\fU$. Then there exists an $\RR$-linear map $\tilde{f}:\fX \ra \RR$ such that $\tilde{f}(x) \leq p(x)$ and $\tilde{f}_{\mid \fU} = f$. 
\end{theorem}
\noindent
We need the following result.

\begin{lemma}\label{lemma:sublinear_induces_seminorm_and_is_continuous_with_respect_to_it}
Let $\fX$ be a vector space over $\RR$ and let $p:\fX \ra \RR$ be a sublinear map. Consider $q:\fX \ra \RR$ given by formula
$$q(x) = \max\{p(x),p(-x)\}$$
for $x \in \fX$. Then $q$ is a seminorm on $\fX$ and $p$ is continuous with respect to $q$. 
\end{lemma}
\begin{proof}[Proof of the lemma]
Note that $q$ is a sublinear map. Since 
$$0\leq p(x) + p(-x)$$
for $x \in \fX$, we derive that the image of $q$ is $\RR_+\cup \{0\}$. Moreover, $q(x) = q(-x)$ for each $x$ in $\fX$. Therefore, $q$ is a seminorm on $\fX$. Observe that
$$|p(x_1) - p(x_2)|\leq q(x_1 - x_2)$$
and hence $p$ is continuous with respect to topology induced by $q$ on $\fX$.
\end{proof}

\begin{proof}[Proof of the theorem]
By Lemma \ref{lemma:sublinear_induces_seminorm_and_is_continuous_with_respect_to_it} we may assume that $\fX$ is a topological vector space over $\RR$ and $p$ is continuous map. Define
$$U = \big\{(x,r)\in \fX\times \RR\,\big|\,p(x) < r\big\},\,\fZ = \big\{(x,f(x))\in \fX\times \RR\,\big|\,x\in \fU\big\}$$ 
It follows that $U$ is a convex open subset of $\fX\times \RR$ and $\fZ$ is an $\RR$-subspace of $\fX\times \RR$ such that $U \cap \fZ = \emptyset$. By Theorem \ref{theorem:Mazurs_hyperplane_separation} there exists an $\RR$-linear continuous map $\tilde{g}:\fX\times \RR\ra \RR$ such that $\fZ\subseteq \ker(\tilde{g})$ and $0 \not \in \tilde{f}(U)$. Since $U$ is convex, without loss of generality we may assume that $\tilde{g}(U)\subseteq \RR_+$. There exists $u \in \RR$ and $\RR$-linear map $g:\fX\ra \RR$ such that 
$$\tilde{g}\big(x, r\big) = g(x) + u\cdot r$$
for every $x \in \fX$ and $r \in \RR$. Suppose now that $u \leq 0$. We have
$$g(x) + u\cdot r = \tilde{g}(x,r) > 0$$
for each $(x,r) \in U$. Hence $g(x) > (- u) \cdot r$ for every $(x,r) \in U$. Fix now $x\in \fX$ and pick $r \in \RR_+$ such that $r > p(x)$. Then 
$$g(x) > (-u)\cdot r \geq 0$$ 
and this shows that $g(x) > 0$ for $x\in \fX$ and this contradicts the fact that $g$ is an $\RR$-linear map. Thus $u > 0$. We define $\tilde{f}:\fX\ra \RR$ by formula $\tilde{f}(x) = -\frac{1}{u}\cdot g(x)$. Then $f$ is an $\RR$-linear map and $$\tilde{g}(x, r) = u \cdot \left(r - \tilde{f}(x)\right)$$
for every $(x,r) \in \fX\times \RR$. For each $x \in \fU$ we have
$$0 = \tilde{g}\big(x,f(x)\big) = u \cdot \left(f(x) - \tilde{f}(x)\right)$$
Hence $\tilde{f}_{\mid \fU} = f$. Moreover, for $(x,r) \in U$ we have 
$$u\cdot \left(r - \tilde{f}(x)\right) = \tilde{g}(x, r) > 0$$
and hence
$$r > \tilde{f}(x)$$
for every $(x,r)\in U$. We deduce that $\tilde{f}(x) \leq p(x)$ for all $x\in \fX$. 

   
\end{proof}


















\end{document}