\input ../pree.tex

\begin{document}

\title{Hahn-Banach theorem}
\date{}
\maketitle

\section{Introduction}

\section{Hahn-Banach theorem}
\noindent
We start by introducing certain notions concerning real maps defined on $\RR$-vector spaces.

\begin{definition}
Let $V$ be an $\RR$-vector space. A map $p:V\ra \RR$ is \textit{subadditive} if 
$$p(v_1 + v_2)\leq p(v_1) + p(v_2)$$
for any vectors $v_1,v_2$ in $V$.
\end{definition}

\begin{definition}
Let $V$ be an $\RR$-vector space. A map $p:V\ra \RR$ is \textit{positive homogeneous} if 
$$p(\alpha \cdot v) = \alpha \cdot p(v)$$
for every $\alpha \in \RR_+$ and every $v$ in $V$.
\end{definition}
\noindent
The following is central result of these notes.

\begin{theorem}[Hahn-Banach]\label{theorem:hahn_banach_theorem}
Let $V$ be an $\RR$-vector space and let $p:V\ra \RR$ be a subadditive and positive homogeneous map. Suppose that $W$ is an $\RR$-subspace of $V$ and $f:W\ra \RR$ is an $\RR$-linear map such that
$$f(w) \leq p(w)$$
for every $w$ in $W$. Then there exists $\RR$-linear map $\tilde{f}:V\ra \RR$ such that $\tilde{f}_{\mid W} = f$ and $\tilde{f}(v) \leq p(v)$ for every $v$ in $V$.
\end{theorem}
\noindent
The heart of the proof is the following result.

\begin{lemma}\label{lemma:Hahn_Banach_extension_in_codimension_one}
Let $V$ be an $\RR$-vector space and let $p:V\ra \RR$ be a subadditive and positive homogeneous map. Suppose that $W$ is an $\RR$-subspace of $V$ and $f:W\ra \RR$ is an $\RR$-linear map such that
$$f(w) \leq p(w)$$
for every $w$ in $W$. Then for every vector $\tilde{v}\in V\setminus W$ there exists $\RR$-linear map $\tilde{f}:W + \RR\cdot \tilde{v} \ra \RR$ such that $\tilde{f}_{\mid W} = f$ and $\tilde{f}(v) \leq p(v)$ for every $v$ in $W + \RR\cdot \tilde{v}$.
\end{lemma}
\begin{proof}[Proof of the lemma]
We claim that the set of $\lambda \in \RR$ such that for every $\gamma \in \RR$ and every $w \in W$ the following condition is satisfied
$$f(w) + \gamma \cdot \lambda  \leq p\big(w + \gamma \cdot \tilde{v}\big)$$
is nonempty. In order to prove this we analyze this condition. Note that for $\gamma = 0$ the condition holds by assumption of the theorem. Thus we may assume that $\gamma \neq 0$. Let $\alpha = |\gamma|$. Now we consider two cases.
\begin{itemize}
\item For $\gamma > 0$ the condition is equivalent to
$$\lambda \leq p\left(\frac{w}{\alpha} + \tilde{v}\right) - f\left(\frac{w}{\alpha}\right)$$
Since $W$ is an $\RR$-vector space, it can be equivalently stated as
$$\lambda \leq p\left(w + \tilde{v}\right) - f\left(w\right)$$
for every $w \in W$.
\item For $\gamma < 0$ the condition is equivalent to
$$-p\left(\frac{w}{\alpha} - \tilde{v}\right) + f\left(\frac{w}{\alpha}\right) \leq  \lambda$$
We invoke the fact that $W$ is an $\RR$-vector space one again and obtain equivalent condition
$$-p\left(w - \tilde{v}\right) + f\left(w\right) \leq \lambda$$
for every $w \in W$.
\end{itemize}
Thus in order to prove our claim it suffices to prove that
$$\sup_{w\in W}-p\left(w - \tilde{v}\right) + f\left(w\right) \leq \inf_{w\in W}p\left(w + \tilde{v}\right) - f\left(w\right)$$
Therefore, it suffices to prove that
$$p\left(w_1 - \tilde{v}\right) + f\left(w_1\right) \leq p\left(w_2 + \tilde{v}\right) - f\left(w_2\right)$$
for any $w_1,w_2\in W$. Fix arbitrary $w_1,w_2\in W$. The inequality
$$p\left(w_1 - \tilde{v}\right) + f\left(w_1\right) \leq p\left(w_2 + \tilde{v}\right) - f\left(w_2\right)$$
is equivalent to
$$f(w_1 + w_2) \leq p\left(w_2 + \tilde{v}\right) + p\left(w_1 - \tilde{v}\right)$$
which holds according to
$$f(w_1 + w_2) \leq p(w_1 + w_2) = p\left(w_2 + \tilde{v} + w_1 - \tilde{v}\right) \leq p\left(w_2 + \tilde{v}\right) + p\left(w_1 - \tilde{v}\right)$$
Thus the claim is proved. We infer the statement from the claim as follows. Pick $\lambda \in \RR$ such that
$$f(w) + \gamma \cdot \lambda  \leq p\big(w + \gamma \cdot \tilde{v}\big)$$
for every $\gamma \in \RR$ and every $w \in W$. Then define $\tilde{f}:W + \RR\cdot \tilde{v} \ra \RR$ by $\tilde{f}(w + \gamma \cdot \tilde{v}) = f(w) + \gamma \cdot \lambda$ for every $w\in W$ and $\gamma \in \RR$. Then $\tilde{f}$ satisfies the assertion.
\end{proof}

\begin{proof}[Proof of the theorem]
Consider the family $\cG$ which consists of $\RR$-linear maps $g:U\ra \RR$ such that $U$ is a $\RR$-subspace of $V$ containing $W$, $g_{\mid W} = f$ and $g(u) \leq p(u)$ for every $u\in U$. For $g_1:U_1\ra \RR$ and $g_2:U_2\ra \RR$ in $\cG$ we define $g_1\preceq g_2$ if and only if $U_1\subseteq U_2$ and $\left(g_2\right)_{\mid U_1} = g_1$. Clearly $\preceq$ is a partial order on $\cG$. By Zorn's lemma there exists element $\tilde{f}:\tilde{V}\ra \RR$ in $\cG$ maximal with respect to $\preceq$. If $\tilde{V}\subsetneq V$, then by Lemma \ref{lemma:Hahn_Banach_extension_in_codimension_one} there exists element of $\cG$ greater than $\tilde{f}$ with respect to $\preceq$. This is a contradiction. Hence $\tilde{V} = V$ and $\tilde{f}$ satisfies the assertion of the theorem.
\end{proof}

\section{Normed version of Hahn-Banach theorem}










\end{document}