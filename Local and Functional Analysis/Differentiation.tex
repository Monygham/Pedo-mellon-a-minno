\input ../pree.tex

\begin{document}

\title{Differentiability}
\date{}
\maketitle

\section{Introduction}
\noindent
In these notes we collect basic results on derivatives of functions defined on open subsets of real or complex Banach spaces.\\
Symbol $\mathbb{K}$ denotes the base field which is either $\RR$ or $\CC$.

\section{Preliminaries on bounded multilinear forms on normed spaces over $\mathbb{K}$}
\noindent
In this section we fix a positive integer $n$ and consider normed spaces $\fD_1,...,\fD_n, \fX$ over $\mathbb{K}$. 

\begin{definition}
Let $L:\fD_1\times ... \times \fD_n\ra \fX$ be a $\mathbb{K}$-multilinear form. Suppose that there exists $C > 0$ such that
$$||L(x_1,...,x_n)|| \leq C\cdot ||x_1||\cdot ... \cdot ||x_n||$$
for every $x_i \in \fD_i$ for $i\in \{1,...,n\}$. Then $L$ is \textit{bounded}.
\end{definition}
\noindent
The following result characterizes bounded multilinear forms.

\begin{theorem}\label{theorem:characterization_of_bounded_multilinear_maps}
Let $L:\fD_1\times ... \times \fD_n\ra \fX$ be a $\mathbb{K}$-multilinear form. Then the following assertions are equivalent.
\begin{enumerate}[label=\emph{\textbf{(\roman*)}}, leftmargin=*]
\item $L$ is continuous.
\item $L$ is continuous at zero $n$-tuple in $\fD_1\times ... \times \fD_n$.
\item $L$ is bounded.
\end{enumerate}
\end{theorem}
\begin{proof}
The implication $\textbf{(i)}\Rightarrow \textbf{(ii)}$ is obvious.\\
Suppose that $L$ is continuous at zero $n$-tuple in $\fD_1\times ... \times \fD_n$. Assume that there exists a sequence $\big\{(x_{m,1},...,x_{m,n})\big\}_{m\in \NN_+}$ such that $x_{m,i} \in \fD_i$, $||x_{m,i}||=1$ for each $i$ and
$$||L(x_{m,1},...,x_{m,n})|| \geq m$$
for each $m\in \NN_+$. Define $y_{m,i} = \frac{1}{\sqrt[n]{m}}\cdot x_{m,i}$ for every $i\in \{1,...,n\}$ and every $m\in \NN_+$. Then $\{y_{m,i}\}_{m\in \NN_+}$ tends to zero for every $i\in \{1,...,n\}$ and
$$||L(y_{m,1},...,y_{m,n})|| \geq 1$$
for every $m\in \NN_+$. This is a contradiction with the assumption that $L$ is continuous at zero $n$-tuple in $\fD_1\times ... \times \fD_n$. Therefore, there exists $C > 0$ such that
$$||L(x_1,...,x_n)|| \leq C$$
for every $x_i \in \fD_i$ with $||x_i|| = 1$ for $i\in \{1,...,n\}$. Thus the implication $\textbf{(ii)}\Rightarrow \textbf{(iii)}$ holds.\\
Assume that $L$ is bounded. Pick $x_i \in \fD_i$ and $h_i\in \fD_i$ for $i\in \{1,...,n\}$. Define
$$z_0  = (x_1,...,x_n),\,z_i = (x_1+h_1,...,x_i+h_i,x_{i+1},...,x_n)$$
for $i\in \{1,...,n\}$. Then
$$\big|\big|L\left(x_1+h_1,...,x_n+h_n\right) - L\left(x_1,...,x_n\right)\big|\big| = \bigg|\bigg|\sum_{i=1}^n\left(L(z_{i}) - L(z_{i-1})\right)\bigg|\bigg| \leq $$
$$\leq\sum_{i=1}^n\big|\big|L(z_{i}) - L(z_{i-1})\big|\big| \leq \sum_{i=1}^nC\cdot ||x_1||\cdot ...\cdot ||x_{i-1}||\cdot ||h_i||\cdot ||x_{i+1}||\cdot ... \cdot ||x_n||$$
Thus if $(h_1,...,h_n)\ra 0$, then $L\left(x_1+h_1,...,x_n+h_n\right) - L\left(x_1,...,x_n\right) \ra 0$. This shows that $L$ is continuous. Hence the proof of the implication $\textbf{(iii)}\Rightarrow \textbf{(i)}$ is completed.
\end{proof}

\begin{definition}
Let $L:\fD_1\times ... \times \fD_n\ra \fX$ be a bounded $\mathbb{K}$-multilinear form. Then we define
$$||L|| = \sup \big\{||L\left(x_1,...x_n\right)||\,\big|\,\forall_{i\in \{1,...,n\}}\,x_i\in \fD_i\mbox{ and }||x_i|| = 1 \big\}$$
and call it \textit{the operator norm of $L$}.
\end{definition}

\begin{fact}\label{fact:operator_norm_is_lipschitz_constant}
Let $L:\fD_1\times ... \times \fD_n\ra \fX$ be a bounded $\mathbb{K}$-multilinear form. Then
$$||L(x_1,...,x_n)|| \leq ||L||\cdot ||x_1||\cdot ...\cdot ||x_n||$$
for every $(x_1,...,x_n) \in \fD_1\times ... \times \fD_n$.
\end{fact}
\begin{proof}
Left for the reader as an exercise.
\end{proof}

\begin{theorem}\label{theorem:bounded_multilinear_maps_form_Banach_space_if_target_is_Banach}
Let $L\left(\fD_1,...,\fD_n;\fX\right)$ be a $\mathbb{K}$-vector space of bounded multilinear forms $\fD_1\times ... \times \fD_n\ra \fX$ with respect to operations defined pointwise. Suppose that $\fX$ is a Banach space over $\mathbb{K}$. Then 
$$L\left(\fD_1,...,\fD_n;\fX\right) \ni L\mapsto ||L||\in [0,+\infty)$$
is a norm which makes $L\left(\fD_1,...,\fD_n;\fX\right)$ into a Banach space over $\mathbb{K}$.
\end{theorem}
\begin{proof}
We left as an exercise the proof that operator norm is well defined vector space norm on $L\left(\fD_1,...,\fD_n;\fX\right)$. Consider a Cauchy's sequence $\{L_m\}_{m\in \NN}$ with respect to operator norm. Then $\{||L_m||\}_{m\in \NN}$ is Cauchy's sequence and hence is convergent in $\mathbb{R}$. Fix $(x_1,...x_n)\in \fD_1\times ... \times \fD_n$. Then by Fact \ref{fact:operator_norm_is_lipschitz_constant}
$$||\left(L_m - L_k\right)\left(x_1,...,x_n\right)|| \leq ||L_m-L_k||\cdot ||x_1||\cdot ...\cdot ||x_n||$$
for every $m,k\in \NN$. This implies that $\big\{L_m\left(x_1,...,x_n\right)\big\}_{m\in \NN}$ is a Cauchy's sequence in $\fX$. Since $\fX$ is a Banach space over $\mathbb{K}$, we derive that this sequence is convergent. We define
$$L\left(x_1,...,x_n\right) = \lim_{m\ra +\infty}L_m\left(x_1,...,x_n\right)$$
Note that we have
$$||L(x_1,...,x_n)|| = \lim_{m\ra +\infty}||L_m\left(x_1,...,x_n\right)|| \leq \left(\lim_{m\ra +\infty}||L_m||\right) \cdot ||x_1||\cdot ...\cdot ||x_n||$$
Therefore, $L:\fD_1\times ... \times \fD_n\ra \fX$ is a bounded $\mathbb{K}$-multilinear form. We claim that $L$ is the limit of $\{L_m\}_{m\in \NN}$ with respect to operator norm. For the proof fix $(x_1,...x_n)\in \fD_1\times ... \times \fD_n$ such that $||x_1||=...=||x_n|| = 1$. Then
$$||\left(L - L_m\right)(x_1,...x_n)|| \leq ||L(x_1,...,x_n) - L_k(x_1,...,x_m)|| + ||L_k - L_m||$$
Thus we have
$$||\left(L - L_m\right)(x_1,...x_n)|| \leq \limsup_{k\ra +\infty}||L_k - L_m||$$
The left hand side does not depend on $x_1,...,x_n$ and we deduce that
$$||L - L_m|| \leq \limsup_{k\ra +\infty}||L_k - L_m||$$
Invoking once again the assumption that $\{||L_m||\}_{m\in \NN}$ is Cauchy's sequence we infer
$$\lim_{m\ra +\infty}||L - L_m|| \leq \lim_{m\ra +\infty}\limsup_{k\ra +\infty}||L_k - L_m|| = 0$$
This completes the proof.
\end{proof}

\begin{proposition}\label{proposition:multilinear_maps_canonical_isometry}
The canonical map $L\left(\fD_1,...,\fD_{n-1};L\left(\fD_n,\fX\right)\right) \ra L\left(\fD_1,...,\fD_n;\fX\right)$
which sends $L$ in $L\left(\fD_1,...,\fD_{n-1};L\left(\fD_n,\fX\right)\right)$ to a $\mathbb{K}$-multilinear form given by formula 
$$\fD_1\times ...\times \fD_n \ni (x_1,...,x_n) \mapsto L(x_1,...,x_{n-1})(x_n) \in \fX$$
is an isometry of normed spaces.
\end{proposition}
\begin{proof}
Left for the reader as an exercise.
\end{proof}

\section{Notion of Fr{\'e}chet derivatives}
\noindent
In this section we introduce derivatives and prove their basic properties. We fix Banach spaces $\fD,\fX$ over $\mathbb{K}$. Let $U$ be an open subset of $\fD$ and let $V$ be an open subset of $\fX$.

\begin{fact}\label{fact:uniqueness_of_derivative}
Let $x$ be a point in $U$ and let $f:U\ra V$ be a function. Suppose that there are continuous $\mathbb{K}$-linear maps $L_i:\fD\ra \fX$ for $i=1,2$. If both functions
$$\big\{h\in \fD\,\big|\,h\neq 0\mbox{ and }x+h\in U\big\} \ni h\mapsto \frac{f(x+h) - f(x) - L_i(h)}{||h||} \in \fX$$
tend to zero as $h\ra 0$, then $L_1 = L_2$.
\end{fact}
\begin{proof}
By assumption the function
$$\big\{h\in \fD\,\big|\,h\neq 0\mbox{ and }x+h\in U\big\} \ni h\mapsto \left(L_1 - L_2\right)\left(\frac{h}{||h||}\right) \in \fX$$
tends to zero as $h\ra 0$. This implies that $L_1 - L_2$ sends each vector of the unit sphere in $\fD$ to zero. Thus $L_1 - L_2 = 0$ and hence $L_1 = L_2$. 
\end{proof}

\begin{definition}
Let $x$ be a point in $U$. A function $f:U\ra V$ is \textit{differentiable at point $x$} if there exists a continuous $\mathbb{K}$-linear map $L:\fD\ra \fX$ such that the function 
$$\big\{h\in \fD\,\big|\,h\neq 0\mbox{ and }x+h\in U\big\} \ni h\mapsto \frac{f(x+h) - f(x) - L(h)}{||h||} \in \fX$$
tends to zero as $h\ra 0$. Moreover, the unique continuous $\mathbb{K}$-linear map $L$ is \textit{the derivative of $f$ at $x$}.
\end{definition}

\begin{remark}\label{remark:Fr{\'e}chet_derivative}
Notion of differentiability defined above is named by some authors \textit{Fr{\'e}chet differentiability} after french mathematician Maurice Fr{\'e}chet.
\end{remark}
    
\begin{remark}\label{remark:notation_for_derivative}
Let $x$ be a point in $U$ and let $f:U\ra V$ be a function differentiable at point $x$. Then the derivative of $f$ at $x$ is usually denoted by $f'(x)$.
\end{remark}

\begin{fact}\label{fact:differentiable_is_continuous}
Let $x$ be a point in $U$ and let $f:U\ra V$ be a function differentiable at $x$. Then $f$ is continuous at $x$.
\end{fact}
\begin{proof}
Consider the function $\phi_f(h)$ defined on the set 
$$\big\{h\in \fD\,\big|\,h\neq 0\mbox{ and }x+h\in U\big\}$$
by formula $f(x+h) - f(x) = f'(x)(h) + \phi_f(h)\cdot ||h||$. By definition $\phi_f$ is continuous at zero. In order to complete the argument it suffices to note that the set $\big\{h\in \fD\,\big|\,x+h\in U\big\}$ contains a neighborhood of zero in $\fD$.
\end{proof}

\begin{definition}
A function $f:U\ra V$ is \textit{differentiable} if it is differentiable at each point of $U$.
\end{definition}

\section{Chain rule}
\noindent
Chain rule is a basic tools for calculating Fr{\'e}chet derivatives of a more complex functions.

\begin{theorem}\label{theorem:chain_rule_for_differentiable}
Let $U\subseteq \fD$, $V\subseteq \fX$, $W\subseteq \fZ$ be open subsets of Banach spaces over $\mathbb{K}$ and let $f:U\ra V$, $g:V\ra W$ be functions. Suppose that $f$ is differentiable at some point $x$ in $U$ and $g$ is differentiable at $f(x)$. Then $g\cdot f$ is differentiable at $x$ and the chain rule
$$\left(g\cdot f\right)'(x) = g'\left(f(x)\right)\cdot f'(x)$$
holds.
\end{theorem}
\begin{proof}
Let $L$ be derivative of $f$ at $x$ and let $K$ be a derivative of $g$ at $f(x)$. For $h$ in $\fD$ such that $x+h\in U$ define $\phi_f(h)$ by formula
$$f(x + h) - f(x) - L(h) = \phi_f(h)\cdot ||h||$$
Similarly for $s$ in $\fX$ such that $f(x) + s \in V$ define $\phi_g(s)$ by formula
$$g(f(x) + s) - g(f(x)) - K(s) = \phi_g(s)\cdot ||s||$$
Now pick nonzero $h$ in $\fD$ such that $x+h \in U$ and $f(x + h) \in V$. Then
$$\big|\big|g\left(f(x+h)\right)- g\left(f(x)\right) - K\left(L(h)\right)\big|\big| = \bigg|\bigg|\phi_g\big(f(x+h) - f(x)\big)\cdot ||f(x+h) - f(x)|| + K\big(\phi_f(h)\cdot ||h||\big)\bigg|\bigg| \leq$$
$$\leq \big|\big|\phi_g\big(f(x+h) - f(x)\big)\big|\big| \cdot ||f(x+h) - f(x)|| + \big|\big| K\big(\phi_f(h)\big)\big|\big|\cdot ||h|| \leq$$
$$\leq \big|\big|\phi_g\big(f(x+h) - f(x)\big)\big|\big| \cdot ||f(x+h) - f(x) - L(h)|| + \big|\big|\phi_g\big(f(x+h) - f(x)\big)\big|\big|\cdot ||L(h)|| + \big|\big| K\big(\phi_f(h)\big)\big|\big|\cdot ||h|| \leq $$
$$\leq \bigg(\big|\big|\phi_g\big(f(x+h) - f(x)\big)\big|\big| \cdot ||\phi_f(h)|| + \big|\big|\phi_g\big(f(x+h) - f(x)\big)\big|\big|\cdot ||L|| + ||K||\cdot ||\phi_f(h)||\bigg)\cdot ||h||$$
According to Fact \ref{fact:differentiable_is_continuous} we have $f(x+h) - f(x) \ra 0$ as $h\ra 0$. Hence by differentiability of $f$ at $x$ and $g$ at $f(x)$ we derive that
$$\phi_g\big(f(x+h) - f(x)\big)\ra 0,\,\phi_f(h)\ra 0$$
as $h\ra 0$. Since
$$\big\{h\in \fD\,\big|\,x+h\in U\mbox{ and }f(x+h)\in V\big\}$$
contains open neighborhood of zero in $\fD$, we derive that
$$\big\{h\in \fD\,\big|\,h\neq 0\mbox{ and }x+h\in U\big\} \ni h \mapsto \frac{g\left(f(x+h)\right)- g\left(f(x)\right) - K\left(L(h)\right)}{||h||} \in \fZ$$
tends to zero as $h\ra 0$. This completes the proof.
\end{proof}

\section{Mean value inequality}
\noindent
The main topic of this section is extremely useful inequality, which connects derivatives with local change of a function.

\begin{definition}
Let $\fD$ be an affine space over $\mathbb{K}$ and let $x_1,x_2$ are points in $\fD$. The subsets
$$[x_1,x_2] = \big\{t\cdot x_1 + (1 - t)\cdot x_2\in \fD\,\big|\,t\in [0,1]\big\},\,(x_1,x_2) = \big\{t\cdot x_1 + (1 - t)\cdot x_2\in \fD\,\big|\,t\in (0,1)\big\}$$
of $\fD$ are called \textit{the closed} and \textit{the open interval with endpoints $x_1,x_2$}, respectively.
\end{definition}

\begin{theorem}\label{theorem:mean_value_inequality}
Let $U\subseteq \fD,V\subseteq \fX$ be open subsets of Banach spaces over $\mathbb{K}$ and let $f:U\ra V$ be a continuous function. Suppose that $x_1,x_2$ are points of $\fD$ such that $[x_1,x_2] \subseteq U$ and $f$ is differentiable at every point in $(x_1,x_2)$. Then the mean value inequality
$$||f(x_1) - f(x_2)||\leq ||x_1 - x_2||\cdot \sup_{z \in (x_1,x_2)}\big|\big|f'\left(z\right)\big|\big|$$
holds.
\end{theorem}
\begin{proof}
For every $t\in [0,1]$ we define $x(t) = t\cdot x_1 + (1-t)\cdot x_2$. Consider the continuous function $g:[0,1] \ra V$ given by formula $g(t) = f\left(x(t)\right)$. Theorem \ref{theorem:chain_rule_for_differentiable} together with assumptions imply that $g$ is differentiable over $\RR$ at each point of $(0,1)$ as the composition of functions
$$[0,1]\ni t\mapsto x(t) \in U,\,f:U\ra V$$
Moreover, its derivative is the map
$$\RR\ni h \mapsto h\cdot f'\left(x(t)\right)\big(x_1 - x_2\big) \in \fX$$
for every $t\in (0,1)$. Thus the mean value inequality is implied by the inequality
$$||g(1) - g(0)||\leq \sup_{t\in (0,1)}\big|\big|g'(t)\big|\big|$$
Fix $\epsilon > 0$ and consider the set
$$S = \bigg\{s\in [0,1]\,\bigg|\,||g(s) - g(0)||\leq s\cdot \sup_{t\in (0,s)}||g'(t)||+ s\cdot \epsilon + \epsilon\bigg\}$$
We shall prove that $S$ satisfies the following assertions.
\begin{enumerate}[label=\textbf{(\arabic*)}, leftmargin=*]
\item There exists $h> 0$ such that $[0,h)\subseteq S$. 
\item For every $s$ in $S \cap (0,1)$ there exists $h > 0$ such that $s+h$ is contained in $S$.
\item For every increasing sequence $\{s_n\}_{n\in \NN}$ of elements of $S$ its limit is contained in $S$.
\end{enumerate}
The assertion \textbf{(1)} holds by continuity of $g$ at zero.\\
Let us prove \textbf{(2)}. Write
$$g(s+h) - g(s) - g'(s)\cdot h = \phi_g(h)\cdot |h|$$
for $h > 0$ such that $s + h \leq 1$. Then $\phi_g(h)$ tends to zero as $h\ra 0$ according to differentiability of $g$ at $(0,1)$. Thus
$$||g(s+h) - g(0)|| \leq ||g(s+h) - g(s)|| + ||g(s) - g(0)|| \leq$$
$$\leq ||\phi_g(h)\cdot h + g'(s)\cdot h|| + s\cdot \sup_{t\in (0,s)}||g'(t)||+ s\cdot \epsilon + \epsilon \leq$$
$$\leq h\cdot \bigg(||\phi_g(h)|| + ||g'(s)||\bigg) + s\cdot \sup_{t\in (0,s)}||g'(t)||+ s\cdot \epsilon + \epsilon$$
Since $\phi_g(h)$ tends to zero as $h\ra 0$, we may pick $h > 0$ such that $s + h \leq 1$ and $||\phi_g(h)|| \leq \epsilon$. Then
$$||g(s+h) - g(0)|| \leq h\cdot \bigg(\epsilon + ||g'(s)||\bigg) + s\cdot \sup_{t\in (0,s)}||g'(t)||+ s\cdot \epsilon + \epsilon \leq$$
$$\leq (s+h)\cdot \sup_{t\in (0,s]}||g'(t)|| + \epsilon \cdot (s + h) + \epsilon \leq (s+h)\cdot \sup_{t\in (0,s+h)}||g'(t)|| + \epsilon \cdot (s + h) + \epsilon$$
and hence clearly $s+h$ is in $S$.\\
For the proof of \textbf{(3)} fix $\{s_n\}_{n\in \NN}$ an increasing sequence of elements of $S$. Let $s$ be its limit. For every $n\in \NN$ we have
$$||g(s_n) - g(0)||\leq s_n\cdot \sup_{t\in (0,s_n)}||g'(t)||+ s_n\cdot \epsilon + \epsilon \leq s\cdot \sup_{t\in (0,s)}||g'(t)||+ s\cdot \epsilon + \epsilon$$
For $n\ra +\infty$ we obtain
$$||g(s) - g(0)|| \leq s\cdot \sup_{t\in (0,s)}||g'(t)||+ s\cdot \epsilon + \epsilon$$
by continuity of $g$ on $[0,1]$. Thus the proof of \textbf{(3)} is complete.\\
Using these three assertions we complete the proof. Note first that by \textbf{(1)} the set $S$ contains some elements of $(0,1)$ and by \textbf{(3)} it contains its least upper bound. According to \textbf{(2)} the least upper bound of $S$ cannot be contained in $(0,1)$. Thus the least upper bound of $S$ is $1$. This proves that
$$||g(1) - g(0)||\leq \sup_{t\in (0,1)}||g'(t)||+ 2\cdot \epsilon$$
for every $\epsilon > 0$ and thus
$$||g(1) - g(0)||\leq \sup_{t\in (0,1)}||g'(t)||$$
The proof is complete.
\end{proof}

\begin{corollary}\label{corollary:derivative_zero_on_connected_implies_that_function_is_constant}
Let $U\subseteq \fD,V\subseteq \fX$ be open subsets of Banach spaces over $\mathbb{K}$ and let $f:U\ra V$ be a differentiable function. If $U$ is connected and derivative of $f$ at each point of $U$ is the zero map, then $f$ is constant. 
\end{corollary}
\begin{proof}
Theorem \ref{theorem:mean_value_inequality} shows that for every open convex set $W\subseteq U$ the restriction $f_{\mid W}$ is constant. Let $y$ be some element of $f(U)$. It follows that the set $f^{-1}(y)$ is open. Let $\{x_n\}_{n\in \NN}$ be a sequence of elements of $f^{-1}(y)$ which is convergent to some point $x$ in $U$. Pick an open and convex neighborhood $W$ of $x$. Then for sufficiently large $n\in \NN$ we have $x_n\in W$ and thus $x \in f^{-1}(y)$. Therefore, $f^{-1}(y)$ is closed. Hence $f^{-1}(y)$ is a clopen nonempty subset of a connected set $U$. This shows that $U = f^{-1}(y)$. 
\end{proof}

\section{Convergence of sequences of differentiable functions}
\noindent
In this section we prove important result concerning convergence of differentiable functions. 

\begin{definition}
Let $X$ be a topological space and let $Y$ be a metric space. Let $\{f_n:X\ra Y\}_{n\in \NN}$ be a sequence of functions and $f:X\ra Y$ be a function. Suppose that for every point $x$ in $X$ there exists an open neighborhood $W$ of $x$ in $X$ such that the sequence $\{{f_n}_{\mid W}\}_{n\in \NN}$ converges uniformly to $f_{\mid W}$. Then the sequence $\{f_n\}_{n\in \NN}$ is \textit{locally uniformly convergent to $f$}.
\end{definition}

\begin{theorem}\label{theorem:convergence_of_differentiable_functions}
Let $U\subseteq \fD$ be open subset of a Banach space $\fD$ over $\mathbb{K}$, let $\fX$ be a Banach space over $\mathbb{K}$  and let $\{f_n:U\ra \fX\}_{n\in \NN}$ be a sequence of functions. Assume that the following assertions hold.
\begin{enumerate}[label=\emph{\textbf{(\arabic*)}}, leftmargin=*]
\item $U$ is connected.
\item There exists $u$ in $U$ such that the sequence $\{f_n(u)\}_{n\in \NN}$ is convergent to some element of $\fX$.
\item $f_n$ is differentiable for every $n\in \NN$.
\item The sequence of maps
$$\big\{U\ni x \mapsto f_n'(x) \in L\left(\fD,\fX\right)\big\}_{n\in \NN}$$
is locally uniformly convergent to a continuous map $g:U\ra L\left(\fD,\fX\right)$.
\end{enumerate}
Then the sequence $\{f_n\}_{n\in \NN}$ converges locally uniformly to a differentiable function $f:U\ra \fX$ and $f'(x) = g(x)$ for every $x\in U$.
\end{theorem}
\begin{proof}
Suppose that $z$ is a point of $U$ such that $\{f_n(z)\}_{n\in \NN}$ is convergent. Let $W$ be a bounded, open and convex neighborhood of $z$ in $\fD$ and assume that
$$\big\{W\ni x \mapsto f_n'(x) \in L\left(\fD,\fX\right)\big\}_{n\in \NN}$$
converges uniformly. Let $x$ be a point of $W$. Then
$$||f_n(x) - f_m(x)|| \leq ||\left(f_n(x) - f_m(x)\right) - \left(f_n(z) - f_m(z)\right)|| + ||f_n(z) - f_m(z)|| \leq$$
$$\leq ||x - z||\cdot \sup_{y\in (x,z)}\big|\big|f_n'\left(y\right) - f_m'\left(y\right)\big|\big| + ||f_n(z) - f_m(z)||$$
Hence
$$\sup_{x\in W}||f_n(x) - f_m(x)|| \leq \mathrm{diam}\left(W\right)\cdot \sup_{x\in W}\big|\big|f_n'\left(x\right) - f_m'\left(x\right)\big|\big| + ||f_n(z) - f_m(z)||$$
Since $\{f_n(z)\}_{n\in \NN}$ is convergent, $\fX$ is complete and
$$\big\{W\ni x \mapsto f_n'(x) \in L\left(\fD,\fX\right)\big\}_{n\in \NN}$$
converges uniformly, we derive that $\{{f_n}_{\mid W}\}_{n\in \NN}$ converges uniformly. This proves that the sequence $\{{f_n}_{\mid W}\}_{n\in \NN}$ is uniformly convergent for every bounded, open and convex subset $W$ of $U$ such that the sequence
$$\big\{W\ni x \mapsto f_n'(x) \in L\left(\fD,\fX\right)\big\}_{n\in \NN}$$
converges uniformly and there exists $z \in W$ such that $\{f_n(z)\}_{n\in \NN}$ is convergent.\\
We define $\cW$ as the largest open subset of $U$ such that $\{{f_n}_{\mid \cW}\}_{n\in \NN}$ converges locally uniformly. Note that $u \in \cW$ according to the first part of the proof. Suppose that $\{z_n\}_{n\in \NN}$ is a sequence of elements of $\cW$ convergent to some point $z$ in $U$. Pick a bounded, open and convex neighborhood $W$ of $z$ such that
$$\big\{W\ni x \mapsto f_n'(x) \in L\left(\fD,\fX\right)\big\}_{n\in \NN}$$
converges uniformly. Then for sufficiently large $n\in \NN$ we have $z_n \in W$. Thus $\{{f_n}_{\mid W}\}_{n\in \NN}$ is uniformly convergent by the first part of the proof and hence $z$ is in $\cW$. This implies that $\cW$ is a closed subset of $U$. Hence it is a clopen and nonempty subset of $U$. Since $U$ is connected, we have $U = \cW$ and $\{f_n\}_{n\in \NN}$ is locally uniformly convergent to some function $f:U\ra \fX$.\\
Fix $x$ in $U$ and let $W$ be an open neighborhood of zero in $\fD$ such that $[x,x+h]\subseteq U$ for every $h\in W$. We apply Theorem \ref{theorem:mean_value_inequality} to a function $k_n:W\ra \fX$ given by formula
$$k_n(h) = f_n(x+h) - f_n'(x)(h)$$
with derivative $k_n'(h) = f_n'(x+h) - f_n'(x)$ for all $h\in W$. We deduce that
$$\big|\big|f_n(x+h) - f_n(x) - f_n'(x)(h)\big|\big| = ||k_n(h) - k_n(0)|| \leq ||h||\cdot \sup_{z\in (0,h)}||k'_n(z)|| =$$
$$= ||h||\cdot \sup_{z\in (0,h)}||f_n'(x+z) - f_n'(x)|| = ||h||\cdot \sup_{z\in (x,x+h)}||f_n'(z) - f_n'(x)||$$
For $n\ra +\infty$ we obtain that
$$\big|\big|f(x+h) - f(x) - g(x)(h)\big|\big| \leq ||h||\cdot \sup_{z\in (x,x+h)}||g(z) - g(x)||$$
Since $g$ is continuous, we derive that 
$$\lim_{h\ra 0}\,\sup_{z\in (x,x+h)}||g(z) - g(x)|| = 0$$
and thus $f'(x) = g(x)$.
\end{proof}

\section{Partial derivatives}
\noindent
We fix Banach spaces $\fD_1,...,\fD_n,\fX$ over $\mathbb{K}$ for some positive integer $n$. For each $i\in \{1,...,n\}$ let $U_i\subseteq \fD_i$ be an open subset and let $V$ be an open subset of $\fX$.

\begin{definition}
Consider a function $f:U_1\times ... \times U_n\ra V$ and a point $x = (x^1,...,x^n) \in U_1\times ...\times U_n$. Fix $i$ in $\{1,...,n\}$. Suppose that the restriction 
$$f_{\mid \{(x^1,...,x^{i-1})\}\times U_i\times \{(x^{i+1},...,x^n)\}}:U_i\ra V$$
is differentiable at $x^i$. Then its derivative is \textit{the partial derivative of $f$ at $x$ along $i$-th axis}. 
\end{definition}

\begin{remark}\label{remark:notation_for_partial_derivatives}
In the situation of the definition above we usually denote the partial derivative of $f$ at $x$ along $i$-th axis by the symbol
$$\frac{\partial f}{\partial x_i}(x)$$
Note that $\frac{\partial f}{\partial x_i}(x):\fD_i\ra \fX$ is a continuous $\mathbb{K}$-linear map.
\end{remark}

\begin{proposition}\label{proposition:derivative_implies_the_existence_of_partial_derivatives}
Let $f:U_1\times ... \times U_n\ra V$ be a function differentiable at some point $x \in U_1\times ...\times U_n$. Fix $i\in \{1,...,n\}$. Then
$$\frac{\partial f}{\partial x_i}(x):\fD_i\ra \fX$$
exists and is the composition of the canonical inclusion $\fD_i \hookrightarrow \fD_1\times ... \times \fD_n$ with $f'(x):\fD_1\times ... \times \fD_n \ra \fX$.
\end{proposition}
\begin{proof}
Suppose that $\phi_f$ is a function given by formula
$$\big\{s\in \fD_1\times ...\times \fD_n\,\big|\,s\neq 0\mbox{ and }x+s\in U\big\} \ni h\mapsto \frac{f(x+s) - f(x) - f'(x)(s)}{||s||} \in \fX$$
Denote the restriction $f_{\mid \{(x^1,...,x^{i-1})\}\times U_i\times \{(x^{i+1},...,x^n)\}}$ by $f_i$ and denote the inclusion $\fD_i \hookrightarrow \fD_1\times ... \times \fD_n$ by $j_i$. Recall that for every $h\in \fD_i$ we have $||h|| = ||j_i(h)||$. Pick $h \in \fD_i$ such that $h\neq 0$ and $x_i+h \in U_i$. We have
$$\frac{f_i(x_i + h) - f_i(x_i) - \left(f'(x)\cdot j_i\right)(h)}{||h||} = \frac{f\left(x + j_i(h)\right) - f(x) - f'(x)\left(j_i(h)\right)}{||j_i(h)||} = \phi_f\left(j_i(h)\right)$$
Since by definition of $f'(x)$ the function $\phi_f(s)$ tends to zero as $s\ra 0$, we derive that $\phi_f\left(j_i(h)\right)$ tends to zero as $h\ra 0$. Thus the partial derivative of $f$ at $x$ along $i$-th axis exists and is given by formula $f'(x)\cdot j_i$.
\end{proof}
\noindent
It is reasonable to ask for the converse of Proposition \ref{proposition:derivative_implies_the_existence_of_partial_derivatives}. The next theorem gives useful answer to this question under some additional assumptions.

\begin{theorem}\label{theorem:if_partial_derivatives_exist_and_are_continuous_then_function_is_differentiable}
Let $f:U_1\times ... \times U_n\ra V$ be a function and let $x$ be a point in $U_1\times..\times U_n$. Suppose that the following two assertions hold.
\begin{enumerate}[label=\emph{\textbf{(\arabic*)}}, leftmargin=*]
\item $\frac{\partial f}{\partial x_i}(u)$ exist for each $i\in \{1,...,n\}$ and every point $u \in U_1\times ...\times U_n$. 
\item For each $i \in \{1,...,n\}$ the map
$$U_1\times ... \times U_n \ni u \mapsto \frac{\partial f}{\partial x_i}(u) \in L(\fD_i,\fX)$$
is continuous at $x$.
\end{enumerate}
Then $f$ is differentiable at $x$ and
$$f'(x) = \sum_{i=1}^n\frac{\partial f}{\partial x_i}(x)\cdot \mathrm{pr}_i$$
where $\mathrm{pr}_i:\fD_1\times ...\times \fD_n\ra \fD_i$ is the projection onto $i$-th axis.
\end{theorem}
\begin{proof}
Pick $h \in \fD_1\times ... \times \fD_n$ such that $x + h \in U_1\times ...\times U_n$. Write $x = (x^1,...,x^n)$ and $h = (h^1,...,h^n)$. Let $z_0 = x$ and 
$$z_i = (x^n,...,x^{i+1},x^{i}+h^{i},...,x^1 + h^1)$$
for each $i\in \{1,...,n\}$. Then
$$f(x + h) - f(x) - \sum_{i=1}^n\frac{\partial f}{\partial x_i}(x)(h^i) =$$
$$= \sum_{i=1}^n\left(f(z_{i}) - f(z_{i-1}) - \frac{\partial f}{\partial x_i}(z_{i-1})(h^i)\right) + \sum_{i=1}^n\left(\frac{\partial f}{\partial x_i}(z_{i-1})(h^i) - \frac{\partial f}{\partial x_i}(x)(h^i)\right)$$
For each $i\in \{1,...,n\}$ define $\phi_i(h^i)$ by formula
$$f(z_{i}) - f(z_{i-1}) - \frac{\partial f}{\partial x_i}(z_{i-1})(h^i) = \phi_i(h^i)\cdot ||h^i||$$
Then we have
$$\bigg|\bigg|f(x + h) - f(x) - \sum_{i=1}^n\frac{\partial f}{\partial x_i}(x)(h^i)\bigg|\bigg| \leq $$
$$\leq ||h||\cdot \sum_{i=1}^n\left(\big|\big|\phi_i(h^i)\big|\big| + \bigg|\bigg|\frac{\partial f}{\partial x_i}(z_{i-1}) - \frac{\partial f}{\partial x_i}(x)\bigg|\bigg|\right)$$
By definition of partial derivative along $i$-th axis at $z_{i-1}$ we derive that $\phi_i(h_i)\ra 0$ as $h_i$ tends to zero. If $h\ra 0$, then by continuity of partial derivatives we have
$$\frac{\partial f}{\partial x_i}(z_{i-1}) - \frac{\partial f}{\partial x_i}(x)\ra 0$$
for every $i$. These results imply that
$$\sum_{i=1}^n\left(\big|\big|\phi_i(h^i)\big|\big| + \bigg|\bigg|\frac{\partial f}{\partial x_i}(z_{i-1}) - \frac{\partial f}{\partial x_i}(x)\bigg|\bigg|\right) \ra 0$$
for $h\ra 0$. This completes the proof.
\end{proof}

\section{Higher order derivatives}
\noindent
We introduce higher order Fr{\'e}chet derivatives. We fix Banach spaces $\fD,\fX$ over $\mathbb{K}$. Let $U$ be an open subset of $\fD$ and let $V$ be an open subset of $\fX$.

\begin{definition}
Let $f:U\ra V$ be a function. For each natural number $m$ we define \textit{$m$-th derivative $f^{(m)}$ of $f$} by recursive formula
$$f^{(0)} = f,\,f^{(m)} = \left(f^{(m-1)}\right)'\mbox{ for }m>1$$
If $f^{(m)}$ exists for some natural number $m$, then $f$ is \textit{$m$-times differentiable on $U$}.
\end{definition}
\noindent
Note that the definition above gives $m$-th derivative as a function defined on the whole domain.

\begin{remark}\label{remark:higher_derivative_takes_values_in_multilinear_forms}
Let $f:U\ra V$ be a $m$-times differentiable function on $U$. Then $f^{(m)}$ can be identified with a function
$$f^{(m)}:U\ra L\big(\underbrace{\fD,...,\fD}_{m\,\mathrm{times}};\fX\big)$$
Indeed, the original codomain of $f^{(m)}$ is $\underbrace{L\left(\fD,L\left(\fD,...,L\left(\fD,\fX\right)...\right)\right)}_{m\,\mathrm{times}\,\fD\,\mathrm{symbol}}$ and according to Proposition \ref{proposition:multilinear_maps_canonical_isometry} we have canonical isometry
$$\underbrace{L\left(\fD,L\left(\fD,...,L\left(\fD,\fX\right)...\right)\right)}_{m\,\mathrm{times}\,\fD\,\mathrm{symbol}} = L\big(\underbrace{\fD,...,\fD}_{m\,\mathrm{times}};\fX\big)$$
Thus we can regard $f^{(m)}$ as a function on $U$ taking values in $L\big(\underbrace{\fD,...,\fD}_{m\,\mathrm{times}};\fX\big)$.
\end{remark}
\noindent
Now we introduce the notion of the higher derivative defined locally for a point in the domain.

\begin{definition}
Let $f:U\ra V$ be a function on $U$ and let $x$ be a point of $U$. Let $m$ be a positive integer. Suppose that $f$ is $(m-1)$-times differentiable on some open neighborhood of $x$ in $U$. Then \textit{$m$-th derivative of $f$ at $x$} is the derivative of $f^{(m-1)}$ at $x$. If it exists, then $f$ is \textit{$m$-times differentiable at $x$}.
\end{definition}

\begin{remark}\label{remark:higher_order_derivative_notation}
Let $x$ be a point in $U$ and let $f:U\ra V$ be a function. Let $m$ be a positive integer. Assume that $f$ is $m$-times differentiable at $x$. Then the $m$-th derivative of $f$ at $x$ is usually denoted by $f^{(m)}(x)$. Similarly to Remark \ref{remark:higher_derivative_takes_values_in_multilinear_forms} we identify $f^{(m)}(x)$ with an element in $L\big(\underbrace{\fD,...,\fD}_{m\,\mathrm{times}};\fX\big)$.
\end{remark}

\begin{theorem}\label{theorem:higher_order_derivatives_are_symmetric}
Let $f:U\ra V$ be a function on $U$ and let $x$ be a point of $U$. Suppose that $f$ is $m$-times differentiable at $x$ for some integer $m$ greater or equal $2$. Then 
$$f^{(m)}(x) \in L\big(\underbrace{\fD,...,\fD}_{m\,\mathrm{times}};\fX\big)$$
is a symmetric $\mathbb{K}$-multilinear form.
\end{theorem}
\begin{proof}[Proof for the second derivative]
Pick $h,s\in \fD$. Assume that $r$ is a positive real number greater than norms of $h$ and $s$. Let $I$ be an open interval in $\RR$ containing zero and such that
$$I \subseteq \bigg\{t\in \RR\,\bigg|\,\forall_{\xi,\eta\in \fD}\,\big(||\xi|| < r\mbox{ and }||\eta|| < r\big)\Rightarrow x + t\cdot \xi + t\cdot \eta \in U\bigg\}$$
Consider the expression
$$F(t) = f(x + t\cdot h + t\cdot s) - f(x + t\cdot s) - f(x + t\cdot h) + f(x) - t^2 \cdot f''(x)(s,h)$$
defined for $t \in I$. For fixed $t\in I$ define a function 
$$g(\xi) = f(x + t\cdot \xi + t\cdot s) - f(x + t\cdot \xi) - t^2 \cdot f''(x)(s,\xi)$$
Then $g$ is defined for each $\xi\in \fD$ with $||\xi||< r$. It is differentiable function and we have formula
$$g'(\xi) = t\cdot f'(x + t\cdot \xi + t\cdot s) - t\cdot f'(x + t\cdot \xi) - t^2\cdot f''(x)(s)$$
which follows from Theorem \ref{theorem:chain_rule_for_differentiable}. Thus by Theorem \ref{theorem:mean_value_inequality}
$$||F(t)|| = ||g(h) - g(0)|| \leq ||h||\cdot \sup_{\xi \in (0,h)}||g'(\xi)|| =$$
$$= t\cdot ||h||\cdot \sup_{\xi\in (0,h)}||f'(x + t\cdot \xi + t\cdot s) - f'(x + t\cdot \xi) - t\cdot f''(x)(s)||$$
We write
$$f'(x + t\cdot \xi + t\cdot s) = f'(x) + f''(x)(t\cdot \xi + t\cdot s) + \phi(t\cdot \xi + t\cdot s) \cdot t\cdot || \xi + s||$$
and
$$f'(x + t\cdot \xi) = f'(x) + f''(x)(t\cdot \xi) + \phi(t\cdot \xi) \cdot t\cdot ||\xi||$$
Therefore, we have
$$||F(t)|| \leq t\cdot ||h||\cdot \sup_{\xi \in (0,h)}||f'(x + t\cdot \xi + t\cdot s) - f'(x + t\cdot \xi) - t\cdot f''(x)(s)|| =$$
$$= t\cdot ||h||\cdot \sup_{\xi \in (0,h)}\big|\big|\phi(t\cdot \xi + t\cdot s) \cdot t\cdot || \xi + s|| - \phi(t\cdot \xi) \cdot t\cdot ||\xi||\big|\big| = $$
$$=t^2\cdot ||h||\cdot \sup_{\xi \in (0,h)}\big|\big|\phi(t\cdot \xi + t\cdot s)\cdot || \xi + s|| - \phi(t\cdot \xi) \cdot ||\xi||\big|\big|$$
Since $f'$ is differentiable at $x$, we derive that 
$$\lim_{t\ra 0}\phi(t\cdot \xi + t\cdot s) = \lim_{t\ra 0}\phi(t\cdot \xi) = 0$$
and hence 
$$\lim_{t\ra 0}\frac{F(t)}{t^2} = 0$$
This implies that
$$\lim_{t\ra 0}\frac{f(x + t\cdot h + t\cdot s) - f(x + t\cdot s) - f(x + t\cdot h) + f(x)}{t^2} = f''(x)(s,h)$$
Since the left hand side is symmetric with respect to $s$ and $h$, we deduce that it also converges to $f''(x)(h,s)$ as $t \ra 0$. Thus $f''(x)(s,h) = f''(x)(h,s)$. According to the fact that $h$ and $s$ are arbitrary we infer that $f''(x)$ is a symmetric $\mathbb{K}$-bilinear form.
\end{proof}

\begin{proof}[Proof of the general case]
We proved the theorem for $m = 2$. Suppose that it holds for some $m \geq 2$. We prove it for $m + 1$. For this assume that $f$ is $(m+1)$-times differentiable at $x$. By shrinking domain of $f$ we may assume that $f$ is $m$-times differentiable function on $U$. Pick elements $h_1,h_2,h_3,...,h_{m+1}\in \fD$ and fix a permutation $\sigma$ of the set $\{2,3,...,m+1\}$. Consider the composition of $f^{(m)}:U\ra L\big(\underbrace{\fD,...,\fD}_{m\,\mathrm{times}};\fX\big)$ with the map $\mathrm{ev}_{h_2,h_3,...,h_{m+1}}:L\big(\underbrace{\fD,...,\fD}_{m\,\mathrm{times}};\fX\big) \ra \fX$ given by formula $L\mapsto L(h_2,h_3,...,h_{m+1})$. According to Theorem \ref{theorem:chain_rule_for_differentiable} we derive that the derivative of this composition at $x$ is a $\mathbb{K}$-linear map 
$$f^{(m+1)}(x)(-,h_2,h_3,...,h_{m+1}):\fD\ra \fX$$
Similarly the derivative at $x$ of the composition of $f^{(m)}:U\ra L\big(\underbrace{\fD,...,\fD}_{m\,\mathrm{times}};\fX\big)$ with the map $\mathrm{ev}_{h_{\sigma(2)},h_{\sigma(3)},...,h_{\sigma(m+1)}}:L\big(\underbrace{\fD,...,\fD}_{m\,\mathrm{times}};\fX\big) \ra \fX$ given by formula $L\mapsto L(h_{\sigma(2)},h_{\sigma(3)},...,h_{\sigma(m+1)})$ is a $\mathbb{K}$-linear map
$$f^{(m+1)}(x)(-,h_{\sigma(2)},h_{\sigma(3)},...,h_{\sigma(m+1)}):\fD\ra \fX$$
Since we have $\mathrm{ev}_{h_2,h_3,...,h_{m+1}} \cdot f^{(m)} = \mathrm{ev}_{h_{\sigma(2)},h_{\sigma(3)},...,h_{\sigma(m+1)}} \cdot f^{(m)}$ (by inductive assumption), we deduce that $f^{(m+1)}(x)(-,h_2,h_3,...,h_{m+1}) = f^{(m+1)}(x)(-,h_{\sigma(2)},h_{\sigma(3)},...,h_{\sigma(m+1)})$. In particular, we derive that 
$$f^{(m+1)}(x)(h_1,h_2,h_3,...,h_{m+1}) = f^{(m+1)}(x)(h_1,h_{\sigma(2)},h_{\sigma(3)},...,h_{\sigma(m+1)})$$
for every permutation $\sigma$ of the set $\{2,...,m,m+1\}$. Next observe that
$$f^{(m+1)}(x) = \left(f^{(m-1)}\right)''(x)$$
and hence
$$f^{(m+1)}(x)(h_1,h_2,h_3,...,h_{m+1}) = \left(f^{(m-1)}\right)''(x)(h_1,h_2)(h_3,...,h_{m+1}) =$$
$$= \left(f^{(m-1)}\right)''(x)(h_2,h_1)(h_3,...,h_{m+1}) = f^{(m+1)}(x)(h_2,h_1,h_3,...,h_{m+1})$$
by the symmetry of the second derivative. Let us summarize these results in slightly different form. For every elements $h_1,h_2,h_3,...,h_{m+1}\in \fD$ we have
$$f^{(m+1)}(x)(h_1,h_2,h_3,...,h_{m+1}) = f^{(m+1)}(x)(h_{\sigma(1)},h_{\sigma(2)},h_{\sigma(3)},...,h_{\sigma(m+1)})$$
for each permutation $\sigma$ of $\{1,...,m+1\}$ such that $\sigma(1) = 1$ and 
$$f^{(m+1)}(x)(h_1,h_2,h_3,...,h_{m+1}) = f^{(m+1)}(x)(h_2,h_1,h_3,...,h_{m+1})$$
This implies that
$$f^{(m+1)}(x)(h_1,h_2,h_3,...,h_{m+1}) = f^{(m+1)}(x)(h_{\sigma(1)},h_{\sigma(2)},h_{\sigma(3)},...,h_{\sigma(m+1)})$$
for every permutation $\sigma$ of $\{1,...,m+1\}$ and every elements $h_1,h_2,h_3,...,h_{m+1}\in \fD$. This completes the proof that $f^{(m+1)}(x)$ is a symmetric $\mathbb{K}$-multilinear form.
\end{proof}

\section{Taylor formulas}
\noindent
In this section we fix Banach spaces $\fD,\fX$ over $\mathbb{K}$. Let $U$ be an open subset of $\fD$ and let $V$ be an open subset of $\fX$.

\begin{remark}\label{remark:power_notation_for_higher_order_derivative}
Let $x$ be a point in $U$ and let $f:U\ra V$ be a function. Let $m$ be a positive integer. Assume that $f$ is $m$-times differentiable at $x$. For every $h$ in $\fD$ we denote 
$$f^{(m)}(x)(\underbrace{h,...,h}_{\mathrm{m\,times}})$$
by $f^{(m)}(x)\cdot h^m$. 
\end{remark}

\begin{theorem}[Taylor's theorem]\label{theorem:Taylor_formula_with_Peano_remainder}
Let $x$ be a point in $U$ and let $f:U\ra V$ be a function. Let $m$ be a positive integer. Assume that $f$ is $m$-times differentiable at $x$. Consider the function $\phi:\big\{h\in \fD\,\big|\,h\neq 0\mbox{ and }x+h\in U\big\} \ra \fX$ defined by formula 
$$f(x + h) - \sum_{i=0}^m\frac{1}{i!}\cdot f^{(i)}(x)\cdot h^i = \phi(h)\cdot ||h||^m$$
Then $\phi(h)\ra 0$ as $h\ra 0$.
\end{theorem}
\begin{proof}
The proof goes by induction. The case $m = 0$ follows from the definition of Fr{\'e}chet derivative. Suppose that the result holds for some $m\in \NN$ and assume that $f$ is $(m+1)$-times differentiable at $x$. Without loss of generality we may assume that $f$ is $m$-times differentiable on $U$. Consider the function $g:\big\{h\in \fD\,\big|\,h\neq 0\mbox{ and }x+h\in U\big\} \ra \fX$ given by formula 
$$g(h) = f(x + h) - \sum_{i=0}^{m+1}\frac{1}{i!}\cdot f^{(i)}(x)\cdot h^i$$
There exists open subset $W$ of $\fD$ such that $W \subseteq \big\{h\in \fD\,\big|\,h\neq 0\mbox{ and }x+h\in U\big\}$ and $W\cup \{0\}$ is an open neighborhood of zero in $\fD$. According to Theorem \ref{theorem:higher_order_derivatives_are_symmetric} we have
$$g'(h) = f'(x + h) - \sum_{i=1}^{m+1}\frac{1}{(i-1)!}\cdot f^{(i)}(x)\left(\underbrace{h,...,h}_{i-1\,\mathrm{times}},-\right) = f'(x + h) - \sum_{i=0}^{m}\frac{1}{i!}\cdot \left(f'\right)^{(i)}(x)\cdot h^i$$
for every $h \in W$. Suppose that $\psi:W\ra \fX$ is a function given by formula
$$f'(x + h) - \sum_{i=0}^{m}\frac{1}{i!}\cdot \left(f'\right)^{(i)}(x)\cdot h^i = \psi(h)\cdot ||h||^m$$
Pick $h\in W$. By Theorem \ref{theorem:mean_value_inequality}
$$||h||^{m+1}\cdot ||\phi(h)|| = \bigg|\bigg|f(x + h) - \sum_{i=0}^{m+1}\frac{1}{i!}\cdot f^{(i)}(x)\cdot h^i \bigg|\bigg| = ||g(h) - g(0)|| \leq$$
$$\leq ||h||\cdot \sup_{s \in (0,h)}||g'(s)|| = ||h||\cdot \sup_{s \in (0,h)}\bigg(||s||^m \cdot ||\psi(s)||\bigg) = ||h||^{m+1}\cdot \sup_{s \in (0,h)}||\psi(s)||$$
and thus $||\phi(h)||\leq \sup_{s \in (0,h)}||\psi(s)||$. Induction hypothesis applied to $f':U\ra L(\fD,\fX)$ and point $x$ takes form
$$\lim_{h\ra 0}\psi(h) = 0$$
Therefore, we also have $\sup_{s \in (0,h)}||\psi(s)||\ra 0$ as $h\ra 0$. Hence by inequality $||\phi(h)||\leq \sup_{s \in (0,h)}||\psi(s)||$ which holds for every $h\in W$ we infer $\phi(h)\ra 0$ for $h\ra 0$. This completes the proof.
\end{proof}

\begin{theorem}\label{theorem:Taylors_inequality_with_Lagrange_remainder}
Let $f:U\ra V$ be $m$-times differentiable function for some $m\in \NN$. Suppose that $x$ is a point in $U$ and $h$ is an element of $\fD$ such that $[x,x+h] \subseteq U$ and $f$ is $(m+1)$-times differentiable at every point of $(x,x+h)$. Then 
$$\bigg|\bigg|f(x + h) - \sum_{i=0}^m\frac{1}{i!}\cdot f^{(i)}(x)\cdot h^i \bigg|\bigg| \leq \frac{||h||^{m+1}}{(m+1)!} \cdot \sup_{\xi \in (x,x+h)}||f^{(m+1)}(\xi)||$$
\end{theorem}
\begin{proof}
The proof goes by induction. The case $m = 0$ follows from Theorem \ref{theorem:mean_value_inequality}. Suppose that the result holds for some $m\in \NN$ and assume that $f$ is $(m+1)$-times differentiable on $U$ and $(m+2)$-times differentiable at every point of $(x,x+h)$, where $x \in U$ and $h\in \fD$ are such that $[x,x+h]\subseteq U$. Consider the function $g:[0,1] \ra \fX$ given by formula 
$$g(t) = f(x + t\cdot h) - \sum_{i=0}^{m+1}\frac{1}{i!}\cdot f^{(i)}(x)\cdot \left(t\cdot h\right)^i$$
According to Theorem \ref{theorem:higher_order_derivatives_are_symmetric} we have
$$g'(t) = f'(x + t\cdot h)(h) - \sum_{i=1}^{m+1}\frac{1}{(i-1)!}\cdot f^{(i)}(x)\left(\underbrace{t\cdot h,...,t\cdot h}_{i-1\,\mathrm{times}},h\right) =$$
$$= \bigg(f'(x + t\cdot h) - \sum_{i=0}^{m}\frac{1}{i!}\cdot \left(f'\right)^{(i)}(x)\cdot \left(t\cdot h\right)^i\bigg)(h)$$
for every $t \in (0,1)$. Induction hypothesis applied to $f':U\ra L(\fD,\fX)$ and point $x$ implies
$$\bigg|\bigg|f'(x + t\cdot h) - \sum_{i=0}^{m}\frac{1}{i!}\cdot \left(f'\right)^{(i)}(x)\cdot \left(t\cdot h\right)^i\bigg|\bigg| \leq \frac{t^{m+1}\cdot ||h||^{m+1}}{(m+1)!}\cdot \sup_{\xi \in (x,x+h)}\big|\big|\left(f'\right)^{(m+1)}(\xi)\big|\big|$$
Now Theorem \ref{theorem:mean_value_inequality} implies that
$$\bigg|\bigg|f(x + h) - \sum_{i=0}^{m+1}\frac{1}{i!}\cdot f^{(i)}(x)\cdot h^i \bigg|\bigg| = ||g(1) - g(0)|| \leq$$
$$\leq \sup_{t \in (0,1)}||g'(t)|| \leq ||h||\cdot  \sup_{t \in (0,1)}\bigg|\bigg|f'(x + t\cdot h) - \sum_{i=0}^{m}\frac{1}{i!}\cdot \left(f'\right)^{(i)}(x)\cdot \left(t\cdot h\right)^i\bigg|\bigg| \leq $$
$$\leq \sup_{t \in (0,1)}\frac{t^{m+1}\cdot ||h||^{m+2}}{(m+1)!} \cdot \sup_{\xi \in (x,x+h)}||\left(f'\right)^{(m+1)}(\xi)|| = \frac{t^{m+1}\cdot ||h||^{m+2}}{(m+1)!} \cdot \sup_{\xi \in (x,x+h)}||f^{(m+2)}(\xi)||$$
\end{proof}















\end{document}