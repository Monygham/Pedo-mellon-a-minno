\input ../pree.tex

\begin{document}

\title{Charges and their integrals}
\date{}
\maketitle

\section{Introduction}
\noindent
These notes are devoted to study charges i.e. finitely additive functions of sets. Here we study only extended real and complex valued charges. Our main aim is to present contents of an important Fichtenholz and Kantorovich paper \cite{fichtenholz1934operations} in a modern form. We also prove related results due to Banach, which were originally published in \cite{banach1923problemelameasure} and \cite{banach1979theorieoperationslineaires}.

\section{Extended real line charges and Jordan decomposition}

\begin{definition}
    Let $X$ be a set and let $\Sigma$ be an algebra of its subsets. Let $\mu:\Sigma \ra \ol{\RR}$ be a function. Suppose that $\mu(\emptyset) = 0$ and
    $$\mu(A \cup B) = \mu(A) + \mu(B)$$
    for every pair of disjoint sets $A,B \in \Sigma$. Then $\mu$ is \textit{a charge on $\Sigma$}.
\end{definition}

\begin{fact}\label{fact:one_side_infinity_only_for_finitely_additive}
    Let $X$ be a set and let $\Sigma$ be an algebra of its subsets. Suppose that $\mu:\Sigma \ra \ol{\RR}$ is a charge. Then the image of $\mu$ is not a superset of $\{-\infty,+\infty\}$.
\end{fact}
\begin{proof}
    Left for the reader as an exercise.
\end{proof}

\begin{definition}
    Let $X$ be a set and let $\Sigma$ be an algebra of its subsets. Let $\mu$ be a charge on $\Sigma$. If there exists $\kappa \in \RR$ such that $\mu(A) \geq
        \kappa$ for every $A \in \Sigma$, then $\mu$ is \textit{bounded from below}.
\end{definition}

\begin{definition}
    Let $X$ be a set and let $\Sigma$ be an algebra of its subsets. Let $\mu$ be a charge on $\Sigma$. If there exists $\kappa \in \RR$ such that $\mu(A) \leq
        \kappa$ for every $A \in \Sigma$, then $\mu$ is \textit{bounded from above}.
\end{definition}

\begin{definition}
    Let $X$ be a set and let $\Sigma$ be an algebra of its subsets. Let $\mu$ be a charge on $\Sigma$. If $\mu$ is bounded from below and from above, then $\mu$ is \textit{bounded}.
\end{definition}

\begin{definition}
    Let $X$ be a set and let $\Sigma$ be an algebra of its subsets. Let $\mu$ be a charge on $\Sigma$. If $\mu(A) \in [0,+\infty]$ for every $A \in \Sigma$, then $\mu$ is \textit{a nonnegative charge on $\Sigma$}.
\end{definition}
\noindent
Now we prove Jordan decomposition for charges. Our approach closely follows Stanis{\l}aw Saks \cite{saks1937theory}.

\begin{theorem}[Jordan decomposition]\label{theorem:Jordan_decomposition}
    Let $X$ be a set and let $\Sigma$ be an algebra of its subsets. Let $\mu:\Sigma \ra \ol{\RR}$ be a charge. For every $A \in \Sigma$ we define
    $$\mu_+(A) = \sup\big\{\mu(B)\,\big|\,B\in \Sigma\mbox{ and }B\subseteq A\big\},\,\mu_-(A) = \sup\big\{-\mu(B)\,\big|\,B\in \Sigma\mbox{ and }B\subseteq A\big\}$$
    Then the following assertions hold.
    \begin{enumerate}[label=\emph{\textbf{(\arabic*)}}, leftmargin=*]
        \item $\mu_+$ and $\mu_-$ are nonnegative charges on $\Sigma$.
        \item For every $A \in \Sigma$ we define
              $$|\mu|(A) = \sup \bigg\{\sum_{P\in \PP}|\mu(P)|\,\bigg|\,\PP\mbox{ is a finite partition of }A\mbox{ onto sets in }\Sigma\bigg\}$$
              Then $|\mu|$ is a nonnegative charge on $\Sigma$ and
              $$|\mu| = \mu_+ + \mu_-$$
        \item If $\mu$ is bounded from below, then $\mu_-$ is a bounded charge and
              $$\mu(A) = \mu_+(A) - \mu_-(A)$$
              for every $A \in \Sigma$.
        \item If $\mu$ is bounded from above, then $\mu_+$ is a bounded charge and
              $$\mu(A) = \mu_+(A) - \mu_-(A)$$
              for every $A \in \Sigma$.
    \end{enumerate}
\end{theorem}
\begin{proof}
    We left for the reader the proof of \textbf{(1)}.

    Fix $A \in \Sigma$. Let $\PP$ be a finite partition of $A$ onto a sets in $\Sigma$. Consider families
    $$\PP_+ = \big\{P\in \PP\,\big|\,\mu(P) > 0\big\},\,\PP_- = \big\{P\in \PP\,\big|\,\mu(P) \leq 0\big\}$$
    Clearly $\PP = \PP_+ \cup \PP_-$ and $\PP_+ \cap \PP_- = \emptyset$. Moreover, we have
    $$\sum_{P\in \PP}|\mu(P)| = \sum_{P \in \PP_+}\mu(P) - \sum_{P \in \PP_-}\mu(P) = \mu\left(\bigcup_{P\in \PP_+}P\right) - \mu\left(\bigcup_{P\in \PP_-}P\right) \leq \mu_+(A) + \mu_-(A)$$
    and thus $|\mu|(A) \leq \mu_+(A) + \mu_-(A)$ for every $A \in \Sigma$.

    Again fix arbitrary $A \in \Sigma$. There exists a sequence $\{B_n\}_{n\in \NN}$ of subsets of $A$ contained in $\Sigma$ such that $\mu(B_n) \geq 0$ for every $n\in \NN$ and $\{\mu(B_n)\}_{n\in \NN}$ is convergent to $\mu_+(A)$. Similarly there exists a sequence $\{C_n\}_{n\in \NN}$ of subsets of $A$ contained in $\Sigma$ such that $\mu(C_n) \leq 0$ for every $n \in \NN$ and $\{\mu(C_n)\}_{n\in \NN}$ is convergent to $-\mu_-(A)$. For each $n\in \NN$ we define
    $$\cS_n = \big\{B_n\setminus C_n,B_n \cap C_n,C_n \setminus B_n, A \setminus (B_n\cup C_n) \big\}$$
    and
    $$\tilde{B}_n = \bigcup\big\{S\in \cS_n\,\big|\,\mu(S) > 0\big\},\,\tilde{C}_n = \bigcup\big\{S\in \cS_n\,\big|\,\mu(S) \leq 0\big\}$$
    Then $A = \tilde{B}_n\cup \tilde{C}_n,\tilde{B}_n\cap \tilde{C}_n = \emptyset,\mu(\tilde{B}_n) \geq \mu(B_n),\mu(\tilde{C}_n) \leq \mu(C_n)$ for every $n \in \NN$. It follows from inequalities that $\{\mu(\tilde{B}_n)\}_{n\in \NN}$ is convergent to $\mu_+(A)$ and $\{\mu(\tilde{C}_n)\}_{n\in \NN}$ is convergent to $-\mu_-(A)$.

    Now we have
    $$\mu_+(A) + \mu_-(A) = \lim_{n\ra +\infty}\left(\mu(\tilde{B}_n) - \mu(\tilde{C}_n)\right) = \lim_{n\ra +\infty}\left(|\mu(\tilde{B}_n)| + |\mu(\tilde{C}_n)|\right) \leq |\mu|(A)$$
    Hence $\mu_+(A) + \mu_-(A) \leq |\mu|(A)$ for every $A \in \Sigma$. This completes the proof of \textbf{(2)}.

    Now in order to prove \textbf{(3)} assume that $\mu$ is bounded from below. Then clearly $\mu_-$ is bounded. Fix $A \in \Sigma$. As above there exist sequences $\{\tilde{B}_n\}_{n\in \NN}$ and $\{\tilde{C}_n\}_{n\in \NN}$ of subsets of $A$ contained in $\Sigma$ such that $A = \tilde{B}_n\cup \tilde{C}_n,\tilde{B}_n\cap \tilde{C}_n = \emptyset$ and
    $$\mu_+(A) = \lim_{n\ra +\infty} \mu(\tilde{B}_n),\,\mu_-(A) = \lim_{n\ra +\infty}\mu(\tilde{C}_n)$$
    Using the fact that $\mu_-(A) \in \RR$ we derive
    $$\mu(A) = \lim_{n\ra +\infty}\left(\mu(\tilde{B}_n) + \mu(\tilde{C}_n)\right) = \mu_+(A) - \mu_-(A)$$
    Since $A \in \Sigma$ is arbitrary, we deduced \textbf{(3)}.

    The proof of \textbf{(4)} is analogical to the proof of \textbf{(3)} and is omitted.
\end{proof}

\begin{definition}
    Let $X$ be a set and let $\Sigma$ be an algebra of its subsets. Let $\mu$ be a charge on $\Sigma$. If $\mu(A) \in \RR$ for every $A \in \Sigma$, then $\mu$ is \textit{a real charge on $\Sigma$}.
\end{definition}


\section{Countably additive charges and signed measures}

\begin{definition}
    Let $X$ be a set and let $\Sigma$ be an algebra of its subsets. Let $\mu:\Sigma \ra \ol{\RR}$ be a charge. Suppose that
    $$\mu\left(\bigcup_{n\in \NN}A_n\right) = \sum_{n \in \NN}\mu(A_n)$$
    for every sequence $\{A_n\}_{n\in \NN}$ of pairwise disjoint sets in $\Sigma$ such that
    $$\bigcup_{n\in \NN}A_n \in \Sigma$$
    Then $\mu$ is \textit{a $\sigma$-additive charge on $\Sigma$}.
\end{definition}

\begin{proposition}\label{proposition:sigma_additive_have_positive_and_negative_parts_which_are_sigma_additive}
    Let $X$ be a set and let $\Sigma$ be an algebra of its subsets. Let $\mu:\Sigma \ra \ol{\RR}$ be a $\sigma$-additive charge. Then $\mu_+,\mu_-$ and $|\mu|$ are $\sigma$-additive charges.
\end{proposition}
\begin{proof}
    Suppose that $\{A_n\}_{n\in \NN}$ is a sequence of pairwise disjoint subsets in $\Sigma$ such that
    $$A = \bigcup_{n\in \NN}A_n \in \Sigma$$
    Let $B \in \Sigma$ be a subset of $A$. Since $\mu$ is $\sigma$-additive, we derive
    $$\mu(B) = \sum_{n\in \NN}\mu(A_n\cap B) \leq \sum_{n\in \NN}\mu_+(A_n)$$
    Thus $\mu_+(A) \leq  \sum_{n\in \NN}\mu_+(A_n)$. On the other hand pick a family $\{B_n\}_{n\in \NN}$ of sets in $\Sigma$ such that $B_n\subseteq A_n$ and $\mu(B_n) \geq 0$ for each $n\in \NN$. Then
    $$\sum_{n \in \NN}\mu(B_n) = \lim_{N\ra +\infty}\sum_{n\leq N}\mu(B_n) = \lim_{N \ra +\infty}\mu\left(\bigcup_{n\leq N} B_n\right) \leq \mu^+(A)$$
    and hence $\sum_{n \in \NN}\mu_+(A_n) \leq \mu_+(A)$. This proves that $\mu_+$ is $\sigma$-additive.

    Since $(-\mu)_+ = \mu_-$ and $-\mu$ is $\sigma$-additive, we derive that $\mu_-$ is $\sigma$-additive by the case considered above.

    According to Theorem \ref{theorem:Jordan_decomposition} we have $|\mu| = \mu_+ + \mu_-$. Hence also $|\mu|$ is $\sigma$-additive.
\end{proof}

\begin{definition}
    Let $X$ be a set and let $\Sigma$ be a $\sigma$-algebra of its subsets. Let $\mu:\Sigma \ra \ol{\RR}$ be a $\sigma$-additive charge. Then $\mu$ is \textit{a signed measure on $\Sigma$}.
\end{definition}
\noindent
The following notion plays central role in studying structure of signed measures.

\begin{definition}
    Let $X$ be a set and let $\Sigma$ be an algebra of its subsets. Let $\mu:\Sigma \ra \ol{\RR}$ be a charge. \textit{A positive set for $\mu$} is a set $P \in \Sigma$ such that
    $$\mu(A\cap P) \geq 0,\,\mu(A\setminus P)\leq 0$$
    for every $A \in \Sigma$.
\end{definition}
\noindent
The following important result shows the existence of positive sets for signed measures.

\begin{theorem}[Hahn]\label{theorem:Hahn_decomposition}
    Let $X$ be a set and let $\Sigma$ be a $\sigma$-algebra of its subsets. Let $\mu:\Sigma \ra \ol{\RR}$ be a signed measure. Then there exists a positive set for $\mu$.
\end{theorem}
\noindent
The proof proceeds by constructing approximations for a positive set.

\begin{lemma}\label{lemma:approximate_positive_set}
    Let $X$ be a set and let $\Sigma$ be a $\sigma$-algebra of its subsets. Let $\mu:\Sigma \ra \ol{\RR}$ be a signed measure. Suppose that $\mu(A) \geq 0$ for some $A \in \Sigma$. Then for each $\epsilon > 0$ there exists a subset $Q_{\epsilon}$ of $A$ such that the following assertions hold.
    \begin{enumerate}[label=\emph{\textbf{(\arabic*)}}, leftmargin=3.0em]
        \item $Q_{\epsilon} \in \Sigma$ and $\mu(Q_{\epsilon}) \geq \mu(A)$.
        \item If $B \in \Sigma$ and $B \subseteq Q_{\epsilon}$, then $\mu(B) \geq -\epsilon$.
    \end{enumerate}
\end{lemma}
\begin{proof}[Proof of the lemma]
    Let $\fF$ be a family of all sets in $\Sigma$ contained in $A$. For any two sets $F_1,F_2\in \fF$ we define
    $$F_1 \sqsubseteq_{\epsilon}F_2$$
    if and only if $F_2 \subseteq F_1$ and $\mu(F_1 \setminus F_2) < -\epsilon$. Clearly $\sqsubseteq_{\epsilon}$ is transitive and antireflexive. Suppose that $\{F_n\}_{n\in \NN}$ is a sequence of sets in $\fF$ which is a chain with respect to $\sqsubseteq_{\epsilon}$. Then
    $$\bigcup_{n\in \NN}\left(F_n\setminus F_{n+1}\right) \in \fF$$
    and
    $$\mu\left(\bigcup_{n\in \NN}\left(F_n\setminus F_{n+1}\right)\right) = \sum_{n\in \NN}\mu\left(F_n\setminus F_{n+1}\right) < -\sum_{n\in \NN}\epsilon$$
    This contradicts the fact that $\mu(A) \geq 0$. Hence there are no infinite chains in $\fF$ with respect to $\sqsubseteq_{\epsilon}$. Thus there exists $Q_{\epsilon} \in \fF$ which is maximal with respect to $\sqsubseteq_{\epsilon}$ and is contained in a $\sqsubseteq_{\epsilon}$-chain which starts with $A$. Then $Q_{\epsilon}$ satisfies assertions.
\end{proof}

\begin{lemma}\label{lemma:positive_set}
    Let $X$ be a set and let $\Sigma$ be a $\sigma$-algebra of its subsets. Let $\mu:\Sigma \ra \ol{\RR}$ be a signed measure. Suppose that $\mu(A) \geq 0$ for some $A \in \Sigma$. Then there exists a subset $Q$ of $A$ such that the following assertions hold.
    \begin{enumerate}[label=\emph{\textbf{(\arabic*)}}, leftmargin=3.0em]
        \item $Q \in \Sigma$ and $\mu(Q) \geq \mu(A)$.
        \item If $B \in \Sigma$ and $B \subseteq Q$, then $\mu(B) \geq 0$.
    \end{enumerate}
\end{lemma}
\begin{proof}[Proof of the lemma]
    We define a sequence $\{Q_n\}_{n\in \NN}$ of sets in $\Sigma$ which are contained in $A$. We set $Q_0 = A$ and if $Q_n$ is defined for some $n \in \NN$, then we pick $Q_{n+1} \subseteq Q_n$ such that $\mu(Q_n) \leq \mu(Q_{n+1})$ and
    $$\mu\left(B\right) \geq - \frac{1}{n+1}$$
    for every $B \in \Sigma$ and $B\subseteq Q_{n+1}$. This construction is possible due to Lemma \ref{lemma:approximate_positive_set}. Define
    $$Q = \bigcap_{n\in \NN}Q_n$$
    Then $Q \in \Sigma$ and $Q\subseteq A$. Since $\{\mu(Q_n)\}_{n\in \NN}$ is nondecreasing and $Q_0 = A$, we derive
    $$\mu(A) \leq \lim_{n\ra +\infty}\mu(Q_n) = \mu(Q) $$
    Now if $B \in \Sigma$ and $B \subseteq Q$, then
    $$\mu(B) \geq -\frac{1}{n + 1}$$
    for every $n \in \NN$. Thus $\mu(B) \geq 0$. This proves that $Q$ satisfies assertions.
\end{proof}

\begin{proof}[Proof of the theorem]
    By Fact \ref{fact:one_side_infinity_only_for_finitely_additive} and changing $\mu$ to $-\mu$ if necessary, we may assume that there is no set $A \in \Sigma$ such that $\mu(A) = +\infty$. Consider the family
    $$\cP = \big\{Q \in \Sigma\,\big|\,\mu(B)\geq 0\mbox{ for each }B\subseteq Q\mbox{ such that }B \in \Sigma\big\}$$
    Denote by $\alpha$ the least upper bound of $\mu(Q)$ for $Q \in \cP$. There exists a sequence $\{Q_n\}_{n\in \NN}$ such that
    $$\lim_{n\ra +\infty}\mu(Q_n) = \alpha$$
    Define
    $$P = \bigcup_{n\in \NN}Q_n$$
    Then $P \in \cP$ and $\mu(P) = \alpha$. Since by assumption $\mu(P)$ is finite, we derive that $\alpha \in \RR$. Assume that there exists a set $A \in \Sigma$ such that $\mu(A) > 0$ and $A \subseteq X\setminus P$. Then by Lemma \ref{lemma:positive_set} there exists $Q \in \cP$ such that $Q \subseteq A$ and $\mu(A) \leq \mu(Q)$. Then $Q\cup P \in \cP$ and
    $$\alpha = \mu(P) < \mu(P) + \mu(Q) = \mu(Q\cup P) \leq \alpha$$
    This is a contradiction. Hence $P$ is a positive set for $\mu$.
\end{proof}

\begin{corollary}\label{corollary:signed_measures_are_one_sided_bounded}
    Let $X$ be a set and let $\Sigma$ be a $\sigma$-algebra of its subsets. Let $\mu:\Sigma \ra \ol{\RR}$ be a signed measure. Then $\mu$ is either bounded from below or from above.
\end{corollary}
\begin{proof}
    Indeed, let $P \in \Sigma$ be a positive set of $\mu$. Then $\mu_+(X) = \mu(P),\,\mu_-(X) = \mu(X\setminus P)$ and both cannot be infinite by Fact \ref{fact:one_side_infinity_only_for_finitely_additive}.
\end{proof}

\begin{definition}
    Let $X$ be a set and let $\Sigma$ be a $\sigma$-algebra of its subsets. Let $\mu$ be a signed measure on $\Sigma$ which is at the same time real charge. Then $\mu$ is \textit{a real measure on $\Sigma$}.
\end{definition}

\begin{corollary}\label{corollary:real_measures_are_bounded}
    Let $X$ be a set and let $\Sigma$ be a $\sigma$-algebra of its subsets. Let $\mu:\Sigma \ra \RR$ be a real measure. Then $\mu$ is bounded.
\end{corollary}
\begin{proof}
    Indeed, let $P \in \Sigma$ be a positive set of $\mu$. Then $\mu_+(X) = \mu(P),\,\mu_-(X) = \mu(X\setminus P)$ and both are finite, since $\mu$ is real.
\end{proof}

\begin{remark}\label{remark:measure_is_nonnegative_signed_measure}
    Note that a measure as it was introduced in \cite{Integration} is a nonnegative, signed meaasure.
\end{remark}


\section{Examples of charges}
\noindent
We now give to examples of charges. The first is set-theoretic and is an interesting application of axiom of choice. 

\begin{example}\label{example:ultrafilter_charge}
    For the notion of ultrafilter we refer to \cite{Filters_in_topology}. Let $X$ be a set and let $\cF$ be an ultrafilter of subsets of $X$. Consider a function given by formula
    $$\mu(A) = \begin{cases}
        1 & \mbox{ if }A \in \cF \\
        0 & \mbox{ otherwise }   \\
    \end{cases}
    $$
    for every $A \subseteq X$. Then $\mu$ is a $\{0,1\}$-valued charge on the algebra of all subsets of $X$.
\end{example}
\noindent
Our second example has analytical nature.

\begin{example}\label{example:charge_from_series}
    Let $\{a_n\}_{n\in \NN}$ be a sequence of real numbers such that the series
    $$\sum_{n\in \NN}a_n$$
    is convergent. Let $\Sigma$ be an algebra of all finite and cofinite subsets in $\NN$. We define
    $$\mu(A) = \sum_{n \in A} a_n$$
    for every $A \in \Sigma$. Then $\mu:\Sigma \ra \ol{\RR}$ is a charge.
\end{example}

\begin{remark}\label{remark:unbounded_charge_from_not_absolutely_convergent_series}
    Consider a sequence $\{a_n\}_{n\in \NN}$ such that the series
    $$\sum_{n\in \NN}a_n$$
    is convergent, but not absolutely convergent. Then the charge defined by $\{a_n\}_{n\in \NN}$ as in Example \ref{example:charge_from_series} is real but not bounded from below or above. In particular, both $\mu_+$ and $\mu_-$ assume infinite values and hence $\mu_+ - \mu_-$ is undefined.
\end{remark}

\begin{remark}\label{remark:one_sided_bounded_charge_without_positive_set}
    Consider a sequence $\{a_n\}_{n\in \NN}$ such that the series
    $$\sum_{n\in \NN}a_n$$
    is absolutely convergent. Assume also that for for every $N \in \NN$ there exist $n,m\geq N$ such that $a_n > 0$ and $a_m < 0$. Then the charge defined by $\{a_n\}_{n\in \NN}$ as in Example \ref{example:charge_from_series} is real, bounded and $\sigma$-additive. On the other hand it has no positive sets.
\end{remark}
\noindent
The remaining part of this section is devoted to example which is of number theoretic nature.

\begin{example}\label{example:natural_density_charge}
    For each $n \in \NN_+$ let $[n)$ denote the subset of $\NN$ consisting of consecutive numbers from $0$ to $n-1$. Let $A \subseteq \NN$ be a subset. We define \textit{the upper density of $A$} and \textit{the lower density of $A$}, respectively, by formulas
    $$\ol{d}(A) = \limsup_{n\ra +\infty}\frac{|A\cap [n)|}{n},\,\underline{d}(A) = \liminf_{n\ra +\infty}\frac{|A\cap [n)|}{n}$$
    If $\ol{d}(A) = \underline{d}(A)$ for some $A \subseteq \NN$, then their value is \textit{the density of $A$} and is also denoted by $d(A)$. We set
    $$\Sigma = \big\{A\subseteq \NN\,\big|\,d(A)\mbox{ exists }\big\}$$
    Then $\Sigma$ is an algebra of subsets of $\NN$. Moreover, $d$ is a real and nonnegative charge on $\Sigma$.
\end{example}

\begin{proposition}\label{proposition:natural_density_charges_are_not_sigma_additive}
    Let $\Sigma$ be an algebra of subsets of $\NN$ which contains each finite subset of $\NN$ and a family $\{d\cdot \NN\}_{d \in \NN_+}$. Suppose that $\mu$ is a charge on $\Sigma$ such that
    $$\mu(d\cdot \NN) = \frac{1}{d}$$
    for every $d \in \NN_+$. Then $\mu$ is not $\sigma$-additive.
\end{proposition}
\begin{proof}
    Suppose that $\mu$ is a charge on $\Sigma$ such that
    $$\mu(d\cdot \NN) = \frac{1}{d}$$
    for every $d \in \NN_+$. Assume that $d_1,...,d_s \in \NN_+$ are pairwise coprime. Then inclusion-exclusion principle implies that
    $$\mu\left(\bigcup_{k=1}^sd_k\cdot \NN\right) = 1 - \prod_{i=1}^s\left(1 - \frac{1}{d_i}\right)$$

    Let $\PP$ be the set of all primes. For each $n \in \NN_+$ let $\nu_p(n) \in \NN$ be the exponent of $p \in \PP$ in prime factorization of $n$. Fix now a sequence $\alpha = \{\alpha_p\}_{p\in \PP}$ of elements in $\NN_+$ such that $\alpha_p = 1$ for all but finitely many $p \in \PP$. Consider the set
    $$\Gamma_{\alpha} = \big\{n\in \NN_+\,\big|\,\nu_p(n) \geq \alpha_p\mbox{ for some }p\in \PP\big\}$$
    Clearly $\Gamma_{\alpha}$ is cofinite and
    $$\Gamma_{\alpha} = \bigcup_{p\in \PP}p^{\alpha_p}\cdot \NN$$
    If $\mu$ is $\sigma$-additive, then
    $$\mu(\Gamma_{\alpha}) = \lim_{N \ra +\infty}\mu\left(\bigcup_{p < N}p^{\alpha_p}\cdot \NN\right) = 1 - \lim_{N\ra +\infty}\prod_{p < N}\left(1 - \frac{1}{p^{\alpha_p}}\right) = 1$$
    Now for fixed $n \in \NN \cap (1,+\infty)$ we pick $\alpha = \{\alpha_p\}_{p\in \PP}$ and $\beta = \{\beta_p\}_{p\in \PP}$ such that
    $$\alpha_p = \begin{cases}
            \nu_p(n) & \mbox{ if }\nu_p(n) > 0 \\
            1        & \mbox{ otherwise }
        \end{cases}$$
    for each $p \in \PP$ and
    $$\beta_p = \begin{cases}
            \nu_p(n) + 1 & \mbox{ if }\nu_p(n) > 0 \\
            1            & \mbox{ otherwise }
        \end{cases}$$
    Then $\mu(\Gamma_{\alpha}) = \mu(\Gamma_{\beta}) = 1$ and hence $\mu(\{n\}) = \mu\left(\Gamma_{\alpha} \setminus \Gamma_{\beta}\right) = 0$. This holds for all $n \in \NN \cap (1,+\infty)$. Moreover, by $\sigma$-additivity it follows that
    $$\mu(\{0\}) = \mu\left(\bigcap_{n \in \NN}2^n \cdot \NN\right) = \lim_{n\ra +\infty}\mu(2^n\cdot \NN) = \lim_{n\ra +\infty}\frac{1}{2^n} = 0$$
    and hence
    $$\mu(2\cdot \NN) = \sum_{n\in \NN}\mu(\{2\cdot n\}) = 0$$
    This contradicts the fact that $\mu(2\cdot \NN) \neq 0$.
\end{proof}

\begin{remark}\label{remark:natural_density_is_not_sigma_additive}
    Let $d$ be the density charge defined in Example \ref{example:natural_density_charge}. Then Proposition \ref{proposition:natural_density_charges_are_not_sigma_additive} implies that $d$ is not $\sigma$-additive.
\end{remark}

\section{Complex charges and spaces of bounded charges}

\begin{definition}
    Let $X$ be a set and let $\Sigma$ be an algebra of its subsets. Let $\mu:\Sigma \ra \CC$ be a function. Suppose that $\mu(\emptyset) = 0$ and
    $$\mu(A \cup B) = \mu(A) + \mu(B)$$
    for every pair of disjoint sets $A,B \in \Sigma$. Then $\mu$ is \textit{a complex charge on $\Sigma$}.
\end{definition}

\begin{remark}\label{remark:each_real_charge_is_complex}
    Let $X$ be a set and let $\Sigma$ be an algebra of its subsets. Each real charge on $\Sigma$ is complex.
\end{remark}

\begin{definition}
    Let $X$ be a set and let $\Sigma$ be an algebra of its subsets. Let $\mu:\Sigma \ra \CC$ be a charge. Suppose that
    $$\mu\left(\bigcup_{n\in \NN}A_n\right) = \sum_{n \in \NN}\mu(A_n)$$
    for every sequence $\{A_n\}_{n\in \NN}$ of pairwise disjoint sets in $\Sigma$ such that
    $$\bigcup_{n\in \NN}A_n \in \Sigma$$
    Then $\mu$ is \textit{a $\sigma$-additive charge on $\Sigma$}.
\end{definition}

\begin{definition}
    Let $X$ be a set and let $\Sigma$ be a $\sigma$-algebra of its subsets. Let $\mu:\Sigma \ra \CC$ be a charge. If $\mu$ is $\sigma$-additive, then $\mu$ is \textit{a complex measure on $\Sigma$}.
\end{definition}

\begin{fact}\label{fact:total_variation_is_a_charge}
    Let $X$ be a set and let $\Sigma$ be an algebra of its subsets. Let $\mu:\Sigma \ra \CC$ be a charge. For every $A \in \Sigma$ we define
    $$|\mu|(A) = \sup \bigg\{\sum_{P\in \PP}|\mu(P)|\,\bigg|\,\PP\mbox{ is a finite partition of }A\mbox{ onto sets in }\Sigma\bigg\}$$
    Then $|\mu|$ is a nonnegative charge on $\Sigma$.

    Moreover, if $\mu$ is $\sigma$-additive, then also $|\mu|$ is $\sigma$-additive.
\end{fact}
\begin{proof}
    The fact that $|\mu|$ is a charge is left for the reader as an exercise.

    Assume now that $\mu$ is $\sigma$-additive. Suppose that $\{A_n\}_{n\in \NN}$ is a sequence of pairwise disjoint subsets in $\Sigma$ such that
    $$A = \bigcup_{n\in \NN}A_n \in \Sigma$$
    Pick a finite partition $\PP$ of $A$ onto sets in $\Sigma$. Since $\mu$ is $\sigma$-additive, we derive that
    $$\sum_{P\in \PP}|\mu(P)| = \sum_{P\in \PP}\bigg|\sum_{n\in \NN}\mu(A_n \cap P)\bigg| \leq$$
    $$\leq \sum_{P\in \PP}\sum_{n\in \NN}|\mu(A_n \cap P)| = \sum_{n\in \NN}\sum_{P\in \PP}|\mu(A_n \cap P)| \leq \sum_{n \in \NN}|\mu|(A_n)$$
    This proves that $|\mu|(A) \leq \sum_{n \in \NN}|\mu|(A_n)$. On the other hand for each $n \in \NN$ pick a finite partition $\PP_n$ of $A_n$ onto a sets in $\Sigma$. Then
    $$\sum_{n \in \NN}\sum_{P \in \PP_n}|\mu(P)| = \lim_{N\ra +\infty}\sum_{n\leq N}\sum_{P \in \PP_n}|\mu(P)| \leq $$
    $$\leq \limsup_{N \ra +\infty}\left(\sum_{n\leq N}\sum_{P \in \PP_n}|\mu(P)| + \bigg|\mu\left(A\setminus \bigcup_{n\leq N}A_n\right)\bigg|\right) \leq |\mu|(A)$$
    Hence $\sum_{n \in \NN}|\mu|(A_n) \leq |\mu|(A)$. This completes the proof of $\sigma$-additivity of $\mu$.
\end{proof}

\begin{theorem}\label{theorem:charge_is_bounded_if_variation_is_finite}
    Let $X$ be a set and let $\Sigma$ be an algebra of its subsets. Let $\mu:\Sigma \ra \CC$ be a charge. Then the following assertions are equivalent.
    \begin{enumerate}[label=\emph{\textbf{(\roman*)}}, leftmargin=*]
        \item There exists $\kappa \in \RR_+$ such that
              $$|\mu(A)|\leq \kappa$$
              for every $A \in \Sigma$.
        \item $|\mu|$ is a bounded charge.
    \end{enumerate}
\end{theorem}
\begin{proof}
    Assume that there exists $\kappa \in \RR_+$ such that $|\mu(A)| \leq \kappa$ for every $A \in \Sigma$. For each $A \in \Sigma$ write
    $$\mu(A) = \mu_r(A) + \sqrt{-1}\cdot \mu_i(A)$$
    where $\mu_r(A),\mu_i(A) \in \RR$. Then $\mu_r,\mu_i:\Sigma \ra \RR$ are real charges and $|\mu_r(A)|,|\mu_i(A)| \leq \kappa$ for every $A \in \Sigma$. Part \textbf{(2)} of Theorem \ref{theorem:Jordan_decomposition} implies that $|\mu_r|,|\mu_i|$ are bounded. Note that
    $$|\mu|(A) \leq |\mu_r|(A) + |\mu_i|(A)$$
    for every $A \in \Sigma$. Hence $|\mu|$ is bounded. This proves that $\textbf{(i)}\Rightarrow \textbf{(ii)}$.

    Suppose now that $|\mu|$ is a bounded charge. Then there exists $\kappa \in \RR_+$ such that $|\mu|(A) \leq \kappa$ for every $A \in \Sigma$. Since $|\mu(A)| \leq |\mu|(A)$ for every $A \in \Sigma$, we deduce that $|\mu(A)|\leq \kappa$ for each $A \in \Sigma$. This completes the proof of $\textbf{(ii)}\Rightarrow \textbf{(i)}$.
\end{proof}

\begin{definition}
    Let $X$ be a set and let $\Sigma$ be an algebra of its subsets. Let $\mu:\Sigma \ra \CC$ be a charge. If $|\mu|$ is bounded, then $\mu$ is \textit{a bounded complex charge on $\Sigma$}.
\end{definition}

\begin{definition}
    Let $X$ be a set and let $\Sigma$ be an algebra of its subsets. Let $\mu:\Sigma \ra \CC$ be a charge. We define
    $$\lVert \mu \rVert = |\mu|(X)$$
    Then $\lVert \mu \rVert$ is \textit{the total variation of $\mu$}.
\end{definition}

\begin{theorem}\label{theorem:space_of_bounded_complex_charges}
    Let $X$ be a set and let $\Sigma$ be an algebra of its subsets. Consider the set
    $$\mathrm{ba}(\Sigma, \CC) = \big\{\mu:\Sigma \ra \CC\,\big|\,\mu\mbox{ is a bounded charge on }\Sigma\big\}$$
    Then the following assertions hold.
    \begin{enumerate}[label=\emph{\textbf{(\arabic*)}}, leftmargin=*]
        \item $\mathrm{ba}(\Sigma, \CC)$ is a $\CC$-linear space with respect to canonical operations of addition of charges and multiplication by complex scalars.
        \item The map
              $$\mathrm{ba}(\Sigma, \CC) \ni \mu \mapsto \lVert \mu\rVert \in [0,+\infty)$$
              is a norm.
        \item Let $\{\mu_n\}_{n\in \NN}$ be a Cauchy sequence with respect to $\lVert-\rVert$. Then $\{\mu_n\}_{n\in \NN}$ is convergent to some $\mu \in \mathrm{ba}(\Sigma, \CC)$. Moreover, if $\{\mu_n\}_{n\in \NN}$ are $\sigma$-additive, then $\mu$ is $\sigma$-additive.
        \item Let $\mathrm{ba}(\Sigma,\RR)$ be an $\RR$-linear subspace of $\mathrm{ba}(\Sigma, \CC)$ that consists of real bounded charges. Then $\mathrm{ba}(\Sigma,\RR)$ is closed with respect to $\lVert-\rVert$.
    \end{enumerate}
\end{theorem}
\begin{proof}
    Proofs of \textbf{(1)} and \textbf{(2)} are left for the reader.

    Let $\{\mu_n\}_{n\in \NN}$ be a Cauchy sequence with respect to $\lVert-\rVert$. For every $A \in \Sigma$ and each $n,m\in \NN$ we have
    $$|\mu_n(A) - \mu_m(A)|\leq \lVert \mu_n - \mu_m\rVert$$
    Since $\CC$ with the usual absolute value is complete, we derive that there exists $\mu(A) \in \CC$ such that $\{\mu_n(A)\}_{n\in \NN}$ converges to $\mu(A)$. Now pick at most countable family $\cF$ of pairwise disjoint sets in $\Sigma$ such that
    $$\bigcup_{F \in \cF}F\in \Sigma$$
    Suppose also that
    $$\mu_n\left(\bigcup_{F\in \cF}F\right) = \sum_{F\in \cF}\mu_n(F)$$
    for every $n \in \NN$. We define a measure $u$ on the power set of $\cF$ by formula
    $$u(Z) = |Z|$$
    for every $Z \subseteq \cF$. Let $L^1(u,\CC)$ is a space of complex valued functions defined on $\cF$ which are integrable with respect to $u$. In particular, $L^1(u,\CC)$ is a Banach space over $\CC$ with norm
    $$\lVert f \rVert_1 = \int_{\cF} f\,du = \sum_{F\in \cF}|f(F)|$$
    and there is integral functional
    $$\int_{\cF}f\,du = \sum_{F\in \cF}f(F)$$
    with respect to $u$. For the details we refer to \cite{Integration}. For each $n \in \NN$ let $f_n:\cF\ra \CC$ be a function given by formula $f_n(F) = \mu_n(F)$ for $F \in \cF$. Since $\lVert \mu_n\rVert$ is finite for each $n \in \NN$ by Theorem \ref{theorem:charge_is_bounded_if_variation_is_finite}, we derive that $f_n$ is an element of $L^1(u,\CC)$ for every $n \in \NN$. Moreover, the distance of $f_n$ and $f_m$ in $L^1(u,\CC)$ is bounded by $\lVert \mu_n - \mu_m\rVert$ for all pairs $n,m \in \NN$. Hence the sequence $\{f_n\}_{n \in \NN}$ is convergent in $L^1(u,\CC)$. It is also pointwise convergent to a function $f:\cF \ra \CC$ given by formula $f(F) = \mu(F)$ for $F \in \cF$. By general results in \cite{Integration} we deduce that $f$ is a limit of $\{f_n\}_{n\in \NN}$ in $L^1(u,\CC)$ and from considerations above we have inequality
    $$\lVert f - f_n\rVert_1 = \lim_{m\ra +\infty}\lVert f_m - f_n\rVert_1 \leq \limsup_{m \ra +\infty}\lVert \mu_m - \mu_n\rVert$$
    Let us note some consequences of this fact.
    \begin{itemize}
        \item From the convergence of integrals with respect to $u$ we deduce
              $$\mu\left(\bigcup_{F\in \cF}F\right) = \lim_{n\ra +\infty}\mu_n\left(\bigcup_{F\in \cF}F\right) = \lim_{n\ra +\infty}\sum_{F\in \cF}\mu_n(F) = \sum_{F\in \cF}\mu(F)$$
        \item The convergence in $\lVert-\rVert_1$ implies that
              $$\sum_{F \in \cF}|\mu(F)| = \lim_{n\ra +\infty}\sum_{F\in \cF}|\mu_n(F)| = \lim_{n\ra +\infty}\lVert \mu_n\rVert \leq \sup_{n\in \NN}\lVert \mu_n\rVert$$
        \item Moreover, we have
              $$\sum_{F\in \cF}|(\mu - \mu_n)(F)| = \lVert f - f_n\rVert_1 \leq \limsup_{m\ra +\infty}\lVert \mu_m - \mu_n\rVert$$
    \end{itemize}
    Note that these assertions hold for every family $\cF$ which satisfies the conditions specified above. Hence from the first assertion it follows that $\mu$ is a charge and if $\{\mu_n\}_{n\in \NN}$ are $\sigma$-additive, then also $\mu$ is $\sigma$-additive. Next the second statement shows that $\mu$ is bounded. From the last assertion we deduce that $\mu$ is a limit of $\{\mu_n\}_{n\in \NN}$ with respect to $\lVert-\rVert$. This completes the proof of \textbf{(3)}.

    The proof of \textbf{(4)} follows from the investigation of the proof of \textbf{(3)} above. The details are left for the reader.
\end{proof}

\begin{corollary}\label{corollary:spaces_of_sigma_additive_charges}
    Let $X$ be a set and let $\Sigma$ be an algebra of its subsets. Consider the set
    $$\mathrm{bca}(\Sigma, \CC) = \big\{\mu:\Sigma \ra \CC\,\big|\,\mu\mbox{ is a bounded and $\sigma$-additive charge on }\Sigma\big\}$$
    Then $\mathrm{bca}(\Sigma,\CC)$ is a $\CC$-linear subspace of $\mathrm{ba}(\Sigma,\CC)$ closed with respect to total variation norm.
\end{corollary}
\begin{proof}
    Closedeness follows from Theorem \ref{theorem:space_of_bounded_complex_charges}. The fact that $\mathrm{bca}(\Sigma, \CC)$ is $\CC$-linear subspace of $\mathrm{ba}(\Sigma,\CC)$ is left as an exercise for the reader.
\end{proof}

\begin{remark}\label{remark:spaces_of_bounded_charges}
    Let $X$ be a set and let $\Sigma$ be an algebra of its subsets. We have the following diagram of Banach spaces and their inclusions.
    \begin{center}
        \begin{tikzpicture}
            [description/.style={fill=white,inner sep=2pt}]
            \matrix (m) [matrix of math nodes, row sep=3em, column sep=1em,text height=1.5ex, text depth=0.25ex]
            {                                                                                                                                                                                                                                                                                                                                                                                                                                           & \mathrm{ba}(\Sigma,\CC)  & \\
                                                                                                                                                                                                                                                                                                                                                                                                                               \mathrm{bca}(\Sigma,\CC) &                          & \mathrm{ba}(\Sigma,\RR)                                                                                                                                                                                                                                                                                                                                                                                            \\
                                                                                                                                                                                                                                                                                                                                                                                                                                                        & \mathrm{bca}(\Sigma,\RR) & \\} ;
            \path[right hook->,line width=0.8pt,font=\scriptsize]
            (m-2-3) edge node {$  $} (m-1-2)
            (m-3-2) edge node {$  $} (m-2-3);
            \path[left hook->,line width=0.8pt,font=\scriptsize]
            (m-2-1) edge node {$  $} (m-1-2)
            (m-3-2) edge node {$  $} (m-2-1);
        \end{tikzpicture}
    \end{center}
    In the diagram $\mathrm{bca}(\Sigma, \RR)$ is the intersection of $\mathrm{ba}(\Sigma,\RR)$ and $\mathrm{bca}(\Sigma,\CC)$ i.e. a Banach space over $\RR$ of all real, bounded and $\sigma$-additive charges on $\Sigma$.
\end{remark}

\section{Integration and Fichtenholz-Kantorovich theorem}

\begin{definition}
    Let $X$ be a set and let $\Sigma$ be an algebra of its subsets. Suppose that $Y$ is a set. Consider a function $s:X\ra Y$ such that $s(X)$ is finite and $s^{-1}(y) \in \Sigma$ for every $y \in Y$. Then $s$ is \textit{a $\Sigma$-simple function}.
\end{definition}
\noindent
In this section we denote by $\mathbb{K}$ either $\RR$ or $\CC$ with their usual absolute values. We also fix a set $X$ and a $\sigma$-algebra $\Sigma$ of subsets of $X$.

We define
$$B(\Sigma,\mathbb{K}) = \big\{f:X\ra \mathbb{K}\,\big|\,f\mbox{ is measurable and bounded}\big\}$$
Clearly $B(\Sigma, \mathbb{K})$ is a $\mathbb{K}$-linear subspace of the space of all functions $X\ra \mathbb{K}$. Moreover, the function
$$B(\Sigma,\mathbb{K}) \ni f \mapsto \sup_{x\in X}|f(x)|\in [0,+\infty)$$
is a norm on $B(\Sigma, \mathbb{K})$. We denote it by $\lVert-\rVert_{\infty}$. We also define $S(\Sigma,\mathbb{K})\subseteq B(\Sigma,\mathbb{K})$ as a subset consisting of all $\Sigma$-simple $\mathbb{K}$-valued functions.

\begin{theorem}\label{theorem:space_of_bounded_measurable_functions}
    The $\mathbb{K}$-linear space $B(\Sigma,\mathbb{K})$ with norm $\lVert-\rVert_{\infty}$ is a Banach space over $\mathbb{K}$. Moreover, $S(\Sigma,\mathbb{K})$ is its $\mathbb{K}$-linear and dense subspace.
\end{theorem}
\begin{proof}
    In order to show that $\lVert-\rVert_{\infty}$ is complete we fix a sequence $\{f_n\}_{n\in \NN}$ of elements of $B(\Sigma,\mathbb{K})$ which is Cauchy with respect to $\lVert-\rVert_{\infty}$. For each $x \in X$ and every $n,m\in \NN$ we have
    $$|f_n(x) - f_m(x)|\leq \sup_{x\in X}|f_n(x) - f_m(x)| = \lVert f_n - f_m \rVert_{\infty}$$
    In particular, the sequence $\{f_n(x)\}_{n\in \NN}$ is a Cauchy sequence in $\mathbb{K}$ with respect to $|-|$ for every $x \in X$. We define $f(x)$ as the limit of $\{f_n(x)\}_{n\in \NN}$ in $\mathbb{K}$ for every $x \in X$. Then $f:X\ra \mathbb{K}$ is a function. Since $\{f_n\}_{n\in \NN}$ is a sequence of measurable functions pointwise convergent to $f$, we deduce by results in \cite{Integration} that $f$ is measurable. Fix $\epsilon > 0$ and consider $N \in \NN$ such that for all $n,m\geq N$ the inequality
    $$\sup_{x\in X}|f_m(x) - f_n(x)|\leq \epsilon$$
    holds. Then for $n \geq N$ we have
    $$\sup_{x\in X}|f(x) - f_n(x)| = \sup_{x\in X}\lim_{m\ra +\infty}|f_m(x) - f_n(x)| \leq \epsilon$$
    This proves that $f$ is bounded and $\{f_n\}_{n\in \NN}$ converges to $f$ with respect to $\lVert -\rVert_{\infty}$. Thus $B(\Sigma,\mathbb{K})$ is a Banach space over $\mathbb{K}$ with respect to $\lVert-\rVert_{\infty}$.

    Obviously $S(\Sigma,\mathbb{K})$ is a $\mathbb{K}$-linear subspace of $B(\Sigma,\mathbb{K})$. For $\alpha \in \mathbb{K}$ and $r \in \RR_+$ we denote by $D(\alpha,r)$ the open disc in $\mathbb{K}$ centered in $\alpha$ and with radius $r$. Fix $f \in B(\Sigma,\mathbb{K})$ and let $\epsilon > 0$. Since $\mathbb{K}$ is locally compact, there exist $n\in \NN_+$ and elements $\alpha_1,...,\alpha_n \in \mathbb{K}$ such that
    $$\big\{\alpha \in \mathbb{K}\,\big|\,|\alpha| \leq \lVert f\rVert_{\infty}\big\}\subseteq \bigcup_{i=1}^nD(\alpha_i,\epsilon)$$
    Define
    $$B_i = D(\alpha_i,\epsilon)\setminus \bigcup_{j < i}D(\alpha_j,\epsilon)$$
    for every $1\leq i \leq n$. We define
    $$s = \sum_{i=1}^n\alpha_i\cdot \mathbb{1}_{f^{-1}(B_i)}$$
    Then $s:X\ra \mathbb{K}$ is a well defined element of $S(\Sigma,\mathbb{K})$. Next pick $x \in X$. Since $X$ is the union of $\{f^{-1}(B_i)\}_{i=1}^n$,
    there exists $i$ such that $x \in f^{-1}(B_i)$. Then $s(x) = \alpha_i$ and $f(x) \in B_i\subseteq D(\alpha_i,\epsilon)$. Thus
    $$|f(x) - s(x)| = |f(x) - \alpha_i|\leq \epsilon$$
    Since $x$ is arbitrary, we derive that $\lVert f - s\rVert_{\infty} \leq \epsilon$. This implies that $S(\Sigma,\mathbb{K})$ is dense in $B(\Sigma,\mathbb{K})$.
\end{proof}
\noindent
Theorem \ref{theorem:space_of_bounded_measurable_functions} implies the existence of integration with respect to bounded and $\mathbb{K}$-valued charges on $\Sigma$. To explain this we fix a charge $\mu:\Sigma \ra \mathbb{K}$.

\begin{definition}
    For $s \in S(\Sigma,\mathbb{K})$ we set
    $$\int_Xs\,d\mu = \sum_{\alpha \in \mathbb{K}}\alpha\cdot \mu\left(s^{-1}(\alpha)\right)$$
    and call it \textit{the integral of $s$ with respect to $\mu$}.
\end{definition}

\begin{fact}\label{fact:charge_integral_is_linear_and_continuous_on_space_of_simple}
    The map
    $$S(\Sigma, \mathbb{K})\ni s \mapsto \int_Xs\,d\mu\in \mathbb{K}$$
    is $\mathbb{K}$-linear and its norm is equal to $\lVert \mu \rVert$.
\end{fact}
\begin{proof}
    The fact that map is $\mathbb{K}$-linear is clear.

    Pick $s \in S(\Sigma,\mathbb{K})$ such that $\lVert s \rVert_{\infty} = 1$. Suppose that $s(X) = \{\alpha_1,...,\alpha_n\}$ for some $n \in \NN_+$. Then
    $$\left|\int s\,d\mu\right| = \left|\sum_{i=1}^n\alpha_i\cdot \mu(s^{-1}(\alpha_i))\right| \leq \sum_{i=1}^n|\alpha_i|\cdot |\mu(s^{-1}(\alpha_i))| \leq \sum_{i=1}^n|\mu(s^{-1}(\alpha_i))| \leq \lVert \mu \rVert$$
    On the other hand fix an arbitrary finite partition $\PP$ of $X$ onto sets in $\Sigma$. For each $P \in \PP$ pick $\alpha_p \in \mathbb{K}$ such that $|\alpha_p| = 1$ and $\alpha_P\cdot \mu(P) = |\mu(P)|$. Define $s \in S(\Sigma,\mathbb{K})$ by formula
    $$s = \sum_{P\in \PP}\alpha_P\cdot \mathbb{1}_{P}$$
    Then $\lVert s \rVert_{\infty} = 1$ and
    $$\int_Xs\,d\mu = \sum_{P \in \PP}\alpha_P\cdot \mu(P) = \sum_{P\in \PP}|\mu(P)|$$
    This proves that
    $$\lVert \mu \rVert \leq \sup_{s\in S(\Sigma,\mathbb{K})\,\mathrm{s.t.}\,\lVert s \rVert_{\infty}=1}\left|\int_Xs\,d\mu\right|$$
    Hence the norm of the $\mathbb{K}$-linear map in question is equal to $\lVert \mu \rVert$.
\end{proof}

\begin{definition}
    Let $\mu:\Sigma \ra \mathbb{K}$ be a bounded charge. Fact \ref{fact:charge_integral_is_linear_and_continuous_on_space_of_simple} implies that there exists a unique continuous extension of the integral with respect to $\mu$ to all functions in $B(\Sigma, \mathbb{K})$. For every $f \in B(\Sigma,\mathbb{K})$ the value
    $$\int_X f\,d\mu$$
    of this extension for $f$ is \textit{the integral of $f$ with respect to $\mu$}.
\end{definition}

\begin{remark}\label{remark:norm_of_integration_is_total_variation}
    Let $\mu:\Sigma \ra \mathbb{K}$ be a bounded charge. The norm of the integral
    $$B(\Sigma,\mathbb{K})\ni f \mapsto \int_X f\,d\mu\in \mathbb{K}$$
    is $\lVert \mu \rVert$.
\end{remark}

\begin{theorem}[Fichtenholz-Kantorovich]\label{theorem:fichtenholz_kantorovich_for_bounded_functions}
    The map
    $$\mathrm{ba}(\Sigma,\mathbb{K})\ni \mu \mapsto \left(f\mapsto \int_Xf\,d\mu\right) \in B(\Sigma,\mathbb{K})^*$$
    is an isometry of Banach spaces over $\mathbb{K}$.
\end{theorem}
\begin{proof}
    Let $\Lambda:B(\Sigma,\mathbb{K})\ra \mathbb{K}$ be a continuous $\mathbb{K}$-linear map. For every $A \in \Sigma$ we define $\mu(A) = \Lambda(\mathbb{1}_A)$. Then $\mu:\Sigma \ra \mathbb{K}$ is a charge. Moreover, we have
    $$|\mu(A)| = \big|\Lambda(\mathbb{1}_A)\big| \leq \lVert \Lambda \rVert$$
    for every $A \in \Sigma$. Thus $\mu$ is bounded. By definition
    $$\Lambda(s) = \int_Xs\,d\mu$$
    for every $s \in S(\Sigma,\mathbb{K})$. It follows that $\Lambda$ and the integration with respect to $\mu$ coincide on $S(\Sigma,\mathbb{K})$. By Theorem \ref{theorem:space_of_bounded_measurable_functions} we derive that they are equal. Remark \ref{remark:norm_of_integration_is_total_variation} implies that $\lVert \Lambda \rVert = \lVert \mu \rVert$.

    Next note that if $\mu \in \mathrm{ba}(\Sigma,\mathbb{K})$ is such that
    $$\int_Xf\,d\mu = 0$$
    for every $f \in B(\Sigma,\mathbb{K})$, then $\mu$ is the zero charge.
\end{proof}

\section{Space of essentially bounded functions}
\noindent
In this section we denote by $\mathbb{K}$ either $\RR$ or $\CC$ with their usual absolute values. We also fix a space $(X,\Sigma,\nu)$ with measure.

\begin{definition}
    Let $f:X\ra \mathbb{K}$ be a measurable function. Then
    $$\lVert f \rVert_{\mathrm{ess}} = \sup \bigg\{r\in \RR_+\cup \{0\}\,\bigg|\,\nu\left(\big\{x\in X\,\big|\,|f(x)| \geq r\big\}\right) > 0\bigg\}$$
    is \textit{the essential supremum of $f$ with respect to $\nu$}.
\end{definition}

\begin{proposition}\label{proposition:essential_supremum_is_a_seminorm}
    The following assertions hold.
    \begin{enumerate}[label=\emph{\textbf{(\arabic*)}}, leftmargin=*]
        \item If $\alpha \in \mathbb{K}$ and $f:X\ra \mathbb{K}$ is a measurable function, then
              $$\lVert \alpha \cdot f\rVert_{\mathrm{ess}} = |\alpha|\cdot \lVert f\rVert_{\mathrm{ess}}$$
        \item If $f,g:X\ra \mathbb{K}$ are measurable functions, then
              $$\lVert f + g \rVert_{\mathrm{ess}} \leq \lVert f \rVert_{\mathrm{ess}} + \lVert g \rVert_{\mathrm{ess}}$$
    \end{enumerate}
\end{proposition}
\begin{proof}
    Fix $\alpha \in \mathbb{K}\setminus \{0\}$ and a measurable function $f:X\ra \mathbb{K}$. Then
    $$\{x\in X\,\big|\,| (\alpha \cdot f)(x)| \geq r \big\} = \bigg\{x\in X\,\bigg|\,|f(x)| \geq \frac{r}{|\alpha|}\bigg\}$$
    for every $r \in \RR_+ \cup \{0\}$. Hence
    $$\lVert \alpha \cdot f\rVert_{\mathrm{ess}} = \sup \bigg\{r\in \RR_+ \cup \{0\}\,\bigg|\,\nu\left(\big\{x\in X\,\big|\,|(\alpha \cdot f)(x)| \geq r \big\}\right) > 0\bigg\} =$$
    $$= \sup \bigg\{r \in \RR_+\cup \{0\}\,\bigg|\,\nu\left(\bigg\{x\in X\,\bigg|\,|f(x)| \geq \frac{r}{|\alpha|}\bigg\}\right) > 0\bigg\} = $$
    $$= |\alpha| \cdot \sup \bigg\{r \in \RR_+\cup \{0\}\,\bigg|\,\nu\left(\big\{x\in X\,\big|\,|f(x)| \geq r\big\}\right) > 0\bigg\} = |\alpha|\cdot \lVert f\rVert_{\mathrm{ess}}$$
    It follows that
    $$\lVert \alpha \cdot f\rVert_{\mathrm{ess}} = |\alpha|\cdot \lVert f\rVert_{\mathrm{ess}}$$
    for every $\alpha \in \mathbb{K}\setminus \{0\}$. For $\alpha = 0$ this also holds for trivial reasons. Hence \textbf{(1)} is proved.

    Suppose that $f,g:X\ra \mathrm{K}$ are measurable functions. Assume that $r\in \RR_+$ is such that
    $$\lVert f \rVert_{\mathrm{ess}} + \lVert g \rVert_{\mathrm{ess}} < r$$
    We may pick $r_f,r_g\in \RR_+$ such that $r_f + r_g = r$ and $\lVert f \rVert_{\mathrm{ess}} < r_f$ and $\lVert g \rVert_{\mathrm{ess}} < r_g$. Then
    $$\{x\in X\,\big|\,|(f + g)(x)| \geq r \big\} \subseteq \big\{x\in X\,\big|\,|f(x)| + |g(x)| \geq r_f + r_g \big\} \subseteq $$
    $$\subseteq \big\{x\in X\,\big|\,|f(x)|  \geq r_f\big\} \cup \big\{x\in X\,\big|\,|g(x)|  \geq r_g\big\}$$
    Since $\lVert f \rVert_{\mathrm{ess}} < r_f$ and $\lVert g \rVert_{\mathrm{ess}} < r_g$, we deduce that
    $$\nu\left(\big\{x\in X\,\big|\,|f(x)|  \geq r_f\big\}\right) = \nu\left(\big\{x\in X\,\big|\,|g(x)|\geq r_g\big\}\right) = 0$$
    This implies that
    $$\nu\left(\{x\in X\,\big|\,|(f + g)(x)| \geq r\big\}\right) = 0$$
    and thus $\lVert f + g\rVert_{\mathrm{ess}} < r$. This proves that
    $$\lVert f + g \rVert_{\mathrm{ess}} \leq \lVert f \rVert_{\mathrm{ess}} + \lVert g \rVert_{\mathrm{ess}}$$
    if right hand side is finite. Clearly the inequality holds if the right hand side is infinite. This completes the proof of \textbf{(2)}.
\end{proof}

\begin{definition}
    Let $f:X\ra \mathbb{K}$ be a measurable function. If
    $$\lVert f \rVert_{\mathrm{ess}} \in \RR$$
    then $f$ is \textit{essentially bounded with respect to $\nu$} or shortly \textit{$\nu$-essentially bounded}.
\end{definition}
\noindent
According to Proposition \ref{proposition:essential_supremum_is_a_seminorm} the set of all $\mathbb{K}$-valued and $\nu$-essentially bounded functions is a seminormed space over $\mathbb{K}$ with respect to seminorm $\lVert-\rVert_{\mathrm{ess}}$.

\begin{definition}
    The seminormed space of all $\mathbb{K}$-valued and $\nu$-essentially bounded functions is denoted by $L^{\infty}(\nu,\mathbb{K})$ and is called \textit{the Lebesgue space of $\nu$-essentially bounded $\mathbb{K}$-valued functions}.
\end{definition}
\noindent
For every $f \in L^{\infty}(\nu,\mathbb{K})$ define
$$B_f = \big\{x \in X\,\big|\,|f(x)| \leq \lVert f \rVert_{\mathrm{ess}}\big\}$$
and $f_b = \mathbb{1}_{B_f}\cdot f \in B(\Sigma,\mathbb{K})$.

\begin{theorem}\label{theorem:canonical_isometry_of_quotients}
    Consider a set
    $$\cN = \big\{f\in L^{\infty}(\nu,\mathbb{K})\,\big|\,f\mbox{ is $\nu$-almost equal to zero}\big\}$$
    Then the following assertions hold.
    \begin{enumerate}[label=\emph{\textbf{(\arabic*)}}, leftmargin=*]
        \item $\cN\cap B(\Sigma,\mathbb{K})$ is a closed $\mathbb{K}$-linear subspace of $B(\Sigma, \mathbb{K})$.
        \item The canonical inclusion $B(\Sigma, \mathbb{K})\hookrightarrow L^{\infty}(\nu,\mathbb{K})$ induces a bijective isometry
              $$B(\Sigma,\mathbb{K})/\left(\cN\cap B(\Sigma,\mathbb{K})\right) \cong L^{\infty}(\nu,\mathbb{K})/\cN$$
              of normed spaces over $\mathbb{K}$.
        \item The inverse of the isometry described above is given by formula
              $$f + \cN \mapsto f_b + \left(\cN \cap B(\Sigma, \mathbb{K})\right)$$
              for $f \in L^{\infty}(\nu,\mathbb{K})$.
    \end{enumerate}
\end{theorem}
\noindent
We first prove result which is useful in clarifying the main argument.

\begin{lemma}\label{lemma:special_bounded_representative_has_minimal_uniform_norm}
    Let $f$ be a function in $B(\Sigma, \mathbb{K})$ such that $\lVert f \rVert_{\infty} = \lVert f \rVert_{\mathrm{ess}}$. Then
    $$\lVert f \rVert_{\infty} \leq \lVert f + g \rVert_{\infty}$$
    for every $g \in \cN \cap B(\Sigma, \mathbb{K})$.
\end{lemma}
\begin{proof}[Proof of the lemma]
    Note that for every $g \in B(\Sigma, \mathbb{K})$ we have $\lVert g \rVert_{\mathrm{ess}} \leq \lVert g \rVert_{\infty}$. Thus
    $$\lVert f \rVert_{\infty} = \lVert f \rVert_{\mathrm{ess}} = \lVert f + g \rVert_{\mathrm{ess}} \leq \lVert f + g \rVert_{\infty}$$
    for every $g \in \cN \cap B(\Sigma, \mathbb{K})$.
\end{proof}

\begin{proof}[Proof of the theorem]
    Since $B(\Sigma,\mathbb{K})$ is a $\mathbb{K}$-linear subspace of $L^{\infty}(\nu,\mathbb{K})$, we derive that $\cN\cap B(\Sigma, \mathbb{K})$ is a $\mathbb{K}$-linear subspace of $B(\Sigma,\mathbb{K})$. Suppose next that $\{f_n\}_{n \in \NN}$ is a sequence of elements in $\cN\cap B(\Sigma,\mathbb{K})$ which is convergent to $f \in B(\Sigma,\mathbb{K})$ with respect to $\lVert - \rVert_{\infty}$. Since $\{f_n\}_{n\in \NN}$ is pointwise convergent to $f$, we derive that
    $$\big\{x\in X\,\big|\,f(x) \neq 0\big\} \subseteq \bigcup_{n\in \NN}\big\{x\in X\,\big|\,f_n(x) \neq 0\big\}$$
    Clearly the right hand side is a set in $\Sigma$ with measure $\nu$ equal to zero. Thus $f$ is zero $\nu$-almost everywhere. This shows that $f \in \cN$ and the proof of \textbf{(1)} is completed.

    Note that the map
    $$B(\Sigma,\mathbb{K})/\left(\cN\cap B(\Sigma,\mathbb{K})\right) \ra L^{\infty}(\nu,\mathbb{K})/\cN$$
    induced by $B(\Sigma, \mathbb{K})\hookrightarrow L^{\infty}(\nu,\mathbb{K})$ is injective. Fix $f \in L^{\infty}(\nu,\mathbb{K})$. Then $f_b$ and $f$ are equal $\nu$-almost everywhere and $\lVert f \rVert_{\mathrm{ess}} = \lVert f_b\rVert_{\mathrm{ess}} = \lVert f_b \rVert_{\infty}$. This implies that
    $$f_b + \cN = f + \cN$$
    and
    $$\inf_{g \in \cN} \lVert f + g \rVert_{\mathrm{ess}} = \lVert f \rVert_{\mathrm{ess}} = \lVert f_b \rVert_{\mathrm{ess}} = \lVert f_b \rVert_{\infty} =  \inf_{g \in \cN \cap B(\Sigma, \mathbb{K})}\lVert f_b + g\rVert_{\infty}$$
    where the last equality follows from Lemma \ref{lemma:special_bounded_representative_has_minimal_uniform_norm}. This holds for arbitrary $f \in L^{\infty}(\nu,\mathbb{K})$. Therefore, we derive that
    $$B(\Sigma,\mathbb{K})/\left(\cN\cap B(\Sigma,\mathbb{K})\right) \ra L^{\infty}(\nu,\mathbb{K})/\cN$$
    is a surjective isometry of normed spaces over $\mathbb{K}$ and its inverse is given by formula presented in \textbf{(3)}. Hence \textbf{(2)} and \textbf{(3)} hold.
\end{proof}

\begin{corollary}\label{corollary:L_infty_is_complete}
    The space $L^{\infty}(\nu,\mathbb{K})$ is a complete space with respect to norm $\lVert - \rVert_{\mathrm{ess}}$.
\end{corollary}
\begin{proof}
    Theorem \ref{theorem:canonical_isometry_of_quotients} implies that $L^{\infty}(\nu,\mathbb{K})/\cN$ is a Banach space over $\mathbb{K}$. From this we deduce that $L^{\infty}(\nu,\mathbb{K})$ is complete.
\end{proof}
\noindent
Consider now the subspace $\mathrm{ba}_{\ll \nu}(\Sigma, \mathbb{K})$ of $\mathrm{ba}(\Sigma, \mathbb{K})$ which consists of all charges $\mu$ such that $\mu(A) = 0$ for every $A \in \Sigma$ that also satisfy $\nu(A) = 0$. Clearly $\mathrm{ba}_{\ll \nu}(\Sigma, \mathbb{K})$ is a $\mathbb{K}$-linear subspace of $\mathrm{ba}(\Sigma, \mathbb{K})$ and hence a normed subspace of $\mathrm{ba}(\Sigma, \mathbb{K})$ with respect to total variation norm.

\begin{corollary}[Fichtenholz-Kantorovich]\label{corollary:functionals_for_L_infinity}
    The map
    $$\mathrm{ba}_{\ll \nu}(\Sigma, \mathbb{K}) \ni \mu \mapsto \left(f \mapsto \int_X f_b\,d\mu\right) \in L^{\infty}(\nu,\mathbb{K})^*$$
    is a bijective isometry of normed spaces over $\mathbb{K}$.
\end{corollary}
\begin{proof}
    Let $\mu:\Sigma \ra \mathbb{K}$ be a bounded charge. Then
    $$\int_Xf\,d\mu = 0$$
    for every $f \in \cN \cap B(\Sigma,\mathbb{K})$ if and only if $\mu \in \mathrm{ba}_{\ll \nu}(\Sigma, \mathbb{K})$. From this fact and from Theorem \ref{theorem:fichtenholz_kantorovich_for_bounded_functions} we infer that there is a bijective isometry
    $$\mathrm{ba}_{\ll \nu}(\Sigma, \mathbb{K}) \ni \mu \mapsto \left(f \mapsto \int_Xf\,d\mu\right) \in
        \bigg(B(\Sigma,\mathbb{K})/\left(\cN\cap B(\Sigma,\mathbb{K})\right)\bigg)^*$$
    According to Theorem \ref{theorem:canonical_isometry_of_quotients} the formula
    $$L^{\infty}(\nu,\mathbb{K})/\cN \ni f + \cN \mapsto f_b + \cN \in B(\Sigma,\mathbb{K})/\left(\cN\cap B(\Sigma,\mathbb{K})\right)$$
    gives rise to a bijective isometry. Combining these two results with obvious identification of $L^{\infty}(\nu,\mathbb{K})^*$ with $\left(L^{\infty}(\nu,\mathbb{K})/\cN\right)^*$ completes the proof of the corollary.
\end{proof}






\small
\bibliographystyle{apalike}
\bibliography{../zzz}

\end{document}