\input ../pree.tex

\begin{document}

\title{Charges and their integrals}
\date{}
\maketitle


\section{Charges with values in extended real line}

\begin{definition}
    Let $X$ be a set and let $\Sigma$ be an algebra of its subsets. Let $\mu:\Sigma \ra \ol{\RR}$ be a function. Suppose that $\mu(\emptyset) = 0$ and
    $$\mu(A \cup B) = \mu(A) + \mu(B)$$
    for every pair of disjoint sets $A,B \in \Sigma$. Then $\mu$ is \textit{a charge on $\Sigma$}.
\end{definition}

\begin{fact}\label{fact:one_side_infinity_only_for_finitely_additive}
    Let $X$ be a set and let $\Sigma$ be an algebra of its subsets. Suppose that $\mu:\Sigma \ra \ol{\RR}$ is a charge. Then the image of $\mu$ is not a superset of $\{-\infty,+\infty\}$.
\end{fact}
\begin{proof}
    Left for the reader as an exercise.
\end{proof}

\begin{example}\label{example:natural_density_charge}
    For each $n \in \NN_+$ we denote the subset of $\NN$ consisting of consecutive numbers from $0$ to $n-1$ by $[n]$. Let $A \subseteq \NN$ be a subset. We define \textit{the upper density of $A$} and \textit{the lower density of $A$} as the following numbers respectively
    $$\ol{d}(A) = \limsup_{n\ra +\infty}\frac{|A\cap [n]|}{n},\,\underline{d}(A) = \liminf_{n\ra +\infty}\frac{|A\cap [n]|}{n}$$
    If $\ol{d}(A) = \underline{d}(A)$ for some $A \subseteq \NN$, then we denote their value by $d(A)$ and \textit{the density of $A$}. We set
    $$\Sigma = \big\{A\subseteq \NN\,\big|\,d(A)\mbox{ exists }\big\}$$
    Then $\Sigma$ is an algebra of subsets of $\NN$. Moreover, $d$ is a real and nonnegative charge on $\Sigma$.
\end{example}

\begin{example}\label{example:ultrafilter_charge}
    For the notion of ultrafilter we refer to \cite{Filters_in_topology}. Let $X$ be a set and let $\cF$ be an ultrafilter of subsets of $X$  Consider a function given by formula
    $$\mu(A) = \begin{cases}
            1 & \mbox{ if }A \in \cF \\
            0 & \mbox{ otherwise }   \\
        \end{cases}
    $$
    for every $A \subseteq X$. Then $\mu$ is a $\{0,1\}$-valued charge on the algebra of all subsets of $X$.
\end{example}

\begin{example}\label{example:charge_from_series}
    Let $\{a_n\}_{n\in \NN}$ be a sequence of real numbers such that the series
    $$\sum_{n\in \NN}a_n$$
    is convergent. Let $\Sigma$ be an algebra of all finite and cofinite subsets in $\NN$. We define
    $$\mu(A) = \sum_{n \in A} a_n$$
    for every $A \in \Sigma$. Then $\mu:\Sigma \ra \ol{\RR}$ is a charge.
\end{example}

\begin{definition}
    Let $X$ be a set and let $\Sigma$ be an algebra of its subsets. Let $\mu$ be a charge on $\Sigma$. If $\mu(A) \in \RR$ for every $A \in \Sigma$, then $\mu$ is \textit{a real charge on $\Sigma$}.
\end{definition}

\begin{definition}
    Let $X$ be a set and let $\Sigma$ be an algebra of its subsets. Let $\mu$ be a charge on $\Sigma$. If $\mu(A) \in [0,+\infty]$ for every $A \in \Sigma$, then $\mu$ is \textit{a nonnegative charge on $\Sigma$}.
\end{definition}

\begin{definition}
    Let $X$ be a set and let $\Sigma$ be an algebra of its subsets. Let $\mu$ be a charge on $\Sigma$. If there exists $\kappa \in \RR$ such that $\mu(A) \geq
        \kappa$ for every $A \in \Sigma$, then $\mu$ is \textit{bounded from below}.
\end{definition}

\begin{definition}
    Let $X$ be a set and let $\Sigma$ be an algebra of its subsets. Let $\mu$ be a charge on $\Sigma$. If there exists $\kappa \in \RR$ such that $\mu(A) \leq
        \kappa$ for every $A \in \Sigma$, then $\mu$ is \textit{bounded from above}.
\end{definition}

\begin{definition}
    Let $X$ be a set and let $\Sigma$ be an algebra of its subsets. Let $\mu$ be a charge on $\Sigma$. If $\mu$ is bounded from below and from above, then $\mu$ is \textit{bounded}.
\end{definition}

\begin{example}\label{example:examples_of_bounded_and_nonnegative_charges}
    Charges defined in Examples \ref{example:natural_density_charge} and \ref{example:ultrafilter_charge} are real, bounded and nonnegative.
\end{example}

\begin{example}\label{example:unbounded_charge_from_not_absolutely_convergent_series}
    Consider a sequence $\{a_n\}_{n\in \NN}$ such that the series
    $$\sum_{n\in \NN}a_n$$
    is convergent, but not absolutely convergent. Then the charge defined by $\{a_n\}_{n\in \NN}$ as in Example \ref{example:charge_from_series} is real but not bounded from below or above.
\end{example}
\noindent
Now we prove important Jordan decomposition for charges. Our approach closely follows Stanis{\l}aw Saks \cite{saks1937theory}.

\begin{theorem}[Jordan decomposition]\label{theorem:Jordan_decomposition}
    Let $X$ be a set and let $\Sigma$ be an algebra of its subsets. Let $\mu:\Sigma \ra \ol{\RR}$ be a charge. For every $A \in \Sigma$ set
    $$\mu_+(A) = \sup\big\{\mu(B)\,\big|\,B\in \Sigma\mbox{ and }B\subseteq A\big\},\,\mu_-(A) = \sup\big\{-\mu(B)\,\big|\,B\in \Sigma\mbox{ and }B\subseteq A\big\}$$
    Then the following assertions hold.
    \begin{enumerate}[label=\emph{\textbf{(\arabic*)}}, leftmargin=*]
        \item $\mu_+$ and $\mu_-$ are nonnegative charges on $\Sigma$.
        \item For every $A \in \Sigma$ set
              $$|\mu|(A) = \sup \bigg\{\sum_{P\in \PP}|\mu(P)|\,\bigg|\,\PP\mbox{ is a finite partition of }A\mbox{ onto sets in }\Sigma\bigg\}$$
              Then $|\mu|$ is a nonnegative charge on $\Sigma$ and
              $$|\mu|(A) = \mu_+(A) + \mu_-(A)$$
              for every $A \in \Sigma$.
        \item If $\mu$ is bounded from below, then $\mu_-$ is a bounded charge and
              $$\mu(A) = \mu_+(A) - \mu_-(A)$$
              for every $A \in \Sigma$.
        \item If $\mu$ is bounded from above, then $\mu_+$ is a bounded charge and
              $$\mu(A) = \mu_+(A) - \mu_-(A)$$
              for every $A \in \Sigma$.
    \end{enumerate}
\end{theorem}
\begin{proof}
    We left for the reader the proof of \textbf{(1)}.

    Fix $A \in \Sigma$. Let $\PP$ be a finite partition of $A$ onto a sets in $\Sigma$. Consider families
    $$\PP_+ = \big\{P\in \PP\,\big|\,\mu(P) > 0\big\},\,\PP_- = \big\{P\in \PP\,\big|\,\mu(P) \leq 0\big\}$$
    Clearly $\PP = \PP_+ \cup \PP_-$ and $\PP_+ \cap \PP_- = \emptyset$. Moreover, we have
    $$\sum_{P\in \PP}|\mu(P)| = \sum_{P \in \PP_+}\mu(P) - \sum_{P \in \PP_-}\mu(P) = \mu\left(\bigcup_{P\in \PP_+}P\right) - \mu\left(\bigcup_{P\in \PP_-}P\right) \leq \mu_+(A) + \mu_-(A)$$
    and thus $|\mu|(A) \leq \mu_+(A) + \mu_-(A)$ for every $A \in \Sigma$.

    Again fix arbitrary $A \in \Sigma$. There exists a sequence $\{B_n\}_{n\in \NN}$ of subsets of $A$ contained in $\Sigma$ such that $\mu(B_n) \geq 0$ for every $n\in \NN$ and $\{\mu(B_n)\}_{n\in \NN}$ is convergent to $\mu_+(A)$. Similarly there exists a sequence $\{C_n\}_{n\in \NN}$ of subsets of $A$ contained in $\Sigma$ such that $\mu(C_n) < 0$ for every $n \in \NN$ and $\{\mu(C_n)\}_{n\in \NN}$ is convergent to $-\mu_-(A)$. For each $n\in \NN$ we define
    $$\cS_n = \big\{B_n\setminus C_n,B_n \cap C_n,C_n \setminus B_n, A \setminus (B_n\cup C_n) \big\}$$
    and
    $$\tilde{B}_n = \bigcup\big\{S\in \cS_n\,\big|\,\mu(S) > 0\big\},\,\tilde{C}_n = \bigcup\big\{S\in \cS_n\,\big|\,\mu(S) \leq 0\big\}$$
    Then $A = \tilde{B}_n\cup \tilde{C}_n,\tilde{B}_n\cap \tilde{C}_n = \emptyset,\mu(\tilde{B}_n) \geq \mu(B_n),\mu(\tilde{C}_n) \leq \mu(C_n)$ for every $n \in \NN$. It follows from inequalities that $\{\mu(\tilde{B}_n)\}_{n\in \NN}$ is convergent to $\mu_+(A)$ and $\{\mu(\tilde{C}_n)\}_{n\in \NN}$ is convergent to $-\mu_-(A)$.

    Now we have
    $$\mu_+(A) + \mu_-(A) = \lim_{n\ra +\infty}\left(\mu(\tilde{B}_n) - \mu(\tilde{C}_n)\right) = \lim_{n\ra +\infty}\left(|\mu(\tilde{B}_n)| + |\mu(\tilde{C}_n)|\right) \leq |\mu|(A)$$
    Hence $\mu_+(A) + \mu_-(A) \leq |\mu|(A)$ for every $A \in \Sigma$. This completes the proof of \textbf{(2)}.

    Now in order to prove \textbf{(3)} assume that $\mu$ is bounded from below. Then clearly $\mu_-$ is bounded. Fix $A \in \Sigma$. As above there exist sequences $\{\tilde{B}_n\}_{n\in \NN}$ and $\{\tilde{C}_n\}_{n\in \NN}$ of subsets of $A$ contained in $\Sigma$ such that $A = \tilde{B}_n\cup \tilde{C}_n,\tilde{B}_n\cap \tilde{C}_n = \emptyset$ and
    $$\mu_+(A) = \lim_{n\ra +\infty} \mu(\tilde{B}_n),\,\mu_-(A) = \lim_{n\ra +\infty}\mu(\tilde{C}_n)$$
    Using the fact that $\mu_-(A) \in \RR$ we derive
    $$\mu(A) = \lim_{n\ra +\infty}\left(\mu(\tilde{B}_n) + \mu(\tilde{C}_n)\right) = \mu_+(A) - \mu_-(A)$$
    Since $A \in \Sigma$ is arbitrary, we deduced \textbf{(3)}.

    The proof of \textbf{(4)} is analogical to the proof of \textbf{(3)} and is omited.
\end{proof}

\begin{example}\label{example:charge_without_Jordan_decomposition}
    If $\mu$ is the charge from Example \ref{example:unbounded_charge_from_not_absolutely_convergent_series}, then for every cofinite $A \subseteq \NN$ we have $\mu_+(A) = +\infty$ and $\mu_-(A) = +\infty$. Thus $\mu_+ - \mu_-$ is undefined.
\end{example}


\section{$\sigma$-additive charges and signed measures}

\begin{definition}
    Let $X$ be a set and let $\Sigma$ be an algebra of its subsets. Let $\mu:\Sigma \ra \ol{\RR}$ be a charge. Suppose that for every sequence $\{A_n\}_{n\in \NN}$ of pairwise disjoint sets in $\Sigma$ such that
    $$\bigcup_{n\in \NN}A_n \in \Sigma$$
    the equality
    $$\mu\left(\bigcup_{n\in \NN}A_n\right) = \sum_{n \in \NN}\mu(A_n)$$
    holds. Then $\mu$ is \textit{a $\sigma$-additive charge on $\Sigma$}.
\end{definition}
\noindent
For the sake of giving a counterexample we first prove the following result.

\begin{proposition}\label{proposition:natural_density_charges_are_not_sigma_additive}
    Let $\Sigma$ be an algebra of subsets of $\NN$ which contains each finite subset of $\NN$ and a family $\{d\cdot \NN\}_{d \in \NN_+}$. Suppose that $\mu$ is a charge on $\Sigma$ such that
    $$\mu(d\cdot \NN) = \frac{1}{d}$$
    for every $d \in \NN_+$. Then $\mu$ is not $\sigma$-additive.
\end{proposition}
\begin{proof}
    Suppose that $\mu$ is a charge on $\Sigma$ such that
    $$\mu(d\cdot \NN) = \frac{1}{d}$$
    for every $d \in \NN_+$. Assume that $d_1,...,d_s \in \NN_+$ are pairwise coprime. Then inclusion-exclusion principle implies that
    $$\mu\left(\bigcup_{k=1}^sd_k\cdot \NN\right) = 1 - \prod_{i=1}^s\left(1 - \frac{1}{d_i}\right)$$

    Let $\PP$ be the set of all primes. For each $n \in \NN_+$ let $\nu_p(n) \in \NN$ be the exponent of $p \in \PP$ in prime factorization of $n$. Fix now a sequence $\alpha = \{\alpha_p\}_{p\in \PP}$ of elements in $\NN_+$ such that $\alpha_p = 1$ for all but finitely many $p \in \PP$. Consider the set
    $$\Gamma_{\alpha} = \big\{n\in \NN_+\,\big|\,\nu_p(n) \geq \alpha_p\mbox{ for some }p\in \PP\big\}$$
    Clearly $\Gamma_{\alpha}$ is cofinite and
    $$\Gamma_{\alpha} = \bigcup_{p\in \PP}p^{\alpha_p}\cdot \NN$$
    If $\mu$ is $\sigma$-additive, then
    $$\mu(\Gamma_{\alpha}) = \lim_{N \ra +\infty}\mu\left(\bigcup_{p < N}p^{\alpha_p}\cdot \NN\right) = 1 - \lim_{N\ra +\infty}\prod_{p < N}\left(1 - \frac{1}{p^{\alpha_p}}\right) = 1$$
    Now for fixed $n \in \NN \cap (1,+\infty)$ we pick $\alpha = \{\alpha_p\}_{p\in \PP}$ and $\beta = \{\beta_p\}_{p\in \PP}$ such that
    $$\alpha_p = \begin{cases}
            \nu_p(n) & \mbox{ if }\nu_p(n) > 0 \\
            1        & \mbox{ otherwise }
        \end{cases}$$
    for each $p \in \PP$ and
    $$\beta_p = \begin{cases}
            \nu_p(n) + 1 & \mbox{ if }\nu_p(n) > 0 \\
            1            & \mbox{ otherwise }
        \end{cases}$$
    Then $\mu(\Gamma_{\alpha}) = \mu(\Gamma_{\beta}) = 1$ and hence $\mu(\{n\}) = \mu\left(\Gamma_{\alpha} \setminus \Gamma_{\beta}\right) = 0$. This holds for all $n \in \NN \cap (1,+\infty)$. Moreover, by $\sigma$-additivity it follows that
    $$\mu(\{0\}) = \mu\left(\bigcap_{n \in \NN}2^n \cdot \NN\right) = \lim_{n\ra +\infty}\mu(2^n\cdot \NN) = \lim_{n\ra +\infty}\frac{1}{2^n} = 0$$
    and hence
    $$\mu(2\cdot \NN) = \sum_{n\in \NN}\mu(\{2\cdot n\}) = 0$$
    This contradicts the fact that $\mu(2\cdot \NN) \neq 0$.
\end{proof}

\begin{example}\label{example:natural_density_is_not_sigma_additive}
    Let $d$ be the density charge defined in Example \ref{example:natural_density_charge}. Then Proposition \ref{proposition:natural_density_charges_are_not_sigma_additive} implies that $d$ is not $\sigma$-additive.
\end{example}

\begin{proposition}\label{proposition:sigma_additive_have_positive_and_negative_parts_which_are_sigma_additive}
    Let $X$ be a set and let $\Sigma$ be an algebra of its subsets. Let $\mu:\Sigma \ra \ol{\RR}$ be a $\sigma$-additive charge. Then $\mu_+,\mu_-$ and $|\mu|$ are $\sigma$-additive charges.
\end{proposition}
\begin{proof}
    Suppose that $\{A_n\}_{n\in \NN}$ is a sequence of pairwise disjoint subsets in $\Sigma$ such that
    $$A = \bigcup_{n\in \NN}A_n \in \Sigma$$
    Let $B \in \Sigma$ be a subset of $A$. Since $\mu$ is $\sigma$-additive, we derive
    $$\mu(B) = \sum_{n\in \NN}\mu(A_n\cap B) \leq \sum_{n\in \NN}\mu_+(A_n)$$
    Thus $\mu_+(A) \leq  \sum_{n\in \NN}\mu_+(A_n)$. On the other hand pick a family $\{B_n\}_{n\in \NN}$ of sets in $\Sigma$ such that $B_n\subseteq A_n$ and $\mu(B_n) \geq 0$ for each $n\in \NN$. Then
    $$\sum_{n \in \NN}\mu(B_n) = \lim_{N\ra +\infty}\sum_{n\leq N}\mu(B_n) = \lim_{N \ra +\infty}\mu\left(\bigcup_{n\leq N} B_n\right) \leq \mu^+(A)$$
    and hence $\sum_{n \in \NN}\mu_+(A_n) \leq \mu_+(A)$. This proves that $\mu_+$ is $\sigma$-additive.

    Since $(-\mu)_+ = \mu_-$ and $-\mu$ is $\sigma$-additive, we derive that $\mu_-$ is $\sigma$-additive by the case considered above.

    According to Theorem \ref{theorem:Jordan_decomposition} we have $|\mu| = \mu_+ + \mu_-$. Hence also $|\mu|$ is $\sigma$-additive.
\end{proof}

\begin{definition}
    Let $X$ be a set and let $\Sigma$ be a $\sigma$-algebra of its subsets. Let $\mu:\Sigma \ra \ol{\RR}$ be a $\sigma$-additive charge. Then $\mu$ is \textit{a signed measure on $\Sigma$}.
\end{definition}

\begin{example}\label{example:each_measure_is_signed_measure}
    Measures are defined in \cite{Integration}. Note that each measure is a nonnegative, signed measure.
\end{example}
\noindent
The following notion plays central role in studying structure of signed measures.

\begin{definition}
    Let $X$ be a set and let $\Sigma$ be an algebra of its subsets. Let $\mu:\Sigma \ra \ol{\RR}$ be a charge. \textit{A positive set for $\mu$} is a set $P \in \Sigma$ such that
    $$\mu(A\cap P) \geq 0,\,\mu(A\setminus P)\leq 0$$
    for every $A \in \Sigma$.
\end{definition}

\begin{example}\label{example:not_every_charge_has_positive_set}
    The charge in Example \ref{example:unbounded_charge_from_not_absolutely_convergent_series} does not have positive sets.
\end{example}
\noindent
The following important result shows the existence of positive sets for signed measures.

\begin{theorem}[Hahn]\label{theorem:Hahn_decomposition}
    Let $X$ be a set and let $\Sigma$ be a $\sigma$-algebra of its subsets. Let $\mu:\Sigma \ra \ol{\RR}$ be a signed measure. Then there exists a positive set for $\mu$.
\end{theorem}
\noindent
The proof proceeds by constructing approximations for a positive set.

\begin{lemma}\label{lemma:approximate_positive_set}
    Let $X$ be a set and let $\Sigma$ be a $\sigma$-algebra of its subsets. Let $\mu:\Sigma \ra \ol{\RR}$ be a signed measure. Suppose that $\mu(A) \geq 0$ for some $A \in \Sigma$. Then for each $\epsilon > 0$ there exists a subset $Q_{\epsilon}$ of $A$ such that the following assertions hold.
    \begin{enumerate}[label=\emph{\textbf{(\arabic*)}}, leftmargin=3.0em]
        \item $Q_{\epsilon} \in \Sigma$ and $\mu(Q_{\epsilon}) \geq \mu(A)$.
        \item If $B \in \Sigma$ and $B \subseteq Q_{\epsilon}$, then $\mu(B) \geq -\epsilon$.
    \end{enumerate}
\end{lemma}
\begin{proof}[Proof of the lemma]
    Let $\fF$ be a family of all sets in $\Sigma$ contained in $A$. For any two sets $F_1,F_2\in \fF$ we define
    $$F_1 \sqsubseteq_{\epsilon}F_2$$
    if and only if $F_2 \subseteq F_1$ and $\mu(F_1 \setminus F_2) < -\epsilon$. Clearly $\sqsubseteq_{\epsilon}$ is transitive and antireflexive. Suppose that $\{F_n\}_{n\in \NN}$ is a sequence of sets in $\fF$ which is a chain with respect to $\sqsubseteq_{\epsilon}$. Then
    $$\bigcup_{n\in \NN}\left(F_n\setminus F_{n+1}\right) \in \fF$$
    and
    $$\mu\left(\bigcup_{n\in \NN}\left(F_n\setminus F_{n+1}\right)\right) = \sum_{n\in \NN}\mu\left(F_n\setminus F_{n+1}\right) < -\sum_{n\in \NN}\epsilon$$
    This contradicts the fact that $\mu(A) \geq 0$. Hence there are no infinite chains in $\fF$ with respect to $\sqsubseteq_{\epsilon}$. Thus there exists $Q_{\epsilon} \in \fF$ which is maximal with respect to $\sqsubseteq_{\epsilon}$ and is contained in a chain with respect to $\sqsubseteq_{\epsilon}$ which starts with $A$. Then $Q_{\epsilon}$ satisfies assertions.
\end{proof}

\begin{lemma}\label{lemma:positive_set}
    Let $X$ be a set and let $\Sigma$ be a $\sigma$-algebra of its subsets. Let $\mu:\Sigma \ra \ol{\RR}$ be a signed measure. Suppose that $\mu(A) > 0$ for some $A \in \Sigma$. Then there exists a subset $Q$ of $A$ such that the following assertions hold.
    \begin{enumerate}[label=\emph{\textbf{(\arabic*)}}, leftmargin=3.0em]
        \item $Q \in \Sigma$ and $\mu(Q) \geq \mu(A)$.
        \item If $B \in \Sigma$ and $B \subseteq Q$, then $\mu(B) \geq 0$.
    \end{enumerate}
\end{lemma}
\begin{proof}[Proof of the lemma]
    We define a sequence $\{Q_n\}_{n\in \NN}$ of sets in $\Sigma$ which are contained in $A$. We set $Q_0 = A$ and if $Q_n$ is defined for some $n \in \NN$, then we pick $Q_{n+1} \subseteq Q_n$ such that $\mu(Q_n) \leq \mu(Q_{n+1})$ and
    $$\mu\left(B\right) \geq - \frac{1}{n+1}$$
    for every $B \in \Sigma$ and $B\subseteq Q_{n+1}$. This construction is possible due to Lemma \ref{lemma:approximate_positive_set}. Define
    $$Q = \bigcap_{n\in \NN}Q_n$$
    Then $Q \in \Sigma$ and $Q\subseteq A$. Since $\{\mu(Q_n)\}_{n\in \NN}$ is nondecreasing and $Q_0 = A$, we derive
    $$\mu(A) \leq \lim_{n\ra +\infty}\mu(Q_n) = \mu(Q) $$
    Now if $B \in \Sigma$ and $B \subseteq Q$, then
    $$\mu(B) \geq -\frac{1}{n + 1}$$
    for every $n \in \NN$. Thus $\mu(B) \geq 0$. This proves that $Q$ satisfies assertions.
\end{proof}

\begin{proof}[Proof of the theorem]
    By Fact \ref{fact:one_side_infinity_only_for_finitely_additive} and changing $\mu$ to $-\mu$ if necessary, we may assume that there is no set $A \in \Sigma$ such that $\mu(A) = +\infty$. Consider the family
    $$\cP = \big\{Q \in \Sigma\,\big|\,\mu(B)\geq 0\mbox{ for each }B\subseteq Q\mbox{ such that }B \in \Sigma\big\}$$
    Denote by $\alpha$ the least upper bound of $\mu(Q)$ for $Q \in \cP$. There exists a sequence $\{Q_n\}_{n\in \NN}$ such that
    $$\lim_{n\ra +\infty}\mu(Q_n) = \alpha$$
    Define
    $$P = \bigcup_{n\in \NN}Q_n$$
    Then $P \in \cP$ and $\mu(P) = \alpha$. Since by assumption $\mu(P)$ is finite, we derive that $\alpha \in \RR$. Assume that there exists a set $A \in \Sigma$ such that $\mu(A) > 0$ and $A \subseteq X\setminus P$. Then by Lemma \ref{lemma:positive_set} there exists $Q \in \cP$ such that $Q \subseteq A$ and $\mu(A) \leq \mu(Q)$. Then $Q\cup P \in \cP$ and
    $$\alpha = \mu(P) < \mu(P) + \mu(Q) = \mu(Q\cup P) \leq \alpha$$
    This is a contradiction. Hence $P$ is a positive set for $\mu$.
\end{proof}

\begin{corollary}\label{corollary:signed_measures_are_one_sided_bounded}
    Let $X$ be a set and let $\Sigma$ be a $\sigma$-algebra of its subsets. Let $\mu:\Sigma \ra \ol{\RR}$ be a signed measure. Then $\mu$ is either bounded from below or from above.
\end{corollary}
\begin{proof}
    Indeed, let $P \in \Sigma$ be a positive set of $\mu$. Then $\mu_+(X) = \mu(P),\,\mu_-(X) = \mu(X\setminus P)$ and both cannot be infinite by Fact \ref{fact:one_side_infinity_only_for_finitely_additive}.
\end{proof}

\section{Complex charges and spaces of bounded charges}

\begin{definition}
    Let $X$ be a set and let $\Sigma$ be an algebra of its subsets. Let $\mu:\Sigma \ra \CC$ be a function. Suppose that $\mu(\emptyset) = 0$ and
    $$\mu(A \cup B) = \mu(A) + \mu(B)$$
    for every pair of disjoint sets $A,B \in \Sigma$. Then $\mu$ is \textit{a complex charge on $\Sigma$}.
\end{definition}

\begin{remark}\label{remark:each_real_charge_is_complex}
    Let $X$ be a set and let $\Sigma$ be an algebra of its subsets. Each real charge on $\Sigma$ is a complex on $\Sigma$.
\end{remark}

\begin{definition}
    Let $X$ be a set and let $\Sigma$ be an algebra of its subsets. Let $\mu:\Sigma \ra \CC$ be a charge. Suppose that for every sequence $\{A_n\}_{n\in \NN}$ of pairwise disjoint sets in $\Sigma$ such that
    $$\bigcup_{n\in \NN}A_n \in \Sigma$$
    the equality
    $$\mu\left(\bigcup_{n\in \NN}A_n\right) = \sum_{n \in \NN}\mu(A_n)$$
    holds. Then $\mu$ is \textit{a $\sigma$-additive complex charge on $\Sigma$}.
\end{definition}

\begin{definition}
    Let $X$ be a set and let $\Sigma$ be a $\sigma$-algebra of its subsets. Let $\mu:\Sigma \ra \CC$ be a charge. If $\mu$ is $\sigma$-additive, then $\mu$ is \textit{a complex measure on $\Sigma$}.
\end{definition}

\begin{fact}\label{fact:total_variation_is_a_charge}
    Let $X$ be a set and let $\Sigma$ be an algebra of its subsets. Let $\mu:\Sigma \ra \CC$ be a charge. For every $A \in \Sigma$ we define
    $$|\mu|(A) = \sup \bigg\{\sum_{P\in \PP}|\mu(P)|\,\bigg|\,\PP\mbox{ is a finite partition of }A\mbox{ onto sets in }\Sigma\bigg\}$$
    Then $|\mu|$ is a nonnegative charge on $\Sigma$.

    Moreover, if $\mu$ is $\sigma$-additive, then also $|\mu|$ is $\sigma$-additive.
\end{fact}
\begin{proof}
    The fact that $|\mu|$ is a charge is left for the reader as an exercise.

    Assume now that $\mu$ is $\sigma$-additive. Suppose that $\{A_n\}_{n\in \NN}$ is a sequence of pairwise disjoint subsets in $\Sigma$ such that
    $$A = \bigcup_{n\in \NN}A_n \in \Sigma$$
    Pick a finite partition $\PP$ of $A$ onto sets in $\Sigma$. Since $\mu$ is $\sigma$-additive, we derive that
    $$\sum_{P\in \PP}|\mu(P)| = \sum_{P\in \PP}\bigg|\sum_{n\in \NN}\mu(A_n \cap P)\bigg| \leq$$
    $$\leq \sum_{P\in \PP}\sum_{n\in \NN}|\mu(A_n \cap P)| = \sum_{n\in \NN}\sum_{P\in \PP}|\mu(A_n \cap P)| \leq \sum_{n \in \NN}|\mu|(A_n)$$
    This proves that $|\mu|(A) \leq \sum_{n \in \NN}|\mu|(A_n)$. On the other hand for each $n \in \NN$ pick a finite partition $\PP_n$ of $A_n$ onto a sets in $\Sigma$. Then
    $$\sum_{n \in \NN}\sum_{P \in \PP_n}|\mu(P)| = \lim_{N\ra +\infty}\sum_{n\leq N}\sum_{P \in \PP_n}|\mu(P)| \leq $$
    $$\leq \limsup_{N \ra +\infty}\left(\sum_{n\leq N}\sum_{P \in \PP_n}|\mu(P)| + \bigg|\mu\left(A\setminus \bigcup_{n\leq N}A_n\right)\bigg|\right) \leq |\mu|(A)$$
    Hence $\sum_{n \in \NN}|\mu|(A_n) \leq |\mu|(A)$. This completes the proof of $\sigma$-additivity of $\mu$.
\end{proof}

\begin{theorem}\label{theorem:charge_is_bounded_if_variation_is_finite}
    Let $X$ be a set and let $\Sigma$ be an algebra of its subsets. Let $\mu:\Sigma \ra \CC$ be a charge. Then the following assertions are equivalent.
    \begin{enumerate}[label=\emph{\textbf{(\roman*)}}, leftmargin=*]
        \item There exists $\kappa \in \RR_+$ such that
              $$|\mu(A)|\leq \kappa$$
              for every $A \in \Sigma$.
        \item $|\mu|$ is a bounded charge.
    \end{enumerate}
\end{theorem}
\begin{proof}
    Assume that there exists $\kappa \in \RR_+$ such that $|\mu(A)| \leq \kappa$ for every $A \in \Sigma$. For each $A \in \Sigma$ write
    $$\mu(A) = \mu_r(A) + \sqrt{-1}\cdot \mu_i(A)$$
    where $\mu_r(A),\mu_i(A) \in \RR$. Then $\mu_r,\mu_i:\Sigma \ra \RR$ are real charges and $|\mu_r(A)|,|\mu_i(A)| \leq \kappa$ for every $A \in \Sigma$. Part \textbf{(2)} of Theorem \ref{theorem:Jordan_decomposition} implies that $|\mu_r|,|\mu_i|$ are bounded. Note that
    $$|\mu|(A) \leq |\mu_r|(A) + |\mu_i|(A)$$
    for every $A \in \Sigma$. Hence $|\mu|$ is bounded. This proves that $\textbf{(i)}\Rightarrow \textbf{(ii)}$.

    Suppose now that $|\mu|$ is a bounded charge. Then there exists $\kappa \in \RR_+$ such that $|\mu|(A) \leq \kappa$ for every $A \in \Sigma$. Since $|\mu|(A)\leq |\mu|(A)$ for every $A \in \Sigma$, we deduce that $|\mu(A)|\leq \kappa$ for each $A \in \Sigma$. This completes the proof of $\textbf{(ii)}\Rightarrow \textbf{(i)}$.
\end{proof}

\begin{definition}
    Let $X$ be a set and let $\Sigma$ be an algebra of its subsets. Let $\mu:\Sigma \ra \CC$ be a charge. If $|\mu|$ is bounded, then $\mu$ is \textit{a bounded complex charge on $\Sigma$}.
\end{definition}

\begin{definition}
    Let $X$ be a set and let $\Sigma$ be an algebra of its subsets. Let $\mu:\Sigma \ra \CC$ be a charge. We define
    $$\lVert \mu \rVert = |\mu|(X)$$
    Then $\lVert \mu \rVert$ is \textit{the total variation of $\mu$}.
\end{definition}

\begin{theorem}\label{theorem:space_of_bounded_complex_charges}
    Let $X$ be a set and let $\Sigma$ be an algebra of its subsets. Consider the set
    $$\mathrm{ba}(\Sigma, \CC) = \big\{\mu:\Sigma \ra \CC\,\big|\,\mu\mbox{ is a bounded charge on }\Sigma\big\}$$
    Then the following assertions hold.
    \begin{enumerate}[label=\emph{\textbf{(\arabic*)}}, leftmargin=*]
        \item $\mathrm{ba}(\Sigma, \CC)$ is a $\CC$-linear space with respect to canonical operations of addition of charges and multiplication by complex scalars.
        \item Then
              $$\mathrm{ba}(\Sigma, \CC) \ni \mu \mapsto \lVert \mu\rVert \in [0,+\infty)$$
              is a norm.
        \item Let $\{\mu_n\}_{n\in \NN}$ be a Cauchy sequence with respect to $\lVert-\rVert$. Then $\{\mu_n\}_{n\in \NN}$ is convergent to some $\mu \in \mathrm{ba}(\Sigma, \CC)$. Moreover, if $\{\mu_n\}_{n\in \NN}$ are $\sigma$-additive, then $\mu$ is $\sigma$-additive.
        \item Let $\mathrm{ba}(\Sigma,\RR)$ be an $\RR$-linear subspace of $\mathrm{ba}(\Sigma, \CC)$ that consists of real bounded charges. Then $\mathrm{ba}(\Sigma,\RR)$ is closed with respect to $\lVert-\rVert$.
    \end{enumerate}
\end{theorem}
\begin{proof}
    Proofs of \textbf{(1)} and \textbf{(2)} are left for the reader.

    Let $\{\mu_n\}_{n\in \NN}$ be a Cauchy sequence with respect to $\lVert-\rVert$. For every $A \in \Sigma$ and each $n,m\in \NN$ we have
    $$|\mu_n(A) - \mu_m(A)|\leq \lVert \mu_n - \mu_m\rVert$$
    Since $\CC$ with the usual absolute value is complete, we derive that there exists $\mu(A) \in \CC$ such that $\{\mu_n(A)\}_{n\in \NN}$ converges to $\mu(A)$. Now pick at most countable family $\cF$ of pairwise disjoint sets in $\Sigma$ such that
    $$\bigcup_{F \in \cF}F\in \Sigma$$
    Suppose also that
    $$\mu_n(A) = \sum_{F\in \cF}\mu_n(F)$$
    for every $n \in \NN$. We define a measure $u$ on the power set of $\cF$ by formula
    $$u(Z) = |Z|$$
    for every $Z \subseteq \cF$. Let $L^1(u,\CC)$ is a space of complex valued functions defined on $\cF$ which are integrable with respect to $u$. In particular, $L^1(u,\CC)$ is a Banach space over $\CC$ with norm
    $$\lVert f \rVert_1 = \int_{\cF} f\,du = \sum_{F\in \cF}|f(F)|$$
    and integral
    $$\int_{\cF}f\,du = \sum_{F\in \cF}f(F)$$
    For the details we refer to \cite{Integration}. Since $\lVert \mu_n\rVert$ is finite for each $n \in \NN$ by Theorem \ref{theorem:charge_is_bounded_if_variation_is_finite}, we derive that the function $\cF \ni F \mapsto \mu_n(F)\in \CC$, which we denote by $f_n$, is an element of $L^1(u,\CC)$ for every $n \in \NN$. Moreover, the distance of $f_n$ and $f_m$ in $L^1(u,\CC)$ is bounded by $\lVert \mu_n - \mu_m\rVert$ for all pairs $n,m \in \NN$. Hence the sequence $\{f_n\}_{n \in \NN}$ is convergent in $L^1(u,\CC)$. It is also pointwise convergent to a function $\cF \ni F \mapsto \mu(F)\in \CC$, which we denote by $f$. By general results in \cite{Integration} we deduce that $f$ is a limit of $\{f_n\}_{n\in \NN}$ in $L^1(\mu,\CC)$ and from considerations above we have inequality
    $$\lVert f - f_n\rVert_1 = \lim_{m\ra +\infty}\lVert f_m - f_n\rVert_1 \leq \limsup_{m \ra +\infty}\lVert \mu_n - \mu_m\rVert$$
    Let us note some consequences of this fact.
    \begin{itemize}
        \item From the convergence of integrals with respect to $u$ we deduce
              $$\mu\left(\bigcup_{F\in \cF}F\right) = \lim_{n\ra +\infty}\mu_n(\bigcup_{F\in \cF}F) = \lim_{n\ra +\infty}\sum_{F\in \cF}\mu_n(F) = \sum_{F\in \cF}\mu(F)$$
        \item The convergence in $\lVert-\rVert_1$ implies that
              $$\sum_{F \in \cF}|\mu(F)| = \lim_{n\ra +\infty}\sum_{F\in \cF}|\mu_n(F)| \leq \sup_{n\in \NN}\lVert \mu_n\rVert$$
        \item From the inequality above we derive that
              $$\sum_{F\in \cF}|(\mu - \mu_n)(F)| = \lVert f - f_n\rVert_1 \leq \limsup_{m\ra +\infty}\lVert \mu_m - \mu_n\rVert$$
    \end{itemize}
    Note that these assertions hold for every family $\cF$ which satisfies the conditions specified above. Hence from the first assertion it follows that $\mu$ is a charge and if $\{\mu_n\}_{n\in \NN}$ are $\sigma$-additive, then also $\mu$ is $\sigma$-additive. Next the second statement shows that $\mu$ is bounded. From the last assertion we deduce that $\mu$ is a limit of $\{\mu_n\}_{n\in \NN}$ with respect to $\lVert-\rVert$. This completes the proof of \textbf{(3)}.

    The proof of \textbf{(4)} follows from the investigation of the proof of \textbf{(3)} above. The details are left for the reader.
\end{proof}

\begin{corollary}\label{corollary:spaces_of_sigma_additive_charges}
    Let $X$ be a set and let $\Sigma$ be an algebra of its subsets. Consider the set
    $$\mathrm{bca}(\Sigma, \CC) = \big\{\mu:\Sigma \ra \CC\,\big|\,\mu\mbox{ is a bounded and $\sigma$-additive charge on }\Sigma\big\}$$
    Then $\mathrm{bca}(\Sigma,\CC)$ is a $\CC$-linear subspace of $\mathrm{ba}(\Sigma,\CC)$ closed with respect to total variation norm.
\end{corollary}
\begin{proof}
    Closedness follows from Theorem \ref{theorem:space_of_bounded_complex_charges}. The fact that $\mathrm{bca}(\Sigma, \CC)$ is $\CC$-linear subspace of $\mathrm{ba}(\Sigma,\CC)$ is left as an exercise for the reader.
\end{proof}

\begin{remark}\label{remark:spaces_of_bounded_charges}
    Let $X$ be a set and let $\Sigma$ be an algebra of its subsets. We have the following diagram of Banach spaces and their inclusions.
    \begin{center}
        \begin{tikzpicture}
            [description/.style={fill=white,inner sep=2pt}]
            \matrix (m) [matrix of math nodes, row sep=3em, column sep=1em,text height=1.5ex, text depth=0.25ex]
            {                                                    & \mathrm{ba}(\Sigma,\CC) & \\
                                        \mathrm{bca}(\Sigma,\CC) &                         & \mathrm{ba}(\Sigma,\RR)     \\
                                                                 & \mathrm{ba}(\Sigma,\CC) & \\} ;
            \path[right hook->,line width=0.8pt,font=\scriptsize]
            (m-2-3) edge node {$  $} (m-1-2)
            (m-3-2) edge node {$  $} (m-2-3);
            \path[left hook->,line width=0.8pt,font=\scriptsize]
            (m-2-1) edge node {$  $} (m-1-2)
            (m-3-2) edge node {$  $} (m-2-1);
        \end{tikzpicture}
    \end{center}
    In the diagram $\mathrm{cba}(\Sigma, \RR)$ is the intersection of $\mathrm{ba}(\Sigma,\RR)$ and $\mathrm{bca}(\Sigma,\CC)$ i.e. a Banach space over $\RR$ of all real, bounded and $\sigma$-additive charges on $\Sigma$.
\end{remark}

\section{Space of essentially bounded functions}
\noindent
In this section we extend the notion of Lebesgue space to $p = +\infty$. We fix a Banach space $Y$ with norm $\lVert-\rVert$ over a field $\mathbb{K}$ with absolute value $|-|$.

\begin{definition}
    Let $f:X\ra Y$ be a strongly measurable function on a space $(X,\Sigma,\mu)$ with measure. Then
    $$\lVert f \rVert_{\infty} = \sup \bigg\{r\in \RR_+\cup \{0\}\,\bigg|\,\mu\left(\big\{x\in X\,\big|\,\lVert f(x)\rVert \geq r\big\}\right) > 0\bigg\}$$
    is \textit{the essential supremum of $f$ with respect to $\mu$}.
\end{definition}

\begin{proposition}\label{proposition:L_infinity_norm_is_seminorm}
    Let $(X,\Sigma,\mu)$ be a space with measure. Then
    \begin{enumerate}[label=\emph{\textbf{(\arabic*)}}, leftmargin=*]
        \item If $\alpha \in \mathbb{K}$ and $f:X\ra Y$ is a strongly measurable function on $(X,\Sigma)$, then
              $$\lVert \alpha \cdot f\rVert_{\infty} = |\alpha|\cdot \lVert f\rVert_{\infty}$$
        \item If $f,g:X\ra Y$ are strongly measurable functions on $(X,\Sigma)$, then
              $$\lVert f + g \rVert_{\infty} \leq \lVert f \rVert_{\infty} + \lVert g \rVert_{\infty}$$
    \end{enumerate}
\end{proposition}
\begin{proof}
    Fix $\alpha \in \mathbb{K}\setminus \{0\}$ and $f:X\ra Y$ be a strongly measurable function on $(X,\Sigma)$. Then
    $$\{x\in X\,\big|\,\lVert (\alpha \cdot f)(x)\rVert \geq r\big\} = \bigg\{x\in X\,\bigg|\,\lVert f(x)\rVert \geq \frac{r}{|\alpha|}\bigg\}$$
    for every $r \in \RR_+\cup \{0\}$. Hence
    $$\lVert \alpha \cdot f\rVert_{\infty} = \sup \bigg\{r\in \RR_+\cup \{0\}\,\bigg|\,\mu\left(\big\{x\in X\,\big|\,\lVert (\alpha \cdot f)(x)\rVert \geq r\big\}\right) > 0\bigg\} =$$
    $$= \sup \bigg\{r \in \RR_+\cup \{0\}\,\bigg|\,\mu\left(\bigg\{x\in X\,\bigg|\,\lVert f(x)\rVert \geq \frac{r}{|\alpha|}\bigg\}\right) > 0\bigg\} = $$
    $$= |\alpha| \cdot \sup \bigg\{r \in \RR_+\cup \{0\}\,\bigg|\,\mu\left(\big\{x\in X\,\big|\,\lVert f(x)\rVert \geq r\big\}\right) > 0\bigg\} = |\alpha|\cdot \lVert f\rVert_{\infty}$$
    It follows that
    $$\lVert \alpha \cdot f\rVert_{\infty} = |\alpha|\cdot \lVert f\rVert_{\infty}$$
    for every $\alpha \in \mathbb{K}\setminus \{0\}$. For $\alpha = 0$ this also holds for trivial reasons. Hence \textbf{(1)} is proved.

    Suppose that $f,g:X\ra Y$ are strongly measurable functions on $(X,\Sigma)$. Assume that $r\in \RR_+$ is such that
    $$\lVert f \rVert_{\infty} + \lVert g \rVert_{\infty} < r$$
    We may pick $r_f,r_g\in \RR_+$ such that $r_f+r_g = r$ and $\lVert f \rVert_{\infty} < r_f$ and $\lVert g \rVert_{\infty} < r_g$. Then
    $$\{x\in X\,\big|\,\lVert (f + g)(x)\rVert \geq r\big\} \subseteq \big\{x\in X\,\big|\,\lVert f(x)\rVert + \lVert g(x)\rVert \geq r_f + r_g\big\} \subseteq $$
    $$\subseteq \big\{x\in X\,\big|\,\lVert f(x)\rVert  \geq r_f\big\} \cup \big\{x\in X\,\big|\,\lVert g(x)\rVert  \geq r_g\big\}$$
    Since $\lVert f \rVert_{\infty} < r_f$ and $\lVert g \rVert_{\infty} < r_g$, we deduce that
    $$\mu\left(\big\{x\in X\,\big|\,\lVert f(x)\rVert  \geq r_f\big\}\right) = \mu\left(\big\{x\in X\,\big|\,\lVert g(x)\rVert  \geq r_g\big\}\right) = 0$$
    This implies that
    $$\mu\left(\{x\in X\,\big|\,\lVert (f + g)(x)\rVert \geq r\big\}\right) = 0$$
    and thus $\lVert f + g\rVert_{\infty} < r$. This proves that
    $$\lVert f + g \rVert_{\infty} \leq \lVert f \rVert_{\infty} + \lVert g \rVert_{\infty}$$
    if right hand side is finite. Clearly the inequality holds if the right hand side is infinite. This completes the proof of \textbf{(2)}.
\end{proof}

\begin{definition}
    Let $f:X\ra Y$ be a strongly measurable function on a space $(X,\Sigma,\mu)$ with measure. If
    $$\lVert f \rVert_{\infty} \in \RR$$
    then $f$ is \textit{essentially bounded with respect to $\mu$} or shortly \textit{$\mu$-essentially bounded}.
\end{definition}

\begin{definition}
    Let $(X,\Sigma,\mu)$ be a space with measure. Then the set of all $Y$-valued and $\mu$-essentially bounded functions is denoted by $L^{\infty}(\mu,Y)$ and is called \textit{the Lebesgue space of $\mu$-essentially bounded functions for $Y$}.
\end{definition}

\begin{corollary}\label{corollary:L_infinity_is_seminormed_topological_vector_space}
    Let $(X,\Sigma,\mu)$ be a space with measure. Then $L^{\infty}(\mu,Y)$ is a $\mathbb{K}$-vector subspace of the $\mathbb{K}$-vector space of all strongly measurable functions on $(X,\Sigma)$ and
    $$\lVert -\rVert_{\infty}:L^{\infty}(\mu,Y)\ra \RR_+\cup \{0\}$$
    is a seminorm.
\end{corollary}
\begin{proof}
    This follows immediately from Proposition \ref{proposition:L_infinity_norm_is_seminorm}.
\end{proof}

\begin{theorem}[Riesz]\label{theorem:Riesz_theorem_for_L_infinity}
    Let $(X,\Sigma,\mu)$ be a space with measure and let $\{f_n:X\ra Y\}_{n\in \NN}$ be a Cauchy sequence of elements of $L^{\infty}(\mu,Y)$. Then $\{f_n\}_{n\in \NN}$ converges in $L^{\infty}(\mu,Y)$.
\end{theorem}
\begin{proof}
    Consider an increasing sequence $\{n_k\}_{k\in \NN}$ of natural numbers such that
    $$\lVert f_{n} - f_{m}\rVert_{\infty} \leq 2^{-k}$$
    for every $n,m\geq n_k$ and for every $k \in \NN$. For every $k\in \NN$ sets
    $$A_{k} = \bigcup_{n=n_k}^{+\infty}\bigcup_{m=n_k}^{+\infty}\big\{x\in X\,\big|\,\lVert f_{n}(x) - f_{m}(x) \rVert > 2^{-k} \big\}$$
    and
    $$B_k = \big\{x\in X\,\big|\,\lVert f_k(x) \rVert > \lVert f_k \rVert_{\infty}\big\}$$
    are in $\Sigma$ and have measure $\mu$ equal to zero. Hence
    $$A = \bigcup_{k\in \NN}\left(A_k\cup B_k\right)$$
    have measure $\mu$ equal to zero. Now $\{{f_n}_{\mid X\setminus A}\}_{n\in \NN}$ is a sequence of bounded functions which is Cauchy with respect to uniform norm. Since $Y$ is complete with respect to $\lVert-\rVert$, sequence $\{{f_n}_{\mid X\setminus A}\}_{n\in \NN}$ converges uniformly to some function $X\setminus A\ra Y$. We extend this function to a function $f:X\ra Y$ by setting it equal to zero on $A$. Note that $f$ is strongly measurable by Proposition \ref{proposition:strongly_measurable_functions_closed_under_pointwise_limits}. Moreover, $\{{f_n}_{\mid X\setminus A}\}_{n\in \NN}$ converges uniformly to $f_{\mid X\setminus A}$. Thus $f_{X\setminus A}$ is bounded and hence $f\in L^{\infty}(\mu,Y)$. For the same reason $f$ is a limit of $\{f_n\}_{n\in \NN}$ in $L^{\infty}(\mu,Y)$.
\end{proof}

\section{Integration with respect to charges}






\small
\bibliographystyle{apalike}
\bibliography{../zzz}

\end{document}