\input ../pree.tex

\begin{document}

\title{Filters in topology}
\date{}
\maketitle

\section{Introduction}
\noindent
In these short notes we study filters of subsets with their applications to topological spaces. Filters were introduced in \cite{cartan1937filters} as an effective tool in studying general topological spaces. Here we recapitulate Cartan's approach. Our main goal is to give a concise proof of Tychonoff's theorem on compact spaces.

\section{Filters}

\begin{definition}
Let $X$ be a set and let $\cF$ be a nonempty family of subsets of $X$. Assume that the following assertions hold.
\begin{enumerate}[label=\textbf{(\arabic*)}, leftmargin=*]
\item $\cF$ is closed under finite intersections.
\item If $F_1$ and $F_2$ are subsets of $X$ such that $F_1 \in \cF$ and $F_1\subseteq F_2$, then $F_2 \in \cF$.
\end{enumerate}
Then $\cF$ is \textit{a filter of subsets of $X$}.
\end{definition}

\begin{definition}
Let $X$ be a set and let $\cF$ be a filter of subsets of $X$. If $\emptyset \not \in \cF$, then $\cF$ is \textit{a proper filter}.
\end{definition}
\noindent
Filters are functorial as it is displayed in the following notion. 

\begin{definition}
Let $\cF$ be a filter of subsets of a set $X$ and let $f:X\ra Y$ be a map. Then a filter
$$f(\cF) = \big\{G\subseteq Y\,\big|\,\mbox{ there exists }F\in \cF\mbox{ such that }f(F)\subseteq G\big\}$$
of subsets of $Y$ is \textit{the image of $\cF$ under $f$}.
\end{definition}
\noindent
Let us note the following results.

\begin{fact}\label{fact:image_of_a_proper_filter_is_proper}
Let $\cF$ be a filter of subsets of a set $X$ and let $f:X\ra Y$ be a map. If $\cF$ is a proper filter, then $f(\cF)$ is a proper filter.
\end{fact}
\begin{proof}
Left for the reader as an exercise.
\end{proof}
\noindent
Now we introduce the notion of ultrafilter and prove by invoking axiom of choice that they exist.

\begin{definition}
Let $\cF$ be a proper filter of subsets of a set $X$ such that for every proper filter $\tilde{\cF}$ if $\cF \subseteq \tilde{\cF}$, then $\cF = \tilde{\cF}$. Then $\cF$ is \textit{an ultrafilter of subsets of $X$}.
\end{definition}

\begin{proposition}\label{proposition:existence_of_ultrafilters}
Let $X$ be a set and let $\cF$ be a proper filter of subsets of $X$. Then there exists an ultrafilter $\tilde{\cF}$ of subsets of $X$ such that $\cF \subseteq \tilde{\cF}$.
\end{proposition}
\begin{proof}
Consider the family
$$\mathrm{F} = \big\{\cG\,\big|\,\cG\mbox{ is a proper filter of subsets of }X\mbox{ and }\cF \subseteq \cG\big\}$$
Note that $\mathrm{F}$ is nonempty, because $\cF \in \mathrm{F}$. The inclusion introduces partial order on $\mathrm{F}$ and if $\mathrm{L}\subseteq \mathrm{F}$ is a linearly ordered subset, then
$$\bigcup \mathrm{L}$$
is a proper filter. Hence each chain in $\left(\mathrm{F},\subseteq\right)$ admits an upper bound. By Zorn's lemma implies that $\left(\mathrm{F},\subseteq\right)$ has a maximal element $\tilde{\cF}$. Clearly $\tilde{\cF}$ is an ultrafilter of subsets of $X$ which contains $\cF$.
\end{proof}

\section{Filters and convergence in topological spaces}
\noindent
The following notion play an important role.

\begin{definition}
Let $(X,\tau)$ be a topological space and let $\cF$ be a proper filter of subsets of $X$. Consider a point $x$ in $X$. Suppose that for every open neighborhood $U$ of $x$ we have $U \in \cF$. Then filter $\cF$ \textit{converges to $x$ with respect to $\tau$}.
\end{definition}

\begin{proposition}\label{proposition:characterization_of_continuous_maps_in_terms_of_filters}
Let $(X,\tau),(Y,\theta)$ be topological spaces and let $f:X\ra Y$ be a map. Then the following assertions are equivalent.
\begin{enumerate}[label=\emph{\textbf{(\roman*)}}, leftmargin=*]
\item $f$ is a continuous map $\left(X,\tau\right)\ra \left(Y,\theta\right)$.
\item If $\cF$ is a proper filter of subsets of $X$ convergent to some point $x$ with respect to $\tau$, then $f(\cF)$ converges to $f(x)$ with respect to $\theta$.
\end{enumerate}
\end{proposition}
\begin{proof}
Suppose that $f$ is a continuous map $\left(X,\tau\right)\ra \left(Y,\theta\right)$. Fix a proper filter $\cF$ of subsets of $X$ convergent to $x$ with respect to $\tau$. Fix an open neighborhood $V$ of $f(x)$ with respect to $\theta$. By continuity of $f$ we have $f^{-1}(V) \in \tau$. Thus $f^{-1}(V)$ is an open neighborhood of $x$ with respect to $\tau$. Hence $f^{-1}(V) \in \cF$ and we infer that $V \in f(\cF)$. Since $V$ is arbitrary open neighborhood of $f(x)$ with respect to $\theta$, we derive that $f(\cF)$ converges to $f(x)$ with respect to $\theta$. This proves the implication $\textbf{(i)}\Rightarrow \textbf{(ii)}$.\\
Suppose now that \textbf{(ii)} holds. Fix a point $x$ in $X$ and consider an open neighborhood $V$ of $f(x)$ with respect to $\theta$. Define
$$\cF = \big\{F\subseteq X\,\big|\,U\setminus f^{-1}(V)\subseteq F\mbox{ for some open neighborhood }U\mbox{ of }x\mbox{ with respect to }\tau\big\}$$
Then $\cF$ is a filter of subsets of $X$. Note that
$$Y\setminus V = f\left(X\setminus f^{-1}(V)\right) \in f(\cF)$$
This implies that $V \not \in f(\cF)$. If $\cF$ is a proper filter, then it converges to $x$ with respect $\tau$ and thus $f(\cF)$ converges to $f(x)$ with respect to $\theta$. Since $V \not \in f(\cF)$, the filter $f(\cF)$ cannot converge to $f(x)$ with respect to $\theta$. Therefore, $\cF$ is not a proper filter. This means that there exists an open neighborhood $U$ of $x$ with respect to $\tau$ such that $U \subseteq f^{-1}(V)$. This proves that $f$ is continuous at $x$ as a map $\left(X,\tau\right)\ra \left(Y,\theta\right)$. Since $x\in X$ is arbitrary, we derive the implication $\textbf{(ii)}\Rightarrow \textbf{(i)}$.
\end{proof}













\small
\bibliographystyle{apalike}
\bibliography{../zzz}

\end{document}