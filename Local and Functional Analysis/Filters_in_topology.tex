\input ../pree.tex

\begin{document}

\title{Filters in topology}
\date{}
\maketitle

\section{Introduction}
\noindent
In these short notes we study filters of subsets with their applications to topological spaces. Filters were introduced in \cite{cartan1937filters} as an effective tool in studying general topological spaces. Here we recapitulate Cartan's results. In particular, we give a concise proof of Tychonoff's theorem on compact spaces.

\section{Filters}

\begin{definition}
Let $X$ be a set and let $\cF$ be a nonempty family of subsets of $X$. Assume that the following assertions hold.
\begin{enumerate}[label=\textbf{(\arabic*)}, leftmargin=*]
\item $\cF$ is closed under finite intersections.
\item If $F_1$ and $F_2$ are subsets of $X$ such that $F_1 \in \cF$ and $F_1\subseteq F_2$, then $F_2 \in \cF$.
\end{enumerate}
Then $\cF$ is \textit{a filter of subsets of $X$}.
\end{definition}
\noindent
We note the following fact. 

\begin{fact}\label{fact:filters_are_closed_under_intersections}
Let $X$ be a set and let $\{\cF_i\}_{i\in I}$ be a family of filters of subsets of $X$. Then 
$$\bigcap_{i\in I}\cF_i$$
is a filter of subsets of $X$.
\end{fact}
\begin{proof}
Left for the reader as an exercise.
\end{proof}

\begin{definition}
Let $X$ be a set and let $\cF$ be a filter of subsets of $X$. If $\emptyset \not \in \cF$, then $\cF$ is \textit{a proper filter}.
\end{definition}
\noindent
Filters are functorial as it is displayed in the following notion. 

\begin{definition}
Let $\cF$ be a filter of subsets of a set $X$ and let $f:X\ra Y$ be a map. Then a filter
$$f(\cF) = \big\{Z \subseteq Y\,\big|\,\mbox{ there exists }F\in \cF\mbox{ such that }f(F)\subseteq Z\big\}$$
of subsets of $Y$ is \textit{the image of $\cF$ under $f$}.
\end{definition}
\noindent
Let us note the following results.

\begin{fact}\label{fact:image_of_a_proper_filter_is_proper}
Let $\cF$ be a filter of subsets of a set $X$ and let $f:X\ra Y$ be a map. If $\cF$ is a proper filter, then $f(\cF)$ is a proper filter.
\end{fact}
\begin{proof}
Left for the reader as an exercise.
\end{proof}
\noindent
Now we introduce the notion of ultrafilter and prove its properties. Finally by invoking axiom of choice we prove that ultrafilters exist.

\begin{definition}
Let $\cF$ be a proper filter of subsets of a set $X$ such that for every proper filter $\tilde{\cF}$ of subsets of $X$ if $\cF \subseteq \tilde{\cF}$, then $\cF = \tilde{\cF}$. Then $\cF$ is \textit{an ultrafilter of subsets of $X$}.
\end{definition}

\begin{proposition}\label{proposition:ultrafilter_contains_either_susbet_or_its_complement}
Let $X$ be a set and let $\cF$ be a proper filter of subsets of $X$. The following assertions are equivalent.
\begin{enumerate}[label=\emph{\textbf{(\roman*)}}, leftmargin=*]
\item $\cF$ is an ultrafilter of subsets of $X$.
\item For each subset $F$ of $X$ either $F \in \cF$ or $X\setminus F \in \cF$.
\end{enumerate}
\end{proposition}
\begin{proof}
Assume that $\cF$ is an ultrafilter and let $F$ be a subset of $X$. Suppose that $F\not \in \cF$. Then the smallest filter containing $\{F\}\cup \cF$, which exists according to Fact \ref{fact:filters_are_closed_under_intersections}, is not a proper filter. This implies that there exists $F' \in \cF$ such that $F\cap F' = \emptyset$. Since $F'\subseteq X\setminus F$ and $\cF$ is a filter, we derive that $X\setminus F \in \cF$. This proves that $\textbf{(i)}\Rightarrow \textbf{(ii)}$.\\
Suppose that \textbf{(ii)} holds. Consider a filter $\tilde{\cF}$ such that $\cF \subsetneq \tilde{\cF}$. If $F \in \tilde{\cF}\setminus \cF$, then $X\setminus F \in \cF$ and hence $\emptyset = F\cap \left(X\setminus F\right) \in \tilde{\cF}$. This implies that $\tilde{\cF}$ is not a proper filter. Thus $\cF$ is an ultrafilter of subsets of $X$. This completes the proof of $\textbf{(ii)}\Rightarrow \textbf{(i)}$.
\end{proof}

\begin{corollary}\label{corollary:ultrafilters_are_preserved_by_images}
Let $f:X\ra Y$ be a map of sets and let $\cF$ be an ultrafilter of subsets of $X$. Then $f(\cF)$ is an ultrafilter.
\end{corollary}
\begin{proof}
Filter $f(\cF)$ is proper according to Fact \ref{fact:image_of_a_proper_filter_is_proper}. Fix a subset $Z$ of $Y$.  By Proposition \ref{proposition:ultrafilter_contains_either_susbet_or_its_complement} either $f^{-1}(Z) \in \cF$ or $f^{-1}(Y\setminus Z) \in \cF$ . Thus either $Z \in f(\cF)$ or $Y\setminus Z \in f(\cF)$. Proposition \ref{proposition:ultrafilter_contains_either_susbet_or_its_complement} implies that $f(\cF)$ is an ultrafilter.
\end{proof}

\begin{proposition}\label{proposition:existence_of_ultrafilters}
Let $X$ be a set and let $\cF$ be a proper filter of subsets of $X$. Then there exists an ultrafilter $\tilde{\cF}$ of subsets of $X$ such that $\cF \subseteq \tilde{\cF}$.
\end{proposition}
\begin{proof}
Consider the family
$$\mathrm{F} = \big\{\cG\,\big|\,\cG\mbox{ is a proper filter of subsets of }X\mbox{ and }\cF \subseteq \cG\big\}$$
Note that $\mathrm{F}$ is nonempty because $\cF \in \mathrm{F}$. The inclusion introduces partial order on $\mathrm{F}$ and if $\mathrm{L}\subseteq \mathrm{F}$ is a linearly ordered subset, then
$$\bigcup \mathrm{L}$$
is a proper filter. Hence each chain in $\left(\mathrm{F},\subseteq\right)$ admits an upper bound. Zorn's lemma implies that $\left(\mathrm{F},\subseteq\right)$ has a maximal element $\tilde{\cF}$. Clearly $\tilde{\cF}$ is an ultrafilter of subsets of $X$ which contains $\cF$.
\end{proof}

\section{Filters and convergence in topological spaces}

\begin{definition}
Let $(X,\tau)$ be a topological space and let $\cF$ be a proper filter of subsets of $X$. Consider a point $x$ in $X$. Suppose that for every open neighborhood $U$ of $x$ with respect to $\tau$ we have $U \in \cF$. Then $\cF$ \textit{converges to $x$ with respect to $\tau$}.
\end{definition}

\begin{proposition}\label{proposition:characterization_of_continuous_maps_in_terms_of_filters}
Let $(X,\tau),(Y,\theta)$ be topological spaces and let $f:X\ra Y$ be a map. Then the following assertions are equivalent.
\begin{enumerate}[label=\emph{\textbf{(\roman*)}}, leftmargin=*]
\item $f$ is a continuous map $\left(X,\tau\right)\ra \left(Y,\theta\right)$.
\item If $\cF$ is a proper filter of subsets of $X$ convergent to some point $x$ with respect to $\tau$, then $f(\cF)$ converges to $f(x)$ with respect to $\theta$.
\item If $\cF$ is an ultrafilter of subsets of $X$ convergent to some point $x$ with respect to $\tau$, then $f(\cF)$ converges to $f(x)$ with respect to $\theta$.
\end{enumerate}
\end{proposition}
\begin{proof}
Suppose that $f$ is a continuous map $\left(X,\tau\right)\ra \left(Y,\theta\right)$. Fix a proper filter $\cF$ of subsets of $X$ convergent to $x$ with respect to $\tau$. Fix an open neighborhood $V$ of $f(x)$ with respect to $\theta$. By continuity of $f$ we have $f^{-1}(V) \in \tau$. Thus $f^{-1}(V)$ is an open neighborhood of $x$ with respect to $\tau$. Hence $f^{-1}(V) \in \cF$ and we infer that $V \in f(\cF)$. Since $V$ is arbitrary open neighborhood of $f(x)$ with respect to $\theta$, we derive that $f(\cF)$ converges to $f(x)$ with respect to $\theta$. This proves the implication $\textbf{(i)}\Rightarrow \textbf{(ii)}$.\\
The implication $\textbf{(ii)}\Rightarrow \textbf{(iii)}$ follows by definition of ultrafilter.\\
Suppose now that \textbf{(iii)} holds. Fix a point $x$ in $X$ and consider an open neighborhood $V$ of $f(x)$ with respect to $\theta$. Define
$$\cF = \big\{F\subseteq X\,\big|\,U\setminus f^{-1}(V)\subseteq F\mbox{ for some open neighborhood }U\mbox{ of }x\mbox{ with respect to }\tau\big\}$$
Then $\cF$ is a filter of subsets of $X$. If $\cF$ is a proper filter, then Proposition \ref{proposition:existence_of_ultrafilters} asserts that there exists an ultrafilter $\tilde{\cF}$ containing $\cF$. Since $\cF$ converges to $x$ with respect $\tau$, we derive that $\tilde{\cF}$ converges to $x$ with respect to $\tau$. Thus $f(\tilde{\cF})$ converges to $f(x)$ with respect to $\theta$. Note that
$$f\left(X\setminus f^{-1}(V)\right) \in f(\tilde{\cF})$$
This implies that $Y\setminus V \in f(\tilde{\cF})$ and hence $V \not \in f(\tilde{\cF})$. It follows that the filter $f(\tilde{\cF})$ cannot converge to $f(x)$ with respect to $\theta$. Therefore, $\cF$ is not a proper filter. This means that there exists an open neighborhood $U$ of $x$ with respect to $\tau$ such that $U \subseteq f^{-1}(V)$. This proves that $f$ is continuous at $x$ as a map $\left(X,\tau\right)\ra \left(Y,\theta\right)$. Since $x\in X$ is arbitrary, we derive $\textbf{(iii)}\Rightarrow \textbf{(i)}$.
\end{proof}

\begin{theorem}\label{theorem:quasi_compact_in_terms_of_ultrafilters}
Let $(X,\tau)$ be a topological space. Then the following assertions are equivalent.
\begin{enumerate}[label=\emph{\textbf{(\roman*)}}, leftmargin=*]
\item Each ultrafilter of subsets of $X$ is convergent to some point of $X$ with respect to $\tau$.
\item $(X,\tau)$ is a quasi-compact topological space.
\end{enumerate}
\end{theorem}
\begin{proof}
Suppose that \textbf{(i)} holds. Pick a family $\{F_i\}_{i\in I}$ of closed and nonempty subsets of $\left(X,\tau\right)$ which is closed under finite intersections. Then the family
$$\big\{F\subseteq X\,\big|\,F_i\subseteq F\mbox{ for some }i\in I\big\}$$
is a proper filter of subsets of $X$. By Proposition \ref{proposition:existence_of_ultrafilters} there exists an ultrafilter $\cF$ of subsets of $X$ which contains the filter defined above. According to \textbf{(i)} ultrafilter $\cF$ is convergent to some point $x$ in $X$ with respect to $\tau$. Then for every open neighborhood $U$ of $x$ with respect to $\tau$ we have $U \in \cF$. In particular, $U\cap F_i \neq \emptyset$ for every $i\in I$ and for every open neighborhood $U$ of $x$ with respect to $\tau$. Since $F_i$ is closed for each $i\in I$, this implies that $x \in F_i$ for every $i \in I$. Thus
$$x \in  \bigcap_{i\in I}F_i$$
and this implies that $\left(X,\tau\right)$ is quasi-compact. This completes the proof of $\textbf{(i)}\Rightarrow \textbf{(ii)}$.\\
Assume that $(X,\tau)$ is quasi-compact and suppose that $\cF$ is an ultrafilter of subsets of $X$. Suppose that $\cF$ is not convergent. Then for every $x \in X$ there exists open neighborhood $U_x$ of $x$ with respect to $\tau$ such that $U_x \not \in \cF$. Since $\left(X,\tau\right)$ is quasi-compact, we deduce that there exist finite subset $\{x_1,...,x_n\} \in X$ such that 
$$X = \bigcup_{i=1}^nU_{x_i}$$
According to Proposition \ref{proposition:ultrafilter_contains_either_susbet_or_its_complement} we derive that $X\setminus U_x \in \cF$ for every $x \in X$. Hence
$$\bigcap_{i=1}^n\left(X\setminus U_{x_i}\right) \in \cF$$
On the other hand we have
$$\bigcap_{i=1}^n\left(X\setminus U_{x_i}\right) = X \setminus \bigcup_{i=1}^nU_{x_i} = \emptyset$$
This is contradiction. Thus the implication $\textbf{(ii)}\Rightarrow \textbf{(i)}$ holds.
\end{proof}

\section{Tychonoff's theorem}
\noindent
The following result is a celebrated theorem due to Tychonoff.

\begin{theorem}\label{theorem:Tychonoff_theorem}
Let $\big\{\left(X_i,\tau_i\right)\big\}_{i\in I}$ be a family of quasi-compact topological spaces. Then the product
$$\prod_{i\in I}\left(X_i,\tau_i\right)$$
is quasi-compact.
\end{theorem}
\begin{proof}
We denote $\prod_{i\in I}X_i$ by $X$ and let $\tau$ be the product of topologies $\{\tau_i\}_{i\in I}$. For each $i$ in $I$ we denote by $pr_i:X \ra X_i$ the canonical projection onto $i$-th factor. Suppose that $\left(X_i,\tau_i\right)$ is a quasi-compact for every $i\in I$. Pick an ultrafilter $\cF$ of subsets of $X$. Fix $i$ in $I$. According to Corollary \ref{corollary:ultrafilters_are_preserved_by_images} the filter $pr_i(\cF)$ is an ultrafilter. Since $\left(X_i,\tau_i\right)$ is quasi-compact, we derive that $pr_i(\cF)$ is convergent to some point $x_i\in X_i$ with respect to $\tau_i$. Let $x$ be a point of $X$ such that $pr_i(x) = x_i$ for each $i\in I$. Fix finite subset $\{i_1,...,i_n\}\subseteq I$. Consider open neighborhood $U_j$ of $x_{i_j}$ with respect to $\tau_{i_j}$ for $j=1,...,n$. Then $U_{i_j}\in pr_{i_j}(\cF)$ for each $j$ and hence $pr_{i_j}^{-1}(U_{i_j}) \in \cF$ for each $j$. Since $\cF$ is a filter, we derive that
$$\prod_{j=1}^nU_{i_j}\times \prod_{i\in I\setminus \{i_1,...,i_n\}}X_i = \bigcap_{j=1}^npr^{-1}_{i_j}(U_{i_j}) \in \cF$$
This implies that $\cF$ is convergent to $x$ with respect to $\tau$. Thus every ultrafilter in $(X,\tau)$ is convergent and hence Theorem \ref{theorem:quasi_compact_in_terms_of_ultrafilters} shows that $\left(X,\tau\right)$ is a quasi-compact topological space.
\end{proof}


\begin{theorem}\label{theorem:Tychonoff_theorem_converse}
Let $\big\{\left(X_i,\tau_i\right)\big\}_{i\in I}$ be a family of nonempty topological spaces. If the product
$$\prod_{i\in I}\left(X_i,\tau_i\right)$$
is quasi-compact, then $\left(X_i,\tau_i\right)$ is quasi-compact for every $i\in I$.
\end{theorem}
\begin{proof}
We denote $\prod_{i\in I}X_i$ by $X$ and let $\tau$ be the product of topologies $\{\tau_i\}_{i\in I}$. For each $i$ in $I$ we denote by $pr_i:X \ra X_i$ the canonical projection onto $i$-th factor. Assume that $\left(X,\tau\right)$ is quasi-compact. Since $X_i \neq \emptyset$ for every $i\in I$, we derive that $pr_i:\left(X,\tau\right)\ra \left(X_i,\tau_i\right)$ is a continuous and surjective map for every $i\in I$. Hence for each $i\in I$ space $\left(X_i,\tau_i\right)$ is quasi-compact as an image of a quasi-compact space under continuous map.      
\end{proof}











\small
\bibliographystyle{apalike}
\bibliography{../zzz}

\end{document}