\input ../pree.tex

\begin{document}

\title{Radon-Nikodym and related topics}
\date{}
\maketitle

\section{Introduction}
\noindent
These notes are devoted to more advanced topics in measure theory. Tools presented here are indispensable in probability theory, statistics and applications to geometry. We refer to our notes \cite{Introduction_to_measure_theory} for basic measure theory and to \cite{Integration} for integration theory.

\section{Hahn-Jordan decomposition}

\begin{definition}
    Let $\left(X,\Sigma\right)$ be a measurable space. \textit{A signed measure on $\Sigma$} is a function $\nu:\Sigma\ra \ol{\RR}$ such that $$\nu(\emptyset)=0$$
    and
    $$\nu\left(\bigcup_{n\in \NN}A_n\right)=\sum_{n\in \NN}\nu(A_n)$$
    for every family $\{A_n\}_{n\in \NN}$ of pairwise disjoint subsets of $\Sigma$.
\end{definition}

\begin{fact}\label{fact:one_side_infinity_only_for_signed_measure}
    Let $\left(X,\Sigma\right)$ be a measurable space and let $\nu$ be a signed measure on $\Sigma$. Then the image of $\nu$ does not contain $\{-\infty,+\infty\}$.
\end{fact}
\begin{proof}
    Left for the reader as an exercise.
\end{proof}
\noindent
The following notion plays central role in studying structure of signed measures.

\begin{definition}
    Let $\left(X,\Sigma\right)$ be a measurable space and let $\nu$ be a signed measure on $\Sigma$. \textit{A positive set for $\nu$} is a set $P \in \Sigma$ such that
    $$\nu(A\cap P) \geq 0,\,\nu(A\setminus P)\leq 0$$
    for every $A \in \Sigma$.
\end{definition}

\begin{theorem}[Hahn decomposition]\label{theorem:Hahn_decomposition}
    Let $\left(X,\Sigma\right)$ be a measurable space and let $\nu$ be a signed measure on $\Sigma$. Then there exists a positive set for $\nu$.
\end{theorem}
\noindent
The proof proceeds by constructing approximations for a positive set.

\begin{lemma}\label{lemma:approximate_positive_set}
    Let $\left(X,\Sigma\right)$ be a measurable space and let $\nu$ be a signed measure on $\Sigma$. Suppose that $\nu(A) \geq 0$ for some $A \in \Sigma$. Then for each $\epsilon > 0$ there exists a subset $Q_{\epsilon}$ of $A$ such that the following assertions hold.
    \begin{enumerate}[label=\emph{\textbf{(\arabic*)}}, leftmargin=3.0em]
        \item $Q_{\epsilon} \in \Sigma$ and $\nu(Q_{\epsilon}) \geq \nu(A)$.
        \item If $B \in \Sigma$ and $B \subseteq Q_{\epsilon}$, then $\nu(B) \geq -\epsilon$.
    \end{enumerate}
\end{lemma}
\begin{proof}[Proof of the lemma]
    Let $\fF$ be a family of all sets in $\Sigma$ contained in $A$. For any two sets $F_1,F_2\in \fF$ we define
    $$F_1 \sqsubseteq_{\epsilon}F_2$$
    if and only if $F_2 \subseteq F_1$ and $\nu(F_1 \setminus F_2) < -\epsilon$. Clearly $\sqsubseteq_{\epsilon}$ is transitive and antireflexive. Suppose that $\{F_n\}_{n\in \NN}$ is a sequence of sets in $\fF$ which is a chain with respect to $\sqsubseteq_{\epsilon}$. Then
    $$\bigcup_{n\in \NN}\left(F_n\setminus F_{n+1}\right) \in \fF$$
    and
    $$\nu\left(\bigcup_{n\in \NN}\left(F_n\setminus F_{n+1}\right)\right) = \sum_{n\in \NN}\nu\left(F_n\setminus F_{n+1}\right) < -\sum_{n\in \NN}\epsilon$$
    This contradicts the fact that $\nu(A) \geq 0$. Hence there are no infinite chains in $\fF$ with respect to $\sqsubseteq_{\epsilon}$. Thus there exists $Q_{\epsilon} \in \fF$ which is maximal with respect to $\sqsubseteq_{\epsilon}$ and is contained in a chain with respect to $\sqsubseteq_{\epsilon}$ which starts with $A$. Then $Q_{\epsilon}$ satisfies assertions.
\end{proof}

\begin{lemma}\label{lemma:positive_set}
    Let $\left(X,\Sigma\right)$ be a measurable space and let $\nu$ be a signed measure on $\Sigma$. Suppose that $\nu(A) > 0$ for some $A \in \Sigma$. Then there exists a subset $Q$ of $A$ such that the following assertions hold.
    \begin{enumerate}[label=\emph{\textbf{(\arabic*)}}, leftmargin=3.0em]
        \item $Q \in \Sigma$ and $\nu(Q) \geq \nu(A)$.
        \item If $B \in \Sigma$ and $B \subseteq Q$, then $\nu(B) \geq 0$.
    \end{enumerate}
\end{lemma}
\begin{proof}[Proof of the lemma]
    We define a sequence $\{Q_n\}_{n\in \NN}$ of sets in $\Sigma$ which are contained in $A$. We set $Q_0 = A$ and if $Q_n$ is defined for some $n \in \NN$, then we pick $Q_{n+1} \subseteq Q_n$ such that $\nu(Q_n) \leq \nu(Q_{n+1})$ and
    $$\nu\left(B\right) \geq - \frac{1}{n+1}$$
    for every $B \in \Sigma$ and $B\subseteq Q_{n+1}$. This construction is possible due to Lemma \ref{lemma:approximate_positive_set}. Define
    $$Q = \bigcap_{n\in \NN}Q_n$$
    Then $Q \in \Sigma$ and $Q\subseteq A$. Since $\{\nu(Q_n)\}_{n\in \NN}$ is nondecreasing and $Q_0 = A$, we derive
    $$\nu(A) \leq \lim_{n\ra +\infty}\nu(Q_n) = \nu(Q) $$
    Now if $B \in \Sigma$ and $B \subseteq Q$, then
    $$\nu(B) \geq -\frac{1}{n + 1}$$
    for every $n \in \NN$. Thus $\nu(B) \geq 0$. This proves that $Q$ satisfies assertions.
\end{proof}

\begin{proof}[Proof of the theorem]
    By Fact \ref{fact:one_side_infinity_only_for_signed_measure} and changing $\nu$ to $-\nu$ if necessary, we may assume that there is no set $A \in \Sigma$ such that $\nu(A) = +\infty$. Consider the family
    $$\cP = \big\{Q \in \Sigma\,\big|\,\nu(B)\geq 0\mbox{ for each }B\subseteq Q\mbox{ such that }B \in \Sigma\big\}$$
    Denote by $\alpha$ the least upper bound of $\nu(Q)$ for $Q \in \cP$. There exists a sequence $\{Q_n\}_{n\in \NN}$ such that
    $$\lim_{n\ra +\infty}\nu(Q_n) = \alpha$$
    Define
    $$P = \bigcup_{n\in \NN}Q_n$$
    Then $P \in \cP$ and $\nu(P) = \alpha$. Since by assumption $\nu(P)$ is finite, we derive that $\alpha \in \RR$. Assume that there exists a set $A \in \Sigma$ such that $\nu(A) > 0$ and $A \subseteq X\setminus P$. Then by Lemma \ref{lemma:positive_set} there exists $Q \in \cP$ such that $Q \subseteq A$ and $\nu(A) \leq \nu(Q)$. Then $Q\cup P \in \cP$ and
    $$\alpha = \nu(P) < \nu(P) + \nu(Q) = \nu(Q\cup P) \leq \alpha$$
    This is a contradiction. Hence $P$ is a positive set for $\nu$.
\end{proof}
\noindent
For the future use we introduce here important notion.

\begin{definition}
    Let $(X,\Sigma)$ be a measurable space and let $\nu:\Sigma \ra \ol{\RR}$ be a signed measure. Suppose that there there exists a decomposition
    $$X = \bigcup_{n\in \NN}X_n$$
    onto pairwise disjoint elements of $\Sigma$ such that $\nu(X_n) \in \RR$ for every $n\in \NN$. Then $\nu$ is \textit{$\sigma$-finite}.
\end{definition}

\section{Radon-Nikodym theorem}
\noindent
In this section we apply Hahn decomposition i.e. Theorem \ref{theorem:Hahn_decomposition} and prove one of the central results of measure theory.

\begin{definition}
    \textit{A real measure on $\Sigma$} is a function $\nu:\Sigma\ra \RR$ such that $$\nu(\emptyset)=0$$
    and
    $$\nu\left(\bigcup_{n\in \NN}A_n\right)=\sum_{n\in \NN}\nu(A_n)$$
    for every family $\{A_n\}_{n\in \NN}$ of pairwise disjoint subsets of $\Sigma$.
\end{definition}
\noindent
Note that real measures are special class of signed measures.

\begin{definition}
    Let $(X,\Sigma)$ be a measurable space and let $\mu$ be a measure on $\Sigma$. Let $\nu$ be a signed measure on $\Sigma$. Suppose that for every $A \in \Sigma$ if $\mu(A) = 0$, then $\nu(A) = 0$. Then $\nu$ is \textit{absolutely continuous with respect to $\mu$}.
\end{definition}

\begin{definition}
    Let $(X,\Sigma)$ be a measurable space and let $\mu$ be a measure on $\Sigma$. Let $\nu$ be a signed measure on $\Sigma$. Suppose that for every $A \in \Sigma$ if $\nu(A\cap E) = 0$ for every $E \in \Sigma$ such that $\mu(E)$ is finite, then $\nu(A) = 0$. Then $\nu$ is \textit{inner regular with respect to $\mu$}.
\end{definition}
\noindent
The following is one of central results of classical measure theory.

\begin{theorem}[Radon-Nikodym]\label{theorem:general_Radon_Nikodym}
    Let $(X,\Sigma)$ be a measurable space and let $\mu$ be a measure on $\Sigma$. Let $\nu$ be a real measure on $\Sigma$. Then the following are equivalent.
    \begin{enumerate}[label=\emph{\textbf{(\roman*)}}, leftmargin=3.0em]
        \item There exists a $\mu$-integrable function $g:X\ra \RR$ such that
              $$\nu(A) = \int_Ag\,d\mu$$
              for every $A \in \Sigma$.
        \item $\nu$ is absolutely continuous and inner regular with respect to $\mu$.
    \end{enumerate}
\end{theorem}
\noindent
For the proof we need the following result.

\begin{lemma}\label{lemma:nontrivial_mu_simple_function_below_nu}
    Let $(X,\Sigma)$ be a measurable space and let $\mu$ be a measure on $\Sigma$. Let $\nu$ be a nonzero finite measure on $\Sigma$ which is absolutely continuous and inner regular with respect to $\mu$. Then there exists a $\mu$-integrable and nonnegative function $f:X\ra \RR$ such that the following assertions hold.
    \begin{enumerate}[label=\emph{\textbf{(\roman*)}}, leftmargin=3.0em]
        \item Inequality
              $$\int_Af\,d\mu \leq \nu(A)$$
              holds for each $A \in \Sigma$
        \item The integral of $f$ with respect to $\mu$ is positive.
    \end{enumerate}
\end{lemma}
\begin{proof}[Proof of the lemma]
    For each $n \in \NN$ consider signed measure $\nu_n$ on $\Sigma$ given by formula
    $$\nu_n(A) = \nu(A) - \frac{1}{n+1}\cdot \mu(A)$$
    for every $A \in \Sigma$. By Theorem \ref{theorem:Hahn_decomposition} let $P_n \in \Sigma$ be a positive set of $\nu_n$ for each $n \in \NN$. Assume that $\mu_n(P_n) = 0$ for every $n\in \NN$. Let $P$ be the union of sets $\{P_n\}_{n\in \NN}$. Then $P \in \Sigma$ and $\mu(P) = 0$. Since $\nu$ is absolutely continuous with respect to $\mu$, we derive that $\nu(P) = 0$. Pick $E \in \Sigma$ such that $\mu(E) \in \RR$. Then
    $$\nu(E\setminus P) \leq \frac{1}{n+1} \cdot \mu(E\setminus P)$$
    for each $n \in \NN$ and hence $\nu(E\setminus P) = 0$. Since this holds for each $E \in \Sigma$ such that $\mu(E) \in \RR$ and $\nu$ is inner regular with respect to $\mu$, we derive that $\nu(X\setminus P) = 0$. Thus $\nu$ is the zero measure on $\Sigma$. This contradicts the assumption that $\nu$ is nonzero. Therefore, there exists $n \in \NN$ such that $\mu_n(P_n) > 0$. Define $\Sigma$-measurable function
    $$f = \frac{1}{n + 1}\cdot \chi_{P_n}$$
    We have
    $$\int_Af\,d\mu = \frac{1}{n+1}\cdot \mu(A\cap P_n) \leq \nu(A\cap P_n) \leq \nu(A)$$
    for each $A \in \Sigma$. In particular, we have
    $$\int_Xf\,d\mu = \frac{1}{n+1}\cdot \mu(P_n) \leq \nu(P_n) \in \RR$$
    Thus $f$ is $\mu$-integrable and its integral with respect to $\mu$ is positive. It follows that $f$ satisfies assertions \textbf{(1)} and \textbf{(2)}.
\end{proof}

\begin{proof}[Proof of the theorem]
    First we prove that $\textbf{(i)}\Rightarrow \textbf{(ii)}$. We assume that there exists a $\mu$-integrable function $g:X\ra \RR$ such that
    $$\nu(A) = \int_Ag\,d\mu$$
    for every $A\in \Sigma$. Since every $\mu$-integrable function is a difference of two nonnegative $\mu$-integrable functions, we may assume that $g$ is nonnegative. If $A \in \Sigma$ satisfies $\mu(A) = 0$, then
    $$\nu(A) = \int_Ag\,d\mu = 0$$
    Thus $\nu$ is absolutely continuous with respect to $\mu$. Assume now that $A \in \Sigma$ satisfies $\nu(A\cap E) = 0$ for every set of $E \in \Sigma$ such that $\mu(E) \in \RR$. Define sets
    $$P_n = \bigg\{x\in X\,\bigg|\,g(x) \geq \frac{1}{n + 1}\bigg\}, P = \bigg\{x\in X\,\bigg|\,g(x) > 0\bigg\}$$
    Then $P_n \in \Sigma$ and $\mu(P_n) \in \RR$ for all $n \in \NN$. This last assertion holds according to the fact that $g$ is $\mu$-integrable. We have
    $$\int_{A}\chi_{P_n}\cdot g\,d\mu = \nu(A\cap P_n) = 0$$
    for each $n \in \NN$. Hence
    $$0 = \lim_{n\ra +\infty}\int_A\chi_{P_n}\cdot g\,d\mu = \int_A\chi_P\cdot g\,d\mu = \int_Ag\,d\mu = \nu(A)$$
    This proves that $\nu$ is inner regular with respect to $\mu$ and completes the proof of the implication.\\
    Now we prove that $\textbf{(ii)}\Rightarrow \textbf{(i)}$. First assume that $\nu$ takes nonnegative values. Define
    $$\cF = \bigg\{f:X\ra \RR\,\big|\,f\mbox{ is }\mu\mbox{-integrable, nonnegative and }\int_Ag\,d\mu \leq \nu(A)\mbox{ for every }A \in \Sigma\bigg\}$$
    and
    $$\alpha = \sup_{f \in \cF}\int_Xf\,d\mu \leq \nu(X)$$
    Clearly $\alpha \leq \nu(X)$. Next there exists a nondecreasing sequence $\{g_n\}_{n\in \NN}$ in $\cF$ such that
    $$\alpha = \lim_{n\ra +\infty}\int_Xg_n\,d\mu$$
    Let $g$ be a pointwise limit of $\{g_n\}_{n\in \NN}$. Then $g$ is $\Sigma$-measurable and nonnegative. Moreover, $g$ can potentially take $+\infty$ as value. By monotone convergence theorem we have
    $$\int_Ag\,d\mu = \lim_{n\ra +\infty}\int_Ag_n\,d\mu \leq \nu(A)$$
    This proves that $\mu(\{x\in X\,|\,g(x) = +\infty\}) = 0$. By modyfying all functions in $\{g_n\}_{n\in \NN}$ on a set of measure $\mu$ equal to zero, we may achieve that $g:X \ra \RR$. Then $g \in \cF$ and
    $$\alpha = \int_Xg\,d\mu$$
    Now define a measure $\eta$ on $\Sigma$ by formula
    $$\eta(A) = \nu(A) - \int_Ag\,d\mu$$
    for each $A \in \Sigma$. Then $\eta$ is absolutely continuous and inner regular with respect to $\mu$. Indeed, $eta$ is a difference of measures having these properties and hence it also has them. If $\eta$ is nonzero, then by Lemma \ref{lemma:nontrivial_mu_simple_function_below_nu} there exists $\mu$-integrable nonnegative and function $f:X\ra \RR$ such that $g + f \in \cF$ and
    $$\int_X\left(g + f\right)\,d\mu > \alpha$$
    This is contradiction. Thus $\eta$ is the zero measure. Hence
    $$\nu(A) = \int_Ag\,d\mu$$
    for every $A \in \Sigma$. The proof of nonnegative valued $\nu$ is completed. Now if $\nu$ is arbitrary real measure which is both absolutely continuous and inner regular with respect to $\mu$, then by Theorem \ref{theorem:Hahn_decomposition} we pick a positive set $P \in \Sigma$ of $\nu$. We define nonnegative measures $\nu_+.\nu_-$ on $\Sigma$ by formulas
    $$\nu_+(A) = \nu(A\cap P),\,\nu_-(A) = -\nu(A\setminus P)$$
    for $A \in \Sigma$. Then both $\nu_+,\nu_-$ are absolutely continuous and inner regular with respect to $\mu$. Hence there exist $\mu$-integrable functions $g_+,g_-:X\ra \RR$ such that
    $$\nu_+(A) = \int_Ag_+\,d\mu,\,\nu_-(A) = \int_Ag_-\,d\mu$$
    for every $A \in \Sigma$. Let $g:X\ra \RR$ be defined as a sum $g_+ - g_-$. Then
    $$\nu(A) = \int_Ag\,d\mu$$
    for every $A \in \Sigma$. Clearly $g$ is $\mu$-integrable. This proves $\textbf{(ii)}\Rightarrow \textbf{(i)}$.
\end{proof}
\noindent
Now we introduce generalization of real measures and then we extend Radon-Nikodym to this setting.

\begin{definition}
    \textit{A complex measure on $\Sigma$} is a function $\nu:\Sigma\ra \CC$ such that $$\nu(\emptyset)=0$$
    and
    $$\nu\left(\bigcup_{n\in \NN}A_n\right)=\sum_{n\in \NN}\nu(A_n)$$
    for every family $\{A_n\}_{n\in \NN}$ of pairwise disjoint subsets of $\Sigma$.
\end{definition}

\begin{remark}\label{remark:absolute_continuity_and_inner_regularity_can_be_extended_to_complex_measures}
    Let $(X,\Sigma)$ be a measurable space and let $\mu$ be a measure on $\Sigma$. One can immediately extend notions of absolute continuity and inner regularity on $\mu$ to complex measures on $\Sigma$.
\end{remark}

\begin{theorem}\label{theorem:Radon_Nikodym_complex_case}
    Let $(X,\Sigma)$ be a measurable space and let $\mu$ be a measure on $\Sigma$. Let $\nu$ be a complex measure on $\Sigma$. Then the following are equivalent.
    \begin{enumerate}[label=\emph{\textbf{(\roman*)}}, leftmargin=3.0em]
        \item There exists a $\mu$-integrable function $g:X\ra \CC$ such that
              $$\nu(A) = \int_Ag\,d\mu$$
              for every $A \in \Sigma$.
        \item $\nu$ is absolutely continuous and inner regular with respect to $\mu$.
    \end{enumerate}
\end{theorem}
\begin{proof}
    In order to prove that $\textbf{(i)}\Rightarrow \textbf{(ii)}$ it suffices to decompose complex valued $\mu$-integrable function on its real and imaginary parts and invoke the corresponding part of Theorem \ref{theorem:general_Radon_Nikodym}.\\
    For the implication $\textbf{(ii)}\Rightarrow \textbf{(i)}$ we decompose $\nu$ onto its real and imaginary parts. These parts are real measures which are absolutely continuous and inner regular with respect to $\mu$. Next we apply the corresponding part of Theorem \ref{theorem:general_Radon_Nikodym} to the real and imaginary parts to derive the implication.
\end{proof}

\begin{remark}\label{remark:Radon_Nikodym_for_sigma_finite_measures}
    Let $(X,\Sigma)$ be a measurable space and let $\mu$ be a measure on $\Sigma$. Let $\nu$ be either signed or complex measure on $\Sigma$. If $\mu$ is $\sigma$-finite and $\nu$ is absolutely continuous with respect to $\mu$, then $\nu$ is inner regular with respect to $\mu$. In particular, if $\mu$ is $\sigma$-finite, then in Theorems \ref{theorem:general_Radon_Nikodym} and \ref{theorem:Radon_Nikodym_complex_case} assumption on inner regularity is redundant. This leads to versions of Radon-Nikodym theorem which are usually presented in textbooks.
\end{remark}

\section{Lebesgue decomposition theorem}

\begin{definition}
    Let $(X,\Sigma,\mu)$ be a space with measure. Let $\nu$ be either signed or complex measure on $\Sigma$. Suppose that there exists a set $S\in \Sigma$ such that
    $$\mu(A\cap S) = 0,\,\nu(A\setminus S) = 0$$
    for every $A \in \Sigma$. Then $\nu$ is \textit{singular with respect to $\mu$}.
\end{definition}

\begin{theorem}[Lebesgue decomposition]\label{theorem:lebesguede_composition}
    Let $(X,\Sigma,\mu)$ be a space with measure and let $\nu$ be a signed and $\sigma$-finite measure or a complex measure on $(X,\Sigma)$. Then there exists a unique decomposition
    $$\nu = \nu_s + \nu_a$$
    of measure $\nu$ such that $\nu_s$ is singular with respect to $\mu$ and $\nu_a$ is absolutely continuous with respect to $\mu$.
\end{theorem}
\begin{proof}
    Suppose first that $\nu:\Sigma \ra \RR$ and $\nu$ is nonnegative. Consider
    $$\alpha = \sup_{A\in \Sigma,\,\mu(A)=0}\nu(A)$$
    We have that $\alpha \in \RR$. Consider a sequence $\{A_n\}_{n\in \NN}$ of sets in $\Sigma$ such that $\mu(A_n)=0$ for every $n\in \NN$ and 
    $$\lim_{n\ra +\infty}\nu(A_n)=\alpha$$
    Define $S = \bigcup_{n\in \NN}A_n$. Then $\mu(S) = 0$ and $\nu(S) = \alpha$. Fix now $A\in \Sigma$ such that $A \subseteq X\setminus S$. If $\mu(A) = 0$ and $\nu(A) > 0$, then 
    $$\mu(A\cup S) = 0,\,\alpha = \nu(S) < \nu(S) + \nu(A) = \nu(A\cup S)$$
    This is a contradiction. Hence $\mu(A) = 0$ implies that $\nu(A) = 0$. Now we define $\nu_s:\Sigma \ra \RR$ and $\nu_a:\Sigma \ra \RR$ by formulas
    $$\nu_s(A) = \nu(A\cap S),\,\nu_a(A) = \nu(A\setminus S)$$
    for every $A\in \Sigma$. Then $\nu_s,\nu_a$ satsify the assertion. This completes the proof for real and nonnegative $\nu$. One can easily extended the result to real $\nu$ by means of Theorem \ref{theorem:Hahn_decomposition}. From this the assertion for complex $\nu$ follows by considering decomposition on real and imaginary parts, which are real. If $\nu$ is signed and $\sigma$-finite, then 
    $$\nu(A) = \sum_{n\in \NN}\nu_n(A)$$
    for every $A \in \Sigma$, where $\nu_n$ is a real measure on $\Sigma$ for each $n\in \NN$. Since each $\nu_n$ admits decomposition on singular and absolutely continuous part with respect to $\mu$, one can take sum these singular and absolutely continuous components to derive the singular and absolutely continuous component of $\nu$. This completes the proof of existence. The proof of uniqueness is left for the reader.
\end{proof}

\section{Space of complex measures}

\begin{proposition}\label{proposition:variationismeasure}
    Let $\mu$ be a complex measure on a measurable space $(X,\Sigma)$. For every $A\in \Sigma$ we define
    $$|\mu|(A) = \sup \bigg\{\sum_{n\in \NN}|\mu(A_n)|\,\bigg|\,A = \bigcup_{n\in \NN}A_n\mbox{ is a partition of }A\mbox{ onto subsets in }\Sigma\bigg\}$$
    Then $|\mu|$ is a finite measure on $(X,\Sigma)$.
\end{proposition}
\begin{proof}
    Left for the reader. It is consequence of Theorem \ref{theorem:Hahn_decomposition}.
\end{proof}

\begin{definition}
    Let $\mu$ be a complex measure on $(X,\Sigma)$. Then we define
    $$||\mu|| = |\mu|(X)$$
    and call it \textit{the total variation of $\mu$}.
\end{definition}

\begin{theorem}\label{theorem:Banachspaceofmeasures}
    Let $(X,\Sigma)$ be a measurable space and $\cM(X,\Sigma)$ be a set of all complex measures on $(X,\Sigma)$. Then the following assertions hold.
    \begin{enumerate}[label=\emph{\textbf{(\arabic*)}}, leftmargin=3.0em]
        \item $\cM(X,\Sigma)$ is a $\CC$-linear space.
        \item The mapping
              $$\cM(X,\Sigma)\ni \mu \mapsto ||\mu||\in [0,+\infty)$$
              is a norm.
        \item Suppose that $\{\mu_n\}_{n\in \NN}$ is a sequence of complex measures on $(X,\Sigma)$ that is a Cauchy sequence with respect to total variation. Then there exists a complex measure $\mu$ such that
              $$\lim_{n\ra +\infty}\mu_n = \mu$$
              Moreover, for every $A\in \Sigma$ we have
              $$\lim_{n\ra +\infty}\mu_n(A) = \mu(A)$$
    \end{enumerate}
\end{theorem}
\begin{proof}
    We left \textbf{(1)} and \textbf{(2)} for the reader as an exercise. Fix $A\in \Sigma$. Then
    $$|\mu_n(A) - \mu_m(A)| \leq |\mu_n - \mu_m|(A) \leq ||\mu_n - \mu_m||$$
    for every $n,m\in \NN$. Since $\{\mu_n\}_{n\in \NN}$ is a Cauchy sequence with respect to total variation, we derive that there exists the limit $\mu(A)$ of $\{\mu_n(A)\}_{n\in \NN}$. Suppose that
    $$A = \bigcup_{k\in \NN}A_k$$
    for $A\in \Sigma$ and $A_k\in \Sigma$ for $k\in \NN$. Assume that sets $\{A_k\}_{k \in \NN}$ are disjoint. Pick $N\in \NN$. Then
    $$\sum_{k=0}^N|\mu_n(A_k) - \mu(A_k)| = \lim_{m\ra +\infty}\sum_{k=0}^N|\mu_n(A_k) - \mu_m(A_k)| \leq$$
    $$\leq \limsup_{m\ra +\infty}\sum_{k\in \NN}|\mu_n(A_k) - \mu_m(A_k)|\leq \limsup_{m\ra +\infty}|\mu_n - \mu_m|(A) = \limsup_{m\ra +\infty}||\mu_n - \mu_m||$$
    This implies that
    $$\sum_{k\in \NN}|\mu_n(A_k) - \mu(A_k)| \leq  \limsup_{m\ra +\infty}||\mu_n - \mu_m||$$
    regardless of set $A$ and partition $\{A_k\}_{k\in \NN}$. Thus we deduce that there exists a sequence $\{a_n\}_{n\in \NN}$ of real numbers, convergent to zero such that
    $$\sum_{k\in \NN}|\mu_n(A_k) - \mu(A_k)| \leq a_n$$
    for every $n\in \NN$, $A\in \Sigma$ and partition $\{A_k\}_{k\in \NN}$ as above. Therefore, for fixed $N\in \NN$ we have
    $$\big|\mu(A) - \sum_{k=0}^N\mu(A_k)\big| \leq |\mu(A) - \mu_n(A)| + \big|\mu_n(A) - \sum_{k=0}^N\mu_n(A_k)\big| + \sum_{k=0}^N |\mu_n(A_k) - \mu(A_k)\big| \leq $$
    $$\leq |\mu(A) - \mu_n(A)| + \big|\mu_n(A) - \sum_{k=0}^N\mu_n(A_k)\big| + \sum_{k\in \NN} |\mu_n(A_k) - \mu(A_k)\big| \leq  2a_n +  \big|\mu_n(A) - \sum_{k=0}^N\mu_n(A_k)\big|$$
    Hence we derive that
    $$\mu(A) = \sum_{k\in \NN}\mu(A_k)$$
    thus $\mu$ is a complex measure and according to
    $$\sum_{k\in \NN}|\mu_n(A_k) - \mu(A_k)| \leq a_n$$
    for every $n\in \NN$ we deduce that
    $$\lim_{n\ra +\infty}|\mu_n - \mu|(A) = 0$$
    for every $A\in \Sigma$. Hence also $\lim_{n\ra +\infty}||\mu_n-\mu|| = 0$. This finishes the proof of \textbf{(3)}.
\end{proof}

\begin{theorem}\label{theorem:isometrical_embedding_of_L1_into_space_of_complex_measures}
    Let $(X,\Sigma)$ be a measurable space and let $\mu$ be a measure on $\Sigma$. Then the map
    $$L^1(X,\mu)\ni f\mapsto \left(\Sigma \ni A \mapsto \int_Af\,d\mu\in \CC\right)\in \cM(X,\Sigma)$$
    is a $\CC$-linear isometry.
\end{theorem}
\begin{proof}
    Let $\Phi:L^1(\mu,\CC)\ra \cM(X,\Sigma)$ denote the map described in the statement. Clearly $\Phi$ is well defined and $\CC$-linear. Moreover, for every $f \in L^1(\mu,\CC)$ we have
    $$\lVert \Phi(f) \rVert \leq \int_X|f|\,d\mu = \lVert f \rVert_1$$
    and hence $\Phi$ is continuous. If $s \in \cS(\mu,\CC)$, then (we left it as an exercise for the reader) we have
    $$\lVert \Phi(s) \rVert = \int_X|s|\,d\mu = \lVert s \rVert_1$$
    Since $\cS(\mu,\CC) \subseteq L^1(\mu,\CC)$ is dense, we derive that $\Phi$ is a $\CC$-linear isometry.
\end{proof}

\small
\bibliographystyle{apalike}
\bibliography{../zzz}

\end{document}